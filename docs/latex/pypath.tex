%% Generated by Sphinx.
\def\sphinxdocclass{report}
\documentclass[letterpaper,10pt,english]{sphinxmanual}
\ifdefined\pdfpxdimen
   \let\sphinxpxdimen\pdfpxdimen\else\newdimen\sphinxpxdimen
\fi \sphinxpxdimen=.75bp\relax

\PassOptionsToPackage{warn}{textcomp}
\usepackage[utf8]{inputenc}
\ifdefined\DeclareUnicodeCharacter
% support both utf8 and utf8x syntaxes
  \ifdefined\DeclareUnicodeCharacterAsOptional
    \def\sphinxDUC#1{\DeclareUnicodeCharacter{"#1}}
  \else
    \let\sphinxDUC\DeclareUnicodeCharacter
  \fi
  \sphinxDUC{00A0}{\nobreakspace}
  \sphinxDUC{2500}{\sphinxunichar{2500}}
  \sphinxDUC{2502}{\sphinxunichar{2502}}
  \sphinxDUC{2514}{\sphinxunichar{2514}}
  \sphinxDUC{251C}{\sphinxunichar{251C}}
  \sphinxDUC{2572}{\textbackslash}
\fi
\usepackage{cmap}
\usepackage[T1]{fontenc}
\usepackage{amsmath,amssymb,amstext}
\usepackage{babel}



\usepackage{times}
\expandafter\ifx\csname T@LGR\endcsname\relax
\else
% LGR was declared as font encoding
  \substitutefont{LGR}{\rmdefault}{cmr}
  \substitutefont{LGR}{\sfdefault}{cmss}
  \substitutefont{LGR}{\ttdefault}{cmtt}
\fi
\expandafter\ifx\csname T@X2\endcsname\relax
  \expandafter\ifx\csname T@T2A\endcsname\relax
  \else
  % T2A was declared as font encoding
    \substitutefont{T2A}{\rmdefault}{cmr}
    \substitutefont{T2A}{\sfdefault}{cmss}
    \substitutefont{T2A}{\ttdefault}{cmtt}
  \fi
\else
% X2 was declared as font encoding
  \substitutefont{X2}{\rmdefault}{cmr}
  \substitutefont{X2}{\sfdefault}{cmss}
  \substitutefont{X2}{\ttdefault}{cmtt}
\fi


\usepackage[Bjarne]{fncychap}
\usepackage{sphinx}

\fvset{fontsize=\small}
\usepackage{geometry}

% Include hyperref last.
\usepackage{hyperref}
% Fix anchor placement for figures with captions.
\usepackage{hypcap}% it must be loaded after hyperref.
% Set up styles of URL: it should be placed after hyperref.
\urlstyle{same}
\addto\captionsenglish{\renewcommand{\contentsname}{Contents:}}

\usepackage{sphinxmessages}
\setcounter{tocdepth}{4}
\setcounter{secnumdepth}{4}


\title{pypath Documentation}
\date{Feb 25, 2020}
\release{0.10.6}
\author{Dénes Türei}
\newcommand{\sphinxlogo}{\vbox{}}
\renewcommand{\releasename}{Release}
\makeindex
\begin{document}

\pagestyle{empty}
\sphinxmaketitle
\pagestyle{plain}
\sphinxtableofcontents
\pagestyle{normal}
\phantomsection\label{\detokenize{index::doc}}

\begin{quote}\begin{description}
\item[{note}] \leavevmode
\sphinxcode{\sphinxupquote{pypath}} supports both Python 2.7 and Python 3.6+. In the beginning,
pypath has been developed only for Python 2.7. Then the code have been
adjusted to Py3 and for a few years we develop and test \sphinxcode{\sphinxupquote{pypath}} in
Python 3. Therefore this is the better supported Python variant.

\item[{documentation}] \leavevmode
\sphinxurl{http://saezlab.github.io/pypath}

\item[{issues}] \leavevmode
\sphinxurl{https://github.com/saezlab/pypath/issues}

\end{description}\end{quote}


\chapter{Installation}
\label{\detokenize{installation:installation}}\label{\detokenize{installation::doc}}

\section{Linux}
\label{\detokenize{installation:linux}}
In almost any up-to-date Linux distribution the dependencies of \sphinxstylestrong{pypath} are
built-in, or provided by the distributors. You only need to install a couple
of things in your package manager (cairo, py(2)cairo, igraph,
python(2)-igraph, graphviz, pygraphviz), and after install \sphinxstylestrong{pypath} by \sphinxstyleemphasis{pip}
(see below). If any module still missing, you can install them the usual way
by \sphinxstyleemphasis{pip} or your package manager.


\subsection{igraph C library, cairo and pycairo}
\label{\detokenize{installation:igraph-c-library-cairo-and-pycairo}}
\sphinxstyleemphasis{python(2)-igraph} is a Python interface to use the igraph C library. The
C library must be installed. The same goes for \sphinxstyleemphasis{cairo}, \sphinxstyleemphasis{py(2)cairo} and
\sphinxstyleemphasis{graphviz}.


\subsection{Directly from git}
\label{\detokenize{installation:directly-from-git}}
\begin{sphinxVerbatim}[commandchars=\\\{\}]
pip install git+https://github.com/saezlab/pypath.git
\end{sphinxVerbatim}


\subsection{With pip}
\label{\detokenize{installation:with-pip}}
Download the package from /dist, and install with pip:

\begin{sphinxVerbatim}[commandchars=\\\{\}]
pip install pypath\PYGZhy{}x.y.z.tar.gz
\end{sphinxVerbatim}


\subsection{Build source distribution}
\label{\detokenize{installation:build-source-distribution}}
Clone the git repo, and run setup.py:

\begin{sphinxVerbatim}[commandchars=\\\{\}]
python setup.py sdist
\end{sphinxVerbatim}


\section{Mac OS X}
\label{\detokenize{installation:mac-os-x}}
On OS X installation is not straightforward primarily because cairo needs to
be compiled from source. We provide 2 scripts here: the
\sphinxstylestrong{mac-install-brew.sh} installs everything with HomeBrew, and
\sphinxstylestrong{mac-install-conda.sh} installs from Anaconda distribution. With these
scripts installation of igraph, cairo and graphviz goes smoothly most of the
time, and options are available for omitting the 2 latter. To know more see
the description in the script header. There is a third script
\sphinxstylestrong{mac-install-source.sh} which compiles everything from source and presumes
only Python 2.7 and Xcode installed. We do not recommend this as it is time
consuming and troubleshooting requires expertise.


\subsection{Troubleshooting}
\label{\detokenize{installation:troubleshooting}}\begin{itemize}
\item {} 
\sphinxcode{\sphinxupquote{no module named ...}} when you try to load a module in Python. Did
theinstallation of the module run without error? Try to run again the specific
part from the mac install shell script to see if any error comes up. Is the
path where the module has been installed in your \sphinxcode{\sphinxupquote{\$PYTHONPATH}}? Try \sphinxcode{\sphinxupquote{echo
\$PYTHONPATH}} to see the current paths. Add your local install directories if
those are not there, e.g.
\sphinxcode{\sphinxupquote{export PYTHONPATH="/Users/me/local/python2.7/site-packages:\$PYTHONPATH"}}.
If it works afterwards, don’t forget to append these export path statements to
your \sphinxcode{\sphinxupquote{\textasciitilde{}/.bash\_profile}}, so these will be set every time you launch a new
shell.

\item {} 
\sphinxcode{\sphinxupquote{pkgconfig}} not found. Check if the \sphinxcode{\sphinxupquote{\$PKG\_CONFIG\_PATH}} variable is
set correctly, and pointing on a directory where pkgconfig really can be
found.

\item {} 
Error while trying to install py(2)cairo by pip. py(2)cairo could not be
installed by pip, but only by waf. Please set the \sphinxcode{\sphinxupquote{\$PKG\_CONFIG\_PATH}} before.
See \sphinxstylestrong{mac-install-source.sh} on how to install with waf.

\item {} 
Error at pygraphviz build: \sphinxcode{\sphinxupquote{graphviz/cgraph.h file not found}}. This is
because the directory of graphviz detected wrong by pkgconfig. See
\sphinxstylestrong{mac-install-source.sh} how to set include dirs and library dirs by
\sphinxcode{\sphinxupquote{-{-}global-option}} parameters.

\item {} 
Can not install bioservices, because installation of jurko-suds fails. Ok,
this fails because pip is not able to install the recent version of
setuptools, because a very old version present in the system path. The
development version of jurko-suds does not require setuptools, so you can
install it directly from git as it is done in \sphinxstylestrong{mac-install-source.sh}.

\item {} 
In \sphinxstylestrong{Anaconda}, \sphinxstyleemphasis{pypath} can be imported, but the modules and classes are
missing. Apparently Anaconda has some built-in stuff called \sphinxstyleemphasis{pypath}. This
has nothing to do with this module. Please be aware that Anaconda installs a
completely separated Python distribution, and does not detect modules in the
main Python installation. You need to install all modules within Anaconda’s
directory. \sphinxstylestrong{mac-install-conda.sh} does exactly this. If you still
experience issues, please contact us.

\end{itemize}


\section{Microsoft Windows}
\label{\detokenize{installation:microsoft-windows}}
Not many people have used \sphinxstyleemphasis{pypath} on Microsoft computers so far. Please share
your experiences and contact us if you encounter any issue. We appreciate
your feedback, and it would be nice to have better support for other computer
systems.


\subsection{With Anaconda}
\label{\detokenize{installation:with-anaconda}}
The same workflow like you see in \sphinxcode{\sphinxupquote{mac-install-conda.sh}} should work for
Anaconda on Windows. The only problem you certainly will encounter is that not
all the channels have packages for all platforms. If certain channel provides
no package for Windows, or for your Python version, you just need to find an
other one. For this, do a search:

\begin{sphinxVerbatim}[commandchars=\\\{\}]
anaconda search \PYGZhy{}t conda \PYGZlt{}package name\PYGZgt{}
\end{sphinxVerbatim}

For example, if you search for \sphinxstyleemphasis{pycairo}, you will find out that \sphinxstyleemphasis{vgauther}
provides it for osx-64, but only for Python 3.4, while \sphinxstyleemphasis{richlewis} provides
also for Python 3.5. And for win-64 platform, there is the channel of
\sphinxstyleemphasis{KristanAmstrong}. Go along all the commands in \sphinxcode{\sphinxupquote{mac-install-conda.sh}}, and
modify the channel if necessary, until all packages install successfully.


\subsection{With other Python distributions}
\label{\detokenize{installation:with-other-python-distributions}}
Here the basic principles are the same as everywhere: first try to install all
external dependencies, after \sphinxstyleemphasis{pip} install should work. On Windows certain
packages can not be installed by compiled from source by \sphinxstyleemphasis{pip}, instead the
easiest to install them precompiled. These are in our case \sphinxstyleemphasis{fisher, lxml,
numpy (mkl version), pycairo, igraph, pygraphviz, scipy and statsmodels}. The
precompiled packages are available here:
\sphinxurl{http://www.lfd.uci.edu/~gohlke/pythonlibs/}. We tested the setup with Python
3.4.3 and Python 2.7.11. The former should just work fine, while with the
latter we have issues to be resolved.


\subsection{Known issues}
\label{\detokenize{installation:known-issues}}\begin{itemize}
\item {} 
\sphinxstyleemphasis{“No module fabric available.”} \textendash{} or \sphinxstyleemphasis{pysftp} missing: this is not

\end{itemize}

important, only certain data download methods rely on these modules, but
likely you won’t call those at all.
* Progress indicator floods terminal: sorry about that, will be fixed soon.
* Encoding related exceptions in Python2: these might occur at some points in
the module, please send the traceback if you encounter one, and we will fix
as soon as possible.

\sphinxstyleemphasis{Special thanks to Jorge Ferreira for testing pypath on Windows!}


\chapter{Reference}
\label{\detokenize{reference:reference}}\label{\detokenize{reference::doc}}

\section{annot\_formats}
\label{\detokenize{reference:annot-formats}}

\section{annot}
\label{\detokenize{reference:module-pypath.core.annot}}\label{\detokenize{reference:annot}}\index{pypath.core.annot (module)@\spxentry{pypath.core.annot}\spxextra{module}}\index{Adhesome (class in pypath.core.annot)@\spxentry{Adhesome}\spxextra{class in pypath.core.annot}}

\begin{fulllineitems}
\phantomsection\label{\detokenize{reference:pypath.core.annot.Adhesome}}\pysiglinewithargsret{\sphinxbfcode{\sphinxupquote{class }}\sphinxcode{\sphinxupquote{pypath.core.annot.}}\sphinxbfcode{\sphinxupquote{Adhesome}}}{\emph{**kwargs}}{}
\end{fulllineitems}

\index{AnnotationBase (class in pypath.core.annot)@\spxentry{AnnotationBase}\spxextra{class in pypath.core.annot}}

\begin{fulllineitems}
\phantomsection\label{\detokenize{reference:pypath.core.annot.AnnotationBase}}\pysiglinewithargsret{\sphinxbfcode{\sphinxupquote{class }}\sphinxcode{\sphinxupquote{pypath.core.annot.}}\sphinxbfcode{\sphinxupquote{AnnotationBase}}}{\emph{name}, \emph{ncbi\_tax\_id=9606}, \emph{input\_method=None}, \emph{input\_args=None}, \emph{entity\_type='protein'}, \emph{swissprot\_only=True}, \emph{proteins=()}, \emph{complexes=()}, \emph{reference\_set=()}, \emph{infer\_complexes=True}, \emph{dump=None}, \emph{**kwargs}}{}~\index{add\_complexes\_by\_inference() (pypath.core.annot.AnnotationBase method)@\spxentry{add\_complexes\_by\_inference()}\spxextra{pypath.core.annot.AnnotationBase method}}

\begin{fulllineitems}
\phantomsection\label{\detokenize{reference:pypath.core.annot.AnnotationBase.add_complexes_by_inference}}\pysiglinewithargsret{\sphinxbfcode{\sphinxupquote{add\_complexes\_by\_inference}}}{\emph{complexes=None}}{}
Creates complex annotations by in silico inference and adds them
to this annotation set.

\end{fulllineitems}

\index{all\_proteins() (pypath.core.annot.AnnotationBase method)@\spxentry{all\_proteins()}\spxextra{pypath.core.annot.AnnotationBase method}}

\begin{fulllineitems}
\phantomsection\label{\detokenize{reference:pypath.core.annot.AnnotationBase.all_proteins}}\pysiglinewithargsret{\sphinxbfcode{\sphinxupquote{all\_proteins}}}{}{}
All UniProt IDs annotated in this resource.

\end{fulllineitems}

\index{annotate\_complex() (pypath.core.annot.AnnotationBase method)@\spxentry{annotate\_complex()}\spxextra{pypath.core.annot.AnnotationBase method}}

\begin{fulllineitems}
\phantomsection\label{\detokenize{reference:pypath.core.annot.AnnotationBase.annotate_complex}}\pysiglinewithargsret{\sphinxbfcode{\sphinxupquote{annotate\_complex}}}{\emph{cplex}}{}
Infers annotations for a single complex.

\end{fulllineitems}

\index{complex\_inference() (pypath.core.annot.AnnotationBase method)@\spxentry{complex\_inference()}\spxextra{pypath.core.annot.AnnotationBase method}}

\begin{fulllineitems}
\phantomsection\label{\detokenize{reference:pypath.core.annot.AnnotationBase.complex_inference}}\pysiglinewithargsret{\sphinxbfcode{\sphinxupquote{complex\_inference}}}{\emph{complexes=None}}{}
Annotates all complexes in \sphinxtitleref{complexes}, by default in the default
complex database (existing in the \sphinxtitleref{complex} module or generated
on demand according to the module’s current settings).

Dict with complexes as keys and sets of annotations as values.
Complexes with no valid information in this annotation resource
won’t be in the dict.
\begin{description}
\item[{complexes}] \leavevmode{[}iterable{]}
Iterable yielding complexes.

\end{description}

\end{fulllineitems}

\index{get\_subset() (pypath.core.annot.AnnotationBase method)@\spxentry{get\_subset()}\spxextra{pypath.core.annot.AnnotationBase method}}

\begin{fulllineitems}
\phantomsection\label{\detokenize{reference:pypath.core.annot.AnnotationBase.get_subset}}\pysiglinewithargsret{\sphinxbfcode{\sphinxupquote{get\_subset}}}{\emph{method=None}, \emph{**kwargs}}{}
Retrieves a subset by filtering based on \sphinxcode{\sphinxupquote{kwargs}}.
Each argument should be a name and a value or set of values.
Elements having the provided values in the annotation will be
returned.
Returns a set of UniProt IDs.

\end{fulllineitems}

\index{load\_proteins() (pypath.core.annot.AnnotationBase method)@\spxentry{load\_proteins()}\spxextra{pypath.core.annot.AnnotationBase method}}

\begin{fulllineitems}
\phantomsection\label{\detokenize{reference:pypath.core.annot.AnnotationBase.load_proteins}}\pysiglinewithargsret{\sphinxbfcode{\sphinxupquote{load\_proteins}}}{}{}
Retrieves a set of all UniProt IDs to have a base set of the entire
proteome.

\end{fulllineitems}

\index{reload() (pypath.core.annot.AnnotationBase method)@\spxentry{reload()}\spxextra{pypath.core.annot.AnnotationBase method}}

\begin{fulllineitems}
\phantomsection\label{\detokenize{reference:pypath.core.annot.AnnotationBase.reload}}\pysiglinewithargsret{\sphinxbfcode{\sphinxupquote{reload}}}{}{}
Reloads the object from the module level.

\end{fulllineitems}


\end{fulllineitems}

\index{Baccin2019 (class in pypath.core.annot)@\spxentry{Baccin2019}\spxextra{class in pypath.core.annot}}

\begin{fulllineitems}
\phantomsection\label{\detokenize{reference:pypath.core.annot.Baccin2019}}\pysiglinewithargsret{\sphinxbfcode{\sphinxupquote{class }}\sphinxcode{\sphinxupquote{pypath.core.annot.}}\sphinxbfcode{\sphinxupquote{Baccin2019}}}{\emph{ncbi\_tax\_id=9606}, \emph{**kwargs}}{}
\end{fulllineitems}

\index{CancerGeneCensus (class in pypath.core.annot)@\spxentry{CancerGeneCensus}\spxextra{class in pypath.core.annot}}

\begin{fulllineitems}
\phantomsection\label{\detokenize{reference:pypath.core.annot.CancerGeneCensus}}\pysiglinewithargsret{\sphinxbfcode{\sphinxupquote{class }}\sphinxcode{\sphinxupquote{pypath.core.annot.}}\sphinxbfcode{\sphinxupquote{CancerGeneCensus}}}{\emph{**kwargs}}{}
\end{fulllineitems}

\index{Cancersea (class in pypath.core.annot)@\spxentry{Cancersea}\spxextra{class in pypath.core.annot}}

\begin{fulllineitems}
\phantomsection\label{\detokenize{reference:pypath.core.annot.Cancersea}}\pysiglinewithargsret{\sphinxbfcode{\sphinxupquote{class }}\sphinxcode{\sphinxupquote{pypath.core.annot.}}\sphinxbfcode{\sphinxupquote{Cancersea}}}{\emph{**kwargs}}{}
\end{fulllineitems}

\index{CellPhoneDB (class in pypath.core.annot)@\spxentry{CellPhoneDB}\spxextra{class in pypath.core.annot}}

\begin{fulllineitems}
\phantomsection\label{\detokenize{reference:pypath.core.annot.CellPhoneDB}}\pysiglinewithargsret{\sphinxbfcode{\sphinxupquote{class }}\sphinxcode{\sphinxupquote{pypath.core.annot.}}\sphinxbfcode{\sphinxupquote{CellPhoneDB}}}{\emph{**kwargs}}{}~\index{record (pypath.core.annot.CellPhoneDB attribute)@\spxentry{record}\spxextra{pypath.core.annot.CellPhoneDB attribute}}

\begin{fulllineitems}
\phantomsection\label{\detokenize{reference:pypath.core.annot.CellPhoneDB.record}}\pysigline{\sphinxbfcode{\sphinxupquote{record}}}
alias of {\hyperref[\detokenize{reference:pypath.inputs.main.CellPhoneDBAnnotation}]{\sphinxcrossref{\sphinxcode{\sphinxupquote{pypath.inputs.main.CellPhoneDBAnnotation}}}}}

\end{fulllineitems}


\end{fulllineitems}

\index{CellPhoneDBComplex (class in pypath.core.annot)@\spxentry{CellPhoneDBComplex}\spxextra{class in pypath.core.annot}}

\begin{fulllineitems}
\phantomsection\label{\detokenize{reference:pypath.core.annot.CellPhoneDBComplex}}\pysiglinewithargsret{\sphinxbfcode{\sphinxupquote{class }}\sphinxcode{\sphinxupquote{pypath.core.annot.}}\sphinxbfcode{\sphinxupquote{CellPhoneDBComplex}}}{\emph{**kwargs}}{}
\end{fulllineitems}

\index{CellSurfaceProteinAtlas (class in pypath.core.annot)@\spxentry{CellSurfaceProteinAtlas}\spxextra{class in pypath.core.annot}}

\begin{fulllineitems}
\phantomsection\label{\detokenize{reference:pypath.core.annot.CellSurfaceProteinAtlas}}\pysiglinewithargsret{\sphinxbfcode{\sphinxupquote{class }}\sphinxcode{\sphinxupquote{pypath.core.annot.}}\sphinxbfcode{\sphinxupquote{CellSurfaceProteinAtlas}}}{\emph{ncbi\_tax\_id=9606}, \emph{**kwargs}}{}
\end{fulllineitems}

\index{Comppi (class in pypath.core.annot)@\spxentry{Comppi}\spxextra{class in pypath.core.annot}}

\begin{fulllineitems}
\phantomsection\label{\detokenize{reference:pypath.core.annot.Comppi}}\pysiglinewithargsret{\sphinxbfcode{\sphinxupquote{class }}\sphinxcode{\sphinxupquote{pypath.core.annot.}}\sphinxbfcode{\sphinxupquote{Comppi}}}{\emph{**kwargs}}{}
\end{fulllineitems}

\index{Corum (class in pypath.core.annot)@\spxentry{Corum}\spxextra{class in pypath.core.annot}}

\begin{fulllineitems}
\phantomsection\label{\detokenize{reference:pypath.core.annot.Corum}}\pysiglinewithargsret{\sphinxbfcode{\sphinxupquote{class }}\sphinxcode{\sphinxupquote{pypath.core.annot.}}\sphinxbfcode{\sphinxupquote{Corum}}}{\emph{name}, \emph{annot\_attr}, \emph{**kwargs}}{}
\end{fulllineitems}

\index{CorumFuncat (class in pypath.core.annot)@\spxentry{CorumFuncat}\spxextra{class in pypath.core.annot}}

\begin{fulllineitems}
\phantomsection\label{\detokenize{reference:pypath.core.annot.CorumFuncat}}\pysiglinewithargsret{\sphinxbfcode{\sphinxupquote{class }}\sphinxcode{\sphinxupquote{pypath.core.annot.}}\sphinxbfcode{\sphinxupquote{CorumFuncat}}}{\emph{**kwargs}}{}
\end{fulllineitems}

\index{CorumGO (class in pypath.core.annot)@\spxentry{CorumGO}\spxextra{class in pypath.core.annot}}

\begin{fulllineitems}
\phantomsection\label{\detokenize{reference:pypath.core.annot.CorumGO}}\pysiglinewithargsret{\sphinxbfcode{\sphinxupquote{class }}\sphinxcode{\sphinxupquote{pypath.core.annot.}}\sphinxbfcode{\sphinxupquote{CorumGO}}}{\emph{**kwargs}}{}
\end{fulllineitems}

\index{Cpad (class in pypath.core.annot)@\spxentry{Cpad}\spxextra{class in pypath.core.annot}}

\begin{fulllineitems}
\phantomsection\label{\detokenize{reference:pypath.core.annot.Cpad}}\pysiglinewithargsret{\sphinxbfcode{\sphinxupquote{class }}\sphinxcode{\sphinxupquote{pypath.core.annot.}}\sphinxbfcode{\sphinxupquote{Cpad}}}{\emph{ncbi\_tax\_id=9606}, \emph{**kwargs}}{}
\end{fulllineitems}

\index{Dgidb (class in pypath.core.annot)@\spxentry{Dgidb}\spxextra{class in pypath.core.annot}}

\begin{fulllineitems}
\phantomsection\label{\detokenize{reference:pypath.core.annot.Dgidb}}\pysiglinewithargsret{\sphinxbfcode{\sphinxupquote{class }}\sphinxcode{\sphinxupquote{pypath.core.annot.}}\sphinxbfcode{\sphinxupquote{Dgidb}}}{\emph{**kwargs}}{}
\end{fulllineitems}

\index{Disgenet (class in pypath.core.annot)@\spxentry{Disgenet}\spxextra{class in pypath.core.annot}}

\begin{fulllineitems}
\phantomsection\label{\detokenize{reference:pypath.core.annot.Disgenet}}\pysiglinewithargsret{\sphinxbfcode{\sphinxupquote{class }}\sphinxcode{\sphinxupquote{pypath.core.annot.}}\sphinxbfcode{\sphinxupquote{Disgenet}}}{\emph{ncbi\_tax\_id=9606}, \emph{**kwargs}}{}
\end{fulllineitems}

\index{Exocarta (class in pypath.core.annot)@\spxentry{Exocarta}\spxextra{class in pypath.core.annot}}

\begin{fulllineitems}
\phantomsection\label{\detokenize{reference:pypath.core.annot.Exocarta}}\pysiglinewithargsret{\sphinxbfcode{\sphinxupquote{class }}\sphinxcode{\sphinxupquote{pypath.core.annot.}}\sphinxbfcode{\sphinxupquote{Exocarta}}}{\emph{ncbi\_tax\_id=9606}, \emph{**kwargs}}{}
\end{fulllineitems}

\index{GOIntercell (class in pypath.core.annot)@\spxentry{GOIntercell}\spxextra{class in pypath.core.annot}}

\begin{fulllineitems}
\phantomsection\label{\detokenize{reference:pypath.core.annot.GOIntercell}}\pysiglinewithargsret{\sphinxbfcode{\sphinxupquote{class }}\sphinxcode{\sphinxupquote{pypath.core.annot.}}\sphinxbfcode{\sphinxupquote{GOIntercell}}}{\emph{categories=None}, \emph{go\_annot=None}, \emph{ncbi\_tax\_id=9606}, \emph{**kwargs}}{}
\end{fulllineitems}

\index{GuideToPharmacology (class in pypath.core.annot)@\spxentry{GuideToPharmacology}\spxextra{class in pypath.core.annot}}

\begin{fulllineitems}
\phantomsection\label{\detokenize{reference:pypath.core.annot.GuideToPharmacology}}\pysiglinewithargsret{\sphinxbfcode{\sphinxupquote{class }}\sphinxcode{\sphinxupquote{pypath.core.annot.}}\sphinxbfcode{\sphinxupquote{GuideToPharmacology}}}{\emph{load\_sources=False}, \emph{**kwargs}}{}
\end{fulllineitems}

\index{Hgnc (class in pypath.core.annot)@\spxentry{Hgnc}\spxextra{class in pypath.core.annot}}

\begin{fulllineitems}
\phantomsection\label{\detokenize{reference:pypath.core.annot.Hgnc}}\pysiglinewithargsret{\sphinxbfcode{\sphinxupquote{class }}\sphinxcode{\sphinxupquote{pypath.core.annot.}}\sphinxbfcode{\sphinxupquote{Hgnc}}}{\emph{**kwargs}}{}
\end{fulllineitems}

\index{HpmrComplex (class in pypath.core.annot)@\spxentry{HpmrComplex}\spxextra{class in pypath.core.annot}}

\begin{fulllineitems}
\phantomsection\label{\detokenize{reference:pypath.core.annot.HpmrComplex}}\pysiglinewithargsret{\sphinxbfcode{\sphinxupquote{class }}\sphinxcode{\sphinxupquote{pypath.core.annot.}}\sphinxbfcode{\sphinxupquote{HpmrComplex}}}{\emph{**kwargs}}{}
\end{fulllineitems}

\index{HumanPlasmaMembraneReceptome (class in pypath.core.annot)@\spxentry{HumanPlasmaMembraneReceptome}\spxextra{class in pypath.core.annot}}

\begin{fulllineitems}
\phantomsection\label{\detokenize{reference:pypath.core.annot.HumanPlasmaMembraneReceptome}}\pysiglinewithargsret{\sphinxbfcode{\sphinxupquote{class }}\sphinxcode{\sphinxupquote{pypath.core.annot.}}\sphinxbfcode{\sphinxupquote{HumanPlasmaMembraneReceptome}}}{\emph{**kwargs}}{}
\end{fulllineitems}

\index{HumanProteinAtlas (class in pypath.core.annot)@\spxentry{HumanProteinAtlas}\spxextra{class in pypath.core.annot}}

\begin{fulllineitems}
\phantomsection\label{\detokenize{reference:pypath.core.annot.HumanProteinAtlas}}\pysiglinewithargsret{\sphinxbfcode{\sphinxupquote{class }}\sphinxcode{\sphinxupquote{pypath.core.annot.}}\sphinxbfcode{\sphinxupquote{HumanProteinAtlas}}}{\emph{**kwargs}}{}
\end{fulllineitems}

\index{HumanProteinAtlasSecretome (class in pypath.core.annot)@\spxentry{HumanProteinAtlasSecretome}\spxextra{class in pypath.core.annot}}

\begin{fulllineitems}
\phantomsection\label{\detokenize{reference:pypath.core.annot.HumanProteinAtlasSecretome}}\pysiglinewithargsret{\sphinxbfcode{\sphinxupquote{class }}\sphinxcode{\sphinxupquote{pypath.core.annot.}}\sphinxbfcode{\sphinxupquote{HumanProteinAtlasSecretome}}}{\emph{**kwargs}}{}
\end{fulllineitems}

\index{HumanProteinAtlasSubcellular (class in pypath.core.annot)@\spxentry{HumanProteinAtlasSubcellular}\spxextra{class in pypath.core.annot}}

\begin{fulllineitems}
\phantomsection\label{\detokenize{reference:pypath.core.annot.HumanProteinAtlasSubcellular}}\pysiglinewithargsret{\sphinxbfcode{\sphinxupquote{class }}\sphinxcode{\sphinxupquote{pypath.core.annot.}}\sphinxbfcode{\sphinxupquote{HumanProteinAtlasSubcellular}}}{\emph{**kwargs}}{}
\end{fulllineitems}

\index{Integrins (class in pypath.core.annot)@\spxentry{Integrins}\spxextra{class in pypath.core.annot}}

\begin{fulllineitems}
\phantomsection\label{\detokenize{reference:pypath.core.annot.Integrins}}\pysiglinewithargsret{\sphinxbfcode{\sphinxupquote{class }}\sphinxcode{\sphinxupquote{pypath.core.annot.}}\sphinxbfcode{\sphinxupquote{Integrins}}}{\emph{**kwargs}}{}
\end{fulllineitems}

\index{Intogen (class in pypath.core.annot)@\spxentry{Intogen}\spxextra{class in pypath.core.annot}}

\begin{fulllineitems}
\phantomsection\label{\detokenize{reference:pypath.core.annot.Intogen}}\pysiglinewithargsret{\sphinxbfcode{\sphinxupquote{class }}\sphinxcode{\sphinxupquote{pypath.core.annot.}}\sphinxbfcode{\sphinxupquote{Intogen}}}{\emph{**kwargs}}{}
\end{fulllineitems}

\index{KeggPathways (class in pypath.core.annot)@\spxentry{KeggPathways}\spxextra{class in pypath.core.annot}}

\begin{fulllineitems}
\phantomsection\label{\detokenize{reference:pypath.core.annot.KeggPathways}}\pysiglinewithargsret{\sphinxbfcode{\sphinxupquote{class }}\sphinxcode{\sphinxupquote{pypath.core.annot.}}\sphinxbfcode{\sphinxupquote{KeggPathways}}}{\emph{ncbi\_tax\_id=9606}, \emph{**kwargs}}{}
\end{fulllineitems}

\index{Kinasedotcom (class in pypath.core.annot)@\spxentry{Kinasedotcom}\spxextra{class in pypath.core.annot}}

\begin{fulllineitems}
\phantomsection\label{\detokenize{reference:pypath.core.annot.Kinasedotcom}}\pysiglinewithargsret{\sphinxbfcode{\sphinxupquote{class }}\sphinxcode{\sphinxupquote{pypath.core.annot.}}\sphinxbfcode{\sphinxupquote{Kinasedotcom}}}{\emph{**kwargs}}{}
\end{fulllineitems}

\index{Kirouac2010 (class in pypath.core.annot)@\spxentry{Kirouac2010}\spxextra{class in pypath.core.annot}}

\begin{fulllineitems}
\phantomsection\label{\detokenize{reference:pypath.core.annot.Kirouac2010}}\pysiglinewithargsret{\sphinxbfcode{\sphinxupquote{class }}\sphinxcode{\sphinxupquote{pypath.core.annot.}}\sphinxbfcode{\sphinxupquote{Kirouac2010}}}{\emph{load\_sources=False}, \emph{**kwargs}}{}
\end{fulllineitems}

\index{LigandReceptor (class in pypath.core.annot)@\spxentry{LigandReceptor}\spxextra{class in pypath.core.annot}}

\begin{fulllineitems}
\phantomsection\label{\detokenize{reference:pypath.core.annot.LigandReceptor}}\pysiglinewithargsret{\sphinxbfcode{\sphinxupquote{class }}\sphinxcode{\sphinxupquote{pypath.core.annot.}}\sphinxbfcode{\sphinxupquote{LigandReceptor}}}{\emph{name}, \emph{ligand\_col=None}, \emph{receptor\_col=None}, \emph{ligand\_id\_type=None}, \emph{receptor\_id\_type=None}, \emph{record\_processor\_method=None}, \emph{record\_extra\_fields=None}, \emph{record\_defaults=None}, \emph{extra\_fields\_methods=None}, \emph{**kwargs}}{}
\end{fulllineitems}

\index{Locate (class in pypath.core.annot)@\spxentry{Locate}\spxextra{class in pypath.core.annot}}

\begin{fulllineitems}
\phantomsection\label{\detokenize{reference:pypath.core.annot.Locate}}\pysiglinewithargsret{\sphinxbfcode{\sphinxupquote{class }}\sphinxcode{\sphinxupquote{pypath.core.annot.}}\sphinxbfcode{\sphinxupquote{Locate}}}{\emph{ncbi\_tax\_id=9606}, \emph{literature=True}, \emph{external=True}, \emph{predictions=False}, \emph{**kwargs}}{}
\end{fulllineitems}

\index{Lrdb (class in pypath.core.annot)@\spxentry{Lrdb}\spxextra{class in pypath.core.annot}}

\begin{fulllineitems}
\phantomsection\label{\detokenize{reference:pypath.core.annot.Lrdb}}\pysiglinewithargsret{\sphinxbfcode{\sphinxupquote{class }}\sphinxcode{\sphinxupquote{pypath.core.annot.}}\sphinxbfcode{\sphinxupquote{Lrdb}}}{\emph{**kwargs}}{}
\end{fulllineitems}

\index{Matrisome (class in pypath.core.annot)@\spxentry{Matrisome}\spxextra{class in pypath.core.annot}}

\begin{fulllineitems}
\phantomsection\label{\detokenize{reference:pypath.core.annot.Matrisome}}\pysiglinewithargsret{\sphinxbfcode{\sphinxupquote{class }}\sphinxcode{\sphinxupquote{pypath.core.annot.}}\sphinxbfcode{\sphinxupquote{Matrisome}}}{\emph{ncbi\_tax\_id=9606}, \emph{**kwargs}}{}
\end{fulllineitems}

\index{Matrixdb (class in pypath.core.annot)@\spxentry{Matrixdb}\spxextra{class in pypath.core.annot}}

\begin{fulllineitems}
\phantomsection\label{\detokenize{reference:pypath.core.annot.Matrixdb}}\pysiglinewithargsret{\sphinxbfcode{\sphinxupquote{class }}\sphinxcode{\sphinxupquote{pypath.core.annot.}}\sphinxbfcode{\sphinxupquote{Matrixdb}}}{\emph{ncbi\_tax\_id=9606}, \emph{**kwargs}}{}
\end{fulllineitems}

\index{Membranome (class in pypath.core.annot)@\spxentry{Membranome}\spxextra{class in pypath.core.annot}}

\begin{fulllineitems}
\phantomsection\label{\detokenize{reference:pypath.core.annot.Membranome}}\pysiglinewithargsret{\sphinxbfcode{\sphinxupquote{class }}\sphinxcode{\sphinxupquote{pypath.core.annot.}}\sphinxbfcode{\sphinxupquote{Membranome}}}{\emph{**kwargs}}{}
\end{fulllineitems}

\index{Msigdb (class in pypath.core.annot)@\spxentry{Msigdb}\spxextra{class in pypath.core.annot}}

\begin{fulllineitems}
\phantomsection\label{\detokenize{reference:pypath.core.annot.Msigdb}}\pysiglinewithargsret{\sphinxbfcode{\sphinxupquote{class }}\sphinxcode{\sphinxupquote{pypath.core.annot.}}\sphinxbfcode{\sphinxupquote{Msigdb}}}{\emph{ncbi\_tax\_id=9606}, \emph{**kwargs}}{}
\end{fulllineitems}

\index{NetpathPathways (class in pypath.core.annot)@\spxentry{NetpathPathways}\spxextra{class in pypath.core.annot}}

\begin{fulllineitems}
\phantomsection\label{\detokenize{reference:pypath.core.annot.NetpathPathways}}\pysiglinewithargsret{\sphinxbfcode{\sphinxupquote{class }}\sphinxcode{\sphinxupquote{pypath.core.annot.}}\sphinxbfcode{\sphinxupquote{NetpathPathways}}}{\emph{ncbi\_tax\_id=9606}, \emph{**kwargs}}{}
\end{fulllineitems}

\index{Opm (class in pypath.core.annot)@\spxentry{Opm}\spxextra{class in pypath.core.annot}}

\begin{fulllineitems}
\phantomsection\label{\detokenize{reference:pypath.core.annot.Opm}}\pysiglinewithargsret{\sphinxbfcode{\sphinxupquote{class }}\sphinxcode{\sphinxupquote{pypath.core.annot.}}\sphinxbfcode{\sphinxupquote{Opm}}}{\emph{ncbi\_tax\_id=9606}, \emph{**kwargs}}{}
\end{fulllineitems}

\index{Phosphatome (class in pypath.core.annot)@\spxentry{Phosphatome}\spxextra{class in pypath.core.annot}}

\begin{fulllineitems}
\phantomsection\label{\detokenize{reference:pypath.core.annot.Phosphatome}}\pysiglinewithargsret{\sphinxbfcode{\sphinxupquote{class }}\sphinxcode{\sphinxupquote{pypath.core.annot.}}\sphinxbfcode{\sphinxupquote{Phosphatome}}}{\emph{**kwargs}}{}
\end{fulllineitems}

\index{Ramilowski2015 (class in pypath.core.annot)@\spxentry{Ramilowski2015}\spxextra{class in pypath.core.annot}}

\begin{fulllineitems}
\phantomsection\label{\detokenize{reference:pypath.core.annot.Ramilowski2015}}\pysiglinewithargsret{\sphinxbfcode{\sphinxupquote{class }}\sphinxcode{\sphinxupquote{pypath.core.annot.}}\sphinxbfcode{\sphinxupquote{Ramilowski2015}}}{\emph{load\_sources=False}, \emph{**kwargs}}{}
\end{fulllineitems}

\index{Ramilowski2015Location (class in pypath.core.annot)@\spxentry{Ramilowski2015Location}\spxextra{class in pypath.core.annot}}

\begin{fulllineitems}
\phantomsection\label{\detokenize{reference:pypath.core.annot.Ramilowski2015Location}}\pysiglinewithargsret{\sphinxbfcode{\sphinxupquote{class }}\sphinxcode{\sphinxupquote{pypath.core.annot.}}\sphinxbfcode{\sphinxupquote{Ramilowski2015Location}}}{\emph{**kwargs}}{}
\end{fulllineitems}

\index{SignalinkPathways (class in pypath.core.annot)@\spxentry{SignalinkPathways}\spxextra{class in pypath.core.annot}}

\begin{fulllineitems}
\phantomsection\label{\detokenize{reference:pypath.core.annot.SignalinkPathways}}\pysiglinewithargsret{\sphinxbfcode{\sphinxupquote{class }}\sphinxcode{\sphinxupquote{pypath.core.annot.}}\sphinxbfcode{\sphinxupquote{SignalinkPathways}}}{\emph{ncbi\_tax\_id=9606}, \emph{**kwargs}}{}
\end{fulllineitems}

\index{SignorPathways (class in pypath.core.annot)@\spxentry{SignorPathways}\spxextra{class in pypath.core.annot}}

\begin{fulllineitems}
\phantomsection\label{\detokenize{reference:pypath.core.annot.SignorPathways}}\pysiglinewithargsret{\sphinxbfcode{\sphinxupquote{class }}\sphinxcode{\sphinxupquote{pypath.core.annot.}}\sphinxbfcode{\sphinxupquote{SignorPathways}}}{\emph{ncbi\_tax\_id=9606}, \emph{**kwargs}}{}
\end{fulllineitems}

\index{Surfaceome (class in pypath.core.annot)@\spxentry{Surfaceome}\spxextra{class in pypath.core.annot}}

\begin{fulllineitems}
\phantomsection\label{\detokenize{reference:pypath.core.annot.Surfaceome}}\pysiglinewithargsret{\sphinxbfcode{\sphinxupquote{class }}\sphinxcode{\sphinxupquote{pypath.core.annot.}}\sphinxbfcode{\sphinxupquote{Surfaceome}}}{\emph{**kwargs}}{}
\end{fulllineitems}

\index{Tfcensus (class in pypath.core.annot)@\spxentry{Tfcensus}\spxextra{class in pypath.core.annot}}

\begin{fulllineitems}
\phantomsection\label{\detokenize{reference:pypath.core.annot.Tfcensus}}\pysiglinewithargsret{\sphinxbfcode{\sphinxupquote{class }}\sphinxcode{\sphinxupquote{pypath.core.annot.}}\sphinxbfcode{\sphinxupquote{Tfcensus}}}{\emph{**kwargs}}{}
\end{fulllineitems}

\index{Topdb (class in pypath.core.annot)@\spxentry{Topdb}\spxextra{class in pypath.core.annot}}

\begin{fulllineitems}
\phantomsection\label{\detokenize{reference:pypath.core.annot.Topdb}}\pysiglinewithargsret{\sphinxbfcode{\sphinxupquote{class }}\sphinxcode{\sphinxupquote{pypath.core.annot.}}\sphinxbfcode{\sphinxupquote{Topdb}}}{\emph{ncbi\_tax\_id=9606}, \emph{**kwargs}}{}
\end{fulllineitems}

\index{Vesiclepedia (class in pypath.core.annot)@\spxentry{Vesiclepedia}\spxextra{class in pypath.core.annot}}

\begin{fulllineitems}
\phantomsection\label{\detokenize{reference:pypath.core.annot.Vesiclepedia}}\pysiglinewithargsret{\sphinxbfcode{\sphinxupquote{class }}\sphinxcode{\sphinxupquote{pypath.core.annot.}}\sphinxbfcode{\sphinxupquote{Vesiclepedia}}}{\emph{ncbi\_tax\_id=9606}, \emph{**kwargs}}{}
\end{fulllineitems}

\index{Zhong2015 (class in pypath.core.annot)@\spxentry{Zhong2015}\spxextra{class in pypath.core.annot}}

\begin{fulllineitems}
\phantomsection\label{\detokenize{reference:pypath.core.annot.Zhong2015}}\pysiglinewithargsret{\sphinxbfcode{\sphinxupquote{class }}\sphinxcode{\sphinxupquote{pypath.core.annot.}}\sphinxbfcode{\sphinxupquote{Zhong2015}}}{\emph{**kwargs}}{}
\end{fulllineitems}

\index{get\_db() (in module pypath.core.annot)@\spxentry{get\_db()}\spxextra{in module pypath.core.annot}}

\begin{fulllineitems}
\phantomsection\label{\detokenize{reference:pypath.core.annot.get_db}}\pysiglinewithargsret{\sphinxcode{\sphinxupquote{pypath.core.annot.}}\sphinxbfcode{\sphinxupquote{get\_db}}}{\emph{keep\_annotators=True}, \emph{create\_dataframe=False}, \emph{use\_complexes=True}, \emph{**kwargs}}{}
Retrieves the current database instance and initializes it if does
not exist yet.

\end{fulllineitems}

\index{init\_db() (in module pypath.core.annot)@\spxentry{init\_db()}\spxextra{in module pypath.core.annot}}

\begin{fulllineitems}
\phantomsection\label{\detokenize{reference:pypath.core.annot.init_db}}\pysiglinewithargsret{\sphinxcode{\sphinxupquote{pypath.core.annot.}}\sphinxbfcode{\sphinxupquote{init\_db}}}{\emph{keep\_annotators=True}, \emph{create\_dataframe=False}, \emph{use\_complexes=True}, \emph{**kwargs}}{}
Initializes or reloads the annotation database.
The database will be assigned to the \sphinxcode{\sphinxupquote{db}} attribute of this module.

\end{fulllineitems}



\section{bel}
\label{\detokenize{reference:bel}}

\section{cache}
\label{\detokenize{reference:module-pypath.share.cache}}\label{\detokenize{reference:cache}}\index{pypath.share.cache (module)@\spxentry{pypath.share.cache}\spxextra{module}}\index{get\_cachedir() (in module pypath.share.cache)@\spxentry{get\_cachedir()}\spxextra{in module pypath.share.cache}}

\begin{fulllineitems}
\phantomsection\label{\detokenize{reference:pypath.share.cache.get_cachedir}}\pysiglinewithargsret{\sphinxcode{\sphinxupquote{pypath.share.cache.}}\sphinxbfcode{\sphinxupquote{get\_cachedir}}}{\emph{cachedir=None}}{}
Ensures the cache directory exists and returns its path.

\end{fulllineitems}



\section{cellphonedb}
\label{\detokenize{reference:cellphonedb}}

\section{common}
\label{\detokenize{reference:module-pypath.share.common}}\label{\detokenize{reference:common}}\index{pypath.share.common (module)@\spxentry{pypath.share.common}\spxextra{module}}\index{uniq\_list() (in module pypath.share.common)@\spxentry{uniq\_list()}\spxextra{in module pypath.share.common}}

\begin{fulllineitems}
\phantomsection\label{\detokenize{reference:pypath.share.common.uniq_list}}\pysiglinewithargsret{\sphinxcode{\sphinxupquote{pypath.share.common.}}\sphinxbfcode{\sphinxupquote{uniq\_list}}}{\emph{seq}}{}
Reduces a list to its unique elements.

Takes any iterable and returns a list of unique elements on it. If
the argument is a dictionary, returns a list of unique keys.
\sphinxstylestrong{NOTE:} Does not preserve the order of the elements.
\begin{quote}\begin{description}
\item[{Parameters}] \leavevmode
\sphinxstyleliteralstrong{\sphinxupquote{seq}} (\sphinxstyleliteralemphasis{\sphinxupquote{list}}) \textendash{} Sequence to be processed, can be any iterable type.

\item[{Returns}] \leavevmode
(\sphinxstyleemphasis{list}) \textendash{} List of unique elements in the sequence \sphinxstyleemphasis{seq}.

\end{description}\end{quote}
\begin{description}
\item[{\sphinxstylestrong{Examples:}}] \leavevmode
\begin{sphinxVerbatim}[commandchars=\\\{\}]
\PYG{g+gp}{\PYGZgt{}\PYGZgt{}\PYGZgt{} }\PYG{n}{uniq\PYGZus{}list}\PYG{p}{(}\PYG{l+s+s1}{\PYGZsq{}}\PYG{l+s+s1}{aba}\PYG{l+s+s1}{\PYGZsq{}}\PYG{p}{)}
\PYG{g+go}{[\PYGZsq{}a\PYGZsq{}, \PYGZsq{}b\PYGZsq{}]}
\PYG{g+gp}{\PYGZgt{}\PYGZgt{}\PYGZgt{} }\PYG{n}{uniq\PYGZus{}list}\PYG{p}{(}\PYG{p}{[}\PYG{l+m+mi}{0}\PYG{p}{,} \PYG{l+m+mi}{1}\PYG{p}{,} \PYG{l+m+mi}{2}\PYG{p}{,} \PYG{l+m+mi}{1}\PYG{p}{,} \PYG{l+m+mi}{0}\PYG{p}{]}\PYG{p}{)}
\PYG{g+go}{[0, 1, 2]}
\end{sphinxVerbatim}

\end{description}

\end{fulllineitems}

\index{add\_to\_list() (in module pypath.share.common)@\spxentry{add\_to\_list()}\spxextra{in module pypath.share.common}}

\begin{fulllineitems}
\phantomsection\label{\detokenize{reference:pypath.share.common.add_to_list}}\pysiglinewithargsret{\sphinxcode{\sphinxupquote{pypath.share.common.}}\sphinxbfcode{\sphinxupquote{add\_to\_list}}}{\emph{lst}, \emph{toadd}}{}
Adds elements to a list.

Appends \sphinxstyleemphasis{toadd} to \sphinxstyleemphasis{lst}. Function differs from
\sphinxcode{\sphinxupquote{list.append()}} since is capable to handle different data
types. This is, if \sphinxstyleemphasis{lst} is not a list, it will be converted to one.
Similarly, if \sphinxstyleemphasis{toadd} is not a list, it will be converted and added.
If \sphinxstyleemphasis{toadd} is or contains \sphinxcode{\sphinxupquote{None}}, these will be ommited. The
returned list will only contain unique elements and does not
necessarily preserve order.
\begin{quote}\begin{description}
\item[{Parameters}] \leavevmode\begin{itemize}
\item {} 
\sphinxstyleliteralstrong{\sphinxupquote{lst}} (\sphinxstyleliteralemphasis{\sphinxupquote{list}}) \textendash{} List or any other type (will be converted into a list). Original
sequence to which \sphinxstyleemphasis{toadd} will be appended.

\item {} 
\sphinxstyleliteralstrong{\sphinxupquote{toadd}} (\sphinxstyleliteralemphasis{\sphinxupquote{any}}) \textendash{} Element(s) to be added into \sphinxstyleemphasis{lst}.

\end{itemize}

\item[{Returns}] \leavevmode
(\sphinxstyleemphasis{list}) \textendash{} Contains the unique element(s) from the union of
\sphinxstyleemphasis{lst} and \sphinxstyleemphasis{toadd}. \sphinxstylestrong{NOTE:} Makes use of
\sphinxcode{\sphinxupquote{common.uniq\_list()}}, does not preserve order of elements.

\end{description}\end{quote}
\begin{description}
\item[{\sphinxstylestrong{Examples:}}] \leavevmode
\begin{sphinxVerbatim}[commandchars=\\\{\}]
\PYG{g+gp}{\PYGZgt{}\PYGZgt{}\PYGZgt{} }\PYG{n}{add\PYGZus{}to\PYGZus{}list}\PYG{p}{(}\PYG{l+s+s1}{\PYGZsq{}}\PYG{l+s+s1}{ab}\PYG{l+s+s1}{\PYGZsq{}}\PYG{p}{,} \PYG{l+s+s1}{\PYGZsq{}}\PYG{l+s+s1}{cd}\PYG{l+s+s1}{\PYGZsq{}}\PYG{p}{)}
\PYG{g+go}{[\PYGZsq{}ab\PYGZsq{}, \PYGZsq{}cd\PYGZsq{}]}
\PYG{g+gp}{\PYGZgt{}\PYGZgt{}\PYGZgt{} }\PYG{n}{add\PYGZus{}to\PYGZus{}list}\PYG{p}{(}\PYG{l+s+s1}{\PYGZsq{}}\PYG{l+s+s1}{ab}\PYG{l+s+s1}{\PYGZsq{}}\PYG{p}{,} \PYG{p}{[}\PYG{l+s+s1}{\PYGZsq{}}\PYG{l+s+s1}{cd}\PYG{l+s+s1}{\PYGZsq{}}\PYG{p}{,} \PYG{k+kc}{None}\PYG{p}{,} \PYG{l+s+s1}{\PYGZsq{}}\PYG{l+s+s1}{ab}\PYG{l+s+s1}{\PYGZsq{}}\PYG{p}{,} \PYG{l+s+s1}{\PYGZsq{}}\PYG{l+s+s1}{ef}\PYG{l+s+s1}{\PYGZsq{}}\PYG{p}{]}\PYG{p}{)}
\PYG{g+go}{[\PYGZsq{}ab\PYGZsq{}, \PYGZsq{}ef\PYGZsq{}, \PYGZsq{}cd\PYGZsq{}]}
\PYG{g+gp}{\PYGZgt{}\PYGZgt{}\PYGZgt{} }\PYG{n}{add\PYGZus{}to\PYGZus{}list}\PYG{p}{(}\PYG{p}{(}\PYG{l+m+mi}{0}\PYG{p}{,} \PYG{l+m+mi}{1}\PYG{p}{,} \PYG{l+m+mi}{2}\PYG{p}{)}\PYG{p}{,} \PYG{l+m+mi}{4}\PYG{p}{)}
\PYG{g+go}{[0, 1, 2, 4]}
\end{sphinxVerbatim}

\end{description}

\end{fulllineitems}

\index{add\_to\_set() (in module pypath.share.common)@\spxentry{add\_to\_set()}\spxextra{in module pypath.share.common}}

\begin{fulllineitems}
\phantomsection\label{\detokenize{reference:pypath.share.common.add_to_set}}\pysiglinewithargsret{\sphinxcode{\sphinxupquote{pypath.share.common.}}\sphinxbfcode{\sphinxupquote{add\_to\_set}}}{\emph{st}, \emph{toadd}}{}
Adds elements to a set.

Appends \sphinxstyleemphasis{toadd} to \sphinxstyleemphasis{st}. Function is capable to handle different
input data types. This is, if \sphinxstyleemphasis{toadd} is a list, it will be
converted to a set and added.
\begin{quote}\begin{description}
\item[{Parameters}] \leavevmode\begin{itemize}
\item {} 
\sphinxstyleliteralstrong{\sphinxupquote{st}} (\sphinxstyleliteralemphasis{\sphinxupquote{set}}) \textendash{} Original set to which \sphinxstyleemphasis{toadd} will be added.

\item {} 
\sphinxstyleliteralstrong{\sphinxupquote{toadd}} (\sphinxstyleliteralemphasis{\sphinxupquote{any}}) \textendash{} Element(s) to be added into \sphinxstyleemphasis{st}.

\end{itemize}

\item[{Returns}] \leavevmode
(\sphinxstyleemphasis{set}) \textendash{} Contains the element(s) from the union of \sphinxstyleemphasis{st} and
\sphinxstyleemphasis{toadd}.

\end{description}\end{quote}
\begin{description}
\item[{\sphinxstylestrong{Examples:}}] \leavevmode
\begin{sphinxVerbatim}[commandchars=\\\{\}]
\PYG{g+gp}{\PYGZgt{}\PYGZgt{}\PYGZgt{} }\PYG{n}{st} \PYG{o}{=} \PYG{n+nb}{set}\PYG{p}{(}\PYG{p}{[}\PYG{l+m+mi}{0}\PYG{p}{,} \PYG{l+m+mi}{1}\PYG{p}{,} \PYG{l+m+mi}{2}\PYG{p}{]}\PYG{p}{)}
\PYG{g+gp}{\PYGZgt{}\PYGZgt{}\PYGZgt{} }\PYG{n}{add\PYGZus{}to\PYGZus{}set}\PYG{p}{(}\PYG{n}{st}\PYG{p}{,} \PYG{l+m+mi}{3}\PYG{p}{)}
\PYG{g+go}{set([0, 1, 2, 3])}
\PYG{g+gp}{\PYGZgt{}\PYGZgt{}\PYGZgt{} }\PYG{n}{add\PYGZus{}to\PYGZus{}set}\PYG{p}{(}\PYG{n}{st}\PYG{p}{,} \PYG{p}{[}\PYG{l+m+mi}{4}\PYG{p}{,} \PYG{l+m+mi}{2}\PYG{p}{,} \PYG{l+m+mi}{5}\PYG{p}{]}\PYG{p}{)}
\PYG{g+go}{set([0, 1, 2, 4, 5])}
\end{sphinxVerbatim}

\end{description}

\end{fulllineitems}

\index{gen\_session\_id() (in module pypath.share.common)@\spxentry{gen\_session\_id()}\spxextra{in module pypath.share.common}}

\begin{fulllineitems}
\phantomsection\label{\detokenize{reference:pypath.share.common.gen_session_id}}\pysiglinewithargsret{\sphinxcode{\sphinxupquote{pypath.share.common.}}\sphinxbfcode{\sphinxupquote{gen\_session\_id}}}{\emph{length=5}}{}
Generates a random alphanumeric string.
\begin{quote}\begin{description}
\item[{Parameters}] \leavevmode
\sphinxstyleliteralstrong{\sphinxupquote{length}} (\sphinxstyleliteralemphasis{\sphinxupquote{int}}) \textendash{} Optional, \sphinxcode{\sphinxupquote{5}} by default. Specifies the length of the random
string.

\item[{Returns}] \leavevmode
(\sphinxstyleemphasis{str}) \textendash{} Random alphanumeric string of the specified length.

\end{description}\end{quote}

\end{fulllineitems}

\index{sorensen\_index() (in module pypath.share.common)@\spxentry{sorensen\_index()}\spxextra{in module pypath.share.common}}

\begin{fulllineitems}
\phantomsection\label{\detokenize{reference:pypath.share.common.sorensen_index}}\pysiglinewithargsret{\sphinxcode{\sphinxupquote{pypath.share.common.}}\sphinxbfcode{\sphinxupquote{sorensen\_index}}}{\emph{a}, \emph{b}}{}
Computes the Sorensen index.

Computes the Sorensen-Dice coefficient between two sets \sphinxstyleemphasis{a} and \sphinxstyleemphasis{b}.
\begin{quote}\begin{description}
\item[{Parameters}] \leavevmode\begin{itemize}
\item {} 
\sphinxstyleliteralstrong{\sphinxupquote{a}} (\sphinxstyleliteralemphasis{\sphinxupquote{set}}) \textendash{} Or any iterable type (will be converted to set).

\item {} 
\sphinxstyleliteralstrong{\sphinxupquote{b}} (\sphinxstyleliteralemphasis{\sphinxupquote{set}}) \textendash{} Or any iterable type (will be converted to set).

\end{itemize}

\item[{Returns}] \leavevmode
(\sphinxstyleemphasis{float}) \textendash{} The Sorensen-Dice coefficient between \sphinxstyleemphasis{a} and \sphinxstyleemphasis{b}.

\end{description}\end{quote}

\end{fulllineitems}

\index{simpson\_index() (in module pypath.share.common)@\spxentry{simpson\_index()}\spxextra{in module pypath.share.common}}

\begin{fulllineitems}
\phantomsection\label{\detokenize{reference:pypath.share.common.simpson_index}}\pysiglinewithargsret{\sphinxcode{\sphinxupquote{pypath.share.common.}}\sphinxbfcode{\sphinxupquote{simpson\_index}}}{\emph{a}, \emph{b}}{}
Computes Simpson’s index.

Given two sets \sphinxstyleemphasis{a} and \sphinxstyleemphasis{b}, returns the Simpson index.
\begin{quote}\begin{description}
\item[{Parameters}] \leavevmode\begin{itemize}
\item {} 
\sphinxstyleliteralstrong{\sphinxupquote{a}} (\sphinxstyleliteralemphasis{\sphinxupquote{set}}) \textendash{} Or any iterable type (will be converted to set).

\item {} 
\sphinxstyleliteralstrong{\sphinxupquote{b}} (\sphinxstyleliteralemphasis{\sphinxupquote{set}}) \textendash{} Or any iterable type (will be converted to set).

\end{itemize}

\item[{Returns}] \leavevmode
(\sphinxstyleemphasis{float}) \textendash{} The Simpson index between \sphinxstyleemphasis{a} and \sphinxstyleemphasis{b}.

\end{description}\end{quote}

\end{fulllineitems}

\index{simpson\_index\_counts() (in module pypath.share.common)@\spxentry{simpson\_index\_counts()}\spxextra{in module pypath.share.common}}

\begin{fulllineitems}
\phantomsection\label{\detokenize{reference:pypath.share.common.simpson_index_counts}}\pysiglinewithargsret{\sphinxcode{\sphinxupquote{pypath.share.common.}}\sphinxbfcode{\sphinxupquote{simpson\_index\_counts}}}{\emph{a}, \emph{b}, \emph{c}}{}~\begin{quote}\begin{description}
\item[{Parameters}] \leavevmode\begin{itemize}
\item {} 
\sphinxstyleliteralstrong{\sphinxupquote{a}} \textendash{} 

\item {} 
\sphinxstyleliteralstrong{\sphinxupquote{b}} \textendash{} 

\item {} 
\sphinxstyleliteralstrong{\sphinxupquote{c}} \textendash{} 

\end{itemize}

\item[{Returns}] \leavevmode
(\sphinxstyleemphasis{float}) \textendash{}

\end{description}\end{quote}

\end{fulllineitems}

\index{jaccard\_index() (in module pypath.share.common)@\spxentry{jaccard\_index()}\spxextra{in module pypath.share.common}}

\begin{fulllineitems}
\phantomsection\label{\detokenize{reference:pypath.share.common.jaccard_index}}\pysiglinewithargsret{\sphinxcode{\sphinxupquote{pypath.share.common.}}\sphinxbfcode{\sphinxupquote{jaccard\_index}}}{\emph{a}, \emph{b}}{}
Computes the Jaccard index.

Computes the Jaccard index between two sets \sphinxstyleemphasis{a} and \sphinxstyleemphasis{b}.
\begin{quote}\begin{description}
\item[{Parameters}] \leavevmode\begin{itemize}
\item {} 
\sphinxstyleliteralstrong{\sphinxupquote{a}} (\sphinxstyleliteralemphasis{\sphinxupquote{set}}) \textendash{} Or any iterable type (will be converted to set).

\item {} 
\sphinxstyleliteralstrong{\sphinxupquote{b}} (\sphinxstyleliteralemphasis{\sphinxupquote{set}}) \textendash{} Or any iterable type (will be converted to set).

\end{itemize}

\item[{Returns}] \leavevmode
(\sphinxstyleemphasis{float}) \textendash{} The Jaccard index between \sphinxstyleemphasis{a} and \sphinxstyleemphasis{b}.

\end{description}\end{quote}

\end{fulllineitems}

\index{console() (in module pypath.share.common)@\spxentry{console()}\spxextra{in module pypath.share.common}}

\begin{fulllineitems}
\phantomsection\label{\detokenize{reference:pypath.share.common.console}}\pysiglinewithargsret{\sphinxcode{\sphinxupquote{pypath.share.common.}}\sphinxbfcode{\sphinxupquote{console}}}{\emph{message}}{}
Prints a message on the terminal.

Prints a \sphinxstyleemphasis{message} to the standard output (e.g. terminal) formatted
to 80 characters per line plus first-level indentation.
\begin{quote}\begin{description}
\item[{Parameters}] \leavevmode
\sphinxstyleliteralstrong{\sphinxupquote{message}} (\sphinxstyleliteralemphasis{\sphinxupquote{str}}) \textendash{} The message to be printed.

\end{description}\end{quote}

\end{fulllineitems}

\index{wcl() (in module pypath.share.common)@\spxentry{wcl()}\spxextra{in module pypath.share.common}}

\begin{fulllineitems}
\phantomsection\label{\detokenize{reference:pypath.share.common.wcl}}\pysiglinewithargsret{\sphinxcode{\sphinxupquote{pypath.share.common.}}\sphinxbfcode{\sphinxupquote{wcl}}}{\emph{f}}{}
\end{fulllineitems}

\index{flat\_list() (in module pypath.share.common)@\spxentry{flat\_list()}\spxextra{in module pypath.share.common}}

\begin{fulllineitems}
\phantomsection\label{\detokenize{reference:pypath.share.common.flat_list}}\pysiglinewithargsret{\sphinxcode{\sphinxupquote{pypath.share.common.}}\sphinxbfcode{\sphinxupquote{flat\_list}}}{\emph{lst}}{}
Coerces the elements of a list of iterables into a single list.
\begin{quote}\begin{description}
\item[{Parameters}] \leavevmode
\sphinxstyleliteralstrong{\sphinxupquote{lst}} (\sphinxstyleliteralemphasis{\sphinxupquote{lsit}}) \textendash{} List to be flattened. Its elements can be also lists or any
other iterable.

\item[{Returns}] \leavevmode
(\sphinxstyleemphasis{list}) \textendash{} Flattened list of \sphinxstyleemphasis{lst}.

\end{description}\end{quote}
\begin{description}
\item[{\sphinxstylestrong{Examples:}}] \leavevmode
\begin{sphinxVerbatim}[commandchars=\\\{\}]
\PYG{g+gp}{\PYGZgt{}\PYGZgt{}\PYGZgt{} }\PYG{n}{flat\PYGZus{}list}\PYG{p}{(}\PYG{p}{[}\PYG{p}{(}\PYG{l+m+mi}{0}\PYG{p}{,} \PYG{l+m+mi}{1}\PYG{p}{)}\PYG{p}{,} \PYG{p}{(}\PYG{l+m+mi}{1}\PYG{p}{,} \PYG{l+m+mi}{1}\PYG{p}{)}\PYG{p}{,} \PYG{p}{(}\PYG{l+m+mi}{2}\PYG{p}{,} \PYG{l+m+mi}{1}\PYG{p}{)}\PYG{p}{]}\PYG{p}{)}
\PYG{g+go}{[0, 1, 1, 1, 2, 1]}
\PYG{g+go}{\PYGZgt{}\PYGZgt{} flat\PYGZus{}list([\PYGZsq{}abc\PYGZsq{}, \PYGZsq{}def\PYGZsq{}])}
\PYG{g+go}{[\PYGZsq{}a\PYGZsq{}, \PYGZsq{}b\PYGZsq{}, \PYGZsq{}c\PYGZsq{}, \PYGZsq{}d\PYGZsq{}, \PYGZsq{}e\PYGZsq{}, \PYGZsq{}f\PYGZsq{}]}
\end{sphinxVerbatim}

\end{description}

\end{fulllineitems}

\index{del\_empty() (in module pypath.share.common)@\spxentry{del\_empty()}\spxextra{in module pypath.share.common}}

\begin{fulllineitems}
\phantomsection\label{\detokenize{reference:pypath.share.common.del_empty}}\pysiglinewithargsret{\sphinxcode{\sphinxupquote{pypath.share.common.}}\sphinxbfcode{\sphinxupquote{del\_empty}}}{\emph{lst}}{}
Removes empty entries of a list.

It is assumed that elemenst of \sphinxstyleemphasis{lst} are iterables (e.g. {[}str{]} or
{[}list{]}).
\begin{quote}\begin{description}
\item[{Parameters}] \leavevmode
\sphinxstyleliteralstrong{\sphinxupquote{lst}} (\sphinxstyleliteralemphasis{\sphinxupquote{list}}) \textendash{} List from which empty elements will be removed.

\item[{Returns}] \leavevmode
(\sphinxstyleemphasis{list}) \textendash{} Copy of \sphinxstyleemphasis{lst} without elements whose length was
zero.

\end{description}\end{quote}
\begin{description}
\item[{\sphinxstylestrong{Example:}}] \leavevmode
\begin{sphinxVerbatim}[commandchars=\\\{\}]
\PYG{g+gp}{\PYGZgt{}\PYGZgt{}\PYGZgt{} }\PYG{n}{del\PYGZus{}empty}\PYG{p}{(}\PYG{p}{[}\PYG{l+s+s1}{\PYGZsq{}}\PYG{l+s+s1}{a}\PYG{l+s+s1}{\PYGZsq{}}\PYG{p}{,} \PYG{l+s+s1}{\PYGZsq{}}\PYG{l+s+s1}{\PYGZsq{}}\PYG{p}{,} \PYG{l+s+s1}{\PYGZsq{}}\PYG{l+s+s1}{b}\PYG{l+s+s1}{\PYGZsq{}}\PYG{p}{,} \PYG{l+s+s1}{\PYGZsq{}}\PYG{l+s+s1}{c}\PYG{l+s+s1}{\PYGZsq{}}\PYG{p}{]}\PYG{p}{)}
\PYG{g+go}{[\PYGZsq{}a\PYGZsq{}, \PYGZsq{}b\PYGZsq{}, \PYGZsq{}c\PYGZsq{}]}
\end{sphinxVerbatim}

\end{description}

\end{fulllineitems}

\index{get\_args() (in module pypath.share.common)@\spxentry{get\_args()}\spxextra{in module pypath.share.common}}

\begin{fulllineitems}
\phantomsection\label{\detokenize{reference:pypath.share.common.get_args}}\pysiglinewithargsret{\sphinxcode{\sphinxupquote{pypath.share.common.}}\sphinxbfcode{\sphinxupquote{get\_args}}}{\emph{loc\_dict}, \emph{remove=\{\}}}{}
Given a dictionary of local variables, returns a copy of it without
\sphinxcode{\sphinxupquote{'self'}}, \sphinxcode{\sphinxupquote{'kwargs'}} (in the scope of a \sphinxcode{\sphinxupquote{class}}) plus
any other specified in the keyword argument \sphinxstyleemphasis{remove}.
\begin{quote}\begin{description}
\item[{Parameters}] \leavevmode\begin{itemize}
\item {} 
\sphinxstyleliteralstrong{\sphinxupquote{loc\_dict}} (\sphinxstyleliteralemphasis{\sphinxupquote{dict}}) \textendash{} Dictionary containing the local variables (e.g. a call to
\sphinxcode{\sphinxupquote{locals()}} in a given scope).

\item {} 
\sphinxstyleliteralstrong{\sphinxupquote{remove}} (\sphinxstyleliteralemphasis{\sphinxupquote{set}}) \textendash{} Optional, \sphinxcode{\sphinxupquote{set({[}{]})}} by default. Can also be a list. Contains
the keys of the elements in \sphinxstyleemphasis{loc\_dict} that will be removed.

\end{itemize}

\item[{Returns}] \leavevmode
(\sphinxstyleemphasis{dict}) \textendash{} Copy of \sphinxstyleemphasis{loc\_dict} without \sphinxcode{\sphinxupquote{'self'}}, \sphinxcode{\sphinxupquote{'kwargs'}}
and any other element specified in \sphinxstyleemphasis{remove}.

\end{description}\end{quote}

\end{fulllineitems}

\index{something() (in module pypath.share.common)@\spxentry{something()}\spxextra{in module pypath.share.common}}

\begin{fulllineitems}
\phantomsection\label{\detokenize{reference:pypath.share.common.something}}\pysiglinewithargsret{\sphinxcode{\sphinxupquote{pypath.share.common.}}\sphinxbfcode{\sphinxupquote{something}}}{\emph{anything}}{}
Checks if argument is empty.

Checks if \sphinxstyleemphasis{anything} is empty or \sphinxcode{\sphinxupquote{None}}.
\begin{quote}\begin{description}
\item[{Parameters}] \leavevmode
\sphinxstyleliteralstrong{\sphinxupquote{anything}} (\sphinxstyleliteralemphasis{\sphinxupquote{any}}) \textendash{} Self-explanatory.

\item[{Returns}] \leavevmode
(\sphinxstyleemphasis{bool}) \textendash{} \sphinxcode{\sphinxupquote{False}} if \sphinxstyleemphasis{anyhting} is \sphinxcode{\sphinxupquote{None}} or any empty
data type.

\end{description}\end{quote}
\begin{description}
\item[{\sphinxstylestrong{Examples:}}] \leavevmode
\begin{sphinxVerbatim}[commandchars=\\\{\}]
\PYG{g+gp}{\PYGZgt{}\PYGZgt{}\PYGZgt{} }\PYG{n}{something}\PYG{p}{(}\PYG{k+kc}{None}\PYG{p}{)}
\PYG{g+go}{False}
\PYG{g+gp}{\PYGZgt{}\PYGZgt{}\PYGZgt{} }\PYG{n}{something}\PYG{p}{(}\PYG{l+m+mi}{123}\PYG{p}{)}
\PYG{g+go}{True}
\PYG{g+gp}{\PYGZgt{}\PYGZgt{}\PYGZgt{} }\PYG{n}{something}\PYG{p}{(}\PYG{l+s+s1}{\PYGZsq{}}\PYG{l+s+s1}{Hello world!}\PYG{l+s+s1}{\PYGZsq{}}\PYG{p}{)}
\PYG{g+go}{True}
\PYG{g+gp}{\PYGZgt{}\PYGZgt{}\PYGZgt{} }\PYG{n}{something}\PYG{p}{(}\PYG{l+s+s1}{\PYGZsq{}}\PYG{l+s+s1}{\PYGZsq{}}\PYG{p}{)}
\PYG{g+go}{False}
\PYG{g+gp}{\PYGZgt{}\PYGZgt{}\PYGZgt{} }\PYG{n}{something}\PYG{p}{(}\PYG{p}{[}\PYG{p}{]}\PYG{p}{)}
\PYG{g+go}{False}
\end{sphinxVerbatim}

\end{description}

\end{fulllineitems}

\index{rotate() (in module pypath.share.common)@\spxentry{rotate()}\spxextra{in module pypath.share.common}}

\begin{fulllineitems}
\phantomsection\label{\detokenize{reference:pypath.share.common.rotate}}\pysiglinewithargsret{\sphinxcode{\sphinxupquote{pypath.share.common.}}\sphinxbfcode{\sphinxupquote{rotate}}}{\emph{point}, \emph{angle}, \emph{center=(0.0}, \emph{0.0)}}{}
Rotates a point with respect to a center.

Rotates a given \sphinxstyleemphasis{point} around a \sphinxstyleemphasis{center} according to the specified
\sphinxstyleemphasis{angle} (in degrees) in a two-dimensional space. The rotation is
counter-clockwise.
\begin{quote}\begin{description}
\item[{Parameters}] \leavevmode\begin{itemize}
\item {} 
\sphinxstyleliteralstrong{\sphinxupquote{point}} (\sphinxstyleliteralemphasis{\sphinxupquote{tuple}}) \textendash{} Or list. Contains the two coordinates of the point to be
rotated.

\item {} 
\sphinxstyleliteralstrong{\sphinxupquote{angle}} (\sphinxstyleliteralemphasis{\sphinxupquote{float}}) \textendash{} Angle (in degrees) from which the point will be rotated with
respect to \sphinxstyleemphasis{center} (counter-clockwise).

\item {} 
\sphinxstyleliteralstrong{\sphinxupquote{center}} (\sphinxstyleliteralemphasis{\sphinxupquote{tuple}}) \textendash{} Optional, \sphinxcode{\sphinxupquote{(0.0, 0.0)}} by default. Can also be a list.
Determines the two coordinates of the center relative to which
the point has to be rotated.

\end{itemize}

\item[{Returns}] \leavevmode
(\sphinxstyleemphasis{tuple}) \textendash{} Pair of coordinates of the rotated point.

\end{description}\end{quote}

\end{fulllineitems}

\index{clean\_dict() (in module pypath.share.common)@\spxentry{clean\_dict()}\spxextra{in module pypath.share.common}}

\begin{fulllineitems}
\phantomsection\label{\detokenize{reference:pypath.share.common.clean_dict}}\pysiglinewithargsret{\sphinxcode{\sphinxupquote{pypath.share.common.}}\sphinxbfcode{\sphinxupquote{clean\_dict}}}{\emph{dct}}{}
Cleans a dictionary of \sphinxcode{\sphinxupquote{None}} values.

Removes \sphinxcode{\sphinxupquote{None}} values from  a dictionary \sphinxstyleemphasis{dct} and casts all other
values to strings.
\begin{quote}\begin{description}
\item[{Parameters}] \leavevmode
\sphinxstyleliteralstrong{\sphinxupquote{dct}} (\sphinxstyleliteralemphasis{\sphinxupquote{dict}}) \textendash{} Dictionary to be cleaned from \sphinxcode{\sphinxupquote{None}} values.

\item[{Returns}] \leavevmode
(\sphinxstyleemphasis{dict}) \textendash{} Copy of \sphinxstyleemphasis{dct} without \sphinxcode{\sphinxupquote{None}} value entries and all
other values formatted to strings.

\end{description}\end{quote}

\end{fulllineitems}

\index{md5() (in module pypath.share.common)@\spxentry{md5()}\spxextra{in module pypath.share.common}}

\begin{fulllineitems}
\phantomsection\label{\detokenize{reference:pypath.share.common.md5}}\pysiglinewithargsret{\sphinxcode{\sphinxupquote{pypath.share.common.}}\sphinxbfcode{\sphinxupquote{md5}}}{\emph{value}}{}
Computes the sum of MD5 hash of a given string \sphinxstyleemphasis{value}.
\begin{quote}\begin{description}
\item[{Parameters}] \leavevmode
\sphinxstyleliteralstrong{\sphinxupquote{value}} (\sphinxstyleliteralemphasis{\sphinxupquote{str}}) \textendash{} Or any other type (will be converted to string). Value for which
the MD5 sum will be computed. Must follow ASCII encoding.

\item[{Returns}] \leavevmode
(\sphinxstyleemphasis{str}) \textendash{} Hash value resulting from the MD5 sum of the \sphinxstyleemphasis{value}
string.

\end{description}\end{quote}

\end{fulllineitems}

\index{uniq\_ord\_list() (in module pypath.share.common)@\spxentry{uniq\_ord\_list()}\spxextra{in module pypath.share.common}}

\begin{fulllineitems}
\phantomsection\label{\detokenize{reference:pypath.share.common.uniq_ord_list}}\pysiglinewithargsret{\sphinxcode{\sphinxupquote{pypath.share.common.}}\sphinxbfcode{\sphinxupquote{uniq\_ord\_list}}}{\emph{seq}, \emph{idfun=None}}{}
Reduces a list to its unique elements keeping their order.

Returns a copy of \sphinxstyleemphasis{seq} without repeated elements. Preserves the
order. If any element is repeated, the first instance is kept.
\begin{quote}\begin{description}
\item[{Parameters}] \leavevmode\begin{itemize}
\item {} 
\sphinxstyleliteralstrong{\sphinxupquote{seq}} (\sphinxstyleliteralemphasis{\sphinxupquote{list}}) \textendash{} Or any other iterable type. The sequence from which repeated
elements are to be removed.

\item {} 
\sphinxstyleliteralstrong{\sphinxupquote{idfun}} (\sphinxstyleliteralemphasis{\sphinxupquote{function}}) \textendash{} Optional, \sphinxcode{\sphinxupquote{None}} by default. Identifier function, for each
entry of \sphinxstyleemphasis{seq}, returns a identifier of that entry from which
uniqueness is determined. Default behavior is f(x) = x. See
examples below.

\end{itemize}

\item[{Returns}] \leavevmode
(\sphinxstyleemphasis{list}) \textendash{} Copy of \sphinxstyleemphasis{seq} without the repeated elements
(according to \sphinxstyleemphasis{idfun}).

\end{description}\end{quote}
\begin{description}
\item[{\sphinxstylestrong{Examples:}}] \leavevmode
\begin{sphinxVerbatim}[commandchars=\\\{\}]
\PYG{g+gp}{\PYGZgt{}\PYGZgt{}\PYGZgt{} }\PYG{n}{uniq\PYGZus{}ord\PYGZus{}list}\PYG{p}{(}\PYG{p}{[}\PYG{l+m+mi}{0}\PYG{p}{,} \PYG{l+m+mi}{1}\PYG{p}{,} \PYG{l+m+mi}{2}\PYG{p}{,} \PYG{l+m+mi}{1}\PYG{p}{,} \PYG{l+m+mi}{5}\PYG{p}{]}\PYG{p}{)}
\PYG{g+go}{[0, 1, 2, 5]}
\PYG{g+gp}{\PYGZgt{}\PYGZgt{}\PYGZgt{} }\PYG{n}{uniq\PYGZus{}ord\PYGZus{}list}\PYG{p}{(}\PYG{l+s+s1}{\PYGZsq{}}\PYG{l+s+s1}{abracadabra}\PYG{l+s+s1}{\PYGZsq{}}\PYG{p}{)}
\PYG{g+go}{[\PYGZsq{}a\PYGZsq{}, \PYGZsq{}b\PYGZsq{}, \PYGZsq{}r\PYGZsq{}, \PYGZsq{}c\PYGZsq{}, \PYGZsq{}d\PYGZsq{}]}
\PYG{g+gp}{\PYGZgt{}\PYGZgt{}\PYGZgt{} }\PYG{k}{def} \PYG{n+nf}{f}\PYG{p}{(}\PYG{n}{x}\PYG{p}{)}\PYG{p}{:}
\PYG{g+gp}{... }    \PYG{k}{if} \PYG{n}{x} \PYG{o}{\PYGZgt{}} \PYG{l+m+mi}{0}\PYG{p}{:}
\PYG{g+gp}{... }            \PYG{k}{return} \PYG{l+m+mi}{0}
\PYG{g+gp}{... }    \PYG{k}{else}\PYG{p}{:}
\PYG{g+gp}{... }            \PYG{k}{return} \PYG{l+m+mi}{1}
\PYG{g+gp}{\PYGZgt{}\PYGZgt{}\PYGZgt{} }\PYG{n}{uniq\PYGZus{}ord\PYGZus{}list}\PYG{p}{(}\PYG{p}{[}\PYG{o}{\PYGZhy{}}\PYG{l+m+mi}{32}\PYG{p}{,} \PYG{o}{\PYGZhy{}}\PYG{l+m+mi}{42}\PYG{p}{,} \PYG{l+m+mi}{1}\PYG{p}{,} \PYG{l+m+mi}{15}\PYG{p}{,} \PYG{o}{\PYGZhy{}}\PYG{l+m+mi}{12}\PYG{p}{]}\PYG{p}{,} \PYG{n}{idfun}\PYG{o}{=}\PYG{n}{f}\PYG{p}{)}
\PYG{g+go}{[\PYGZhy{}32, 1]}
\PYG{g+gp}{\PYGZgt{}\PYGZgt{}\PYGZgt{} }\PYG{k}{def} \PYG{n+nf}{g}\PYG{p}{(}\PYG{n}{x}\PYG{p}{)}\PYG{p}{:} \PYG{c+c1}{\PYGZsh{} Given a file name, return it without extension}
\PYG{g+gp}{... }   \PYG{k}{return} \PYG{n}{x}\PYG{o}{.}\PYG{n}{split}\PYG{p}{(}\PYG{l+s+s1}{\PYGZsq{}}\PYG{l+s+s1}{.}\PYG{l+s+s1}{\PYGZsq{}}\PYG{p}{)}\PYG{p}{[}\PYG{l+m+mi}{0}\PYG{p}{]}
\PYG{g+gp}{\PYGZgt{}\PYGZgt{}\PYGZgt{} }\PYG{n}{uniq\PYGZus{}ord\PYGZus{}list}\PYG{p}{(}\PYG{p}{[}\PYG{l+s+s1}{\PYGZsq{}}\PYG{l+s+s1}{a.png}\PYG{l+s+s1}{\PYGZsq{}}\PYG{p}{,} \PYG{l+s+s1}{\PYGZsq{}}\PYG{l+s+s1}{a.txt}\PYG{l+s+s1}{\PYGZsq{}}\PYG{p}{,} \PYG{l+s+s1}{\PYGZsq{}}\PYG{l+s+s1}{b.pdf}\PYG{l+s+s1}{\PYGZsq{}}\PYG{p}{]}\PYG{p}{,} \PYG{n}{idfun}\PYG{o}{=}\PYG{n}{g}\PYG{p}{)}
\PYG{g+go}{[\PYGZsq{}a.png\PYGZsq{}, \PYGZsq{}b.pdf\PYGZsq{}]}
\end{sphinxVerbatim}

\end{description}

\end{fulllineitems}

\index{dict\_diff() (in module pypath.share.common)@\spxentry{dict\_diff()}\spxextra{in module pypath.share.common}}

\begin{fulllineitems}
\phantomsection\label{\detokenize{reference:pypath.share.common.dict_diff}}\pysiglinewithargsret{\sphinxcode{\sphinxupquote{pypath.share.common.}}\sphinxbfcode{\sphinxupquote{dict\_diff}}}{\emph{d1}, \emph{d2}}{}
Compares two dictionaries.

Compares two given dictionaries \sphinxstyleemphasis{d1} and \sphinxstyleemphasis{d2} whose values are sets
or dictionaries (in such case the function is called recursively).
\sphinxstylestrong{NOTE:} The comparison is only performed on the values of the
keys that are common in \sphinxstyleemphasis{d1} and \sphinxstyleemphasis{d2} (see example below).
\begin{quote}\begin{description}
\item[{Parameters}] \leavevmode\begin{itemize}
\item {} 
\sphinxstyleliteralstrong{\sphinxupquote{d1}} (\sphinxstyleliteralemphasis{\sphinxupquote{dict}}) \textendash{} First dictionary of the comparison.

\item {} 
\sphinxstyleliteralstrong{\sphinxupquote{d2}} (\sphinxstyleliteralemphasis{\sphinxupquote{dict}}) \textendash{} Second dictionary of the comparison.

\end{itemize}

\item[{Returns}] \leavevmode
\begin{itemize}
\item {} 
(\sphinxstyleemphasis{dict}) \textendash{} Unique elements of \sphinxstyleemphasis{d1} when compared to \sphinxstyleemphasis{d2}.

\item {} 
(\sphinxstyleemphasis{dict}) \textendash{} Unique elements of \sphinxstyleemphasis{d2} when compared to \sphinxstyleemphasis{d1}.

\end{itemize}


\end{description}\end{quote}
\begin{description}
\item[{\sphinxstylestrong{Examples:}}] \leavevmode
\begin{sphinxVerbatim}[commandchars=\\\{\}]
\PYG{g+gp}{\PYGZgt{}\PYGZgt{}\PYGZgt{} }\PYG{n}{d1} \PYG{o}{=} \PYG{p}{\PYGZob{}}\PYG{l+s+s1}{\PYGZsq{}}\PYG{l+s+s1}{a}\PYG{l+s+s1}{\PYGZsq{}}\PYG{p}{:} \PYG{p}{\PYGZob{}}\PYG{l+m+mi}{1}\PYG{p}{\PYGZcb{}}\PYG{p}{,} \PYG{l+s+s1}{\PYGZsq{}}\PYG{l+s+s1}{b}\PYG{l+s+s1}{\PYGZsq{}}\PYG{p}{:} \PYG{p}{\PYGZob{}}\PYG{l+m+mi}{2}\PYG{p}{\PYGZcb{}}\PYG{p}{,} \PYG{l+s+s1}{\PYGZsq{}}\PYG{l+s+s1}{c}\PYG{l+s+s1}{\PYGZsq{}}\PYG{p}{:} \PYG{p}{\PYGZob{}}\PYG{l+m+mi}{3}\PYG{p}{\PYGZcb{}}\PYG{p}{\PYGZcb{}} \PYG{c+c1}{\PYGZsh{} \PYGZsq{}c\PYGZsq{} is unique to d1}
\PYG{g+gp}{\PYGZgt{}\PYGZgt{}\PYGZgt{} }\PYG{n}{d2} \PYG{o}{=} \PYG{p}{\PYGZob{}}\PYG{l+s+s1}{\PYGZsq{}}\PYG{l+s+s1}{a}\PYG{l+s+s1}{\PYGZsq{}}\PYG{p}{:} \PYG{p}{\PYGZob{}}\PYG{l+m+mi}{1}\PYG{p}{\PYGZcb{}}\PYG{p}{,} \PYG{l+s+s1}{\PYGZsq{}}\PYG{l+s+s1}{b}\PYG{l+s+s1}{\PYGZsq{}}\PYG{p}{:} \PYG{p}{\PYGZob{}}\PYG{l+m+mi}{3}\PYG{p}{\PYGZcb{}}\PYG{p}{\PYGZcb{}}
\PYG{g+gp}{\PYGZgt{}\PYGZgt{}\PYGZgt{} }\PYG{n}{dict\PYGZus{}diff}\PYG{p}{(}\PYG{n}{d1}\PYG{p}{,} \PYG{n}{d2}\PYG{p}{)}
\PYG{g+go}{(\PYGZob{}\PYGZsq{}a\PYGZsq{}: set([]), \PYGZsq{}b\PYGZsq{}: set([2])\PYGZcb{}, \PYGZob{}\PYGZsq{}a\PYGZsq{}: set([2]), \PYGZsq{}b\PYGZsq{}: set([3])\PYGZcb{})}
\end{sphinxVerbatim}

\end{description}

\end{fulllineitems}

\index{to\_set() (in module pypath.share.common)@\spxentry{to\_set()}\spxextra{in module pypath.share.common}}

\begin{fulllineitems}
\phantomsection\label{\detokenize{reference:pypath.share.common.to_set}}\pysiglinewithargsret{\sphinxcode{\sphinxupquote{pypath.share.common.}}\sphinxbfcode{\sphinxupquote{to\_set}}}{\emph{var}}{}
Makes sure the object \sphinxtitleref{var} is a set, if it is a list converts it to set,
otherwise it creates a single element set out of it.
If \sphinxtitleref{var} is None returns empty set.

\end{fulllineitems}

\index{to\_list() (in module pypath.share.common)@\spxentry{to\_list()}\spxextra{in module pypath.share.common}}

\begin{fulllineitems}
\phantomsection\label{\detokenize{reference:pypath.share.common.to_list}}\pysiglinewithargsret{\sphinxcode{\sphinxupquote{pypath.share.common.}}\sphinxbfcode{\sphinxupquote{to\_list}}}{\emph{var}}{}
Makes sure \sphinxtitleref{var} is a list otherwise creates a single element list
out of it. If \sphinxtitleref{var} is None returns empty list.

\end{fulllineitems}

\index{unique\_list() (in module pypath.share.common)@\spxentry{unique\_list()}\spxextra{in module pypath.share.common}}

\begin{fulllineitems}
\phantomsection\label{\detokenize{reference:pypath.share.common.unique_list}}\pysiglinewithargsret{\sphinxcode{\sphinxupquote{pypath.share.common.}}\sphinxbfcode{\sphinxupquote{unique\_list}}}{\emph{seq}}{}
Reduces a list to its unique elements.

Takes any iterable and returns a list of unique elements on it. If
the argument is a dictionary, returns a list of unique keys.
\sphinxstylestrong{NOTE:} Does not preserve the order of the elements.
\begin{quote}\begin{description}
\item[{Parameters}] \leavevmode
\sphinxstyleliteralstrong{\sphinxupquote{seq}} (\sphinxstyleliteralemphasis{\sphinxupquote{list}}) \textendash{} Sequence to be processed, can be any iterable type.

\item[{Returns}] \leavevmode
(\sphinxstyleemphasis{list}) \textendash{} List of unique elements in the sequence \sphinxstyleemphasis{seq}.

\end{description}\end{quote}
\begin{description}
\item[{\sphinxstylestrong{Examples:}}] \leavevmode
\begin{sphinxVerbatim}[commandchars=\\\{\}]
\PYG{g+gp}{\PYGZgt{}\PYGZgt{}\PYGZgt{} }\PYG{n}{uniq\PYGZus{}list}\PYG{p}{(}\PYG{l+s+s1}{\PYGZsq{}}\PYG{l+s+s1}{aba}\PYG{l+s+s1}{\PYGZsq{}}\PYG{p}{)}
\PYG{g+go}{[\PYGZsq{}a\PYGZsq{}, \PYGZsq{}b\PYGZsq{}]}
\PYG{g+gp}{\PYGZgt{}\PYGZgt{}\PYGZgt{} }\PYG{n}{uniq\PYGZus{}list}\PYG{p}{(}\PYG{p}{[}\PYG{l+m+mi}{0}\PYG{p}{,} \PYG{l+m+mi}{1}\PYG{p}{,} \PYG{l+m+mi}{2}\PYG{p}{,} \PYG{l+m+mi}{1}\PYG{p}{,} \PYG{l+m+mi}{0}\PYG{p}{]}\PYG{p}{)}
\PYG{g+go}{[0, 1, 2]}
\end{sphinxVerbatim}

\end{description}

\end{fulllineitems}

\index{basestring (in module pypath.share.common)@\spxentry{basestring}\spxextra{in module pypath.share.common}}

\begin{fulllineitems}
\phantomsection\label{\detokenize{reference:pypath.share.common.basestring}}\pysigline{\sphinxcode{\sphinxupquote{pypath.share.common.}}\sphinxbfcode{\sphinxupquote{basestring}}}
alias of \sphinxcode{\sphinxupquote{builtins.str}}

\end{fulllineitems}

\index{is\_float() (in module pypath.share.common)@\spxentry{is\_float()}\spxextra{in module pypath.share.common}}

\begin{fulllineitems}
\phantomsection\label{\detokenize{reference:pypath.share.common.is_float}}\pysiglinewithargsret{\sphinxcode{\sphinxupquote{pypath.share.common.}}\sphinxbfcode{\sphinxupquote{is\_float}}}{\emph{num}}{}
Tells if a string represents a floating point number,
i.e. it can be converted by \sphinxtitleref{float}.

\end{fulllineitems}

\index{is\_int() (in module pypath.share.common)@\spxentry{is\_int()}\spxextra{in module pypath.share.common}}

\begin{fulllineitems}
\phantomsection\label{\detokenize{reference:pypath.share.common.is_int}}\pysiglinewithargsret{\sphinxcode{\sphinxupquote{pypath.share.common.}}\sphinxbfcode{\sphinxupquote{is\_int}}}{\emph{num}}{}
Tells if a string represents an integer,
i.e. it can be converted by \sphinxtitleref{int}.

\end{fulllineitems}

\index{float\_or\_nan() (in module pypath.share.common)@\spxentry{float\_or\_nan()}\spxextra{in module pypath.share.common}}

\begin{fulllineitems}
\phantomsection\label{\detokenize{reference:pypath.share.common.float_or_nan}}\pysiglinewithargsret{\sphinxcode{\sphinxupquote{pypath.share.common.}}\sphinxbfcode{\sphinxupquote{float\_or\_nan}}}{\emph{num}}{}
Returns \sphinxtitleref{num} converted from string to float if \sphinxtitleref{num} represents a
float otherwise \sphinxtitleref{numpy.nan}.

\end{fulllineitems}



\section{complex}
\label{\detokenize{reference:module-pypath.core.complex}}\label{\detokenize{reference:complex}}\index{pypath.core.complex (module)@\spxentry{pypath.core.complex}\spxextra{module}}\index{AbstractComplexResource (class in pypath.core.complex)@\spxentry{AbstractComplexResource}\spxextra{class in pypath.core.complex}}

\begin{fulllineitems}
\phantomsection\label{\detokenize{reference:pypath.core.complex.AbstractComplexResource}}\pysiglinewithargsret{\sphinxbfcode{\sphinxupquote{class }}\sphinxcode{\sphinxupquote{pypath.core.complex.}}\sphinxbfcode{\sphinxupquote{AbstractComplexResource}}}{\emph{name}, \emph{ncbi\_tax\_id=9606}, \emph{input\_method=None}, \emph{input\_args=None}, \emph{dump=None}, \emph{**kwargs}}{}
A resource which provides information about molecular complexes.

\end{fulllineitems}

\index{CellPhoneDB (class in pypath.core.complex)@\spxentry{CellPhoneDB}\spxextra{class in pypath.core.complex}}

\begin{fulllineitems}
\phantomsection\label{\detokenize{reference:pypath.core.complex.CellPhoneDB}}\pysiglinewithargsret{\sphinxbfcode{\sphinxupquote{class }}\sphinxcode{\sphinxupquote{pypath.core.complex.}}\sphinxbfcode{\sphinxupquote{CellPhoneDB}}}{\emph{**kwargs}}{}
\end{fulllineitems}

\index{Compleat (class in pypath.core.complex)@\spxentry{Compleat}\spxextra{class in pypath.core.complex}}

\begin{fulllineitems}
\phantomsection\label{\detokenize{reference:pypath.core.complex.Compleat}}\pysiglinewithargsret{\sphinxbfcode{\sphinxupquote{class }}\sphinxcode{\sphinxupquote{pypath.core.complex.}}\sphinxbfcode{\sphinxupquote{Compleat}}}{\emph{input\_args=None}, \emph{**kwargs}}{}
\end{fulllineitems}

\index{ComplexAggregator (class in pypath.core.complex)@\spxentry{ComplexAggregator}\spxextra{class in pypath.core.complex}}

\begin{fulllineitems}
\phantomsection\label{\detokenize{reference:pypath.core.complex.ComplexAggregator}}\pysiglinewithargsret{\sphinxbfcode{\sphinxupquote{class }}\sphinxcode{\sphinxupquote{pypath.core.complex.}}\sphinxbfcode{\sphinxupquote{ComplexAggregator}}}{\emph{resources=None}, \emph{pickle\_file=None}}{}~\index{reload() (pypath.core.complex.ComplexAggregator method)@\spxentry{reload()}\spxextra{pypath.core.complex.ComplexAggregator method}}

\begin{fulllineitems}
\phantomsection\label{\detokenize{reference:pypath.core.complex.ComplexAggregator.reload}}\pysiglinewithargsret{\sphinxbfcode{\sphinxupquote{reload}}}{}{}
Reloads the object from the module level.

\end{fulllineitems}


\end{fulllineitems}

\index{ComplexPortal (class in pypath.core.complex)@\spxentry{ComplexPortal}\spxextra{class in pypath.core.complex}}

\begin{fulllineitems}
\phantomsection\label{\detokenize{reference:pypath.core.complex.ComplexPortal}}\pysiglinewithargsret{\sphinxbfcode{\sphinxupquote{class }}\sphinxcode{\sphinxupquote{pypath.core.complex.}}\sphinxbfcode{\sphinxupquote{ComplexPortal}}}{\emph{input\_args=None}, \emph{**kwargs}}{}
\end{fulllineitems}

\index{Corum (class in pypath.core.complex)@\spxentry{Corum}\spxextra{class in pypath.core.complex}}

\begin{fulllineitems}
\phantomsection\label{\detokenize{reference:pypath.core.complex.Corum}}\pysiglinewithargsret{\sphinxbfcode{\sphinxupquote{class }}\sphinxcode{\sphinxupquote{pypath.core.complex.}}\sphinxbfcode{\sphinxupquote{Corum}}}{\emph{input\_args=None}, \emph{**kwargs}}{}
\end{fulllineitems}

\index{GuideToPharmacology (class in pypath.core.complex)@\spxentry{GuideToPharmacology}\spxextra{class in pypath.core.complex}}

\begin{fulllineitems}
\phantomsection\label{\detokenize{reference:pypath.core.complex.GuideToPharmacology}}\pysiglinewithargsret{\sphinxbfcode{\sphinxupquote{class }}\sphinxcode{\sphinxupquote{pypath.core.complex.}}\sphinxbfcode{\sphinxupquote{GuideToPharmacology}}}{\emph{input\_args=None}, \emph{**kwargs}}{}
\end{fulllineitems}

\index{Havugimana (class in pypath.core.complex)@\spxentry{Havugimana}\spxextra{class in pypath.core.complex}}

\begin{fulllineitems}
\phantomsection\label{\detokenize{reference:pypath.core.complex.Havugimana}}\pysiglinewithargsret{\sphinxbfcode{\sphinxupquote{class }}\sphinxcode{\sphinxupquote{pypath.core.complex.}}\sphinxbfcode{\sphinxupquote{Havugimana}}}{\emph{input\_args=None}, \emph{**kwargs}}{}
\end{fulllineitems}

\index{Hpmr (class in pypath.core.complex)@\spxentry{Hpmr}\spxextra{class in pypath.core.complex}}

\begin{fulllineitems}
\phantomsection\label{\detokenize{reference:pypath.core.complex.Hpmr}}\pysiglinewithargsret{\sphinxbfcode{\sphinxupquote{class }}\sphinxcode{\sphinxupquote{pypath.core.complex.}}\sphinxbfcode{\sphinxupquote{Hpmr}}}{\emph{input\_args=None}, \emph{**kwargs}}{}
\end{fulllineitems}

\index{Humap (class in pypath.core.complex)@\spxentry{Humap}\spxextra{class in pypath.core.complex}}

\begin{fulllineitems}
\phantomsection\label{\detokenize{reference:pypath.core.complex.Humap}}\pysiglinewithargsret{\sphinxbfcode{\sphinxupquote{class }}\sphinxcode{\sphinxupquote{pypath.core.complex.}}\sphinxbfcode{\sphinxupquote{Humap}}}{\emph{input\_args=None}, \emph{**kwargs}}{}
\end{fulllineitems}

\index{Pdb (class in pypath.core.complex)@\spxentry{Pdb}\spxextra{class in pypath.core.complex}}

\begin{fulllineitems}
\phantomsection\label{\detokenize{reference:pypath.core.complex.Pdb}}\pysiglinewithargsret{\sphinxbfcode{\sphinxupquote{class }}\sphinxcode{\sphinxupquote{pypath.core.complex.}}\sphinxbfcode{\sphinxupquote{Pdb}}}{\emph{input\_args=None}, \emph{**kwargs}}{}
\end{fulllineitems}

\index{Signor (class in pypath.core.complex)@\spxentry{Signor}\spxextra{class in pypath.core.complex}}

\begin{fulllineitems}
\phantomsection\label{\detokenize{reference:pypath.core.complex.Signor}}\pysiglinewithargsret{\sphinxbfcode{\sphinxupquote{class }}\sphinxcode{\sphinxupquote{pypath.core.complex.}}\sphinxbfcode{\sphinxupquote{Signor}}}{\emph{input\_args=None}, \emph{**kwargs}}{}
\end{fulllineitems}

\index{all\_complexes() (in module pypath.core.complex)@\spxentry{all\_complexes()}\spxextra{in module pypath.core.complex}}

\begin{fulllineitems}
\phantomsection\label{\detokenize{reference:pypath.core.complex.all_complexes}}\pysiglinewithargsret{\sphinxcode{\sphinxupquote{pypath.core.complex.}}\sphinxbfcode{\sphinxupquote{all\_complexes}}}{}{}
Returns a set of all complexes in the database which serves as a
reference set for many methods, just like \sphinxcode{\sphinxupquote{inputs.uniprot.all\_uniprots}}
represents the proteome.

\end{fulllineitems}

\index{get\_db() (in module pypath.core.complex)@\spxentry{get\_db()}\spxextra{in module pypath.core.complex}}

\begin{fulllineitems}
\phantomsection\label{\detokenize{reference:pypath.core.complex.get_db}}\pysiglinewithargsret{\sphinxcode{\sphinxupquote{pypath.core.complex.}}\sphinxbfcode{\sphinxupquote{get\_db}}}{\emph{**kwargs}}{}
Retrieves the current database instance and initializes it if does
not exist yet.

\end{fulllineitems}

\index{init\_db() (in module pypath.core.complex)@\spxentry{init\_db()}\spxextra{in module pypath.core.complex}}

\begin{fulllineitems}
\phantomsection\label{\detokenize{reference:pypath.core.complex.init_db}}\pysiglinewithargsret{\sphinxcode{\sphinxupquote{pypath.core.complex.}}\sphinxbfcode{\sphinxupquote{init\_db}}}{\emph{**kwargs}}{}
Initializes or reloads the complex database.
The database will be assigned to the \sphinxcode{\sphinxupquote{db}} attribute of this module.

\end{fulllineitems}



\section{curl}
\label{\detokenize{reference:module-pypath.share.curl}}\label{\detokenize{reference:curl}}\index{pypath.share.curl (module)@\spxentry{pypath.share.curl}\spxextra{module}}\index{Curl (class in pypath.share.curl)@\spxentry{Curl}\spxextra{class in pypath.share.curl}}

\begin{fulllineitems}
\phantomsection\label{\detokenize{reference:pypath.share.curl.Curl}}\pysiglinewithargsret{\sphinxbfcode{\sphinxupquote{class }}\sphinxcode{\sphinxupquote{pypath.share.curl.}}\sphinxbfcode{\sphinxupquote{Curl}}}{\emph{url}, \emph{silent=True}, \emph{get=None}, \emph{post=None}, \emph{req\_headers=None}, \emph{cache=True}, \emph{debug=False}, \emph{outf=None}, \emph{compr=None}, \emph{encoding=None}, \emph{files\_needed=None}, \emph{connect\_timeout=300}, \emph{timeout=2400}, \emph{ignore\_content\_length=False}, \emph{init\_url=None}, \emph{init\_fun='get\_jsessionid'}, \emph{init\_use\_cache=False}, \emph{follow=True}, \emph{large=False}, \emph{default\_mode='r'}, \emph{override\_post=False}, \emph{init\_headers=False}, \emph{return\_headers=False}, \emph{compressed=False}, \emph{binary\_data=None}, \emph{write\_cache=True}, \emph{force\_quote=False}, \emph{sftp\_user=None}, \emph{sftp\_passwd=None}, \emph{sftp\_passwd\_file='.secrets'}, \emph{sftp\_port=22}, \emph{sftp\_host=None}, \emph{sftp\_ask=None}, \emph{setup=True}, \emph{call=True}, \emph{process=True}, \emph{retries=3}, \emph{cache\_dir=None}, \emph{bypass\_url\_encoding=False}, \emph{empty\_attempt\_again=True}}{}
This class is a wrapper around pycurl.
You can set a vast amount of parameters.
In addition it has a caching functionality: using this downloads
of databases/resources is performed only once.
It handles HTTP, FTP, cookies, headers, GET and POST params,
multipart/form data, URL quoting, redirects, timeouts, retries,
encodings, debugging.
It returns either downloaded data, file pointer, files extracted
from archives (gzip, tar.gz, zip).
It is able to show a progress and status indicator on the console.
\index{close() (pypath.share.curl.Curl method)@\spxentry{close()}\spxextra{pypath.share.curl.Curl method}}

\begin{fulllineitems}
\phantomsection\label{\detokenize{reference:pypath.share.curl.Curl.close}}\pysiglinewithargsret{\sphinxbfcode{\sphinxupquote{close}}}{}{}
Closes all file objects.

\end{fulllineitems}

\index{construct\_binary\_data() (pypath.share.curl.Curl method)@\spxentry{construct\_binary\_data()}\spxextra{pypath.share.curl.Curl method}}

\begin{fulllineitems}
\phantomsection\label{\detokenize{reference:pypath.share.curl.Curl.construct_binary_data}}\pysiglinewithargsret{\sphinxbfcode{\sphinxupquote{construct\_binary\_data}}}{}{}
The binary data content of a \sphinxtitleref{form/multipart} type request
can be constructed from a list of tuples (\textless{}field name\textgreater{}, \textless{}field value\textgreater{}),
where field name and value are both type of bytes.

\end{fulllineitems}

\index{is\_quoted() (pypath.share.curl.Curl method)@\spxentry{is\_quoted()}\spxextra{pypath.share.curl.Curl method}}

\begin{fulllineitems}
\phantomsection\label{\detokenize{reference:pypath.share.curl.Curl.is_quoted}}\pysiglinewithargsret{\sphinxbfcode{\sphinxupquote{is\_quoted}}}{\emph{string}}{}
From \sphinxurl{http://stackoverflow.com/questions/}
1637762/test-if-string-is-url-encoded-in-php

\end{fulllineitems}

\index{set\_binary\_data() (pypath.share.curl.Curl method)@\spxentry{set\_binary\_data()}\spxextra{pypath.share.curl.Curl method}}

\begin{fulllineitems}
\phantomsection\label{\detokenize{reference:pypath.share.curl.Curl.set_binary_data}}\pysiglinewithargsret{\sphinxbfcode{\sphinxupquote{set\_binary\_data}}}{}{}
Set binary data to be transmitted attached to POST request.

\sphinxtitleref{binary\_data} is either a bytes string, or a filename, or
a list of key-value pairs of a multipart form.

\end{fulllineitems}

\index{url\_fix() (pypath.share.curl.Curl method)@\spxentry{url\_fix()}\spxextra{pypath.share.curl.Curl method}}

\begin{fulllineitems}
\phantomsection\label{\detokenize{reference:pypath.share.curl.Curl.url_fix}}\pysiglinewithargsret{\sphinxbfcode{\sphinxupquote{url\_fix}}}{\emph{charset='utf-8'}}{}
From \sphinxurl{http://stackoverflow.com/a/121017/854988}

\end{fulllineitems}


\end{fulllineitems}

\index{FileOpener (class in pypath.share.curl)@\spxentry{FileOpener}\spxextra{class in pypath.share.curl}}

\begin{fulllineitems}
\phantomsection\label{\detokenize{reference:pypath.share.curl.FileOpener}}\pysiglinewithargsret{\sphinxbfcode{\sphinxupquote{class }}\sphinxcode{\sphinxupquote{pypath.share.curl.}}\sphinxbfcode{\sphinxupquote{FileOpener}}}{\emph{file\_param}, \emph{compr=None}, \emph{extract=True}, \emph{\_open=True}, \emph{set\_fileobj=True}, \emph{files\_needed=None}, \emph{large=True}, \emph{default\_mode='r'}, \emph{encoding='utf-8'}}{}
This class opens a file, extracts it in case it is a
gzip, tar.gz, tar.bz2 or zip archive, selects the requested
files if you only need certain files from a multifile archive,
reads the data from the file, or returns the file pointer,
as you request. It examines the file type and size.
\index{extract() (pypath.share.curl.FileOpener method)@\spxentry{extract()}\spxextra{pypath.share.curl.FileOpener method}}

\begin{fulllineitems}
\phantomsection\label{\detokenize{reference:pypath.share.curl.FileOpener.extract}}\pysiglinewithargsret{\sphinxbfcode{\sphinxupquote{extract}}}{}{}
Calls the extracting method for compressed files.

\end{fulllineitems}

\index{open() (pypath.share.curl.FileOpener method)@\spxentry{open()}\spxextra{pypath.share.curl.FileOpener method}}

\begin{fulllineitems}
\phantomsection\label{\detokenize{reference:pypath.share.curl.FileOpener.open}}\pysiglinewithargsret{\sphinxbfcode{\sphinxupquote{open}}}{}{}
Opens the file if exists.

\end{fulllineitems}

\index{open\_tgz() (pypath.share.curl.FileOpener method)@\spxentry{open\_tgz()}\spxextra{pypath.share.curl.FileOpener method}}

\begin{fulllineitems}
\phantomsection\label{\detokenize{reference:pypath.share.curl.FileOpener.open_tgz}}\pysiglinewithargsret{\sphinxbfcode{\sphinxupquote{open\_tgz}}}{}{}
Extracts files from tar gz.

\end{fulllineitems}


\end{fulllineitems}

\index{cache\_delete\_off (class in pypath.share.curl)@\spxentry{cache\_delete\_off}\spxextra{class in pypath.share.curl}}

\begin{fulllineitems}
\phantomsection\label{\detokenize{reference:pypath.share.curl.cache_delete_off}}\pysigline{\sphinxbfcode{\sphinxupquote{class }}\sphinxcode{\sphinxupquote{pypath.share.curl.}}\sphinxbfcode{\sphinxupquote{cache\_delete\_off}}}
This is a context handler which stops pypath.curl.Curl() deleting the
cache files. This is the default behaviour, so this context won’t
change anything by default.

Behind the scenes it sets the value of the \sphinxtitleref{pypath.curl.CACHEDEL}
module level variable to \sphinxtitleref{False}.

Example:

\begin{sphinxVerbatim}[commandchars=\\\{\}]
\PYG{k+kn}{import} \PYG{n+nn}{pypath}
\PYG{k+kn}{from} \PYG{n+nn}{pypath} \PYG{k}{import} \PYG{n}{curl}\PYG{p}{,} \PYG{n}{data\PYGZus{}formats}

\PYG{n}{pa} \PYG{o}{=} \PYG{n}{pypath}\PYG{o}{.}\PYG{n}{PyPath}\PYG{p}{(}\PYG{p}{)}

\PYG{k}{with} \PYG{n}{curl}\PYG{o}{.}\PYG{n}{cache\PYGZus{}delete\PYGZus{}off}\PYG{p}{(}\PYG{p}{)}\PYG{p}{:}
    \PYG{n}{pa}\PYG{o}{.}\PYG{n}{load\PYGZus{}resources}\PYG{p}{(}\PYG{p}{\PYGZob{}}\PYG{l+s+s1}{\PYGZsq{}}\PYG{l+s+s1}{signor}\PYG{l+s+s1}{\PYGZsq{}}\PYG{p}{:} \PYG{n}{data\PYGZus{}formats}\PYG{o}{.}\PYG{n}{pathway}\PYG{p}{[}\PYG{l+s+s1}{\PYGZsq{}}\PYG{l+s+s1}{signor}\PYG{l+s+s1}{\PYGZsq{}}\PYG{p}{]}\PYG{p}{\PYGZcb{}}\PYG{p}{)}
\end{sphinxVerbatim}

\end{fulllineitems}

\index{cache\_delete\_on (class in pypath.share.curl)@\spxentry{cache\_delete\_on}\spxextra{class in pypath.share.curl}}

\begin{fulllineitems}
\phantomsection\label{\detokenize{reference:pypath.share.curl.cache_delete_on}}\pysigline{\sphinxbfcode{\sphinxupquote{class }}\sphinxcode{\sphinxupquote{pypath.share.curl.}}\sphinxbfcode{\sphinxupquote{cache\_delete\_on}}}
This is a context handler which results pypath.curl.Curl() deleting the
cache files instead of reading it. Then it downloads the data again,
or does nothing if the \sphinxtitleref{DRYRUN} context is turned on. Upon deleting
cache files console messages will let you know which files have been
deleted.

Behind the scenes it sets the value of the \sphinxtitleref{pypath.curl.CACHEDEL}
module level variable to \sphinxtitleref{True} (by default it is \sphinxtitleref{False}).

Example:

\begin{sphinxVerbatim}[commandchars=\\\{\}]
\PYG{k+kn}{import} \PYG{n+nn}{pypath}
\PYG{k+kn}{from} \PYG{n+nn}{pypath} \PYG{k}{import} \PYG{n}{curl}\PYG{p}{,} \PYG{n}{data\PYGZus{}formats}

\PYG{n}{pa} \PYG{o}{=} \PYG{n}{pypath}\PYG{o}{.}\PYG{n}{PyPath}\PYG{p}{(}\PYG{p}{)}

\PYG{k}{with} \PYG{n}{curl}\PYG{o}{.}\PYG{n}{cache\PYGZus{}delete\PYGZus{}on}\PYG{p}{(}\PYG{p}{)}\PYG{p}{:}
    \PYG{n}{pa}\PYG{o}{.}\PYG{n}{load\PYGZus{}resources}\PYG{p}{(}\PYG{p}{\PYGZob{}}\PYG{l+s+s1}{\PYGZsq{}}\PYG{l+s+s1}{signor}\PYG{l+s+s1}{\PYGZsq{}}\PYG{p}{:} \PYG{n}{data\PYGZus{}formats}\PYG{o}{.}\PYG{n}{pathway}\PYG{p}{[}\PYG{l+s+s1}{\PYGZsq{}}\PYG{l+s+s1}{signor}\PYG{l+s+s1}{\PYGZsq{}}\PYG{p}{]}\PYG{p}{\PYGZcb{}}\PYG{p}{)}
\end{sphinxVerbatim}

\end{fulllineitems}

\index{cache\_off (class in pypath.share.curl)@\spxentry{cache\_off}\spxextra{class in pypath.share.curl}}

\begin{fulllineitems}
\phantomsection\label{\detokenize{reference:pypath.share.curl.cache_off}}\pysigline{\sphinxbfcode{\sphinxupquote{class }}\sphinxcode{\sphinxupquote{pypath.share.curl.}}\sphinxbfcode{\sphinxupquote{cache\_off}}}
This is a context handler to turn off pypath.curl.Curl() cache.
Data will be downloaded even if it exists in cache.

Behind the scenes it sets the value of the \sphinxtitleref{pypath.curl.CACHE}
module level variable to \sphinxtitleref{False} (by default it is \sphinxtitleref{None}).

Example:

\begin{sphinxVerbatim}[commandchars=\\\{\}]
\PYG{k+kn}{import} \PYG{n+nn}{pypath}
\PYG{k+kn}{from} \PYG{n+nn}{pypath} \PYG{k}{import} \PYG{n}{curl}\PYG{p}{,} \PYG{n}{data\PYGZus{}formats}

\PYG{n}{pa} \PYG{o}{=} \PYG{n}{pypath}\PYG{o}{.}\PYG{n}{PyPath}\PYG{p}{(}\PYG{p}{)}

\PYG{n+nb}{print}\PYG{p}{(}\PYG{l+s+s1}{\PYGZsq{}}\PYG{l+s+s1}{{}`curl.CACHE{}` is }\PYG{l+s+s1}{\PYGZsq{}}\PYG{p}{,} \PYG{n}{curl}\PYG{o}{.}\PYG{n}{CACHE}\PYG{p}{)}

\PYG{k}{with} \PYG{n}{curl}\PYG{o}{.}\PYG{n}{cache\PYGZus{}on}\PYG{p}{(}\PYG{p}{)}\PYG{p}{:}
    \PYG{n+nb}{print}\PYG{p}{(}\PYG{l+s+s1}{\PYGZsq{}}\PYG{l+s+s1}{{}`curl.CACHE{}` is }\PYG{l+s+s1}{\PYGZsq{}}\PYG{p}{,} \PYG{n}{curl}\PYG{o}{.}\PYG{n}{CACHE}\PYG{p}{)}
    \PYG{n}{pa}\PYG{o}{.}\PYG{n}{load\PYGZus{}resources}\PYG{p}{(}\PYG{p}{\PYGZob{}}\PYG{l+s+s1}{\PYGZsq{}}\PYG{l+s+s1}{signor}\PYG{l+s+s1}{\PYGZsq{}}\PYG{p}{:} \PYG{n}{data\PYGZus{}formats}\PYG{o}{.}\PYG{n}{pathway}\PYG{p}{[}\PYG{l+s+s1}{\PYGZsq{}}\PYG{l+s+s1}{signor}\PYG{l+s+s1}{\PYGZsq{}}\PYG{p}{]}\PYG{p}{\PYGZcb{}}\PYG{p}{)}
\end{sphinxVerbatim}

\end{fulllineitems}

\index{cache\_on (class in pypath.share.curl)@\spxentry{cache\_on}\spxextra{class in pypath.share.curl}}

\begin{fulllineitems}
\phantomsection\label{\detokenize{reference:pypath.share.curl.cache_on}}\pysigline{\sphinxbfcode{\sphinxupquote{class }}\sphinxcode{\sphinxupquote{pypath.share.curl.}}\sphinxbfcode{\sphinxupquote{cache\_on}}}
This is a context handler to turn on pypath.curl.Curl() cache.
As most of the methods use cache as their default behaviour,
probably it won’t change anything.

Behind the scenes it sets the value of the \sphinxtitleref{pypath.curl.CACHE}
module level variable to \sphinxtitleref{True} (by default it is \sphinxtitleref{None}).

Example:

\begin{sphinxVerbatim}[commandchars=\\\{\}]
\PYG{k+kn}{import} \PYG{n+nn}{pypath}
\PYG{k+kn}{from} \PYG{n+nn}{pypath} \PYG{k}{import} \PYG{n}{curl}\PYG{p}{,} \PYG{n}{data\PYGZus{}formats}

\PYG{n}{pa} \PYG{o}{=} \PYG{n}{pypath}\PYG{o}{.}\PYG{n}{PyPath}\PYG{p}{(}\PYG{p}{)}

\PYG{n+nb}{print}\PYG{p}{(}\PYG{l+s+s1}{\PYGZsq{}}\PYG{l+s+s1}{{}`curl.CACHE{}` is }\PYG{l+s+s1}{\PYGZsq{}}\PYG{p}{,} \PYG{n}{curl}\PYG{o}{.}\PYG{n}{CACHE}\PYG{p}{)}

\PYG{k}{with} \PYG{n}{curl}\PYG{o}{.}\PYG{n}{cache\PYGZus{}on}\PYG{p}{(}\PYG{p}{)}\PYG{p}{:}
    \PYG{n+nb}{print}\PYG{p}{(}\PYG{l+s+s1}{\PYGZsq{}}\PYG{l+s+s1}{{}`curl.CACHE{}` is }\PYG{l+s+s1}{\PYGZsq{}}\PYG{p}{,} \PYG{n}{curl}\PYG{o}{.}\PYG{n}{CACHE}\PYG{p}{)}
    \PYG{n}{pa}\PYG{o}{.}\PYG{n}{load\PYGZus{}resources}\PYG{p}{(}\PYG{p}{\PYGZob{}}\PYG{l+s+s1}{\PYGZsq{}}\PYG{l+s+s1}{signor}\PYG{l+s+s1}{\PYGZsq{}}\PYG{p}{:} \PYG{n}{data\PYGZus{}formats}\PYG{o}{.}\PYG{n}{pathway}\PYG{p}{[}\PYG{l+s+s1}{\PYGZsq{}}\PYG{l+s+s1}{signor}\PYG{l+s+s1}{\PYGZsq{}}\PYG{p}{]}\PYG{p}{\PYGZcb{}}\PYG{p}{)}
\end{sphinxVerbatim}

\end{fulllineitems}

\index{cache\_print\_off (class in pypath.share.curl)@\spxentry{cache\_print\_off}\spxextra{class in pypath.share.curl}}

\begin{fulllineitems}
\phantomsection\label{\detokenize{reference:pypath.share.curl.cache_print_off}}\pysigline{\sphinxbfcode{\sphinxupquote{class }}\sphinxcode{\sphinxupquote{pypath.share.curl.}}\sphinxbfcode{\sphinxupquote{cache\_print\_off}}}
This is a context handler which stops pypath.curl.Curl() to print
verbose messages about its cache.

Behind the scenes it sets the value of the \sphinxtitleref{pypath.curl.CACHEPRINT}
module level variable to \sphinxtitleref{False}. As by default it is \sphinxtitleref{False}, this
context won’t modify the default behaviour.

Example:

\begin{sphinxVerbatim}[commandchars=\\\{\}]
\PYG{k+kn}{import} \PYG{n+nn}{pypath}
\PYG{k+kn}{from} \PYG{n+nn}{pypath} \PYG{k}{import} \PYG{n}{curl}\PYG{p}{,} \PYG{n}{data\PYGZus{}formats}

\PYG{n}{pa} \PYG{o}{=} \PYG{n}{pypath}\PYG{o}{.}\PYG{n}{PyPath}\PYG{p}{(}\PYG{p}{)}

\PYG{k}{with} \PYG{n}{curl}\PYG{o}{.}\PYG{n}{cache\PYGZus{}print\PYGZus{}off}\PYG{p}{(}\PYG{p}{)}\PYG{p}{:}
    \PYG{n}{pa}\PYG{o}{.}\PYG{n}{load\PYGZus{}resources}\PYG{p}{(}\PYG{p}{\PYGZob{}}\PYG{l+s+s1}{\PYGZsq{}}\PYG{l+s+s1}{signor}\PYG{l+s+s1}{\PYGZsq{}}\PYG{p}{:} \PYG{n}{data\PYGZus{}formats}\PYG{o}{.}\PYG{n}{pathway}\PYG{p}{[}\PYG{l+s+s1}{\PYGZsq{}}\PYG{l+s+s1}{signor}\PYG{l+s+s1}{\PYGZsq{}}\PYG{p}{]}\PYG{p}{\PYGZcb{}}\PYG{p}{)}
\end{sphinxVerbatim}

\end{fulllineitems}

\index{cache\_print\_on (class in pypath.share.curl)@\spxentry{cache\_print\_on}\spxextra{class in pypath.share.curl}}

\begin{fulllineitems}
\phantomsection\label{\detokenize{reference:pypath.share.curl.cache_print_on}}\pysigline{\sphinxbfcode{\sphinxupquote{class }}\sphinxcode{\sphinxupquote{pypath.share.curl.}}\sphinxbfcode{\sphinxupquote{cache\_print\_on}}}
This is a context handler which makes pypath.curl.Curl() print
verbose messages about its cache.

Behind the scenes it sets the value of the \sphinxtitleref{pypath.curl.CACHEPRINT}
module level variable to \sphinxtitleref{True} (by default it is \sphinxtitleref{False}).

Example:

\begin{sphinxVerbatim}[commandchars=\\\{\}]
\PYG{k+kn}{import} \PYG{n+nn}{pypath}
\PYG{k+kn}{from} \PYG{n+nn}{pypath} \PYG{k}{import} \PYG{n}{curl}\PYG{p}{,} \PYG{n}{data\PYGZus{}formats}

\PYG{n}{pa} \PYG{o}{=} \PYG{n}{pypath}\PYG{o}{.}\PYG{n}{PyPath}\PYG{p}{(}\PYG{p}{)}

\PYG{k}{with} \PYG{n}{curl}\PYG{o}{.}\PYG{n}{cache\PYGZus{}print\PYGZus{}on}\PYG{p}{(}\PYG{p}{)}\PYG{p}{:}
    \PYG{n}{pa}\PYG{o}{.}\PYG{n}{load\PYGZus{}resources}\PYG{p}{(}\PYG{p}{\PYGZob{}}\PYG{l+s+s1}{\PYGZsq{}}\PYG{l+s+s1}{signor}\PYG{l+s+s1}{\PYGZsq{}}\PYG{p}{:} \PYG{n}{data\PYGZus{}formats}\PYG{o}{.}\PYG{n}{pathway}\PYG{p}{[}\PYG{l+s+s1}{\PYGZsq{}}\PYG{l+s+s1}{signor}\PYG{l+s+s1}{\PYGZsq{}}\PYG{p}{]}\PYG{p}{\PYGZcb{}}\PYG{p}{)}
\end{sphinxVerbatim}

\end{fulllineitems}

\index{debug\_off (class in pypath.share.curl)@\spxentry{debug\_off}\spxextra{class in pypath.share.curl}}

\begin{fulllineitems}
\phantomsection\label{\detokenize{reference:pypath.share.curl.debug_off}}\pysigline{\sphinxbfcode{\sphinxupquote{class }}\sphinxcode{\sphinxupquote{pypath.share.curl.}}\sphinxbfcode{\sphinxupquote{debug\_off}}}
This is a context handler which avoids pypath.curl.Curl() to print
debug information.
By default it does not do this, so this context only restores the
default.

Behind the scenes it sets the value of the \sphinxtitleref{pypath.curl.DEBUG}
module level variable to \sphinxtitleref{False}.

Example:

\begin{sphinxVerbatim}[commandchars=\\\{\}]
\PYG{k+kn}{import} \PYG{n+nn}{pypath}
\PYG{k+kn}{from} \PYG{n+nn}{pypath} \PYG{k}{import} \PYG{n}{curl}\PYG{p}{,} \PYG{n}{data\PYGZus{}formats}

\PYG{n}{pa} \PYG{o}{=} \PYG{n}{pypath}\PYG{o}{.}\PYG{n}{PyPath}\PYG{p}{(}\PYG{p}{)}

\PYG{k}{with} \PYG{n}{curl}\PYG{o}{.}\PYG{n}{cache\PYGZus{}debug\PYGZus{}off}\PYG{p}{(}\PYG{p}{)}\PYG{p}{:}
    \PYG{n}{pa}\PYG{o}{.}\PYG{n}{load\PYGZus{}resources}\PYG{p}{(}\PYG{p}{\PYGZob{}}\PYG{l+s+s1}{\PYGZsq{}}\PYG{l+s+s1}{signor}\PYG{l+s+s1}{\PYGZsq{}}\PYG{p}{:} \PYG{n}{data\PYGZus{}formats}\PYG{o}{.}\PYG{n}{pathway}\PYG{p}{[}\PYG{l+s+s1}{\PYGZsq{}}\PYG{l+s+s1}{signor}\PYG{l+s+s1}{\PYGZsq{}}\PYG{p}{]}\PYG{p}{\PYGZcb{}}\PYG{p}{)}
\end{sphinxVerbatim}

\end{fulllineitems}

\index{debug\_on (class in pypath.share.curl)@\spxentry{debug\_on}\spxextra{class in pypath.share.curl}}

\begin{fulllineitems}
\phantomsection\label{\detokenize{reference:pypath.share.curl.debug_on}}\pysigline{\sphinxbfcode{\sphinxupquote{class }}\sphinxcode{\sphinxupquote{pypath.share.curl.}}\sphinxbfcode{\sphinxupquote{debug\_on}}}
This is a context handler which results pypath.curl.Curl() to print
debug information. 
This is useful if you have some issue with \sphinxtitleref{Curl}, and you want to
see what{}`s going on.

Behind the scenes it sets the value of the \sphinxtitleref{pypath.curl.DEBUG}
module level variable to \sphinxtitleref{True} (by default it is \sphinxtitleref{False}).

Example:

\begin{sphinxVerbatim}[commandchars=\\\{\}]
\PYG{k+kn}{import} \PYG{n+nn}{pypath}
\PYG{k+kn}{from} \PYG{n+nn}{pypath} \PYG{k}{import} \PYG{n}{curl}\PYG{p}{,} \PYG{n}{data\PYGZus{}formats}

\PYG{n}{pa} \PYG{o}{=} \PYG{n}{pypath}\PYG{o}{.}\PYG{n}{PyPath}\PYG{p}{(}\PYG{p}{)}

\PYG{k}{with} \PYG{n}{curl}\PYG{o}{.}\PYG{n}{cache\PYGZus{}debug\PYGZus{}on}\PYG{p}{(}\PYG{p}{)}\PYG{p}{:}
    \PYG{n}{pa}\PYG{o}{.}\PYG{n}{load\PYGZus{}resources}\PYG{p}{(}\PYG{p}{\PYGZob{}}\PYG{l+s+s1}{\PYGZsq{}}\PYG{l+s+s1}{signor}\PYG{l+s+s1}{\PYGZsq{}}\PYG{p}{:} \PYG{n}{data\PYGZus{}formats}\PYG{o}{.}\PYG{n}{pathway}\PYG{p}{[}\PYG{l+s+s1}{\PYGZsq{}}\PYG{l+s+s1}{signor}\PYG{l+s+s1}{\PYGZsq{}}\PYG{p}{]}\PYG{p}{\PYGZcb{}}\PYG{p}{)}
\end{sphinxVerbatim}

\end{fulllineitems}

\index{dryrun\_off (class in pypath.share.curl)@\spxentry{dryrun\_off}\spxextra{class in pypath.share.curl}}

\begin{fulllineitems}
\phantomsection\label{\detokenize{reference:pypath.share.curl.dryrun_off}}\pysigline{\sphinxbfcode{\sphinxupquote{class }}\sphinxcode{\sphinxupquote{pypath.share.curl.}}\sphinxbfcode{\sphinxupquote{dryrun\_off}}}
This is a context handler which results pypath.curl.Curl() to
perform download or cache read. This is the default behaviour,
so applying this context restores the default.

Behind the scenes it sets the value of the \sphinxtitleref{pypath.curl.DRYRUN}
module level variable to \sphinxtitleref{False}.

Example:

\begin{sphinxVerbatim}[commandchars=\\\{\}]
\PYG{k+kn}{import} \PYG{n+nn}{pypath}
\PYG{k+kn}{from} \PYG{n+nn}{pypath} \PYG{k}{import} \PYG{n}{curl}\PYG{p}{,} \PYG{n}{data\PYGZus{}formats}

\PYG{n}{pa} \PYG{o}{=} \PYG{n}{pypath}\PYG{o}{.}\PYG{n}{PyPath}\PYG{p}{(}\PYG{p}{)}

\PYG{k}{with} \PYG{n}{curl}\PYG{o}{.}\PYG{n}{cache\PYGZus{}dryrun\PYGZus{}off}\PYG{p}{(}\PYG{p}{)}\PYG{p}{:}
    \PYG{n}{pa}\PYG{o}{.}\PYG{n}{load\PYGZus{}resources}\PYG{p}{(}\PYG{p}{\PYGZob{}}\PYG{l+s+s1}{\PYGZsq{}}\PYG{l+s+s1}{signor}\PYG{l+s+s1}{\PYGZsq{}}\PYG{p}{:} \PYG{n}{data\PYGZus{}formats}\PYG{o}{.}\PYG{n}{pathway}\PYG{p}{[}\PYG{l+s+s1}{\PYGZsq{}}\PYG{l+s+s1}{signor}\PYG{l+s+s1}{\PYGZsq{}}\PYG{p}{]}\PYG{p}{\PYGZcb{}}\PYG{p}{)}
\end{sphinxVerbatim}

\end{fulllineitems}

\index{dryrun\_on (class in pypath.share.curl)@\spxentry{dryrun\_on}\spxextra{class in pypath.share.curl}}

\begin{fulllineitems}
\phantomsection\label{\detokenize{reference:pypath.share.curl.dryrun_on}}\pysigline{\sphinxbfcode{\sphinxupquote{class }}\sphinxcode{\sphinxupquote{pypath.share.curl.}}\sphinxbfcode{\sphinxupquote{dryrun\_on}}}
This is a context handler which results pypath.curl.Curl() to do all
setup steps, but do not perform download or cache read.

Behind the scenes it sets the value of the \sphinxtitleref{pypath.curl.DRYRUN}
module level variable to \sphinxtitleref{True} (by default it is \sphinxtitleref{False}).

Example:

\begin{sphinxVerbatim}[commandchars=\\\{\}]
\PYG{k+kn}{import} \PYG{n+nn}{pypath}
\PYG{k+kn}{from} \PYG{n+nn}{pypath} \PYG{k}{import} \PYG{n}{curl}\PYG{p}{,} \PYG{n}{data\PYGZus{}formats}

\PYG{n}{pa} \PYG{o}{=} \PYG{n}{pypath}\PYG{o}{.}\PYG{n}{PyPath}\PYG{p}{(}\PYG{p}{)}

\PYG{k}{with} \PYG{n}{curl}\PYG{o}{.}\PYG{n}{cache\PYGZus{}dryrun\PYGZus{}on}\PYG{p}{(}\PYG{p}{)}\PYG{p}{:}
    \PYG{n}{pa}\PYG{o}{.}\PYG{n}{load\PYGZus{}resources}\PYG{p}{(}\PYG{p}{\PYGZob{}}\PYG{l+s+s1}{\PYGZsq{}}\PYG{l+s+s1}{signor}\PYG{l+s+s1}{\PYGZsq{}}\PYG{p}{:} \PYG{n}{data\PYGZus{}formats}\PYG{o}{.}\PYG{n}{pathway}\PYG{p}{[}\PYG{l+s+s1}{\PYGZsq{}}\PYG{l+s+s1}{signor}\PYG{l+s+s1}{\PYGZsq{}}\PYG{p}{]}\PYG{p}{\PYGZcb{}}\PYG{p}{)}
\end{sphinxVerbatim}

\end{fulllineitems}

\index{preserve\_off (class in pypath.share.curl)@\spxentry{preserve\_off}\spxextra{class in pypath.share.curl}}

\begin{fulllineitems}
\phantomsection\label{\detokenize{reference:pypath.share.curl.preserve_off}}\pysigline{\sphinxbfcode{\sphinxupquote{class }}\sphinxcode{\sphinxupquote{pypath.share.curl.}}\sphinxbfcode{\sphinxupquote{preserve\_off}}}
This is a context handler which avoids pypath.curl.Curl() to make
a reference to itself in the module level variable \sphinxtitleref{LASTCURL}. By
default it does not do this, so this context only restores the
default.

Behind the scenes it sets the value of the \sphinxtitleref{pypath.curl.PRESERVE}
module level variable to \sphinxtitleref{False}.

Example:

\begin{sphinxVerbatim}[commandchars=\\\{\}]
\PYG{k+kn}{import} \PYG{n+nn}{pypath}
\PYG{k+kn}{from} \PYG{n+nn}{pypath} \PYG{k}{import} \PYG{n}{curl}\PYG{p}{,} \PYG{n}{data\PYGZus{}formats}

\PYG{n}{pa} \PYG{o}{=} \PYG{n}{pypath}\PYG{o}{.}\PYG{n}{PyPath}\PYG{p}{(}\PYG{p}{)}

\PYG{k}{with} \PYG{n}{curl}\PYG{o}{.}\PYG{n}{cache\PYGZus{}preserve\PYGZus{}off}\PYG{p}{(}\PYG{p}{)}\PYG{p}{:}
    \PYG{n}{pa}\PYG{o}{.}\PYG{n}{load\PYGZus{}resources}\PYG{p}{(}\PYG{p}{\PYGZob{}}\PYG{l+s+s1}{\PYGZsq{}}\PYG{l+s+s1}{signor}\PYG{l+s+s1}{\PYGZsq{}}\PYG{p}{:} \PYG{n}{data\PYGZus{}formats}\PYG{o}{.}\PYG{n}{pathway}\PYG{p}{[}\PYG{l+s+s1}{\PYGZsq{}}\PYG{l+s+s1}{signor}\PYG{l+s+s1}{\PYGZsq{}}\PYG{p}{]}\PYG{p}{\PYGZcb{}}\PYG{p}{)}
\end{sphinxVerbatim}

\end{fulllineitems}

\index{preserve\_on (class in pypath.share.curl)@\spxentry{preserve\_on}\spxextra{class in pypath.share.curl}}

\begin{fulllineitems}
\phantomsection\label{\detokenize{reference:pypath.share.curl.preserve_on}}\pysigline{\sphinxbfcode{\sphinxupquote{class }}\sphinxcode{\sphinxupquote{pypath.share.curl.}}\sphinxbfcode{\sphinxupquote{preserve\_on}}}
This is a context handler which results pypath.curl.Curl() to make
a reference to itself in the module level variable \sphinxtitleref{LASTCURL}. This
is useful if you have some issue with \sphinxtitleref{Curl}, and you want to access
the instance for debugging.

Behind the scenes it sets the value of the \sphinxtitleref{pypath.curl.PRESERVE}
module level variable to \sphinxtitleref{True} (by default it is \sphinxtitleref{False}).

Example:

\begin{sphinxVerbatim}[commandchars=\\\{\}]
\PYG{k+kn}{import} \PYG{n+nn}{pypath}
\PYG{k+kn}{from} \PYG{n+nn}{pypath} \PYG{k}{import} \PYG{n}{curl}\PYG{p}{,} \PYG{n}{data\PYGZus{}formats}

\PYG{n}{pa} \PYG{o}{=} \PYG{n}{pypath}\PYG{o}{.}\PYG{n}{PyPath}\PYG{p}{(}\PYG{p}{)}

\PYG{k}{with} \PYG{n}{curl}\PYG{o}{.}\PYG{n}{cache\PYGZus{}preserve\PYGZus{}on}\PYG{p}{(}\PYG{p}{)}\PYG{p}{:}
    \PYG{n}{pa}\PYG{o}{.}\PYG{n}{load\PYGZus{}resources}\PYG{p}{(}\PYG{p}{\PYGZob{}}\PYG{l+s+s1}{\PYGZsq{}}\PYG{l+s+s1}{signor}\PYG{l+s+s1}{\PYGZsq{}}\PYG{p}{:} \PYG{n}{data\PYGZus{}formats}\PYG{o}{.}\PYG{n}{pathway}\PYG{p}{[}\PYG{l+s+s1}{\PYGZsq{}}\PYG{l+s+s1}{signor}\PYG{l+s+s1}{\PYGZsq{}}\PYG{p}{]}\PYG{p}{\PYGZcb{}}\PYG{p}{)}
\end{sphinxVerbatim}

\end{fulllineitems}



\section{data\_formats}
\label{\detokenize{reference:module-pypath.resources.data_formats}}\label{\detokenize{reference:data-formats}}\index{pypath.resources.data\_formats (module)@\spxentry{pypath.resources.data\_formats}\spxextra{module}}\index{interaction (in module pypath.resources.data\_formats)@\spxentry{interaction}\spxextra{in module pypath.resources.data\_formats}}

\begin{fulllineitems}
\phantomsection\label{\detokenize{reference:pypath.resources.data_formats.interaction}}\pysigline{\sphinxcode{\sphinxupquote{pypath.resources.data\_formats.}}\sphinxbfcode{\sphinxupquote{interaction}}\sphinxbfcode{\sphinxupquote{ = \{'alz': \textless{}pypath.internals.input\_formats.NetworkInput object\textgreater{}, 'biogrid': \textless{}pypath.internals.input\_formats.NetworkInput object\textgreater{}, 'ccmap': \textless{}pypath.internals.input\_formats.NetworkInput object\textgreater{}, 'dip': \textless{}pypath.internals.input\_formats.NetworkInput object\textgreater{}, 'innatedb': \textless{}pypath.internals.input\_formats.NetworkInput object\textgreater{}, 'matrixdb': \textless{}pypath.internals.input\_formats.NetworkInput object\textgreater{}, 'mppi': \textless{}pypath.internals.input\_formats.NetworkInput object\textgreater{}, 'netpath': \textless{}pypath.internals.input\_formats.NetworkInput object\textgreater{}\}}}}
PTM databases included in OmniPath.
These supply large sets of directed interactions.

\end{fulllineitems}

\index{interaction\_htp (in module pypath.resources.data\_formats)@\spxentry{interaction\_htp}\spxextra{in module pypath.resources.data\_formats}}

\begin{fulllineitems}
\phantomsection\label{\detokenize{reference:pypath.resources.data_formats.interaction_htp}}\pysigline{\sphinxcode{\sphinxupquote{pypath.resources.data\_formats.}}\sphinxbfcode{\sphinxupquote{interaction\_htp}}\sphinxbfcode{\sphinxupquote{ = \{'biogrid': \textless{}pypath.internals.input\_formats.NetworkInput object\textgreater{}, 'ccmap': \textless{}pypath.internals.input\_formats.NetworkInput object\textgreater{}, 'dip': \textless{}pypath.internals.input\_formats.NetworkInput object\textgreater{}, 'hprd': \textless{}pypath.internals.input\_formats.NetworkInput object\textgreater{}, 'innatedb': \textless{}pypath.internals.input\_formats.NetworkInput object\textgreater{}, 'intact': \textless{}pypath.internals.input\_formats.NetworkInput object\textgreater{}, 'matrixdb': \textless{}pypath.internals.input\_formats.NetworkInput object\textgreater{}, 'mppi': \textless{}pypath.internals.input\_formats.NetworkInput object\textgreater{}\}}}}
Transcriptional regulatory interactions.

\end{fulllineitems}

\index{obsolate (in module pypath.resources.data\_formats)@\spxentry{obsolate}\spxextra{in module pypath.resources.data\_formats}}

\begin{fulllineitems}
\phantomsection\label{\detokenize{reference:pypath.resources.data_formats.obsolate}}\pysigline{\sphinxcode{\sphinxupquote{pypath.resources.data\_formats.}}\sphinxbfcode{\sphinxupquote{obsolate}}\sphinxbfcode{\sphinxupquote{ = \{'nci\_pid': \textless{}pypath.internals.input\_formats.NetworkInput object\textgreater{}, 'signalink2': \textless{}pypath.internals.input\_formats.NetworkInput object\textgreater{}\}}}}
Reaction databases.
These are not included in OmniPath, because only a minor
part of their content can be used when processing along
strict conditions to have only binary interactions with
references.

\end{fulllineitems}

\index{transcription\_deprecated (in module pypath.resources.data\_formats)@\spxentry{transcription\_deprecated}\spxextra{in module pypath.resources.data\_formats}}

\begin{fulllineitems}
\phantomsection\label{\detokenize{reference:pypath.resources.data_formats.transcription_deprecated}}\pysigline{\sphinxcode{\sphinxupquote{pypath.resources.data\_formats.}}\sphinxbfcode{\sphinxupquote{transcription\_deprecated}}\sphinxbfcode{\sphinxupquote{ = \{'oreganno\_old': \textless{}pypath.internals.input\_formats.NetworkInput object\textgreater{}\}}}}
miRNA-target resources

\end{fulllineitems}

\index{transcription\_onebyone (in module pypath.resources.data\_formats)@\spxentry{transcription\_onebyone}\spxextra{in module pypath.resources.data\_formats}}

\begin{fulllineitems}
\phantomsection\label{\detokenize{reference:pypath.resources.data_formats.transcription_onebyone}}\pysigline{\sphinxcode{\sphinxupquote{pypath.resources.data\_formats.}}\sphinxbfcode{\sphinxupquote{transcription\_onebyone}}\sphinxbfcode{\sphinxupquote{ = \{'abs': \textless{}pypath.internals.input\_formats.NetworkInput object\textgreater{}, 'encode\_dist': \textless{}pypath.internals.input\_formats.NetworkInput object\textgreater{}, 'encode\_prox': \textless{}pypath.internals.input\_formats.NetworkInput object\textgreater{}, 'htri': \textless{}pypath.internals.input\_formats.NetworkInput object\textgreater{}, 'oreganno': \textless{}pypath.internals.input\_formats.NetworkInput object\textgreater{}, 'pazar': \textless{}pypath.internals.input\_formats.NetworkInput object\textgreater{}, 'signor': \textless{}pypath.internals.input\_formats.NetworkInput object\textgreater{}\}}}}
New default transcription dataset is only TFregulons
as it is already an integrated resource and
has sufficient coverage.

import pypath

\# load only \sphinxtitleref{A} confidence level:
pypath.data\_formats.transcription{[}‘tfregulons’{]}.input\_args{[}‘levels’{]} = \{‘A’\}
pa = pypath.PyPath()
pa.init\_network(pypath.data\_formats.transcription)
\begin{description}
\item[{pypath.data\_formats.transcription{[}‘tfregulons’{]}.input\_args{[}‘levels’{]} = \{}] \leavevmode
‘A’, ‘B’, ‘C’, ‘D’

\end{description}

\}
pa = pypath.PyPath()
pa.init\_network(pypath.data\_formats.transcription)

\end{fulllineitems}



\section{dataio}
\label{\detokenize{reference:module-pypath.inputs.main}}\label{\detokenize{reference:dataio}}\index{pypath.inputs.main (module)@\spxentry{pypath.inputs.main}\spxextra{module}}\index{CellPhoneDBAnnotation (class in pypath.inputs.main)@\spxentry{CellPhoneDBAnnotation}\spxextra{class in pypath.inputs.main}}

\begin{fulllineitems}
\phantomsection\label{\detokenize{reference:pypath.inputs.main.CellPhoneDBAnnotation}}\pysiglinewithargsret{\sphinxbfcode{\sphinxupquote{class }}\sphinxcode{\sphinxupquote{pypath.inputs.main.}}\sphinxbfcode{\sphinxupquote{CellPhoneDBAnnotation}}}{\emph{receptor}, \emph{receptor\_class}, \emph{peripheral}, \emph{secreted}, \emph{secreted\_class}, \emph{transmembrane}, \emph{integrin}}{}~\index{integrin (pypath.inputs.main.CellPhoneDBAnnotation attribute)@\spxentry{integrin}\spxextra{pypath.inputs.main.CellPhoneDBAnnotation attribute}}

\begin{fulllineitems}
\phantomsection\label{\detokenize{reference:pypath.inputs.main.CellPhoneDBAnnotation.integrin}}\pysigline{\sphinxbfcode{\sphinxupquote{integrin}}}
Alias for field number 6

\end{fulllineitems}

\index{peripheral (pypath.inputs.main.CellPhoneDBAnnotation attribute)@\spxentry{peripheral}\spxextra{pypath.inputs.main.CellPhoneDBAnnotation attribute}}

\begin{fulllineitems}
\phantomsection\label{\detokenize{reference:pypath.inputs.main.CellPhoneDBAnnotation.peripheral}}\pysigline{\sphinxbfcode{\sphinxupquote{peripheral}}}
Alias for field number 2

\end{fulllineitems}

\index{receptor (pypath.inputs.main.CellPhoneDBAnnotation attribute)@\spxentry{receptor}\spxextra{pypath.inputs.main.CellPhoneDBAnnotation attribute}}

\begin{fulllineitems}
\phantomsection\label{\detokenize{reference:pypath.inputs.main.CellPhoneDBAnnotation.receptor}}\pysigline{\sphinxbfcode{\sphinxupquote{receptor}}}
Alias for field number 0

\end{fulllineitems}

\index{receptor\_class (pypath.inputs.main.CellPhoneDBAnnotation attribute)@\spxentry{receptor\_class}\spxextra{pypath.inputs.main.CellPhoneDBAnnotation attribute}}

\begin{fulllineitems}
\phantomsection\label{\detokenize{reference:pypath.inputs.main.CellPhoneDBAnnotation.receptor_class}}\pysigline{\sphinxbfcode{\sphinxupquote{receptor\_class}}}
Alias for field number 1

\end{fulllineitems}

\index{secreted (pypath.inputs.main.CellPhoneDBAnnotation attribute)@\spxentry{secreted}\spxextra{pypath.inputs.main.CellPhoneDBAnnotation attribute}}

\begin{fulllineitems}
\phantomsection\label{\detokenize{reference:pypath.inputs.main.CellPhoneDBAnnotation.secreted}}\pysigline{\sphinxbfcode{\sphinxupquote{secreted}}}
Alias for field number 3

\end{fulllineitems}

\index{secreted\_class (pypath.inputs.main.CellPhoneDBAnnotation attribute)@\spxentry{secreted\_class}\spxextra{pypath.inputs.main.CellPhoneDBAnnotation attribute}}

\begin{fulllineitems}
\phantomsection\label{\detokenize{reference:pypath.inputs.main.CellPhoneDBAnnotation.secreted_class}}\pysigline{\sphinxbfcode{\sphinxupquote{secreted\_class}}}
Alias for field number 4

\end{fulllineitems}

\index{transmembrane (pypath.inputs.main.CellPhoneDBAnnotation attribute)@\spxentry{transmembrane}\spxextra{pypath.inputs.main.CellPhoneDBAnnotation attribute}}

\begin{fulllineitems}
\phantomsection\label{\detokenize{reference:pypath.inputs.main.CellPhoneDBAnnotation.transmembrane}}\pysigline{\sphinxbfcode{\sphinxupquote{transmembrane}}}
Alias for field number 5

\end{fulllineitems}


\end{fulllineitems}

\index{ResidueMapper (class in pypath.inputs.main)@\spxentry{ResidueMapper}\spxextra{class in pypath.inputs.main}}

\begin{fulllineitems}
\phantomsection\label{\detokenize{reference:pypath.inputs.main.ResidueMapper}}\pysigline{\sphinxbfcode{\sphinxupquote{class }}\sphinxcode{\sphinxupquote{pypath.inputs.main.}}\sphinxbfcode{\sphinxupquote{ResidueMapper}}}
This class stores and serves the PDB \textendash{}\textgreater{} UniProt
residue level mapping. Attempts to download the
mapping, and stores it for further use. Converts
PDB residue numbers to the corresponding UniProt ones.
\index{clean() (pypath.inputs.main.ResidueMapper method)@\spxentry{clean()}\spxextra{pypath.inputs.main.ResidueMapper method}}

\begin{fulllineitems}
\phantomsection\label{\detokenize{reference:pypath.inputs.main.ResidueMapper.clean}}\pysiglinewithargsret{\sphinxbfcode{\sphinxupquote{clean}}}{}{}
Removes cached mappings, freeing up memory.

\end{fulllineitems}


\end{fulllineitems}

\index{acsn\_interactions() (in module pypath.inputs.main)@\spxentry{acsn\_interactions()}\spxextra{in module pypath.inputs.main}}

\begin{fulllineitems}
\phantomsection\label{\detokenize{reference:pypath.inputs.main.acsn_interactions}}\pysiglinewithargsret{\sphinxcode{\sphinxupquote{pypath.inputs.main.}}\sphinxbfcode{\sphinxupquote{acsn\_interactions}}}{\emph{keep\_in\_complex\_interactions=True}}{}
Processes ACSN data from local file.
Returns list of interactions.
\begin{description}
\item[{@keep\_in\_complex\_interactions}] \leavevmode{[}bool{]}
Whether to include interactions from complex expansion.

\end{description}

\end{fulllineitems}

\index{biogrid\_interactions() (in module pypath.inputs.main)@\spxentry{biogrid\_interactions()}\spxextra{in module pypath.inputs.main}}

\begin{fulllineitems}
\phantomsection\label{\detokenize{reference:pypath.inputs.main.biogrid_interactions}}\pysiglinewithargsret{\sphinxcode{\sphinxupquote{pypath.inputs.main.}}\sphinxbfcode{\sphinxupquote{biogrid\_interactions}}}{\emph{organism=9606}, \emph{htp\_limit=1}, \emph{ltp=True}}{}
Downloads and processes BioGRID interactions.
Keeps only the “low throughput” interactions.
Returns list of interactions.
\begin{description}
\item[{@organism}] \leavevmode{[}int{]}
NCBI Taxonomy ID of organism.

\item[{@htp\_limit}] \leavevmode{[}int{]}
Exclude interactions only from references
cited at more than this number of interactions.

\end{description}

\end{fulllineitems}

\index{cancer\_gene\_census\_annotations() (in module pypath.inputs.main)@\spxentry{cancer\_gene\_census\_annotations()}\spxextra{in module pypath.inputs.main}}

\begin{fulllineitems}
\phantomsection\label{\detokenize{reference:pypath.inputs.main.cancer_gene_census_annotations}}\pysiglinewithargsret{\sphinxcode{\sphinxupquote{pypath.inputs.main.}}\sphinxbfcode{\sphinxupquote{cancer\_gene\_census\_annotations}}}{\emph{user=None}, \emph{passwd=None}, \emph{credentials\_fname='cosmic\_credentials'}}{}
Retrieves a list of cancer driver genes (Cancer Gene Census) from
the Sanger COSMIC (Catalogue of Somatic Mutations in Cancer) database.

Does not work at the moment (signature does not match error).

\end{fulllineitems}

\index{cancersea\_annotations() (in module pypath.inputs.main)@\spxentry{cancersea\_annotations()}\spxextra{in module pypath.inputs.main}}

\begin{fulllineitems}
\phantomsection\label{\detokenize{reference:pypath.inputs.main.cancersea_annotations}}\pysiglinewithargsret{\sphinxcode{\sphinxupquote{pypath.inputs.main.}}\sphinxbfcode{\sphinxupquote{cancersea\_annotations}}}{}{}
Retrieves genes annotated with cancer funcitonal states from the
CancerSEA database.

\end{fulllineitems}

\index{cellphonedb\_ligands\_receptors() (in module pypath.inputs.main)@\spxentry{cellphonedb\_ligands\_receptors()}\spxextra{in module pypath.inputs.main}}

\begin{fulllineitems}
\phantomsection\label{\detokenize{reference:pypath.inputs.main.cellphonedb_ligands_receptors}}\pysiglinewithargsret{\sphinxcode{\sphinxupquote{pypath.inputs.main.}}\sphinxbfcode{\sphinxupquote{cellphonedb\_ligands\_receptors}}}{}{}
Retrieves the set of ligands and receptors from CellPhoneDB.
Returns tuple of sets.

\end{fulllineitems}

\index{cellphonedb\_protein\_annotations() (in module pypath.inputs.main)@\spxentry{cellphonedb\_protein\_annotations()}\spxextra{in module pypath.inputs.main}}

\begin{fulllineitems}
\phantomsection\label{\detokenize{reference:pypath.inputs.main.cellphonedb_protein_annotations}}\pysiglinewithargsret{\sphinxcode{\sphinxupquote{pypath.inputs.main.}}\sphinxbfcode{\sphinxupquote{cellphonedb\_protein\_annotations}}}{\emph{add\_complex\_annotations=True}}{}~\begin{quote}\begin{description}
\item[{Parameters}] \leavevmode
\sphinxstyleliteralstrong{\sphinxupquote{add\_complex\_annotations}} (\sphinxstyleliteralemphasis{\sphinxupquote{bool}}) \textendash{} Copy the annotations of complexes to each of their member proteins.

\end{description}\end{quote}

\end{fulllineitems}

\index{compleat\_complexes() (in module pypath.inputs.main)@\spxentry{compleat\_complexes()}\spxextra{in module pypath.inputs.main}}

\begin{fulllineitems}
\phantomsection\label{\detokenize{reference:pypath.inputs.main.compleat_complexes}}\pysiglinewithargsret{\sphinxcode{\sphinxupquote{pypath.inputs.main.}}\sphinxbfcode{\sphinxupquote{compleat\_complexes}}}{\emph{predicted=True}}{}
Retrieves complexes from the Compleat database.

\end{fulllineitems}

\index{complexportal\_complexes() (in module pypath.inputs.main)@\spxentry{complexportal\_complexes()}\spxextra{in module pypath.inputs.main}}

\begin{fulllineitems}
\phantomsection\label{\detokenize{reference:pypath.inputs.main.complexportal_complexes}}\pysiglinewithargsret{\sphinxcode{\sphinxupquote{pypath.inputs.main.}}\sphinxbfcode{\sphinxupquote{complexportal\_complexes}}}{\emph{organism=9606}, \emph{return\_details=False}}{}
Complex dataset from IntAct.
See more:
\sphinxurl{http://www.ebi.ac.uk/intact/complex/}
\sphinxurl{http://nar.oxfordjournals.org/content/early/2014/10/13/nar.gku975.full.pdf}

\end{fulllineitems}

\index{comppi\_interaction\_locations() (in module pypath.inputs.main)@\spxentry{comppi\_interaction\_locations()}\spxextra{in module pypath.inputs.main}}

\begin{fulllineitems}
\phantomsection\label{\detokenize{reference:pypath.inputs.main.comppi_interaction_locations}}\pysiglinewithargsret{\sphinxcode{\sphinxupquote{pypath.inputs.main.}}\sphinxbfcode{\sphinxupquote{comppi\_interaction\_locations}}}{\emph{organism=9606}}{}
Downloads and preprocesses protein interaction and cellular compartment
association data from the ComPPI database.
This data provides scores for occurrence of protein-protein interactions
in various compartments.

\end{fulllineitems}

\index{cpad\_pathway\_cancer() (in module pypath.inputs.main)@\spxentry{cpad\_pathway\_cancer()}\spxextra{in module pypath.inputs.main}}

\begin{fulllineitems}
\phantomsection\label{\detokenize{reference:pypath.inputs.main.cpad_pathway_cancer}}\pysiglinewithargsret{\sphinxcode{\sphinxupquote{pypath.inputs.main.}}\sphinxbfcode{\sphinxupquote{cpad\_pathway\_cancer}}}{}{}
Collects only the pathway-cancer relationships. Returns sets of records
grouped in dicts by cancer and by pathway.

\end{fulllineitems}

\index{dgidb\_annotations() (in module pypath.inputs.main)@\spxentry{dgidb\_annotations()}\spxextra{in module pypath.inputs.main}}

\begin{fulllineitems}
\phantomsection\label{\detokenize{reference:pypath.inputs.main.dgidb_annotations}}\pysiglinewithargsret{\sphinxcode{\sphinxupquote{pypath.inputs.main.}}\sphinxbfcode{\sphinxupquote{dgidb\_annotations}}}{}{}
Downloads druggable protein annotations from DGIdb.

\end{fulllineitems}

\index{dip\_login() (in module pypath.inputs.main)@\spxentry{dip\_login()}\spxextra{in module pypath.inputs.main}}

\begin{fulllineitems}
\phantomsection\label{\detokenize{reference:pypath.inputs.main.dip_login}}\pysiglinewithargsret{\sphinxcode{\sphinxupquote{pypath.inputs.main.}}\sphinxbfcode{\sphinxupquote{dip\_login}}}{\emph{user}, \emph{passwd}}{}
This does not work for unknown reasons.

In addition, the binary\_data parameter of Curl().\_\_init\_\_() has been changed,
below updates are necessary.

\end{fulllineitems}

\index{disgenet\_annotations() (in module pypath.inputs.main)@\spxentry{disgenet\_annotations()}\spxextra{in module pypath.inputs.main}}

\begin{fulllineitems}
\phantomsection\label{\detokenize{reference:pypath.inputs.main.disgenet_annotations}}\pysiglinewithargsret{\sphinxcode{\sphinxupquote{pypath.inputs.main.}}\sphinxbfcode{\sphinxupquote{disgenet\_annotations}}}{\emph{dataset='curated'}}{}
Downloads and processes the list of all human disease related proteins
from DisGeNet.
Returns dict of dicts.
\begin{description}
\item[{@dataset}] \leavevmode{[}str{]}
Name of DisGeNet dataset to be obtained:
\sphinxtitleref{curated}, \sphinxtitleref{literature}, \sphinxtitleref{befree} or \sphinxtitleref{all}.

\end{description}

\end{fulllineitems}

\index{dorothea\_interactions() (in module pypath.inputs.main)@\spxentry{dorothea\_interactions()}\spxextra{in module pypath.inputs.main}}

\begin{fulllineitems}
\phantomsection\label{\detokenize{reference:pypath.inputs.main.dorothea_interactions}}\pysiglinewithargsret{\sphinxcode{\sphinxupquote{pypath.inputs.main.}}\sphinxbfcode{\sphinxupquote{dorothea\_interactions}}}{\emph{levels=\{'A'}, \emph{'B'\}}, \emph{only\_curated=False}}{}
Retrieves TF-target interactions from TF regulons.
\begin{quote}\begin{description}
\item[{Parameters}] \leavevmode\begin{itemize}
\item {} 
\sphinxstyleliteralstrong{\sphinxupquote{levels}} (\sphinxstyleliteralemphasis{\sphinxupquote{set}}) \textendash{} Confidence levels to be used.

\item {} 
\sphinxstyleliteralstrong{\sphinxupquote{only\_curated}} (\sphinxstyleliteralemphasis{\sphinxupquote{bool}}) \textendash{} Retrieve only literature curated interactions.

\end{itemize}

\end{description}\end{quote}

TF regulons is a comprehensive resource of TF-target interactions
combining multiple lines of evidences: literature curated databases,
ChIP-Seq data, PWM based prediction using HOCOMOCO and JASPAR matrices
and prediction from GTEx expression data by ARACNe.

For details see \sphinxurl{https://github.com/saezlab/DoRothEA}.

\end{fulllineitems}

\index{elm\_interactions() (in module pypath.inputs.main)@\spxentry{elm\_interactions()}\spxextra{in module pypath.inputs.main}}

\begin{fulllineitems}
\phantomsection\label{\detokenize{reference:pypath.inputs.main.elm_interactions}}\pysiglinewithargsret{\sphinxcode{\sphinxupquote{pypath.inputs.main.}}\sphinxbfcode{\sphinxupquote{elm\_interactions}}}{}{}
Downlods manually curated interactions from ELM.
This is the gold standard set of ELM.

\end{fulllineitems}

\index{get\_3dcomplex() (in module pypath.inputs.main)@\spxentry{get\_3dcomplex()}\spxextra{in module pypath.inputs.main}}

\begin{fulllineitems}
\phantomsection\label{\detokenize{reference:pypath.inputs.main.get_3dcomplex}}\pysiglinewithargsret{\sphinxcode{\sphinxupquote{pypath.inputs.main.}}\sphinxbfcode{\sphinxupquote{get\_3dcomplex}}}{}{}
Downloads and preprocesses data from the 3DComplex database.

Returns dict of dicts where top level keys are PDB IDs, second level
keys are pairs of tuples of UniProt IDs and values are list with the
number of amino acids in contact.

\end{fulllineitems}

\index{get\_acsn\_effects() (in module pypath.inputs.main)@\spxentry{get\_acsn\_effects()}\spxextra{in module pypath.inputs.main}}

\begin{fulllineitems}
\phantomsection\label{\detokenize{reference:pypath.inputs.main.get_acsn_effects}}\pysiglinewithargsret{\sphinxcode{\sphinxupquote{pypath.inputs.main.}}\sphinxbfcode{\sphinxupquote{get\_acsn\_effects}}}{}{}
Processes ACSN data, returns list of effects.

\end{fulllineitems}

\index{get\_ca1() (in module pypath.inputs.main)@\spxentry{get\_ca1()}\spxextra{in module pypath.inputs.main}}

\begin{fulllineitems}
\phantomsection\label{\detokenize{reference:pypath.inputs.main.get_ca1}}\pysiglinewithargsret{\sphinxcode{\sphinxupquote{pypath.inputs.main.}}\sphinxbfcode{\sphinxupquote{get\_ca1}}}{}{}
Downloads and processes the CA1 signaling network (Ma’ayan 2005).
Returns list of interactions.

\end{fulllineitems}

\index{get\_ccmap() (in module pypath.inputs.main)@\spxentry{get\_ccmap()}\spxextra{in module pypath.inputs.main}}

\begin{fulllineitems}
\phantomsection\label{\detokenize{reference:pypath.inputs.main.get_ccmap}}\pysiglinewithargsret{\sphinxcode{\sphinxupquote{pypath.inputs.main.}}\sphinxbfcode{\sphinxupquote{get\_ccmap}}}{\emph{organism=9606}}{}
Downloads and processes CancerCellMap.
Returns list of interactions.
\begin{description}
\item[{@organism}] \leavevmode{[}int{]}
NCBI Taxonomy ID to match column \#7 in nodes file.

\end{description}

\end{fulllineitems}

\index{get\_csa() (in module pypath.inputs.main)@\spxentry{get\_csa()}\spxextra{in module pypath.inputs.main}}

\begin{fulllineitems}
\phantomsection\label{\detokenize{reference:pypath.inputs.main.get_csa}}\pysiglinewithargsret{\sphinxcode{\sphinxupquote{pypath.inputs.main.}}\sphinxbfcode{\sphinxupquote{get\_csa}}}{\emph{uniprots=None}}{}
Downloads and preprocesses catalytic sites data.
This data tells which residues are involved in the catalytic
activity of one protein.

\end{fulllineitems}

\index{get\_dgidb\_old() (in module pypath.inputs.main)@\spxentry{get\_dgidb\_old()}\spxextra{in module pypath.inputs.main}}

\begin{fulllineitems}
\phantomsection\label{\detokenize{reference:pypath.inputs.main.get_dgidb_old}}\pysiglinewithargsret{\sphinxcode{\sphinxupquote{pypath.inputs.main.}}\sphinxbfcode{\sphinxupquote{get\_dgidb\_old}}}{}{}
Deprecated. Will be removed soon.

Downloads and processes the list of all human druggable proteins.
Returns a list of GeneSymbols.

\end{fulllineitems}

\index{get\_domino\_ptms() (in module pypath.inputs.main)@\spxentry{get\_domino\_ptms()}\spxextra{in module pypath.inputs.main}}

\begin{fulllineitems}
\phantomsection\label{\detokenize{reference:pypath.inputs.main.get_domino_ptms}}\pysiglinewithargsret{\sphinxcode{\sphinxupquote{pypath.inputs.main.}}\sphinxbfcode{\sphinxupquote{get\_domino\_ptms}}}{}{}
The table comes from dataio.get\_domino(), having the following fields:
header = {[}‘uniprot-A’, ‘uniprot-B’, ‘isoform-A’, ‘isoform-B’, \#3
‘exp. method’, ‘references’, ‘taxon-A’, ‘taxon-B’, \#7
‘role-A’, ‘role-B’, ‘binding-site-range-A’, ‘binding-site-range-B’, \#11
‘domains-A’, ‘domains-B’, ‘ptm-residue-A’, ‘ptm-residue-B’, \#15
‘ptm-type-mi-A’, ‘ptm-type-mi-B’, ‘ptm-type-A’, ‘ptm-type-B’, \#19
‘ptm-res-name-A’, ‘ptm-res-name-B’, ‘mutations-A’, ‘mutations-B’, \#23
‘mutation-effects-A’, ‘mutation-effects-B’, ‘domains-interpro-A’, \#26
‘domains-interpro-B’, ‘negative’{]} \#28

\end{fulllineitems}

\index{get\_dorothea() (in module pypath.inputs.main)@\spxentry{get\_dorothea()}\spxextra{in module pypath.inputs.main}}

\begin{fulllineitems}
\phantomsection\label{\detokenize{reference:pypath.inputs.main.get_dorothea}}\pysiglinewithargsret{\sphinxcode{\sphinxupquote{pypath.inputs.main.}}\sphinxbfcode{\sphinxupquote{get\_dorothea}}}{\emph{levels=\{'A'}, \emph{'B'\}}, \emph{only\_curated=False}}{}
Retrieves TF-target interactions from TF regulons.
\begin{quote}\begin{description}
\item[{Parameters}] \leavevmode\begin{itemize}
\item {} 
\sphinxstyleliteralstrong{\sphinxupquote{levels}} (\sphinxstyleliteralemphasis{\sphinxupquote{set}}) \textendash{} Confidence levels to be used.

\item {} 
\sphinxstyleliteralstrong{\sphinxupquote{only\_curated}} (\sphinxstyleliteralemphasis{\sphinxupquote{bool}}) \textendash{} Retrieve only literature curated interactions.

\end{itemize}

\end{description}\end{quote}

TF regulons is a comprehensive resource of TF-target interactions
combining multiple lines of evidences: literature curated databases,
ChIP-Seq data, PWM based prediction using HOCOMOCO and JASPAR matrices
and prediction from GTEx expression data by ARACNe.

For details see \sphinxurl{https://github.com/saezlab/DoRothEA}.

\end{fulllineitems}

\index{get\_exocarta() (in module pypath.inputs.main)@\spxentry{get\_exocarta()}\spxextra{in module pypath.inputs.main}}

\begin{fulllineitems}
\phantomsection\label{\detokenize{reference:pypath.inputs.main.get_exocarta}}\pysiglinewithargsret{\sphinxcode{\sphinxupquote{pypath.inputs.main.}}\sphinxbfcode{\sphinxupquote{get\_exocarta}}}{\emph{organism=9606}, \emph{types=None}}{}~\begin{quote}\begin{description}
\item[{Parameters}] \leavevmode
\sphinxstyleliteralstrong{\sphinxupquote{types}} (\sphinxstyleliteralemphasis{\sphinxupquote{set}}) \textendash{} Molecule types to retrieve. Possible values: \sphinxtitleref{protein}, \sphinxtitleref{mrna}.

\end{description}\end{quote}

\end{fulllineitems}

\index{get\_go\_desc() (in module pypath.inputs.main)@\spxentry{get\_go\_desc()}\spxextra{in module pypath.inputs.main}}

\begin{fulllineitems}
\phantomsection\label{\detokenize{reference:pypath.inputs.main.get_go_desc}}\pysiglinewithargsret{\sphinxcode{\sphinxupquote{pypath.inputs.main.}}\sphinxbfcode{\sphinxupquote{get\_go\_desc}}}{\emph{go\_ids}, \emph{organism=9606}}{}
Deprecated, should be removed soon.

\end{fulllineitems}

\index{get\_go\_quick() (in module pypath.inputs.main)@\spxentry{get\_go\_quick()}\spxextra{in module pypath.inputs.main}}

\begin{fulllineitems}
\phantomsection\label{\detokenize{reference:pypath.inputs.main.get_go_quick}}\pysiglinewithargsret{\sphinxcode{\sphinxupquote{pypath.inputs.main.}}\sphinxbfcode{\sphinxupquote{get\_go\_quick}}}{\emph{organism=9606}, \emph{slim=False}, \emph{names\_only=False}, \emph{aspects=('C'}, \emph{'F'}, \emph{'P')}}{}
Deprecated, should be removed soon.

Loads GO terms and annotations from QuickGO.
Returns 2 dicts: \sphinxtitleref{names} are GO terms by their IDs,
\sphinxtitleref{terms} are proteins GO IDs by UniProt IDs.

\end{fulllineitems}

\index{get\_graphviz\_attrs() (in module pypath.inputs.main)@\spxentry{get\_graphviz\_attrs()}\spxextra{in module pypath.inputs.main}}

\begin{fulllineitems}
\phantomsection\label{\detokenize{reference:pypath.inputs.main.get_graphviz_attrs}}\pysiglinewithargsret{\sphinxcode{\sphinxupquote{pypath.inputs.main.}}\sphinxbfcode{\sphinxupquote{get\_graphviz\_attrs}}}{}{}
Downloads graphviz attribute list from graphviz.org.
Returns 3 dicts of dicts: graph\_attrs, vertex\_attrs and edge\_attrs.

\end{fulllineitems}

\index{get\_guide2pharma() (in module pypath.inputs.main)@\spxentry{get\_guide2pharma()}\spxextra{in module pypath.inputs.main}}

\begin{fulllineitems}
\phantomsection\label{\detokenize{reference:pypath.inputs.main.get_guide2pharma}}\pysiglinewithargsret{\sphinxcode{\sphinxupquote{pypath.inputs.main.}}\sphinxbfcode{\sphinxupquote{get\_guide2pharma}}}{\emph{organism='human'}, \emph{endogenous=True}, \emph{process\_interactions=True}, \emph{process\_complexes=True}}{}
Downloads and processes Guide to Pharmacology data.
Returns list of dicts.
\begin{description}
\item[{@organism}] \leavevmode{[}str{]}
Name of the organism, e.g. \sphinxtitleref{human}.

\item[{@endogenous}] \leavevmode{[}bool{]}
Whether to include only endogenous ligands interactions.

\end{description}

\end{fulllineitems}

\index{get\_havugimana() (in module pypath.inputs.main)@\spxentry{get\_havugimana()}\spxextra{in module pypath.inputs.main}}

\begin{fulllineitems}
\phantomsection\label{\detokenize{reference:pypath.inputs.main.get_havugimana}}\pysiglinewithargsret{\sphinxcode{\sphinxupquote{pypath.inputs.main.}}\sphinxbfcode{\sphinxupquote{get\_havugimana}}}{}{}
Downloads data from
Supplement Table S3/1 from Havugimana 2012
Cell. 150(5): 1068\textendash{}1081.

\end{fulllineitems}

\index{get\_hpmr() (in module pypath.inputs.main)@\spxentry{get\_hpmr()}\spxextra{in module pypath.inputs.main}}

\begin{fulllineitems}
\phantomsection\label{\detokenize{reference:pypath.inputs.main.get_hpmr}}\pysiglinewithargsret{\sphinxcode{\sphinxupquote{pypath.inputs.main.}}\sphinxbfcode{\sphinxupquote{get\_hpmr}}}{\emph{use\_cache=None}}{}
Downloads ligand-receptor and receptor-receptor interactions from the
Human Plasma Membrane Receptome database.

\end{fulllineitems}

\index{get\_hpmr\_old() (in module pypath.inputs.main)@\spxentry{get\_hpmr\_old()}\spxextra{in module pypath.inputs.main}}

\begin{fulllineitems}
\phantomsection\label{\detokenize{reference:pypath.inputs.main.get_hpmr_old}}\pysiglinewithargsret{\sphinxcode{\sphinxupquote{pypath.inputs.main.}}\sphinxbfcode{\sphinxupquote{get\_hpmr\_old}}}{}{}
Deprecated, should be removed soon.

Downloads and processes the list of all human receptors from
human receptor census (HPMR \textendash{} Human Plasma Membrane Receptome).
Returns list of GeneSymbols.

\end{fulllineitems}

\index{get\_hsn() (in module pypath.inputs.main)@\spxentry{get\_hsn()}\spxextra{in module pypath.inputs.main}}

\begin{fulllineitems}
\phantomsection\label{\detokenize{reference:pypath.inputs.main.get_hsn}}\pysiglinewithargsret{\sphinxcode{\sphinxupquote{pypath.inputs.main.}}\sphinxbfcode{\sphinxupquote{get\_hsn}}}{}{}
Downloads and processes HumanSignalingNetwork version 6
(published 2014 Jan by Edwin Wang).
Returns list of interactions.

\end{fulllineitems}

\index{get\_i3d() (in module pypath.inputs.main)@\spxentry{get\_i3d()}\spxextra{in module pypath.inputs.main}}

\begin{fulllineitems}
\phantomsection\label{\detokenize{reference:pypath.inputs.main.get_i3d}}\pysiglinewithargsret{\sphinxcode{\sphinxupquote{pypath.inputs.main.}}\sphinxbfcode{\sphinxupquote{get\_i3d}}}{}{}
Interaction3D contains residue numbers in given chains in
given PDB stuctures, so we need to add an offset to get the residue
numbers valid for UniProt sequences. Offsets can be obtained from
Instruct, or from the Pfam PDB-chain-UniProt mapping table.

\end{fulllineitems}

\index{get\_ielm() (in module pypath.inputs.main)@\spxentry{get\_ielm()}\spxextra{in module pypath.inputs.main}}

\begin{fulllineitems}
\phantomsection\label{\detokenize{reference:pypath.inputs.main.get_ielm}}\pysiglinewithargsret{\sphinxcode{\sphinxupquote{pypath.inputs.main.}}\sphinxbfcode{\sphinxupquote{get\_ielm}}}{\emph{ppi}, \emph{id\_type='UniProtKB\_AC'}, \emph{mydomains='HMMS'}, \emph{maxwait=180}, \emph{cache=True}, \emph{part=False}, \emph{part\_size=500}, \emph{headers=None}}{}
Performs one query to iELM. Parameters are the same as at get\_ielm\_huge().

\end{fulllineitems}

\index{get\_ielm\_huge() (in module pypath.inputs.main)@\spxentry{get\_ielm\_huge()}\spxextra{in module pypath.inputs.main}}

\begin{fulllineitems}
\phantomsection\label{\detokenize{reference:pypath.inputs.main.get_ielm_huge}}\pysiglinewithargsret{\sphinxcode{\sphinxupquote{pypath.inputs.main.}}\sphinxbfcode{\sphinxupquote{get\_ielm\_huge}}}{\emph{ppi}, \emph{id\_type='UniProtKB\_AC'}, \emph{mydomains='HMMS'}, \emph{maxwait=180}, \emph{cache=True}, \emph{part\_size=500}, \emph{headers=None}}{}
Loads iELM predicted domain-motif interaction data for a set of
protein-protein interactions. This method breaks the list into
reasonable sized chunks and performs multiple requests to iELM,
and also retries in case of failure, with reducing the request
size. Provides feedback on the console.
\begin{quote}\begin{description}
\item[{Parameters}] \leavevmode\begin{itemize}
\item {} 
\sphinxstyleliteralstrong{\sphinxupquote{id\_type}} (\sphinxstyleliteralemphasis{\sphinxupquote{str}}) \textendash{} The type of the IDs in the supplied interaction list.
Default is ‘UniProtKB\_AC’.
Please refer to iELM what type of IDs it does understand.

\item {} 
\sphinxstyleliteralstrong{\sphinxupquote{mydomains}} (\sphinxstyleliteralemphasis{\sphinxupquote{str}}) \textendash{} The type of the domain detection method.
Defaults to ‘HMMS’.
Please refer to iELM for alternatives.

\item {} 
\sphinxstyleliteralstrong{\sphinxupquote{maxwait}} (\sphinxstyleliteralemphasis{\sphinxupquote{int}}) \textendash{} The limit of the waiting time in seconds.

\item {} 
\sphinxstyleliteralstrong{\sphinxupquote{cache}} (\sphinxstyleliteralemphasis{\sphinxupquote{bool}}) \textendash{} Whether to use the cache or download everything again.

\item {} 
\sphinxstyleliteralstrong{\sphinxupquote{part\_size}} (\sphinxstyleliteralemphasis{\sphinxupquote{int}}) \textendash{} The number of interactions to be queried in one request.

\item {} 
\sphinxstyleliteralstrong{\sphinxupquote{headers}} (\sphinxstyleliteralemphasis{\sphinxupquote{list}}) \textendash{} Additional HTTP headers to send to iELM with each request.

\end{itemize}

\end{description}\end{quote}

\end{fulllineitems}

\index{get\_instruct() (in module pypath.inputs.main)@\spxentry{get\_instruct()}\spxextra{in module pypath.inputs.main}}

\begin{fulllineitems}
\phantomsection\label{\detokenize{reference:pypath.inputs.main.get_instruct}}\pysiglinewithargsret{\sphinxcode{\sphinxupquote{pypath.inputs.main.}}\sphinxbfcode{\sphinxupquote{get\_instruct}}}{}{}
Instruct contains residue numbers in UniProt sequences, it means
no further calculations of offsets in chains of PDB structures needed.
Chains are not given, only a set of PDB structures supporting the
domain-domain // protein-protein interaction.

\end{fulllineitems}

\index{get\_instruct\_offsets() (in module pypath.inputs.main)@\spxentry{get\_instruct\_offsets()}\spxextra{in module pypath.inputs.main}}

\begin{fulllineitems}
\phantomsection\label{\detokenize{reference:pypath.inputs.main.get_instruct_offsets}}\pysiglinewithargsret{\sphinxcode{\sphinxupquote{pypath.inputs.main.}}\sphinxbfcode{\sphinxupquote{get\_instruct\_offsets}}}{}{}
These offsets should be understood as from UniProt to PDB.

\end{fulllineitems}

\index{get\_integrins() (in module pypath.inputs.main)@\spxentry{get\_integrins()}\spxextra{in module pypath.inputs.main}}

\begin{fulllineitems}
\phantomsection\label{\detokenize{reference:pypath.inputs.main.get_integrins}}\pysiglinewithargsret{\sphinxcode{\sphinxupquote{pypath.inputs.main.}}\sphinxbfcode{\sphinxupquote{get\_integrins}}}{}{}
Returns a set of the UniProt IDs of the human integrins from
Table 1 of Takada et al 2007 (10.1186/gb-2007-8-5-215).

\end{fulllineitems}

\index{get\_laudanna\_directions() (in module pypath.inputs.main)@\spxentry{get\_laudanna\_directions()}\spxextra{in module pypath.inputs.main}}

\begin{fulllineitems}
\phantomsection\label{\detokenize{reference:pypath.inputs.main.get_laudanna_directions}}\pysiglinewithargsret{\sphinxcode{\sphinxupquote{pypath.inputs.main.}}\sphinxbfcode{\sphinxupquote{get\_laudanna\_directions}}}{}{}
Downloads and processes the SignalingFlow edge attributes
from Laudanna Lab.
Returns list of directions.

\end{fulllineitems}

\index{get\_laudanna\_effects() (in module pypath.inputs.main)@\spxentry{get\_laudanna\_effects()}\spxextra{in module pypath.inputs.main}}

\begin{fulllineitems}
\phantomsection\label{\detokenize{reference:pypath.inputs.main.get_laudanna_effects}}\pysiglinewithargsret{\sphinxcode{\sphinxupquote{pypath.inputs.main.}}\sphinxbfcode{\sphinxupquote{get\_laudanna\_effects}}}{}{}
Downloads and processes the SignalingDirection edge attributes
from Laudanna Lab.
Returns list of effects.

\end{fulllineitems}

\index{get\_listof\_ontologies() (in module pypath.inputs.main)@\spxentry{get\_listof\_ontologies()}\spxextra{in module pypath.inputs.main}}

\begin{fulllineitems}
\phantomsection\label{\detokenize{reference:pypath.inputs.main.get_listof_ontologies}}\pysiglinewithargsret{\sphinxcode{\sphinxupquote{pypath.inputs.main.}}\sphinxbfcode{\sphinxupquote{get\_listof\_ontologies}}}{}{}
Returns a list of available ontologies using the bioservices module.

\end{fulllineitems}

\index{get\_ontology() (in module pypath.inputs.main)@\spxentry{get\_ontology()}\spxextra{in module pypath.inputs.main}}

\begin{fulllineitems}
\phantomsection\label{\detokenize{reference:pypath.inputs.main.get_ontology}}\pysiglinewithargsret{\sphinxcode{\sphinxupquote{pypath.inputs.main.}}\sphinxbfcode{\sphinxupquote{get\_ontology}}}{\emph{ontology}}{}
Downloads an ontology using the bioservices module.

\end{fulllineitems}

\index{get\_pepcyber() (in module pypath.inputs.main)@\spxentry{get\_pepcyber()}\spxextra{in module pypath.inputs.main}}

\begin{fulllineitems}
\phantomsection\label{\detokenize{reference:pypath.inputs.main.get_pepcyber}}\pysiglinewithargsret{\sphinxcode{\sphinxupquote{pypath.inputs.main.}}\sphinxbfcode{\sphinxupquote{get\_pepcyber}}}{\emph{cache=None}}{}
Downloads phosphoprotein binding protein interactions
from the PEPCyber database.

\end{fulllineitems}

\index{get\_pmid() (in module pypath.inputs.main)@\spxentry{get\_pmid()}\spxextra{in module pypath.inputs.main}}

\begin{fulllineitems}
\phantomsection\label{\detokenize{reference:pypath.inputs.main.get_pmid}}\pysiglinewithargsret{\sphinxcode{\sphinxupquote{pypath.inputs.main.}}\sphinxbfcode{\sphinxupquote{get\_pmid}}}{\emph{idList}}{}
For a list of doi or PMC IDs
fetches the corresponding PMIDs.

\end{fulllineitems}

\index{get\_switches\_elm() (in module pypath.inputs.main)@\spxentry{get\_switches\_elm()}\spxextra{in module pypath.inputs.main}}

\begin{fulllineitems}
\phantomsection\label{\detokenize{reference:pypath.inputs.main.get_switches_elm}}\pysiglinewithargsret{\sphinxcode{\sphinxupquote{pypath.inputs.main.}}\sphinxbfcode{\sphinxupquote{get\_switches\_elm}}}{}{}
switches.elm is a resource containing functional switches in molecular regulation,
in domain-motif level resolution, classified into categories according to their
mechanism.

\end{fulllineitems}

\index{get\_tfcensus() (in module pypath.inputs.main)@\spxentry{get\_tfcensus()}\spxextra{in module pypath.inputs.main}}

\begin{fulllineitems}
\phantomsection\label{\detokenize{reference:pypath.inputs.main.get_tfcensus}}\pysiglinewithargsret{\sphinxcode{\sphinxupquote{pypath.inputs.main.}}\sphinxbfcode{\sphinxupquote{get\_tfcensus}}}{\emph{classes=('a'}, \emph{'b'}, \emph{'other')}}{}
Downloads and processes the list of all known transcription factors from
TF census (Vaquerizas 2009). This resource is human only.
Returns set of UniProt IDs.

\end{fulllineitems}

\index{get\_tfregulons() (in module pypath.inputs.main)@\spxentry{get\_tfregulons()}\spxextra{in module pypath.inputs.main}}

\begin{fulllineitems}
\phantomsection\label{\detokenize{reference:pypath.inputs.main.get_tfregulons}}\pysiglinewithargsret{\sphinxcode{\sphinxupquote{pypath.inputs.main.}}\sphinxbfcode{\sphinxupquote{get\_tfregulons}}}{\emph{levels=\{'A'}, \emph{'B'\}}, \emph{only\_curated=False}}{}
Retrieves TF-target interactions from TF regulons.
\begin{quote}\begin{description}
\item[{Parameters}] \leavevmode\begin{itemize}
\item {} 
\sphinxstyleliteralstrong{\sphinxupquote{levels}} (\sphinxstyleliteralemphasis{\sphinxupquote{set}}) \textendash{} Confidence levels to be used.

\item {} 
\sphinxstyleliteralstrong{\sphinxupquote{only\_curated}} (\sphinxstyleliteralemphasis{\sphinxupquote{bool}}) \textendash{} Retrieve only literature curated interactions.

\end{itemize}

\end{description}\end{quote}

TF regulons is a comprehensive resource of TF-target interactions
combining multiple lines of evidences: literature curated databases,
ChIP-Seq data, PWM based prediction using HOCOMOCO and JASPAR matrices
and prediction from GTEx expression data by ARACNe.

For details see \sphinxurl{https://github.com/saezlab/DoRothEA}.

\end{fulllineitems}

\index{get\_tfregulons\_old() (in module pypath.inputs.main)@\spxentry{get\_tfregulons\_old()}\spxextra{in module pypath.inputs.main}}

\begin{fulllineitems}
\phantomsection\label{\detokenize{reference:pypath.inputs.main.get_tfregulons_old}}\pysiglinewithargsret{\sphinxcode{\sphinxupquote{pypath.inputs.main.}}\sphinxbfcode{\sphinxupquote{get\_tfregulons\_old}}}{\emph{levels=\{'A'}, \emph{'B'\}}, \emph{only\_curated=False}}{}
Retrieves TF-target interactions from TF regulons.
\begin{quote}\begin{description}
\item[{Parameters}] \leavevmode\begin{itemize}
\item {} 
\sphinxstyleliteralstrong{\sphinxupquote{levels}} (\sphinxstyleliteralemphasis{\sphinxupquote{set}}) \textendash{} Confidence levels to be used.

\item {} 
\sphinxstyleliteralstrong{\sphinxupquote{only\_curated}} (\sphinxstyleliteralemphasis{\sphinxupquote{bool}}) \textendash{} Retrieve only literature curated interactions.

\end{itemize}

\end{description}\end{quote}

TF regulons is a comprehensive resource of TF-target interactions
combining multiple lines of evidences: literature curated databases,
ChIP-Seq data, PWM based prediction using HOCOMOCO and JASPAR matrices
and prediction from GTEx expression data by ARACNe.

For details see \sphinxurl{https://github.com/saezlab/DoRothEA}.

\end{fulllineitems}

\index{get\_vesiclepedia() (in module pypath.inputs.main)@\spxentry{get\_vesiclepedia()}\spxextra{in module pypath.inputs.main}}

\begin{fulllineitems}
\phantomsection\label{\detokenize{reference:pypath.inputs.main.get_vesiclepedia}}\pysiglinewithargsret{\sphinxcode{\sphinxupquote{pypath.inputs.main.}}\sphinxbfcode{\sphinxupquote{get\_vesiclepedia}}}{\emph{organism=9606}, \emph{types=None}}{}~\begin{quote}\begin{description}
\item[{Parameters}] \leavevmode
\sphinxstyleliteralstrong{\sphinxupquote{types}} (\sphinxstyleliteralemphasis{\sphinxupquote{set}}) \textendash{} Molecule types to retrieve. Possible values: \sphinxtitleref{protein}, \sphinxtitleref{mrna}.

\end{description}\end{quote}

\end{fulllineitems}

\index{go\_ancestors() (in module pypath.inputs.main)@\spxentry{go\_ancestors()}\spxextra{in module pypath.inputs.main}}

\begin{fulllineitems}
\phantomsection\label{\detokenize{reference:pypath.inputs.main.go_ancestors}}\pysiglinewithargsret{\sphinxcode{\sphinxupquote{pypath.inputs.main.}}\sphinxbfcode{\sphinxupquote{go\_ancestors}}}{\emph{aspects=('C'}, \emph{'F'}, \emph{'P')}}{}
Queries the ancestors of GO terms by QuickGO REST API.

Returns dict of sets where keys are GO accessions and values are sets
of their ancestors.
\begin{quote}\begin{description}
\item[{Parameters}] \leavevmode
\sphinxstyleliteralstrong{\sphinxupquote{aspects}} (\sphinxstyleliteralemphasis{\sphinxupquote{tuple}}) \textendash{} GO aspects: \sphinxtitleref{C}, \sphinxtitleref{F} and \sphinxtitleref{P} for cellular\_component,
molecular\_function and biological\_process, respectively.

\end{description}\end{quote}

\end{fulllineitems}

\index{go\_ancestors\_goose() (in module pypath.inputs.main)@\spxentry{go\_ancestors\_goose()}\spxextra{in module pypath.inputs.main}}

\begin{fulllineitems}
\phantomsection\label{\detokenize{reference:pypath.inputs.main.go_ancestors_goose}}\pysiglinewithargsret{\sphinxcode{\sphinxupquote{pypath.inputs.main.}}\sphinxbfcode{\sphinxupquote{go\_ancestors\_goose}}}{\emph{aspects=('C'}, \emph{'F'}, \emph{'P')}}{}
Queries the ancestors of GO terms by AmiGO goose.

Returns dict of sets where keys are GO accessions and values are sets
of their ancestors.
\begin{quote}\begin{description}
\item[{Parameters}] \leavevmode
\sphinxstyleliteralstrong{\sphinxupquote{aspects}} (\sphinxstyleliteralemphasis{\sphinxupquote{tuple}}) \textendash{} GO aspects: \sphinxtitleref{C}, \sphinxtitleref{F} and \sphinxtitleref{P} for cellular\_component,
molecular\_function and biological\_process, respectively.

\end{description}\end{quote}

\end{fulllineitems}

\index{go\_ancestors\_quickgo() (in module pypath.inputs.main)@\spxentry{go\_ancestors\_quickgo()}\spxextra{in module pypath.inputs.main}}

\begin{fulllineitems}
\phantomsection\label{\detokenize{reference:pypath.inputs.main.go_ancestors_quickgo}}\pysiglinewithargsret{\sphinxcode{\sphinxupquote{pypath.inputs.main.}}\sphinxbfcode{\sphinxupquote{go\_ancestors\_quickgo}}}{\emph{aspects=('C'}, \emph{'F'}, \emph{'P')}}{}
Queries the ancestors of GO terms by QuickGO REST API.

Returns dict of sets where keys are GO accessions and values are sets
of their ancestors.
\begin{quote}\begin{description}
\item[{Parameters}] \leavevmode
\sphinxstyleliteralstrong{\sphinxupquote{aspects}} (\sphinxstyleliteralemphasis{\sphinxupquote{tuple}}) \textendash{} GO aspects: \sphinxtitleref{C}, \sphinxtitleref{F} and \sphinxtitleref{P} for cellular\_component,
molecular\_function and biological\_process, respectively.

\end{description}\end{quote}

\end{fulllineitems}

\index{go\_annotations() (in module pypath.inputs.main)@\spxentry{go\_annotations()}\spxextra{in module pypath.inputs.main}}

\begin{fulllineitems}
\phantomsection\label{\detokenize{reference:pypath.inputs.main.go_annotations}}\pysiglinewithargsret{\sphinxcode{\sphinxupquote{pypath.inputs.main.}}\sphinxbfcode{\sphinxupquote{go\_annotations}}}{\emph{organism='human'}}{}
Downloads GO annotation from UniProt GOA.

\end{fulllineitems}

\index{go\_annotations\_goa() (in module pypath.inputs.main)@\spxentry{go\_annotations\_goa()}\spxextra{in module pypath.inputs.main}}

\begin{fulllineitems}
\phantomsection\label{\detokenize{reference:pypath.inputs.main.go_annotations_goa}}\pysiglinewithargsret{\sphinxcode{\sphinxupquote{pypath.inputs.main.}}\sphinxbfcode{\sphinxupquote{go\_annotations\_goa}}}{\emph{organism='human'}}{}
Downloads GO annotation from UniProt GOA.

\end{fulllineitems}

\index{go\_annotations\_goose() (in module pypath.inputs.main)@\spxentry{go\_annotations\_goose()}\spxextra{in module pypath.inputs.main}}

\begin{fulllineitems}
\phantomsection\label{\detokenize{reference:pypath.inputs.main.go_annotations_goose}}\pysiglinewithargsret{\sphinxcode{\sphinxupquote{pypath.inputs.main.}}\sphinxbfcode{\sphinxupquote{go\_annotations\_goose}}}{\emph{organism=9606}, \emph{aspects=('C'}, \emph{'F'}, \emph{'P')}, \emph{uniprots=None}}{}
Queries GO annotations by AmiGO goose.

IMPORTANT:
This is not the preferred method any more to get terms and annotations.
Recently the preferred method to access GO annotations is
\sphinxcode{\sphinxupquote{pypath.dataio.go\_annotations\_solr()}}.
The data in GO MySQL instances has not been updated since Dec 2016.
Unfortunately the providers ceased to support MySQL, the most flexible
and highest performance access to GO data. The replacement is Solr
which is far from providing the same features as MySQL.

Returns terms in dict of dicts and annotations in dict of dicts of sets.
In both dicts the keys are aspects by their one letter codes.
In the term dicts keys are GO accessions and values are their names.
In the annotation dicts keys are UniProt IDs and values are sets
of GO accessions.
\begin{quote}\begin{description}
\item[{Parameters}] \leavevmode\begin{itemize}
\item {} 
\sphinxstyleliteralstrong{\sphinxupquote{organism}} (\sphinxstyleliteralemphasis{\sphinxupquote{int}}) \textendash{} NCBI Taxonomy ID of one organism. Default is human (9606).

\item {} 
\sphinxstyleliteralstrong{\sphinxupquote{aspects}} (\sphinxstyleliteralemphasis{\sphinxupquote{tuple}}) \textendash{} GO aspects: \sphinxtitleref{C}, \sphinxtitleref{F} and \sphinxtitleref{P} for cellular\_component,
molecular\_function and biological\_process, respectively.

\item {} 
\sphinxstyleliteralstrong{\sphinxupquote{uniprots}} (\sphinxstyleliteralemphasis{\sphinxupquote{list}}) \textendash{} Optionally a list of UniProt IDs. If \sphinxtitleref{None}, results for all proteins
returned.

\end{itemize}

\end{description}\end{quote}

\end{fulllineitems}

\index{go\_annotations\_quickgo() (in module pypath.inputs.main)@\spxentry{go\_annotations\_quickgo()}\spxextra{in module pypath.inputs.main}}

\begin{fulllineitems}
\phantomsection\label{\detokenize{reference:pypath.inputs.main.go_annotations_quickgo}}\pysiglinewithargsret{\sphinxcode{\sphinxupquote{pypath.inputs.main.}}\sphinxbfcode{\sphinxupquote{go\_annotations\_quickgo}}}{\emph{organism=9606}, \emph{aspects=('C'}, \emph{'F'}, \emph{'P')}, \emph{relations=('is\_a'}, \emph{'part\_of')}}{}
Queries GO annotations by QuickGO REST API.

IMPORTANT:
Recently the preferred method to access GO annotations is
\sphinxcode{\sphinxupquote{pypath.dataio.go\_annotations\_goa()}}.
Contrary to its name QuickGO is super slow, otherwise it should yield
up to date data, identical to the GOA file.

Returns terms in dict of dicts and annotations in dict of dicts of sets.
In both dicts the keys are aspects by their one letter codes.
In the term dicts keys are GO accessions and values are their names.
In the annotation dicts keys are UniProt IDs and values are sets
of GO accessions.
\begin{quote}\begin{description}
\item[{Parameters}] \leavevmode\begin{itemize}
\item {} 
\sphinxstyleliteralstrong{\sphinxupquote{organism}} (\sphinxstyleliteralemphasis{\sphinxupquote{int}}) \textendash{} NCBI Taxonomy ID of one organism. Default is human (9606).

\item {} 
\sphinxstyleliteralstrong{\sphinxupquote{aspects}} (\sphinxstyleliteralemphasis{\sphinxupquote{tuple}}) \textendash{} GO aspects: \sphinxtitleref{C}, \sphinxtitleref{F} and \sphinxtitleref{P} for cellular\_component,
molecular\_function and biological\_process, respectively.

\item {} 
\sphinxstyleliteralstrong{\sphinxupquote{uniprots}} (\sphinxstyleliteralemphasis{\sphinxupquote{list}}) \textendash{} Optionally a list of UniProt IDs. If \sphinxtitleref{None}, results for all proteins
returned.

\end{itemize}

\end{description}\end{quote}

\end{fulllineitems}

\index{go\_annotations\_solr() (in module pypath.inputs.main)@\spxentry{go\_annotations\_solr()}\spxextra{in module pypath.inputs.main}}

\begin{fulllineitems}
\phantomsection\label{\detokenize{reference:pypath.inputs.main.go_annotations_solr}}\pysiglinewithargsret{\sphinxcode{\sphinxupquote{pypath.inputs.main.}}\sphinxbfcode{\sphinxupquote{go\_annotations\_solr}}}{\emph{organism=9606}, \emph{aspects=('C'}, \emph{'F'}, \emph{'P')}, \emph{references=False}}{}
Queries GO annotations by AmiGO Solr.

Before other methods have been provided to access GO.
Now this is the preferred method to get annotations.
Returns terms in dict of dicts and annotations in dict of dicts of sets.
In both dicts the keys are aspects by their one letter codes.
In the term dicts keys are GO accessions and values are their names.
In the annotation dicts keys are UniProt IDs and values are sets
of GO accessions.
\begin{quote}\begin{description}
\item[{Parameters}] \leavevmode\begin{itemize}
\item {} 
\sphinxstyleliteralstrong{\sphinxupquote{organism}} (\sphinxstyleliteralemphasis{\sphinxupquote{int}}) \textendash{} NCBI Taxonomy ID of one organism. Default is human (9606).

\item {} 
\sphinxstyleliteralstrong{\sphinxupquote{aspects}} (\sphinxstyleliteralemphasis{\sphinxupquote{tuple}}) \textendash{} GO aspects: \sphinxtitleref{C}, \sphinxtitleref{F} and \sphinxtitleref{P} for cellular\_component,
molecular\_function and biological\_process, respectively.

\item {} 
\sphinxstyleliteralstrong{\sphinxupquote{references}} (\sphinxstyleliteralemphasis{\sphinxupquote{bool}}) \textendash{} Retrieve the references (PubMed IDs) for the annotations.
Currently not implemented.

\end{itemize}

\end{description}\end{quote}

\end{fulllineitems}

\index{go\_annotations\_uniprot() (in module pypath.inputs.main)@\spxentry{go\_annotations\_uniprot()}\spxextra{in module pypath.inputs.main}}

\begin{fulllineitems}
\phantomsection\label{\detokenize{reference:pypath.inputs.main.go_annotations_uniprot}}\pysiglinewithargsret{\sphinxcode{\sphinxupquote{pypath.inputs.main.}}\sphinxbfcode{\sphinxupquote{go\_annotations\_uniprot}}}{\emph{organism=9606}, \emph{swissprot='yes'}}{}
Deprecated, should be removed soon.

\end{fulllineitems}

\index{go\_descendants() (in module pypath.inputs.main)@\spxentry{go\_descendants()}\spxextra{in module pypath.inputs.main}}

\begin{fulllineitems}
\phantomsection\label{\detokenize{reference:pypath.inputs.main.go_descendants}}\pysiglinewithargsret{\sphinxcode{\sphinxupquote{pypath.inputs.main.}}\sphinxbfcode{\sphinxupquote{go\_descendants}}}{\emph{aspects=('C'}, \emph{'F'}, \emph{'P')}, \emph{terms=None}, \emph{relations=None}, \emph{quickgo\_download\_size=500}}{}
Queries descendants of GO terms by QuickGO REST API.

Returns dict of sets where keys are GO accessions and values are sets
of their descendants.
\begin{quote}\begin{description}
\item[{Parameters}] \leavevmode\begin{itemize}
\item {} 
\sphinxstyleliteralstrong{\sphinxupquote{aspects}} (\sphinxstyleliteralemphasis{\sphinxupquote{tuple}}) \textendash{} GO aspects: \sphinxtitleref{C}, \sphinxtitleref{F} and \sphinxtitleref{P} for cellular\_component,
molecular\_function and biological\_process, respectively.

\item {} 
\sphinxstyleliteralstrong{\sphinxupquote{terms}} (\sphinxstyleliteralemphasis{\sphinxupquote{dict}}) \textendash{} Result from \sphinxcode{\sphinxupquote{go\_terms\_solr}}. If \sphinxcode{\sphinxupquote{None}} the method will be called.

\end{itemize}

\end{description}\end{quote}

\end{fulllineitems}

\index{go\_descendants\_goose() (in module pypath.inputs.main)@\spxentry{go\_descendants\_goose()}\spxextra{in module pypath.inputs.main}}

\begin{fulllineitems}
\phantomsection\label{\detokenize{reference:pypath.inputs.main.go_descendants_goose}}\pysiglinewithargsret{\sphinxcode{\sphinxupquote{pypath.inputs.main.}}\sphinxbfcode{\sphinxupquote{go\_descendants\_goose}}}{\emph{aspects=('C'}, \emph{'F'}, \emph{'P')}}{}
Queries descendants of GO terms by AmiGO goose.

IMPORTANT:
This is not the preferred method any more to get descendants.
Recently the preferred method to access GO annotations is
\sphinxcode{\sphinxupquote{pypath.dataio.go\_descendants\_quickgo()}}.
The data in GO MySQL instances has not been updated since Dec 2016.
Unfortunately the providers ceased to support MySQL, the most flexible
and highest performance access to GO data. The replacement is Solr
which is far from providing the same features as MySQL, for example
it is unable to provide GO graph relationships. Other service is QuickGO
which is up to date and has nice ways to query the ontology.

Returns dict of sets where keys are GO accessions and values are sets
of their descendants.
\begin{quote}\begin{description}
\item[{Parameters}] \leavevmode
\sphinxstyleliteralstrong{\sphinxupquote{aspects}} (\sphinxstyleliteralemphasis{\sphinxupquote{tuple}}) \textendash{} GO aspects: \sphinxtitleref{C}, \sphinxtitleref{F} and \sphinxtitleref{P} for cellular\_component,
molecular\_function and biological\_process, respectively.

\end{description}\end{quote}

\end{fulllineitems}

\index{go\_descendants\_quickgo() (in module pypath.inputs.main)@\spxentry{go\_descendants\_quickgo()}\spxextra{in module pypath.inputs.main}}

\begin{fulllineitems}
\phantomsection\label{\detokenize{reference:pypath.inputs.main.go_descendants_quickgo}}\pysiglinewithargsret{\sphinxcode{\sphinxupquote{pypath.inputs.main.}}\sphinxbfcode{\sphinxupquote{go\_descendants\_quickgo}}}{\emph{aspects=('C'}, \emph{'F'}, \emph{'P')}, \emph{terms=None}, \emph{relations=None}, \emph{quickgo\_download\_size=500}}{}
Queries descendants of GO terms by QuickGO REST API.

Returns dict of sets where keys are GO accessions and values are sets
of their descendants.
\begin{quote}\begin{description}
\item[{Parameters}] \leavevmode\begin{itemize}
\item {} 
\sphinxstyleliteralstrong{\sphinxupquote{aspects}} (\sphinxstyleliteralemphasis{\sphinxupquote{tuple}}) \textendash{} GO aspects: \sphinxtitleref{C}, \sphinxtitleref{F} and \sphinxtitleref{P} for cellular\_component,
molecular\_function and biological\_process, respectively.

\item {} 
\sphinxstyleliteralstrong{\sphinxupquote{terms}} (\sphinxstyleliteralemphasis{\sphinxupquote{dict}}) \textendash{} Result from \sphinxcode{\sphinxupquote{go\_terms\_solr}}. If \sphinxcode{\sphinxupquote{None}} the method will be called.

\end{itemize}

\end{description}\end{quote}

\end{fulllineitems}

\index{go\_descendants\_to\_ancestors() (in module pypath.inputs.main)@\spxentry{go\_descendants\_to\_ancestors()}\spxextra{in module pypath.inputs.main}}

\begin{fulllineitems}
\phantomsection\label{\detokenize{reference:pypath.inputs.main.go_descendants_to_ancestors}}\pysiglinewithargsret{\sphinxcode{\sphinxupquote{pypath.inputs.main.}}\sphinxbfcode{\sphinxupquote{go\_descendants\_to\_ancestors}}}{\emph{desc}}{}
Turns a dict of descendants to dict of ancestors by swapping the
relationships. This way descendants will be the keys and their ancestors
will be the values.

\end{fulllineitems}

\index{go\_terms() (in module pypath.inputs.main)@\spxentry{go\_terms()}\spxextra{in module pypath.inputs.main}}

\begin{fulllineitems}
\phantomsection\label{\detokenize{reference:pypath.inputs.main.go_terms}}\pysiglinewithargsret{\sphinxcode{\sphinxupquote{pypath.inputs.main.}}\sphinxbfcode{\sphinxupquote{go\_terms}}}{\emph{aspects=('C'}, \emph{'F'}, \emph{'P')}}{}
Queries GO terms by the QuickGO REST API.

Return dict of dicts where upper level keys are one letter codes of the
aspects \sphinxtitleref{C}, \sphinxtitleref{F} and \sphinxtitleref{P} for cellular\_component, molecular\_function and
biological\_process, respectively. Lower level keys are GO accessions
and values are names of the terms.
\begin{quote}\begin{description}
\item[{Parameters}] \leavevmode
\sphinxstyleliteralstrong{\sphinxupquote{aspects}} (\sphinxstyleliteralemphasis{\sphinxupquote{tuple}}) \textendash{} GO aspects: \sphinxtitleref{C}, \sphinxtitleref{F} and \sphinxtitleref{P} for cellular\_component,
molecular\_function and biological\_process, respectively.

\end{description}\end{quote}

\end{fulllineitems}

\index{go\_terms\_goose() (in module pypath.inputs.main)@\spxentry{go\_terms\_goose()}\spxextra{in module pypath.inputs.main}}

\begin{fulllineitems}
\phantomsection\label{\detokenize{reference:pypath.inputs.main.go_terms_goose}}\pysiglinewithargsret{\sphinxcode{\sphinxupquote{pypath.inputs.main.}}\sphinxbfcode{\sphinxupquote{go\_terms\_goose}}}{\emph{aspects=('C'}, \emph{'F'}, \emph{'P')}}{}
Queries GO terms by AmiGO goose.

Return dict of dicts where upper level keys are one letter codes of the
aspects \sphinxtitleref{C}, \sphinxtitleref{F} and \sphinxtitleref{P} for cellular\_component, molecular\_function and
biological\_process, respectively. Lower level keys are GO accessions
and values are names of the terms.
\begin{quote}\begin{description}
\item[{Parameters}] \leavevmode
\sphinxstyleliteralstrong{\sphinxupquote{aspects}} (\sphinxstyleliteralemphasis{\sphinxupquote{tuple}}) \textendash{} GO aspects: \sphinxtitleref{C}, \sphinxtitleref{F} and \sphinxtitleref{P} for cellular\_component,
molecular\_function and biological\_process, respectively.

\end{description}\end{quote}

\end{fulllineitems}

\index{go\_terms\_quickgo() (in module pypath.inputs.main)@\spxentry{go\_terms\_quickgo()}\spxextra{in module pypath.inputs.main}}

\begin{fulllineitems}
\phantomsection\label{\detokenize{reference:pypath.inputs.main.go_terms_quickgo}}\pysiglinewithargsret{\sphinxcode{\sphinxupquote{pypath.inputs.main.}}\sphinxbfcode{\sphinxupquote{go\_terms\_quickgo}}}{\emph{aspects=('C'}, \emph{'F'}, \emph{'P')}}{}
Queries GO terms by the QuickGO REST API.

Return dict of dicts where upper level keys are one letter codes of the
aspects \sphinxtitleref{C}, \sphinxtitleref{F} and \sphinxtitleref{P} for cellular\_component, molecular\_function and
biological\_process, respectively. Lower level keys are GO accessions
and values are names of the terms.
\begin{quote}\begin{description}
\item[{Parameters}] \leavevmode
\sphinxstyleliteralstrong{\sphinxupquote{aspects}} (\sphinxstyleliteralemphasis{\sphinxupquote{tuple}}) \textendash{} GO aspects: \sphinxtitleref{C}, \sphinxtitleref{F} and \sphinxtitleref{P} for cellular\_component,
molecular\_function and biological\_process, respectively.

\end{description}\end{quote}

\end{fulllineitems}

\index{go\_terms\_solr() (in module pypath.inputs.main)@\spxentry{go\_terms\_solr()}\spxextra{in module pypath.inputs.main}}

\begin{fulllineitems}
\phantomsection\label{\detokenize{reference:pypath.inputs.main.go_terms_solr}}\pysiglinewithargsret{\sphinxcode{\sphinxupquote{pypath.inputs.main.}}\sphinxbfcode{\sphinxupquote{go\_terms\_solr}}}{\emph{aspects=('C'}, \emph{'F'}, \emph{'P')}}{}
Queries GO terms by AmiGO Solr.

Returns dict of dicts where upper level keys are one letter codes of the
aspects \sphinxtitleref{C}, \sphinxtitleref{F} and \sphinxtitleref{P} for cellular\_component, molecular\_function and
biological\_process, respectively. Lower level keys are GO accessions
and values are names of the terms.
\begin{quote}\begin{description}
\item[{Parameters}] \leavevmode
\sphinxstyleliteralstrong{\sphinxupquote{aspects}} (\sphinxstyleliteralemphasis{\sphinxupquote{tuple}}) \textendash{} GO aspects: \sphinxtitleref{C}, \sphinxtitleref{F} and \sphinxtitleref{P} for cellular\_component,
molecular\_function and biological\_process, respectively.

\end{description}\end{quote}

\end{fulllineitems}

\index{havugimana\_complexes() (in module pypath.inputs.main)@\spxentry{havugimana\_complexes()}\spxextra{in module pypath.inputs.main}}

\begin{fulllineitems}
\phantomsection\label{\detokenize{reference:pypath.inputs.main.havugimana_complexes}}\pysiglinewithargsret{\sphinxcode{\sphinxupquote{pypath.inputs.main.}}\sphinxbfcode{\sphinxupquote{havugimana\_complexes}}}{}{}
Retrieves complexes from
Supplement Table S3/1 from Havugimana 2012
Cell. 150(5): 1068\textendash{}1081.

\end{fulllineitems}

\index{hpmr\_interactions\_old() (in module pypath.inputs.main)@\spxentry{hpmr\_interactions\_old()}\spxextra{in module pypath.inputs.main}}

\begin{fulllineitems}
\phantomsection\label{\detokenize{reference:pypath.inputs.main.hpmr_interactions_old}}\pysiglinewithargsret{\sphinxcode{\sphinxupquote{pypath.inputs.main.}}\sphinxbfcode{\sphinxupquote{hpmr\_interactions\_old}}}{}{}
Deprecated, should be removed soon.

Downloads ligand-receptor and receptor-receptor interactions from the
Human Plasma Membrane Receptome database.

\end{fulllineitems}

\index{intogen\_annotations() (in module pypath.inputs.main)@\spxentry{intogen\_annotations()}\spxextra{in module pypath.inputs.main}}

\begin{fulllineitems}
\phantomsection\label{\detokenize{reference:pypath.inputs.main.intogen_annotations}}\pysiglinewithargsret{\sphinxcode{\sphinxupquote{pypath.inputs.main.}}\sphinxbfcode{\sphinxupquote{intogen\_annotations}}}{}{}
Returns a list of cancer driver genes according to the IntOGen database.

\end{fulllineitems}

\index{kegg\_interactions() (in module pypath.inputs.main)@\spxentry{kegg\_interactions()}\spxextra{in module pypath.inputs.main}}

\begin{fulllineitems}
\phantomsection\label{\detokenize{reference:pypath.inputs.main.kegg_interactions}}\pysiglinewithargsret{\sphinxcode{\sphinxupquote{pypath.inputs.main.}}\sphinxbfcode{\sphinxupquote{kegg\_interactions}}}{}{}
Downloads and processes KEGG Pathways.
Returns list of interactions.

\end{fulllineitems}

\index{kinasedotcom\_annotations() (in module pypath.inputs.main)@\spxentry{kinasedotcom\_annotations()}\spxextra{in module pypath.inputs.main}}

\begin{fulllineitems}
\phantomsection\label{\detokenize{reference:pypath.inputs.main.kinasedotcom_annotations}}\pysiglinewithargsret{\sphinxcode{\sphinxupquote{pypath.inputs.main.}}\sphinxbfcode{\sphinxupquote{kinasedotcom\_annotations}}}{}{}
Downloads and processes kinase annotations from kinase.com.

\end{fulllineitems}

\index{lmpid\_dmi() (in module pypath.inputs.main)@\spxentry{lmpid\_dmi()}\spxextra{in module pypath.inputs.main}}

\begin{fulllineitems}
\phantomsection\label{\detokenize{reference:pypath.inputs.main.lmpid_dmi}}\pysiglinewithargsret{\sphinxcode{\sphinxupquote{pypath.inputs.main.}}\sphinxbfcode{\sphinxupquote{lmpid\_dmi}}}{\emph{organism=9606}}{}
Converts list of domain-motif interactions supplied by
\sphinxtitleref{pypath.dataio.load\_lmpid()} to list of
{\color{red}\bfseries{}{}`}pypath.intera.DomainMotif() objects.

\end{fulllineitems}

\index{lmpid\_interactions() (in module pypath.inputs.main)@\spxentry{lmpid\_interactions()}\spxextra{in module pypath.inputs.main}}

\begin{fulllineitems}
\phantomsection\label{\detokenize{reference:pypath.inputs.main.lmpid_interactions}}\pysiglinewithargsret{\sphinxcode{\sphinxupquote{pypath.inputs.main.}}\sphinxbfcode{\sphinxupquote{lmpid\_interactions}}}{\emph{organism=9606}}{}
Converts list of domain-motif interactions supplied by
\sphinxtitleref{pypath.dataio.load\_lmpid()} to list of interactions.

\end{fulllineitems}

\index{load\_lmpid() (in module pypath.inputs.main)@\spxentry{load\_lmpid()}\spxextra{in module pypath.inputs.main}}

\begin{fulllineitems}
\phantomsection\label{\detokenize{reference:pypath.inputs.main.load_lmpid}}\pysiglinewithargsret{\sphinxcode{\sphinxupquote{pypath.inputs.main.}}\sphinxbfcode{\sphinxupquote{load\_lmpid}}}{\emph{organism=9606}}{}
Reads and processes LMPID data from local file
\sphinxtitleref{pypath.data/LMPID\_DATA\_pubmed\_ref.xml}.
The file was provided by LMPID authors and is now
redistributed with the module.
Returns list of domain-motif interactions.

\end{fulllineitems}

\index{load\_macrophage() (in module pypath.inputs.main)@\spxentry{load\_macrophage()}\spxextra{in module pypath.inputs.main}}

\begin{fulllineitems}
\phantomsection\label{\detokenize{reference:pypath.inputs.main.load_macrophage}}\pysiglinewithargsret{\sphinxcode{\sphinxupquote{pypath.inputs.main.}}\sphinxbfcode{\sphinxupquote{load\_macrophage}}}{}{}
Loads Macrophage from local file.
Returns list of interactions.

\end{fulllineitems}

\index{matrisome\_annotations() (in module pypath.inputs.main)@\spxentry{matrisome\_annotations()}\spxextra{in module pypath.inputs.main}}

\begin{fulllineitems}
\phantomsection\label{\detokenize{reference:pypath.inputs.main.matrisome_annotations}}\pysiglinewithargsret{\sphinxcode{\sphinxupquote{pypath.inputs.main.}}\sphinxbfcode{\sphinxupquote{matrisome\_annotations}}}{\emph{organism=9606}}{}
Downloads MatrisomeDB 2.0, a database of extracellular matrix proteins.
Returns dict where keys are UniProt IDs and values are tuples of
classes, subclasses and notes.

\end{fulllineitems}

\index{matrixdb\_ecm\_proteins() (in module pypath.inputs.main)@\spxentry{matrixdb\_ecm\_proteins()}\spxextra{in module pypath.inputs.main}}

\begin{fulllineitems}
\phantomsection\label{\detokenize{reference:pypath.inputs.main.matrixdb_ecm_proteins}}\pysiglinewithargsret{\sphinxcode{\sphinxupquote{pypath.inputs.main.}}\sphinxbfcode{\sphinxupquote{matrixdb\_ecm\_proteins}}}{\emph{organism=9606}}{}
Returns a set of ECM (extracellular matrix) protein UniProt IDs
retrieved from MatrixDB.

\end{fulllineitems}

\index{matrixdb\_membrane\_proteins() (in module pypath.inputs.main)@\spxentry{matrixdb\_membrane\_proteins()}\spxextra{in module pypath.inputs.main}}

\begin{fulllineitems}
\phantomsection\label{\detokenize{reference:pypath.inputs.main.matrixdb_membrane_proteins}}\pysiglinewithargsret{\sphinxcode{\sphinxupquote{pypath.inputs.main.}}\sphinxbfcode{\sphinxupquote{matrixdb\_membrane\_proteins}}}{\emph{organism=9606}}{}
Returns a set of membrane protein UniProt IDs retrieved from MatrixDB.

\end{fulllineitems}

\index{matrixdb\_secreted\_proteins() (in module pypath.inputs.main)@\spxentry{matrixdb\_secreted\_proteins()}\spxextra{in module pypath.inputs.main}}

\begin{fulllineitems}
\phantomsection\label{\detokenize{reference:pypath.inputs.main.matrixdb_secreted_proteins}}\pysiglinewithargsret{\sphinxcode{\sphinxupquote{pypath.inputs.main.}}\sphinxbfcode{\sphinxupquote{matrixdb\_secreted\_proteins}}}{\emph{organism=9606}}{}
Returns a set of secreted protein UniProt IDs retrieved from MatrixDB.

\end{fulllineitems}

\index{msigdb\_annotations() (in module pypath.inputs.main)@\spxentry{msigdb\_annotations()}\spxextra{in module pypath.inputs.main}}

\begin{fulllineitems}
\phantomsection\label{\detokenize{reference:pypath.inputs.main.msigdb_annotations}}\pysiglinewithargsret{\sphinxcode{\sphinxupquote{pypath.inputs.main.}}\sphinxbfcode{\sphinxupquote{msigdb\_annotations}}}{\emph{registered\_email=None}, \emph{only\_collections=None}, \emph{exclude=('c5'}, \emph{)}}{}
Downloads all or some MSigDB gene set collections and processes them
to an annotation type dictionary.
\begin{quote}\begin{description}
\item[{Parameters}] \leavevmode\begin{itemize}
\item {} 
\sphinxstyleliteralstrong{\sphinxupquote{registered\_email}} (\sphinxstyleliteralemphasis{\sphinxupquote{str}}\sphinxstyleliteralemphasis{\sphinxupquote{,}}\sphinxstyleliteralemphasis{\sphinxupquote{NoneType}}) \textendash{} An email address registered at MSigDB. If \sphinxtitleref{None} the \sphinxtitleref{msigdb\_email}
from \sphinxcode{\sphinxupquote{pypath.settings}} will be used.

\item {} 
\sphinxstyleliteralstrong{\sphinxupquote{only\_collections}} (\sphinxstyleliteralemphasis{\sphinxupquote{set}}\sphinxstyleliteralemphasis{\sphinxupquote{,}}\sphinxstyleliteralemphasis{\sphinxupquote{NoneType}}) \textendash{} Limit the annotations only to these collections. For available
collections e.g. \sphinxcode{\sphinxupquote{\{'h.all', 'c2cgp'\}}} refer to the MSigDB webpage:
\sphinxurl{http://software.broadinstitute.org/gsea/downloads.jsp\#msigdb}

\item {} 
\sphinxstyleliteralstrong{\sphinxupquote{exclude}} (\sphinxstyleliteralemphasis{\sphinxupquote{tuple}}) \textendash{} Exclude the collections having their name starting with any of the
strings in this tuple. By default \sphinxtitleref{c5} (Gene Ontology) is excluded.

\end{itemize}

\end{description}\end{quote}

\end{fulllineitems}

\index{msigdb\_download() (in module pypath.inputs.main)@\spxentry{msigdb\_download()}\spxextra{in module pypath.inputs.main}}

\begin{fulllineitems}
\phantomsection\label{\detokenize{reference:pypath.inputs.main.msigdb_download}}\pysiglinewithargsret{\sphinxcode{\sphinxupquote{pypath.inputs.main.}}\sphinxbfcode{\sphinxupquote{msigdb\_download}}}{\emph{registered\_email=None}, \emph{collection='msigdb'}, \emph{id\_type='symbols'}, \emph{force\_download=False}}{}
Downloads and preprocesses a collection of gmt format gene sets from
MSigDB. Returns dict of sets with gene set names as keys and molecular
identifiers as values.
\begin{quote}\begin{description}
\item[{Parameters}] \leavevmode\begin{itemize}
\item {} 
\sphinxstyleliteralstrong{\sphinxupquote{registered\_email}} (\sphinxstyleliteralemphasis{\sphinxupquote{str}}\sphinxstyleliteralemphasis{\sphinxupquote{,}}\sphinxstyleliteralemphasis{\sphinxupquote{NoneType}}) \textendash{} An email address registered at MSigDB. If \sphinxtitleref{None} the \sphinxtitleref{msigdb\_email}
from \sphinxcode{\sphinxupquote{pypath.settings}} will be used.

\item {} 
\sphinxstyleliteralstrong{\sphinxupquote{collection}} (\sphinxstyleliteralemphasis{\sphinxupquote{str}}) \textendash{} The name of the gene set collection. For available collections (e.g.
\sphinxtitleref{h.all} or \sphinxtitleref{c2.cpg}) refer to the MSigDB website:
\sphinxurl{http://software.broadinstitute.org/gsea/downloads.jsp\#msigdb}
The default value \sphinxtitleref{msigdb} contains all the genesets however you
won’t be able to distinguish which geneset comes from which
collection. For this you need to download the collections one by one.

\item {} 
\sphinxstyleliteralstrong{\sphinxupquote{id\_type}} (\sphinxstyleliteralemphasis{\sphinxupquote{str}}) \textendash{} MSigDB provides Gene Symbols (\sphinxtitleref{symbols}) and Entrez Gene IDs
(\sphinxtitleref{entrez}).

\item {} 
\sphinxstyleliteralstrong{\sphinxupquote{force\_download}} (\sphinxstyleliteralemphasis{\sphinxupquote{bool}}) \textendash{} Download even if cache content is available.

\end{itemize}

\end{description}\end{quote}

\end{fulllineitems}

\index{msigdb\_download\_collections() (in module pypath.inputs.main)@\spxentry{msigdb\_download\_collections()}\spxextra{in module pypath.inputs.main}}

\begin{fulllineitems}
\phantomsection\label{\detokenize{reference:pypath.inputs.main.msigdb_download_collections}}\pysiglinewithargsret{\sphinxcode{\sphinxupquote{pypath.inputs.main.}}\sphinxbfcode{\sphinxupquote{msigdb\_download\_collections}}}{\emph{registered\_email=None}, \emph{only\_collections=None}, \emph{exclude=('c5'}, \emph{)}, \emph{id\_type='symbols'}}{}
Downloads all or some MSigDB gene set collections.
Returns a dict of dicts where upper level keys are collections while
lower level keys are geneset names and values are molecular identifiers.
\begin{quote}\begin{description}
\item[{Parameters}] \leavevmode\begin{itemize}
\item {} 
\sphinxstyleliteralstrong{\sphinxupquote{registered\_email}} (\sphinxstyleliteralemphasis{\sphinxupquote{str}}\sphinxstyleliteralemphasis{\sphinxupquote{,}}\sphinxstyleliteralemphasis{\sphinxupquote{NoneType}}) \textendash{} An email address registered at MSigDB. If \sphinxtitleref{None} the \sphinxtitleref{msigdb\_email}
from \sphinxcode{\sphinxupquote{pypath.settings}} will be used.

\item {} 
\sphinxstyleliteralstrong{\sphinxupquote{only\_collections}} (\sphinxstyleliteralemphasis{\sphinxupquote{set}}\sphinxstyleliteralemphasis{\sphinxupquote{,}}\sphinxstyleliteralemphasis{\sphinxupquote{NoneType}}) \textendash{} Limit the annotations only to these collections. For available
collections e.g. \sphinxcode{\sphinxupquote{\{'h.all', 'c2cgp'\}}} refer to the MSigDB webpage:
\sphinxurl{http://software.broadinstitute.org/gsea/downloads.jsp\#msigdb}

\item {} 
\sphinxstyleliteralstrong{\sphinxupquote{exclude}} (\sphinxstyleliteralemphasis{\sphinxupquote{tuple}}) \textendash{} Exclude the collections having their name starting with any of the
strings in this tuple. By default \sphinxtitleref{c5} (Gene Ontology) is excluded.

\end{itemize}

\end{description}\end{quote}

\end{fulllineitems}

\index{only\_pmids() (in module pypath.inputs.main)@\spxentry{only\_pmids()}\spxextra{in module pypath.inputs.main}}

\begin{fulllineitems}
\phantomsection\label{\detokenize{reference:pypath.inputs.main.only_pmids}}\pysiglinewithargsret{\sphinxcode{\sphinxupquote{pypath.inputs.main.}}\sphinxbfcode{\sphinxupquote{only\_pmids}}}{\emph{idList}, \emph{strict=True}}{}
Return elements unchanged which comply with the PubMed ID format,
and attempts to translate the DOIs and PMC IDs using NCBI
E-utils.
Returns list containing only PMIDs.
\begin{description}
\item[{@idList}] \leavevmode{[}list, str{]}
List of IDs or one single ID.

\item[{@strict}] \leavevmode{[}bool{]}
Whether keep in the list those IDs which are not PMIDs,
neither DOIs or PMC IDs or NIH manuscript IDs.

\end{description}

\end{fulllineitems}

\index{open\_pubmed() (in module pypath.inputs.main)@\spxentry{open\_pubmed()}\spxextra{in module pypath.inputs.main}}

\begin{fulllineitems}
\phantomsection\label{\detokenize{reference:pypath.inputs.main.open_pubmed}}\pysiglinewithargsret{\sphinxcode{\sphinxupquote{pypath.inputs.main.}}\sphinxbfcode{\sphinxupquote{open\_pubmed}}}{\emph{pmid}}{}
Opens PubMed record in web browser.
\begin{description}
\item[{@pmid}] \leavevmode{[}str or int{]}
PubMed ID

\end{description}

\end{fulllineitems}

\index{pdzbase\_interactions() (in module pypath.inputs.main)@\spxentry{pdzbase\_interactions()}\spxextra{in module pypath.inputs.main}}

\begin{fulllineitems}
\phantomsection\label{\detokenize{reference:pypath.inputs.main.pdzbase_interactions}}\pysiglinewithargsret{\sphinxcode{\sphinxupquote{pypath.inputs.main.}}\sphinxbfcode{\sphinxupquote{pdzbase\_interactions}}}{}{}
Downloads data from PDZbase. Parses data from the HTML tables.

\end{fulllineitems}

\index{phosphatome\_annotations() (in module pypath.inputs.main)@\spxentry{phosphatome\_annotations()}\spxextra{in module pypath.inputs.main}}

\begin{fulllineitems}
\phantomsection\label{\detokenize{reference:pypath.inputs.main.phosphatome_annotations}}\pysiglinewithargsret{\sphinxcode{\sphinxupquote{pypath.inputs.main.}}\sphinxbfcode{\sphinxupquote{phosphatome\_annotations}}}{}{}
Downloads the list of phosphatases from Chen et al, Science Signaling
(2017) Table S1.

\end{fulllineitems}

\index{ramilowski\_interactions() (in module pypath.inputs.main)@\spxentry{ramilowski\_interactions()}\spxextra{in module pypath.inputs.main}}

\begin{fulllineitems}
\phantomsection\label{\detokenize{reference:pypath.inputs.main.ramilowski_interactions}}\pysiglinewithargsret{\sphinxcode{\sphinxupquote{pypath.inputs.main.}}\sphinxbfcode{\sphinxupquote{ramilowski\_interactions}}}{\emph{putative=False}}{}
Downloads and processes ligand-receptor interactions from
Supplementary Table 2 of Ramilowski 2015.

\end{fulllineitems}

\index{reactions\_biopax() (in module pypath.inputs.main)@\spxentry{reactions\_biopax()}\spxextra{in module pypath.inputs.main}}

\begin{fulllineitems}
\phantomsection\label{\detokenize{reference:pypath.inputs.main.reactions_biopax}}\pysiglinewithargsret{\sphinxcode{\sphinxupquote{pypath.inputs.main.}}\sphinxbfcode{\sphinxupquote{reactions\_biopax}}}{\emph{biopax\_file}, \emph{organism=9606}, \emph{protein\_name\_type='UniProt'}, \emph{clean=True}}{}
Processes a BioPAX file and extracts binary interactions.

\end{fulllineitems}

\index{reactome\_biopax() (in module pypath.inputs.main)@\spxentry{reactome\_biopax()}\spxextra{in module pypath.inputs.main}}

\begin{fulllineitems}
\phantomsection\label{\detokenize{reference:pypath.inputs.main.reactome_biopax}}\pysiglinewithargsret{\sphinxcode{\sphinxupquote{pypath.inputs.main.}}\sphinxbfcode{\sphinxupquote{reactome\_biopax}}}{\emph{organism=9606}, \emph{cache=True}}{}
Downloads Reactome human reactions in SBML format.
Returns File object.

\end{fulllineitems}

\index{reactome\_interactions() (in module pypath.inputs.main)@\spxentry{reactome\_interactions()}\spxextra{in module pypath.inputs.main}}

\begin{fulllineitems}
\phantomsection\label{\detokenize{reference:pypath.inputs.main.reactome_interactions}}\pysiglinewithargsret{\sphinxcode{\sphinxupquote{pypath.inputs.main.}}\sphinxbfcode{\sphinxupquote{reactome\_interactions}}}{\emph{cacheFile=None}, \emph{**kwargs}}{}
Downloads and processes Reactome BioPAX.
Extracts binary interactions.
The applied criteria are very stringent, yields very few interactions.
Requires large free memory, approx. 2G.

\end{fulllineitems}

\index{reactome\_sbml() (in module pypath.inputs.main)@\spxentry{reactome\_sbml()}\spxextra{in module pypath.inputs.main}}

\begin{fulllineitems}
\phantomsection\label{\detokenize{reference:pypath.inputs.main.reactome_sbml}}\pysiglinewithargsret{\sphinxcode{\sphinxupquote{pypath.inputs.main.}}\sphinxbfcode{\sphinxupquote{reactome\_sbml}}}{}{}
Downloads Reactome human reactions in SBML format.
Returns gzip.GzipFile object.

\end{fulllineitems}

\index{signalink\_interactions() (in module pypath.inputs.main)@\spxentry{signalink\_interactions()}\spxextra{in module pypath.inputs.main}}

\begin{fulllineitems}
\phantomsection\label{\detokenize{reference:pypath.inputs.main.signalink_interactions}}\pysiglinewithargsret{\sphinxcode{\sphinxupquote{pypath.inputs.main.}}\sphinxbfcode{\sphinxupquote{signalink\_interactions}}}{}{}
Reads and processes SignaLink3 interactions from local file.
Returns list of interactions.

\end{fulllineitems}

\index{surfaceome\_annotations() (in module pypath.inputs.main)@\spxentry{surfaceome\_annotations()}\spxextra{in module pypath.inputs.main}}

\begin{fulllineitems}
\phantomsection\label{\detokenize{reference:pypath.inputs.main.surfaceome_annotations}}\pysiglinewithargsret{\sphinxcode{\sphinxupquote{pypath.inputs.main.}}\sphinxbfcode{\sphinxupquote{surfaceome\_annotations}}}{}{}
Downloads the “In silico human surfaceome”.
Dict with UniProt IDs as key and tuples of surface prediction score,
class and subclass as values (columns B, N, S and T of table S3).

\end{fulllineitems}

\index{take\_a\_trip() (in module pypath.inputs.main)@\spxentry{take\_a\_trip()}\spxextra{in module pypath.inputs.main}}

\begin{fulllineitems}
\phantomsection\label{\detokenize{reference:pypath.inputs.main.take_a_trip}}\pysiglinewithargsret{\sphinxcode{\sphinxupquote{pypath.inputs.main.}}\sphinxbfcode{\sphinxupquote{take\_a\_trip}}}{\emph{cachefile=None}}{}
Downloads TRIP data from webpage and preprocesses it.
Saves preprocessed data into \sphinxtitleref{cachefile} and next
time loads from this file.
\begin{quote}\begin{description}
\item[{Parameters}] \leavevmode
\sphinxstyleliteralstrong{\sphinxupquote{str}} (\sphinxstyleliteralemphasis{\sphinxupquote{cachefile}}) \textendash{} Path to pickle dump of preprocessed TRIP database. If does not exist
the database will be downloaded and saved to this file. By default
the path queried from the \sphinxcode{\sphinxupquote{settings}} module.

\end{description}\end{quote}

\end{fulllineitems}

\index{tfregulons\_interactions() (in module pypath.inputs.main)@\spxentry{tfregulons\_interactions()}\spxextra{in module pypath.inputs.main}}

\begin{fulllineitems}
\phantomsection\label{\detokenize{reference:pypath.inputs.main.tfregulons_interactions}}\pysiglinewithargsret{\sphinxcode{\sphinxupquote{pypath.inputs.main.}}\sphinxbfcode{\sphinxupquote{tfregulons\_interactions}}}{\emph{levels=\{'A'}, \emph{'B'\}}, \emph{only\_curated=False}}{}
Retrieves TF-target interactions from TF regulons.
\begin{quote}\begin{description}
\item[{Parameters}] \leavevmode\begin{itemize}
\item {} 
\sphinxstyleliteralstrong{\sphinxupquote{levels}} (\sphinxstyleliteralemphasis{\sphinxupquote{set}}) \textendash{} Confidence levels to be used.

\item {} 
\sphinxstyleliteralstrong{\sphinxupquote{only\_curated}} (\sphinxstyleliteralemphasis{\sphinxupquote{bool}}) \textendash{} Retrieve only literature curated interactions.

\end{itemize}

\end{description}\end{quote}

TF regulons is a comprehensive resource of TF-target interactions
combining multiple lines of evidences: literature curated databases,
ChIP-Seq data, PWM based prediction using HOCOMOCO and JASPAR matrices
and prediction from GTEx expression data by ARACNe.

For details see \sphinxurl{https://github.com/saezlab/DoRothEA}.

\end{fulllineitems}

\index{trip\_find\_uniprot() (in module pypath.inputs.main)@\spxentry{trip\_find\_uniprot()}\spxextra{in module pypath.inputs.main}}

\begin{fulllineitems}
\phantomsection\label{\detokenize{reference:pypath.inputs.main.trip_find_uniprot}}\pysiglinewithargsret{\sphinxcode{\sphinxupquote{pypath.inputs.main.}}\sphinxbfcode{\sphinxupquote{trip\_find\_uniprot}}}{\emph{soup}}{}
Looks up a UniProt name in table downloaded from TRIP
webpage.
\begin{description}
\item[{@soup}] \leavevmode{[}bs4.BeautifulSoup{]}
The \sphinxtitleref{BeautifulSoup} instance returned by \sphinxtitleref{pypath.dataio.trip\_get\_uniprot()}.

\end{description}

\end{fulllineitems}

\index{trip\_get\_uniprot() (in module pypath.inputs.main)@\spxentry{trip\_get\_uniprot()}\spxextra{in module pypath.inputs.main}}

\begin{fulllineitems}
\phantomsection\label{\detokenize{reference:pypath.inputs.main.trip_get_uniprot}}\pysiglinewithargsret{\sphinxcode{\sphinxupquote{pypath.inputs.main.}}\sphinxbfcode{\sphinxupquote{trip\_get\_uniprot}}}{\emph{syn}}{}
Downloads table from TRIP webpage and UniProt attempts to
look up the UniProt ID for one synonym.
\begin{description}
\item[{@syn}] \leavevmode{[}str{]}
The synonym as shown on TRIP webpage.

\end{description}

\end{fulllineitems}

\index{trip\_interactions() (in module pypath.inputs.main)@\spxentry{trip\_interactions()}\spxextra{in module pypath.inputs.main}}

\begin{fulllineitems}
\phantomsection\label{\detokenize{reference:pypath.inputs.main.trip_interactions}}\pysiglinewithargsret{\sphinxcode{\sphinxupquote{pypath.inputs.main.}}\sphinxbfcode{\sphinxupquote{trip\_interactions}}}{\emph{exclude\_methods={[}'Inference', 'Speculation'{]}, predictions=False, species='Human', strict=False}}{}
Obtains processed TRIP interactions by calling \sphinxtitleref{pypath.dataio.trip\_process()}
and returns list of interactions. All arguments are passed to
\sphinxtitleref{trip\_process()}, see their definition there.

\end{fulllineitems}

\index{trip\_process() (in module pypath.inputs.main)@\spxentry{trip\_process()}\spxextra{in module pypath.inputs.main}}

\begin{fulllineitems}
\phantomsection\label{\detokenize{reference:pypath.inputs.main.trip_process}}\pysiglinewithargsret{\sphinxcode{\sphinxupquote{pypath.inputs.main.}}\sphinxbfcode{\sphinxupquote{trip\_process}}}{\emph{exclude\_methods={[}'Inference', 'Speculation'{]}, predictions=False, species='Human', strict=False}}{}
Downloads TRIP data by calling \sphinxtitleref{pypath.dadio.take\_a\_trip()} and
further provcesses it.
Returns dict of dict with TRIP data.
\begin{description}
\item[{@exclude\_methods}] \leavevmode{[}list{]}
Interaction detection methods to be discarded.

\item[{@predictions}] \leavevmode{[}bool{]}
Whether to include predicted interactions.

\item[{@species}] \leavevmode{[}str{]}
Organism name, e.g. \sphinxtitleref{Human}.

\item[{@strict}] \leavevmode{[}bool{]}
Whether include interactions with species not
used as a bait or not specified.

\end{description}

\end{fulllineitems}

\index{trip\_process\_table() (in module pypath.inputs.main)@\spxentry{trip\_process\_table()}\spxextra{in module pypath.inputs.main}}

\begin{fulllineitems}
\phantomsection\label{\detokenize{reference:pypath.inputs.main.trip_process_table}}\pysiglinewithargsret{\sphinxcode{\sphinxupquote{pypath.inputs.main.}}\sphinxbfcode{\sphinxupquote{trip\_process\_table}}}{\emph{tab}, \emph{result}, \emph{intrs}, \emph{trp\_uniprot}}{}
Processes one HTML table downloaded from TRIP webpage.
\begin{description}
\item[{@tab}] \leavevmode{[}bs4.element.Tag(){]}
One table of interactions from TRIP webpage.

\item[{@result}] \leavevmode{[}dict{]}
Dictionary the data should be filled in.

\item[{@intrs}] \leavevmode{[}dict{]}
Dictionary of already converted interactor IDs.
This serves as a cache so do not need to look up
the same ID twice.

\item[{@trp\_uniprot}] \leavevmode{[}str{]}
UniProt ID of TRP domain containing protein.

\end{description}

\end{fulllineitems}

\index{wang\_interactions() (in module pypath.inputs.main)@\spxentry{wang\_interactions()}\spxextra{in module pypath.inputs.main}}

\begin{fulllineitems}
\phantomsection\label{\detokenize{reference:pypath.inputs.main.wang_interactions}}\pysiglinewithargsret{\sphinxcode{\sphinxupquote{pypath.inputs.main.}}\sphinxbfcode{\sphinxupquote{wang\_interactions}}}{}{}
Downloads and processes Wang Lab HumanSignalingNetwork.
Returns list of interactions.

\end{fulllineitems}

\index{zhong2015\_annotations() (in module pypath.inputs.main)@\spxentry{zhong2015\_annotations()}\spxextra{in module pypath.inputs.main}}

\begin{fulllineitems}
\phantomsection\label{\detokenize{reference:pypath.inputs.main.zhong2015_annotations}}\pysiglinewithargsret{\sphinxcode{\sphinxupquote{pypath.inputs.main.}}\sphinxbfcode{\sphinxupquote{zhong2015\_annotations}}}{}{}
From 10.1111/nyas.12776 (PMID 25988664).

\end{fulllineitems}



\section{descriptions}
\label{\detokenize{reference:module-pypath.resources.descriptions}}\label{\detokenize{reference:descriptions}}\index{pypath.resources.descriptions (module)@\spxentry{pypath.resources.descriptions}\spxextra{module}}\index{gen\_html() (in module pypath.resources.descriptions)@\spxentry{gen\_html()}\spxextra{in module pypath.resources.descriptions}}

\begin{fulllineitems}
\phantomsection\label{\detokenize{reference:pypath.resources.descriptions.gen_html}}\pysiglinewithargsret{\sphinxcode{\sphinxupquote{pypath.resources.descriptions.}}\sphinxbfcode{\sphinxupquote{gen\_html}}}{}{}
Generates a HTML page from the \sphinxtitleref{descriptions} array.
This HTML is provided by the webservice under \sphinxtitleref{/info},
or can be saved locally with \sphinxtitleref{write\_html()}.

\end{fulllineitems}

\index{write\_html() (in module pypath.resources.descriptions)@\spxentry{write\_html()}\spxextra{in module pypath.resources.descriptions}}

\begin{fulllineitems}
\phantomsection\label{\detokenize{reference:pypath.resources.descriptions.write_html}}\pysiglinewithargsret{\sphinxcode{\sphinxupquote{pypath.resources.descriptions.}}\sphinxbfcode{\sphinxupquote{write\_html}}}{\emph{filename='resources.html'}}{}
Saves the HTML descriptions to custom local file.

\end{fulllineitems}



\section{entity}
\label{\detokenize{reference:module-pypath.core.entity}}\label{\detokenize{reference:entity}}\index{pypath.core.entity (module)@\spxentry{pypath.core.entity}\spxextra{module}}
Provides classes for representing molecular entities and their collections.
A molecular entity is defined by its identifier, type and taxon.
\index{Entity (class in pypath.core.entity)@\spxentry{Entity}\spxextra{class in pypath.core.entity}}

\begin{fulllineitems}
\phantomsection\label{\detokenize{reference:pypath.core.entity.Entity}}\pysiglinewithargsret{\sphinxbfcode{\sphinxupquote{class }}\sphinxcode{\sphinxupquote{pypath.core.entity.}}\sphinxbfcode{\sphinxupquote{Entity}}}{\emph{identifier}, \emph{entity\_type='protein'}, \emph{id\_type='uniprot'}, \emph{taxon=9606}, \emph{attrs=None}}{}
Represents a molecular entity such as protein, miRNA, lncRNA or small
molecule.
\begin{quote}\begin{description}
\item[{Parameters}] \leavevmode\begin{itemize}
\item {} 
\sphinxstyleliteralstrong{\sphinxupquote{identifier}} (\sphinxstyleliteralemphasis{\sphinxupquote{str}}) \textendash{} An identifier from the reference database e.g. UniProt ID for
proteins.

\item {} 
\sphinxstyleliteralstrong{\sphinxupquote{entity\_type}} (\sphinxstyleliteralemphasis{\sphinxupquote{str}}) \textendash{} The type of the molecular entity, defaults to \sphinxcode{\sphinxupquote{'protein'}}.

\item {} 
\sphinxstyleliteralstrong{\sphinxupquote{id\_type}} (\sphinxstyleliteralemphasis{\sphinxupquote{str}}) \textendash{} The type of the identifier (the reference database), default is
\sphinxcode{\sphinxupquote{'uniprot'}}.

\item {} 
\sphinxstyleliteralstrong{\sphinxupquote{taxon}} (\sphinxstyleliteralemphasis{\sphinxupquote{int}}) \textendash{} The NCBI Taxonomy Identifier of the molecular entity, e.g. \sphinxcode{\sphinxupquote{9606}}
for human. Use \sphinxcode{\sphinxupquote{0}} for non taxon specific molecules e.g. metabolites
or drug compounds.

\item {} 
\sphinxstyleliteralstrong{\sphinxupquote{attrs}} (\sphinxstyleliteralemphasis{\sphinxupquote{NoneType}}\sphinxstyleliteralemphasis{\sphinxupquote{,}}\sphinxstyleliteralemphasis{\sphinxupquote{dict}}) \textendash{} A dictionary of additional attributes.

\end{itemize}

\end{description}\end{quote}

\end{fulllineitems}

\index{EntityKey (class in pypath.core.entity)@\spxentry{EntityKey}\spxextra{class in pypath.core.entity}}

\begin{fulllineitems}
\phantomsection\label{\detokenize{reference:pypath.core.entity.EntityKey}}\pysiglinewithargsret{\sphinxbfcode{\sphinxupquote{class }}\sphinxcode{\sphinxupquote{pypath.core.entity.}}\sphinxbfcode{\sphinxupquote{EntityKey}}}{\emph{identifier}, \emph{id\_type}, \emph{entity\_type}, \emph{taxon}}{}~\index{entity\_type (pypath.core.entity.EntityKey attribute)@\spxentry{entity\_type}\spxextra{pypath.core.entity.EntityKey attribute}}

\begin{fulllineitems}
\phantomsection\label{\detokenize{reference:pypath.core.entity.EntityKey.entity_type}}\pysigline{\sphinxbfcode{\sphinxupquote{entity\_type}}}
Alias for field number 2

\end{fulllineitems}

\index{id\_type (pypath.core.entity.EntityKey attribute)@\spxentry{id\_type}\spxextra{pypath.core.entity.EntityKey attribute}}

\begin{fulllineitems}
\phantomsection\label{\detokenize{reference:pypath.core.entity.EntityKey.id_type}}\pysigline{\sphinxbfcode{\sphinxupquote{id\_type}}}
Alias for field number 1

\end{fulllineitems}

\index{identifier (pypath.core.entity.EntityKey attribute)@\spxentry{identifier}\spxextra{pypath.core.entity.EntityKey attribute}}

\begin{fulllineitems}
\phantomsection\label{\detokenize{reference:pypath.core.entity.EntityKey.identifier}}\pysigline{\sphinxbfcode{\sphinxupquote{identifier}}}
Alias for field number 0

\end{fulllineitems}

\index{taxon (pypath.core.entity.EntityKey attribute)@\spxentry{taxon}\spxextra{pypath.core.entity.EntityKey attribute}}

\begin{fulllineitems}
\phantomsection\label{\detokenize{reference:pypath.core.entity.EntityKey.taxon}}\pysigline{\sphinxbfcode{\sphinxupquote{taxon}}}
Alias for field number 3

\end{fulllineitems}


\end{fulllineitems}



\section{evidence}
\label{\detokenize{reference:module-pypath.core.evidence}}\label{\detokenize{reference:evidence}}\index{pypath.core.evidence (module)@\spxentry{pypath.core.evidence}\spxextra{module}}
Provides classes for representing and processing evidences supporting
relationships. The evidences hold information about the databases and
literature references, they can be organized into collections. A number
of operations are available on evidences and their collections, for
example they can be combined or filtered.
\index{Evidence (class in pypath.core.evidence)@\spxentry{Evidence}\spxextra{class in pypath.core.evidence}}

\begin{fulllineitems}
\phantomsection\label{\detokenize{reference:pypath.core.evidence.Evidence}}\pysiglinewithargsret{\sphinxbfcode{\sphinxupquote{class }}\sphinxcode{\sphinxupquote{pypath.core.evidence.}}\sphinxbfcode{\sphinxupquote{Evidence}}}{\emph{resource}, \emph{references=None}}{}
Represents an evidence supporting a relationship such as molecular
interaction, molecular complex, enzyme-PTM interaction, annotation, etc.

The evidence consists of two main parts: the database and the literature
references. If a relationship is supprted by multiple databases, for
each one \sphinxtitleref{Evidence} object should be created and
\begin{quote}\begin{description}
\item[{Parameters}] \leavevmode\begin{itemize}
\item {} 
\sphinxstyleliteralstrong{\sphinxupquote{resource}} (\sphinxstyleliteralemphasis{\sphinxupquote{pypath.resource.ResourceAttributes}}) \textendash{} An object derived from \sphinxcode{\sphinxupquote{pypath.resource.ResourceAttributes}}.

\item {} 
\sphinxstyleliteralstrong{\sphinxupquote{references}} (\sphinxstyleliteralemphasis{\sphinxupquote{str}}\sphinxstyleliteralemphasis{\sphinxupquote{,}}\sphinxstyleliteralemphasis{\sphinxupquote{list}}\sphinxstyleliteralemphasis{\sphinxupquote{,}}\sphinxstyleliteralemphasis{\sphinxupquote{set}}\sphinxstyleliteralemphasis{\sphinxupquote{,}}\sphinxstyleliteralemphasis{\sphinxupquote{NoneType}}) \textendash{} Optional, one or more literature references (preferably PubMed IDs).

\end{itemize}

\end{description}\end{quote}
\index{has\_interaction\_type() (pypath.core.evidence.Evidence method)@\spxentry{has\_interaction\_type()}\spxextra{pypath.core.evidence.Evidence method}}

\begin{fulllineitems}
\phantomsection\label{\detokenize{reference:pypath.core.evidence.Evidence.has_interaction_type}}\pysiglinewithargsret{\sphinxbfcode{\sphinxupquote{has\_interaction\_type}}}{\emph{interaction\_type}, \emph{database=None}, \emph{via=False}}{}
If \sphinxcode{\sphinxupquote{via}} is \sphinxcode{\sphinxupquote{False}} then it will be ignored, otherwise if \sphinxcode{\sphinxupquote{None}}
only primary resources are considered.

\end{fulllineitems}

\index{merge() (pypath.core.evidence.Evidence method)@\spxentry{merge()}\spxextra{pypath.core.evidence.Evidence method}}

\begin{fulllineitems}
\phantomsection\label{\detokenize{reference:pypath.core.evidence.Evidence.merge}}\pysiglinewithargsret{\sphinxbfcode{\sphinxupquote{merge}}}{\emph{other}}{}
Merges two evidences. Returns set of either one or two evidences
depending on whether the two evidences are from the same resource.

\end{fulllineitems}

\index{reload() (pypath.core.evidence.Evidence method)@\spxentry{reload()}\spxextra{pypath.core.evidence.Evidence method}}

\begin{fulllineitems}
\phantomsection\label{\detokenize{reference:pypath.core.evidence.Evidence.reload}}\pysiglinewithargsret{\sphinxbfcode{\sphinxupquote{reload}}}{}{}
Reloads the object from the module level.

\end{fulllineitems}


\end{fulllineitems}

\index{Evidences (class in pypath.core.evidence)@\spxentry{Evidences}\spxextra{class in pypath.core.evidence}}

\begin{fulllineitems}
\phantomsection\label{\detokenize{reference:pypath.core.evidence.Evidences}}\pysiglinewithargsret{\sphinxbfcode{\sphinxupquote{class }}\sphinxcode{\sphinxupquote{pypath.core.evidence.}}\sphinxbfcode{\sphinxupquote{Evidences}}}{\emph{evidences=()}}{}
A collection of evidences. All evidences supporting a relationship such
as molecular interaction, molecular complex, enzyme-PTM interaction,
annotation, etc should be collected in one \sphinxtitleref{Evidences} object. This way
the set of evidences can be queried a comprehensive way.
\begin{quote}\begin{description}
\item[{Parameters}] \leavevmode
\sphinxstyleliteralstrong{\sphinxupquote{evidences}} (\sphinxstyleliteralemphasis{\sphinxupquote{tuple}}\sphinxstyleliteralemphasis{\sphinxupquote{,}}\sphinxstyleliteralemphasis{\sphinxupquote{list}}\sphinxstyleliteralemphasis{\sphinxupquote{,}}\sphinxstyleliteralemphasis{\sphinxupquote{set}}\sphinxstyleliteralemphasis{\sphinxupquote{,}}{\hyperref[\detokenize{reference:pypath.core.evidence.Evidences}]{\sphinxcrossref{\sphinxstyleliteralemphasis{\sphinxupquote{Evidences}}}}}) \textendash{} An iterable providing {\hyperref[\detokenize{reference:pypath.core.evidence.Evidence}]{\sphinxcrossref{\sphinxcode{\sphinxupquote{Evidence}}}}} instances. It is possible
to create an empty evidence collection and populate it later or to
show this way that certain relationship has no supporting evidences.

\end{description}\end{quote}
\index{has\_interaction\_type() (pypath.core.evidence.Evidences method)@\spxentry{has\_interaction\_type()}\spxextra{pypath.core.evidence.Evidences method}}

\begin{fulllineitems}
\phantomsection\label{\detokenize{reference:pypath.core.evidence.Evidences.has_interaction_type}}\pysiglinewithargsret{\sphinxbfcode{\sphinxupquote{has\_interaction\_type}}}{\emph{interaction\_type}, \emph{database=None}, \emph{via=False}}{}
If \sphinxcode{\sphinxupquote{via}} is \sphinxcode{\sphinxupquote{False}} then it will be ignored, otherwise if \sphinxcode{\sphinxupquote{None}}
only primary resources are considered.

\end{fulllineitems}

\index{reload() (pypath.core.evidence.Evidences method)@\spxentry{reload()}\spxextra{pypath.core.evidence.Evidences method}}

\begin{fulllineitems}
\phantomsection\label{\detokenize{reference:pypath.core.evidence.Evidences.reload}}\pysiglinewithargsret{\sphinxbfcode{\sphinxupquote{reload}}}{}{}
Reloads the object from the module level.

\end{fulllineitems}


\end{fulllineitems}



\section{export}
\label{\detokenize{reference:module-pypath.omnipath.export}}\label{\detokenize{reference:export}}\index{pypath.omnipath.export (module)@\spxentry{pypath.omnipath.export}\spxextra{module}}

\section{go}
\label{\detokenize{reference:module-pypath.utils.go}}\label{\detokenize{reference:go}}\index{pypath.utils.go (module)@\spxentry{pypath.utils.go}\spxextra{module}}\index{annotate() (in module pypath.utils.go)@\spxentry{annotate()}\spxextra{in module pypath.utils.go}}

\begin{fulllineitems}
\phantomsection\label{\detokenize{reference:pypath.utils.go.annotate}}\pysiglinewithargsret{\sphinxcode{\sphinxupquote{pypath.utils.go.}}\sphinxbfcode{\sphinxupquote{annotate}}}{\emph{graph}, \emph{organism=9606}, \emph{aspects=('C'}, \emph{'F'}, \emph{'P')}}{}
Adds Gene Ontology annotations to the nodes of a graph.
\begin{quote}\begin{description}
\item[{Parameters}] \leavevmode
\sphinxstyleliteralstrong{\sphinxupquote{graph}} (\sphinxstyleliteralemphasis{\sphinxupquote{igraph.Graph}}) \textendash{} Any \sphinxcode{\sphinxupquote{igraph.Graph}} object with uniprot IDs
in its \sphinxcode{\sphinxupquote{name}} vertex attribute.

\end{description}\end{quote}

\end{fulllineitems}

\index{get\_db() (in module pypath.utils.go)@\spxentry{get\_db()}\spxextra{in module pypath.utils.go}}

\begin{fulllineitems}
\phantomsection\label{\detokenize{reference:pypath.utils.go.get_db}}\pysiglinewithargsret{\sphinxcode{\sphinxupquote{pypath.utils.go.}}\sphinxbfcode{\sphinxupquote{get\_db}}}{\emph{organism=9606}, \emph{pickle\_file=None}, \emph{use\_pickle\_cache=True}}{}
Retrieves the current database instance and initializes it if does
not exist yet.

\end{fulllineitems}

\index{init\_db() (in module pypath.utils.go)@\spxentry{init\_db()}\spxextra{in module pypath.utils.go}}

\begin{fulllineitems}
\phantomsection\label{\detokenize{reference:pypath.utils.go.init_db}}\pysiglinewithargsret{\sphinxcode{\sphinxupquote{pypath.utils.go.}}\sphinxbfcode{\sphinxupquote{init\_db}}}{\emph{organism=9606}, \emph{pickle\_file=None}, \emph{use\_pickle\_cache=True}}{}
Initializes or reloads the GO annotation database.
The database will be assigned to the \sphinxcode{\sphinxupquote{db}} attribute of this module.

\end{fulllineitems}

\index{load\_go() (in module pypath.utils.go)@\spxentry{load\_go()}\spxextra{in module pypath.utils.go}}

\begin{fulllineitems}
\phantomsection\label{\detokenize{reference:pypath.utils.go.load_go}}\pysiglinewithargsret{\sphinxcode{\sphinxupquote{pypath.utils.go.}}\sphinxbfcode{\sphinxupquote{load\_go}}}{\emph{graph}, \emph{organism=9606}, \emph{aspects=('C'}, \emph{'F'}, \emph{'P')}}{}
Adds Gene Ontology annotations to the nodes of a graph.
\begin{quote}\begin{description}
\item[{Parameters}] \leavevmode
\sphinxstyleliteralstrong{\sphinxupquote{graph}} (\sphinxstyleliteralemphasis{\sphinxupquote{igraph.Graph}}) \textendash{} Any \sphinxcode{\sphinxupquote{igraph.Graph}} object with uniprot IDs
in its \sphinxcode{\sphinxupquote{name}} vertex attribute.

\end{description}\end{quote}

\end{fulllineitems}



\section{homology}
\label{\detokenize{reference:module-pypath.utils.homology}}\label{\detokenize{reference:homology}}\index{pypath.utils.homology (module)@\spxentry{pypath.utils.homology}\spxextra{module}}\index{get\_homologene() (in module pypath.utils.homology)@\spxentry{get\_homologene()}\spxextra{in module pypath.utils.homology}}

\begin{fulllineitems}
\phantomsection\label{\detokenize{reference:pypath.utils.homology.get_homologene}}\pysiglinewithargsret{\sphinxcode{\sphinxupquote{pypath.utils.homology.}}\sphinxbfcode{\sphinxupquote{get\_homologene}}}{}{}
Downloads the recent release of the NCBI HomoloGene database.
Returns file pointer.

\end{fulllineitems}

\index{homologene\_dict() (in module pypath.utils.homology)@\spxentry{homologene\_dict()}\spxextra{in module pypath.utils.homology}}

\begin{fulllineitems}
\phantomsection\label{\detokenize{reference:pypath.utils.homology.homologene_dict}}\pysiglinewithargsret{\sphinxcode{\sphinxupquote{pypath.utils.homology.}}\sphinxbfcode{\sphinxupquote{homologene\_dict}}}{\emph{source}, \emph{target}, \emph{id\_type}}{}
Returns orthology translation table as dict, obtained
from NCBI HomoloGene data.
\begin{quote}\begin{description}
\item[{Parameters}] \leavevmode\begin{itemize}
\item {} 
\sphinxstyleliteralstrong{\sphinxupquote{source}} (\sphinxstyleliteralemphasis{\sphinxupquote{int}}) \textendash{} NCBI Taxonomy ID of the source species (keys).

\item {} 
\sphinxstyleliteralstrong{\sphinxupquote{target}} (\sphinxstyleliteralemphasis{\sphinxupquote{int}}) \textendash{} NCBI Taxonomy ID of the target species (values).

\item {} 
\sphinxstyleliteralstrong{\sphinxupquote{id\_type}} (\sphinxstyleliteralemphasis{\sphinxupquote{str}}) \textendash{} ID type to be used in the dict. Possible values:
‘RefSeq’, ‘Entrez’, ‘GI’, ‘GeneSymbol’.

\end{itemize}

\end{description}\end{quote}

\end{fulllineitems}

\index{homologene\_uniprot\_dict() (in module pypath.utils.homology)@\spxentry{homologene\_uniprot\_dict()}\spxextra{in module pypath.utils.homology}}

\begin{fulllineitems}
\phantomsection\label{\detokenize{reference:pypath.utils.homology.homologene_uniprot_dict}}\pysiglinewithargsret{\sphinxcode{\sphinxupquote{pypath.utils.homology.}}\sphinxbfcode{\sphinxupquote{homologene\_uniprot\_dict}}}{\emph{source}, \emph{target}, \emph{only\_swissprot=True}}{}
Returns orthology translation table as dict from UniProt to Uniprot,
obtained from NCBI HomoloGene data. Uses RefSeq and Entrez IDs for
translation.
\begin{quote}\begin{description}
\item[{Parameters}] \leavevmode\begin{itemize}
\item {} 
\sphinxstyleliteralstrong{\sphinxupquote{source}} (\sphinxstyleliteralemphasis{\sphinxupquote{int}}) \textendash{} NCBI Taxonomy ID of the source species (keys).

\item {} 
\sphinxstyleliteralstrong{\sphinxupquote{target}} (\sphinxstyleliteralemphasis{\sphinxupquote{int}}) \textendash{} NCBI Taxonomy ID of the target species (values).

\item {} 
\sphinxstyleliteralstrong{\sphinxupquote{only\_swissprot}} (\sphinxstyleliteralemphasis{\sphinxupquote{bool}}) \textendash{} Translate only SwissProt IDs.

\end{itemize}

\end{description}\end{quote}

\end{fulllineitems}



\section{input\_formats}
\label{\detokenize{reference:module-pypath.internals.input_formats}}\label{\detokenize{reference:input-formats}}\index{pypath.internals.input\_formats (module)@\spxentry{pypath.internals.input\_formats}\spxextra{module}}

\section{interaction}
\label{\detokenize{reference:module-pypath.core.interaction}}\label{\detokenize{reference:interaction}}\index{pypath.core.interaction (module)@\spxentry{pypath.core.interaction}\spxextra{module}}
Here we define one class, the \sphinxcode{\sphinxupquote{Interaction{}`}} which provides a
rich API for representing and querying molecular interactions. The
interactions serve as the building elements of the network and the
\sphinxcode{\sphinxupquote{pypath.network.Network}} object largely relies on methods
of the \sphinxcode{\sphinxupquote{Interaction{}`}} objects.
\index{Interaction (class in pypath.core.interaction)@\spxentry{Interaction}\spxextra{class in pypath.core.interaction}}

\begin{fulllineitems}
\phantomsection\label{\detokenize{reference:pypath.core.interaction.Interaction}}\pysiglinewithargsret{\sphinxbfcode{\sphinxupquote{class }}\sphinxcode{\sphinxupquote{pypath.core.interaction.}}\sphinxbfcode{\sphinxupquote{Interaction}}}{\emph{a}, \emph{b}, \emph{id\_type\_a='uniprot'}, \emph{id\_type\_b='uniprot'}, \emph{entity\_type\_a='protein'}, \emph{entity\_type\_b='protein'}, \emph{taxon\_a=9606}, \emph{taxon\_b=9606}}{}
Represents a unique pair of molecular entities interacting with each
other. One {\hyperref[\detokenize{reference:pypath.core.interaction.Interaction}]{\sphinxcrossref{\sphinxcode{\sphinxupquote{Interaction}}}}} object might represent multiple
interactions i.e. with different direction or effect or type (e.g.
transcriptional regulation and post-translational regulation),
each supported by different evidences.
\begin{quote}\begin{description}
\item[{Parameters}] \leavevmode\begin{itemize}
\item {} 
\sphinxstyleliteralstrong{\sphinxupquote{a}}\sphinxstyleliteralstrong{\sphinxupquote{,}}\sphinxstyleliteralstrong{\sphinxupquote{b}} (\sphinxstyleliteralemphasis{\sphinxupquote{str}}\sphinxstyleliteralemphasis{\sphinxupquote{,}}\sphinxstyleliteralemphasis{\sphinxupquote{pypath.entity.Entity}}) \textendash{} The two interacting partners. If an \sphinxcode{\sphinxupquote{pypath.entity.Entity}}
objects provided the other attributes (entity\_type, id\_type, taxon)
will be ignored.

\item {} 
\sphinxstyleliteralstrong{\sphinxupquote{id\_type\_a}}\sphinxstyleliteralstrong{\sphinxupquote{,}}\sphinxstyleliteralstrong{\sphinxupquote{id\_type\_b}} (\sphinxstyleliteralemphasis{\sphinxupquote{str}}) \textendash{} The identifier types for partner \sphinxcode{\sphinxupquote{a}} and \sphinxcode{\sphinxupquote{b}} e.g. \sphinxcode{\sphinxupquote{'uniprot'}}.

\item {} 
\sphinxstyleliteralstrong{\sphinxupquote{entity\_type\_a}}\sphinxstyleliteralstrong{\sphinxupquote{,}}\sphinxstyleliteralstrong{\sphinxupquote{entity\_type\_b}} (\sphinxstyleliteralemphasis{\sphinxupquote{str}}) \textendash{} The types of the molecular entities \sphinxcode{\sphinxupquote{a}} and \sphinxcode{\sphinxupquote{b}}
e.g. \sphinxcode{\sphinxupquote{'protein'}}.

\item {} 
\sphinxstyleliteralstrong{\sphinxupquote{taxon\_a}}\sphinxstyleliteralstrong{\sphinxupquote{,}}\sphinxstyleliteralstrong{\sphinxupquote{taxon\_b}} (\sphinxstyleliteralemphasis{\sphinxupquote{int}}) \textendash{} The NCBI Taxonomy Identifiers of partner \sphinxcode{\sphinxupquote{a}} and \sphinxcode{\sphinxupquote{b}}
e.g. \sphinxcode{\sphinxupquote{9606}} for human.

\end{itemize}

\item[{Details}] \leavevmode
The arguments \sphinxcode{\sphinxupquote{a}} and \sphinxcode{\sphinxupquote{b}} will be assigned to the attribute \sphinxcode{\sphinxupquote{a}}
and \sphinxcode{\sphinxupquote{b}} in an alphabetical order, hence it’s possible that
argument \sphinxcode{\sphinxupquote{a}} becomes attribute \sphinxcode{\sphinxupquote{b}}.

\end{description}\end{quote}
\index{add\_evidence() (pypath.core.interaction.Interaction method)@\spxentry{add\_evidence()}\spxextra{pypath.core.interaction.Interaction method}}

\begin{fulllineitems}
\phantomsection\label{\detokenize{reference:pypath.core.interaction.Interaction.add_evidence}}\pysiglinewithargsret{\sphinxbfcode{\sphinxupquote{add\_evidence}}}{\emph{evidence}, \emph{direction='undirected'}, \emph{effect=0}, \emph{references=None}}{}
Adds directionality information with the corresponding data
source named. Modifies self attributes \sphinxcode{\sphinxupquote{dirs}} and
\sphinxcode{\sphinxupquote{sources}}.
\begin{quote}\begin{description}
\item[{Parameters}] \leavevmode\begin{itemize}
\item {} 
\sphinxstyleliteralstrong{\sphinxupquote{evidence}} (\sphinxstyleliteralemphasis{\sphinxupquote{resource.NetworkResource}}\sphinxstyleliteralemphasis{\sphinxupquote{,}}{\hyperref[\detokenize{reference:pypath.core.evidence.Evidence}]{\sphinxcrossref{\sphinxstyleliteralemphasis{\sphinxupquote{evidence.Evidence}}}}}) \textendash{} Either a \sphinxcode{\sphinxupquote{pypath.evidence.Evidence}} object or a resource as
\sphinxcode{\sphinxupquote{pypath.resource.NetworkResource}} object. In the latter case
the references can be provided in a separate argument.

\item {} 
\sphinxstyleliteralstrong{\sphinxupquote{direction}} (\sphinxstyleliteralemphasis{\sphinxupquote{tuple}}) \textendash{} Or {[}str{]}, the directionality key for which the value on
\sphinxcode{\sphinxupquote{dirs}} has to be set \sphinxcode{\sphinxupquote{True}}.

\item {} 
\sphinxstyleliteralstrong{\sphinxupquote{effect}} (\sphinxstyleliteralemphasis{\sphinxupquote{int}}) \textendash{} The causal effect of the interaction. 1 or ‘stimulation’
corresponds to a stimulatory, -1 or ‘inhibition’ to an
inhibitory while 0 to an unknown or neutral effect.

\item {} 
\sphinxstyleliteralstrong{\sphinxupquote{references}} (\sphinxstyleliteralemphasis{\sphinxupquote{set}}\sphinxstyleliteralemphasis{\sphinxupquote{,}}\sphinxstyleliteralemphasis{\sphinxupquote{NoneType}}) \textendash{} A set of references, used only if the resource have been provided
as \sphinxcode{\sphinxupquote{NetworkResource}} object.

\end{itemize}

\end{description}\end{quote}

\end{fulllineitems}

\index{add\_sign() (pypath.core.interaction.Interaction method)@\spxentry{add\_sign()}\spxextra{pypath.core.interaction.Interaction method}}

\begin{fulllineitems}
\phantomsection\label{\detokenize{reference:pypath.core.interaction.Interaction.add_sign}}\pysiglinewithargsret{\sphinxbfcode{\sphinxupquote{add\_sign}}}{\emph{direction}, \emph{sign}, \emph{resource=None}, \emph{resource\_name=None}, \emph{interaction\_type='PPI'}, \emph{data\_model=None}, \emph{**kwargs}}{}
Sets sign and source information on a given direction of the
edge. Modifies the attributes \sphinxcode{\sphinxupquote{positive}} and
\sphinxcode{\sphinxupquote{positive\_sources}} or \sphinxcode{\sphinxupquote{negative}} and
\sphinxcode{\sphinxupquote{negative\_sources}} depending on the sign. Direction is
also updated accordingly, which also modifies the attributes
\sphinxcode{\sphinxupquote{dirs}} and \sphinxcode{\sphinxupquote{sources}}.
\begin{quote}\begin{description}
\item[{Parameters}] \leavevmode\begin{itemize}
\item {} 
\sphinxstyleliteralstrong{\sphinxupquote{direction}} (\sphinxstyleliteralemphasis{\sphinxupquote{tuple}}) \textendash{} Pair of edge nodes specifying the direction from which the
information is to be set/updated.

\item {} 
\sphinxstyleliteralstrong{\sphinxupquote{sign}} (\sphinxstyleliteralemphasis{\sphinxupquote{str}}) \textendash{} Specifies the type of interaction. Either \sphinxcode{\sphinxupquote{'positive'}} or
\sphinxcode{\sphinxupquote{'negative'}}.

\item {} 
\sphinxstyleliteralstrong{\sphinxupquote{resource}} (\sphinxstyleliteralemphasis{\sphinxupquote{set}}) \textendash{} Contains the name(s) of the source(s) from which the
information was obtained.

\item {} 
\sphinxstyleliteralstrong{\sphinxupquote{**kwargs}} \textendash{} 
Passed to \sphinxcode{\sphinxupquote{pypath.resource.NetworkResource}} if \sphinxcode{\sphinxupquote{resource}}
is not already a \sphinxcode{\sphinxupquote{NetworkResource}} or \sphinxcode{\sphinxupquote{Evidence}}
instance.


\end{itemize}

\end{description}\end{quote}

\end{fulllineitems}

\index{complex\_identifiers\_by\_data\_model() (pypath.core.interaction.Interaction method)@\spxentry{complex\_identifiers\_by\_data\_model()}\spxextra{pypath.core.interaction.Interaction method}}

\begin{fulllineitems}
\phantomsection\label{\detokenize{reference:pypath.core.interaction.Interaction.complex_identifiers_by_data_model}}\pysiglinewithargsret{\sphinxbfcode{\sphinxupquote{complex\_identifiers\_by\_data\_model}}}{\emph{effect=None}, \emph{resources=None}, \emph{data\_model=None}, \emph{interaction\_type=None}, \emph{via=None}, \emph{references=None}}{}
Retrieves the entities involved in interactions matching the criteria.
It either returns both interacting entities in a \sphinxstyleemphasis{set} or an empty
\sphinxstyleemphasis{set}. This may not sound so useful at the level of this object but
becomes more useful once we want to collect entities having certain
kind of interactions across a series of \sphinxtitleref{Interaction} objects.
\begin{quote}\begin{description}
\item[{Parameters}] \leavevmode\begin{itemize}
\item {} 
\sphinxstyleliteralstrong{\sphinxupquote{entity\_type}} (\sphinxstyleliteralemphasis{\sphinxupquote{str}}) \textendash{} The type of the molecular entity. Possible values: \sphinxtitleref{protein},
\sphinxtitleref{complex}, \sphinxtitleref{mirna}, \sphinxtitleref{small\_molecule}.

\item {} 
\sphinxstyleliteralstrong{\sphinxupquote{return\_type}} (\sphinxstyleliteralemphasis{\sphinxupquote{str}}) \textendash{} The type of values to return. Default is
py:class:\sphinxcode{\sphinxupquote{pypath.entity.Entity}} objects, alternatives are
\sphinxcode{\sphinxupquote{labels}}  \sphinxcode{\sphinxupquote{identifiers}}.

\end{itemize}

\end{description}\end{quote}

\end{fulllineitems}

\index{complex\_identifiers\_by\_interaction\_type() (pypath.core.interaction.Interaction method)@\spxentry{complex\_identifiers\_by\_interaction\_type()}\spxextra{pypath.core.interaction.Interaction method}}

\begin{fulllineitems}
\phantomsection\label{\detokenize{reference:pypath.core.interaction.Interaction.complex_identifiers_by_interaction_type}}\pysiglinewithargsret{\sphinxbfcode{\sphinxupquote{complex\_identifiers\_by\_interaction\_type}}}{\emph{effect=None}, \emph{resources=None}, \emph{data\_model=None}, \emph{interaction\_type=None}, \emph{via=None}, \emph{references=None}}{}
Retrieves the entities involved in interactions matching the criteria.
It either returns both interacting entities in a \sphinxstyleemphasis{set} or an empty
\sphinxstyleemphasis{set}. This may not sound so useful at the level of this object but
becomes more useful once we want to collect entities having certain
kind of interactions across a series of \sphinxtitleref{Interaction} objects.
\begin{quote}\begin{description}
\item[{Parameters}] \leavevmode\begin{itemize}
\item {} 
\sphinxstyleliteralstrong{\sphinxupquote{entity\_type}} (\sphinxstyleliteralemphasis{\sphinxupquote{str}}) \textendash{} The type of the molecular entity. Possible values: \sphinxtitleref{protein},
\sphinxtitleref{complex}, \sphinxtitleref{mirna}, \sphinxtitleref{small\_molecule}.

\item {} 
\sphinxstyleliteralstrong{\sphinxupquote{return\_type}} (\sphinxstyleliteralemphasis{\sphinxupquote{str}}) \textendash{} The type of values to return. Default is
py:class:\sphinxcode{\sphinxupquote{pypath.entity.Entity}} objects, alternatives are
\sphinxcode{\sphinxupquote{labels}}  \sphinxcode{\sphinxupquote{identifiers}}.

\end{itemize}

\end{description}\end{quote}

\end{fulllineitems}

\index{complex\_identifiers\_by\_interaction\_type\_and\_data\_model() (pypath.core.interaction.Interaction method)@\spxentry{complex\_identifiers\_by\_interaction\_type\_and\_data\_model()}\spxextra{pypath.core.interaction.Interaction method}}

\begin{fulllineitems}
\phantomsection\label{\detokenize{reference:pypath.core.interaction.Interaction.complex_identifiers_by_interaction_type_and_data_model}}\pysiglinewithargsret{\sphinxbfcode{\sphinxupquote{complex\_identifiers\_by\_interaction\_type\_and\_data\_model}}}{\emph{effect=None}, \emph{resources=None}, \emph{data\_model=None}, \emph{interaction\_type=None}, \emph{via=None}, \emph{references=None}}{}
Retrieves the entities involved in interactions matching the criteria.
It either returns both interacting entities in a \sphinxstyleemphasis{set} or an empty
\sphinxstyleemphasis{set}. This may not sound so useful at the level of this object but
becomes more useful once we want to collect entities having certain
kind of interactions across a series of \sphinxtitleref{Interaction} objects.
\begin{quote}\begin{description}
\item[{Parameters}] \leavevmode\begin{itemize}
\item {} 
\sphinxstyleliteralstrong{\sphinxupquote{entity\_type}} (\sphinxstyleliteralemphasis{\sphinxupquote{str}}) \textendash{} The type of the molecular entity. Possible values: \sphinxtitleref{protein},
\sphinxtitleref{complex}, \sphinxtitleref{mirna}, \sphinxtitleref{small\_molecule}.

\item {} 
\sphinxstyleliteralstrong{\sphinxupquote{return\_type}} (\sphinxstyleliteralemphasis{\sphinxupquote{str}}) \textendash{} The type of values to return. Default is
py:class:\sphinxcode{\sphinxupquote{pypath.entity.Entity}} objects, alternatives are
\sphinxcode{\sphinxupquote{labels}}  \sphinxcode{\sphinxupquote{identifiers}}.

\end{itemize}

\end{description}\end{quote}

\end{fulllineitems}

\index{complex\_identifiers\_by\_interaction\_type\_and\_data\_model\_and\_resource() (pypath.core.interaction.Interaction method)@\spxentry{complex\_identifiers\_by\_interaction\_type\_and\_data\_model\_and\_resource()}\spxextra{pypath.core.interaction.Interaction method}}

\begin{fulllineitems}
\phantomsection\label{\detokenize{reference:pypath.core.interaction.Interaction.complex_identifiers_by_interaction_type_and_data_model_and_resource}}\pysiglinewithargsret{\sphinxbfcode{\sphinxupquote{complex\_identifiers\_by\_interaction\_type\_and\_data\_model\_and\_resource}}}{\emph{effect=None}, \emph{resources=None}, \emph{data\_model=None}, \emph{interaction\_type=None}, \emph{via=None}, \emph{references=None}}{}
Retrieves the entities involved in interactions matching the criteria.
It either returns both interacting entities in a \sphinxstyleemphasis{set} or an empty
\sphinxstyleemphasis{set}. This may not sound so useful at the level of this object but
becomes more useful once we want to collect entities having certain
kind of interactions across a series of \sphinxtitleref{Interaction} objects.
\begin{quote}\begin{description}
\item[{Parameters}] \leavevmode\begin{itemize}
\item {} 
\sphinxstyleliteralstrong{\sphinxupquote{entity\_type}} (\sphinxstyleliteralemphasis{\sphinxupquote{str}}) \textendash{} The type of the molecular entity. Possible values: \sphinxtitleref{protein},
\sphinxtitleref{complex}, \sphinxtitleref{mirna}, \sphinxtitleref{small\_molecule}.

\item {} 
\sphinxstyleliteralstrong{\sphinxupquote{return\_type}} (\sphinxstyleliteralemphasis{\sphinxupquote{str}}) \textendash{} The type of values to return. Default is
py:class:\sphinxcode{\sphinxupquote{pypath.entity.Entity}} objects, alternatives are
\sphinxcode{\sphinxupquote{labels}}  \sphinxcode{\sphinxupquote{identifiers}}.

\end{itemize}

\end{description}\end{quote}

\end{fulllineitems}

\index{complex\_identifiers\_by\_reference() (pypath.core.interaction.Interaction method)@\spxentry{complex\_identifiers\_by\_reference()}\spxextra{pypath.core.interaction.Interaction method}}

\begin{fulllineitems}
\phantomsection\label{\detokenize{reference:pypath.core.interaction.Interaction.complex_identifiers_by_reference}}\pysiglinewithargsret{\sphinxbfcode{\sphinxupquote{complex\_identifiers\_by\_reference}}}{\emph{effect=None}, \emph{resources=None}, \emph{data\_model=None}, \emph{interaction\_type=None}, \emph{via=None}, \emph{references=None}}{}
Retrieves the entities involved in interactions matching the criteria.
It either returns both interacting entities in a \sphinxstyleemphasis{set} or an empty
\sphinxstyleemphasis{set}. This may not sound so useful at the level of this object but
becomes more useful once we want to collect entities having certain
kind of interactions across a series of \sphinxtitleref{Interaction} objects.
\begin{quote}\begin{description}
\item[{Parameters}] \leavevmode\begin{itemize}
\item {} 
\sphinxstyleliteralstrong{\sphinxupquote{entity\_type}} (\sphinxstyleliteralemphasis{\sphinxupquote{str}}) \textendash{} The type of the molecular entity. Possible values: \sphinxtitleref{protein},
\sphinxtitleref{complex}, \sphinxtitleref{mirna}, \sphinxtitleref{small\_molecule}.

\item {} 
\sphinxstyleliteralstrong{\sphinxupquote{return\_type}} (\sphinxstyleliteralemphasis{\sphinxupquote{str}}) \textendash{} The type of values to return. Default is
py:class:\sphinxcode{\sphinxupquote{pypath.entity.Entity}} objects, alternatives are
\sphinxcode{\sphinxupquote{labels}}  \sphinxcode{\sphinxupquote{identifiers}}.

\end{itemize}

\end{description}\end{quote}

\end{fulllineitems}

\index{complex\_identifiers\_by\_resource() (pypath.core.interaction.Interaction method)@\spxentry{complex\_identifiers\_by\_resource()}\spxextra{pypath.core.interaction.Interaction method}}

\begin{fulllineitems}
\phantomsection\label{\detokenize{reference:pypath.core.interaction.Interaction.complex_identifiers_by_resource}}\pysiglinewithargsret{\sphinxbfcode{\sphinxupquote{complex\_identifiers\_by\_resource}}}{\emph{effect=None}, \emph{resources=None}, \emph{data\_model=None}, \emph{interaction\_type=None}, \emph{via=None}, \emph{references=None}}{}
Retrieves the entities involved in interactions matching the criteria.
It either returns both interacting entities in a \sphinxstyleemphasis{set} or an empty
\sphinxstyleemphasis{set}. This may not sound so useful at the level of this object but
becomes more useful once we want to collect entities having certain
kind of interactions across a series of \sphinxtitleref{Interaction} objects.
\begin{quote}\begin{description}
\item[{Parameters}] \leavevmode\begin{itemize}
\item {} 
\sphinxstyleliteralstrong{\sphinxupquote{entity\_type}} (\sphinxstyleliteralemphasis{\sphinxupquote{str}}) \textendash{} The type of the molecular entity. Possible values: \sphinxtitleref{protein},
\sphinxtitleref{complex}, \sphinxtitleref{mirna}, \sphinxtitleref{small\_molecule}.

\item {} 
\sphinxstyleliteralstrong{\sphinxupquote{return\_type}} (\sphinxstyleliteralemphasis{\sphinxupquote{str}}) \textendash{} The type of values to return. Default is
py:class:\sphinxcode{\sphinxupquote{pypath.entity.Entity}} objects, alternatives are
\sphinxcode{\sphinxupquote{labels}}  \sphinxcode{\sphinxupquote{identifiers}}.

\end{itemize}

\end{description}\end{quote}

\end{fulllineitems}

\index{complex\_labels\_by\_data\_model() (pypath.core.interaction.Interaction method)@\spxentry{complex\_labels\_by\_data\_model()}\spxextra{pypath.core.interaction.Interaction method}}

\begin{fulllineitems}
\phantomsection\label{\detokenize{reference:pypath.core.interaction.Interaction.complex_labels_by_data_model}}\pysiglinewithargsret{\sphinxbfcode{\sphinxupquote{complex\_labels\_by\_data\_model}}}{\emph{effect=None}, \emph{resources=None}, \emph{data\_model=None}, \emph{interaction\_type=None}, \emph{via=None}, \emph{references=None}}{}
Retrieves the entities involved in interactions matching the criteria.
It either returns both interacting entities in a \sphinxstyleemphasis{set} or an empty
\sphinxstyleemphasis{set}. This may not sound so useful at the level of this object but
becomes more useful once we want to collect entities having certain
kind of interactions across a series of \sphinxtitleref{Interaction} objects.
\begin{quote}\begin{description}
\item[{Parameters}] \leavevmode\begin{itemize}
\item {} 
\sphinxstyleliteralstrong{\sphinxupquote{entity\_type}} (\sphinxstyleliteralemphasis{\sphinxupquote{str}}) \textendash{} The type of the molecular entity. Possible values: \sphinxtitleref{protein},
\sphinxtitleref{complex}, \sphinxtitleref{mirna}, \sphinxtitleref{small\_molecule}.

\item {} 
\sphinxstyleliteralstrong{\sphinxupquote{return\_type}} (\sphinxstyleliteralemphasis{\sphinxupquote{str}}) \textendash{} The type of values to return. Default is
py:class:\sphinxcode{\sphinxupquote{pypath.entity.Entity}} objects, alternatives are
\sphinxcode{\sphinxupquote{labels}}  \sphinxcode{\sphinxupquote{identifiers}}.

\end{itemize}

\end{description}\end{quote}

\end{fulllineitems}

\index{complex\_labels\_by\_interaction\_type() (pypath.core.interaction.Interaction method)@\spxentry{complex\_labels\_by\_interaction\_type()}\spxextra{pypath.core.interaction.Interaction method}}

\begin{fulllineitems}
\phantomsection\label{\detokenize{reference:pypath.core.interaction.Interaction.complex_labels_by_interaction_type}}\pysiglinewithargsret{\sphinxbfcode{\sphinxupquote{complex\_labels\_by\_interaction\_type}}}{\emph{effect=None}, \emph{resources=None}, \emph{data\_model=None}, \emph{interaction\_type=None}, \emph{via=None}, \emph{references=None}}{}
Retrieves the entities involved in interactions matching the criteria.
It either returns both interacting entities in a \sphinxstyleemphasis{set} or an empty
\sphinxstyleemphasis{set}. This may not sound so useful at the level of this object but
becomes more useful once we want to collect entities having certain
kind of interactions across a series of \sphinxtitleref{Interaction} objects.
\begin{quote}\begin{description}
\item[{Parameters}] \leavevmode\begin{itemize}
\item {} 
\sphinxstyleliteralstrong{\sphinxupquote{entity\_type}} (\sphinxstyleliteralemphasis{\sphinxupquote{str}}) \textendash{} The type of the molecular entity. Possible values: \sphinxtitleref{protein},
\sphinxtitleref{complex}, \sphinxtitleref{mirna}, \sphinxtitleref{small\_molecule}.

\item {} 
\sphinxstyleliteralstrong{\sphinxupquote{return\_type}} (\sphinxstyleliteralemphasis{\sphinxupquote{str}}) \textendash{} The type of values to return. Default is
py:class:\sphinxcode{\sphinxupquote{pypath.entity.Entity}} objects, alternatives are
\sphinxcode{\sphinxupquote{labels}}  \sphinxcode{\sphinxupquote{identifiers}}.

\end{itemize}

\end{description}\end{quote}

\end{fulllineitems}

\index{complex\_labels\_by\_interaction\_type\_and\_data\_model() (pypath.core.interaction.Interaction method)@\spxentry{complex\_labels\_by\_interaction\_type\_and\_data\_model()}\spxextra{pypath.core.interaction.Interaction method}}

\begin{fulllineitems}
\phantomsection\label{\detokenize{reference:pypath.core.interaction.Interaction.complex_labels_by_interaction_type_and_data_model}}\pysiglinewithargsret{\sphinxbfcode{\sphinxupquote{complex\_labels\_by\_interaction\_type\_and\_data\_model}}}{\emph{effect=None}, \emph{resources=None}, \emph{data\_model=None}, \emph{interaction\_type=None}, \emph{via=None}, \emph{references=None}}{}
Retrieves the entities involved in interactions matching the criteria.
It either returns both interacting entities in a \sphinxstyleemphasis{set} or an empty
\sphinxstyleemphasis{set}. This may not sound so useful at the level of this object but
becomes more useful once we want to collect entities having certain
kind of interactions across a series of \sphinxtitleref{Interaction} objects.
\begin{quote}\begin{description}
\item[{Parameters}] \leavevmode\begin{itemize}
\item {} 
\sphinxstyleliteralstrong{\sphinxupquote{entity\_type}} (\sphinxstyleliteralemphasis{\sphinxupquote{str}}) \textendash{} The type of the molecular entity. Possible values: \sphinxtitleref{protein},
\sphinxtitleref{complex}, \sphinxtitleref{mirna}, \sphinxtitleref{small\_molecule}.

\item {} 
\sphinxstyleliteralstrong{\sphinxupquote{return\_type}} (\sphinxstyleliteralemphasis{\sphinxupquote{str}}) \textendash{} The type of values to return. Default is
py:class:\sphinxcode{\sphinxupquote{pypath.entity.Entity}} objects, alternatives are
\sphinxcode{\sphinxupquote{labels}}  \sphinxcode{\sphinxupquote{identifiers}}.

\end{itemize}

\end{description}\end{quote}

\end{fulllineitems}

\index{complex\_labels\_by\_interaction\_type\_and\_data\_model\_and\_resource() (pypath.core.interaction.Interaction method)@\spxentry{complex\_labels\_by\_interaction\_type\_and\_data\_model\_and\_resource()}\spxextra{pypath.core.interaction.Interaction method}}

\begin{fulllineitems}
\phantomsection\label{\detokenize{reference:pypath.core.interaction.Interaction.complex_labels_by_interaction_type_and_data_model_and_resource}}\pysiglinewithargsret{\sphinxbfcode{\sphinxupquote{complex\_labels\_by\_interaction\_type\_and\_data\_model\_and\_resource}}}{\emph{effect=None}, \emph{resources=None}, \emph{data\_model=None}, \emph{interaction\_type=None}, \emph{via=None}, \emph{references=None}}{}
Retrieves the entities involved in interactions matching the criteria.
It either returns both interacting entities in a \sphinxstyleemphasis{set} or an empty
\sphinxstyleemphasis{set}. This may not sound so useful at the level of this object but
becomes more useful once we want to collect entities having certain
kind of interactions across a series of \sphinxtitleref{Interaction} objects.
\begin{quote}\begin{description}
\item[{Parameters}] \leavevmode\begin{itemize}
\item {} 
\sphinxstyleliteralstrong{\sphinxupquote{entity\_type}} (\sphinxstyleliteralemphasis{\sphinxupquote{str}}) \textendash{} The type of the molecular entity. Possible values: \sphinxtitleref{protein},
\sphinxtitleref{complex}, \sphinxtitleref{mirna}, \sphinxtitleref{small\_molecule}.

\item {} 
\sphinxstyleliteralstrong{\sphinxupquote{return\_type}} (\sphinxstyleliteralemphasis{\sphinxupquote{str}}) \textendash{} The type of values to return. Default is
py:class:\sphinxcode{\sphinxupquote{pypath.entity.Entity}} objects, alternatives are
\sphinxcode{\sphinxupquote{labels}}  \sphinxcode{\sphinxupquote{identifiers}}.

\end{itemize}

\end{description}\end{quote}

\end{fulllineitems}

\index{complex\_labels\_by\_reference() (pypath.core.interaction.Interaction method)@\spxentry{complex\_labels\_by\_reference()}\spxextra{pypath.core.interaction.Interaction method}}

\begin{fulllineitems}
\phantomsection\label{\detokenize{reference:pypath.core.interaction.Interaction.complex_labels_by_reference}}\pysiglinewithargsret{\sphinxbfcode{\sphinxupquote{complex\_labels\_by\_reference}}}{\emph{effect=None}, \emph{resources=None}, \emph{data\_model=None}, \emph{interaction\_type=None}, \emph{via=None}, \emph{references=None}}{}
Retrieves the entities involved in interactions matching the criteria.
It either returns both interacting entities in a \sphinxstyleemphasis{set} or an empty
\sphinxstyleemphasis{set}. This may not sound so useful at the level of this object but
becomes more useful once we want to collect entities having certain
kind of interactions across a series of \sphinxtitleref{Interaction} objects.
\begin{quote}\begin{description}
\item[{Parameters}] \leavevmode\begin{itemize}
\item {} 
\sphinxstyleliteralstrong{\sphinxupquote{entity\_type}} (\sphinxstyleliteralemphasis{\sphinxupquote{str}}) \textendash{} The type of the molecular entity. Possible values: \sphinxtitleref{protein},
\sphinxtitleref{complex}, \sphinxtitleref{mirna}, \sphinxtitleref{small\_molecule}.

\item {} 
\sphinxstyleliteralstrong{\sphinxupquote{return\_type}} (\sphinxstyleliteralemphasis{\sphinxupquote{str}}) \textendash{} The type of values to return. Default is
py:class:\sphinxcode{\sphinxupquote{pypath.entity.Entity}} objects, alternatives are
\sphinxcode{\sphinxupquote{labels}}  \sphinxcode{\sphinxupquote{identifiers}}.

\end{itemize}

\end{description}\end{quote}

\end{fulllineitems}

\index{complex\_labels\_by\_resource() (pypath.core.interaction.Interaction method)@\spxentry{complex\_labels\_by\_resource()}\spxextra{pypath.core.interaction.Interaction method}}

\begin{fulllineitems}
\phantomsection\label{\detokenize{reference:pypath.core.interaction.Interaction.complex_labels_by_resource}}\pysiglinewithargsret{\sphinxbfcode{\sphinxupquote{complex\_labels\_by\_resource}}}{\emph{effect=None}, \emph{resources=None}, \emph{data\_model=None}, \emph{interaction\_type=None}, \emph{via=None}, \emph{references=None}}{}
Retrieves the entities involved in interactions matching the criteria.
It either returns both interacting entities in a \sphinxstyleemphasis{set} or an empty
\sphinxstyleemphasis{set}. This may not sound so useful at the level of this object but
becomes more useful once we want to collect entities having certain
kind of interactions across a series of \sphinxtitleref{Interaction} objects.
\begin{quote}\begin{description}
\item[{Parameters}] \leavevmode\begin{itemize}
\item {} 
\sphinxstyleliteralstrong{\sphinxupquote{entity\_type}} (\sphinxstyleliteralemphasis{\sphinxupquote{str}}) \textendash{} The type of the molecular entity. Possible values: \sphinxtitleref{protein},
\sphinxtitleref{complex}, \sphinxtitleref{mirna}, \sphinxtitleref{small\_molecule}.

\item {} 
\sphinxstyleliteralstrong{\sphinxupquote{return\_type}} (\sphinxstyleliteralemphasis{\sphinxupquote{str}}) \textendash{} The type of values to return. Default is
py:class:\sphinxcode{\sphinxupquote{pypath.entity.Entity}} objects, alternatives are
\sphinxcode{\sphinxupquote{labels}}  \sphinxcode{\sphinxupquote{identifiers}}.

\end{itemize}

\end{description}\end{quote}

\end{fulllineitems}

\index{complexes\_by\_data\_model() (pypath.core.interaction.Interaction method)@\spxentry{complexes\_by\_data\_model()}\spxextra{pypath.core.interaction.Interaction method}}

\begin{fulllineitems}
\phantomsection\label{\detokenize{reference:pypath.core.interaction.Interaction.complexes_by_data_model}}\pysiglinewithargsret{\sphinxbfcode{\sphinxupquote{complexes\_by\_data\_model}}}{\emph{effect=None}, \emph{resources=None}, \emph{data\_model=None}, \emph{interaction\_type=None}, \emph{via=None}, \emph{references=None}}{}
Retrieves the entities involved in interactions matching the criteria.
It either returns both interacting entities in a \sphinxstyleemphasis{set} or an empty
\sphinxstyleemphasis{set}. This may not sound so useful at the level of this object but
becomes more useful once we want to collect entities having certain
kind of interactions across a series of \sphinxtitleref{Interaction} objects.
\begin{quote}\begin{description}
\item[{Parameters}] \leavevmode\begin{itemize}
\item {} 
\sphinxstyleliteralstrong{\sphinxupquote{entity\_type}} (\sphinxstyleliteralemphasis{\sphinxupquote{str}}) \textendash{} The type of the molecular entity. Possible values: \sphinxtitleref{protein},
\sphinxtitleref{complex}, \sphinxtitleref{mirna}, \sphinxtitleref{small\_molecule}.

\item {} 
\sphinxstyleliteralstrong{\sphinxupquote{return\_type}} (\sphinxstyleliteralemphasis{\sphinxupquote{str}}) \textendash{} The type of values to return. Default is
py:class:\sphinxcode{\sphinxupquote{pypath.entity.Entity}} objects, alternatives are
\sphinxcode{\sphinxupquote{labels}}  \sphinxcode{\sphinxupquote{identifiers}}.

\end{itemize}

\end{description}\end{quote}

\end{fulllineitems}

\index{complexes\_by\_interaction\_type() (pypath.core.interaction.Interaction method)@\spxentry{complexes\_by\_interaction\_type()}\spxextra{pypath.core.interaction.Interaction method}}

\begin{fulllineitems}
\phantomsection\label{\detokenize{reference:pypath.core.interaction.Interaction.complexes_by_interaction_type}}\pysiglinewithargsret{\sphinxbfcode{\sphinxupquote{complexes\_by\_interaction\_type}}}{\emph{effect=None}, \emph{resources=None}, \emph{data\_model=None}, \emph{interaction\_type=None}, \emph{via=None}, \emph{references=None}}{}
Retrieves the entities involved in interactions matching the criteria.
It either returns both interacting entities in a \sphinxstyleemphasis{set} or an empty
\sphinxstyleemphasis{set}. This may not sound so useful at the level of this object but
becomes more useful once we want to collect entities having certain
kind of interactions across a series of \sphinxtitleref{Interaction} objects.
\begin{quote}\begin{description}
\item[{Parameters}] \leavevmode\begin{itemize}
\item {} 
\sphinxstyleliteralstrong{\sphinxupquote{entity\_type}} (\sphinxstyleliteralemphasis{\sphinxupquote{str}}) \textendash{} The type of the molecular entity. Possible values: \sphinxtitleref{protein},
\sphinxtitleref{complex}, \sphinxtitleref{mirna}, \sphinxtitleref{small\_molecule}.

\item {} 
\sphinxstyleliteralstrong{\sphinxupquote{return\_type}} (\sphinxstyleliteralemphasis{\sphinxupquote{str}}) \textendash{} The type of values to return. Default is
py:class:\sphinxcode{\sphinxupquote{pypath.entity.Entity}} objects, alternatives are
\sphinxcode{\sphinxupquote{labels}}  \sphinxcode{\sphinxupquote{identifiers}}.

\end{itemize}

\end{description}\end{quote}

\end{fulllineitems}

\index{complexes\_by\_interaction\_type\_and\_data\_model() (pypath.core.interaction.Interaction method)@\spxentry{complexes\_by\_interaction\_type\_and\_data\_model()}\spxextra{pypath.core.interaction.Interaction method}}

\begin{fulllineitems}
\phantomsection\label{\detokenize{reference:pypath.core.interaction.Interaction.complexes_by_interaction_type_and_data_model}}\pysiglinewithargsret{\sphinxbfcode{\sphinxupquote{complexes\_by\_interaction\_type\_and\_data\_model}}}{\emph{effect=None}, \emph{resources=None}, \emph{data\_model=None}, \emph{interaction\_type=None}, \emph{via=None}, \emph{references=None}}{}
Retrieves the entities involved in interactions matching the criteria.
It either returns both interacting entities in a \sphinxstyleemphasis{set} or an empty
\sphinxstyleemphasis{set}. This may not sound so useful at the level of this object but
becomes more useful once we want to collect entities having certain
kind of interactions across a series of \sphinxtitleref{Interaction} objects.
\begin{quote}\begin{description}
\item[{Parameters}] \leavevmode\begin{itemize}
\item {} 
\sphinxstyleliteralstrong{\sphinxupquote{entity\_type}} (\sphinxstyleliteralemphasis{\sphinxupquote{str}}) \textendash{} The type of the molecular entity. Possible values: \sphinxtitleref{protein},
\sphinxtitleref{complex}, \sphinxtitleref{mirna}, \sphinxtitleref{small\_molecule}.

\item {} 
\sphinxstyleliteralstrong{\sphinxupquote{return\_type}} (\sphinxstyleliteralemphasis{\sphinxupquote{str}}) \textendash{} The type of values to return. Default is
py:class:\sphinxcode{\sphinxupquote{pypath.entity.Entity}} objects, alternatives are
\sphinxcode{\sphinxupquote{labels}}  \sphinxcode{\sphinxupquote{identifiers}}.

\end{itemize}

\end{description}\end{quote}

\end{fulllineitems}

\index{complexes\_by\_interaction\_type\_and\_data\_model\_and\_resource() (pypath.core.interaction.Interaction method)@\spxentry{complexes\_by\_interaction\_type\_and\_data\_model\_and\_resource()}\spxextra{pypath.core.interaction.Interaction method}}

\begin{fulllineitems}
\phantomsection\label{\detokenize{reference:pypath.core.interaction.Interaction.complexes_by_interaction_type_and_data_model_and_resource}}\pysiglinewithargsret{\sphinxbfcode{\sphinxupquote{complexes\_by\_interaction\_type\_and\_data\_model\_and\_resource}}}{\emph{effect=None}, \emph{resources=None}, \emph{data\_model=None}, \emph{interaction\_type=None}, \emph{via=None}, \emph{references=None}}{}
Retrieves the entities involved in interactions matching the criteria.
It either returns both interacting entities in a \sphinxstyleemphasis{set} or an empty
\sphinxstyleemphasis{set}. This may not sound so useful at the level of this object but
becomes more useful once we want to collect entities having certain
kind of interactions across a series of \sphinxtitleref{Interaction} objects.
\begin{quote}\begin{description}
\item[{Parameters}] \leavevmode\begin{itemize}
\item {} 
\sphinxstyleliteralstrong{\sphinxupquote{entity\_type}} (\sphinxstyleliteralemphasis{\sphinxupquote{str}}) \textendash{} The type of the molecular entity. Possible values: \sphinxtitleref{protein},
\sphinxtitleref{complex}, \sphinxtitleref{mirna}, \sphinxtitleref{small\_molecule}.

\item {} 
\sphinxstyleliteralstrong{\sphinxupquote{return\_type}} (\sphinxstyleliteralemphasis{\sphinxupquote{str}}) \textendash{} The type of values to return. Default is
py:class:\sphinxcode{\sphinxupquote{pypath.entity.Entity}} objects, alternatives are
\sphinxcode{\sphinxupquote{labels}}  \sphinxcode{\sphinxupquote{identifiers}}.

\end{itemize}

\end{description}\end{quote}

\end{fulllineitems}

\index{complexes\_by\_reference() (pypath.core.interaction.Interaction method)@\spxentry{complexes\_by\_reference()}\spxextra{pypath.core.interaction.Interaction method}}

\begin{fulllineitems}
\phantomsection\label{\detokenize{reference:pypath.core.interaction.Interaction.complexes_by_reference}}\pysiglinewithargsret{\sphinxbfcode{\sphinxupquote{complexes\_by\_reference}}}{\emph{effect=None}, \emph{resources=None}, \emph{data\_model=None}, \emph{interaction\_type=None}, \emph{via=None}, \emph{references=None}}{}
Retrieves the entities involved in interactions matching the criteria.
It either returns both interacting entities in a \sphinxstyleemphasis{set} or an empty
\sphinxstyleemphasis{set}. This may not sound so useful at the level of this object but
becomes more useful once we want to collect entities having certain
kind of interactions across a series of \sphinxtitleref{Interaction} objects.
\begin{quote}\begin{description}
\item[{Parameters}] \leavevmode\begin{itemize}
\item {} 
\sphinxstyleliteralstrong{\sphinxupquote{entity\_type}} (\sphinxstyleliteralemphasis{\sphinxupquote{str}}) \textendash{} The type of the molecular entity. Possible values: \sphinxtitleref{protein},
\sphinxtitleref{complex}, \sphinxtitleref{mirna}, \sphinxtitleref{small\_molecule}.

\item {} 
\sphinxstyleliteralstrong{\sphinxupquote{return\_type}} (\sphinxstyleliteralemphasis{\sphinxupquote{str}}) \textendash{} The type of values to return. Default is
py:class:\sphinxcode{\sphinxupquote{pypath.entity.Entity}} objects, alternatives are
\sphinxcode{\sphinxupquote{labels}}  \sphinxcode{\sphinxupquote{identifiers}}.

\end{itemize}

\end{description}\end{quote}

\end{fulllineitems}

\index{complexes\_by\_resource() (pypath.core.interaction.Interaction method)@\spxentry{complexes\_by\_resource()}\spxextra{pypath.core.interaction.Interaction method}}

\begin{fulllineitems}
\phantomsection\label{\detokenize{reference:pypath.core.interaction.Interaction.complexes_by_resource}}\pysiglinewithargsret{\sphinxbfcode{\sphinxupquote{complexes\_by\_resource}}}{\emph{effect=None}, \emph{resources=None}, \emph{data\_model=None}, \emph{interaction\_type=None}, \emph{via=None}, \emph{references=None}}{}
Retrieves the entities involved in interactions matching the criteria.
It either returns both interacting entities in a \sphinxstyleemphasis{set} or an empty
\sphinxstyleemphasis{set}. This may not sound so useful at the level of this object but
becomes more useful once we want to collect entities having certain
kind of interactions across a series of \sphinxtitleref{Interaction} objects.
\begin{quote}\begin{description}
\item[{Parameters}] \leavevmode\begin{itemize}
\item {} 
\sphinxstyleliteralstrong{\sphinxupquote{entity\_type}} (\sphinxstyleliteralemphasis{\sphinxupquote{str}}) \textendash{} The type of the molecular entity. Possible values: \sphinxtitleref{protein},
\sphinxtitleref{complex}, \sphinxtitleref{mirna}, \sphinxtitleref{small\_molecule}.

\item {} 
\sphinxstyleliteralstrong{\sphinxupquote{return\_type}} (\sphinxstyleliteralemphasis{\sphinxupquote{str}}) \textendash{} The type of values to return. Default is
py:class:\sphinxcode{\sphinxupquote{pypath.entity.Entity}} objects, alternatives are
\sphinxcode{\sphinxupquote{labels}}  \sphinxcode{\sphinxupquote{identifiers}}.

\end{itemize}

\end{description}\end{quote}

\end{fulllineitems}

\index{consensus() (pypath.core.interaction.Interaction method)@\spxentry{consensus()}\spxextra{pypath.core.interaction.Interaction method}}

\begin{fulllineitems}
\phantomsection\label{\detokenize{reference:pypath.core.interaction.Interaction.consensus}}\pysiglinewithargsret{\sphinxbfcode{\sphinxupquote{consensus}}}{\emph{only\_interaction\_type=None}, \emph{only\_primary=False}, \emph{by\_references=False}, \emph{by\_reference\_resource\_pairs=True}}{}
Infers the consensus edge(s) according to the number of
supporting sources. This includes direction and sign.
\begin{quote}\begin{description}
\item[{Returns}] \leavevmode
(\sphinxstyleemphasis{list}) \textendash{} Contains the consensus edge(s) along with the
consensus sign. If there is no major directionality, both
are returned. The structure is as follows:
\sphinxcode{\sphinxupquote{{[}'\textless{}source\textgreater{}', '\textless{}target\textgreater{}', '\textless{}(un)directed\textgreater{}', '\textless{}sign\textgreater{}'{]}}}

\end{description}\end{quote}

\end{fulllineitems}

\index{consensus\_edges() (pypath.core.interaction.Interaction method)@\spxentry{consensus\_edges()}\spxextra{pypath.core.interaction.Interaction method}}

\begin{fulllineitems}
\phantomsection\label{\detokenize{reference:pypath.core.interaction.Interaction.consensus_edges}}\pysiglinewithargsret{\sphinxbfcode{\sphinxupquote{consensus\_edges}}}{\emph{only\_interaction\_type=None}, \emph{only\_primary=False}, \emph{by\_references=False}, \emph{by\_reference\_resource\_pairs=True}}{}
Infers the consensus edge(s) according to the number of
supporting sources. This includes direction and sign.
\begin{quote}\begin{description}
\item[{Returns}] \leavevmode
(\sphinxstyleemphasis{list}) \textendash{} Contains the consensus edge(s) along with the
consensus sign. If there is no major directionality, both
are returned. The structure is as follows:
\sphinxcode{\sphinxupquote{{[}'\textless{}source\textgreater{}', '\textless{}target\textgreater{}', '\textless{}(un)directed\textgreater{}', '\textless{}sign\textgreater{}'{]}}}

\end{description}\end{quote}

\end{fulllineitems}

\index{count\_complex\_identifiers() (pypath.core.interaction.Interaction method)@\spxentry{count\_complex\_identifiers()}\spxextra{pypath.core.interaction.Interaction method}}

\begin{fulllineitems}
\phantomsection\label{\detokenize{reference:pypath.core.interaction.Interaction.count_complex_identifiers}}\pysiglinewithargsret{\sphinxbfcode{\sphinxupquote{count\_complex\_identifiers}}}{\emph{effect=None}, \emph{resources=None}, \emph{data\_model=None}, \emph{interaction\_type=None}, \emph{via=None}, \emph{references=None}}{}
Retrieves the entities involved in interactions matching the criteria.
It either returns both interacting entities in a \sphinxstyleemphasis{set} or an empty
\sphinxstyleemphasis{set}. This may not sound so useful at the level of this object but
becomes more useful once we want to collect entities having certain
kind of interactions across a series of \sphinxtitleref{Interaction} objects.
\begin{quote}\begin{description}
\item[{Parameters}] \leavevmode\begin{itemize}
\item {} 
\sphinxstyleliteralstrong{\sphinxupquote{entity\_type}} (\sphinxstyleliteralemphasis{\sphinxupquote{str}}) \textendash{} The type of the molecular entity. Possible values: \sphinxtitleref{protein},
\sphinxtitleref{complex}, \sphinxtitleref{mirna}, \sphinxtitleref{small\_molecule}.

\item {} 
\sphinxstyleliteralstrong{\sphinxupquote{return\_type}} (\sphinxstyleliteralemphasis{\sphinxupquote{str}}) \textendash{} The type of values to return. Default is
py:class:\sphinxcode{\sphinxupquote{pypath.entity.Entity}} objects, alternatives are
\sphinxcode{\sphinxupquote{labels}}  \sphinxcode{\sphinxupquote{identifiers}}.

\end{itemize}

\end{description}\end{quote}

\end{fulllineitems}

\index{count\_complex\_labels() (pypath.core.interaction.Interaction method)@\spxentry{count\_complex\_labels()}\spxextra{pypath.core.interaction.Interaction method}}

\begin{fulllineitems}
\phantomsection\label{\detokenize{reference:pypath.core.interaction.Interaction.count_complex_labels}}\pysiglinewithargsret{\sphinxbfcode{\sphinxupquote{count\_complex\_labels}}}{\emph{effect=None}, \emph{resources=None}, \emph{data\_model=None}, \emph{interaction\_type=None}, \emph{via=None}, \emph{references=None}}{}
Retrieves the entities involved in interactions matching the criteria.
It either returns both interacting entities in a \sphinxstyleemphasis{set} or an empty
\sphinxstyleemphasis{set}. This may not sound so useful at the level of this object but
becomes more useful once we want to collect entities having certain
kind of interactions across a series of \sphinxtitleref{Interaction} objects.
\begin{quote}\begin{description}
\item[{Parameters}] \leavevmode\begin{itemize}
\item {} 
\sphinxstyleliteralstrong{\sphinxupquote{entity\_type}} (\sphinxstyleliteralemphasis{\sphinxupquote{str}}) \textendash{} The type of the molecular entity. Possible values: \sphinxtitleref{protein},
\sphinxtitleref{complex}, \sphinxtitleref{mirna}, \sphinxtitleref{small\_molecule}.

\item {} 
\sphinxstyleliteralstrong{\sphinxupquote{return\_type}} (\sphinxstyleliteralemphasis{\sphinxupquote{str}}) \textendash{} The type of values to return. Default is
py:class:\sphinxcode{\sphinxupquote{pypath.entity.Entity}} objects, alternatives are
\sphinxcode{\sphinxupquote{labels}}  \sphinxcode{\sphinxupquote{identifiers}}.

\end{itemize}

\end{description}\end{quote}

\end{fulllineitems}

\index{count\_complexes() (pypath.core.interaction.Interaction method)@\spxentry{count\_complexes()}\spxextra{pypath.core.interaction.Interaction method}}

\begin{fulllineitems}
\phantomsection\label{\detokenize{reference:pypath.core.interaction.Interaction.count_complexes}}\pysiglinewithargsret{\sphinxbfcode{\sphinxupquote{count\_complexes}}}{\emph{effect=None}, \emph{resources=None}, \emph{data\_model=None}, \emph{interaction\_type=None}, \emph{via=None}, \emph{references=None}}{}
Retrieves the entities involved in interactions matching the criteria.
It either returns both interacting entities in a \sphinxstyleemphasis{set} or an empty
\sphinxstyleemphasis{set}. This may not sound so useful at the level of this object but
becomes more useful once we want to collect entities having certain
kind of interactions across a series of \sphinxtitleref{Interaction} objects.
\begin{quote}\begin{description}
\item[{Parameters}] \leavevmode\begin{itemize}
\item {} 
\sphinxstyleliteralstrong{\sphinxupquote{entity\_type}} (\sphinxstyleliteralemphasis{\sphinxupquote{str}}) \textendash{} The type of the molecular entity. Possible values: \sphinxtitleref{protein},
\sphinxtitleref{complex}, \sphinxtitleref{mirna}, \sphinxtitleref{small\_molecule}.

\item {} 
\sphinxstyleliteralstrong{\sphinxupquote{return\_type}} (\sphinxstyleliteralemphasis{\sphinxupquote{str}}) \textendash{} The type of values to return. Default is
py:class:\sphinxcode{\sphinxupquote{pypath.entity.Entity}} objects, alternatives are
\sphinxcode{\sphinxupquote{labels}}  \sphinxcode{\sphinxupquote{identifiers}}.

\end{itemize}

\end{description}\end{quote}

\end{fulllineitems}

\index{count\_data\_models() (pypath.core.interaction.Interaction method)@\spxentry{count\_data\_models()}\spxextra{pypath.core.interaction.Interaction method}}

\begin{fulllineitems}
\phantomsection\label{\detokenize{reference:pypath.core.interaction.Interaction.count_data_models}}\pysiglinewithargsret{\sphinxbfcode{\sphinxupquote{count\_data\_models}}}{\emph{effect=None}, \emph{resources=None}, \emph{data\_model=None}, \emph{interaction\_type=None}, \emph{via=None}, \emph{references=None}}{}
Retrieves data models matching the criteria.

\end{fulllineitems}

\index{count\_degrees\_directed() (pypath.core.interaction.Interaction method)@\spxentry{count\_degrees\_directed()}\spxextra{pypath.core.interaction.Interaction method}}

\begin{fulllineitems}
\phantomsection\label{\detokenize{reference:pypath.core.interaction.Interaction.count_degrees_directed}}\pysiglinewithargsret{\sphinxbfcode{\sphinxupquote{count\_degrees\_directed}}}{\emph{effect=None}, \emph{resources=None}, \emph{data\_model=None}, \emph{interaction\_type=None}, \emph{via=None}, \emph{references=None}}{}
Returns a \sphinxstyleemphasis{set} of nodes with the connections matching the direction,
effect and evidence criteria. E.g. if the query concerns the incoming
degrees with positive effect and the matching evidences show A
activates B, but not the other way around, only “B” will be returned.
\begin{quote}\begin{description}
\item[{Parameters}] \leavevmode
\sphinxstyleliteralstrong{\sphinxupquote{mode}} (\sphinxstyleliteralemphasis{\sphinxupquote{str}}) \textendash{} The type of degrees to be considered. Three possible values are
\sphinxcode{\sphinxupquote{'IN'}}, \sphinxtitleref{‘OUT’{}`} and \sphinxcode{\sphinxupquote{'ALL'}} for incoming, outgoing and all
connections, respectively. If the \sphinxcode{\sphinxupquote{direction}} is \sphinxcode{\sphinxupquote{False}} the
only possible mode is \sphinxcode{\sphinxupquote{ALL}}. If the \sphinxcode{\sphinxupquote{direction}} is \sphinxcode{\sphinxupquote{None}}
and also directed evidence(s) match the criteria these will
overwrite the undirected evidences and only the directed result
will be returned.

\end{description}\end{quote}

\end{fulllineitems}

\index{count\_degrees\_directed\_in() (pypath.core.interaction.Interaction method)@\spxentry{count\_degrees\_directed\_in()}\spxextra{pypath.core.interaction.Interaction method}}

\begin{fulllineitems}
\phantomsection\label{\detokenize{reference:pypath.core.interaction.Interaction.count_degrees_directed_in}}\pysiglinewithargsret{\sphinxbfcode{\sphinxupquote{count\_degrees\_directed\_in}}}{\emph{effect=None}, \emph{resources=None}, \emph{data\_model=None}, \emph{interaction\_type=None}, \emph{via=None}, \emph{references=None}}{}
Returns a \sphinxstyleemphasis{set} of nodes with the connections matching the direction,
effect and evidence criteria. E.g. if the query concerns the incoming
degrees with positive effect and the matching evidences show A
activates B, but not the other way around, only “B” will be returned.
\begin{quote}\begin{description}
\item[{Parameters}] \leavevmode
\sphinxstyleliteralstrong{\sphinxupquote{mode}} (\sphinxstyleliteralemphasis{\sphinxupquote{str}}) \textendash{} The type of degrees to be considered. Three possible values are
\sphinxcode{\sphinxupquote{'IN'}}, \sphinxtitleref{‘OUT’{}`} and \sphinxcode{\sphinxupquote{'ALL'}} for incoming, outgoing and all
connections, respectively. If the \sphinxcode{\sphinxupquote{direction}} is \sphinxcode{\sphinxupquote{False}} the
only possible mode is \sphinxcode{\sphinxupquote{ALL}}. If the \sphinxcode{\sphinxupquote{direction}} is \sphinxcode{\sphinxupquote{None}}
and also directed evidence(s) match the criteria these will
overwrite the undirected evidences and only the directed result
will be returned.

\end{description}\end{quote}

\end{fulllineitems}

\index{count\_degrees\_directed\_out() (pypath.core.interaction.Interaction method)@\spxentry{count\_degrees\_directed\_out()}\spxextra{pypath.core.interaction.Interaction method}}

\begin{fulllineitems}
\phantomsection\label{\detokenize{reference:pypath.core.interaction.Interaction.count_degrees_directed_out}}\pysiglinewithargsret{\sphinxbfcode{\sphinxupquote{count\_degrees\_directed\_out}}}{\emph{effect=None}, \emph{resources=None}, \emph{data\_model=None}, \emph{interaction\_type=None}, \emph{via=None}, \emph{references=None}}{}
Returns a \sphinxstyleemphasis{set} of nodes with the connections matching the direction,
effect and evidence criteria. E.g. if the query concerns the incoming
degrees with positive effect and the matching evidences show A
activates B, but not the other way around, only “B” will be returned.
\begin{quote}\begin{description}
\item[{Parameters}] \leavevmode
\sphinxstyleliteralstrong{\sphinxupquote{mode}} (\sphinxstyleliteralemphasis{\sphinxupquote{str}}) \textendash{} The type of degrees to be considered. Three possible values are
\sphinxcode{\sphinxupquote{'IN'}}, \sphinxtitleref{‘OUT’{}`} and \sphinxcode{\sphinxupquote{'ALL'}} for incoming, outgoing and all
connections, respectively. If the \sphinxcode{\sphinxupquote{direction}} is \sphinxcode{\sphinxupquote{False}} the
only possible mode is \sphinxcode{\sphinxupquote{ALL}}. If the \sphinxcode{\sphinxupquote{direction}} is \sphinxcode{\sphinxupquote{None}}
and also directed evidence(s) match the criteria these will
overwrite the undirected evidences and only the directed result
will be returned.

\end{description}\end{quote}

\end{fulllineitems}

\index{count\_degrees\_negative() (pypath.core.interaction.Interaction method)@\spxentry{count\_degrees\_negative()}\spxextra{pypath.core.interaction.Interaction method}}

\begin{fulllineitems}
\phantomsection\label{\detokenize{reference:pypath.core.interaction.Interaction.count_degrees_negative}}\pysiglinewithargsret{\sphinxbfcode{\sphinxupquote{count\_degrees\_negative}}}{\emph{effect=None}, \emph{resources=None}, \emph{data\_model=None}, \emph{interaction\_type=None}, \emph{via=None}, \emph{references=None}}{}
Returns a \sphinxstyleemphasis{set} of nodes with the connections matching the direction,
effect and evidence criteria. E.g. if the query concerns the incoming
degrees with positive effect and the matching evidences show A
activates B, but not the other way around, only “B” will be returned.
\begin{quote}\begin{description}
\item[{Parameters}] \leavevmode
\sphinxstyleliteralstrong{\sphinxupquote{mode}} (\sphinxstyleliteralemphasis{\sphinxupquote{str}}) \textendash{} The type of degrees to be considered. Three possible values are
\sphinxcode{\sphinxupquote{'IN'}}, \sphinxtitleref{‘OUT’{}`} and \sphinxcode{\sphinxupquote{'ALL'}} for incoming, outgoing and all
connections, respectively. If the \sphinxcode{\sphinxupquote{direction}} is \sphinxcode{\sphinxupquote{False}} the
only possible mode is \sphinxcode{\sphinxupquote{ALL}}. If the \sphinxcode{\sphinxupquote{direction}} is \sphinxcode{\sphinxupquote{None}}
and also directed evidence(s) match the criteria these will
overwrite the undirected evidences and only the directed result
will be returned.

\end{description}\end{quote}

\end{fulllineitems}

\index{count\_degrees\_negative\_in() (pypath.core.interaction.Interaction method)@\spxentry{count\_degrees\_negative\_in()}\spxextra{pypath.core.interaction.Interaction method}}

\begin{fulllineitems}
\phantomsection\label{\detokenize{reference:pypath.core.interaction.Interaction.count_degrees_negative_in}}\pysiglinewithargsret{\sphinxbfcode{\sphinxupquote{count\_degrees\_negative\_in}}}{\emph{effect=None}, \emph{resources=None}, \emph{data\_model=None}, \emph{interaction\_type=None}, \emph{via=None}, \emph{references=None}}{}
Returns a \sphinxstyleemphasis{set} of nodes with the connections matching the direction,
effect and evidence criteria. E.g. if the query concerns the incoming
degrees with positive effect and the matching evidences show A
activates B, but not the other way around, only “B” will be returned.
\begin{quote}\begin{description}
\item[{Parameters}] \leavevmode
\sphinxstyleliteralstrong{\sphinxupquote{mode}} (\sphinxstyleliteralemphasis{\sphinxupquote{str}}) \textendash{} The type of degrees to be considered. Three possible values are
\sphinxcode{\sphinxupquote{'IN'}}, \sphinxtitleref{‘OUT’{}`} and \sphinxcode{\sphinxupquote{'ALL'}} for incoming, outgoing and all
connections, respectively. If the \sphinxcode{\sphinxupquote{direction}} is \sphinxcode{\sphinxupquote{False}} the
only possible mode is \sphinxcode{\sphinxupquote{ALL}}. If the \sphinxcode{\sphinxupquote{direction}} is \sphinxcode{\sphinxupquote{None}}
and also directed evidence(s) match the criteria these will
overwrite the undirected evidences and only the directed result
will be returned.

\end{description}\end{quote}

\end{fulllineitems}

\index{count\_degrees\_negative\_out() (pypath.core.interaction.Interaction method)@\spxentry{count\_degrees\_negative\_out()}\spxextra{pypath.core.interaction.Interaction method}}

\begin{fulllineitems}
\phantomsection\label{\detokenize{reference:pypath.core.interaction.Interaction.count_degrees_negative_out}}\pysiglinewithargsret{\sphinxbfcode{\sphinxupquote{count\_degrees\_negative\_out}}}{\emph{effect=None}, \emph{resources=None}, \emph{data\_model=None}, \emph{interaction\_type=None}, \emph{via=None}, \emph{references=None}}{}
Returns a \sphinxstyleemphasis{set} of nodes with the connections matching the direction,
effect and evidence criteria. E.g. if the query concerns the incoming
degrees with positive effect and the matching evidences show A
activates B, but not the other way around, only “B” will be returned.
\begin{quote}\begin{description}
\item[{Parameters}] \leavevmode
\sphinxstyleliteralstrong{\sphinxupquote{mode}} (\sphinxstyleliteralemphasis{\sphinxupquote{str}}) \textendash{} The type of degrees to be considered. Three possible values are
\sphinxcode{\sphinxupquote{'IN'}}, \sphinxtitleref{‘OUT’{}`} and \sphinxcode{\sphinxupquote{'ALL'}} for incoming, outgoing and all
connections, respectively. If the \sphinxcode{\sphinxupquote{direction}} is \sphinxcode{\sphinxupquote{False}} the
only possible mode is \sphinxcode{\sphinxupquote{ALL}}. If the \sphinxcode{\sphinxupquote{direction}} is \sphinxcode{\sphinxupquote{None}}
and also directed evidence(s) match the criteria these will
overwrite the undirected evidences and only the directed result
will be returned.

\end{description}\end{quote}

\end{fulllineitems}

\index{count\_degrees\_non\_directed() (pypath.core.interaction.Interaction method)@\spxentry{count\_degrees\_non\_directed()}\spxextra{pypath.core.interaction.Interaction method}}

\begin{fulllineitems}
\phantomsection\label{\detokenize{reference:pypath.core.interaction.Interaction.count_degrees_non_directed}}\pysiglinewithargsret{\sphinxbfcode{\sphinxupquote{count\_degrees\_non\_directed}}}{\emph{effect=None}, \emph{resources=None}, \emph{data\_model=None}, \emph{interaction\_type=None}, \emph{via=None}, \emph{references=None}}{}
Returns a \sphinxstyleemphasis{set} of nodes with the connections matching the direction,
effect and evidence criteria. E.g. if the query concerns the incoming
degrees with positive effect and the matching evidences show A
activates B, but not the other way around, only “B” will be returned.
\begin{quote}\begin{description}
\item[{Parameters}] \leavevmode
\sphinxstyleliteralstrong{\sphinxupquote{mode}} (\sphinxstyleliteralemphasis{\sphinxupquote{str}}) \textendash{} The type of degrees to be considered. Three possible values are
\sphinxcode{\sphinxupquote{'IN'}}, \sphinxtitleref{‘OUT’{}`} and \sphinxcode{\sphinxupquote{'ALL'}} for incoming, outgoing and all
connections, respectively. If the \sphinxcode{\sphinxupquote{direction}} is \sphinxcode{\sphinxupquote{False}} the
only possible mode is \sphinxcode{\sphinxupquote{ALL}}. If the \sphinxcode{\sphinxupquote{direction}} is \sphinxcode{\sphinxupquote{None}}
and also directed evidence(s) match the criteria these will
overwrite the undirected evidences and only the directed result
will be returned.

\end{description}\end{quote}

\end{fulllineitems}

\index{count\_degrees\_positive() (pypath.core.interaction.Interaction method)@\spxentry{count\_degrees\_positive()}\spxextra{pypath.core.interaction.Interaction method}}

\begin{fulllineitems}
\phantomsection\label{\detokenize{reference:pypath.core.interaction.Interaction.count_degrees_positive}}\pysiglinewithargsret{\sphinxbfcode{\sphinxupquote{count\_degrees\_positive}}}{\emph{effect=None}, \emph{resources=None}, \emph{data\_model=None}, \emph{interaction\_type=None}, \emph{via=None}, \emph{references=None}}{}
Returns a \sphinxstyleemphasis{set} of nodes with the connections matching the direction,
effect and evidence criteria. E.g. if the query concerns the incoming
degrees with positive effect and the matching evidences show A
activates B, but not the other way around, only “B” will be returned.
\begin{quote}\begin{description}
\item[{Parameters}] \leavevmode
\sphinxstyleliteralstrong{\sphinxupquote{mode}} (\sphinxstyleliteralemphasis{\sphinxupquote{str}}) \textendash{} The type of degrees to be considered. Three possible values are
\sphinxcode{\sphinxupquote{'IN'}}, \sphinxtitleref{‘OUT’{}`} and \sphinxcode{\sphinxupquote{'ALL'}} for incoming, outgoing and all
connections, respectively. If the \sphinxcode{\sphinxupquote{direction}} is \sphinxcode{\sphinxupquote{False}} the
only possible mode is \sphinxcode{\sphinxupquote{ALL}}. If the \sphinxcode{\sphinxupquote{direction}} is \sphinxcode{\sphinxupquote{None}}
and also directed evidence(s) match the criteria these will
overwrite the undirected evidences and only the directed result
will be returned.

\end{description}\end{quote}

\end{fulllineitems}

\index{count\_degrees\_positive\_in() (pypath.core.interaction.Interaction method)@\spxentry{count\_degrees\_positive\_in()}\spxextra{pypath.core.interaction.Interaction method}}

\begin{fulllineitems}
\phantomsection\label{\detokenize{reference:pypath.core.interaction.Interaction.count_degrees_positive_in}}\pysiglinewithargsret{\sphinxbfcode{\sphinxupquote{count\_degrees\_positive\_in}}}{\emph{effect=None}, \emph{resources=None}, \emph{data\_model=None}, \emph{interaction\_type=None}, \emph{via=None}, \emph{references=None}}{}
Returns a \sphinxstyleemphasis{set} of nodes with the connections matching the direction,
effect and evidence criteria. E.g. if the query concerns the incoming
degrees with positive effect and the matching evidences show A
activates B, but not the other way around, only “B” will be returned.
\begin{quote}\begin{description}
\item[{Parameters}] \leavevmode
\sphinxstyleliteralstrong{\sphinxupquote{mode}} (\sphinxstyleliteralemphasis{\sphinxupquote{str}}) \textendash{} The type of degrees to be considered. Three possible values are
\sphinxcode{\sphinxupquote{'IN'}}, \sphinxtitleref{‘OUT’{}`} and \sphinxcode{\sphinxupquote{'ALL'}} for incoming, outgoing and all
connections, respectively. If the \sphinxcode{\sphinxupquote{direction}} is \sphinxcode{\sphinxupquote{False}} the
only possible mode is \sphinxcode{\sphinxupquote{ALL}}. If the \sphinxcode{\sphinxupquote{direction}} is \sphinxcode{\sphinxupquote{None}}
and also directed evidence(s) match the criteria these will
overwrite the undirected evidences and only the directed result
will be returned.

\end{description}\end{quote}

\end{fulllineitems}

\index{count\_degrees\_positive\_out() (pypath.core.interaction.Interaction method)@\spxentry{count\_degrees\_positive\_out()}\spxextra{pypath.core.interaction.Interaction method}}

\begin{fulllineitems}
\phantomsection\label{\detokenize{reference:pypath.core.interaction.Interaction.count_degrees_positive_out}}\pysiglinewithargsret{\sphinxbfcode{\sphinxupquote{count\_degrees\_positive\_out}}}{\emph{effect=None}, \emph{resources=None}, \emph{data\_model=None}, \emph{interaction\_type=None}, \emph{via=None}, \emph{references=None}}{}
Returns a \sphinxstyleemphasis{set} of nodes with the connections matching the direction,
effect and evidence criteria. E.g. if the query concerns the incoming
degrees with positive effect and the matching evidences show A
activates B, but not the other way around, only “B” will be returned.
\begin{quote}\begin{description}
\item[{Parameters}] \leavevmode
\sphinxstyleliteralstrong{\sphinxupquote{mode}} (\sphinxstyleliteralemphasis{\sphinxupquote{str}}) \textendash{} The type of degrees to be considered. Three possible values are
\sphinxcode{\sphinxupquote{'IN'}}, \sphinxtitleref{‘OUT’{}`} and \sphinxcode{\sphinxupquote{'ALL'}} for incoming, outgoing and all
connections, respectively. If the \sphinxcode{\sphinxupquote{direction}} is \sphinxcode{\sphinxupquote{False}} the
only possible mode is \sphinxcode{\sphinxupquote{ALL}}. If the \sphinxcode{\sphinxupquote{direction}} is \sphinxcode{\sphinxupquote{None}}
and also directed evidence(s) match the criteria these will
overwrite the undirected evidences and only the directed result
will be returned.

\end{description}\end{quote}

\end{fulllineitems}

\index{count\_degrees\_signed() (pypath.core.interaction.Interaction method)@\spxentry{count\_degrees\_signed()}\spxextra{pypath.core.interaction.Interaction method}}

\begin{fulllineitems}
\phantomsection\label{\detokenize{reference:pypath.core.interaction.Interaction.count_degrees_signed}}\pysiglinewithargsret{\sphinxbfcode{\sphinxupquote{count\_degrees\_signed}}}{\emph{effect=None}, \emph{resources=None}, \emph{data\_model=None}, \emph{interaction\_type=None}, \emph{via=None}, \emph{references=None}}{}
Returns a \sphinxstyleemphasis{set} of nodes with the connections matching the direction,
effect and evidence criteria. E.g. if the query concerns the incoming
degrees with positive effect and the matching evidences show A
activates B, but not the other way around, only “B” will be returned.
\begin{quote}\begin{description}
\item[{Parameters}] \leavevmode
\sphinxstyleliteralstrong{\sphinxupquote{mode}} (\sphinxstyleliteralemphasis{\sphinxupquote{str}}) \textendash{} The type of degrees to be considered. Three possible values are
\sphinxcode{\sphinxupquote{'IN'}}, \sphinxtitleref{‘OUT’{}`} and \sphinxcode{\sphinxupquote{'ALL'}} for incoming, outgoing and all
connections, respectively. If the \sphinxcode{\sphinxupquote{direction}} is \sphinxcode{\sphinxupquote{False}} the
only possible mode is \sphinxcode{\sphinxupquote{ALL}}. If the \sphinxcode{\sphinxupquote{direction}} is \sphinxcode{\sphinxupquote{None}}
and also directed evidence(s) match the criteria these will
overwrite the undirected evidences and only the directed result
will be returned.

\end{description}\end{quote}

\end{fulllineitems}

\index{count\_degrees\_signed\_in() (pypath.core.interaction.Interaction method)@\spxentry{count\_degrees\_signed\_in()}\spxextra{pypath.core.interaction.Interaction method}}

\begin{fulllineitems}
\phantomsection\label{\detokenize{reference:pypath.core.interaction.Interaction.count_degrees_signed_in}}\pysiglinewithargsret{\sphinxbfcode{\sphinxupquote{count\_degrees\_signed\_in}}}{\emph{effect=None}, \emph{resources=None}, \emph{data\_model=None}, \emph{interaction\_type=None}, \emph{via=None}, \emph{references=None}}{}
Returns a \sphinxstyleemphasis{set} of nodes with the connections matching the direction,
effect and evidence criteria. E.g. if the query concerns the incoming
degrees with positive effect and the matching evidences show A
activates B, but not the other way around, only “B” will be returned.
\begin{quote}\begin{description}
\item[{Parameters}] \leavevmode
\sphinxstyleliteralstrong{\sphinxupquote{mode}} (\sphinxstyleliteralemphasis{\sphinxupquote{str}}) \textendash{} The type of degrees to be considered. Three possible values are
\sphinxcode{\sphinxupquote{'IN'}}, \sphinxtitleref{‘OUT’{}`} and \sphinxcode{\sphinxupquote{'ALL'}} for incoming, outgoing and all
connections, respectively. If the \sphinxcode{\sphinxupquote{direction}} is \sphinxcode{\sphinxupquote{False}} the
only possible mode is \sphinxcode{\sphinxupquote{ALL}}. If the \sphinxcode{\sphinxupquote{direction}} is \sphinxcode{\sphinxupquote{None}}
and also directed evidence(s) match the criteria these will
overwrite the undirected evidences and only the directed result
will be returned.

\end{description}\end{quote}

\end{fulllineitems}

\index{count\_degrees\_signed\_out() (pypath.core.interaction.Interaction method)@\spxentry{count\_degrees\_signed\_out()}\spxextra{pypath.core.interaction.Interaction method}}

\begin{fulllineitems}
\phantomsection\label{\detokenize{reference:pypath.core.interaction.Interaction.count_degrees_signed_out}}\pysiglinewithargsret{\sphinxbfcode{\sphinxupquote{count\_degrees\_signed\_out}}}{\emph{effect=None}, \emph{resources=None}, \emph{data\_model=None}, \emph{interaction\_type=None}, \emph{via=None}, \emph{references=None}}{}
Returns a \sphinxstyleemphasis{set} of nodes with the connections matching the direction,
effect and evidence criteria. E.g. if the query concerns the incoming
degrees with positive effect and the matching evidences show A
activates B, but not the other way around, only “B” will be returned.
\begin{quote}\begin{description}
\item[{Parameters}] \leavevmode
\sphinxstyleliteralstrong{\sphinxupquote{mode}} (\sphinxstyleliteralemphasis{\sphinxupquote{str}}) \textendash{} The type of degrees to be considered. Three possible values are
\sphinxcode{\sphinxupquote{'IN'}}, \sphinxtitleref{‘OUT’{}`} and \sphinxcode{\sphinxupquote{'ALL'}} for incoming, outgoing and all
connections, respectively. If the \sphinxcode{\sphinxupquote{direction}} is \sphinxcode{\sphinxupquote{False}} the
only possible mode is \sphinxcode{\sphinxupquote{ALL}}. If the \sphinxcode{\sphinxupquote{direction}} is \sphinxcode{\sphinxupquote{None}}
and also directed evidence(s) match the criteria these will
overwrite the undirected evidences and only the directed result
will be returned.

\end{description}\end{quote}

\end{fulllineitems}

\index{count\_degrees\_undirected() (pypath.core.interaction.Interaction method)@\spxentry{count\_degrees\_undirected()}\spxextra{pypath.core.interaction.Interaction method}}

\begin{fulllineitems}
\phantomsection\label{\detokenize{reference:pypath.core.interaction.Interaction.count_degrees_undirected}}\pysiglinewithargsret{\sphinxbfcode{\sphinxupquote{count\_degrees\_undirected}}}{\emph{effect=None}, \emph{resources=None}, \emph{data\_model=None}, \emph{interaction\_type=None}, \emph{via=None}, \emph{references=None}}{}
Returns a \sphinxstyleemphasis{set} of nodes with the connections matching the direction,
effect and evidence criteria. E.g. if the query concerns the incoming
degrees with positive effect and the matching evidences show A
activates B, but not the other way around, only “B” will be returned.
\begin{quote}\begin{description}
\item[{Parameters}] \leavevmode
\sphinxstyleliteralstrong{\sphinxupquote{mode}} (\sphinxstyleliteralemphasis{\sphinxupquote{str}}) \textendash{} The type of degrees to be considered. Three possible values are
\sphinxcode{\sphinxupquote{'IN'}}, \sphinxtitleref{‘OUT’{}`} and \sphinxcode{\sphinxupquote{'ALL'}} for incoming, outgoing and all
connections, respectively. If the \sphinxcode{\sphinxupquote{direction}} is \sphinxcode{\sphinxupquote{False}} the
only possible mode is \sphinxcode{\sphinxupquote{ALL}}. If the \sphinxcode{\sphinxupquote{direction}} is \sphinxcode{\sphinxupquote{None}}
and also directed evidence(s) match the criteria these will
overwrite the undirected evidences and only the directed result
will be returned.

\end{description}\end{quote}

\end{fulllineitems}

\index{count\_entities() (pypath.core.interaction.Interaction method)@\spxentry{count\_entities()}\spxextra{pypath.core.interaction.Interaction method}}

\begin{fulllineitems}
\phantomsection\label{\detokenize{reference:pypath.core.interaction.Interaction.count_entities}}\pysiglinewithargsret{\sphinxbfcode{\sphinxupquote{count\_entities}}}{\emph{effect=None}, \emph{resources=None}, \emph{data\_model=None}, \emph{interaction\_type=None}, \emph{via=None}, \emph{references=None}}{}
Retrieves the entities involved in interactions matching the criteria.
It either returns both interacting entities in a \sphinxstyleemphasis{set} or an empty
\sphinxstyleemphasis{set}. This may not sound so useful at the level of this object but
becomes more useful once we want to collect entities having certain
kind of interactions across a series of \sphinxtitleref{Interaction} objects.
\begin{quote}\begin{description}
\item[{Parameters}] \leavevmode\begin{itemize}
\item {} 
\sphinxstyleliteralstrong{\sphinxupquote{entity\_type}} (\sphinxstyleliteralemphasis{\sphinxupquote{str}}) \textendash{} The type of the molecular entity. Possible values: \sphinxtitleref{protein},
\sphinxtitleref{complex}, \sphinxtitleref{mirna}, \sphinxtitleref{small\_molecule}.

\item {} 
\sphinxstyleliteralstrong{\sphinxupquote{return\_type}} (\sphinxstyleliteralemphasis{\sphinxupquote{str}}) \textendash{} The type of values to return. Default is
py:class:\sphinxcode{\sphinxupquote{pypath.entity.Entity}} objects, alternatives are
\sphinxcode{\sphinxupquote{labels}}  \sphinxcode{\sphinxupquote{identifiers}}.

\end{itemize}

\end{description}\end{quote}

\end{fulllineitems}

\index{count\_identifiers() (pypath.core.interaction.Interaction method)@\spxentry{count\_identifiers()}\spxextra{pypath.core.interaction.Interaction method}}

\begin{fulllineitems}
\phantomsection\label{\detokenize{reference:pypath.core.interaction.Interaction.count_identifiers}}\pysiglinewithargsret{\sphinxbfcode{\sphinxupquote{count\_identifiers}}}{\emph{effect=None}, \emph{resources=None}, \emph{data\_model=None}, \emph{interaction\_type=None}, \emph{via=None}, \emph{references=None}}{}
Retrieves the entities involved in interactions matching the criteria.
It either returns both interacting entities in a \sphinxstyleemphasis{set} or an empty
\sphinxstyleemphasis{set}. This may not sound so useful at the level of this object but
becomes more useful once we want to collect entities having certain
kind of interactions across a series of \sphinxtitleref{Interaction} objects.
\begin{quote}\begin{description}
\item[{Parameters}] \leavevmode\begin{itemize}
\item {} 
\sphinxstyleliteralstrong{\sphinxupquote{entity\_type}} (\sphinxstyleliteralemphasis{\sphinxupquote{str}}) \textendash{} The type of the molecular entity. Possible values: \sphinxtitleref{protein},
\sphinxtitleref{complex}, \sphinxtitleref{mirna}, \sphinxtitleref{small\_molecule}.

\item {} 
\sphinxstyleliteralstrong{\sphinxupquote{return\_type}} (\sphinxstyleliteralemphasis{\sphinxupquote{str}}) \textendash{} The type of values to return. Default is
py:class:\sphinxcode{\sphinxupquote{pypath.entity.Entity}} objects, alternatives are
\sphinxcode{\sphinxupquote{labels}}  \sphinxcode{\sphinxupquote{identifiers}}.

\end{itemize}

\end{description}\end{quote}

\end{fulllineitems}

\index{count\_interaction\_types() (pypath.core.interaction.Interaction method)@\spxentry{count\_interaction\_types()}\spxextra{pypath.core.interaction.Interaction method}}

\begin{fulllineitems}
\phantomsection\label{\detokenize{reference:pypath.core.interaction.Interaction.count_interaction_types}}\pysiglinewithargsret{\sphinxbfcode{\sphinxupquote{count\_interaction\_types}}}{\emph{effect=None}, \emph{resources=None}, \emph{data\_model=None}, \emph{interaction\_type=None}, \emph{via=None}, \emph{references=None}}{}
Retrieves interaction types matching the criteria.

\end{fulllineitems}

\index{count\_interactions() (pypath.core.interaction.Interaction method)@\spxentry{count\_interactions()}\spxextra{pypath.core.interaction.Interaction method}}

\begin{fulllineitems}
\phantomsection\label{\detokenize{reference:pypath.core.interaction.Interaction.count_interactions}}\pysiglinewithargsret{\sphinxbfcode{\sphinxupquote{count\_interactions}}}{\emph{effect=None}, \emph{resources=None}, \emph{data\_model=None}, \emph{interaction\_type=None}, \emph{via=None}, \emph{references=None}}{}
Returns one or two tuples of the interacting partners: one if only
one direction, two if both directions match the query criteria.
The tuple will be empty if no evidence matches the criteria.
\begin{quote}\begin{description}
\item[{Parameters}] \leavevmode\begin{itemize}
\item {} 
\sphinxstyleliteralstrong{\sphinxupquote{direction}} (\sphinxstyleliteralemphasis{\sphinxupquote{NontType}}\sphinxstyleliteralemphasis{\sphinxupquote{,}}\sphinxstyleliteralemphasis{\sphinxupquote{bool}}\sphinxstyleliteralemphasis{\sphinxupquote{,}}\sphinxstyleliteralemphasis{\sphinxupquote{tuple}}) \textendash{} If \sphinxtitleref{None} both undirected and directed, if \sphinxtitleref{True} only directed,
if a \sphinxstyleemphasis{tuple} of entities only the interactions with that specific
direction will be considered. Unless you set this parameter to
\sphinxtitleref{True} this method will return both directions if one or more
undirected resources present.
If \sphinxtitleref{False}, only the undirected interactions will be considered,
and if any resource annotates this interaction as undirected
both directions will be returned. However the
\sphinxcode{\sphinxupquote{count\_interactions\_undirected}} method will return \sphinxtitleref{1}
in this case.

\item {} 
\sphinxstyleliteralstrong{\sphinxupquote{effect}} (\sphinxstyleliteralemphasis{\sphinxupquote{NoneType}}\sphinxstyleliteralemphasis{\sphinxupquote{,}}\sphinxstyleliteralemphasis{\sphinxupquote{bool}}\sphinxstyleliteralemphasis{\sphinxupquote{,}}\sphinxstyleliteralemphasis{\sphinxupquote{str}}) \textendash{} If \sphinxtitleref{None} also interactions without effect, if \sphinxtitleref{True} only
the ones with any effect, if a string naming an effect only the
interactions with that specific effect will be considered.

\item {} 
\sphinxstyleliteralstrong{\sphinxupquote{resources}} (\sphinxstyleliteralemphasis{\sphinxupquote{NontType}}\sphinxstyleliteralemphasis{\sphinxupquote{,}}\sphinxstyleliteralemphasis{\sphinxupquote{str}}\sphinxstyleliteralemphasis{\sphinxupquote{,}}\sphinxstyleliteralemphasis{\sphinxupquote{set}}) \textendash{} Optionally limit the query to one or more resources.

\item {} 
\sphinxstyleliteralstrong{\sphinxupquote{data\_model}} (\sphinxstyleliteralemphasis{\sphinxupquote{NontType}}\sphinxstyleliteralemphasis{\sphinxupquote{,}}\sphinxstyleliteralemphasis{\sphinxupquote{str}}\sphinxstyleliteralemphasis{\sphinxupquote{,}}\sphinxstyleliteralemphasis{\sphinxupquote{set}}) \textendash{} Optionally limit the query to one or more data models e.g.
\sphinxtitleref{activity\_flow}.

\item {} 
\sphinxstyleliteralstrong{\sphinxupquote{interaction\_type}} (\sphinxstyleliteralemphasis{\sphinxupquote{NontType}}\sphinxstyleliteralemphasis{\sphinxupquote{,}}\sphinxstyleliteralemphasis{\sphinxupquote{str}}\sphinxstyleliteralemphasis{\sphinxupquote{,}}\sphinxstyleliteralemphasis{\sphinxupquote{set}}) \textendash{} Optionally limit the query to one or more interaction types
e.g. \sphinxtitleref{PPI}.

\item {} 
\sphinxstyleliteralstrong{\sphinxupquote{via}} (\sphinxstyleliteralemphasis{\sphinxupquote{NontType}}\sphinxstyleliteralemphasis{\sphinxupquote{,}}\sphinxstyleliteralemphasis{\sphinxupquote{bool}}\sphinxstyleliteralemphasis{\sphinxupquote{,}}\sphinxstyleliteralemphasis{\sphinxupquote{str}}\sphinxstyleliteralemphasis{\sphinxupquote{,}}\sphinxstyleliteralemphasis{\sphinxupquote{set}}) \textendash{} Optionally limit the query to certain secondary databases or
if \sphinxtitleref{False} consider only data from primary databases.

\item {} 
\sphinxstyleliteralstrong{\sphinxupquote{entity\_type}} (\sphinxstyleliteralemphasis{\sphinxupquote{str}}) \textendash{} Molecule type for both of the entities.

\item {} 
\sphinxstyleliteralstrong{\sphinxupquote{source\_entity\_type}} (\sphinxstyleliteralemphasis{\sphinxupquote{str}}) \textendash{} Molecule type for the source entity.

\item {} 
\sphinxstyleliteralstrong{\sphinxupquote{target\_entity\_type}} (\sphinxstyleliteralemphasis{\sphinxupquote{str}}) \textendash{} Molecule type for the target entity.

\end{itemize}

\end{description}\end{quote}

\end{fulllineitems}

\index{count\_interactions\_0() (pypath.core.interaction.Interaction method)@\spxentry{count\_interactions\_0()}\spxextra{pypath.core.interaction.Interaction method}}

\begin{fulllineitems}
\phantomsection\label{\detokenize{reference:pypath.core.interaction.Interaction.count_interactions_0}}\pysiglinewithargsret{\sphinxbfcode{\sphinxupquote{count\_interactions\_0}}}{\emph{effect=None}, \emph{resources=None}, \emph{data\_model=None}, \emph{interaction\_type=None}, \emph{via=None}, \emph{references=None}}{}
Returns unique interacting pairs without being aware of the direction.

\end{fulllineitems}

\index{count\_interactions\_directed() (pypath.core.interaction.Interaction method)@\spxentry{count\_interactions\_directed()}\spxextra{pypath.core.interaction.Interaction method}}

\begin{fulllineitems}
\phantomsection\label{\detokenize{reference:pypath.core.interaction.Interaction.count_interactions_directed}}\pysiglinewithargsret{\sphinxbfcode{\sphinxupquote{count\_interactions\_directed}}}{\emph{effect=None}, \emph{resources=None}, \emph{data\_model=None}, \emph{interaction\_type=None}, \emph{via=None}, \emph{references=None}}{}
{\color{red}\bfseries{}**}kwargs: see the docs of method \sphinxcode{\sphinxupquote{get\_interactions}}.

\end{fulllineitems}

\index{count\_interactions\_mutual() (pypath.core.interaction.Interaction method)@\spxentry{count\_interactions\_mutual()}\spxextra{pypath.core.interaction.Interaction method}}

\begin{fulllineitems}
\phantomsection\label{\detokenize{reference:pypath.core.interaction.Interaction.count_interactions_mutual}}\pysiglinewithargsret{\sphinxbfcode{\sphinxupquote{count\_interactions\_mutual}}}{\emph{**kwargs}}{}
Note: undirected interactions does not count as mutual but only
interactions with explicit direction information for both directions.

{\color{red}\bfseries{}**}kwargs: see the docs of method \sphinxcode{\sphinxupquote{get\_interactions}}.

\end{fulllineitems}

\index{count\_interactions\_negative() (pypath.core.interaction.Interaction method)@\spxentry{count\_interactions\_negative()}\spxextra{pypath.core.interaction.Interaction method}}

\begin{fulllineitems}
\phantomsection\label{\detokenize{reference:pypath.core.interaction.Interaction.count_interactions_negative}}\pysiglinewithargsret{\sphinxbfcode{\sphinxupquote{count\_interactions\_negative}}}{\emph{effect=None}, \emph{resources=None}, \emph{data\_model=None}, \emph{interaction\_type=None}, \emph{via=None}, \emph{references=None}}{}
{\color{red}\bfseries{}**}kwargs: see the docs of method \sphinxcode{\sphinxupquote{get\_interactions}}.

\end{fulllineitems}

\index{count\_interactions\_non\_directed() (pypath.core.interaction.Interaction method)@\spxentry{count\_interactions\_non\_directed()}\spxextra{pypath.core.interaction.Interaction method}}

\begin{fulllineitems}
\phantomsection\label{\detokenize{reference:pypath.core.interaction.Interaction.count_interactions_non_directed}}\pysiglinewithargsret{\sphinxbfcode{\sphinxupquote{count\_interactions\_non\_directed}}}{\emph{**kwargs}}{}
Returns \sphinxtitleref{True} if any resource annotates this interaction without
and no resource with direction.

{\color{red}\bfseries{}**}kwargs: see the docs of method \sphinxcode{\sphinxupquote{get\_interactions}}.

\end{fulllineitems}

\index{count\_interactions\_positive() (pypath.core.interaction.Interaction method)@\spxentry{count\_interactions\_positive()}\spxextra{pypath.core.interaction.Interaction method}}

\begin{fulllineitems}
\phantomsection\label{\detokenize{reference:pypath.core.interaction.Interaction.count_interactions_positive}}\pysiglinewithargsret{\sphinxbfcode{\sphinxupquote{count\_interactions\_positive}}}{\emph{effect=None}, \emph{resources=None}, \emph{data\_model=None}, \emph{interaction\_type=None}, \emph{via=None}, \emph{references=None}}{}
{\color{red}\bfseries{}**}kwargs: see the docs of method \sphinxcode{\sphinxupquote{get\_interactions}}.

\end{fulllineitems}

\index{count\_interactions\_signed() (pypath.core.interaction.Interaction method)@\spxentry{count\_interactions\_signed()}\spxextra{pypath.core.interaction.Interaction method}}

\begin{fulllineitems}
\phantomsection\label{\detokenize{reference:pypath.core.interaction.Interaction.count_interactions_signed}}\pysiglinewithargsret{\sphinxbfcode{\sphinxupquote{count\_interactions\_signed}}}{\emph{effect=None}, \emph{resources=None}, \emph{data\_model=None}, \emph{interaction\_type=None}, \emph{via=None}, \emph{references=None}}{}
{\color{red}\bfseries{}**}kwargs: see the docs of method \sphinxcode{\sphinxupquote{get\_interactions}}.

\end{fulllineitems}

\index{count\_interactions\_undirected() (pypath.core.interaction.Interaction method)@\spxentry{count\_interactions\_undirected()}\spxextra{pypath.core.interaction.Interaction method}}

\begin{fulllineitems}
\phantomsection\label{\detokenize{reference:pypath.core.interaction.Interaction.count_interactions_undirected}}\pysiglinewithargsret{\sphinxbfcode{\sphinxupquote{count\_interactions\_undirected}}}{\emph{**kwargs}}{}
Returns \sphinxtitleref{True} if any resource annotates this interaction without
direction.

{\color{red}\bfseries{}**}kwargs: see the docs of method \sphinxcode{\sphinxupquote{get\_interactions}}.

\end{fulllineitems}

\index{count\_labels() (pypath.core.interaction.Interaction method)@\spxentry{count\_labels()}\spxextra{pypath.core.interaction.Interaction method}}

\begin{fulllineitems}
\phantomsection\label{\detokenize{reference:pypath.core.interaction.Interaction.count_labels}}\pysiglinewithargsret{\sphinxbfcode{\sphinxupquote{count\_labels}}}{\emph{effect=None}, \emph{resources=None}, \emph{data\_model=None}, \emph{interaction\_type=None}, \emph{via=None}, \emph{references=None}}{}
Retrieves the entities involved in interactions matching the criteria.
It either returns both interacting entities in a \sphinxstyleemphasis{set} or an empty
\sphinxstyleemphasis{set}. This may not sound so useful at the level of this object but
becomes more useful once we want to collect entities having certain
kind of interactions across a series of \sphinxtitleref{Interaction} objects.
\begin{quote}\begin{description}
\item[{Parameters}] \leavevmode\begin{itemize}
\item {} 
\sphinxstyleliteralstrong{\sphinxupquote{entity\_type}} (\sphinxstyleliteralemphasis{\sphinxupquote{str}}) \textendash{} The type of the molecular entity. Possible values: \sphinxtitleref{protein},
\sphinxtitleref{complex}, \sphinxtitleref{mirna}, \sphinxtitleref{small\_molecule}.

\item {} 
\sphinxstyleliteralstrong{\sphinxupquote{return\_type}} (\sphinxstyleliteralemphasis{\sphinxupquote{str}}) \textendash{} The type of values to return. Default is
py:class:\sphinxcode{\sphinxupquote{pypath.entity.Entity}} objects, alternatives are
\sphinxcode{\sphinxupquote{labels}}  \sphinxcode{\sphinxupquote{identifiers}}.

\end{itemize}

\end{description}\end{quote}

\end{fulllineitems}

\index{count\_lncrna\_identifiers() (pypath.core.interaction.Interaction method)@\spxentry{count\_lncrna\_identifiers()}\spxextra{pypath.core.interaction.Interaction method}}

\begin{fulllineitems}
\phantomsection\label{\detokenize{reference:pypath.core.interaction.Interaction.count_lncrna_identifiers}}\pysiglinewithargsret{\sphinxbfcode{\sphinxupquote{count\_lncrna\_identifiers}}}{\emph{effect=None}, \emph{resources=None}, \emph{data\_model=None}, \emph{interaction\_type=None}, \emph{via=None}, \emph{references=None}}{}
Retrieves the entities involved in interactions matching the criteria.
It either returns both interacting entities in a \sphinxstyleemphasis{set} or an empty
\sphinxstyleemphasis{set}. This may not sound so useful at the level of this object but
becomes more useful once we want to collect entities having certain
kind of interactions across a series of \sphinxtitleref{Interaction} objects.
\begin{quote}\begin{description}
\item[{Parameters}] \leavevmode\begin{itemize}
\item {} 
\sphinxstyleliteralstrong{\sphinxupquote{entity\_type}} (\sphinxstyleliteralemphasis{\sphinxupquote{str}}) \textendash{} The type of the molecular entity. Possible values: \sphinxtitleref{protein},
\sphinxtitleref{complex}, \sphinxtitleref{mirna}, \sphinxtitleref{small\_molecule}.

\item {} 
\sphinxstyleliteralstrong{\sphinxupquote{return\_type}} (\sphinxstyleliteralemphasis{\sphinxupquote{str}}) \textendash{} The type of values to return. Default is
py:class:\sphinxcode{\sphinxupquote{pypath.entity.Entity}} objects, alternatives are
\sphinxcode{\sphinxupquote{labels}}  \sphinxcode{\sphinxupquote{identifiers}}.

\end{itemize}

\end{description}\end{quote}

\end{fulllineitems}

\index{count\_lncrna\_labels() (pypath.core.interaction.Interaction method)@\spxentry{count\_lncrna\_labels()}\spxextra{pypath.core.interaction.Interaction method}}

\begin{fulllineitems}
\phantomsection\label{\detokenize{reference:pypath.core.interaction.Interaction.count_lncrna_labels}}\pysiglinewithargsret{\sphinxbfcode{\sphinxupquote{count\_lncrna\_labels}}}{\emph{effect=None}, \emph{resources=None}, \emph{data\_model=None}, \emph{interaction\_type=None}, \emph{via=None}, \emph{references=None}}{}
Retrieves the entities involved in interactions matching the criteria.
It either returns both interacting entities in a \sphinxstyleemphasis{set} or an empty
\sphinxstyleemphasis{set}. This may not sound so useful at the level of this object but
becomes more useful once we want to collect entities having certain
kind of interactions across a series of \sphinxtitleref{Interaction} objects.
\begin{quote}\begin{description}
\item[{Parameters}] \leavevmode\begin{itemize}
\item {} 
\sphinxstyleliteralstrong{\sphinxupquote{entity\_type}} (\sphinxstyleliteralemphasis{\sphinxupquote{str}}) \textendash{} The type of the molecular entity. Possible values: \sphinxtitleref{protein},
\sphinxtitleref{complex}, \sphinxtitleref{mirna}, \sphinxtitleref{small\_molecule}.

\item {} 
\sphinxstyleliteralstrong{\sphinxupquote{return\_type}} (\sphinxstyleliteralemphasis{\sphinxupquote{str}}) \textendash{} The type of values to return. Default is
py:class:\sphinxcode{\sphinxupquote{pypath.entity.Entity}} objects, alternatives are
\sphinxcode{\sphinxupquote{labels}}  \sphinxcode{\sphinxupquote{identifiers}}.

\end{itemize}

\end{description}\end{quote}

\end{fulllineitems}

\index{count\_lncrnas() (pypath.core.interaction.Interaction method)@\spxentry{count\_lncrnas()}\spxextra{pypath.core.interaction.Interaction method}}

\begin{fulllineitems}
\phantomsection\label{\detokenize{reference:pypath.core.interaction.Interaction.count_lncrnas}}\pysiglinewithargsret{\sphinxbfcode{\sphinxupquote{count\_lncrnas}}}{\emph{effect=None}, \emph{resources=None}, \emph{data\_model=None}, \emph{interaction\_type=None}, \emph{via=None}, \emph{references=None}}{}
Retrieves the entities involved in interactions matching the criteria.
It either returns both interacting entities in a \sphinxstyleemphasis{set} or an empty
\sphinxstyleemphasis{set}. This may not sound so useful at the level of this object but
becomes more useful once we want to collect entities having certain
kind of interactions across a series of \sphinxtitleref{Interaction} objects.
\begin{quote}\begin{description}
\item[{Parameters}] \leavevmode\begin{itemize}
\item {} 
\sphinxstyleliteralstrong{\sphinxupquote{entity\_type}} (\sphinxstyleliteralemphasis{\sphinxupquote{str}}) \textendash{} The type of the molecular entity. Possible values: \sphinxtitleref{protein},
\sphinxtitleref{complex}, \sphinxtitleref{mirna}, \sphinxtitleref{small\_molecule}.

\item {} 
\sphinxstyleliteralstrong{\sphinxupquote{return\_type}} (\sphinxstyleliteralemphasis{\sphinxupquote{str}}) \textendash{} The type of values to return. Default is
py:class:\sphinxcode{\sphinxupquote{pypath.entity.Entity}} objects, alternatives are
\sphinxcode{\sphinxupquote{labels}}  \sphinxcode{\sphinxupquote{identifiers}}.

\end{itemize}

\end{description}\end{quote}

\end{fulllineitems}

\index{count\_mirna\_identifiers() (pypath.core.interaction.Interaction method)@\spxentry{count\_mirna\_identifiers()}\spxextra{pypath.core.interaction.Interaction method}}

\begin{fulllineitems}
\phantomsection\label{\detokenize{reference:pypath.core.interaction.Interaction.count_mirna_identifiers}}\pysiglinewithargsret{\sphinxbfcode{\sphinxupquote{count\_mirna\_identifiers}}}{\emph{effect=None}, \emph{resources=None}, \emph{data\_model=None}, \emph{interaction\_type=None}, \emph{via=None}, \emph{references=None}}{}
Retrieves the entities involved in interactions matching the criteria.
It either returns both interacting entities in a \sphinxstyleemphasis{set} or an empty
\sphinxstyleemphasis{set}. This may not sound so useful at the level of this object but
becomes more useful once we want to collect entities having certain
kind of interactions across a series of \sphinxtitleref{Interaction} objects.
\begin{quote}\begin{description}
\item[{Parameters}] \leavevmode\begin{itemize}
\item {} 
\sphinxstyleliteralstrong{\sphinxupquote{entity\_type}} (\sphinxstyleliteralemphasis{\sphinxupquote{str}}) \textendash{} The type of the molecular entity. Possible values: \sphinxtitleref{protein},
\sphinxtitleref{complex}, \sphinxtitleref{mirna}, \sphinxtitleref{small\_molecule}.

\item {} 
\sphinxstyleliteralstrong{\sphinxupquote{return\_type}} (\sphinxstyleliteralemphasis{\sphinxupquote{str}}) \textendash{} The type of values to return. Default is
py:class:\sphinxcode{\sphinxupquote{pypath.entity.Entity}} objects, alternatives are
\sphinxcode{\sphinxupquote{labels}}  \sphinxcode{\sphinxupquote{identifiers}}.

\end{itemize}

\end{description}\end{quote}

\end{fulllineitems}

\index{count\_mirna\_labels() (pypath.core.interaction.Interaction method)@\spxentry{count\_mirna\_labels()}\spxextra{pypath.core.interaction.Interaction method}}

\begin{fulllineitems}
\phantomsection\label{\detokenize{reference:pypath.core.interaction.Interaction.count_mirna_labels}}\pysiglinewithargsret{\sphinxbfcode{\sphinxupquote{count\_mirna\_labels}}}{\emph{effect=None}, \emph{resources=None}, \emph{data\_model=None}, \emph{interaction\_type=None}, \emph{via=None}, \emph{references=None}}{}
Retrieves the entities involved in interactions matching the criteria.
It either returns both interacting entities in a \sphinxstyleemphasis{set} or an empty
\sphinxstyleemphasis{set}. This may not sound so useful at the level of this object but
becomes more useful once we want to collect entities having certain
kind of interactions across a series of \sphinxtitleref{Interaction} objects.
\begin{quote}\begin{description}
\item[{Parameters}] \leavevmode\begin{itemize}
\item {} 
\sphinxstyleliteralstrong{\sphinxupquote{entity\_type}} (\sphinxstyleliteralemphasis{\sphinxupquote{str}}) \textendash{} The type of the molecular entity. Possible values: \sphinxtitleref{protein},
\sphinxtitleref{complex}, \sphinxtitleref{mirna}, \sphinxtitleref{small\_molecule}.

\item {} 
\sphinxstyleliteralstrong{\sphinxupquote{return\_type}} (\sphinxstyleliteralemphasis{\sphinxupquote{str}}) \textendash{} The type of values to return. Default is
py:class:\sphinxcode{\sphinxupquote{pypath.entity.Entity}} objects, alternatives are
\sphinxcode{\sphinxupquote{labels}}  \sphinxcode{\sphinxupquote{identifiers}}.

\end{itemize}

\end{description}\end{quote}

\end{fulllineitems}

\index{count\_mirnas() (pypath.core.interaction.Interaction method)@\spxentry{count\_mirnas()}\spxextra{pypath.core.interaction.Interaction method}}

\begin{fulllineitems}
\phantomsection\label{\detokenize{reference:pypath.core.interaction.Interaction.count_mirnas}}\pysiglinewithargsret{\sphinxbfcode{\sphinxupquote{count\_mirnas}}}{\emph{effect=None}, \emph{resources=None}, \emph{data\_model=None}, \emph{interaction\_type=None}, \emph{via=None}, \emph{references=None}}{}
Retrieves the entities involved in interactions matching the criteria.
It either returns both interacting entities in a \sphinxstyleemphasis{set} or an empty
\sphinxstyleemphasis{set}. This may not sound so useful at the level of this object but
becomes more useful once we want to collect entities having certain
kind of interactions across a series of \sphinxtitleref{Interaction} objects.
\begin{quote}\begin{description}
\item[{Parameters}] \leavevmode\begin{itemize}
\item {} 
\sphinxstyleliteralstrong{\sphinxupquote{entity\_type}} (\sphinxstyleliteralemphasis{\sphinxupquote{str}}) \textendash{} The type of the molecular entity. Possible values: \sphinxtitleref{protein},
\sphinxtitleref{complex}, \sphinxtitleref{mirna}, \sphinxtitleref{small\_molecule}.

\item {} 
\sphinxstyleliteralstrong{\sphinxupquote{return\_type}} (\sphinxstyleliteralemphasis{\sphinxupquote{str}}) \textendash{} The type of values to return. Default is
py:class:\sphinxcode{\sphinxupquote{pypath.entity.Entity}} objects, alternatives are
\sphinxcode{\sphinxupquote{labels}}  \sphinxcode{\sphinxupquote{identifiers}}.

\end{itemize}

\end{description}\end{quote}

\end{fulllineitems}

\index{count\_protein\_identifiers() (pypath.core.interaction.Interaction method)@\spxentry{count\_protein\_identifiers()}\spxextra{pypath.core.interaction.Interaction method}}

\begin{fulllineitems}
\phantomsection\label{\detokenize{reference:pypath.core.interaction.Interaction.count_protein_identifiers}}\pysiglinewithargsret{\sphinxbfcode{\sphinxupquote{count\_protein\_identifiers}}}{\emph{effect=None}, \emph{resources=None}, \emph{data\_model=None}, \emph{interaction\_type=None}, \emph{via=None}, \emph{references=None}}{}
Retrieves the entities involved in interactions matching the criteria.
It either returns both interacting entities in a \sphinxstyleemphasis{set} or an empty
\sphinxstyleemphasis{set}. This may not sound so useful at the level of this object but
becomes more useful once we want to collect entities having certain
kind of interactions across a series of \sphinxtitleref{Interaction} objects.
\begin{quote}\begin{description}
\item[{Parameters}] \leavevmode\begin{itemize}
\item {} 
\sphinxstyleliteralstrong{\sphinxupquote{entity\_type}} (\sphinxstyleliteralemphasis{\sphinxupquote{str}}) \textendash{} The type of the molecular entity. Possible values: \sphinxtitleref{protein},
\sphinxtitleref{complex}, \sphinxtitleref{mirna}, \sphinxtitleref{small\_molecule}.

\item {} 
\sphinxstyleliteralstrong{\sphinxupquote{return\_type}} (\sphinxstyleliteralemphasis{\sphinxupquote{str}}) \textendash{} The type of values to return. Default is
py:class:\sphinxcode{\sphinxupquote{pypath.entity.Entity}} objects, alternatives are
\sphinxcode{\sphinxupquote{labels}}  \sphinxcode{\sphinxupquote{identifiers}}.

\end{itemize}

\end{description}\end{quote}

\end{fulllineitems}

\index{count\_protein\_labels() (pypath.core.interaction.Interaction method)@\spxentry{count\_protein\_labels()}\spxextra{pypath.core.interaction.Interaction method}}

\begin{fulllineitems}
\phantomsection\label{\detokenize{reference:pypath.core.interaction.Interaction.count_protein_labels}}\pysiglinewithargsret{\sphinxbfcode{\sphinxupquote{count\_protein\_labels}}}{\emph{effect=None}, \emph{resources=None}, \emph{data\_model=None}, \emph{interaction\_type=None}, \emph{via=None}, \emph{references=None}}{}
Retrieves the entities involved in interactions matching the criteria.
It either returns both interacting entities in a \sphinxstyleemphasis{set} or an empty
\sphinxstyleemphasis{set}. This may not sound so useful at the level of this object but
becomes more useful once we want to collect entities having certain
kind of interactions across a series of \sphinxtitleref{Interaction} objects.
\begin{quote}\begin{description}
\item[{Parameters}] \leavevmode\begin{itemize}
\item {} 
\sphinxstyleliteralstrong{\sphinxupquote{entity\_type}} (\sphinxstyleliteralemphasis{\sphinxupquote{str}}) \textendash{} The type of the molecular entity. Possible values: \sphinxtitleref{protein},
\sphinxtitleref{complex}, \sphinxtitleref{mirna}, \sphinxtitleref{small\_molecule}.

\item {} 
\sphinxstyleliteralstrong{\sphinxupquote{return\_type}} (\sphinxstyleliteralemphasis{\sphinxupquote{str}}) \textendash{} The type of values to return. Default is
py:class:\sphinxcode{\sphinxupquote{pypath.entity.Entity}} objects, alternatives are
\sphinxcode{\sphinxupquote{labels}}  \sphinxcode{\sphinxupquote{identifiers}}.

\end{itemize}

\end{description}\end{quote}

\end{fulllineitems}

\index{count\_proteins() (pypath.core.interaction.Interaction method)@\spxentry{count\_proteins()}\spxextra{pypath.core.interaction.Interaction method}}

\begin{fulllineitems}
\phantomsection\label{\detokenize{reference:pypath.core.interaction.Interaction.count_proteins}}\pysiglinewithargsret{\sphinxbfcode{\sphinxupquote{count\_proteins}}}{\emph{effect=None}, \emph{resources=None}, \emph{data\_model=None}, \emph{interaction\_type=None}, \emph{via=None}, \emph{references=None}}{}
Retrieves the entities involved in interactions matching the criteria.
It either returns both interacting entities in a \sphinxstyleemphasis{set} or an empty
\sphinxstyleemphasis{set}. This may not sound so useful at the level of this object but
becomes more useful once we want to collect entities having certain
kind of interactions across a series of \sphinxtitleref{Interaction} objects.
\begin{quote}\begin{description}
\item[{Parameters}] \leavevmode\begin{itemize}
\item {} 
\sphinxstyleliteralstrong{\sphinxupquote{entity\_type}} (\sphinxstyleliteralemphasis{\sphinxupquote{str}}) \textendash{} The type of the molecular entity. Possible values: \sphinxtitleref{protein},
\sphinxtitleref{complex}, \sphinxtitleref{mirna}, \sphinxtitleref{small\_molecule}.

\item {} 
\sphinxstyleliteralstrong{\sphinxupquote{return\_type}} (\sphinxstyleliteralemphasis{\sphinxupquote{str}}) \textendash{} The type of values to return. Default is
py:class:\sphinxcode{\sphinxupquote{pypath.entity.Entity}} objects, alternatives are
\sphinxcode{\sphinxupquote{labels}}  \sphinxcode{\sphinxupquote{identifiers}}.

\end{itemize}

\end{description}\end{quote}

\end{fulllineitems}

\index{count\_references() (pypath.core.interaction.Interaction method)@\spxentry{count\_references()}\spxextra{pypath.core.interaction.Interaction method}}

\begin{fulllineitems}
\phantomsection\label{\detokenize{reference:pypath.core.interaction.Interaction.count_references}}\pysiglinewithargsret{\sphinxbfcode{\sphinxupquote{count\_references}}}{\emph{effect=None}, \emph{resources=None}, \emph{data\_model=None}, \emph{interaction\_type=None}, \emph{via=None}, \emph{references=None}}{}
Retrieves references matching the criteria.

\end{fulllineitems}

\index{count\_resource\_names() (pypath.core.interaction.Interaction method)@\spxentry{count\_resource\_names()}\spxextra{pypath.core.interaction.Interaction method}}

\begin{fulllineitems}
\phantomsection\label{\detokenize{reference:pypath.core.interaction.Interaction.count_resource_names}}\pysiglinewithargsret{\sphinxbfcode{\sphinxupquote{count\_resource\_names}}}{\emph{effect=None}, \emph{resources=None}, \emph{data\_model=None}, \emph{interaction\_type=None}, \emph{via=None}, \emph{references=None}}{}
Retrieves resource names matching the criteria.

\end{fulllineitems}

\index{count\_resource\_names\_via() (pypath.core.interaction.Interaction method)@\spxentry{count\_resource\_names\_via()}\spxextra{pypath.core.interaction.Interaction method}}

\begin{fulllineitems}
\phantomsection\label{\detokenize{reference:pypath.core.interaction.Interaction.count_resource_names_via}}\pysiglinewithargsret{\sphinxbfcode{\sphinxupquote{count\_resource\_names\_via}}}{\emph{effect=None}, \emph{resources=None}, \emph{data\_model=None}, \emph{interaction\_type=None}, \emph{via=None}, \emph{references=None}}{}
Retrieves resource names via matching the criteria.

\end{fulllineitems}

\index{count\_resources() (pypath.core.interaction.Interaction method)@\spxentry{count\_resources()}\spxextra{pypath.core.interaction.Interaction method}}

\begin{fulllineitems}
\phantomsection\label{\detokenize{reference:pypath.core.interaction.Interaction.count_resources}}\pysiglinewithargsret{\sphinxbfcode{\sphinxupquote{count\_resources}}}{\emph{effect=None}, \emph{resources=None}, \emph{data\_model=None}, \emph{interaction\_type=None}, \emph{via=None}, \emph{references=None}}{}
Retrieves resources matching the criteria.

\end{fulllineitems}

\index{count\_resources\_via() (pypath.core.interaction.Interaction method)@\spxentry{count\_resources\_via()}\spxextra{pypath.core.interaction.Interaction method}}

\begin{fulllineitems}
\phantomsection\label{\detokenize{reference:pypath.core.interaction.Interaction.count_resources_via}}\pysiglinewithargsret{\sphinxbfcode{\sphinxupquote{count\_resources\_via}}}{\emph{effect=None}, \emph{resources=None}, \emph{data\_model=None}, \emph{interaction\_type=None}, \emph{via=None}, \emph{references=None}}{}
Retrieves resources via matching the criteria.

\end{fulllineitems}

\index{count\_small\_molecule\_identifiers() (pypath.core.interaction.Interaction method)@\spxentry{count\_small\_molecule\_identifiers()}\spxextra{pypath.core.interaction.Interaction method}}

\begin{fulllineitems}
\phantomsection\label{\detokenize{reference:pypath.core.interaction.Interaction.count_small_molecule_identifiers}}\pysiglinewithargsret{\sphinxbfcode{\sphinxupquote{count\_small\_molecule\_identifiers}}}{\emph{effect=None}, \emph{resources=None}, \emph{data\_model=None}, \emph{interaction\_type=None}, \emph{via=None}, \emph{references=None}}{}
Retrieves the entities involved in interactions matching the criteria.
It either returns both interacting entities in a \sphinxstyleemphasis{set} or an empty
\sphinxstyleemphasis{set}. This may not sound so useful at the level of this object but
becomes more useful once we want to collect entities having certain
kind of interactions across a series of \sphinxtitleref{Interaction} objects.
\begin{quote}\begin{description}
\item[{Parameters}] \leavevmode\begin{itemize}
\item {} 
\sphinxstyleliteralstrong{\sphinxupquote{entity\_type}} (\sphinxstyleliteralemphasis{\sphinxupquote{str}}) \textendash{} The type of the molecular entity. Possible values: \sphinxtitleref{protein},
\sphinxtitleref{complex}, \sphinxtitleref{mirna}, \sphinxtitleref{small\_molecule}.

\item {} 
\sphinxstyleliteralstrong{\sphinxupquote{return\_type}} (\sphinxstyleliteralemphasis{\sphinxupquote{str}}) \textendash{} The type of values to return. Default is
py:class:\sphinxcode{\sphinxupquote{pypath.entity.Entity}} objects, alternatives are
\sphinxcode{\sphinxupquote{labels}}  \sphinxcode{\sphinxupquote{identifiers}}.

\end{itemize}

\end{description}\end{quote}

\end{fulllineitems}

\index{count\_small\_molecule\_labels() (pypath.core.interaction.Interaction method)@\spxentry{count\_small\_molecule\_labels()}\spxextra{pypath.core.interaction.Interaction method}}

\begin{fulllineitems}
\phantomsection\label{\detokenize{reference:pypath.core.interaction.Interaction.count_small_molecule_labels}}\pysiglinewithargsret{\sphinxbfcode{\sphinxupquote{count\_small\_molecule\_labels}}}{\emph{effect=None}, \emph{resources=None}, \emph{data\_model=None}, \emph{interaction\_type=None}, \emph{via=None}, \emph{references=None}}{}
Retrieves the entities involved in interactions matching the criteria.
It either returns both interacting entities in a \sphinxstyleemphasis{set} or an empty
\sphinxstyleemphasis{set}. This may not sound so useful at the level of this object but
becomes more useful once we want to collect entities having certain
kind of interactions across a series of \sphinxtitleref{Interaction} objects.
\begin{quote}\begin{description}
\item[{Parameters}] \leavevmode\begin{itemize}
\item {} 
\sphinxstyleliteralstrong{\sphinxupquote{entity\_type}} (\sphinxstyleliteralemphasis{\sphinxupquote{str}}) \textendash{} The type of the molecular entity. Possible values: \sphinxtitleref{protein},
\sphinxtitleref{complex}, \sphinxtitleref{mirna}, \sphinxtitleref{small\_molecule}.

\item {} 
\sphinxstyleliteralstrong{\sphinxupquote{return\_type}} (\sphinxstyleliteralemphasis{\sphinxupquote{str}}) \textendash{} The type of values to return. Default is
py:class:\sphinxcode{\sphinxupquote{pypath.entity.Entity}} objects, alternatives are
\sphinxcode{\sphinxupquote{labels}}  \sphinxcode{\sphinxupquote{identifiers}}.

\end{itemize}

\end{description}\end{quote}

\end{fulllineitems}

\index{count\_small\_molecules() (pypath.core.interaction.Interaction method)@\spxentry{count\_small\_molecules()}\spxextra{pypath.core.interaction.Interaction method}}

\begin{fulllineitems}
\phantomsection\label{\detokenize{reference:pypath.core.interaction.Interaction.count_small_molecules}}\pysiglinewithargsret{\sphinxbfcode{\sphinxupquote{count\_small\_molecules}}}{\emph{effect=None}, \emph{resources=None}, \emph{data\_model=None}, \emph{interaction\_type=None}, \emph{via=None}, \emph{references=None}}{}
Retrieves the entities involved in interactions matching the criteria.
It either returns both interacting entities in a \sphinxstyleemphasis{set} or an empty
\sphinxstyleemphasis{set}. This may not sound so useful at the level of this object but
becomes more useful once we want to collect entities having certain
kind of interactions across a series of \sphinxtitleref{Interaction} objects.
\begin{quote}\begin{description}
\item[{Parameters}] \leavevmode\begin{itemize}
\item {} 
\sphinxstyleliteralstrong{\sphinxupquote{entity\_type}} (\sphinxstyleliteralemphasis{\sphinxupquote{str}}) \textendash{} The type of the molecular entity. Possible values: \sphinxtitleref{protein},
\sphinxtitleref{complex}, \sphinxtitleref{mirna}, \sphinxtitleref{small\_molecule}.

\item {} 
\sphinxstyleliteralstrong{\sphinxupquote{return\_type}} (\sphinxstyleliteralemphasis{\sphinxupquote{str}}) \textendash{} The type of values to return. Default is
py:class:\sphinxcode{\sphinxupquote{pypath.entity.Entity}} objects, alternatives are
\sphinxcode{\sphinxupquote{labels}}  \sphinxcode{\sphinxupquote{identifiers}}.

\end{itemize}

\end{description}\end{quote}

\end{fulllineitems}

\index{data\_models\_by\_data\_model() (pypath.core.interaction.Interaction method)@\spxentry{data\_models\_by\_data\_model()}\spxextra{pypath.core.interaction.Interaction method}}

\begin{fulllineitems}
\phantomsection\label{\detokenize{reference:pypath.core.interaction.Interaction.data_models_by_data_model}}\pysiglinewithargsret{\sphinxbfcode{\sphinxupquote{data\_models\_by\_data\_model}}}{\emph{effect=None}, \emph{resources=None}, \emph{data\_model=None}, \emph{interaction\_type=None}, \emph{via=None}, \emph{references=None}}{}
Retrieves data models matching the criteria.

\end{fulllineitems}

\index{data\_models\_by\_interaction\_type() (pypath.core.interaction.Interaction method)@\spxentry{data\_models\_by\_interaction\_type()}\spxextra{pypath.core.interaction.Interaction method}}

\begin{fulllineitems}
\phantomsection\label{\detokenize{reference:pypath.core.interaction.Interaction.data_models_by_interaction_type}}\pysiglinewithargsret{\sphinxbfcode{\sphinxupquote{data\_models\_by\_interaction\_type}}}{\emph{effect=None}, \emph{resources=None}, \emph{data\_model=None}, \emph{interaction\_type=None}, \emph{via=None}, \emph{references=None}}{}
Retrieves data models matching the criteria.

\end{fulllineitems}

\index{data\_models\_by\_interaction\_type\_and\_data\_model() (pypath.core.interaction.Interaction method)@\spxentry{data\_models\_by\_interaction\_type\_and\_data\_model()}\spxextra{pypath.core.interaction.Interaction method}}

\begin{fulllineitems}
\phantomsection\label{\detokenize{reference:pypath.core.interaction.Interaction.data_models_by_interaction_type_and_data_model}}\pysiglinewithargsret{\sphinxbfcode{\sphinxupquote{data\_models\_by\_interaction\_type\_and\_data\_model}}}{\emph{effect=None}, \emph{resources=None}, \emph{data\_model=None}, \emph{interaction\_type=None}, \emph{via=None}, \emph{references=None}}{}
Retrieves data models matching the criteria.

\end{fulllineitems}

\index{data\_models\_by\_interaction\_type\_and\_data\_model\_and\_resource() (pypath.core.interaction.Interaction method)@\spxentry{data\_models\_by\_interaction\_type\_and\_data\_model\_and\_resource()}\spxextra{pypath.core.interaction.Interaction method}}

\begin{fulllineitems}
\phantomsection\label{\detokenize{reference:pypath.core.interaction.Interaction.data_models_by_interaction_type_and_data_model_and_resource}}\pysiglinewithargsret{\sphinxbfcode{\sphinxupquote{data\_models\_by\_interaction\_type\_and\_data\_model\_and\_resource}}}{\emph{effect=None}, \emph{resources=None}, \emph{data\_model=None}, \emph{interaction\_type=None}, \emph{via=None}, \emph{references=None}}{}
Retrieves data models matching the criteria.

\end{fulllineitems}

\index{data\_models\_by\_reference() (pypath.core.interaction.Interaction method)@\spxentry{data\_models\_by\_reference()}\spxextra{pypath.core.interaction.Interaction method}}

\begin{fulllineitems}
\phantomsection\label{\detokenize{reference:pypath.core.interaction.Interaction.data_models_by_reference}}\pysiglinewithargsret{\sphinxbfcode{\sphinxupquote{data\_models\_by\_reference}}}{\emph{effect=None}, \emph{resources=None}, \emph{data\_model=None}, \emph{interaction\_type=None}, \emph{via=None}, \emph{references=None}}{}
Retrieves data models matching the criteria.

\end{fulllineitems}

\index{data\_models\_by\_resource() (pypath.core.interaction.Interaction method)@\spxentry{data\_models\_by\_resource()}\spxextra{pypath.core.interaction.Interaction method}}

\begin{fulllineitems}
\phantomsection\label{\detokenize{reference:pypath.core.interaction.Interaction.data_models_by_resource}}\pysiglinewithargsret{\sphinxbfcode{\sphinxupquote{data\_models\_by\_resource}}}{\emph{effect=None}, \emph{resources=None}, \emph{data\_model=None}, \emph{interaction\_type=None}, \emph{via=None}, \emph{references=None}}{}
Retrieves data models matching the criteria.

\end{fulllineitems}

\index{degrees\_directed\_by\_data\_model() (pypath.core.interaction.Interaction method)@\spxentry{degrees\_directed\_by\_data\_model()}\spxextra{pypath.core.interaction.Interaction method}}

\begin{fulllineitems}
\phantomsection\label{\detokenize{reference:pypath.core.interaction.Interaction.degrees_directed_by_data_model}}\pysiglinewithargsret{\sphinxbfcode{\sphinxupquote{degrees\_directed\_by\_data\_model}}}{\emph{effect=None}, \emph{resources=None}, \emph{data\_model=None}, \emph{interaction\_type=None}, \emph{via=None}, \emph{references=None}}{}
Returns a \sphinxstyleemphasis{set} of nodes with the connections matching the direction,
effect and evidence criteria. E.g. if the query concerns the incoming
degrees with positive effect and the matching evidences show A
activates B, but not the other way around, only “B” will be returned.
\begin{quote}\begin{description}
\item[{Parameters}] \leavevmode
\sphinxstyleliteralstrong{\sphinxupquote{mode}} (\sphinxstyleliteralemphasis{\sphinxupquote{str}}) \textendash{} The type of degrees to be considered. Three possible values are
\sphinxcode{\sphinxupquote{'IN'}}, \sphinxtitleref{‘OUT’{}`} and \sphinxcode{\sphinxupquote{'ALL'}} for incoming, outgoing and all
connections, respectively. If the \sphinxcode{\sphinxupquote{direction}} is \sphinxcode{\sphinxupquote{False}} the
only possible mode is \sphinxcode{\sphinxupquote{ALL}}. If the \sphinxcode{\sphinxupquote{direction}} is \sphinxcode{\sphinxupquote{None}}
and also directed evidence(s) match the criteria these will
overwrite the undirected evidences and only the directed result
will be returned.

\end{description}\end{quote}

\end{fulllineitems}

\index{degrees\_directed\_by\_interaction\_type() (pypath.core.interaction.Interaction method)@\spxentry{degrees\_directed\_by\_interaction\_type()}\spxextra{pypath.core.interaction.Interaction method}}

\begin{fulllineitems}
\phantomsection\label{\detokenize{reference:pypath.core.interaction.Interaction.degrees_directed_by_interaction_type}}\pysiglinewithargsret{\sphinxbfcode{\sphinxupquote{degrees\_directed\_by\_interaction\_type}}}{\emph{effect=None}, \emph{resources=None}, \emph{data\_model=None}, \emph{interaction\_type=None}, \emph{via=None}, \emph{references=None}}{}
Returns a \sphinxstyleemphasis{set} of nodes with the connections matching the direction,
effect and evidence criteria. E.g. if the query concerns the incoming
degrees with positive effect and the matching evidences show A
activates B, but not the other way around, only “B” will be returned.
\begin{quote}\begin{description}
\item[{Parameters}] \leavevmode
\sphinxstyleliteralstrong{\sphinxupquote{mode}} (\sphinxstyleliteralemphasis{\sphinxupquote{str}}) \textendash{} The type of degrees to be considered. Three possible values are
\sphinxcode{\sphinxupquote{'IN'}}, \sphinxtitleref{‘OUT’{}`} and \sphinxcode{\sphinxupquote{'ALL'}} for incoming, outgoing and all
connections, respectively. If the \sphinxcode{\sphinxupquote{direction}} is \sphinxcode{\sphinxupquote{False}} the
only possible mode is \sphinxcode{\sphinxupquote{ALL}}. If the \sphinxcode{\sphinxupquote{direction}} is \sphinxcode{\sphinxupquote{None}}
and also directed evidence(s) match the criteria these will
overwrite the undirected evidences and only the directed result
will be returned.

\end{description}\end{quote}

\end{fulllineitems}

\index{degrees\_directed\_by\_interaction\_type\_and\_data\_model() (pypath.core.interaction.Interaction method)@\spxentry{degrees\_directed\_by\_interaction\_type\_and\_data\_model()}\spxextra{pypath.core.interaction.Interaction method}}

\begin{fulllineitems}
\phantomsection\label{\detokenize{reference:pypath.core.interaction.Interaction.degrees_directed_by_interaction_type_and_data_model}}\pysiglinewithargsret{\sphinxbfcode{\sphinxupquote{degrees\_directed\_by\_interaction\_type\_and\_data\_model}}}{\emph{effect=None}, \emph{resources=None}, \emph{data\_model=None}, \emph{interaction\_type=None}, \emph{via=None}, \emph{references=None}}{}
Returns a \sphinxstyleemphasis{set} of nodes with the connections matching the direction,
effect and evidence criteria. E.g. if the query concerns the incoming
degrees with positive effect and the matching evidences show A
activates B, but not the other way around, only “B” will be returned.
\begin{quote}\begin{description}
\item[{Parameters}] \leavevmode
\sphinxstyleliteralstrong{\sphinxupquote{mode}} (\sphinxstyleliteralemphasis{\sphinxupquote{str}}) \textendash{} The type of degrees to be considered. Three possible values are
\sphinxcode{\sphinxupquote{'IN'}}, \sphinxtitleref{‘OUT’{}`} and \sphinxcode{\sphinxupquote{'ALL'}} for incoming, outgoing and all
connections, respectively. If the \sphinxcode{\sphinxupquote{direction}} is \sphinxcode{\sphinxupquote{False}} the
only possible mode is \sphinxcode{\sphinxupquote{ALL}}. If the \sphinxcode{\sphinxupquote{direction}} is \sphinxcode{\sphinxupquote{None}}
and also directed evidence(s) match the criteria these will
overwrite the undirected evidences and only the directed result
will be returned.

\end{description}\end{quote}

\end{fulllineitems}

\index{degrees\_directed\_by\_interaction\_type\_and\_data\_model\_and\_resource() (pypath.core.interaction.Interaction method)@\spxentry{degrees\_directed\_by\_interaction\_type\_and\_data\_model\_and\_resource()}\spxextra{pypath.core.interaction.Interaction method}}

\begin{fulllineitems}
\phantomsection\label{\detokenize{reference:pypath.core.interaction.Interaction.degrees_directed_by_interaction_type_and_data_model_and_resource}}\pysiglinewithargsret{\sphinxbfcode{\sphinxupquote{degrees\_directed\_by\_interaction\_type\_and\_data\_model\_and\_resource}}}{\emph{effect=None}, \emph{resources=None}, \emph{data\_model=None}, \emph{interaction\_type=None}, \emph{via=None}, \emph{references=None}}{}
Returns a \sphinxstyleemphasis{set} of nodes with the connections matching the direction,
effect and evidence criteria. E.g. if the query concerns the incoming
degrees with positive effect and the matching evidences show A
activates B, but not the other way around, only “B” will be returned.
\begin{quote}\begin{description}
\item[{Parameters}] \leavevmode
\sphinxstyleliteralstrong{\sphinxupquote{mode}} (\sphinxstyleliteralemphasis{\sphinxupquote{str}}) \textendash{} The type of degrees to be considered. Three possible values are
\sphinxcode{\sphinxupquote{'IN'}}, \sphinxtitleref{‘OUT’{}`} and \sphinxcode{\sphinxupquote{'ALL'}} for incoming, outgoing and all
connections, respectively. If the \sphinxcode{\sphinxupquote{direction}} is \sphinxcode{\sphinxupquote{False}} the
only possible mode is \sphinxcode{\sphinxupquote{ALL}}. If the \sphinxcode{\sphinxupquote{direction}} is \sphinxcode{\sphinxupquote{None}}
and also directed evidence(s) match the criteria these will
overwrite the undirected evidences and only the directed result
will be returned.

\end{description}\end{quote}

\end{fulllineitems}

\index{degrees\_directed\_by\_reference() (pypath.core.interaction.Interaction method)@\spxentry{degrees\_directed\_by\_reference()}\spxextra{pypath.core.interaction.Interaction method}}

\begin{fulllineitems}
\phantomsection\label{\detokenize{reference:pypath.core.interaction.Interaction.degrees_directed_by_reference}}\pysiglinewithargsret{\sphinxbfcode{\sphinxupquote{degrees\_directed\_by\_reference}}}{\emph{effect=None}, \emph{resources=None}, \emph{data\_model=None}, \emph{interaction\_type=None}, \emph{via=None}, \emph{references=None}}{}
Returns a \sphinxstyleemphasis{set} of nodes with the connections matching the direction,
effect and evidence criteria. E.g. if the query concerns the incoming
degrees with positive effect and the matching evidences show A
activates B, but not the other way around, only “B” will be returned.
\begin{quote}\begin{description}
\item[{Parameters}] \leavevmode
\sphinxstyleliteralstrong{\sphinxupquote{mode}} (\sphinxstyleliteralemphasis{\sphinxupquote{str}}) \textendash{} The type of degrees to be considered. Three possible values are
\sphinxcode{\sphinxupquote{'IN'}}, \sphinxtitleref{‘OUT’{}`} and \sphinxcode{\sphinxupquote{'ALL'}} for incoming, outgoing and all
connections, respectively. If the \sphinxcode{\sphinxupquote{direction}} is \sphinxcode{\sphinxupquote{False}} the
only possible mode is \sphinxcode{\sphinxupquote{ALL}}. If the \sphinxcode{\sphinxupquote{direction}} is \sphinxcode{\sphinxupquote{None}}
and also directed evidence(s) match the criteria these will
overwrite the undirected evidences and only the directed result
will be returned.

\end{description}\end{quote}

\end{fulllineitems}

\index{degrees\_directed\_by\_resource() (pypath.core.interaction.Interaction method)@\spxentry{degrees\_directed\_by\_resource()}\spxextra{pypath.core.interaction.Interaction method}}

\begin{fulllineitems}
\phantomsection\label{\detokenize{reference:pypath.core.interaction.Interaction.degrees_directed_by_resource}}\pysiglinewithargsret{\sphinxbfcode{\sphinxupquote{degrees\_directed\_by\_resource}}}{\emph{effect=None}, \emph{resources=None}, \emph{data\_model=None}, \emph{interaction\_type=None}, \emph{via=None}, \emph{references=None}}{}
Returns a \sphinxstyleemphasis{set} of nodes with the connections matching the direction,
effect and evidence criteria. E.g. if the query concerns the incoming
degrees with positive effect and the matching evidences show A
activates B, but not the other way around, only “B” will be returned.
\begin{quote}\begin{description}
\item[{Parameters}] \leavevmode
\sphinxstyleliteralstrong{\sphinxupquote{mode}} (\sphinxstyleliteralemphasis{\sphinxupquote{str}}) \textendash{} The type of degrees to be considered. Three possible values are
\sphinxcode{\sphinxupquote{'IN'}}, \sphinxtitleref{‘OUT’{}`} and \sphinxcode{\sphinxupquote{'ALL'}} for incoming, outgoing and all
connections, respectively. If the \sphinxcode{\sphinxupquote{direction}} is \sphinxcode{\sphinxupquote{False}} the
only possible mode is \sphinxcode{\sphinxupquote{ALL}}. If the \sphinxcode{\sphinxupquote{direction}} is \sphinxcode{\sphinxupquote{None}}
and also directed evidence(s) match the criteria these will
overwrite the undirected evidences and only the directed result
will be returned.

\end{description}\end{quote}

\end{fulllineitems}

\index{degrees\_directed\_in\_by\_data\_model() (pypath.core.interaction.Interaction method)@\spxentry{degrees\_directed\_in\_by\_data\_model()}\spxextra{pypath.core.interaction.Interaction method}}

\begin{fulllineitems}
\phantomsection\label{\detokenize{reference:pypath.core.interaction.Interaction.degrees_directed_in_by_data_model}}\pysiglinewithargsret{\sphinxbfcode{\sphinxupquote{degrees\_directed\_in\_by\_data\_model}}}{\emph{effect=None}, \emph{resources=None}, \emph{data\_model=None}, \emph{interaction\_type=None}, \emph{via=None}, \emph{references=None}}{}
Returns a \sphinxstyleemphasis{set} of nodes with the connections matching the direction,
effect and evidence criteria. E.g. if the query concerns the incoming
degrees with positive effect and the matching evidences show A
activates B, but not the other way around, only “B” will be returned.
\begin{quote}\begin{description}
\item[{Parameters}] \leavevmode
\sphinxstyleliteralstrong{\sphinxupquote{mode}} (\sphinxstyleliteralemphasis{\sphinxupquote{str}}) \textendash{} The type of degrees to be considered. Three possible values are
\sphinxcode{\sphinxupquote{'IN'}}, \sphinxtitleref{‘OUT’{}`} and \sphinxcode{\sphinxupquote{'ALL'}} for incoming, outgoing and all
connections, respectively. If the \sphinxcode{\sphinxupquote{direction}} is \sphinxcode{\sphinxupquote{False}} the
only possible mode is \sphinxcode{\sphinxupquote{ALL}}. If the \sphinxcode{\sphinxupquote{direction}} is \sphinxcode{\sphinxupquote{None}}
and also directed evidence(s) match the criteria these will
overwrite the undirected evidences and only the directed result
will be returned.

\end{description}\end{quote}

\end{fulllineitems}

\index{degrees\_directed\_in\_by\_interaction\_type() (pypath.core.interaction.Interaction method)@\spxentry{degrees\_directed\_in\_by\_interaction\_type()}\spxextra{pypath.core.interaction.Interaction method}}

\begin{fulllineitems}
\phantomsection\label{\detokenize{reference:pypath.core.interaction.Interaction.degrees_directed_in_by_interaction_type}}\pysiglinewithargsret{\sphinxbfcode{\sphinxupquote{degrees\_directed\_in\_by\_interaction\_type}}}{\emph{effect=None}, \emph{resources=None}, \emph{data\_model=None}, \emph{interaction\_type=None}, \emph{via=None}, \emph{references=None}}{}
Returns a \sphinxstyleemphasis{set} of nodes with the connections matching the direction,
effect and evidence criteria. E.g. if the query concerns the incoming
degrees with positive effect and the matching evidences show A
activates B, but not the other way around, only “B” will be returned.
\begin{quote}\begin{description}
\item[{Parameters}] \leavevmode
\sphinxstyleliteralstrong{\sphinxupquote{mode}} (\sphinxstyleliteralemphasis{\sphinxupquote{str}}) \textendash{} The type of degrees to be considered. Three possible values are
\sphinxcode{\sphinxupquote{'IN'}}, \sphinxtitleref{‘OUT’{}`} and \sphinxcode{\sphinxupquote{'ALL'}} for incoming, outgoing and all
connections, respectively. If the \sphinxcode{\sphinxupquote{direction}} is \sphinxcode{\sphinxupquote{False}} the
only possible mode is \sphinxcode{\sphinxupquote{ALL}}. If the \sphinxcode{\sphinxupquote{direction}} is \sphinxcode{\sphinxupquote{None}}
and also directed evidence(s) match the criteria these will
overwrite the undirected evidences and only the directed result
will be returned.

\end{description}\end{quote}

\end{fulllineitems}

\index{degrees\_directed\_in\_by\_interaction\_type\_and\_data\_model() (pypath.core.interaction.Interaction method)@\spxentry{degrees\_directed\_in\_by\_interaction\_type\_and\_data\_model()}\spxextra{pypath.core.interaction.Interaction method}}

\begin{fulllineitems}
\phantomsection\label{\detokenize{reference:pypath.core.interaction.Interaction.degrees_directed_in_by_interaction_type_and_data_model}}\pysiglinewithargsret{\sphinxbfcode{\sphinxupquote{degrees\_directed\_in\_by\_interaction\_type\_and\_data\_model}}}{\emph{effect=None}, \emph{resources=None}, \emph{data\_model=None}, \emph{interaction\_type=None}, \emph{via=None}, \emph{references=None}}{}
Returns a \sphinxstyleemphasis{set} of nodes with the connections matching the direction,
effect and evidence criteria. E.g. if the query concerns the incoming
degrees with positive effect and the matching evidences show A
activates B, but not the other way around, only “B” will be returned.
\begin{quote}\begin{description}
\item[{Parameters}] \leavevmode
\sphinxstyleliteralstrong{\sphinxupquote{mode}} (\sphinxstyleliteralemphasis{\sphinxupquote{str}}) \textendash{} The type of degrees to be considered. Three possible values are
\sphinxcode{\sphinxupquote{'IN'}}, \sphinxtitleref{‘OUT’{}`} and \sphinxcode{\sphinxupquote{'ALL'}} for incoming, outgoing and all
connections, respectively. If the \sphinxcode{\sphinxupquote{direction}} is \sphinxcode{\sphinxupquote{False}} the
only possible mode is \sphinxcode{\sphinxupquote{ALL}}. If the \sphinxcode{\sphinxupquote{direction}} is \sphinxcode{\sphinxupquote{None}}
and also directed evidence(s) match the criteria these will
overwrite the undirected evidences and only the directed result
will be returned.

\end{description}\end{quote}

\end{fulllineitems}

\index{degrees\_directed\_in\_by\_interaction\_type\_and\_data\_model\_and\_resource() (pypath.core.interaction.Interaction method)@\spxentry{degrees\_directed\_in\_by\_interaction\_type\_and\_data\_model\_and\_resource()}\spxextra{pypath.core.interaction.Interaction method}}

\begin{fulllineitems}
\phantomsection\label{\detokenize{reference:pypath.core.interaction.Interaction.degrees_directed_in_by_interaction_type_and_data_model_and_resource}}\pysiglinewithargsret{\sphinxbfcode{\sphinxupquote{degrees\_directed\_in\_by\_interaction\_type\_and\_data\_model\_and\_resource}}}{\emph{effect=None}, \emph{resources=None}, \emph{data\_model=None}, \emph{interaction\_type=None}, \emph{via=None}, \emph{references=None}}{}
Returns a \sphinxstyleemphasis{set} of nodes with the connections matching the direction,
effect and evidence criteria. E.g. if the query concerns the incoming
degrees with positive effect and the matching evidences show A
activates B, but not the other way around, only “B” will be returned.
\begin{quote}\begin{description}
\item[{Parameters}] \leavevmode
\sphinxstyleliteralstrong{\sphinxupquote{mode}} (\sphinxstyleliteralemphasis{\sphinxupquote{str}}) \textendash{} The type of degrees to be considered. Three possible values are
\sphinxcode{\sphinxupquote{'IN'}}, \sphinxtitleref{‘OUT’{}`} and \sphinxcode{\sphinxupquote{'ALL'}} for incoming, outgoing and all
connections, respectively. If the \sphinxcode{\sphinxupquote{direction}} is \sphinxcode{\sphinxupquote{False}} the
only possible mode is \sphinxcode{\sphinxupquote{ALL}}. If the \sphinxcode{\sphinxupquote{direction}} is \sphinxcode{\sphinxupquote{None}}
and also directed evidence(s) match the criteria these will
overwrite the undirected evidences and only the directed result
will be returned.

\end{description}\end{quote}

\end{fulllineitems}

\index{degrees\_directed\_in\_by\_reference() (pypath.core.interaction.Interaction method)@\spxentry{degrees\_directed\_in\_by\_reference()}\spxextra{pypath.core.interaction.Interaction method}}

\begin{fulllineitems}
\phantomsection\label{\detokenize{reference:pypath.core.interaction.Interaction.degrees_directed_in_by_reference}}\pysiglinewithargsret{\sphinxbfcode{\sphinxupquote{degrees\_directed\_in\_by\_reference}}}{\emph{effect=None}, \emph{resources=None}, \emph{data\_model=None}, \emph{interaction\_type=None}, \emph{via=None}, \emph{references=None}}{}
Returns a \sphinxstyleemphasis{set} of nodes with the connections matching the direction,
effect and evidence criteria. E.g. if the query concerns the incoming
degrees with positive effect and the matching evidences show A
activates B, but not the other way around, only “B” will be returned.
\begin{quote}\begin{description}
\item[{Parameters}] \leavevmode
\sphinxstyleliteralstrong{\sphinxupquote{mode}} (\sphinxstyleliteralemphasis{\sphinxupquote{str}}) \textendash{} The type of degrees to be considered. Three possible values are
\sphinxcode{\sphinxupquote{'IN'}}, \sphinxtitleref{‘OUT’{}`} and \sphinxcode{\sphinxupquote{'ALL'}} for incoming, outgoing and all
connections, respectively. If the \sphinxcode{\sphinxupquote{direction}} is \sphinxcode{\sphinxupquote{False}} the
only possible mode is \sphinxcode{\sphinxupquote{ALL}}. If the \sphinxcode{\sphinxupquote{direction}} is \sphinxcode{\sphinxupquote{None}}
and also directed evidence(s) match the criteria these will
overwrite the undirected evidences and only the directed result
will be returned.

\end{description}\end{quote}

\end{fulllineitems}

\index{degrees\_directed\_in\_by\_resource() (pypath.core.interaction.Interaction method)@\spxentry{degrees\_directed\_in\_by\_resource()}\spxextra{pypath.core.interaction.Interaction method}}

\begin{fulllineitems}
\phantomsection\label{\detokenize{reference:pypath.core.interaction.Interaction.degrees_directed_in_by_resource}}\pysiglinewithargsret{\sphinxbfcode{\sphinxupquote{degrees\_directed\_in\_by\_resource}}}{\emph{effect=None}, \emph{resources=None}, \emph{data\_model=None}, \emph{interaction\_type=None}, \emph{via=None}, \emph{references=None}}{}
Returns a \sphinxstyleemphasis{set} of nodes with the connections matching the direction,
effect and evidence criteria. E.g. if the query concerns the incoming
degrees with positive effect and the matching evidences show A
activates B, but not the other way around, only “B” will be returned.
\begin{quote}\begin{description}
\item[{Parameters}] \leavevmode
\sphinxstyleliteralstrong{\sphinxupquote{mode}} (\sphinxstyleliteralemphasis{\sphinxupquote{str}}) \textendash{} The type of degrees to be considered. Three possible values are
\sphinxcode{\sphinxupquote{'IN'}}, \sphinxtitleref{‘OUT’{}`} and \sphinxcode{\sphinxupquote{'ALL'}} for incoming, outgoing and all
connections, respectively. If the \sphinxcode{\sphinxupquote{direction}} is \sphinxcode{\sphinxupquote{False}} the
only possible mode is \sphinxcode{\sphinxupquote{ALL}}. If the \sphinxcode{\sphinxupquote{direction}} is \sphinxcode{\sphinxupquote{None}}
and also directed evidence(s) match the criteria these will
overwrite the undirected evidences and only the directed result
will be returned.

\end{description}\end{quote}

\end{fulllineitems}

\index{degrees\_directed\_out\_by\_data\_model() (pypath.core.interaction.Interaction method)@\spxentry{degrees\_directed\_out\_by\_data\_model()}\spxextra{pypath.core.interaction.Interaction method}}

\begin{fulllineitems}
\phantomsection\label{\detokenize{reference:pypath.core.interaction.Interaction.degrees_directed_out_by_data_model}}\pysiglinewithargsret{\sphinxbfcode{\sphinxupquote{degrees\_directed\_out\_by\_data\_model}}}{\emph{effect=None}, \emph{resources=None}, \emph{data\_model=None}, \emph{interaction\_type=None}, \emph{via=None}, \emph{references=None}}{}
Returns a \sphinxstyleemphasis{set} of nodes with the connections matching the direction,
effect and evidence criteria. E.g. if the query concerns the incoming
degrees with positive effect and the matching evidences show A
activates B, but not the other way around, only “B” will be returned.
\begin{quote}\begin{description}
\item[{Parameters}] \leavevmode
\sphinxstyleliteralstrong{\sphinxupquote{mode}} (\sphinxstyleliteralemphasis{\sphinxupquote{str}}) \textendash{} The type of degrees to be considered. Three possible values are
\sphinxcode{\sphinxupquote{'IN'}}, \sphinxtitleref{‘OUT’{}`} and \sphinxcode{\sphinxupquote{'ALL'}} for incoming, outgoing and all
connections, respectively. If the \sphinxcode{\sphinxupquote{direction}} is \sphinxcode{\sphinxupquote{False}} the
only possible mode is \sphinxcode{\sphinxupquote{ALL}}. If the \sphinxcode{\sphinxupquote{direction}} is \sphinxcode{\sphinxupquote{None}}
and also directed evidence(s) match the criteria these will
overwrite the undirected evidences and only the directed result
will be returned.

\end{description}\end{quote}

\end{fulllineitems}

\index{degrees\_directed\_out\_by\_interaction\_type() (pypath.core.interaction.Interaction method)@\spxentry{degrees\_directed\_out\_by\_interaction\_type()}\spxextra{pypath.core.interaction.Interaction method}}

\begin{fulllineitems}
\phantomsection\label{\detokenize{reference:pypath.core.interaction.Interaction.degrees_directed_out_by_interaction_type}}\pysiglinewithargsret{\sphinxbfcode{\sphinxupquote{degrees\_directed\_out\_by\_interaction\_type}}}{\emph{effect=None}, \emph{resources=None}, \emph{data\_model=None}, \emph{interaction\_type=None}, \emph{via=None}, \emph{references=None}}{}
Returns a \sphinxstyleemphasis{set} of nodes with the connections matching the direction,
effect and evidence criteria. E.g. if the query concerns the incoming
degrees with positive effect and the matching evidences show A
activates B, but not the other way around, only “B” will be returned.
\begin{quote}\begin{description}
\item[{Parameters}] \leavevmode
\sphinxstyleliteralstrong{\sphinxupquote{mode}} (\sphinxstyleliteralemphasis{\sphinxupquote{str}}) \textendash{} The type of degrees to be considered. Three possible values are
\sphinxcode{\sphinxupquote{'IN'}}, \sphinxtitleref{‘OUT’{}`} and \sphinxcode{\sphinxupquote{'ALL'}} for incoming, outgoing and all
connections, respectively. If the \sphinxcode{\sphinxupquote{direction}} is \sphinxcode{\sphinxupquote{False}} the
only possible mode is \sphinxcode{\sphinxupquote{ALL}}. If the \sphinxcode{\sphinxupquote{direction}} is \sphinxcode{\sphinxupquote{None}}
and also directed evidence(s) match the criteria these will
overwrite the undirected evidences and only the directed result
will be returned.

\end{description}\end{quote}

\end{fulllineitems}

\index{degrees\_directed\_out\_by\_interaction\_type\_and\_data\_model() (pypath.core.interaction.Interaction method)@\spxentry{degrees\_directed\_out\_by\_interaction\_type\_and\_data\_model()}\spxextra{pypath.core.interaction.Interaction method}}

\begin{fulllineitems}
\phantomsection\label{\detokenize{reference:pypath.core.interaction.Interaction.degrees_directed_out_by_interaction_type_and_data_model}}\pysiglinewithargsret{\sphinxbfcode{\sphinxupquote{degrees\_directed\_out\_by\_interaction\_type\_and\_data\_model}}}{\emph{effect=None}, \emph{resources=None}, \emph{data\_model=None}, \emph{interaction\_type=None}, \emph{via=None}, \emph{references=None}}{}
Returns a \sphinxstyleemphasis{set} of nodes with the connections matching the direction,
effect and evidence criteria. E.g. if the query concerns the incoming
degrees with positive effect and the matching evidences show A
activates B, but not the other way around, only “B” will be returned.
\begin{quote}\begin{description}
\item[{Parameters}] \leavevmode
\sphinxstyleliteralstrong{\sphinxupquote{mode}} (\sphinxstyleliteralemphasis{\sphinxupquote{str}}) \textendash{} The type of degrees to be considered. Three possible values are
\sphinxcode{\sphinxupquote{'IN'}}, \sphinxtitleref{‘OUT’{}`} and \sphinxcode{\sphinxupquote{'ALL'}} for incoming, outgoing and all
connections, respectively. If the \sphinxcode{\sphinxupquote{direction}} is \sphinxcode{\sphinxupquote{False}} the
only possible mode is \sphinxcode{\sphinxupquote{ALL}}. If the \sphinxcode{\sphinxupquote{direction}} is \sphinxcode{\sphinxupquote{None}}
and also directed evidence(s) match the criteria these will
overwrite the undirected evidences and only the directed result
will be returned.

\end{description}\end{quote}

\end{fulllineitems}

\index{degrees\_directed\_out\_by\_interaction\_type\_and\_data\_model\_and\_resource() (pypath.core.interaction.Interaction method)@\spxentry{degrees\_directed\_out\_by\_interaction\_type\_and\_data\_model\_and\_resource()}\spxextra{pypath.core.interaction.Interaction method}}

\begin{fulllineitems}
\phantomsection\label{\detokenize{reference:pypath.core.interaction.Interaction.degrees_directed_out_by_interaction_type_and_data_model_and_resource}}\pysiglinewithargsret{\sphinxbfcode{\sphinxupquote{degrees\_directed\_out\_by\_interaction\_type\_and\_data\_model\_and\_resource}}}{\emph{effect=None}, \emph{resources=None}, \emph{data\_model=None}, \emph{interaction\_type=None}, \emph{via=None}, \emph{references=None}}{}
Returns a \sphinxstyleemphasis{set} of nodes with the connections matching the direction,
effect and evidence criteria. E.g. if the query concerns the incoming
degrees with positive effect and the matching evidences show A
activates B, but not the other way around, only “B” will be returned.
\begin{quote}\begin{description}
\item[{Parameters}] \leavevmode
\sphinxstyleliteralstrong{\sphinxupquote{mode}} (\sphinxstyleliteralemphasis{\sphinxupquote{str}}) \textendash{} The type of degrees to be considered. Three possible values are
\sphinxcode{\sphinxupquote{'IN'}}, \sphinxtitleref{‘OUT’{}`} and \sphinxcode{\sphinxupquote{'ALL'}} for incoming, outgoing and all
connections, respectively. If the \sphinxcode{\sphinxupquote{direction}} is \sphinxcode{\sphinxupquote{False}} the
only possible mode is \sphinxcode{\sphinxupquote{ALL}}. If the \sphinxcode{\sphinxupquote{direction}} is \sphinxcode{\sphinxupquote{None}}
and also directed evidence(s) match the criteria these will
overwrite the undirected evidences and only the directed result
will be returned.

\end{description}\end{quote}

\end{fulllineitems}

\index{degrees\_directed\_out\_by\_reference() (pypath.core.interaction.Interaction method)@\spxentry{degrees\_directed\_out\_by\_reference()}\spxextra{pypath.core.interaction.Interaction method}}

\begin{fulllineitems}
\phantomsection\label{\detokenize{reference:pypath.core.interaction.Interaction.degrees_directed_out_by_reference}}\pysiglinewithargsret{\sphinxbfcode{\sphinxupquote{degrees\_directed\_out\_by\_reference}}}{\emph{effect=None}, \emph{resources=None}, \emph{data\_model=None}, \emph{interaction\_type=None}, \emph{via=None}, \emph{references=None}}{}
Returns a \sphinxstyleemphasis{set} of nodes with the connections matching the direction,
effect and evidence criteria. E.g. if the query concerns the incoming
degrees with positive effect and the matching evidences show A
activates B, but not the other way around, only “B” will be returned.
\begin{quote}\begin{description}
\item[{Parameters}] \leavevmode
\sphinxstyleliteralstrong{\sphinxupquote{mode}} (\sphinxstyleliteralemphasis{\sphinxupquote{str}}) \textendash{} The type of degrees to be considered. Three possible values are
\sphinxcode{\sphinxupquote{'IN'}}, \sphinxtitleref{‘OUT’{}`} and \sphinxcode{\sphinxupquote{'ALL'}} for incoming, outgoing and all
connections, respectively. If the \sphinxcode{\sphinxupquote{direction}} is \sphinxcode{\sphinxupquote{False}} the
only possible mode is \sphinxcode{\sphinxupquote{ALL}}. If the \sphinxcode{\sphinxupquote{direction}} is \sphinxcode{\sphinxupquote{None}}
and also directed evidence(s) match the criteria these will
overwrite the undirected evidences and only the directed result
will be returned.

\end{description}\end{quote}

\end{fulllineitems}

\index{degrees\_directed\_out\_by\_resource() (pypath.core.interaction.Interaction method)@\spxentry{degrees\_directed\_out\_by\_resource()}\spxextra{pypath.core.interaction.Interaction method}}

\begin{fulllineitems}
\phantomsection\label{\detokenize{reference:pypath.core.interaction.Interaction.degrees_directed_out_by_resource}}\pysiglinewithargsret{\sphinxbfcode{\sphinxupquote{degrees\_directed\_out\_by\_resource}}}{\emph{effect=None}, \emph{resources=None}, \emph{data\_model=None}, \emph{interaction\_type=None}, \emph{via=None}, \emph{references=None}}{}
Returns a \sphinxstyleemphasis{set} of nodes with the connections matching the direction,
effect and evidence criteria. E.g. if the query concerns the incoming
degrees with positive effect and the matching evidences show A
activates B, but not the other way around, only “B” will be returned.
\begin{quote}\begin{description}
\item[{Parameters}] \leavevmode
\sphinxstyleliteralstrong{\sphinxupquote{mode}} (\sphinxstyleliteralemphasis{\sphinxupquote{str}}) \textendash{} The type of degrees to be considered. Three possible values are
\sphinxcode{\sphinxupquote{'IN'}}, \sphinxtitleref{‘OUT’{}`} and \sphinxcode{\sphinxupquote{'ALL'}} for incoming, outgoing and all
connections, respectively. If the \sphinxcode{\sphinxupquote{direction}} is \sphinxcode{\sphinxupquote{False}} the
only possible mode is \sphinxcode{\sphinxupquote{ALL}}. If the \sphinxcode{\sphinxupquote{direction}} is \sphinxcode{\sphinxupquote{None}}
and also directed evidence(s) match the criteria these will
overwrite the undirected evidences and only the directed result
will be returned.

\end{description}\end{quote}

\end{fulllineitems}

\index{degrees\_negative\_by\_data\_model() (pypath.core.interaction.Interaction method)@\spxentry{degrees\_negative\_by\_data\_model()}\spxextra{pypath.core.interaction.Interaction method}}

\begin{fulllineitems}
\phantomsection\label{\detokenize{reference:pypath.core.interaction.Interaction.degrees_negative_by_data_model}}\pysiglinewithargsret{\sphinxbfcode{\sphinxupquote{degrees\_negative\_by\_data\_model}}}{\emph{effect=None}, \emph{resources=None}, \emph{data\_model=None}, \emph{interaction\_type=None}, \emph{via=None}, \emph{references=None}}{}
Returns a \sphinxstyleemphasis{set} of nodes with the connections matching the direction,
effect and evidence criteria. E.g. if the query concerns the incoming
degrees with positive effect and the matching evidences show A
activates B, but not the other way around, only “B” will be returned.
\begin{quote}\begin{description}
\item[{Parameters}] \leavevmode
\sphinxstyleliteralstrong{\sphinxupquote{mode}} (\sphinxstyleliteralemphasis{\sphinxupquote{str}}) \textendash{} The type of degrees to be considered. Three possible values are
\sphinxcode{\sphinxupquote{'IN'}}, \sphinxtitleref{‘OUT’{}`} and \sphinxcode{\sphinxupquote{'ALL'}} for incoming, outgoing and all
connections, respectively. If the \sphinxcode{\sphinxupquote{direction}} is \sphinxcode{\sphinxupquote{False}} the
only possible mode is \sphinxcode{\sphinxupquote{ALL}}. If the \sphinxcode{\sphinxupquote{direction}} is \sphinxcode{\sphinxupquote{None}}
and also directed evidence(s) match the criteria these will
overwrite the undirected evidences and only the directed result
will be returned.

\end{description}\end{quote}

\end{fulllineitems}

\index{degrees\_negative\_by\_interaction\_type() (pypath.core.interaction.Interaction method)@\spxentry{degrees\_negative\_by\_interaction\_type()}\spxextra{pypath.core.interaction.Interaction method}}

\begin{fulllineitems}
\phantomsection\label{\detokenize{reference:pypath.core.interaction.Interaction.degrees_negative_by_interaction_type}}\pysiglinewithargsret{\sphinxbfcode{\sphinxupquote{degrees\_negative\_by\_interaction\_type}}}{\emph{effect=None}, \emph{resources=None}, \emph{data\_model=None}, \emph{interaction\_type=None}, \emph{via=None}, \emph{references=None}}{}
Returns a \sphinxstyleemphasis{set} of nodes with the connections matching the direction,
effect and evidence criteria. E.g. if the query concerns the incoming
degrees with positive effect and the matching evidences show A
activates B, but not the other way around, only “B” will be returned.
\begin{quote}\begin{description}
\item[{Parameters}] \leavevmode
\sphinxstyleliteralstrong{\sphinxupquote{mode}} (\sphinxstyleliteralemphasis{\sphinxupquote{str}}) \textendash{} The type of degrees to be considered. Three possible values are
\sphinxcode{\sphinxupquote{'IN'}}, \sphinxtitleref{‘OUT’{}`} and \sphinxcode{\sphinxupquote{'ALL'}} for incoming, outgoing and all
connections, respectively. If the \sphinxcode{\sphinxupquote{direction}} is \sphinxcode{\sphinxupquote{False}} the
only possible mode is \sphinxcode{\sphinxupquote{ALL}}. If the \sphinxcode{\sphinxupquote{direction}} is \sphinxcode{\sphinxupquote{None}}
and also directed evidence(s) match the criteria these will
overwrite the undirected evidences and only the directed result
will be returned.

\end{description}\end{quote}

\end{fulllineitems}

\index{degrees\_negative\_by\_interaction\_type\_and\_data\_model() (pypath.core.interaction.Interaction method)@\spxentry{degrees\_negative\_by\_interaction\_type\_and\_data\_model()}\spxextra{pypath.core.interaction.Interaction method}}

\begin{fulllineitems}
\phantomsection\label{\detokenize{reference:pypath.core.interaction.Interaction.degrees_negative_by_interaction_type_and_data_model}}\pysiglinewithargsret{\sphinxbfcode{\sphinxupquote{degrees\_negative\_by\_interaction\_type\_and\_data\_model}}}{\emph{effect=None}, \emph{resources=None}, \emph{data\_model=None}, \emph{interaction\_type=None}, \emph{via=None}, \emph{references=None}}{}
Returns a \sphinxstyleemphasis{set} of nodes with the connections matching the direction,
effect and evidence criteria. E.g. if the query concerns the incoming
degrees with positive effect and the matching evidences show A
activates B, but not the other way around, only “B” will be returned.
\begin{quote}\begin{description}
\item[{Parameters}] \leavevmode
\sphinxstyleliteralstrong{\sphinxupquote{mode}} (\sphinxstyleliteralemphasis{\sphinxupquote{str}}) \textendash{} The type of degrees to be considered. Three possible values are
\sphinxcode{\sphinxupquote{'IN'}}, \sphinxtitleref{‘OUT’{}`} and \sphinxcode{\sphinxupquote{'ALL'}} for incoming, outgoing and all
connections, respectively. If the \sphinxcode{\sphinxupquote{direction}} is \sphinxcode{\sphinxupquote{False}} the
only possible mode is \sphinxcode{\sphinxupquote{ALL}}. If the \sphinxcode{\sphinxupquote{direction}} is \sphinxcode{\sphinxupquote{None}}
and also directed evidence(s) match the criteria these will
overwrite the undirected evidences and only the directed result
will be returned.

\end{description}\end{quote}

\end{fulllineitems}

\index{degrees\_negative\_by\_interaction\_type\_and\_data\_model\_and\_resource() (pypath.core.interaction.Interaction method)@\spxentry{degrees\_negative\_by\_interaction\_type\_and\_data\_model\_and\_resource()}\spxextra{pypath.core.interaction.Interaction method}}

\begin{fulllineitems}
\phantomsection\label{\detokenize{reference:pypath.core.interaction.Interaction.degrees_negative_by_interaction_type_and_data_model_and_resource}}\pysiglinewithargsret{\sphinxbfcode{\sphinxupquote{degrees\_negative\_by\_interaction\_type\_and\_data\_model\_and\_resource}}}{\emph{effect=None}, \emph{resources=None}, \emph{data\_model=None}, \emph{interaction\_type=None}, \emph{via=None}, \emph{references=None}}{}
Returns a \sphinxstyleemphasis{set} of nodes with the connections matching the direction,
effect and evidence criteria. E.g. if the query concerns the incoming
degrees with positive effect and the matching evidences show A
activates B, but not the other way around, only “B” will be returned.
\begin{quote}\begin{description}
\item[{Parameters}] \leavevmode
\sphinxstyleliteralstrong{\sphinxupquote{mode}} (\sphinxstyleliteralemphasis{\sphinxupquote{str}}) \textendash{} The type of degrees to be considered. Three possible values are
\sphinxcode{\sphinxupquote{'IN'}}, \sphinxtitleref{‘OUT’{}`} and \sphinxcode{\sphinxupquote{'ALL'}} for incoming, outgoing and all
connections, respectively. If the \sphinxcode{\sphinxupquote{direction}} is \sphinxcode{\sphinxupquote{False}} the
only possible mode is \sphinxcode{\sphinxupquote{ALL}}. If the \sphinxcode{\sphinxupquote{direction}} is \sphinxcode{\sphinxupquote{None}}
and also directed evidence(s) match the criteria these will
overwrite the undirected evidences and only the directed result
will be returned.

\end{description}\end{quote}

\end{fulllineitems}

\index{degrees\_negative\_by\_reference() (pypath.core.interaction.Interaction method)@\spxentry{degrees\_negative\_by\_reference()}\spxextra{pypath.core.interaction.Interaction method}}

\begin{fulllineitems}
\phantomsection\label{\detokenize{reference:pypath.core.interaction.Interaction.degrees_negative_by_reference}}\pysiglinewithargsret{\sphinxbfcode{\sphinxupquote{degrees\_negative\_by\_reference}}}{\emph{effect=None}, \emph{resources=None}, \emph{data\_model=None}, \emph{interaction\_type=None}, \emph{via=None}, \emph{references=None}}{}
Returns a \sphinxstyleemphasis{set} of nodes with the connections matching the direction,
effect and evidence criteria. E.g. if the query concerns the incoming
degrees with positive effect and the matching evidences show A
activates B, but not the other way around, only “B” will be returned.
\begin{quote}\begin{description}
\item[{Parameters}] \leavevmode
\sphinxstyleliteralstrong{\sphinxupquote{mode}} (\sphinxstyleliteralemphasis{\sphinxupquote{str}}) \textendash{} The type of degrees to be considered. Three possible values are
\sphinxcode{\sphinxupquote{'IN'}}, \sphinxtitleref{‘OUT’{}`} and \sphinxcode{\sphinxupquote{'ALL'}} for incoming, outgoing and all
connections, respectively. If the \sphinxcode{\sphinxupquote{direction}} is \sphinxcode{\sphinxupquote{False}} the
only possible mode is \sphinxcode{\sphinxupquote{ALL}}. If the \sphinxcode{\sphinxupquote{direction}} is \sphinxcode{\sphinxupquote{None}}
and also directed evidence(s) match the criteria these will
overwrite the undirected evidences and only the directed result
will be returned.

\end{description}\end{quote}

\end{fulllineitems}

\index{degrees\_negative\_by\_resource() (pypath.core.interaction.Interaction method)@\spxentry{degrees\_negative\_by\_resource()}\spxextra{pypath.core.interaction.Interaction method}}

\begin{fulllineitems}
\phantomsection\label{\detokenize{reference:pypath.core.interaction.Interaction.degrees_negative_by_resource}}\pysiglinewithargsret{\sphinxbfcode{\sphinxupquote{degrees\_negative\_by\_resource}}}{\emph{effect=None}, \emph{resources=None}, \emph{data\_model=None}, \emph{interaction\_type=None}, \emph{via=None}, \emph{references=None}}{}
Returns a \sphinxstyleemphasis{set} of nodes with the connections matching the direction,
effect and evidence criteria. E.g. if the query concerns the incoming
degrees with positive effect and the matching evidences show A
activates B, but not the other way around, only “B” will be returned.
\begin{quote}\begin{description}
\item[{Parameters}] \leavevmode
\sphinxstyleliteralstrong{\sphinxupquote{mode}} (\sphinxstyleliteralemphasis{\sphinxupquote{str}}) \textendash{} The type of degrees to be considered. Three possible values are
\sphinxcode{\sphinxupquote{'IN'}}, \sphinxtitleref{‘OUT’{}`} and \sphinxcode{\sphinxupquote{'ALL'}} for incoming, outgoing and all
connections, respectively. If the \sphinxcode{\sphinxupquote{direction}} is \sphinxcode{\sphinxupquote{False}} the
only possible mode is \sphinxcode{\sphinxupquote{ALL}}. If the \sphinxcode{\sphinxupquote{direction}} is \sphinxcode{\sphinxupquote{None}}
and also directed evidence(s) match the criteria these will
overwrite the undirected evidences and only the directed result
will be returned.

\end{description}\end{quote}

\end{fulllineitems}

\index{degrees\_negative\_in\_by\_data\_model() (pypath.core.interaction.Interaction method)@\spxentry{degrees\_negative\_in\_by\_data\_model()}\spxextra{pypath.core.interaction.Interaction method}}

\begin{fulllineitems}
\phantomsection\label{\detokenize{reference:pypath.core.interaction.Interaction.degrees_negative_in_by_data_model}}\pysiglinewithargsret{\sphinxbfcode{\sphinxupquote{degrees\_negative\_in\_by\_data\_model}}}{\emph{effect=None}, \emph{resources=None}, \emph{data\_model=None}, \emph{interaction\_type=None}, \emph{via=None}, \emph{references=None}}{}
Returns a \sphinxstyleemphasis{set} of nodes with the connections matching the direction,
effect and evidence criteria. E.g. if the query concerns the incoming
degrees with positive effect and the matching evidences show A
activates B, but not the other way around, only “B” will be returned.
\begin{quote}\begin{description}
\item[{Parameters}] \leavevmode
\sphinxstyleliteralstrong{\sphinxupquote{mode}} (\sphinxstyleliteralemphasis{\sphinxupquote{str}}) \textendash{} The type of degrees to be considered. Three possible values are
\sphinxcode{\sphinxupquote{'IN'}}, \sphinxtitleref{‘OUT’{}`} and \sphinxcode{\sphinxupquote{'ALL'}} for incoming, outgoing and all
connections, respectively. If the \sphinxcode{\sphinxupquote{direction}} is \sphinxcode{\sphinxupquote{False}} the
only possible mode is \sphinxcode{\sphinxupquote{ALL}}. If the \sphinxcode{\sphinxupquote{direction}} is \sphinxcode{\sphinxupquote{None}}
and also directed evidence(s) match the criteria these will
overwrite the undirected evidences and only the directed result
will be returned.

\end{description}\end{quote}

\end{fulllineitems}

\index{degrees\_negative\_in\_by\_interaction\_type() (pypath.core.interaction.Interaction method)@\spxentry{degrees\_negative\_in\_by\_interaction\_type()}\spxextra{pypath.core.interaction.Interaction method}}

\begin{fulllineitems}
\phantomsection\label{\detokenize{reference:pypath.core.interaction.Interaction.degrees_negative_in_by_interaction_type}}\pysiglinewithargsret{\sphinxbfcode{\sphinxupquote{degrees\_negative\_in\_by\_interaction\_type}}}{\emph{effect=None}, \emph{resources=None}, \emph{data\_model=None}, \emph{interaction\_type=None}, \emph{via=None}, \emph{references=None}}{}
Returns a \sphinxstyleemphasis{set} of nodes with the connections matching the direction,
effect and evidence criteria. E.g. if the query concerns the incoming
degrees with positive effect and the matching evidences show A
activates B, but not the other way around, only “B” will be returned.
\begin{quote}\begin{description}
\item[{Parameters}] \leavevmode
\sphinxstyleliteralstrong{\sphinxupquote{mode}} (\sphinxstyleliteralemphasis{\sphinxupquote{str}}) \textendash{} The type of degrees to be considered. Three possible values are
\sphinxcode{\sphinxupquote{'IN'}}, \sphinxtitleref{‘OUT’{}`} and \sphinxcode{\sphinxupquote{'ALL'}} for incoming, outgoing and all
connections, respectively. If the \sphinxcode{\sphinxupquote{direction}} is \sphinxcode{\sphinxupquote{False}} the
only possible mode is \sphinxcode{\sphinxupquote{ALL}}. If the \sphinxcode{\sphinxupquote{direction}} is \sphinxcode{\sphinxupquote{None}}
and also directed evidence(s) match the criteria these will
overwrite the undirected evidences and only the directed result
will be returned.

\end{description}\end{quote}

\end{fulllineitems}

\index{degrees\_negative\_in\_by\_interaction\_type\_and\_data\_model() (pypath.core.interaction.Interaction method)@\spxentry{degrees\_negative\_in\_by\_interaction\_type\_and\_data\_model()}\spxextra{pypath.core.interaction.Interaction method}}

\begin{fulllineitems}
\phantomsection\label{\detokenize{reference:pypath.core.interaction.Interaction.degrees_negative_in_by_interaction_type_and_data_model}}\pysiglinewithargsret{\sphinxbfcode{\sphinxupquote{degrees\_negative\_in\_by\_interaction\_type\_and\_data\_model}}}{\emph{effect=None}, \emph{resources=None}, \emph{data\_model=None}, \emph{interaction\_type=None}, \emph{via=None}, \emph{references=None}}{}
Returns a \sphinxstyleemphasis{set} of nodes with the connections matching the direction,
effect and evidence criteria. E.g. if the query concerns the incoming
degrees with positive effect and the matching evidences show A
activates B, but not the other way around, only “B” will be returned.
\begin{quote}\begin{description}
\item[{Parameters}] \leavevmode
\sphinxstyleliteralstrong{\sphinxupquote{mode}} (\sphinxstyleliteralemphasis{\sphinxupquote{str}}) \textendash{} The type of degrees to be considered. Three possible values are
\sphinxcode{\sphinxupquote{'IN'}}, \sphinxtitleref{‘OUT’{}`} and \sphinxcode{\sphinxupquote{'ALL'}} for incoming, outgoing and all
connections, respectively. If the \sphinxcode{\sphinxupquote{direction}} is \sphinxcode{\sphinxupquote{False}} the
only possible mode is \sphinxcode{\sphinxupquote{ALL}}. If the \sphinxcode{\sphinxupquote{direction}} is \sphinxcode{\sphinxupquote{None}}
and also directed evidence(s) match the criteria these will
overwrite the undirected evidences and only the directed result
will be returned.

\end{description}\end{quote}

\end{fulllineitems}

\index{degrees\_negative\_in\_by\_interaction\_type\_and\_data\_model\_and\_resource() (pypath.core.interaction.Interaction method)@\spxentry{degrees\_negative\_in\_by\_interaction\_type\_and\_data\_model\_and\_resource()}\spxextra{pypath.core.interaction.Interaction method}}

\begin{fulllineitems}
\phantomsection\label{\detokenize{reference:pypath.core.interaction.Interaction.degrees_negative_in_by_interaction_type_and_data_model_and_resource}}\pysiglinewithargsret{\sphinxbfcode{\sphinxupquote{degrees\_negative\_in\_by\_interaction\_type\_and\_data\_model\_and\_resource}}}{\emph{effect=None}, \emph{resources=None}, \emph{data\_model=None}, \emph{interaction\_type=None}, \emph{via=None}, \emph{references=None}}{}
Returns a \sphinxstyleemphasis{set} of nodes with the connections matching the direction,
effect and evidence criteria. E.g. if the query concerns the incoming
degrees with positive effect and the matching evidences show A
activates B, but not the other way around, only “B” will be returned.
\begin{quote}\begin{description}
\item[{Parameters}] \leavevmode
\sphinxstyleliteralstrong{\sphinxupquote{mode}} (\sphinxstyleliteralemphasis{\sphinxupquote{str}}) \textendash{} The type of degrees to be considered. Three possible values are
\sphinxcode{\sphinxupquote{'IN'}}, \sphinxtitleref{‘OUT’{}`} and \sphinxcode{\sphinxupquote{'ALL'}} for incoming, outgoing and all
connections, respectively. If the \sphinxcode{\sphinxupquote{direction}} is \sphinxcode{\sphinxupquote{False}} the
only possible mode is \sphinxcode{\sphinxupquote{ALL}}. If the \sphinxcode{\sphinxupquote{direction}} is \sphinxcode{\sphinxupquote{None}}
and also directed evidence(s) match the criteria these will
overwrite the undirected evidences and only the directed result
will be returned.

\end{description}\end{quote}

\end{fulllineitems}

\index{degrees\_negative\_in\_by\_reference() (pypath.core.interaction.Interaction method)@\spxentry{degrees\_negative\_in\_by\_reference()}\spxextra{pypath.core.interaction.Interaction method}}

\begin{fulllineitems}
\phantomsection\label{\detokenize{reference:pypath.core.interaction.Interaction.degrees_negative_in_by_reference}}\pysiglinewithargsret{\sphinxbfcode{\sphinxupquote{degrees\_negative\_in\_by\_reference}}}{\emph{effect=None}, \emph{resources=None}, \emph{data\_model=None}, \emph{interaction\_type=None}, \emph{via=None}, \emph{references=None}}{}
Returns a \sphinxstyleemphasis{set} of nodes with the connections matching the direction,
effect and evidence criteria. E.g. if the query concerns the incoming
degrees with positive effect and the matching evidences show A
activates B, but not the other way around, only “B” will be returned.
\begin{quote}\begin{description}
\item[{Parameters}] \leavevmode
\sphinxstyleliteralstrong{\sphinxupquote{mode}} (\sphinxstyleliteralemphasis{\sphinxupquote{str}}) \textendash{} The type of degrees to be considered. Three possible values are
\sphinxcode{\sphinxupquote{'IN'}}, \sphinxtitleref{‘OUT’{}`} and \sphinxcode{\sphinxupquote{'ALL'}} for incoming, outgoing and all
connections, respectively. If the \sphinxcode{\sphinxupquote{direction}} is \sphinxcode{\sphinxupquote{False}} the
only possible mode is \sphinxcode{\sphinxupquote{ALL}}. If the \sphinxcode{\sphinxupquote{direction}} is \sphinxcode{\sphinxupquote{None}}
and also directed evidence(s) match the criteria these will
overwrite the undirected evidences and only the directed result
will be returned.

\end{description}\end{quote}

\end{fulllineitems}

\index{degrees\_negative\_in\_by\_resource() (pypath.core.interaction.Interaction method)@\spxentry{degrees\_negative\_in\_by\_resource()}\spxextra{pypath.core.interaction.Interaction method}}

\begin{fulllineitems}
\phantomsection\label{\detokenize{reference:pypath.core.interaction.Interaction.degrees_negative_in_by_resource}}\pysiglinewithargsret{\sphinxbfcode{\sphinxupquote{degrees\_negative\_in\_by\_resource}}}{\emph{effect=None}, \emph{resources=None}, \emph{data\_model=None}, \emph{interaction\_type=None}, \emph{via=None}, \emph{references=None}}{}
Returns a \sphinxstyleemphasis{set} of nodes with the connections matching the direction,
effect and evidence criteria. E.g. if the query concerns the incoming
degrees with positive effect and the matching evidences show A
activates B, but not the other way around, only “B” will be returned.
\begin{quote}\begin{description}
\item[{Parameters}] \leavevmode
\sphinxstyleliteralstrong{\sphinxupquote{mode}} (\sphinxstyleliteralemphasis{\sphinxupquote{str}}) \textendash{} The type of degrees to be considered. Three possible values are
\sphinxcode{\sphinxupquote{'IN'}}, \sphinxtitleref{‘OUT’{}`} and \sphinxcode{\sphinxupquote{'ALL'}} for incoming, outgoing and all
connections, respectively. If the \sphinxcode{\sphinxupquote{direction}} is \sphinxcode{\sphinxupquote{False}} the
only possible mode is \sphinxcode{\sphinxupquote{ALL}}. If the \sphinxcode{\sphinxupquote{direction}} is \sphinxcode{\sphinxupquote{None}}
and also directed evidence(s) match the criteria these will
overwrite the undirected evidences and only the directed result
will be returned.

\end{description}\end{quote}

\end{fulllineitems}

\index{degrees\_negative\_out\_by\_data\_model() (pypath.core.interaction.Interaction method)@\spxentry{degrees\_negative\_out\_by\_data\_model()}\spxextra{pypath.core.interaction.Interaction method}}

\begin{fulllineitems}
\phantomsection\label{\detokenize{reference:pypath.core.interaction.Interaction.degrees_negative_out_by_data_model}}\pysiglinewithargsret{\sphinxbfcode{\sphinxupquote{degrees\_negative\_out\_by\_data\_model}}}{\emph{effect=None}, \emph{resources=None}, \emph{data\_model=None}, \emph{interaction\_type=None}, \emph{via=None}, \emph{references=None}}{}
Returns a \sphinxstyleemphasis{set} of nodes with the connections matching the direction,
effect and evidence criteria. E.g. if the query concerns the incoming
degrees with positive effect and the matching evidences show A
activates B, but not the other way around, only “B” will be returned.
\begin{quote}\begin{description}
\item[{Parameters}] \leavevmode
\sphinxstyleliteralstrong{\sphinxupquote{mode}} (\sphinxstyleliteralemphasis{\sphinxupquote{str}}) \textendash{} The type of degrees to be considered. Three possible values are
\sphinxcode{\sphinxupquote{'IN'}}, \sphinxtitleref{‘OUT’{}`} and \sphinxcode{\sphinxupquote{'ALL'}} for incoming, outgoing and all
connections, respectively. If the \sphinxcode{\sphinxupquote{direction}} is \sphinxcode{\sphinxupquote{False}} the
only possible mode is \sphinxcode{\sphinxupquote{ALL}}. If the \sphinxcode{\sphinxupquote{direction}} is \sphinxcode{\sphinxupquote{None}}
and also directed evidence(s) match the criteria these will
overwrite the undirected evidences and only the directed result
will be returned.

\end{description}\end{quote}

\end{fulllineitems}

\index{degrees\_negative\_out\_by\_interaction\_type() (pypath.core.interaction.Interaction method)@\spxentry{degrees\_negative\_out\_by\_interaction\_type()}\spxextra{pypath.core.interaction.Interaction method}}

\begin{fulllineitems}
\phantomsection\label{\detokenize{reference:pypath.core.interaction.Interaction.degrees_negative_out_by_interaction_type}}\pysiglinewithargsret{\sphinxbfcode{\sphinxupquote{degrees\_negative\_out\_by\_interaction\_type}}}{\emph{effect=None}, \emph{resources=None}, \emph{data\_model=None}, \emph{interaction\_type=None}, \emph{via=None}, \emph{references=None}}{}
Returns a \sphinxstyleemphasis{set} of nodes with the connections matching the direction,
effect and evidence criteria. E.g. if the query concerns the incoming
degrees with positive effect and the matching evidences show A
activates B, but not the other way around, only “B” will be returned.
\begin{quote}\begin{description}
\item[{Parameters}] \leavevmode
\sphinxstyleliteralstrong{\sphinxupquote{mode}} (\sphinxstyleliteralemphasis{\sphinxupquote{str}}) \textendash{} The type of degrees to be considered. Three possible values are
\sphinxcode{\sphinxupquote{'IN'}}, \sphinxtitleref{‘OUT’{}`} and \sphinxcode{\sphinxupquote{'ALL'}} for incoming, outgoing and all
connections, respectively. If the \sphinxcode{\sphinxupquote{direction}} is \sphinxcode{\sphinxupquote{False}} the
only possible mode is \sphinxcode{\sphinxupquote{ALL}}. If the \sphinxcode{\sphinxupquote{direction}} is \sphinxcode{\sphinxupquote{None}}
and also directed evidence(s) match the criteria these will
overwrite the undirected evidences and only the directed result
will be returned.

\end{description}\end{quote}

\end{fulllineitems}

\index{degrees\_negative\_out\_by\_interaction\_type\_and\_data\_model() (pypath.core.interaction.Interaction method)@\spxentry{degrees\_negative\_out\_by\_interaction\_type\_and\_data\_model()}\spxextra{pypath.core.interaction.Interaction method}}

\begin{fulllineitems}
\phantomsection\label{\detokenize{reference:pypath.core.interaction.Interaction.degrees_negative_out_by_interaction_type_and_data_model}}\pysiglinewithargsret{\sphinxbfcode{\sphinxupquote{degrees\_negative\_out\_by\_interaction\_type\_and\_data\_model}}}{\emph{effect=None}, \emph{resources=None}, \emph{data\_model=None}, \emph{interaction\_type=None}, \emph{via=None}, \emph{references=None}}{}
Returns a \sphinxstyleemphasis{set} of nodes with the connections matching the direction,
effect and evidence criteria. E.g. if the query concerns the incoming
degrees with positive effect and the matching evidences show A
activates B, but not the other way around, only “B” will be returned.
\begin{quote}\begin{description}
\item[{Parameters}] \leavevmode
\sphinxstyleliteralstrong{\sphinxupquote{mode}} (\sphinxstyleliteralemphasis{\sphinxupquote{str}}) \textendash{} The type of degrees to be considered. Three possible values are
\sphinxcode{\sphinxupquote{'IN'}}, \sphinxtitleref{‘OUT’{}`} and \sphinxcode{\sphinxupquote{'ALL'}} for incoming, outgoing and all
connections, respectively. If the \sphinxcode{\sphinxupquote{direction}} is \sphinxcode{\sphinxupquote{False}} the
only possible mode is \sphinxcode{\sphinxupquote{ALL}}. If the \sphinxcode{\sphinxupquote{direction}} is \sphinxcode{\sphinxupquote{None}}
and also directed evidence(s) match the criteria these will
overwrite the undirected evidences and only the directed result
will be returned.

\end{description}\end{quote}

\end{fulllineitems}

\index{degrees\_negative\_out\_by\_interaction\_type\_and\_data\_model\_and\_resource() (pypath.core.interaction.Interaction method)@\spxentry{degrees\_negative\_out\_by\_interaction\_type\_and\_data\_model\_and\_resource()}\spxextra{pypath.core.interaction.Interaction method}}

\begin{fulllineitems}
\phantomsection\label{\detokenize{reference:pypath.core.interaction.Interaction.degrees_negative_out_by_interaction_type_and_data_model_and_resource}}\pysiglinewithargsret{\sphinxbfcode{\sphinxupquote{degrees\_negative\_out\_by\_interaction\_type\_and\_data\_model\_and\_resource}}}{\emph{effect=None}, \emph{resources=None}, \emph{data\_model=None}, \emph{interaction\_type=None}, \emph{via=None}, \emph{references=None}}{}
Returns a \sphinxstyleemphasis{set} of nodes with the connections matching the direction,
effect and evidence criteria. E.g. if the query concerns the incoming
degrees with positive effect and the matching evidences show A
activates B, but not the other way around, only “B” will be returned.
\begin{quote}\begin{description}
\item[{Parameters}] \leavevmode
\sphinxstyleliteralstrong{\sphinxupquote{mode}} (\sphinxstyleliteralemphasis{\sphinxupquote{str}}) \textendash{} The type of degrees to be considered. Three possible values are
\sphinxcode{\sphinxupquote{'IN'}}, \sphinxtitleref{‘OUT’{}`} and \sphinxcode{\sphinxupquote{'ALL'}} for incoming, outgoing and all
connections, respectively. If the \sphinxcode{\sphinxupquote{direction}} is \sphinxcode{\sphinxupquote{False}} the
only possible mode is \sphinxcode{\sphinxupquote{ALL}}. If the \sphinxcode{\sphinxupquote{direction}} is \sphinxcode{\sphinxupquote{None}}
and also directed evidence(s) match the criteria these will
overwrite the undirected evidences and only the directed result
will be returned.

\end{description}\end{quote}

\end{fulllineitems}

\index{degrees\_negative\_out\_by\_reference() (pypath.core.interaction.Interaction method)@\spxentry{degrees\_negative\_out\_by\_reference()}\spxextra{pypath.core.interaction.Interaction method}}

\begin{fulllineitems}
\phantomsection\label{\detokenize{reference:pypath.core.interaction.Interaction.degrees_negative_out_by_reference}}\pysiglinewithargsret{\sphinxbfcode{\sphinxupquote{degrees\_negative\_out\_by\_reference}}}{\emph{effect=None}, \emph{resources=None}, \emph{data\_model=None}, \emph{interaction\_type=None}, \emph{via=None}, \emph{references=None}}{}
Returns a \sphinxstyleemphasis{set} of nodes with the connections matching the direction,
effect and evidence criteria. E.g. if the query concerns the incoming
degrees with positive effect and the matching evidences show A
activates B, but not the other way around, only “B” will be returned.
\begin{quote}\begin{description}
\item[{Parameters}] \leavevmode
\sphinxstyleliteralstrong{\sphinxupquote{mode}} (\sphinxstyleliteralemphasis{\sphinxupquote{str}}) \textendash{} The type of degrees to be considered. Three possible values are
\sphinxcode{\sphinxupquote{'IN'}}, \sphinxtitleref{‘OUT’{}`} and \sphinxcode{\sphinxupquote{'ALL'}} for incoming, outgoing and all
connections, respectively. If the \sphinxcode{\sphinxupquote{direction}} is \sphinxcode{\sphinxupquote{False}} the
only possible mode is \sphinxcode{\sphinxupquote{ALL}}. If the \sphinxcode{\sphinxupquote{direction}} is \sphinxcode{\sphinxupquote{None}}
and also directed evidence(s) match the criteria these will
overwrite the undirected evidences and only the directed result
will be returned.

\end{description}\end{quote}

\end{fulllineitems}

\index{degrees\_negative\_out\_by\_resource() (pypath.core.interaction.Interaction method)@\spxentry{degrees\_negative\_out\_by\_resource()}\spxextra{pypath.core.interaction.Interaction method}}

\begin{fulllineitems}
\phantomsection\label{\detokenize{reference:pypath.core.interaction.Interaction.degrees_negative_out_by_resource}}\pysiglinewithargsret{\sphinxbfcode{\sphinxupquote{degrees\_negative\_out\_by\_resource}}}{\emph{effect=None}, \emph{resources=None}, \emph{data\_model=None}, \emph{interaction\_type=None}, \emph{via=None}, \emph{references=None}}{}
Returns a \sphinxstyleemphasis{set} of nodes with the connections matching the direction,
effect and evidence criteria. E.g. if the query concerns the incoming
degrees with positive effect and the matching evidences show A
activates B, but not the other way around, only “B” will be returned.
\begin{quote}\begin{description}
\item[{Parameters}] \leavevmode
\sphinxstyleliteralstrong{\sphinxupquote{mode}} (\sphinxstyleliteralemphasis{\sphinxupquote{str}}) \textendash{} The type of degrees to be considered. Three possible values are
\sphinxcode{\sphinxupquote{'IN'}}, \sphinxtitleref{‘OUT’{}`} and \sphinxcode{\sphinxupquote{'ALL'}} for incoming, outgoing and all
connections, respectively. If the \sphinxcode{\sphinxupquote{direction}} is \sphinxcode{\sphinxupquote{False}} the
only possible mode is \sphinxcode{\sphinxupquote{ALL}}. If the \sphinxcode{\sphinxupquote{direction}} is \sphinxcode{\sphinxupquote{None}}
and also directed evidence(s) match the criteria these will
overwrite the undirected evidences and only the directed result
will be returned.

\end{description}\end{quote}

\end{fulllineitems}

\index{degrees\_non\_directed\_by\_data\_model() (pypath.core.interaction.Interaction method)@\spxentry{degrees\_non\_directed\_by\_data\_model()}\spxextra{pypath.core.interaction.Interaction method}}

\begin{fulllineitems}
\phantomsection\label{\detokenize{reference:pypath.core.interaction.Interaction.degrees_non_directed_by_data_model}}\pysiglinewithargsret{\sphinxbfcode{\sphinxupquote{degrees\_non\_directed\_by\_data\_model}}}{\emph{effect=None}, \emph{resources=None}, \emph{data\_model=None}, \emph{interaction\_type=None}, \emph{via=None}, \emph{references=None}}{}
Returns a \sphinxstyleemphasis{set} of nodes with the connections matching the direction,
effect and evidence criteria. E.g. if the query concerns the incoming
degrees with positive effect and the matching evidences show A
activates B, but not the other way around, only “B” will be returned.
\begin{quote}\begin{description}
\item[{Parameters}] \leavevmode
\sphinxstyleliteralstrong{\sphinxupquote{mode}} (\sphinxstyleliteralemphasis{\sphinxupquote{str}}) \textendash{} The type of degrees to be considered. Three possible values are
\sphinxcode{\sphinxupquote{'IN'}}, \sphinxtitleref{‘OUT’{}`} and \sphinxcode{\sphinxupquote{'ALL'}} for incoming, outgoing and all
connections, respectively. If the \sphinxcode{\sphinxupquote{direction}} is \sphinxcode{\sphinxupquote{False}} the
only possible mode is \sphinxcode{\sphinxupquote{ALL}}. If the \sphinxcode{\sphinxupquote{direction}} is \sphinxcode{\sphinxupquote{None}}
and also directed evidence(s) match the criteria these will
overwrite the undirected evidences and only the directed result
will be returned.

\end{description}\end{quote}

\end{fulllineitems}

\index{degrees\_non\_directed\_by\_interaction\_type() (pypath.core.interaction.Interaction method)@\spxentry{degrees\_non\_directed\_by\_interaction\_type()}\spxextra{pypath.core.interaction.Interaction method}}

\begin{fulllineitems}
\phantomsection\label{\detokenize{reference:pypath.core.interaction.Interaction.degrees_non_directed_by_interaction_type}}\pysiglinewithargsret{\sphinxbfcode{\sphinxupquote{degrees\_non\_directed\_by\_interaction\_type}}}{\emph{effect=None}, \emph{resources=None}, \emph{data\_model=None}, \emph{interaction\_type=None}, \emph{via=None}, \emph{references=None}}{}
Returns a \sphinxstyleemphasis{set} of nodes with the connections matching the direction,
effect and evidence criteria. E.g. if the query concerns the incoming
degrees with positive effect and the matching evidences show A
activates B, but not the other way around, only “B” will be returned.
\begin{quote}\begin{description}
\item[{Parameters}] \leavevmode
\sphinxstyleliteralstrong{\sphinxupquote{mode}} (\sphinxstyleliteralemphasis{\sphinxupquote{str}}) \textendash{} The type of degrees to be considered. Three possible values are
\sphinxcode{\sphinxupquote{'IN'}}, \sphinxtitleref{‘OUT’{}`} and \sphinxcode{\sphinxupquote{'ALL'}} for incoming, outgoing and all
connections, respectively. If the \sphinxcode{\sphinxupquote{direction}} is \sphinxcode{\sphinxupquote{False}} the
only possible mode is \sphinxcode{\sphinxupquote{ALL}}. If the \sphinxcode{\sphinxupquote{direction}} is \sphinxcode{\sphinxupquote{None}}
and also directed evidence(s) match the criteria these will
overwrite the undirected evidences and only the directed result
will be returned.

\end{description}\end{quote}

\end{fulllineitems}

\index{degrees\_non\_directed\_by\_interaction\_type\_and\_data\_model() (pypath.core.interaction.Interaction method)@\spxentry{degrees\_non\_directed\_by\_interaction\_type\_and\_data\_model()}\spxextra{pypath.core.interaction.Interaction method}}

\begin{fulllineitems}
\phantomsection\label{\detokenize{reference:pypath.core.interaction.Interaction.degrees_non_directed_by_interaction_type_and_data_model}}\pysiglinewithargsret{\sphinxbfcode{\sphinxupquote{degrees\_non\_directed\_by\_interaction\_type\_and\_data\_model}}}{\emph{effect=None}, \emph{resources=None}, \emph{data\_model=None}, \emph{interaction\_type=None}, \emph{via=None}, \emph{references=None}}{}
Returns a \sphinxstyleemphasis{set} of nodes with the connections matching the direction,
effect and evidence criteria. E.g. if the query concerns the incoming
degrees with positive effect and the matching evidences show A
activates B, but not the other way around, only “B” will be returned.
\begin{quote}\begin{description}
\item[{Parameters}] \leavevmode
\sphinxstyleliteralstrong{\sphinxupquote{mode}} (\sphinxstyleliteralemphasis{\sphinxupquote{str}}) \textendash{} The type of degrees to be considered. Three possible values are
\sphinxcode{\sphinxupquote{'IN'}}, \sphinxtitleref{‘OUT’{}`} and \sphinxcode{\sphinxupquote{'ALL'}} for incoming, outgoing and all
connections, respectively. If the \sphinxcode{\sphinxupquote{direction}} is \sphinxcode{\sphinxupquote{False}} the
only possible mode is \sphinxcode{\sphinxupquote{ALL}}. If the \sphinxcode{\sphinxupquote{direction}} is \sphinxcode{\sphinxupquote{None}}
and also directed evidence(s) match the criteria these will
overwrite the undirected evidences and only the directed result
will be returned.

\end{description}\end{quote}

\end{fulllineitems}

\index{degrees\_non\_directed\_by\_interaction\_type\_and\_data\_model\_and\_resource() (pypath.core.interaction.Interaction method)@\spxentry{degrees\_non\_directed\_by\_interaction\_type\_and\_data\_model\_and\_resource()}\spxextra{pypath.core.interaction.Interaction method}}

\begin{fulllineitems}
\phantomsection\label{\detokenize{reference:pypath.core.interaction.Interaction.degrees_non_directed_by_interaction_type_and_data_model_and_resource}}\pysiglinewithargsret{\sphinxbfcode{\sphinxupquote{degrees\_non\_directed\_by\_interaction\_type\_and\_data\_model\_and\_resource}}}{\emph{effect=None}, \emph{resources=None}, \emph{data\_model=None}, \emph{interaction\_type=None}, \emph{via=None}, \emph{references=None}}{}
Returns a \sphinxstyleemphasis{set} of nodes with the connections matching the direction,
effect and evidence criteria. E.g. if the query concerns the incoming
degrees with positive effect and the matching evidences show A
activates B, but not the other way around, only “B” will be returned.
\begin{quote}\begin{description}
\item[{Parameters}] \leavevmode
\sphinxstyleliteralstrong{\sphinxupquote{mode}} (\sphinxstyleliteralemphasis{\sphinxupquote{str}}) \textendash{} The type of degrees to be considered. Three possible values are
\sphinxcode{\sphinxupquote{'IN'}}, \sphinxtitleref{‘OUT’{}`} and \sphinxcode{\sphinxupquote{'ALL'}} for incoming, outgoing and all
connections, respectively. If the \sphinxcode{\sphinxupquote{direction}} is \sphinxcode{\sphinxupquote{False}} the
only possible mode is \sphinxcode{\sphinxupquote{ALL}}. If the \sphinxcode{\sphinxupquote{direction}} is \sphinxcode{\sphinxupquote{None}}
and also directed evidence(s) match the criteria these will
overwrite the undirected evidences and only the directed result
will be returned.

\end{description}\end{quote}

\end{fulllineitems}

\index{degrees\_non\_directed\_by\_reference() (pypath.core.interaction.Interaction method)@\spxentry{degrees\_non\_directed\_by\_reference()}\spxextra{pypath.core.interaction.Interaction method}}

\begin{fulllineitems}
\phantomsection\label{\detokenize{reference:pypath.core.interaction.Interaction.degrees_non_directed_by_reference}}\pysiglinewithargsret{\sphinxbfcode{\sphinxupquote{degrees\_non\_directed\_by\_reference}}}{\emph{effect=None}, \emph{resources=None}, \emph{data\_model=None}, \emph{interaction\_type=None}, \emph{via=None}, \emph{references=None}}{}
Returns a \sphinxstyleemphasis{set} of nodes with the connections matching the direction,
effect and evidence criteria. E.g. if the query concerns the incoming
degrees with positive effect and the matching evidences show A
activates B, but not the other way around, only “B” will be returned.
\begin{quote}\begin{description}
\item[{Parameters}] \leavevmode
\sphinxstyleliteralstrong{\sphinxupquote{mode}} (\sphinxstyleliteralemphasis{\sphinxupquote{str}}) \textendash{} The type of degrees to be considered. Three possible values are
\sphinxcode{\sphinxupquote{'IN'}}, \sphinxtitleref{‘OUT’{}`} and \sphinxcode{\sphinxupquote{'ALL'}} for incoming, outgoing and all
connections, respectively. If the \sphinxcode{\sphinxupquote{direction}} is \sphinxcode{\sphinxupquote{False}} the
only possible mode is \sphinxcode{\sphinxupquote{ALL}}. If the \sphinxcode{\sphinxupquote{direction}} is \sphinxcode{\sphinxupquote{None}}
and also directed evidence(s) match the criteria these will
overwrite the undirected evidences and only the directed result
will be returned.

\end{description}\end{quote}

\end{fulllineitems}

\index{degrees\_non\_directed\_by\_resource() (pypath.core.interaction.Interaction method)@\spxentry{degrees\_non\_directed\_by\_resource()}\spxextra{pypath.core.interaction.Interaction method}}

\begin{fulllineitems}
\phantomsection\label{\detokenize{reference:pypath.core.interaction.Interaction.degrees_non_directed_by_resource}}\pysiglinewithargsret{\sphinxbfcode{\sphinxupquote{degrees\_non\_directed\_by\_resource}}}{\emph{effect=None}, \emph{resources=None}, \emph{data\_model=None}, \emph{interaction\_type=None}, \emph{via=None}, \emph{references=None}}{}
Returns a \sphinxstyleemphasis{set} of nodes with the connections matching the direction,
effect and evidence criteria. E.g. if the query concerns the incoming
degrees with positive effect and the matching evidences show A
activates B, but not the other way around, only “B” will be returned.
\begin{quote}\begin{description}
\item[{Parameters}] \leavevmode
\sphinxstyleliteralstrong{\sphinxupquote{mode}} (\sphinxstyleliteralemphasis{\sphinxupquote{str}}) \textendash{} The type of degrees to be considered. Three possible values are
\sphinxcode{\sphinxupquote{'IN'}}, \sphinxtitleref{‘OUT’{}`} and \sphinxcode{\sphinxupquote{'ALL'}} for incoming, outgoing and all
connections, respectively. If the \sphinxcode{\sphinxupquote{direction}} is \sphinxcode{\sphinxupquote{False}} the
only possible mode is \sphinxcode{\sphinxupquote{ALL}}. If the \sphinxcode{\sphinxupquote{direction}} is \sphinxcode{\sphinxupquote{None}}
and also directed evidence(s) match the criteria these will
overwrite the undirected evidences and only the directed result
will be returned.

\end{description}\end{quote}

\end{fulllineitems}

\index{degrees\_positive\_by\_data\_model() (pypath.core.interaction.Interaction method)@\spxentry{degrees\_positive\_by\_data\_model()}\spxextra{pypath.core.interaction.Interaction method}}

\begin{fulllineitems}
\phantomsection\label{\detokenize{reference:pypath.core.interaction.Interaction.degrees_positive_by_data_model}}\pysiglinewithargsret{\sphinxbfcode{\sphinxupquote{degrees\_positive\_by\_data\_model}}}{\emph{effect=None}, \emph{resources=None}, \emph{data\_model=None}, \emph{interaction\_type=None}, \emph{via=None}, \emph{references=None}}{}
Returns a \sphinxstyleemphasis{set} of nodes with the connections matching the direction,
effect and evidence criteria. E.g. if the query concerns the incoming
degrees with positive effect and the matching evidences show A
activates B, but not the other way around, only “B” will be returned.
\begin{quote}\begin{description}
\item[{Parameters}] \leavevmode
\sphinxstyleliteralstrong{\sphinxupquote{mode}} (\sphinxstyleliteralemphasis{\sphinxupquote{str}}) \textendash{} The type of degrees to be considered. Three possible values are
\sphinxcode{\sphinxupquote{'IN'}}, \sphinxtitleref{‘OUT’{}`} and \sphinxcode{\sphinxupquote{'ALL'}} for incoming, outgoing and all
connections, respectively. If the \sphinxcode{\sphinxupquote{direction}} is \sphinxcode{\sphinxupquote{False}} the
only possible mode is \sphinxcode{\sphinxupquote{ALL}}. If the \sphinxcode{\sphinxupquote{direction}} is \sphinxcode{\sphinxupquote{None}}
and also directed evidence(s) match the criteria these will
overwrite the undirected evidences and only the directed result
will be returned.

\end{description}\end{quote}

\end{fulllineitems}

\index{degrees\_positive\_by\_interaction\_type() (pypath.core.interaction.Interaction method)@\spxentry{degrees\_positive\_by\_interaction\_type()}\spxextra{pypath.core.interaction.Interaction method}}

\begin{fulllineitems}
\phantomsection\label{\detokenize{reference:pypath.core.interaction.Interaction.degrees_positive_by_interaction_type}}\pysiglinewithargsret{\sphinxbfcode{\sphinxupquote{degrees\_positive\_by\_interaction\_type}}}{\emph{effect=None}, \emph{resources=None}, \emph{data\_model=None}, \emph{interaction\_type=None}, \emph{via=None}, \emph{references=None}}{}
Returns a \sphinxstyleemphasis{set} of nodes with the connections matching the direction,
effect and evidence criteria. E.g. if the query concerns the incoming
degrees with positive effect and the matching evidences show A
activates B, but not the other way around, only “B” will be returned.
\begin{quote}\begin{description}
\item[{Parameters}] \leavevmode
\sphinxstyleliteralstrong{\sphinxupquote{mode}} (\sphinxstyleliteralemphasis{\sphinxupquote{str}}) \textendash{} The type of degrees to be considered. Three possible values are
\sphinxcode{\sphinxupquote{'IN'}}, \sphinxtitleref{‘OUT’{}`} and \sphinxcode{\sphinxupquote{'ALL'}} for incoming, outgoing and all
connections, respectively. If the \sphinxcode{\sphinxupquote{direction}} is \sphinxcode{\sphinxupquote{False}} the
only possible mode is \sphinxcode{\sphinxupquote{ALL}}. If the \sphinxcode{\sphinxupquote{direction}} is \sphinxcode{\sphinxupquote{None}}
and also directed evidence(s) match the criteria these will
overwrite the undirected evidences and only the directed result
will be returned.

\end{description}\end{quote}

\end{fulllineitems}

\index{degrees\_positive\_by\_interaction\_type\_and\_data\_model() (pypath.core.interaction.Interaction method)@\spxentry{degrees\_positive\_by\_interaction\_type\_and\_data\_model()}\spxextra{pypath.core.interaction.Interaction method}}

\begin{fulllineitems}
\phantomsection\label{\detokenize{reference:pypath.core.interaction.Interaction.degrees_positive_by_interaction_type_and_data_model}}\pysiglinewithargsret{\sphinxbfcode{\sphinxupquote{degrees\_positive\_by\_interaction\_type\_and\_data\_model}}}{\emph{effect=None}, \emph{resources=None}, \emph{data\_model=None}, \emph{interaction\_type=None}, \emph{via=None}, \emph{references=None}}{}
Returns a \sphinxstyleemphasis{set} of nodes with the connections matching the direction,
effect and evidence criteria. E.g. if the query concerns the incoming
degrees with positive effect and the matching evidences show A
activates B, but not the other way around, only “B” will be returned.
\begin{quote}\begin{description}
\item[{Parameters}] \leavevmode
\sphinxstyleliteralstrong{\sphinxupquote{mode}} (\sphinxstyleliteralemphasis{\sphinxupquote{str}}) \textendash{} The type of degrees to be considered. Three possible values are
\sphinxcode{\sphinxupquote{'IN'}}, \sphinxtitleref{‘OUT’{}`} and \sphinxcode{\sphinxupquote{'ALL'}} for incoming, outgoing and all
connections, respectively. If the \sphinxcode{\sphinxupquote{direction}} is \sphinxcode{\sphinxupquote{False}} the
only possible mode is \sphinxcode{\sphinxupquote{ALL}}. If the \sphinxcode{\sphinxupquote{direction}} is \sphinxcode{\sphinxupquote{None}}
and also directed evidence(s) match the criteria these will
overwrite the undirected evidences and only the directed result
will be returned.

\end{description}\end{quote}

\end{fulllineitems}

\index{degrees\_positive\_by\_interaction\_type\_and\_data\_model\_and\_resource() (pypath.core.interaction.Interaction method)@\spxentry{degrees\_positive\_by\_interaction\_type\_and\_data\_model\_and\_resource()}\spxextra{pypath.core.interaction.Interaction method}}

\begin{fulllineitems}
\phantomsection\label{\detokenize{reference:pypath.core.interaction.Interaction.degrees_positive_by_interaction_type_and_data_model_and_resource}}\pysiglinewithargsret{\sphinxbfcode{\sphinxupquote{degrees\_positive\_by\_interaction\_type\_and\_data\_model\_and\_resource}}}{\emph{effect=None}, \emph{resources=None}, \emph{data\_model=None}, \emph{interaction\_type=None}, \emph{via=None}, \emph{references=None}}{}
Returns a \sphinxstyleemphasis{set} of nodes with the connections matching the direction,
effect and evidence criteria. E.g. if the query concerns the incoming
degrees with positive effect and the matching evidences show A
activates B, but not the other way around, only “B” will be returned.
\begin{quote}\begin{description}
\item[{Parameters}] \leavevmode
\sphinxstyleliteralstrong{\sphinxupquote{mode}} (\sphinxstyleliteralemphasis{\sphinxupquote{str}}) \textendash{} The type of degrees to be considered. Three possible values are
\sphinxcode{\sphinxupquote{'IN'}}, \sphinxtitleref{‘OUT’{}`} and \sphinxcode{\sphinxupquote{'ALL'}} for incoming, outgoing and all
connections, respectively. If the \sphinxcode{\sphinxupquote{direction}} is \sphinxcode{\sphinxupquote{False}} the
only possible mode is \sphinxcode{\sphinxupquote{ALL}}. If the \sphinxcode{\sphinxupquote{direction}} is \sphinxcode{\sphinxupquote{None}}
and also directed evidence(s) match the criteria these will
overwrite the undirected evidences and only the directed result
will be returned.

\end{description}\end{quote}

\end{fulllineitems}

\index{degrees\_positive\_by\_reference() (pypath.core.interaction.Interaction method)@\spxentry{degrees\_positive\_by\_reference()}\spxextra{pypath.core.interaction.Interaction method}}

\begin{fulllineitems}
\phantomsection\label{\detokenize{reference:pypath.core.interaction.Interaction.degrees_positive_by_reference}}\pysiglinewithargsret{\sphinxbfcode{\sphinxupquote{degrees\_positive\_by\_reference}}}{\emph{effect=None}, \emph{resources=None}, \emph{data\_model=None}, \emph{interaction\_type=None}, \emph{via=None}, \emph{references=None}}{}
Returns a \sphinxstyleemphasis{set} of nodes with the connections matching the direction,
effect and evidence criteria. E.g. if the query concerns the incoming
degrees with positive effect and the matching evidences show A
activates B, but not the other way around, only “B” will be returned.
\begin{quote}\begin{description}
\item[{Parameters}] \leavevmode
\sphinxstyleliteralstrong{\sphinxupquote{mode}} (\sphinxstyleliteralemphasis{\sphinxupquote{str}}) \textendash{} The type of degrees to be considered. Three possible values are
\sphinxcode{\sphinxupquote{'IN'}}, \sphinxtitleref{‘OUT’{}`} and \sphinxcode{\sphinxupquote{'ALL'}} for incoming, outgoing and all
connections, respectively. If the \sphinxcode{\sphinxupquote{direction}} is \sphinxcode{\sphinxupquote{False}} the
only possible mode is \sphinxcode{\sphinxupquote{ALL}}. If the \sphinxcode{\sphinxupquote{direction}} is \sphinxcode{\sphinxupquote{None}}
and also directed evidence(s) match the criteria these will
overwrite the undirected evidences and only the directed result
will be returned.

\end{description}\end{quote}

\end{fulllineitems}

\index{degrees\_positive\_by\_resource() (pypath.core.interaction.Interaction method)@\spxentry{degrees\_positive\_by\_resource()}\spxextra{pypath.core.interaction.Interaction method}}

\begin{fulllineitems}
\phantomsection\label{\detokenize{reference:pypath.core.interaction.Interaction.degrees_positive_by_resource}}\pysiglinewithargsret{\sphinxbfcode{\sphinxupquote{degrees\_positive\_by\_resource}}}{\emph{effect=None}, \emph{resources=None}, \emph{data\_model=None}, \emph{interaction\_type=None}, \emph{via=None}, \emph{references=None}}{}
Returns a \sphinxstyleemphasis{set} of nodes with the connections matching the direction,
effect and evidence criteria. E.g. if the query concerns the incoming
degrees with positive effect and the matching evidences show A
activates B, but not the other way around, only “B” will be returned.
\begin{quote}\begin{description}
\item[{Parameters}] \leavevmode
\sphinxstyleliteralstrong{\sphinxupquote{mode}} (\sphinxstyleliteralemphasis{\sphinxupquote{str}}) \textendash{} The type of degrees to be considered. Three possible values are
\sphinxcode{\sphinxupquote{'IN'}}, \sphinxtitleref{‘OUT’{}`} and \sphinxcode{\sphinxupquote{'ALL'}} for incoming, outgoing and all
connections, respectively. If the \sphinxcode{\sphinxupquote{direction}} is \sphinxcode{\sphinxupquote{False}} the
only possible mode is \sphinxcode{\sphinxupquote{ALL}}. If the \sphinxcode{\sphinxupquote{direction}} is \sphinxcode{\sphinxupquote{None}}
and also directed evidence(s) match the criteria these will
overwrite the undirected evidences and only the directed result
will be returned.

\end{description}\end{quote}

\end{fulllineitems}

\index{degrees\_positive\_in\_by\_data\_model() (pypath.core.interaction.Interaction method)@\spxentry{degrees\_positive\_in\_by\_data\_model()}\spxextra{pypath.core.interaction.Interaction method}}

\begin{fulllineitems}
\phantomsection\label{\detokenize{reference:pypath.core.interaction.Interaction.degrees_positive_in_by_data_model}}\pysiglinewithargsret{\sphinxbfcode{\sphinxupquote{degrees\_positive\_in\_by\_data\_model}}}{\emph{effect=None}, \emph{resources=None}, \emph{data\_model=None}, \emph{interaction\_type=None}, \emph{via=None}, \emph{references=None}}{}
Returns a \sphinxstyleemphasis{set} of nodes with the connections matching the direction,
effect and evidence criteria. E.g. if the query concerns the incoming
degrees with positive effect and the matching evidences show A
activates B, but not the other way around, only “B” will be returned.
\begin{quote}\begin{description}
\item[{Parameters}] \leavevmode
\sphinxstyleliteralstrong{\sphinxupquote{mode}} (\sphinxstyleliteralemphasis{\sphinxupquote{str}}) \textendash{} The type of degrees to be considered. Three possible values are
\sphinxcode{\sphinxupquote{'IN'}}, \sphinxtitleref{‘OUT’{}`} and \sphinxcode{\sphinxupquote{'ALL'}} for incoming, outgoing and all
connections, respectively. If the \sphinxcode{\sphinxupquote{direction}} is \sphinxcode{\sphinxupquote{False}} the
only possible mode is \sphinxcode{\sphinxupquote{ALL}}. If the \sphinxcode{\sphinxupquote{direction}} is \sphinxcode{\sphinxupquote{None}}
and also directed evidence(s) match the criteria these will
overwrite the undirected evidences and only the directed result
will be returned.

\end{description}\end{quote}

\end{fulllineitems}

\index{degrees\_positive\_in\_by\_interaction\_type() (pypath.core.interaction.Interaction method)@\spxentry{degrees\_positive\_in\_by\_interaction\_type()}\spxextra{pypath.core.interaction.Interaction method}}

\begin{fulllineitems}
\phantomsection\label{\detokenize{reference:pypath.core.interaction.Interaction.degrees_positive_in_by_interaction_type}}\pysiglinewithargsret{\sphinxbfcode{\sphinxupquote{degrees\_positive\_in\_by\_interaction\_type}}}{\emph{effect=None}, \emph{resources=None}, \emph{data\_model=None}, \emph{interaction\_type=None}, \emph{via=None}, \emph{references=None}}{}
Returns a \sphinxstyleemphasis{set} of nodes with the connections matching the direction,
effect and evidence criteria. E.g. if the query concerns the incoming
degrees with positive effect and the matching evidences show A
activates B, but not the other way around, only “B” will be returned.
\begin{quote}\begin{description}
\item[{Parameters}] \leavevmode
\sphinxstyleliteralstrong{\sphinxupquote{mode}} (\sphinxstyleliteralemphasis{\sphinxupquote{str}}) \textendash{} The type of degrees to be considered. Three possible values are
\sphinxcode{\sphinxupquote{'IN'}}, \sphinxtitleref{‘OUT’{}`} and \sphinxcode{\sphinxupquote{'ALL'}} for incoming, outgoing and all
connections, respectively. If the \sphinxcode{\sphinxupquote{direction}} is \sphinxcode{\sphinxupquote{False}} the
only possible mode is \sphinxcode{\sphinxupquote{ALL}}. If the \sphinxcode{\sphinxupquote{direction}} is \sphinxcode{\sphinxupquote{None}}
and also directed evidence(s) match the criteria these will
overwrite the undirected evidences and only the directed result
will be returned.

\end{description}\end{quote}

\end{fulllineitems}

\index{degrees\_positive\_in\_by\_interaction\_type\_and\_data\_model() (pypath.core.interaction.Interaction method)@\spxentry{degrees\_positive\_in\_by\_interaction\_type\_and\_data\_model()}\spxextra{pypath.core.interaction.Interaction method}}

\begin{fulllineitems}
\phantomsection\label{\detokenize{reference:pypath.core.interaction.Interaction.degrees_positive_in_by_interaction_type_and_data_model}}\pysiglinewithargsret{\sphinxbfcode{\sphinxupquote{degrees\_positive\_in\_by\_interaction\_type\_and\_data\_model}}}{\emph{effect=None}, \emph{resources=None}, \emph{data\_model=None}, \emph{interaction\_type=None}, \emph{via=None}, \emph{references=None}}{}
Returns a \sphinxstyleemphasis{set} of nodes with the connections matching the direction,
effect and evidence criteria. E.g. if the query concerns the incoming
degrees with positive effect and the matching evidences show A
activates B, but not the other way around, only “B” will be returned.
\begin{quote}\begin{description}
\item[{Parameters}] \leavevmode
\sphinxstyleliteralstrong{\sphinxupquote{mode}} (\sphinxstyleliteralemphasis{\sphinxupquote{str}}) \textendash{} The type of degrees to be considered. Three possible values are
\sphinxcode{\sphinxupquote{'IN'}}, \sphinxtitleref{‘OUT’{}`} and \sphinxcode{\sphinxupquote{'ALL'}} for incoming, outgoing and all
connections, respectively. If the \sphinxcode{\sphinxupquote{direction}} is \sphinxcode{\sphinxupquote{False}} the
only possible mode is \sphinxcode{\sphinxupquote{ALL}}. If the \sphinxcode{\sphinxupquote{direction}} is \sphinxcode{\sphinxupquote{None}}
and also directed evidence(s) match the criteria these will
overwrite the undirected evidences and only the directed result
will be returned.

\end{description}\end{quote}

\end{fulllineitems}

\index{degrees\_positive\_in\_by\_interaction\_type\_and\_data\_model\_and\_resource() (pypath.core.interaction.Interaction method)@\spxentry{degrees\_positive\_in\_by\_interaction\_type\_and\_data\_model\_and\_resource()}\spxextra{pypath.core.interaction.Interaction method}}

\begin{fulllineitems}
\phantomsection\label{\detokenize{reference:pypath.core.interaction.Interaction.degrees_positive_in_by_interaction_type_and_data_model_and_resource}}\pysiglinewithargsret{\sphinxbfcode{\sphinxupquote{degrees\_positive\_in\_by\_interaction\_type\_and\_data\_model\_and\_resource}}}{\emph{effect=None}, \emph{resources=None}, \emph{data\_model=None}, \emph{interaction\_type=None}, \emph{via=None}, \emph{references=None}}{}
Returns a \sphinxstyleemphasis{set} of nodes with the connections matching the direction,
effect and evidence criteria. E.g. if the query concerns the incoming
degrees with positive effect and the matching evidences show A
activates B, but not the other way around, only “B” will be returned.
\begin{quote}\begin{description}
\item[{Parameters}] \leavevmode
\sphinxstyleliteralstrong{\sphinxupquote{mode}} (\sphinxstyleliteralemphasis{\sphinxupquote{str}}) \textendash{} The type of degrees to be considered. Three possible values are
\sphinxcode{\sphinxupquote{'IN'}}, \sphinxtitleref{‘OUT’{}`} and \sphinxcode{\sphinxupquote{'ALL'}} for incoming, outgoing and all
connections, respectively. If the \sphinxcode{\sphinxupquote{direction}} is \sphinxcode{\sphinxupquote{False}} the
only possible mode is \sphinxcode{\sphinxupquote{ALL}}. If the \sphinxcode{\sphinxupquote{direction}} is \sphinxcode{\sphinxupquote{None}}
and also directed evidence(s) match the criteria these will
overwrite the undirected evidences and only the directed result
will be returned.

\end{description}\end{quote}

\end{fulllineitems}

\index{degrees\_positive\_in\_by\_reference() (pypath.core.interaction.Interaction method)@\spxentry{degrees\_positive\_in\_by\_reference()}\spxextra{pypath.core.interaction.Interaction method}}

\begin{fulllineitems}
\phantomsection\label{\detokenize{reference:pypath.core.interaction.Interaction.degrees_positive_in_by_reference}}\pysiglinewithargsret{\sphinxbfcode{\sphinxupquote{degrees\_positive\_in\_by\_reference}}}{\emph{effect=None}, \emph{resources=None}, \emph{data\_model=None}, \emph{interaction\_type=None}, \emph{via=None}, \emph{references=None}}{}
Returns a \sphinxstyleemphasis{set} of nodes with the connections matching the direction,
effect and evidence criteria. E.g. if the query concerns the incoming
degrees with positive effect and the matching evidences show A
activates B, but not the other way around, only “B” will be returned.
\begin{quote}\begin{description}
\item[{Parameters}] \leavevmode
\sphinxstyleliteralstrong{\sphinxupquote{mode}} (\sphinxstyleliteralemphasis{\sphinxupquote{str}}) \textendash{} The type of degrees to be considered. Three possible values are
\sphinxcode{\sphinxupquote{'IN'}}, \sphinxtitleref{‘OUT’{}`} and \sphinxcode{\sphinxupquote{'ALL'}} for incoming, outgoing and all
connections, respectively. If the \sphinxcode{\sphinxupquote{direction}} is \sphinxcode{\sphinxupquote{False}} the
only possible mode is \sphinxcode{\sphinxupquote{ALL}}. If the \sphinxcode{\sphinxupquote{direction}} is \sphinxcode{\sphinxupquote{None}}
and also directed evidence(s) match the criteria these will
overwrite the undirected evidences and only the directed result
will be returned.

\end{description}\end{quote}

\end{fulllineitems}

\index{degrees\_positive\_in\_by\_resource() (pypath.core.interaction.Interaction method)@\spxentry{degrees\_positive\_in\_by\_resource()}\spxextra{pypath.core.interaction.Interaction method}}

\begin{fulllineitems}
\phantomsection\label{\detokenize{reference:pypath.core.interaction.Interaction.degrees_positive_in_by_resource}}\pysiglinewithargsret{\sphinxbfcode{\sphinxupquote{degrees\_positive\_in\_by\_resource}}}{\emph{effect=None}, \emph{resources=None}, \emph{data\_model=None}, \emph{interaction\_type=None}, \emph{via=None}, \emph{references=None}}{}
Returns a \sphinxstyleemphasis{set} of nodes with the connections matching the direction,
effect and evidence criteria. E.g. if the query concerns the incoming
degrees with positive effect and the matching evidences show A
activates B, but not the other way around, only “B” will be returned.
\begin{quote}\begin{description}
\item[{Parameters}] \leavevmode
\sphinxstyleliteralstrong{\sphinxupquote{mode}} (\sphinxstyleliteralemphasis{\sphinxupquote{str}}) \textendash{} The type of degrees to be considered. Three possible values are
\sphinxcode{\sphinxupquote{'IN'}}, \sphinxtitleref{‘OUT’{}`} and \sphinxcode{\sphinxupquote{'ALL'}} for incoming, outgoing and all
connections, respectively. If the \sphinxcode{\sphinxupquote{direction}} is \sphinxcode{\sphinxupquote{False}} the
only possible mode is \sphinxcode{\sphinxupquote{ALL}}. If the \sphinxcode{\sphinxupquote{direction}} is \sphinxcode{\sphinxupquote{None}}
and also directed evidence(s) match the criteria these will
overwrite the undirected evidences and only the directed result
will be returned.

\end{description}\end{quote}

\end{fulllineitems}

\index{degrees\_positive\_out\_by\_data\_model() (pypath.core.interaction.Interaction method)@\spxentry{degrees\_positive\_out\_by\_data\_model()}\spxextra{pypath.core.interaction.Interaction method}}

\begin{fulllineitems}
\phantomsection\label{\detokenize{reference:pypath.core.interaction.Interaction.degrees_positive_out_by_data_model}}\pysiglinewithargsret{\sphinxbfcode{\sphinxupquote{degrees\_positive\_out\_by\_data\_model}}}{\emph{effect=None}, \emph{resources=None}, \emph{data\_model=None}, \emph{interaction\_type=None}, \emph{via=None}, \emph{references=None}}{}
Returns a \sphinxstyleemphasis{set} of nodes with the connections matching the direction,
effect and evidence criteria. E.g. if the query concerns the incoming
degrees with positive effect and the matching evidences show A
activates B, but not the other way around, only “B” will be returned.
\begin{quote}\begin{description}
\item[{Parameters}] \leavevmode
\sphinxstyleliteralstrong{\sphinxupquote{mode}} (\sphinxstyleliteralemphasis{\sphinxupquote{str}}) \textendash{} The type of degrees to be considered. Three possible values are
\sphinxcode{\sphinxupquote{'IN'}}, \sphinxtitleref{‘OUT’{}`} and \sphinxcode{\sphinxupquote{'ALL'}} for incoming, outgoing and all
connections, respectively. If the \sphinxcode{\sphinxupquote{direction}} is \sphinxcode{\sphinxupquote{False}} the
only possible mode is \sphinxcode{\sphinxupquote{ALL}}. If the \sphinxcode{\sphinxupquote{direction}} is \sphinxcode{\sphinxupquote{None}}
and also directed evidence(s) match the criteria these will
overwrite the undirected evidences and only the directed result
will be returned.

\end{description}\end{quote}

\end{fulllineitems}

\index{degrees\_positive\_out\_by\_interaction\_type() (pypath.core.interaction.Interaction method)@\spxentry{degrees\_positive\_out\_by\_interaction\_type()}\spxextra{pypath.core.interaction.Interaction method}}

\begin{fulllineitems}
\phantomsection\label{\detokenize{reference:pypath.core.interaction.Interaction.degrees_positive_out_by_interaction_type}}\pysiglinewithargsret{\sphinxbfcode{\sphinxupquote{degrees\_positive\_out\_by\_interaction\_type}}}{\emph{effect=None}, \emph{resources=None}, \emph{data\_model=None}, \emph{interaction\_type=None}, \emph{via=None}, \emph{references=None}}{}
Returns a \sphinxstyleemphasis{set} of nodes with the connections matching the direction,
effect and evidence criteria. E.g. if the query concerns the incoming
degrees with positive effect and the matching evidences show A
activates B, but not the other way around, only “B” will be returned.
\begin{quote}\begin{description}
\item[{Parameters}] \leavevmode
\sphinxstyleliteralstrong{\sphinxupquote{mode}} (\sphinxstyleliteralemphasis{\sphinxupquote{str}}) \textendash{} The type of degrees to be considered. Three possible values are
\sphinxcode{\sphinxupquote{'IN'}}, \sphinxtitleref{‘OUT’{}`} and \sphinxcode{\sphinxupquote{'ALL'}} for incoming, outgoing and all
connections, respectively. If the \sphinxcode{\sphinxupquote{direction}} is \sphinxcode{\sphinxupquote{False}} the
only possible mode is \sphinxcode{\sphinxupquote{ALL}}. If the \sphinxcode{\sphinxupquote{direction}} is \sphinxcode{\sphinxupquote{None}}
and also directed evidence(s) match the criteria these will
overwrite the undirected evidences and only the directed result
will be returned.

\end{description}\end{quote}

\end{fulllineitems}

\index{degrees\_positive\_out\_by\_interaction\_type\_and\_data\_model() (pypath.core.interaction.Interaction method)@\spxentry{degrees\_positive\_out\_by\_interaction\_type\_and\_data\_model()}\spxextra{pypath.core.interaction.Interaction method}}

\begin{fulllineitems}
\phantomsection\label{\detokenize{reference:pypath.core.interaction.Interaction.degrees_positive_out_by_interaction_type_and_data_model}}\pysiglinewithargsret{\sphinxbfcode{\sphinxupquote{degrees\_positive\_out\_by\_interaction\_type\_and\_data\_model}}}{\emph{effect=None}, \emph{resources=None}, \emph{data\_model=None}, \emph{interaction\_type=None}, \emph{via=None}, \emph{references=None}}{}
Returns a \sphinxstyleemphasis{set} of nodes with the connections matching the direction,
effect and evidence criteria. E.g. if the query concerns the incoming
degrees with positive effect and the matching evidences show A
activates B, but not the other way around, only “B” will be returned.
\begin{quote}\begin{description}
\item[{Parameters}] \leavevmode
\sphinxstyleliteralstrong{\sphinxupquote{mode}} (\sphinxstyleliteralemphasis{\sphinxupquote{str}}) \textendash{} The type of degrees to be considered. Three possible values are
\sphinxcode{\sphinxupquote{'IN'}}, \sphinxtitleref{‘OUT’{}`} and \sphinxcode{\sphinxupquote{'ALL'}} for incoming, outgoing and all
connections, respectively. If the \sphinxcode{\sphinxupquote{direction}} is \sphinxcode{\sphinxupquote{False}} the
only possible mode is \sphinxcode{\sphinxupquote{ALL}}. If the \sphinxcode{\sphinxupquote{direction}} is \sphinxcode{\sphinxupquote{None}}
and also directed evidence(s) match the criteria these will
overwrite the undirected evidences and only the directed result
will be returned.

\end{description}\end{quote}

\end{fulllineitems}

\index{degrees\_positive\_out\_by\_interaction\_type\_and\_data\_model\_and\_resource() (pypath.core.interaction.Interaction method)@\spxentry{degrees\_positive\_out\_by\_interaction\_type\_and\_data\_model\_and\_resource()}\spxextra{pypath.core.interaction.Interaction method}}

\begin{fulllineitems}
\phantomsection\label{\detokenize{reference:pypath.core.interaction.Interaction.degrees_positive_out_by_interaction_type_and_data_model_and_resource}}\pysiglinewithargsret{\sphinxbfcode{\sphinxupquote{degrees\_positive\_out\_by\_interaction\_type\_and\_data\_model\_and\_resource}}}{\emph{effect=None}, \emph{resources=None}, \emph{data\_model=None}, \emph{interaction\_type=None}, \emph{via=None}, \emph{references=None}}{}
Returns a \sphinxstyleemphasis{set} of nodes with the connections matching the direction,
effect and evidence criteria. E.g. if the query concerns the incoming
degrees with positive effect and the matching evidences show A
activates B, but not the other way around, only “B” will be returned.
\begin{quote}\begin{description}
\item[{Parameters}] \leavevmode
\sphinxstyleliteralstrong{\sphinxupquote{mode}} (\sphinxstyleliteralemphasis{\sphinxupquote{str}}) \textendash{} The type of degrees to be considered. Three possible values are
\sphinxcode{\sphinxupquote{'IN'}}, \sphinxtitleref{‘OUT’{}`} and \sphinxcode{\sphinxupquote{'ALL'}} for incoming, outgoing and all
connections, respectively. If the \sphinxcode{\sphinxupquote{direction}} is \sphinxcode{\sphinxupquote{False}} the
only possible mode is \sphinxcode{\sphinxupquote{ALL}}. If the \sphinxcode{\sphinxupquote{direction}} is \sphinxcode{\sphinxupquote{None}}
and also directed evidence(s) match the criteria these will
overwrite the undirected evidences and only the directed result
will be returned.

\end{description}\end{quote}

\end{fulllineitems}

\index{degrees\_positive\_out\_by\_reference() (pypath.core.interaction.Interaction method)@\spxentry{degrees\_positive\_out\_by\_reference()}\spxextra{pypath.core.interaction.Interaction method}}

\begin{fulllineitems}
\phantomsection\label{\detokenize{reference:pypath.core.interaction.Interaction.degrees_positive_out_by_reference}}\pysiglinewithargsret{\sphinxbfcode{\sphinxupquote{degrees\_positive\_out\_by\_reference}}}{\emph{effect=None}, \emph{resources=None}, \emph{data\_model=None}, \emph{interaction\_type=None}, \emph{via=None}, \emph{references=None}}{}
Returns a \sphinxstyleemphasis{set} of nodes with the connections matching the direction,
effect and evidence criteria. E.g. if the query concerns the incoming
degrees with positive effect and the matching evidences show A
activates B, but not the other way around, only “B” will be returned.
\begin{quote}\begin{description}
\item[{Parameters}] \leavevmode
\sphinxstyleliteralstrong{\sphinxupquote{mode}} (\sphinxstyleliteralemphasis{\sphinxupquote{str}}) \textendash{} The type of degrees to be considered. Three possible values are
\sphinxcode{\sphinxupquote{'IN'}}, \sphinxtitleref{‘OUT’{}`} and \sphinxcode{\sphinxupquote{'ALL'}} for incoming, outgoing and all
connections, respectively. If the \sphinxcode{\sphinxupquote{direction}} is \sphinxcode{\sphinxupquote{False}} the
only possible mode is \sphinxcode{\sphinxupquote{ALL}}. If the \sphinxcode{\sphinxupquote{direction}} is \sphinxcode{\sphinxupquote{None}}
and also directed evidence(s) match the criteria these will
overwrite the undirected evidences and only the directed result
will be returned.

\end{description}\end{quote}

\end{fulllineitems}

\index{degrees\_positive\_out\_by\_resource() (pypath.core.interaction.Interaction method)@\spxentry{degrees\_positive\_out\_by\_resource()}\spxextra{pypath.core.interaction.Interaction method}}

\begin{fulllineitems}
\phantomsection\label{\detokenize{reference:pypath.core.interaction.Interaction.degrees_positive_out_by_resource}}\pysiglinewithargsret{\sphinxbfcode{\sphinxupquote{degrees\_positive\_out\_by\_resource}}}{\emph{effect=None}, \emph{resources=None}, \emph{data\_model=None}, \emph{interaction\_type=None}, \emph{via=None}, \emph{references=None}}{}
Returns a \sphinxstyleemphasis{set} of nodes with the connections matching the direction,
effect and evidence criteria. E.g. if the query concerns the incoming
degrees with positive effect and the matching evidences show A
activates B, but not the other way around, only “B” will be returned.
\begin{quote}\begin{description}
\item[{Parameters}] \leavevmode
\sphinxstyleliteralstrong{\sphinxupquote{mode}} (\sphinxstyleliteralemphasis{\sphinxupquote{str}}) \textendash{} The type of degrees to be considered. Three possible values are
\sphinxcode{\sphinxupquote{'IN'}}, \sphinxtitleref{‘OUT’{}`} and \sphinxcode{\sphinxupquote{'ALL'}} for incoming, outgoing and all
connections, respectively. If the \sphinxcode{\sphinxupquote{direction}} is \sphinxcode{\sphinxupquote{False}} the
only possible mode is \sphinxcode{\sphinxupquote{ALL}}. If the \sphinxcode{\sphinxupquote{direction}} is \sphinxcode{\sphinxupquote{None}}
and also directed evidence(s) match the criteria these will
overwrite the undirected evidences and only the directed result
will be returned.

\end{description}\end{quote}

\end{fulllineitems}

\index{degrees\_signed\_by\_data\_model() (pypath.core.interaction.Interaction method)@\spxentry{degrees\_signed\_by\_data\_model()}\spxextra{pypath.core.interaction.Interaction method}}

\begin{fulllineitems}
\phantomsection\label{\detokenize{reference:pypath.core.interaction.Interaction.degrees_signed_by_data_model}}\pysiglinewithargsret{\sphinxbfcode{\sphinxupquote{degrees\_signed\_by\_data\_model}}}{\emph{effect=None}, \emph{resources=None}, \emph{data\_model=None}, \emph{interaction\_type=None}, \emph{via=None}, \emph{references=None}}{}
Returns a \sphinxstyleemphasis{set} of nodes with the connections matching the direction,
effect and evidence criteria. E.g. if the query concerns the incoming
degrees with positive effect and the matching evidences show A
activates B, but not the other way around, only “B” will be returned.
\begin{quote}\begin{description}
\item[{Parameters}] \leavevmode
\sphinxstyleliteralstrong{\sphinxupquote{mode}} (\sphinxstyleliteralemphasis{\sphinxupquote{str}}) \textendash{} The type of degrees to be considered. Three possible values are
\sphinxcode{\sphinxupquote{'IN'}}, \sphinxtitleref{‘OUT’{}`} and \sphinxcode{\sphinxupquote{'ALL'}} for incoming, outgoing and all
connections, respectively. If the \sphinxcode{\sphinxupquote{direction}} is \sphinxcode{\sphinxupquote{False}} the
only possible mode is \sphinxcode{\sphinxupquote{ALL}}. If the \sphinxcode{\sphinxupquote{direction}} is \sphinxcode{\sphinxupquote{None}}
and also directed evidence(s) match the criteria these will
overwrite the undirected evidences and only the directed result
will be returned.

\end{description}\end{quote}

\end{fulllineitems}

\index{degrees\_signed\_by\_interaction\_type() (pypath.core.interaction.Interaction method)@\spxentry{degrees\_signed\_by\_interaction\_type()}\spxextra{pypath.core.interaction.Interaction method}}

\begin{fulllineitems}
\phantomsection\label{\detokenize{reference:pypath.core.interaction.Interaction.degrees_signed_by_interaction_type}}\pysiglinewithargsret{\sphinxbfcode{\sphinxupquote{degrees\_signed\_by\_interaction\_type}}}{\emph{effect=None}, \emph{resources=None}, \emph{data\_model=None}, \emph{interaction\_type=None}, \emph{via=None}, \emph{references=None}}{}
Returns a \sphinxstyleemphasis{set} of nodes with the connections matching the direction,
effect and evidence criteria. E.g. if the query concerns the incoming
degrees with positive effect and the matching evidences show A
activates B, but not the other way around, only “B” will be returned.
\begin{quote}\begin{description}
\item[{Parameters}] \leavevmode
\sphinxstyleliteralstrong{\sphinxupquote{mode}} (\sphinxstyleliteralemphasis{\sphinxupquote{str}}) \textendash{} The type of degrees to be considered. Three possible values are
\sphinxcode{\sphinxupquote{'IN'}}, \sphinxtitleref{‘OUT’{}`} and \sphinxcode{\sphinxupquote{'ALL'}} for incoming, outgoing and all
connections, respectively. If the \sphinxcode{\sphinxupquote{direction}} is \sphinxcode{\sphinxupquote{False}} the
only possible mode is \sphinxcode{\sphinxupquote{ALL}}. If the \sphinxcode{\sphinxupquote{direction}} is \sphinxcode{\sphinxupquote{None}}
and also directed evidence(s) match the criteria these will
overwrite the undirected evidences and only the directed result
will be returned.

\end{description}\end{quote}

\end{fulllineitems}

\index{degrees\_signed\_by\_interaction\_type\_and\_data\_model() (pypath.core.interaction.Interaction method)@\spxentry{degrees\_signed\_by\_interaction\_type\_and\_data\_model()}\spxextra{pypath.core.interaction.Interaction method}}

\begin{fulllineitems}
\phantomsection\label{\detokenize{reference:pypath.core.interaction.Interaction.degrees_signed_by_interaction_type_and_data_model}}\pysiglinewithargsret{\sphinxbfcode{\sphinxupquote{degrees\_signed\_by\_interaction\_type\_and\_data\_model}}}{\emph{effect=None}, \emph{resources=None}, \emph{data\_model=None}, \emph{interaction\_type=None}, \emph{via=None}, \emph{references=None}}{}
Returns a \sphinxstyleemphasis{set} of nodes with the connections matching the direction,
effect and evidence criteria. E.g. if the query concerns the incoming
degrees with positive effect and the matching evidences show A
activates B, but not the other way around, only “B” will be returned.
\begin{quote}\begin{description}
\item[{Parameters}] \leavevmode
\sphinxstyleliteralstrong{\sphinxupquote{mode}} (\sphinxstyleliteralemphasis{\sphinxupquote{str}}) \textendash{} The type of degrees to be considered. Three possible values are
\sphinxcode{\sphinxupquote{'IN'}}, \sphinxtitleref{‘OUT’{}`} and \sphinxcode{\sphinxupquote{'ALL'}} for incoming, outgoing and all
connections, respectively. If the \sphinxcode{\sphinxupquote{direction}} is \sphinxcode{\sphinxupquote{False}} the
only possible mode is \sphinxcode{\sphinxupquote{ALL}}. If the \sphinxcode{\sphinxupquote{direction}} is \sphinxcode{\sphinxupquote{None}}
and also directed evidence(s) match the criteria these will
overwrite the undirected evidences and only the directed result
will be returned.

\end{description}\end{quote}

\end{fulllineitems}

\index{degrees\_signed\_by\_interaction\_type\_and\_data\_model\_and\_resource() (pypath.core.interaction.Interaction method)@\spxentry{degrees\_signed\_by\_interaction\_type\_and\_data\_model\_and\_resource()}\spxextra{pypath.core.interaction.Interaction method}}

\begin{fulllineitems}
\phantomsection\label{\detokenize{reference:pypath.core.interaction.Interaction.degrees_signed_by_interaction_type_and_data_model_and_resource}}\pysiglinewithargsret{\sphinxbfcode{\sphinxupquote{degrees\_signed\_by\_interaction\_type\_and\_data\_model\_and\_resource}}}{\emph{effect=None}, \emph{resources=None}, \emph{data\_model=None}, \emph{interaction\_type=None}, \emph{via=None}, \emph{references=None}}{}
Returns a \sphinxstyleemphasis{set} of nodes with the connections matching the direction,
effect and evidence criteria. E.g. if the query concerns the incoming
degrees with positive effect and the matching evidences show A
activates B, but not the other way around, only “B” will be returned.
\begin{quote}\begin{description}
\item[{Parameters}] \leavevmode
\sphinxstyleliteralstrong{\sphinxupquote{mode}} (\sphinxstyleliteralemphasis{\sphinxupquote{str}}) \textendash{} The type of degrees to be considered. Three possible values are
\sphinxcode{\sphinxupquote{'IN'}}, \sphinxtitleref{‘OUT’{}`} and \sphinxcode{\sphinxupquote{'ALL'}} for incoming, outgoing and all
connections, respectively. If the \sphinxcode{\sphinxupquote{direction}} is \sphinxcode{\sphinxupquote{False}} the
only possible mode is \sphinxcode{\sphinxupquote{ALL}}. If the \sphinxcode{\sphinxupquote{direction}} is \sphinxcode{\sphinxupquote{None}}
and also directed evidence(s) match the criteria these will
overwrite the undirected evidences and only the directed result
will be returned.

\end{description}\end{quote}

\end{fulllineitems}

\index{degrees\_signed\_by\_reference() (pypath.core.interaction.Interaction method)@\spxentry{degrees\_signed\_by\_reference()}\spxextra{pypath.core.interaction.Interaction method}}

\begin{fulllineitems}
\phantomsection\label{\detokenize{reference:pypath.core.interaction.Interaction.degrees_signed_by_reference}}\pysiglinewithargsret{\sphinxbfcode{\sphinxupquote{degrees\_signed\_by\_reference}}}{\emph{effect=None}, \emph{resources=None}, \emph{data\_model=None}, \emph{interaction\_type=None}, \emph{via=None}, \emph{references=None}}{}
Returns a \sphinxstyleemphasis{set} of nodes with the connections matching the direction,
effect and evidence criteria. E.g. if the query concerns the incoming
degrees with positive effect and the matching evidences show A
activates B, but not the other way around, only “B” will be returned.
\begin{quote}\begin{description}
\item[{Parameters}] \leavevmode
\sphinxstyleliteralstrong{\sphinxupquote{mode}} (\sphinxstyleliteralemphasis{\sphinxupquote{str}}) \textendash{} The type of degrees to be considered. Three possible values are
\sphinxcode{\sphinxupquote{'IN'}}, \sphinxtitleref{‘OUT’{}`} and \sphinxcode{\sphinxupquote{'ALL'}} for incoming, outgoing and all
connections, respectively. If the \sphinxcode{\sphinxupquote{direction}} is \sphinxcode{\sphinxupquote{False}} the
only possible mode is \sphinxcode{\sphinxupquote{ALL}}. If the \sphinxcode{\sphinxupquote{direction}} is \sphinxcode{\sphinxupquote{None}}
and also directed evidence(s) match the criteria these will
overwrite the undirected evidences and only the directed result
will be returned.

\end{description}\end{quote}

\end{fulllineitems}

\index{degrees\_signed\_by\_resource() (pypath.core.interaction.Interaction method)@\spxentry{degrees\_signed\_by\_resource()}\spxextra{pypath.core.interaction.Interaction method}}

\begin{fulllineitems}
\phantomsection\label{\detokenize{reference:pypath.core.interaction.Interaction.degrees_signed_by_resource}}\pysiglinewithargsret{\sphinxbfcode{\sphinxupquote{degrees\_signed\_by\_resource}}}{\emph{effect=None}, \emph{resources=None}, \emph{data\_model=None}, \emph{interaction\_type=None}, \emph{via=None}, \emph{references=None}}{}
Returns a \sphinxstyleemphasis{set} of nodes with the connections matching the direction,
effect and evidence criteria. E.g. if the query concerns the incoming
degrees with positive effect and the matching evidences show A
activates B, but not the other way around, only “B” will be returned.
\begin{quote}\begin{description}
\item[{Parameters}] \leavevmode
\sphinxstyleliteralstrong{\sphinxupquote{mode}} (\sphinxstyleliteralemphasis{\sphinxupquote{str}}) \textendash{} The type of degrees to be considered. Three possible values are
\sphinxcode{\sphinxupquote{'IN'}}, \sphinxtitleref{‘OUT’{}`} and \sphinxcode{\sphinxupquote{'ALL'}} for incoming, outgoing and all
connections, respectively. If the \sphinxcode{\sphinxupquote{direction}} is \sphinxcode{\sphinxupquote{False}} the
only possible mode is \sphinxcode{\sphinxupquote{ALL}}. If the \sphinxcode{\sphinxupquote{direction}} is \sphinxcode{\sphinxupquote{None}}
and also directed evidence(s) match the criteria these will
overwrite the undirected evidences and only the directed result
will be returned.

\end{description}\end{quote}

\end{fulllineitems}

\index{degrees\_signed\_in\_by\_data\_model() (pypath.core.interaction.Interaction method)@\spxentry{degrees\_signed\_in\_by\_data\_model()}\spxextra{pypath.core.interaction.Interaction method}}

\begin{fulllineitems}
\phantomsection\label{\detokenize{reference:pypath.core.interaction.Interaction.degrees_signed_in_by_data_model}}\pysiglinewithargsret{\sphinxbfcode{\sphinxupquote{degrees\_signed\_in\_by\_data\_model}}}{\emph{effect=None}, \emph{resources=None}, \emph{data\_model=None}, \emph{interaction\_type=None}, \emph{via=None}, \emph{references=None}}{}
Returns a \sphinxstyleemphasis{set} of nodes with the connections matching the direction,
effect and evidence criteria. E.g. if the query concerns the incoming
degrees with positive effect and the matching evidences show A
activates B, but not the other way around, only “B” will be returned.
\begin{quote}\begin{description}
\item[{Parameters}] \leavevmode
\sphinxstyleliteralstrong{\sphinxupquote{mode}} (\sphinxstyleliteralemphasis{\sphinxupquote{str}}) \textendash{} The type of degrees to be considered. Three possible values are
\sphinxcode{\sphinxupquote{'IN'}}, \sphinxtitleref{‘OUT’{}`} and \sphinxcode{\sphinxupquote{'ALL'}} for incoming, outgoing and all
connections, respectively. If the \sphinxcode{\sphinxupquote{direction}} is \sphinxcode{\sphinxupquote{False}} the
only possible mode is \sphinxcode{\sphinxupquote{ALL}}. If the \sphinxcode{\sphinxupquote{direction}} is \sphinxcode{\sphinxupquote{None}}
and also directed evidence(s) match the criteria these will
overwrite the undirected evidences and only the directed result
will be returned.

\end{description}\end{quote}

\end{fulllineitems}

\index{degrees\_signed\_in\_by\_interaction\_type() (pypath.core.interaction.Interaction method)@\spxentry{degrees\_signed\_in\_by\_interaction\_type()}\spxextra{pypath.core.interaction.Interaction method}}

\begin{fulllineitems}
\phantomsection\label{\detokenize{reference:pypath.core.interaction.Interaction.degrees_signed_in_by_interaction_type}}\pysiglinewithargsret{\sphinxbfcode{\sphinxupquote{degrees\_signed\_in\_by\_interaction\_type}}}{\emph{effect=None}, \emph{resources=None}, \emph{data\_model=None}, \emph{interaction\_type=None}, \emph{via=None}, \emph{references=None}}{}
Returns a \sphinxstyleemphasis{set} of nodes with the connections matching the direction,
effect and evidence criteria. E.g. if the query concerns the incoming
degrees with positive effect and the matching evidences show A
activates B, but not the other way around, only “B” will be returned.
\begin{quote}\begin{description}
\item[{Parameters}] \leavevmode
\sphinxstyleliteralstrong{\sphinxupquote{mode}} (\sphinxstyleliteralemphasis{\sphinxupquote{str}}) \textendash{} The type of degrees to be considered. Three possible values are
\sphinxcode{\sphinxupquote{'IN'}}, \sphinxtitleref{‘OUT’{}`} and \sphinxcode{\sphinxupquote{'ALL'}} for incoming, outgoing and all
connections, respectively. If the \sphinxcode{\sphinxupquote{direction}} is \sphinxcode{\sphinxupquote{False}} the
only possible mode is \sphinxcode{\sphinxupquote{ALL}}. If the \sphinxcode{\sphinxupquote{direction}} is \sphinxcode{\sphinxupquote{None}}
and also directed evidence(s) match the criteria these will
overwrite the undirected evidences and only the directed result
will be returned.

\end{description}\end{quote}

\end{fulllineitems}

\index{degrees\_signed\_in\_by\_interaction\_type\_and\_data\_model() (pypath.core.interaction.Interaction method)@\spxentry{degrees\_signed\_in\_by\_interaction\_type\_and\_data\_model()}\spxextra{pypath.core.interaction.Interaction method}}

\begin{fulllineitems}
\phantomsection\label{\detokenize{reference:pypath.core.interaction.Interaction.degrees_signed_in_by_interaction_type_and_data_model}}\pysiglinewithargsret{\sphinxbfcode{\sphinxupquote{degrees\_signed\_in\_by\_interaction\_type\_and\_data\_model}}}{\emph{effect=None}, \emph{resources=None}, \emph{data\_model=None}, \emph{interaction\_type=None}, \emph{via=None}, \emph{references=None}}{}
Returns a \sphinxstyleemphasis{set} of nodes with the connections matching the direction,
effect and evidence criteria. E.g. if the query concerns the incoming
degrees with positive effect and the matching evidences show A
activates B, but not the other way around, only “B” will be returned.
\begin{quote}\begin{description}
\item[{Parameters}] \leavevmode
\sphinxstyleliteralstrong{\sphinxupquote{mode}} (\sphinxstyleliteralemphasis{\sphinxupquote{str}}) \textendash{} The type of degrees to be considered. Three possible values are
\sphinxcode{\sphinxupquote{'IN'}}, \sphinxtitleref{‘OUT’{}`} and \sphinxcode{\sphinxupquote{'ALL'}} for incoming, outgoing and all
connections, respectively. If the \sphinxcode{\sphinxupquote{direction}} is \sphinxcode{\sphinxupquote{False}} the
only possible mode is \sphinxcode{\sphinxupquote{ALL}}. If the \sphinxcode{\sphinxupquote{direction}} is \sphinxcode{\sphinxupquote{None}}
and also directed evidence(s) match the criteria these will
overwrite the undirected evidences and only the directed result
will be returned.

\end{description}\end{quote}

\end{fulllineitems}

\index{degrees\_signed\_in\_by\_interaction\_type\_and\_data\_model\_and\_resource() (pypath.core.interaction.Interaction method)@\spxentry{degrees\_signed\_in\_by\_interaction\_type\_and\_data\_model\_and\_resource()}\spxextra{pypath.core.interaction.Interaction method}}

\begin{fulllineitems}
\phantomsection\label{\detokenize{reference:pypath.core.interaction.Interaction.degrees_signed_in_by_interaction_type_and_data_model_and_resource}}\pysiglinewithargsret{\sphinxbfcode{\sphinxupquote{degrees\_signed\_in\_by\_interaction\_type\_and\_data\_model\_and\_resource}}}{\emph{effect=None}, \emph{resources=None}, \emph{data\_model=None}, \emph{interaction\_type=None}, \emph{via=None}, \emph{references=None}}{}
Returns a \sphinxstyleemphasis{set} of nodes with the connections matching the direction,
effect and evidence criteria. E.g. if the query concerns the incoming
degrees with positive effect and the matching evidences show A
activates B, but not the other way around, only “B” will be returned.
\begin{quote}\begin{description}
\item[{Parameters}] \leavevmode
\sphinxstyleliteralstrong{\sphinxupquote{mode}} (\sphinxstyleliteralemphasis{\sphinxupquote{str}}) \textendash{} The type of degrees to be considered. Three possible values are
\sphinxcode{\sphinxupquote{'IN'}}, \sphinxtitleref{‘OUT’{}`} and \sphinxcode{\sphinxupquote{'ALL'}} for incoming, outgoing and all
connections, respectively. If the \sphinxcode{\sphinxupquote{direction}} is \sphinxcode{\sphinxupquote{False}} the
only possible mode is \sphinxcode{\sphinxupquote{ALL}}. If the \sphinxcode{\sphinxupquote{direction}} is \sphinxcode{\sphinxupquote{None}}
and also directed evidence(s) match the criteria these will
overwrite the undirected evidences and only the directed result
will be returned.

\end{description}\end{quote}

\end{fulllineitems}

\index{degrees\_signed\_in\_by\_reference() (pypath.core.interaction.Interaction method)@\spxentry{degrees\_signed\_in\_by\_reference()}\spxextra{pypath.core.interaction.Interaction method}}

\begin{fulllineitems}
\phantomsection\label{\detokenize{reference:pypath.core.interaction.Interaction.degrees_signed_in_by_reference}}\pysiglinewithargsret{\sphinxbfcode{\sphinxupquote{degrees\_signed\_in\_by\_reference}}}{\emph{effect=None}, \emph{resources=None}, \emph{data\_model=None}, \emph{interaction\_type=None}, \emph{via=None}, \emph{references=None}}{}
Returns a \sphinxstyleemphasis{set} of nodes with the connections matching the direction,
effect and evidence criteria. E.g. if the query concerns the incoming
degrees with positive effect and the matching evidences show A
activates B, but not the other way around, only “B” will be returned.
\begin{quote}\begin{description}
\item[{Parameters}] \leavevmode
\sphinxstyleliteralstrong{\sphinxupquote{mode}} (\sphinxstyleliteralemphasis{\sphinxupquote{str}}) \textendash{} The type of degrees to be considered. Three possible values are
\sphinxcode{\sphinxupquote{'IN'}}, \sphinxtitleref{‘OUT’{}`} and \sphinxcode{\sphinxupquote{'ALL'}} for incoming, outgoing and all
connections, respectively. If the \sphinxcode{\sphinxupquote{direction}} is \sphinxcode{\sphinxupquote{False}} the
only possible mode is \sphinxcode{\sphinxupquote{ALL}}. If the \sphinxcode{\sphinxupquote{direction}} is \sphinxcode{\sphinxupquote{None}}
and also directed evidence(s) match the criteria these will
overwrite the undirected evidences and only the directed result
will be returned.

\end{description}\end{quote}

\end{fulllineitems}

\index{degrees\_signed\_in\_by\_resource() (pypath.core.interaction.Interaction method)@\spxentry{degrees\_signed\_in\_by\_resource()}\spxextra{pypath.core.interaction.Interaction method}}

\begin{fulllineitems}
\phantomsection\label{\detokenize{reference:pypath.core.interaction.Interaction.degrees_signed_in_by_resource}}\pysiglinewithargsret{\sphinxbfcode{\sphinxupquote{degrees\_signed\_in\_by\_resource}}}{\emph{effect=None}, \emph{resources=None}, \emph{data\_model=None}, \emph{interaction\_type=None}, \emph{via=None}, \emph{references=None}}{}
Returns a \sphinxstyleemphasis{set} of nodes with the connections matching the direction,
effect and evidence criteria. E.g. if the query concerns the incoming
degrees with positive effect and the matching evidences show A
activates B, but not the other way around, only “B” will be returned.
\begin{quote}\begin{description}
\item[{Parameters}] \leavevmode
\sphinxstyleliteralstrong{\sphinxupquote{mode}} (\sphinxstyleliteralemphasis{\sphinxupquote{str}}) \textendash{} The type of degrees to be considered. Three possible values are
\sphinxcode{\sphinxupquote{'IN'}}, \sphinxtitleref{‘OUT’{}`} and \sphinxcode{\sphinxupquote{'ALL'}} for incoming, outgoing and all
connections, respectively. If the \sphinxcode{\sphinxupquote{direction}} is \sphinxcode{\sphinxupquote{False}} the
only possible mode is \sphinxcode{\sphinxupquote{ALL}}. If the \sphinxcode{\sphinxupquote{direction}} is \sphinxcode{\sphinxupquote{None}}
and also directed evidence(s) match the criteria these will
overwrite the undirected evidences and only the directed result
will be returned.

\end{description}\end{quote}

\end{fulllineitems}

\index{degrees\_signed\_out\_by\_data\_model() (pypath.core.interaction.Interaction method)@\spxentry{degrees\_signed\_out\_by\_data\_model()}\spxextra{pypath.core.interaction.Interaction method}}

\begin{fulllineitems}
\phantomsection\label{\detokenize{reference:pypath.core.interaction.Interaction.degrees_signed_out_by_data_model}}\pysiglinewithargsret{\sphinxbfcode{\sphinxupquote{degrees\_signed\_out\_by\_data\_model}}}{\emph{effect=None}, \emph{resources=None}, \emph{data\_model=None}, \emph{interaction\_type=None}, \emph{via=None}, \emph{references=None}}{}
Returns a \sphinxstyleemphasis{set} of nodes with the connections matching the direction,
effect and evidence criteria. E.g. if the query concerns the incoming
degrees with positive effect and the matching evidences show A
activates B, but not the other way around, only “B” will be returned.
\begin{quote}\begin{description}
\item[{Parameters}] \leavevmode
\sphinxstyleliteralstrong{\sphinxupquote{mode}} (\sphinxstyleliteralemphasis{\sphinxupquote{str}}) \textendash{} The type of degrees to be considered. Three possible values are
\sphinxcode{\sphinxupquote{'IN'}}, \sphinxtitleref{‘OUT’{}`} and \sphinxcode{\sphinxupquote{'ALL'}} for incoming, outgoing and all
connections, respectively. If the \sphinxcode{\sphinxupquote{direction}} is \sphinxcode{\sphinxupquote{False}} the
only possible mode is \sphinxcode{\sphinxupquote{ALL}}. If the \sphinxcode{\sphinxupquote{direction}} is \sphinxcode{\sphinxupquote{None}}
and also directed evidence(s) match the criteria these will
overwrite the undirected evidences and only the directed result
will be returned.

\end{description}\end{quote}

\end{fulllineitems}

\index{degrees\_signed\_out\_by\_interaction\_type() (pypath.core.interaction.Interaction method)@\spxentry{degrees\_signed\_out\_by\_interaction\_type()}\spxextra{pypath.core.interaction.Interaction method}}

\begin{fulllineitems}
\phantomsection\label{\detokenize{reference:pypath.core.interaction.Interaction.degrees_signed_out_by_interaction_type}}\pysiglinewithargsret{\sphinxbfcode{\sphinxupquote{degrees\_signed\_out\_by\_interaction\_type}}}{\emph{effect=None}, \emph{resources=None}, \emph{data\_model=None}, \emph{interaction\_type=None}, \emph{via=None}, \emph{references=None}}{}
Returns a \sphinxstyleemphasis{set} of nodes with the connections matching the direction,
effect and evidence criteria. E.g. if the query concerns the incoming
degrees with positive effect and the matching evidences show A
activates B, but not the other way around, only “B” will be returned.
\begin{quote}\begin{description}
\item[{Parameters}] \leavevmode
\sphinxstyleliteralstrong{\sphinxupquote{mode}} (\sphinxstyleliteralemphasis{\sphinxupquote{str}}) \textendash{} The type of degrees to be considered. Three possible values are
\sphinxcode{\sphinxupquote{'IN'}}, \sphinxtitleref{‘OUT’{}`} and \sphinxcode{\sphinxupquote{'ALL'}} for incoming, outgoing and all
connections, respectively. If the \sphinxcode{\sphinxupquote{direction}} is \sphinxcode{\sphinxupquote{False}} the
only possible mode is \sphinxcode{\sphinxupquote{ALL}}. If the \sphinxcode{\sphinxupquote{direction}} is \sphinxcode{\sphinxupquote{None}}
and also directed evidence(s) match the criteria these will
overwrite the undirected evidences and only the directed result
will be returned.

\end{description}\end{quote}

\end{fulllineitems}

\index{degrees\_signed\_out\_by\_interaction\_type\_and\_data\_model() (pypath.core.interaction.Interaction method)@\spxentry{degrees\_signed\_out\_by\_interaction\_type\_and\_data\_model()}\spxextra{pypath.core.interaction.Interaction method}}

\begin{fulllineitems}
\phantomsection\label{\detokenize{reference:pypath.core.interaction.Interaction.degrees_signed_out_by_interaction_type_and_data_model}}\pysiglinewithargsret{\sphinxbfcode{\sphinxupquote{degrees\_signed\_out\_by\_interaction\_type\_and\_data\_model}}}{\emph{effect=None}, \emph{resources=None}, \emph{data\_model=None}, \emph{interaction\_type=None}, \emph{via=None}, \emph{references=None}}{}
Returns a \sphinxstyleemphasis{set} of nodes with the connections matching the direction,
effect and evidence criteria. E.g. if the query concerns the incoming
degrees with positive effect and the matching evidences show A
activates B, but not the other way around, only “B” will be returned.
\begin{quote}\begin{description}
\item[{Parameters}] \leavevmode
\sphinxstyleliteralstrong{\sphinxupquote{mode}} (\sphinxstyleliteralemphasis{\sphinxupquote{str}}) \textendash{} The type of degrees to be considered. Three possible values are
\sphinxcode{\sphinxupquote{'IN'}}, \sphinxtitleref{‘OUT’{}`} and \sphinxcode{\sphinxupquote{'ALL'}} for incoming, outgoing and all
connections, respectively. If the \sphinxcode{\sphinxupquote{direction}} is \sphinxcode{\sphinxupquote{False}} the
only possible mode is \sphinxcode{\sphinxupquote{ALL}}. If the \sphinxcode{\sphinxupquote{direction}} is \sphinxcode{\sphinxupquote{None}}
and also directed evidence(s) match the criteria these will
overwrite the undirected evidences and only the directed result
will be returned.

\end{description}\end{quote}

\end{fulllineitems}

\index{degrees\_signed\_out\_by\_interaction\_type\_and\_data\_model\_and\_resource() (pypath.core.interaction.Interaction method)@\spxentry{degrees\_signed\_out\_by\_interaction\_type\_and\_data\_model\_and\_resource()}\spxextra{pypath.core.interaction.Interaction method}}

\begin{fulllineitems}
\phantomsection\label{\detokenize{reference:pypath.core.interaction.Interaction.degrees_signed_out_by_interaction_type_and_data_model_and_resource}}\pysiglinewithargsret{\sphinxbfcode{\sphinxupquote{degrees\_signed\_out\_by\_interaction\_type\_and\_data\_model\_and\_resource}}}{\emph{effect=None}, \emph{resources=None}, \emph{data\_model=None}, \emph{interaction\_type=None}, \emph{via=None}, \emph{references=None}}{}
Returns a \sphinxstyleemphasis{set} of nodes with the connections matching the direction,
effect and evidence criteria. E.g. if the query concerns the incoming
degrees with positive effect and the matching evidences show A
activates B, but not the other way around, only “B” will be returned.
\begin{quote}\begin{description}
\item[{Parameters}] \leavevmode
\sphinxstyleliteralstrong{\sphinxupquote{mode}} (\sphinxstyleliteralemphasis{\sphinxupquote{str}}) \textendash{} The type of degrees to be considered. Three possible values are
\sphinxcode{\sphinxupquote{'IN'}}, \sphinxtitleref{‘OUT’{}`} and \sphinxcode{\sphinxupquote{'ALL'}} for incoming, outgoing and all
connections, respectively. If the \sphinxcode{\sphinxupquote{direction}} is \sphinxcode{\sphinxupquote{False}} the
only possible mode is \sphinxcode{\sphinxupquote{ALL}}. If the \sphinxcode{\sphinxupquote{direction}} is \sphinxcode{\sphinxupquote{None}}
and also directed evidence(s) match the criteria these will
overwrite the undirected evidences and only the directed result
will be returned.

\end{description}\end{quote}

\end{fulllineitems}

\index{degrees\_signed\_out\_by\_reference() (pypath.core.interaction.Interaction method)@\spxentry{degrees\_signed\_out\_by\_reference()}\spxextra{pypath.core.interaction.Interaction method}}

\begin{fulllineitems}
\phantomsection\label{\detokenize{reference:pypath.core.interaction.Interaction.degrees_signed_out_by_reference}}\pysiglinewithargsret{\sphinxbfcode{\sphinxupquote{degrees\_signed\_out\_by\_reference}}}{\emph{effect=None}, \emph{resources=None}, \emph{data\_model=None}, \emph{interaction\_type=None}, \emph{via=None}, \emph{references=None}}{}
Returns a \sphinxstyleemphasis{set} of nodes with the connections matching the direction,
effect and evidence criteria. E.g. if the query concerns the incoming
degrees with positive effect and the matching evidences show A
activates B, but not the other way around, only “B” will be returned.
\begin{quote}\begin{description}
\item[{Parameters}] \leavevmode
\sphinxstyleliteralstrong{\sphinxupquote{mode}} (\sphinxstyleliteralemphasis{\sphinxupquote{str}}) \textendash{} The type of degrees to be considered. Three possible values are
\sphinxcode{\sphinxupquote{'IN'}}, \sphinxtitleref{‘OUT’{}`} and \sphinxcode{\sphinxupquote{'ALL'}} for incoming, outgoing and all
connections, respectively. If the \sphinxcode{\sphinxupquote{direction}} is \sphinxcode{\sphinxupquote{False}} the
only possible mode is \sphinxcode{\sphinxupquote{ALL}}. If the \sphinxcode{\sphinxupquote{direction}} is \sphinxcode{\sphinxupquote{None}}
and also directed evidence(s) match the criteria these will
overwrite the undirected evidences and only the directed result
will be returned.

\end{description}\end{quote}

\end{fulllineitems}

\index{degrees\_signed\_out\_by\_resource() (pypath.core.interaction.Interaction method)@\spxentry{degrees\_signed\_out\_by\_resource()}\spxextra{pypath.core.interaction.Interaction method}}

\begin{fulllineitems}
\phantomsection\label{\detokenize{reference:pypath.core.interaction.Interaction.degrees_signed_out_by_resource}}\pysiglinewithargsret{\sphinxbfcode{\sphinxupquote{degrees\_signed\_out\_by\_resource}}}{\emph{effect=None}, \emph{resources=None}, \emph{data\_model=None}, \emph{interaction\_type=None}, \emph{via=None}, \emph{references=None}}{}
Returns a \sphinxstyleemphasis{set} of nodes with the connections matching the direction,
effect and evidence criteria. E.g. if the query concerns the incoming
degrees with positive effect and the matching evidences show A
activates B, but not the other way around, only “B” will be returned.
\begin{quote}\begin{description}
\item[{Parameters}] \leavevmode
\sphinxstyleliteralstrong{\sphinxupquote{mode}} (\sphinxstyleliteralemphasis{\sphinxupquote{str}}) \textendash{} The type of degrees to be considered. Three possible values are
\sphinxcode{\sphinxupquote{'IN'}}, \sphinxtitleref{‘OUT’{}`} and \sphinxcode{\sphinxupquote{'ALL'}} for incoming, outgoing and all
connections, respectively. If the \sphinxcode{\sphinxupquote{direction}} is \sphinxcode{\sphinxupquote{False}} the
only possible mode is \sphinxcode{\sphinxupquote{ALL}}. If the \sphinxcode{\sphinxupquote{direction}} is \sphinxcode{\sphinxupquote{None}}
and also directed evidence(s) match the criteria these will
overwrite the undirected evidences and only the directed result
will be returned.

\end{description}\end{quote}

\end{fulllineitems}

\index{degrees\_undirected\_by\_data\_model() (pypath.core.interaction.Interaction method)@\spxentry{degrees\_undirected\_by\_data\_model()}\spxextra{pypath.core.interaction.Interaction method}}

\begin{fulllineitems}
\phantomsection\label{\detokenize{reference:pypath.core.interaction.Interaction.degrees_undirected_by_data_model}}\pysiglinewithargsret{\sphinxbfcode{\sphinxupquote{degrees\_undirected\_by\_data\_model}}}{\emph{effect=None}, \emph{resources=None}, \emph{data\_model=None}, \emph{interaction\_type=None}, \emph{via=None}, \emph{references=None}}{}
Returns a \sphinxstyleemphasis{set} of nodes with the connections matching the direction,
effect and evidence criteria. E.g. if the query concerns the incoming
degrees with positive effect and the matching evidences show A
activates B, but not the other way around, only “B” will be returned.
\begin{quote}\begin{description}
\item[{Parameters}] \leavevmode
\sphinxstyleliteralstrong{\sphinxupquote{mode}} (\sphinxstyleliteralemphasis{\sphinxupquote{str}}) \textendash{} The type of degrees to be considered. Three possible values are
\sphinxcode{\sphinxupquote{'IN'}}, \sphinxtitleref{‘OUT’{}`} and \sphinxcode{\sphinxupquote{'ALL'}} for incoming, outgoing and all
connections, respectively. If the \sphinxcode{\sphinxupquote{direction}} is \sphinxcode{\sphinxupquote{False}} the
only possible mode is \sphinxcode{\sphinxupquote{ALL}}. If the \sphinxcode{\sphinxupquote{direction}} is \sphinxcode{\sphinxupquote{None}}
and also directed evidence(s) match the criteria these will
overwrite the undirected evidences and only the directed result
will be returned.

\end{description}\end{quote}

\end{fulllineitems}

\index{degrees\_undirected\_by\_interaction\_type() (pypath.core.interaction.Interaction method)@\spxentry{degrees\_undirected\_by\_interaction\_type()}\spxextra{pypath.core.interaction.Interaction method}}

\begin{fulllineitems}
\phantomsection\label{\detokenize{reference:pypath.core.interaction.Interaction.degrees_undirected_by_interaction_type}}\pysiglinewithargsret{\sphinxbfcode{\sphinxupquote{degrees\_undirected\_by\_interaction\_type}}}{\emph{effect=None}, \emph{resources=None}, \emph{data\_model=None}, \emph{interaction\_type=None}, \emph{via=None}, \emph{references=None}}{}
Returns a \sphinxstyleemphasis{set} of nodes with the connections matching the direction,
effect and evidence criteria. E.g. if the query concerns the incoming
degrees with positive effect and the matching evidences show A
activates B, but not the other way around, only “B” will be returned.
\begin{quote}\begin{description}
\item[{Parameters}] \leavevmode
\sphinxstyleliteralstrong{\sphinxupquote{mode}} (\sphinxstyleliteralemphasis{\sphinxupquote{str}}) \textendash{} The type of degrees to be considered. Three possible values are
\sphinxcode{\sphinxupquote{'IN'}}, \sphinxtitleref{‘OUT’{}`} and \sphinxcode{\sphinxupquote{'ALL'}} for incoming, outgoing and all
connections, respectively. If the \sphinxcode{\sphinxupquote{direction}} is \sphinxcode{\sphinxupquote{False}} the
only possible mode is \sphinxcode{\sphinxupquote{ALL}}. If the \sphinxcode{\sphinxupquote{direction}} is \sphinxcode{\sphinxupquote{None}}
and also directed evidence(s) match the criteria these will
overwrite the undirected evidences and only the directed result
will be returned.

\end{description}\end{quote}

\end{fulllineitems}

\index{degrees\_undirected\_by\_interaction\_type\_and\_data\_model() (pypath.core.interaction.Interaction method)@\spxentry{degrees\_undirected\_by\_interaction\_type\_and\_data\_model()}\spxextra{pypath.core.interaction.Interaction method}}

\begin{fulllineitems}
\phantomsection\label{\detokenize{reference:pypath.core.interaction.Interaction.degrees_undirected_by_interaction_type_and_data_model}}\pysiglinewithargsret{\sphinxbfcode{\sphinxupquote{degrees\_undirected\_by\_interaction\_type\_and\_data\_model}}}{\emph{effect=None}, \emph{resources=None}, \emph{data\_model=None}, \emph{interaction\_type=None}, \emph{via=None}, \emph{references=None}}{}
Returns a \sphinxstyleemphasis{set} of nodes with the connections matching the direction,
effect and evidence criteria. E.g. if the query concerns the incoming
degrees with positive effect and the matching evidences show A
activates B, but not the other way around, only “B” will be returned.
\begin{quote}\begin{description}
\item[{Parameters}] \leavevmode
\sphinxstyleliteralstrong{\sphinxupquote{mode}} (\sphinxstyleliteralemphasis{\sphinxupquote{str}}) \textendash{} The type of degrees to be considered. Three possible values are
\sphinxcode{\sphinxupquote{'IN'}}, \sphinxtitleref{‘OUT’{}`} and \sphinxcode{\sphinxupquote{'ALL'}} for incoming, outgoing and all
connections, respectively. If the \sphinxcode{\sphinxupquote{direction}} is \sphinxcode{\sphinxupquote{False}} the
only possible mode is \sphinxcode{\sphinxupquote{ALL}}. If the \sphinxcode{\sphinxupquote{direction}} is \sphinxcode{\sphinxupquote{None}}
and also directed evidence(s) match the criteria these will
overwrite the undirected evidences and only the directed result
will be returned.

\end{description}\end{quote}

\end{fulllineitems}

\index{degrees\_undirected\_by\_interaction\_type\_and\_data\_model\_and\_resource() (pypath.core.interaction.Interaction method)@\spxentry{degrees\_undirected\_by\_interaction\_type\_and\_data\_model\_and\_resource()}\spxextra{pypath.core.interaction.Interaction method}}

\begin{fulllineitems}
\phantomsection\label{\detokenize{reference:pypath.core.interaction.Interaction.degrees_undirected_by_interaction_type_and_data_model_and_resource}}\pysiglinewithargsret{\sphinxbfcode{\sphinxupquote{degrees\_undirected\_by\_interaction\_type\_and\_data\_model\_and\_resource}}}{\emph{effect=None}, \emph{resources=None}, \emph{data\_model=None}, \emph{interaction\_type=None}, \emph{via=None}, \emph{references=None}}{}
Returns a \sphinxstyleemphasis{set} of nodes with the connections matching the direction,
effect and evidence criteria. E.g. if the query concerns the incoming
degrees with positive effect and the matching evidences show A
activates B, but not the other way around, only “B” will be returned.
\begin{quote}\begin{description}
\item[{Parameters}] \leavevmode
\sphinxstyleliteralstrong{\sphinxupquote{mode}} (\sphinxstyleliteralemphasis{\sphinxupquote{str}}) \textendash{} The type of degrees to be considered. Three possible values are
\sphinxcode{\sphinxupquote{'IN'}}, \sphinxtitleref{‘OUT’{}`} and \sphinxcode{\sphinxupquote{'ALL'}} for incoming, outgoing and all
connections, respectively. If the \sphinxcode{\sphinxupquote{direction}} is \sphinxcode{\sphinxupquote{False}} the
only possible mode is \sphinxcode{\sphinxupquote{ALL}}. If the \sphinxcode{\sphinxupquote{direction}} is \sphinxcode{\sphinxupquote{None}}
and also directed evidence(s) match the criteria these will
overwrite the undirected evidences and only the directed result
will be returned.

\end{description}\end{quote}

\end{fulllineitems}

\index{degrees\_undirected\_by\_reference() (pypath.core.interaction.Interaction method)@\spxentry{degrees\_undirected\_by\_reference()}\spxextra{pypath.core.interaction.Interaction method}}

\begin{fulllineitems}
\phantomsection\label{\detokenize{reference:pypath.core.interaction.Interaction.degrees_undirected_by_reference}}\pysiglinewithargsret{\sphinxbfcode{\sphinxupquote{degrees\_undirected\_by\_reference}}}{\emph{effect=None}, \emph{resources=None}, \emph{data\_model=None}, \emph{interaction\_type=None}, \emph{via=None}, \emph{references=None}}{}
Returns a \sphinxstyleemphasis{set} of nodes with the connections matching the direction,
effect and evidence criteria. E.g. if the query concerns the incoming
degrees with positive effect and the matching evidences show A
activates B, but not the other way around, only “B” will be returned.
\begin{quote}\begin{description}
\item[{Parameters}] \leavevmode
\sphinxstyleliteralstrong{\sphinxupquote{mode}} (\sphinxstyleliteralemphasis{\sphinxupquote{str}}) \textendash{} The type of degrees to be considered. Three possible values are
\sphinxcode{\sphinxupquote{'IN'}}, \sphinxtitleref{‘OUT’{}`} and \sphinxcode{\sphinxupquote{'ALL'}} for incoming, outgoing and all
connections, respectively. If the \sphinxcode{\sphinxupquote{direction}} is \sphinxcode{\sphinxupquote{False}} the
only possible mode is \sphinxcode{\sphinxupquote{ALL}}. If the \sphinxcode{\sphinxupquote{direction}} is \sphinxcode{\sphinxupquote{None}}
and also directed evidence(s) match the criteria these will
overwrite the undirected evidences and only the directed result
will be returned.

\end{description}\end{quote}

\end{fulllineitems}

\index{degrees\_undirected\_by\_resource() (pypath.core.interaction.Interaction method)@\spxentry{degrees\_undirected\_by\_resource()}\spxextra{pypath.core.interaction.Interaction method}}

\begin{fulllineitems}
\phantomsection\label{\detokenize{reference:pypath.core.interaction.Interaction.degrees_undirected_by_resource}}\pysiglinewithargsret{\sphinxbfcode{\sphinxupquote{degrees\_undirected\_by\_resource}}}{\emph{effect=None}, \emph{resources=None}, \emph{data\_model=None}, \emph{interaction\_type=None}, \emph{via=None}, \emph{references=None}}{}
Returns a \sphinxstyleemphasis{set} of nodes with the connections matching the direction,
effect and evidence criteria. E.g. if the query concerns the incoming
degrees with positive effect and the matching evidences show A
activates B, but not the other way around, only “B” will be returned.
\begin{quote}\begin{description}
\item[{Parameters}] \leavevmode
\sphinxstyleliteralstrong{\sphinxupquote{mode}} (\sphinxstyleliteralemphasis{\sphinxupquote{str}}) \textendash{} The type of degrees to be considered. Three possible values are
\sphinxcode{\sphinxupquote{'IN'}}, \sphinxtitleref{‘OUT’{}`} and \sphinxcode{\sphinxupquote{'ALL'}} for incoming, outgoing and all
connections, respectively. If the \sphinxcode{\sphinxupquote{direction}} is \sphinxcode{\sphinxupquote{False}} the
only possible mode is \sphinxcode{\sphinxupquote{ALL}}. If the \sphinxcode{\sphinxupquote{direction}} is \sphinxcode{\sphinxupquote{None}}
and also directed evidence(s) match the criteria these will
overwrite the undirected evidences and only the directed result
will be returned.

\end{description}\end{quote}

\end{fulllineitems}

\index{entities\_by\_data\_model() (pypath.core.interaction.Interaction method)@\spxentry{entities\_by\_data\_model()}\spxextra{pypath.core.interaction.Interaction method}}

\begin{fulllineitems}
\phantomsection\label{\detokenize{reference:pypath.core.interaction.Interaction.entities_by_data_model}}\pysiglinewithargsret{\sphinxbfcode{\sphinxupquote{entities\_by\_data\_model}}}{\emph{effect=None}, \emph{resources=None}, \emph{data\_model=None}, \emph{interaction\_type=None}, \emph{via=None}, \emph{references=None}}{}
Retrieves the entities involved in interactions matching the criteria.
It either returns both interacting entities in a \sphinxstyleemphasis{set} or an empty
\sphinxstyleemphasis{set}. This may not sound so useful at the level of this object but
becomes more useful once we want to collect entities having certain
kind of interactions across a series of \sphinxtitleref{Interaction} objects.
\begin{quote}\begin{description}
\item[{Parameters}] \leavevmode\begin{itemize}
\item {} 
\sphinxstyleliteralstrong{\sphinxupquote{entity\_type}} (\sphinxstyleliteralemphasis{\sphinxupquote{str}}) \textendash{} The type of the molecular entity. Possible values: \sphinxtitleref{protein},
\sphinxtitleref{complex}, \sphinxtitleref{mirna}, \sphinxtitleref{small\_molecule}.

\item {} 
\sphinxstyleliteralstrong{\sphinxupquote{return\_type}} (\sphinxstyleliteralemphasis{\sphinxupquote{str}}) \textendash{} The type of values to return. Default is
py:class:\sphinxcode{\sphinxupquote{pypath.entity.Entity}} objects, alternatives are
\sphinxcode{\sphinxupquote{labels}}  \sphinxcode{\sphinxupquote{identifiers}}.

\end{itemize}

\end{description}\end{quote}

\end{fulllineitems}

\index{entities\_by\_interaction\_type() (pypath.core.interaction.Interaction method)@\spxentry{entities\_by\_interaction\_type()}\spxextra{pypath.core.interaction.Interaction method}}

\begin{fulllineitems}
\phantomsection\label{\detokenize{reference:pypath.core.interaction.Interaction.entities_by_interaction_type}}\pysiglinewithargsret{\sphinxbfcode{\sphinxupquote{entities\_by\_interaction\_type}}}{\emph{effect=None}, \emph{resources=None}, \emph{data\_model=None}, \emph{interaction\_type=None}, \emph{via=None}, \emph{references=None}}{}
Retrieves the entities involved in interactions matching the criteria.
It either returns both interacting entities in a \sphinxstyleemphasis{set} or an empty
\sphinxstyleemphasis{set}. This may not sound so useful at the level of this object but
becomes more useful once we want to collect entities having certain
kind of interactions across a series of \sphinxtitleref{Interaction} objects.
\begin{quote}\begin{description}
\item[{Parameters}] \leavevmode\begin{itemize}
\item {} 
\sphinxstyleliteralstrong{\sphinxupquote{entity\_type}} (\sphinxstyleliteralemphasis{\sphinxupquote{str}}) \textendash{} The type of the molecular entity. Possible values: \sphinxtitleref{protein},
\sphinxtitleref{complex}, \sphinxtitleref{mirna}, \sphinxtitleref{small\_molecule}.

\item {} 
\sphinxstyleliteralstrong{\sphinxupquote{return\_type}} (\sphinxstyleliteralemphasis{\sphinxupquote{str}}) \textendash{} The type of values to return. Default is
py:class:\sphinxcode{\sphinxupquote{pypath.entity.Entity}} objects, alternatives are
\sphinxcode{\sphinxupquote{labels}}  \sphinxcode{\sphinxupquote{identifiers}}.

\end{itemize}

\end{description}\end{quote}

\end{fulllineitems}

\index{entities\_by\_interaction\_type\_and\_data\_model() (pypath.core.interaction.Interaction method)@\spxentry{entities\_by\_interaction\_type\_and\_data\_model()}\spxextra{pypath.core.interaction.Interaction method}}

\begin{fulllineitems}
\phantomsection\label{\detokenize{reference:pypath.core.interaction.Interaction.entities_by_interaction_type_and_data_model}}\pysiglinewithargsret{\sphinxbfcode{\sphinxupquote{entities\_by\_interaction\_type\_and\_data\_model}}}{\emph{effect=None}, \emph{resources=None}, \emph{data\_model=None}, \emph{interaction\_type=None}, \emph{via=None}, \emph{references=None}}{}
Retrieves the entities involved in interactions matching the criteria.
It either returns both interacting entities in a \sphinxstyleemphasis{set} or an empty
\sphinxstyleemphasis{set}. This may not sound so useful at the level of this object but
becomes more useful once we want to collect entities having certain
kind of interactions across a series of \sphinxtitleref{Interaction} objects.
\begin{quote}\begin{description}
\item[{Parameters}] \leavevmode\begin{itemize}
\item {} 
\sphinxstyleliteralstrong{\sphinxupquote{entity\_type}} (\sphinxstyleliteralemphasis{\sphinxupquote{str}}) \textendash{} The type of the molecular entity. Possible values: \sphinxtitleref{protein},
\sphinxtitleref{complex}, \sphinxtitleref{mirna}, \sphinxtitleref{small\_molecule}.

\item {} 
\sphinxstyleliteralstrong{\sphinxupquote{return\_type}} (\sphinxstyleliteralemphasis{\sphinxupquote{str}}) \textendash{} The type of values to return. Default is
py:class:\sphinxcode{\sphinxupquote{pypath.entity.Entity}} objects, alternatives are
\sphinxcode{\sphinxupquote{labels}}  \sphinxcode{\sphinxupquote{identifiers}}.

\end{itemize}

\end{description}\end{quote}

\end{fulllineitems}

\index{entities\_by\_interaction\_type\_and\_data\_model\_and\_resource() (pypath.core.interaction.Interaction method)@\spxentry{entities\_by\_interaction\_type\_and\_data\_model\_and\_resource()}\spxextra{pypath.core.interaction.Interaction method}}

\begin{fulllineitems}
\phantomsection\label{\detokenize{reference:pypath.core.interaction.Interaction.entities_by_interaction_type_and_data_model_and_resource}}\pysiglinewithargsret{\sphinxbfcode{\sphinxupquote{entities\_by\_interaction\_type\_and\_data\_model\_and\_resource}}}{\emph{effect=None}, \emph{resources=None}, \emph{data\_model=None}, \emph{interaction\_type=None}, \emph{via=None}, \emph{references=None}}{}
Retrieves the entities involved in interactions matching the criteria.
It either returns both interacting entities in a \sphinxstyleemphasis{set} or an empty
\sphinxstyleemphasis{set}. This may not sound so useful at the level of this object but
becomes more useful once we want to collect entities having certain
kind of interactions across a series of \sphinxtitleref{Interaction} objects.
\begin{quote}\begin{description}
\item[{Parameters}] \leavevmode\begin{itemize}
\item {} 
\sphinxstyleliteralstrong{\sphinxupquote{entity\_type}} (\sphinxstyleliteralemphasis{\sphinxupquote{str}}) \textendash{} The type of the molecular entity. Possible values: \sphinxtitleref{protein},
\sphinxtitleref{complex}, \sphinxtitleref{mirna}, \sphinxtitleref{small\_molecule}.

\item {} 
\sphinxstyleliteralstrong{\sphinxupquote{return\_type}} (\sphinxstyleliteralemphasis{\sphinxupquote{str}}) \textendash{} The type of values to return. Default is
py:class:\sphinxcode{\sphinxupquote{pypath.entity.Entity}} objects, alternatives are
\sphinxcode{\sphinxupquote{labels}}  \sphinxcode{\sphinxupquote{identifiers}}.

\end{itemize}

\end{description}\end{quote}

\end{fulllineitems}

\index{entities\_by\_reference() (pypath.core.interaction.Interaction method)@\spxentry{entities\_by\_reference()}\spxextra{pypath.core.interaction.Interaction method}}

\begin{fulllineitems}
\phantomsection\label{\detokenize{reference:pypath.core.interaction.Interaction.entities_by_reference}}\pysiglinewithargsret{\sphinxbfcode{\sphinxupquote{entities\_by\_reference}}}{\emph{effect=None}, \emph{resources=None}, \emph{data\_model=None}, \emph{interaction\_type=None}, \emph{via=None}, \emph{references=None}}{}
Retrieves the entities involved in interactions matching the criteria.
It either returns both interacting entities in a \sphinxstyleemphasis{set} or an empty
\sphinxstyleemphasis{set}. This may not sound so useful at the level of this object but
becomes more useful once we want to collect entities having certain
kind of interactions across a series of \sphinxtitleref{Interaction} objects.
\begin{quote}\begin{description}
\item[{Parameters}] \leavevmode\begin{itemize}
\item {} 
\sphinxstyleliteralstrong{\sphinxupquote{entity\_type}} (\sphinxstyleliteralemphasis{\sphinxupquote{str}}) \textendash{} The type of the molecular entity. Possible values: \sphinxtitleref{protein},
\sphinxtitleref{complex}, \sphinxtitleref{mirna}, \sphinxtitleref{small\_molecule}.

\item {} 
\sphinxstyleliteralstrong{\sphinxupquote{return\_type}} (\sphinxstyleliteralemphasis{\sphinxupquote{str}}) \textendash{} The type of values to return. Default is
py:class:\sphinxcode{\sphinxupquote{pypath.entity.Entity}} objects, alternatives are
\sphinxcode{\sphinxupquote{labels}}  \sphinxcode{\sphinxupquote{identifiers}}.

\end{itemize}

\end{description}\end{quote}

\end{fulllineitems}

\index{entities\_by\_resource() (pypath.core.interaction.Interaction method)@\spxentry{entities\_by\_resource()}\spxextra{pypath.core.interaction.Interaction method}}

\begin{fulllineitems}
\phantomsection\label{\detokenize{reference:pypath.core.interaction.Interaction.entities_by_resource}}\pysiglinewithargsret{\sphinxbfcode{\sphinxupquote{entities\_by\_resource}}}{\emph{effect=None}, \emph{resources=None}, \emph{data\_model=None}, \emph{interaction\_type=None}, \emph{via=None}, \emph{references=None}}{}
Retrieves the entities involved in interactions matching the criteria.
It either returns both interacting entities in a \sphinxstyleemphasis{set} or an empty
\sphinxstyleemphasis{set}. This may not sound so useful at the level of this object but
becomes more useful once we want to collect entities having certain
kind of interactions across a series of \sphinxtitleref{Interaction} objects.
\begin{quote}\begin{description}
\item[{Parameters}] \leavevmode\begin{itemize}
\item {} 
\sphinxstyleliteralstrong{\sphinxupquote{entity\_type}} (\sphinxstyleliteralemphasis{\sphinxupquote{str}}) \textendash{} The type of the molecular entity. Possible values: \sphinxtitleref{protein},
\sphinxtitleref{complex}, \sphinxtitleref{mirna}, \sphinxtitleref{small\_molecule}.

\item {} 
\sphinxstyleliteralstrong{\sphinxupquote{return\_type}} (\sphinxstyleliteralemphasis{\sphinxupquote{str}}) \textendash{} The type of values to return. Default is
py:class:\sphinxcode{\sphinxupquote{pypath.entity.Entity}} objects, alternatives are
\sphinxcode{\sphinxupquote{labels}}  \sphinxcode{\sphinxupquote{identifiers}}.

\end{itemize}

\end{description}\end{quote}

\end{fulllineitems}

\index{evaluate\_evidences() (pypath.core.interaction.Interaction method)@\spxentry{evaluate\_evidences()}\spxextra{pypath.core.interaction.Interaction method}}

\begin{fulllineitems}
\phantomsection\label{\detokenize{reference:pypath.core.interaction.Interaction.evaluate_evidences}}\pysiglinewithargsret{\sphinxbfcode{\sphinxupquote{evaluate\_evidences}}}{\emph{this\_direction}, \emph{direction=None}, \emph{effect=None}, \emph{resources=None}, \emph{data\_model=None}, \emph{interaction\_type=None}, \emph{via=None}, \emph{references=None}}{}
Selects the evidence collections matching the direction and effect
criteria and then evaluates if any of the evidences in these
collections match the evidence criteria.

\end{fulllineitems}

\index{generate\_df\_records() (pypath.core.interaction.Interaction method)@\spxentry{generate\_df\_records()}\spxextra{pypath.core.interaction.Interaction method}}

\begin{fulllineitems}
\phantomsection\label{\detokenize{reference:pypath.core.interaction.Interaction.generate_df_records}}\pysiglinewithargsret{\sphinxbfcode{\sphinxupquote{generate\_df\_records}}}{\emph{by\_source=False}, \emph{with\_references=False}}{}
Yields interaction records. It is a generator because one edge can
be represented by one or more records depending on the signs and
directions and other parameters
\begin{quote}\begin{description}
\item[{Parameters}] \leavevmode\begin{itemize}
\item {} 
\sphinxstyleliteralstrong{\sphinxupquote{by\_source}} (\sphinxstyleliteralemphasis{\sphinxupquote{bool}}) \textendash{} Yield separate records by resources. This way the node pairs
will be redundant and you need to group later if you want
unique interacting pairs. By default is \sphinxcode{\sphinxupquote{False}} because for
most applications unique interactions are preferred.
If \sphinxcode{\sphinxupquote{False}} the \sphinxstyleemphasis{refrences} field will still be present
but with \sphinxcode{\sphinxupquote{None}} values.

\item {} 
\sphinxstyleliteralstrong{\sphinxupquote{with\_references}} (\sphinxstyleliteralemphasis{\sphinxupquote{bool}}) \textendash{} Include the literature references. By default is \sphinxcode{\sphinxupquote{False}}
because you rarely need these and they increase the data size
significantly.

\end{itemize}

\end{description}\end{quote}

\end{fulllineitems}

\index{get\_complex\_identifiers() (pypath.core.interaction.Interaction method)@\spxentry{get\_complex\_identifiers()}\spxextra{pypath.core.interaction.Interaction method}}

\begin{fulllineitems}
\phantomsection\label{\detokenize{reference:pypath.core.interaction.Interaction.get_complex_identifiers}}\pysiglinewithargsret{\sphinxbfcode{\sphinxupquote{get\_complex\_identifiers}}}{\emph{entity\_type=None}, \emph{direction=None}, \emph{effect=None}, \emph{resources=None}, \emph{data\_model=None}, \emph{interaction\_type=None}, \emph{via=None}, \emph{references=None}, \emph{return\_type=None}}{}
Retrieves the entities involved in interactions matching the criteria.
It either returns both interacting entities in a \sphinxstyleemphasis{set} or an empty
\sphinxstyleemphasis{set}. This may not sound so useful at the level of this object but
becomes more useful once we want to collect entities having certain
kind of interactions across a series of \sphinxtitleref{Interaction} objects.
\begin{quote}\begin{description}
\item[{Parameters}] \leavevmode\begin{itemize}
\item {} 
\sphinxstyleliteralstrong{\sphinxupquote{entity\_type}} (\sphinxstyleliteralemphasis{\sphinxupquote{str}}) \textendash{} The type of the molecular entity. Possible values: \sphinxtitleref{protein},
\sphinxtitleref{complex}, \sphinxtitleref{mirna}, \sphinxtitleref{small\_molecule}.

\item {} 
\sphinxstyleliteralstrong{\sphinxupquote{return\_type}} (\sphinxstyleliteralemphasis{\sphinxupquote{str}}) \textendash{} The type of values to return. Default is
py:class:\sphinxcode{\sphinxupquote{pypath.entity.Entity}} objects, alternatives are
\sphinxcode{\sphinxupquote{labels}}  \sphinxcode{\sphinxupquote{identifiers}}.

\end{itemize}

\end{description}\end{quote}

\end{fulllineitems}

\index{get\_complex\_labels() (pypath.core.interaction.Interaction method)@\spxentry{get\_complex\_labels()}\spxextra{pypath.core.interaction.Interaction method}}

\begin{fulllineitems}
\phantomsection\label{\detokenize{reference:pypath.core.interaction.Interaction.get_complex_labels}}\pysiglinewithargsret{\sphinxbfcode{\sphinxupquote{get\_complex\_labels}}}{\emph{entity\_type=None}, \emph{direction=None}, \emph{effect=None}, \emph{resources=None}, \emph{data\_model=None}, \emph{interaction\_type=None}, \emph{via=None}, \emph{references=None}, \emph{return\_type=None}}{}
Retrieves the entities involved in interactions matching the criteria.
It either returns both interacting entities in a \sphinxstyleemphasis{set} or an empty
\sphinxstyleemphasis{set}. This may not sound so useful at the level of this object but
becomes more useful once we want to collect entities having certain
kind of interactions across a series of \sphinxtitleref{Interaction} objects.
\begin{quote}\begin{description}
\item[{Parameters}] \leavevmode\begin{itemize}
\item {} 
\sphinxstyleliteralstrong{\sphinxupquote{entity\_type}} (\sphinxstyleliteralemphasis{\sphinxupquote{str}}) \textendash{} The type of the molecular entity. Possible values: \sphinxtitleref{protein},
\sphinxtitleref{complex}, \sphinxtitleref{mirna}, \sphinxtitleref{small\_molecule}.

\item {} 
\sphinxstyleliteralstrong{\sphinxupquote{return\_type}} (\sphinxstyleliteralemphasis{\sphinxupquote{str}}) \textendash{} The type of values to return. Default is
py:class:\sphinxcode{\sphinxupquote{pypath.entity.Entity}} objects, alternatives are
\sphinxcode{\sphinxupquote{labels}}  \sphinxcode{\sphinxupquote{identifiers}}.

\end{itemize}

\end{description}\end{quote}

\end{fulllineitems}

\index{get\_complexes() (pypath.core.interaction.Interaction method)@\spxentry{get\_complexes()}\spxextra{pypath.core.interaction.Interaction method}}

\begin{fulllineitems}
\phantomsection\label{\detokenize{reference:pypath.core.interaction.Interaction.get_complexes}}\pysiglinewithargsret{\sphinxbfcode{\sphinxupquote{get\_complexes}}}{\emph{entity\_type=None}, \emph{direction=None}, \emph{effect=None}, \emph{resources=None}, \emph{data\_model=None}, \emph{interaction\_type=None}, \emph{via=None}, \emph{references=None}, \emph{return\_type=None}}{}
Retrieves the entities involved in interactions matching the criteria.
It either returns both interacting entities in a \sphinxstyleemphasis{set} or an empty
\sphinxstyleemphasis{set}. This may not sound so useful at the level of this object but
becomes more useful once we want to collect entities having certain
kind of interactions across a series of \sphinxtitleref{Interaction} objects.
\begin{quote}\begin{description}
\item[{Parameters}] \leavevmode\begin{itemize}
\item {} 
\sphinxstyleliteralstrong{\sphinxupquote{entity\_type}} (\sphinxstyleliteralemphasis{\sphinxupquote{str}}) \textendash{} The type of the molecular entity. Possible values: \sphinxtitleref{protein},
\sphinxtitleref{complex}, \sphinxtitleref{mirna}, \sphinxtitleref{small\_molecule}.

\item {} 
\sphinxstyleliteralstrong{\sphinxupquote{return\_type}} (\sphinxstyleliteralemphasis{\sphinxupquote{str}}) \textendash{} The type of values to return. Default is
py:class:\sphinxcode{\sphinxupquote{pypath.entity.Entity}} objects, alternatives are
\sphinxcode{\sphinxupquote{labels}}  \sphinxcode{\sphinxupquote{identifiers}}.

\end{itemize}

\end{description}\end{quote}

\end{fulllineitems}

\index{get\_data\_models() (pypath.core.interaction.Interaction method)@\spxentry{get\_data\_models()}\spxextra{pypath.core.interaction.Interaction method}}

\begin{fulllineitems}
\phantomsection\label{\detokenize{reference:pypath.core.interaction.Interaction.get_data_models}}\pysiglinewithargsret{\sphinxbfcode{\sphinxupquote{get\_data\_models}}}{\emph{effect=None}, \emph{resources=None}, \emph{data\_model=None}, \emph{interaction\_type=None}, \emph{via=None}, \emph{references=None}}{}
Retrieves data models matching the criteria.

\end{fulllineitems}

\index{get\_degrees() (pypath.core.interaction.Interaction method)@\spxentry{get\_degrees()}\spxextra{pypath.core.interaction.Interaction method}}

\begin{fulllineitems}
\phantomsection\label{\detokenize{reference:pypath.core.interaction.Interaction.get_degrees}}\pysiglinewithargsret{\sphinxbfcode{\sphinxupquote{get\_degrees}}}{\emph{mode}, \emph{direction=None}, \emph{effect=None}, \emph{resources=None}, \emph{data\_model=None}, \emph{interaction\_type=None}, \emph{via=None}, \emph{references=None}}{}
Returns a \sphinxstyleemphasis{set} of nodes with the connections matching the direction,
effect and evidence criteria. E.g. if the query concerns the incoming
degrees with positive effect and the matching evidences show A
activates B, but not the other way around, only “B” will be returned.
\begin{quote}\begin{description}
\item[{Parameters}] \leavevmode
\sphinxstyleliteralstrong{\sphinxupquote{mode}} (\sphinxstyleliteralemphasis{\sphinxupquote{str}}) \textendash{} The type of degrees to be considered. Three possible values are
\sphinxcode{\sphinxupquote{'IN'}}, \sphinxtitleref{‘OUT’{}`} and \sphinxcode{\sphinxupquote{'ALL'}} for incoming, outgoing and all
connections, respectively. If the \sphinxcode{\sphinxupquote{direction}} is \sphinxcode{\sphinxupquote{False}} the
only possible mode is \sphinxcode{\sphinxupquote{ALL}}. If the \sphinxcode{\sphinxupquote{direction}} is \sphinxcode{\sphinxupquote{None}}
and also directed evidence(s) match the criteria these will
overwrite the undirected evidences and only the directed result
will be returned.

\end{description}\end{quote}

\end{fulllineitems}

\index{get\_degrees\_directed() (pypath.core.interaction.Interaction method)@\spxentry{get\_degrees\_directed()}\spxextra{pypath.core.interaction.Interaction method}}

\begin{fulllineitems}
\phantomsection\label{\detokenize{reference:pypath.core.interaction.Interaction.get_degrees_directed}}\pysiglinewithargsret{\sphinxbfcode{\sphinxupquote{get\_degrees\_directed}}}{\emph{direction=None}, \emph{effect=None}, \emph{resources=None}, \emph{data\_model=None}, \emph{interaction\_type=None}, \emph{via=None}, \emph{references=None}}{}
Returns a \sphinxstyleemphasis{set} of nodes with the connections matching the direction,
effect and evidence criteria. E.g. if the query concerns the incoming
degrees with positive effect and the matching evidences show A
activates B, but not the other way around, only “B” will be returned.
\begin{quote}\begin{description}
\item[{Parameters}] \leavevmode
\sphinxstyleliteralstrong{\sphinxupquote{mode}} (\sphinxstyleliteralemphasis{\sphinxupquote{str}}) \textendash{} The type of degrees to be considered. Three possible values are
\sphinxcode{\sphinxupquote{'IN'}}, \sphinxtitleref{‘OUT’{}`} and \sphinxcode{\sphinxupquote{'ALL'}} for incoming, outgoing and all
connections, respectively. If the \sphinxcode{\sphinxupquote{direction}} is \sphinxcode{\sphinxupquote{False}} the
only possible mode is \sphinxcode{\sphinxupquote{ALL}}. If the \sphinxcode{\sphinxupquote{direction}} is \sphinxcode{\sphinxupquote{None}}
and also directed evidence(s) match the criteria these will
overwrite the undirected evidences and only the directed result
will be returned.

\end{description}\end{quote}

\end{fulllineitems}

\index{get\_degrees\_directed\_in() (pypath.core.interaction.Interaction method)@\spxentry{get\_degrees\_directed\_in()}\spxextra{pypath.core.interaction.Interaction method}}

\begin{fulllineitems}
\phantomsection\label{\detokenize{reference:pypath.core.interaction.Interaction.get_degrees_directed_in}}\pysiglinewithargsret{\sphinxbfcode{\sphinxupquote{get\_degrees\_directed\_in}}}{\emph{direction=None}, \emph{effect=None}, \emph{resources=None}, \emph{data\_model=None}, \emph{interaction\_type=None}, \emph{via=None}, \emph{references=None}}{}
Returns a \sphinxstyleemphasis{set} of nodes with the connections matching the direction,
effect and evidence criteria. E.g. if the query concerns the incoming
degrees with positive effect and the matching evidences show A
activates B, but not the other way around, only “B” will be returned.
\begin{quote}\begin{description}
\item[{Parameters}] \leavevmode
\sphinxstyleliteralstrong{\sphinxupquote{mode}} (\sphinxstyleliteralemphasis{\sphinxupquote{str}}) \textendash{} The type of degrees to be considered. Three possible values are
\sphinxcode{\sphinxupquote{'IN'}}, \sphinxtitleref{‘OUT’{}`} and \sphinxcode{\sphinxupquote{'ALL'}} for incoming, outgoing and all
connections, respectively. If the \sphinxcode{\sphinxupquote{direction}} is \sphinxcode{\sphinxupquote{False}} the
only possible mode is \sphinxcode{\sphinxupquote{ALL}}. If the \sphinxcode{\sphinxupquote{direction}} is \sphinxcode{\sphinxupquote{None}}
and also directed evidence(s) match the criteria these will
overwrite the undirected evidences and only the directed result
will be returned.

\end{description}\end{quote}

\end{fulllineitems}

\index{get\_degrees\_directed\_out() (pypath.core.interaction.Interaction method)@\spxentry{get\_degrees\_directed\_out()}\spxextra{pypath.core.interaction.Interaction method}}

\begin{fulllineitems}
\phantomsection\label{\detokenize{reference:pypath.core.interaction.Interaction.get_degrees_directed_out}}\pysiglinewithargsret{\sphinxbfcode{\sphinxupquote{get\_degrees\_directed\_out}}}{\emph{direction=None}, \emph{effect=None}, \emph{resources=None}, \emph{data\_model=None}, \emph{interaction\_type=None}, \emph{via=None}, \emph{references=None}}{}
Returns a \sphinxstyleemphasis{set} of nodes with the connections matching the direction,
effect and evidence criteria. E.g. if the query concerns the incoming
degrees with positive effect and the matching evidences show A
activates B, but not the other way around, only “B” will be returned.
\begin{quote}\begin{description}
\item[{Parameters}] \leavevmode
\sphinxstyleliteralstrong{\sphinxupquote{mode}} (\sphinxstyleliteralemphasis{\sphinxupquote{str}}) \textendash{} The type of degrees to be considered. Three possible values are
\sphinxcode{\sphinxupquote{'IN'}}, \sphinxtitleref{‘OUT’{}`} and \sphinxcode{\sphinxupquote{'ALL'}} for incoming, outgoing and all
connections, respectively. If the \sphinxcode{\sphinxupquote{direction}} is \sphinxcode{\sphinxupquote{False}} the
only possible mode is \sphinxcode{\sphinxupquote{ALL}}. If the \sphinxcode{\sphinxupquote{direction}} is \sphinxcode{\sphinxupquote{None}}
and also directed evidence(s) match the criteria these will
overwrite the undirected evidences and only the directed result
will be returned.

\end{description}\end{quote}

\end{fulllineitems}

\index{get\_degrees\_negative() (pypath.core.interaction.Interaction method)@\spxentry{get\_degrees\_negative()}\spxextra{pypath.core.interaction.Interaction method}}

\begin{fulllineitems}
\phantomsection\label{\detokenize{reference:pypath.core.interaction.Interaction.get_degrees_negative}}\pysiglinewithargsret{\sphinxbfcode{\sphinxupquote{get\_degrees\_negative}}}{\emph{direction=None}, \emph{effect=None}, \emph{resources=None}, \emph{data\_model=None}, \emph{interaction\_type=None}, \emph{via=None}, \emph{references=None}}{}
Returns a \sphinxstyleemphasis{set} of nodes with the connections matching the direction,
effect and evidence criteria. E.g. if the query concerns the incoming
degrees with positive effect and the matching evidences show A
activates B, but not the other way around, only “B” will be returned.
\begin{quote}\begin{description}
\item[{Parameters}] \leavevmode
\sphinxstyleliteralstrong{\sphinxupquote{mode}} (\sphinxstyleliteralemphasis{\sphinxupquote{str}}) \textendash{} The type of degrees to be considered. Three possible values are
\sphinxcode{\sphinxupquote{'IN'}}, \sphinxtitleref{‘OUT’{}`} and \sphinxcode{\sphinxupquote{'ALL'}} for incoming, outgoing and all
connections, respectively. If the \sphinxcode{\sphinxupquote{direction}} is \sphinxcode{\sphinxupquote{False}} the
only possible mode is \sphinxcode{\sphinxupquote{ALL}}. If the \sphinxcode{\sphinxupquote{direction}} is \sphinxcode{\sphinxupquote{None}}
and also directed evidence(s) match the criteria these will
overwrite the undirected evidences and only the directed result
will be returned.

\end{description}\end{quote}

\end{fulllineitems}

\index{get\_degrees\_negative\_in() (pypath.core.interaction.Interaction method)@\spxentry{get\_degrees\_negative\_in()}\spxextra{pypath.core.interaction.Interaction method}}

\begin{fulllineitems}
\phantomsection\label{\detokenize{reference:pypath.core.interaction.Interaction.get_degrees_negative_in}}\pysiglinewithargsret{\sphinxbfcode{\sphinxupquote{get\_degrees\_negative\_in}}}{\emph{direction=None}, \emph{effect=None}, \emph{resources=None}, \emph{data\_model=None}, \emph{interaction\_type=None}, \emph{via=None}, \emph{references=None}}{}
Returns a \sphinxstyleemphasis{set} of nodes with the connections matching the direction,
effect and evidence criteria. E.g. if the query concerns the incoming
degrees with positive effect and the matching evidences show A
activates B, but not the other way around, only “B” will be returned.
\begin{quote}\begin{description}
\item[{Parameters}] \leavevmode
\sphinxstyleliteralstrong{\sphinxupquote{mode}} (\sphinxstyleliteralemphasis{\sphinxupquote{str}}) \textendash{} The type of degrees to be considered. Three possible values are
\sphinxcode{\sphinxupquote{'IN'}}, \sphinxtitleref{‘OUT’{}`} and \sphinxcode{\sphinxupquote{'ALL'}} for incoming, outgoing and all
connections, respectively. If the \sphinxcode{\sphinxupquote{direction}} is \sphinxcode{\sphinxupquote{False}} the
only possible mode is \sphinxcode{\sphinxupquote{ALL}}. If the \sphinxcode{\sphinxupquote{direction}} is \sphinxcode{\sphinxupquote{None}}
and also directed evidence(s) match the criteria these will
overwrite the undirected evidences and only the directed result
will be returned.

\end{description}\end{quote}

\end{fulllineitems}

\index{get\_degrees\_negative\_out() (pypath.core.interaction.Interaction method)@\spxentry{get\_degrees\_negative\_out()}\spxextra{pypath.core.interaction.Interaction method}}

\begin{fulllineitems}
\phantomsection\label{\detokenize{reference:pypath.core.interaction.Interaction.get_degrees_negative_out}}\pysiglinewithargsret{\sphinxbfcode{\sphinxupquote{get\_degrees\_negative\_out}}}{\emph{direction=None}, \emph{effect=None}, \emph{resources=None}, \emph{data\_model=None}, \emph{interaction\_type=None}, \emph{via=None}, \emph{references=None}}{}
Returns a \sphinxstyleemphasis{set} of nodes with the connections matching the direction,
effect and evidence criteria. E.g. if the query concerns the incoming
degrees with positive effect and the matching evidences show A
activates B, but not the other way around, only “B” will be returned.
\begin{quote}\begin{description}
\item[{Parameters}] \leavevmode
\sphinxstyleliteralstrong{\sphinxupquote{mode}} (\sphinxstyleliteralemphasis{\sphinxupquote{str}}) \textendash{} The type of degrees to be considered. Three possible values are
\sphinxcode{\sphinxupquote{'IN'}}, \sphinxtitleref{‘OUT’{}`} and \sphinxcode{\sphinxupquote{'ALL'}} for incoming, outgoing and all
connections, respectively. If the \sphinxcode{\sphinxupquote{direction}} is \sphinxcode{\sphinxupquote{False}} the
only possible mode is \sphinxcode{\sphinxupquote{ALL}}. If the \sphinxcode{\sphinxupquote{direction}} is \sphinxcode{\sphinxupquote{None}}
and also directed evidence(s) match the criteria these will
overwrite the undirected evidences and only the directed result
will be returned.

\end{description}\end{quote}

\end{fulllineitems}

\index{get\_degrees\_non\_directed() (pypath.core.interaction.Interaction method)@\spxentry{get\_degrees\_non\_directed()}\spxextra{pypath.core.interaction.Interaction method}}

\begin{fulllineitems}
\phantomsection\label{\detokenize{reference:pypath.core.interaction.Interaction.get_degrees_non_directed}}\pysiglinewithargsret{\sphinxbfcode{\sphinxupquote{get\_degrees\_non\_directed}}}{\emph{direction=None}, \emph{effect=None}, \emph{resources=None}, \emph{data\_model=None}, \emph{interaction\_type=None}, \emph{via=None}, \emph{references=None}}{}
Returns a \sphinxstyleemphasis{set} of nodes with the connections matching the direction,
effect and evidence criteria. E.g. if the query concerns the incoming
degrees with positive effect and the matching evidences show A
activates B, but not the other way around, only “B” will be returned.
\begin{quote}\begin{description}
\item[{Parameters}] \leavevmode
\sphinxstyleliteralstrong{\sphinxupquote{mode}} (\sphinxstyleliteralemphasis{\sphinxupquote{str}}) \textendash{} The type of degrees to be considered. Three possible values are
\sphinxcode{\sphinxupquote{'IN'}}, \sphinxtitleref{‘OUT’{}`} and \sphinxcode{\sphinxupquote{'ALL'}} for incoming, outgoing and all
connections, respectively. If the \sphinxcode{\sphinxupquote{direction}} is \sphinxcode{\sphinxupquote{False}} the
only possible mode is \sphinxcode{\sphinxupquote{ALL}}. If the \sphinxcode{\sphinxupquote{direction}} is \sphinxcode{\sphinxupquote{None}}
and also directed evidence(s) match the criteria these will
overwrite the undirected evidences and only the directed result
will be returned.

\end{description}\end{quote}

\end{fulllineitems}

\index{get\_degrees\_positive() (pypath.core.interaction.Interaction method)@\spxentry{get\_degrees\_positive()}\spxextra{pypath.core.interaction.Interaction method}}

\begin{fulllineitems}
\phantomsection\label{\detokenize{reference:pypath.core.interaction.Interaction.get_degrees_positive}}\pysiglinewithargsret{\sphinxbfcode{\sphinxupquote{get\_degrees\_positive}}}{\emph{direction=None}, \emph{effect=None}, \emph{resources=None}, \emph{data\_model=None}, \emph{interaction\_type=None}, \emph{via=None}, \emph{references=None}}{}
Returns a \sphinxstyleemphasis{set} of nodes with the connections matching the direction,
effect and evidence criteria. E.g. if the query concerns the incoming
degrees with positive effect and the matching evidences show A
activates B, but not the other way around, only “B” will be returned.
\begin{quote}\begin{description}
\item[{Parameters}] \leavevmode
\sphinxstyleliteralstrong{\sphinxupquote{mode}} (\sphinxstyleliteralemphasis{\sphinxupquote{str}}) \textendash{} The type of degrees to be considered. Three possible values are
\sphinxcode{\sphinxupquote{'IN'}}, \sphinxtitleref{‘OUT’{}`} and \sphinxcode{\sphinxupquote{'ALL'}} for incoming, outgoing and all
connections, respectively. If the \sphinxcode{\sphinxupquote{direction}} is \sphinxcode{\sphinxupquote{False}} the
only possible mode is \sphinxcode{\sphinxupquote{ALL}}. If the \sphinxcode{\sphinxupquote{direction}} is \sphinxcode{\sphinxupquote{None}}
and also directed evidence(s) match the criteria these will
overwrite the undirected evidences and only the directed result
will be returned.

\end{description}\end{quote}

\end{fulllineitems}

\index{get\_degrees\_positive\_in() (pypath.core.interaction.Interaction method)@\spxentry{get\_degrees\_positive\_in()}\spxextra{pypath.core.interaction.Interaction method}}

\begin{fulllineitems}
\phantomsection\label{\detokenize{reference:pypath.core.interaction.Interaction.get_degrees_positive_in}}\pysiglinewithargsret{\sphinxbfcode{\sphinxupquote{get\_degrees\_positive\_in}}}{\emph{direction=None}, \emph{effect=None}, \emph{resources=None}, \emph{data\_model=None}, \emph{interaction\_type=None}, \emph{via=None}, \emph{references=None}}{}
Returns a \sphinxstyleemphasis{set} of nodes with the connections matching the direction,
effect and evidence criteria. E.g. if the query concerns the incoming
degrees with positive effect and the matching evidences show A
activates B, but not the other way around, only “B” will be returned.
\begin{quote}\begin{description}
\item[{Parameters}] \leavevmode
\sphinxstyleliteralstrong{\sphinxupquote{mode}} (\sphinxstyleliteralemphasis{\sphinxupquote{str}}) \textendash{} The type of degrees to be considered. Three possible values are
\sphinxcode{\sphinxupquote{'IN'}}, \sphinxtitleref{‘OUT’{}`} and \sphinxcode{\sphinxupquote{'ALL'}} for incoming, outgoing and all
connections, respectively. If the \sphinxcode{\sphinxupquote{direction}} is \sphinxcode{\sphinxupquote{False}} the
only possible mode is \sphinxcode{\sphinxupquote{ALL}}. If the \sphinxcode{\sphinxupquote{direction}} is \sphinxcode{\sphinxupquote{None}}
and also directed evidence(s) match the criteria these will
overwrite the undirected evidences and only the directed result
will be returned.

\end{description}\end{quote}

\end{fulllineitems}

\index{get\_degrees\_positive\_out() (pypath.core.interaction.Interaction method)@\spxentry{get\_degrees\_positive\_out()}\spxextra{pypath.core.interaction.Interaction method}}

\begin{fulllineitems}
\phantomsection\label{\detokenize{reference:pypath.core.interaction.Interaction.get_degrees_positive_out}}\pysiglinewithargsret{\sphinxbfcode{\sphinxupquote{get\_degrees\_positive\_out}}}{\emph{direction=None}, \emph{effect=None}, \emph{resources=None}, \emph{data\_model=None}, \emph{interaction\_type=None}, \emph{via=None}, \emph{references=None}}{}
Returns a \sphinxstyleemphasis{set} of nodes with the connections matching the direction,
effect and evidence criteria. E.g. if the query concerns the incoming
degrees with positive effect and the matching evidences show A
activates B, but not the other way around, only “B” will be returned.
\begin{quote}\begin{description}
\item[{Parameters}] \leavevmode
\sphinxstyleliteralstrong{\sphinxupquote{mode}} (\sphinxstyleliteralemphasis{\sphinxupquote{str}}) \textendash{} The type of degrees to be considered. Three possible values are
\sphinxcode{\sphinxupquote{'IN'}}, \sphinxtitleref{‘OUT’{}`} and \sphinxcode{\sphinxupquote{'ALL'}} for incoming, outgoing and all
connections, respectively. If the \sphinxcode{\sphinxupquote{direction}} is \sphinxcode{\sphinxupquote{False}} the
only possible mode is \sphinxcode{\sphinxupquote{ALL}}. If the \sphinxcode{\sphinxupquote{direction}} is \sphinxcode{\sphinxupquote{None}}
and also directed evidence(s) match the criteria these will
overwrite the undirected evidences and only the directed result
will be returned.

\end{description}\end{quote}

\end{fulllineitems}

\index{get\_degrees\_signed() (pypath.core.interaction.Interaction method)@\spxentry{get\_degrees\_signed()}\spxextra{pypath.core.interaction.Interaction method}}

\begin{fulllineitems}
\phantomsection\label{\detokenize{reference:pypath.core.interaction.Interaction.get_degrees_signed}}\pysiglinewithargsret{\sphinxbfcode{\sphinxupquote{get\_degrees\_signed}}}{\emph{direction=None}, \emph{effect=None}, \emph{resources=None}, \emph{data\_model=None}, \emph{interaction\_type=None}, \emph{via=None}, \emph{references=None}}{}
Returns a \sphinxstyleemphasis{set} of nodes with the connections matching the direction,
effect and evidence criteria. E.g. if the query concerns the incoming
degrees with positive effect and the matching evidences show A
activates B, but not the other way around, only “B” will be returned.
\begin{quote}\begin{description}
\item[{Parameters}] \leavevmode
\sphinxstyleliteralstrong{\sphinxupquote{mode}} (\sphinxstyleliteralemphasis{\sphinxupquote{str}}) \textendash{} The type of degrees to be considered. Three possible values are
\sphinxcode{\sphinxupquote{'IN'}}, \sphinxtitleref{‘OUT’{}`} and \sphinxcode{\sphinxupquote{'ALL'}} for incoming, outgoing and all
connections, respectively. If the \sphinxcode{\sphinxupquote{direction}} is \sphinxcode{\sphinxupquote{False}} the
only possible mode is \sphinxcode{\sphinxupquote{ALL}}. If the \sphinxcode{\sphinxupquote{direction}} is \sphinxcode{\sphinxupquote{None}}
and also directed evidence(s) match the criteria these will
overwrite the undirected evidences and only the directed result
will be returned.

\end{description}\end{quote}

\end{fulllineitems}

\index{get\_degrees\_signed\_in() (pypath.core.interaction.Interaction method)@\spxentry{get\_degrees\_signed\_in()}\spxextra{pypath.core.interaction.Interaction method}}

\begin{fulllineitems}
\phantomsection\label{\detokenize{reference:pypath.core.interaction.Interaction.get_degrees_signed_in}}\pysiglinewithargsret{\sphinxbfcode{\sphinxupquote{get\_degrees\_signed\_in}}}{\emph{direction=None}, \emph{effect=None}, \emph{resources=None}, \emph{data\_model=None}, \emph{interaction\_type=None}, \emph{via=None}, \emph{references=None}}{}
Returns a \sphinxstyleemphasis{set} of nodes with the connections matching the direction,
effect and evidence criteria. E.g. if the query concerns the incoming
degrees with positive effect and the matching evidences show A
activates B, but not the other way around, only “B” will be returned.
\begin{quote}\begin{description}
\item[{Parameters}] \leavevmode
\sphinxstyleliteralstrong{\sphinxupquote{mode}} (\sphinxstyleliteralemphasis{\sphinxupquote{str}}) \textendash{} The type of degrees to be considered. Three possible values are
\sphinxcode{\sphinxupquote{'IN'}}, \sphinxtitleref{‘OUT’{}`} and \sphinxcode{\sphinxupquote{'ALL'}} for incoming, outgoing and all
connections, respectively. If the \sphinxcode{\sphinxupquote{direction}} is \sphinxcode{\sphinxupquote{False}} the
only possible mode is \sphinxcode{\sphinxupquote{ALL}}. If the \sphinxcode{\sphinxupquote{direction}} is \sphinxcode{\sphinxupquote{None}}
and also directed evidence(s) match the criteria these will
overwrite the undirected evidences and only the directed result
will be returned.

\end{description}\end{quote}

\end{fulllineitems}

\index{get\_degrees\_signed\_out() (pypath.core.interaction.Interaction method)@\spxentry{get\_degrees\_signed\_out()}\spxextra{pypath.core.interaction.Interaction method}}

\begin{fulllineitems}
\phantomsection\label{\detokenize{reference:pypath.core.interaction.Interaction.get_degrees_signed_out}}\pysiglinewithargsret{\sphinxbfcode{\sphinxupquote{get\_degrees\_signed\_out}}}{\emph{direction=None}, \emph{effect=None}, \emph{resources=None}, \emph{data\_model=None}, \emph{interaction\_type=None}, \emph{via=None}, \emph{references=None}}{}
Returns a \sphinxstyleemphasis{set} of nodes with the connections matching the direction,
effect and evidence criteria. E.g. if the query concerns the incoming
degrees with positive effect and the matching evidences show A
activates B, but not the other way around, only “B” will be returned.
\begin{quote}\begin{description}
\item[{Parameters}] \leavevmode
\sphinxstyleliteralstrong{\sphinxupquote{mode}} (\sphinxstyleliteralemphasis{\sphinxupquote{str}}) \textendash{} The type of degrees to be considered. Three possible values are
\sphinxcode{\sphinxupquote{'IN'}}, \sphinxtitleref{‘OUT’{}`} and \sphinxcode{\sphinxupquote{'ALL'}} for incoming, outgoing and all
connections, respectively. If the \sphinxcode{\sphinxupquote{direction}} is \sphinxcode{\sphinxupquote{False}} the
only possible mode is \sphinxcode{\sphinxupquote{ALL}}. If the \sphinxcode{\sphinxupquote{direction}} is \sphinxcode{\sphinxupquote{None}}
and also directed evidence(s) match the criteria these will
overwrite the undirected evidences and only the directed result
will be returned.

\end{description}\end{quote}

\end{fulllineitems}

\index{get\_degrees\_undirected() (pypath.core.interaction.Interaction method)@\spxentry{get\_degrees\_undirected()}\spxextra{pypath.core.interaction.Interaction method}}

\begin{fulllineitems}
\phantomsection\label{\detokenize{reference:pypath.core.interaction.Interaction.get_degrees_undirected}}\pysiglinewithargsret{\sphinxbfcode{\sphinxupquote{get\_degrees\_undirected}}}{\emph{direction=None}, \emph{effect=None}, \emph{resources=None}, \emph{data\_model=None}, \emph{interaction\_type=None}, \emph{via=None}, \emph{references=None}}{}
Returns a \sphinxstyleemphasis{set} of nodes with the connections matching the direction,
effect and evidence criteria. E.g. if the query concerns the incoming
degrees with positive effect and the matching evidences show A
activates B, but not the other way around, only “B” will be returned.
\begin{quote}\begin{description}
\item[{Parameters}] \leavevmode
\sphinxstyleliteralstrong{\sphinxupquote{mode}} (\sphinxstyleliteralemphasis{\sphinxupquote{str}}) \textendash{} The type of degrees to be considered. Three possible values are
\sphinxcode{\sphinxupquote{'IN'}}, \sphinxtitleref{‘OUT’{}`} and \sphinxcode{\sphinxupquote{'ALL'}} for incoming, outgoing and all
connections, respectively. If the \sphinxcode{\sphinxupquote{direction}} is \sphinxcode{\sphinxupquote{False}} the
only possible mode is \sphinxcode{\sphinxupquote{ALL}}. If the \sphinxcode{\sphinxupquote{direction}} is \sphinxcode{\sphinxupquote{None}}
and also directed evidence(s) match the criteria these will
overwrite the undirected evidences and only the directed result
will be returned.

\end{description}\end{quote}

\end{fulllineitems}

\index{get\_direction() (pypath.core.interaction.Interaction method)@\spxentry{get\_direction()}\spxextra{pypath.core.interaction.Interaction method}}

\begin{fulllineitems}
\phantomsection\label{\detokenize{reference:pypath.core.interaction.Interaction.get_direction}}\pysiglinewithargsret{\sphinxbfcode{\sphinxupquote{get\_direction}}}{\emph{direction}, \emph{resources=False}, \emph{evidences=False}, \emph{sources=False}, \emph{resource\_names=False}}{}
Returns the state (or \sphinxstyleemphasis{resources} if specified) of the given
\sphinxstyleemphasis{direction}.
\begin{quote}\begin{description}
\item[{Parameters}] \leavevmode\begin{itemize}
\item {} 
\sphinxstyleliteralstrong{\sphinxupquote{direction}} (\sphinxstyleliteralemphasis{\sphinxupquote{tuple}}) \textendash{} Or {[}str{]} (if \sphinxcode{\sphinxupquote{'undirected'}}). Pair of nodes from which
direction information is to be retrieved.

\item {} 
\sphinxstyleliteralstrong{\sphinxupquote{resources}} (\sphinxstyleliteralemphasis{\sphinxupquote{bool}}) \textendash{} Optional, \sphinxcode{\sphinxupquote{'False'}} by default. Specifies if the
\sphinxcode{\sphinxupquote{resources}} information of the given direction is to
be retrieved instead.

\end{itemize}

\item[{Returns}] \leavevmode
(\sphinxstyleemphasis{bool} or \sphinxstyleemphasis{set}) \textendash{} (if \sphinxcode{\sphinxupquote{resources=True}}). Presence/absence
of the requested direction (or the list of resources if
specified). Returns \sphinxcode{\sphinxupquote{None}} if \sphinxstyleemphasis{direction} is not valid.

\end{description}\end{quote}

\end{fulllineitems}

\index{get\_directions() (pypath.core.interaction.Interaction method)@\spxentry{get\_directions()}\spxextra{pypath.core.interaction.Interaction method}}

\begin{fulllineitems}
\phantomsection\label{\detokenize{reference:pypath.core.interaction.Interaction.get_directions}}\pysiglinewithargsret{\sphinxbfcode{\sphinxupquote{get\_directions}}}{\emph{src}, \emph{tgt}, \emph{resources=False}, \emph{evidences=False}, \emph{resource\_names=False}, \emph{sources=False}}{}
Returns all directions with boolean values or list of sources.
\begin{quote}\begin{description}
\item[{Parameters}] \leavevmode\begin{itemize}
\item {} 
\sphinxstyleliteralstrong{\sphinxupquote{src}} (\sphinxstyleliteralemphasis{\sphinxupquote{str}}) \textendash{} Source node.

\item {} 
\sphinxstyleliteralstrong{\sphinxupquote{tgt}} (\sphinxstyleliteralemphasis{\sphinxupquote{str}}) \textendash{} Target node.

\item {} 
\sphinxstyleliteralstrong{\sphinxupquote{resources}} (\sphinxstyleliteralemphasis{\sphinxupquote{bool}}) \textendash{} Optional, \sphinxcode{\sphinxupquote{False}} by default. Specifies whether to return
the \sphinxcode{\sphinxupquote{resources}} attribute instead of \sphinxcode{\sphinxupquote{dirs}}.

\end{itemize}

\item[{Returns}] \leavevmode
Contains the \sphinxcode{\sphinxupquote{dirs}} (or \sphinxcode{\sphinxupquote{resources}} if
specified) of the given edge.

\end{description}\end{quote}

\end{fulllineitems}

\index{get\_entities() (pypath.core.interaction.Interaction method)@\spxentry{get\_entities()}\spxextra{pypath.core.interaction.Interaction method}}

\begin{fulllineitems}
\phantomsection\label{\detokenize{reference:pypath.core.interaction.Interaction.get_entities}}\pysiglinewithargsret{\sphinxbfcode{\sphinxupquote{get\_entities}}}{\emph{entity\_type=None}, \emph{direction=None}, \emph{effect=None}, \emph{resources=None}, \emph{data\_model=None}, \emph{interaction\_type=None}, \emph{via=None}, \emph{references=None}, \emph{return\_type=None}}{}
Retrieves the entities involved in interactions matching the criteria.
It either returns both interacting entities in a \sphinxstyleemphasis{set} or an empty
\sphinxstyleemphasis{set}. This may not sound so useful at the level of this object but
becomes more useful once we want to collect entities having certain
kind of interactions across a series of \sphinxtitleref{Interaction} objects.
\begin{quote}\begin{description}
\item[{Parameters}] \leavevmode\begin{itemize}
\item {} 
\sphinxstyleliteralstrong{\sphinxupquote{entity\_type}} (\sphinxstyleliteralemphasis{\sphinxupquote{str}}) \textendash{} The type of the molecular entity. Possible values: \sphinxtitleref{protein},
\sphinxtitleref{complex}, \sphinxtitleref{mirna}, \sphinxtitleref{small\_molecule}.

\item {} 
\sphinxstyleliteralstrong{\sphinxupquote{return\_type}} (\sphinxstyleliteralemphasis{\sphinxupquote{str}}) \textendash{} The type of values to return. Default is
py:class:\sphinxcode{\sphinxupquote{pypath.entity.Entity}} objects, alternatives are
\sphinxcode{\sphinxupquote{labels}}  \sphinxcode{\sphinxupquote{identifiers}}.

\end{itemize}

\end{description}\end{quote}

\end{fulllineitems}

\index{get\_identifiers() (pypath.core.interaction.Interaction method)@\spxentry{get\_identifiers()}\spxextra{pypath.core.interaction.Interaction method}}

\begin{fulllineitems}
\phantomsection\label{\detokenize{reference:pypath.core.interaction.Interaction.get_identifiers}}\pysiglinewithargsret{\sphinxbfcode{\sphinxupquote{get\_identifiers}}}{\emph{entity\_type=None}, \emph{direction=None}, \emph{effect=None}, \emph{resources=None}, \emph{data\_model=None}, \emph{interaction\_type=None}, \emph{via=None}, \emph{references=None}, \emph{return\_type=None}}{}
Retrieves the entities involved in interactions matching the criteria.
It either returns both interacting entities in a \sphinxstyleemphasis{set} or an empty
\sphinxstyleemphasis{set}. This may not sound so useful at the level of this object but
becomes more useful once we want to collect entities having certain
kind of interactions across a series of \sphinxtitleref{Interaction} objects.
\begin{quote}\begin{description}
\item[{Parameters}] \leavevmode\begin{itemize}
\item {} 
\sphinxstyleliteralstrong{\sphinxupquote{entity\_type}} (\sphinxstyleliteralemphasis{\sphinxupquote{str}}) \textendash{} The type of the molecular entity. Possible values: \sphinxtitleref{protein},
\sphinxtitleref{complex}, \sphinxtitleref{mirna}, \sphinxtitleref{small\_molecule}.

\item {} 
\sphinxstyleliteralstrong{\sphinxupquote{return\_type}} (\sphinxstyleliteralemphasis{\sphinxupquote{str}}) \textendash{} The type of values to return. Default is
py:class:\sphinxcode{\sphinxupquote{pypath.entity.Entity}} objects, alternatives are
\sphinxcode{\sphinxupquote{labels}}  \sphinxcode{\sphinxupquote{identifiers}}.

\end{itemize}

\end{description}\end{quote}

\end{fulllineitems}

\index{get\_interaction\_types() (pypath.core.interaction.Interaction method)@\spxentry{get\_interaction\_types()}\spxextra{pypath.core.interaction.Interaction method}}

\begin{fulllineitems}
\phantomsection\label{\detokenize{reference:pypath.core.interaction.Interaction.get_interaction_types}}\pysiglinewithargsret{\sphinxbfcode{\sphinxupquote{get\_interaction\_types}}}{\emph{effect=None}, \emph{resources=None}, \emph{data\_model=None}, \emph{interaction\_type=None}, \emph{via=None}, \emph{references=None}, \emph{entity\_type=None}, \emph{source\_entity\_type=None}, \emph{target\_entity\_type=None}}{}
Retrieves interaction types matching the criteria.

\end{fulllineitems}

\index{get\_interactions() (pypath.core.interaction.Interaction method)@\spxentry{get\_interactions()}\spxextra{pypath.core.interaction.Interaction method}}

\begin{fulllineitems}
\phantomsection\label{\detokenize{reference:pypath.core.interaction.Interaction.get_interactions}}\pysiglinewithargsret{\sphinxbfcode{\sphinxupquote{get\_interactions}}}{\emph{direction=None}, \emph{effect=None}, \emph{resources=None}, \emph{data\_model=None}, \emph{interaction\_type=None}, \emph{via=None}, \emph{references=None}, \emph{entity\_type=None}, \emph{source\_entity\_type=None}, \emph{target\_entity\_type=None}}{}
Returns one or two tuples of the interacting partners: one if only
one direction, two if both directions match the query criteria.
The tuple will be empty if no evidence matches the criteria.
\begin{quote}\begin{description}
\item[{Parameters}] \leavevmode\begin{itemize}
\item {} 
\sphinxstyleliteralstrong{\sphinxupquote{direction}} (\sphinxstyleliteralemphasis{\sphinxupquote{NontType}}\sphinxstyleliteralemphasis{\sphinxupquote{,}}\sphinxstyleliteralemphasis{\sphinxupquote{bool}}\sphinxstyleliteralemphasis{\sphinxupquote{,}}\sphinxstyleliteralemphasis{\sphinxupquote{tuple}}) \textendash{} If \sphinxtitleref{None} both undirected and directed, if \sphinxtitleref{True} only directed,
if a \sphinxstyleemphasis{tuple} of entities only the interactions with that specific
direction will be considered. Unless you set this parameter to
\sphinxtitleref{True} this method will return both directions if one or more
undirected resources present.
If \sphinxtitleref{False}, only the undirected interactions will be considered,
and if any resource annotates this interaction as undirected
both directions will be returned. However the
\sphinxcode{\sphinxupquote{count\_interactions\_undirected}} method will return \sphinxtitleref{1}
in this case.

\item {} 
\sphinxstyleliteralstrong{\sphinxupquote{effect}} (\sphinxstyleliteralemphasis{\sphinxupquote{NoneType}}\sphinxstyleliteralemphasis{\sphinxupquote{,}}\sphinxstyleliteralemphasis{\sphinxupquote{bool}}\sphinxstyleliteralemphasis{\sphinxupquote{,}}\sphinxstyleliteralemphasis{\sphinxupquote{str}}) \textendash{} If \sphinxtitleref{None} also interactions without effect, if \sphinxtitleref{True} only
the ones with any effect, if a string naming an effect only the
interactions with that specific effect will be considered.

\item {} 
\sphinxstyleliteralstrong{\sphinxupquote{resources}} (\sphinxstyleliteralemphasis{\sphinxupquote{NontType}}\sphinxstyleliteralemphasis{\sphinxupquote{,}}\sphinxstyleliteralemphasis{\sphinxupquote{str}}\sphinxstyleliteralemphasis{\sphinxupquote{,}}\sphinxstyleliteralemphasis{\sphinxupquote{set}}) \textendash{} Optionally limit the query to one or more resources.

\item {} 
\sphinxstyleliteralstrong{\sphinxupquote{data\_model}} (\sphinxstyleliteralemphasis{\sphinxupquote{NontType}}\sphinxstyleliteralemphasis{\sphinxupquote{,}}\sphinxstyleliteralemphasis{\sphinxupquote{str}}\sphinxstyleliteralemphasis{\sphinxupquote{,}}\sphinxstyleliteralemphasis{\sphinxupquote{set}}) \textendash{} Optionally limit the query to one or more data models e.g.
\sphinxtitleref{activity\_flow}.

\item {} 
\sphinxstyleliteralstrong{\sphinxupquote{interaction\_type}} (\sphinxstyleliteralemphasis{\sphinxupquote{NontType}}\sphinxstyleliteralemphasis{\sphinxupquote{,}}\sphinxstyleliteralemphasis{\sphinxupquote{str}}\sphinxstyleliteralemphasis{\sphinxupquote{,}}\sphinxstyleliteralemphasis{\sphinxupquote{set}}) \textendash{} Optionally limit the query to one or more interaction types
e.g. \sphinxtitleref{PPI}.

\item {} 
\sphinxstyleliteralstrong{\sphinxupquote{via}} (\sphinxstyleliteralemphasis{\sphinxupquote{NontType}}\sphinxstyleliteralemphasis{\sphinxupquote{,}}\sphinxstyleliteralemphasis{\sphinxupquote{bool}}\sphinxstyleliteralemphasis{\sphinxupquote{,}}\sphinxstyleliteralemphasis{\sphinxupquote{str}}\sphinxstyleliteralemphasis{\sphinxupquote{,}}\sphinxstyleliteralemphasis{\sphinxupquote{set}}) \textendash{} Optionally limit the query to certain secondary databases or
if \sphinxtitleref{False} consider only data from primary databases.

\item {} 
\sphinxstyleliteralstrong{\sphinxupquote{entity\_type}} (\sphinxstyleliteralemphasis{\sphinxupquote{str}}) \textendash{} Molecule type for both of the entities.

\item {} 
\sphinxstyleliteralstrong{\sphinxupquote{source\_entity\_type}} (\sphinxstyleliteralemphasis{\sphinxupquote{str}}) \textendash{} Molecule type for the source entity.

\item {} 
\sphinxstyleliteralstrong{\sphinxupquote{target\_entity\_type}} (\sphinxstyleliteralemphasis{\sphinxupquote{str}}) \textendash{} Molecule type for the target entity.

\end{itemize}

\end{description}\end{quote}

\end{fulllineitems}

\index{get\_interactions\_0() (pypath.core.interaction.Interaction method)@\spxentry{get\_interactions\_0()}\spxextra{pypath.core.interaction.Interaction method}}

\begin{fulllineitems}
\phantomsection\label{\detokenize{reference:pypath.core.interaction.Interaction.get_interactions_0}}\pysiglinewithargsret{\sphinxbfcode{\sphinxupquote{get\_interactions\_0}}}{\emph{**kwargs}}{}
Returns unique interacting pairs without being aware of the direction.

\end{fulllineitems}

\index{get\_interactions\_directed() (pypath.core.interaction.Interaction method)@\spxentry{get\_interactions\_directed()}\spxextra{pypath.core.interaction.Interaction method}}

\begin{fulllineitems}
\phantomsection\label{\detokenize{reference:pypath.core.interaction.Interaction.get_interactions_directed}}\pysiglinewithargsret{\sphinxbfcode{\sphinxupquote{get\_interactions\_directed}}}{\emph{**kwargs}}{}
{\color{red}\bfseries{}**}kwargs: see the docs of method \sphinxcode{\sphinxupquote{get\_interactions}}.

\end{fulllineitems}

\index{get\_interactions\_mutual() (pypath.core.interaction.Interaction method)@\spxentry{get\_interactions\_mutual()}\spxextra{pypath.core.interaction.Interaction method}}

\begin{fulllineitems}
\phantomsection\label{\detokenize{reference:pypath.core.interaction.Interaction.get_interactions_mutual}}\pysiglinewithargsret{\sphinxbfcode{\sphinxupquote{get\_interactions\_mutual}}}{\emph{**kwargs}}{}
Note: undirected interactions does not count as mutual but only
interactions with explicit direction information for both directions.

{\color{red}\bfseries{}**}kwargs: see the docs of method \sphinxcode{\sphinxupquote{get\_interactions}}.

\end{fulllineitems}

\index{get\_interactions\_negative() (pypath.core.interaction.Interaction method)@\spxentry{get\_interactions\_negative()}\spxextra{pypath.core.interaction.Interaction method}}

\begin{fulllineitems}
\phantomsection\label{\detokenize{reference:pypath.core.interaction.Interaction.get_interactions_negative}}\pysiglinewithargsret{\sphinxbfcode{\sphinxupquote{get\_interactions\_negative}}}{\emph{**kwargs}}{}
{\color{red}\bfseries{}**}kwargs: see the docs of method \sphinxcode{\sphinxupquote{get\_interactions}}.

\end{fulllineitems}

\index{get\_interactions\_non\_directed() (pypath.core.interaction.Interaction method)@\spxentry{get\_interactions\_non\_directed()}\spxextra{pypath.core.interaction.Interaction method}}

\begin{fulllineitems}
\phantomsection\label{\detokenize{reference:pypath.core.interaction.Interaction.get_interactions_non_directed}}\pysiglinewithargsret{\sphinxbfcode{\sphinxupquote{get\_interactions\_non\_directed}}}{\emph{**kwargs}}{}
Only the undirected interactions will be considered, if any resource
annotates this interaction as undirected both directions will be
returned, but only if no resource provide direction. However
the \sphinxcode{\sphinxupquote{count\_interactions\_non\_directed}} method will return \sphinxtitleref{1} in
this case.

{\color{red}\bfseries{}**}kwargs: see the docs of method \sphinxcode{\sphinxupquote{get\_interactions}}.

\end{fulllineitems}

\index{get\_interactions\_non\_directed\_0() (pypath.core.interaction.Interaction method)@\spxentry{get\_interactions\_non\_directed\_0()}\spxextra{pypath.core.interaction.Interaction method}}

\begin{fulllineitems}
\phantomsection\label{\detokenize{reference:pypath.core.interaction.Interaction.get_interactions_non_directed_0}}\pysiglinewithargsret{\sphinxbfcode{\sphinxupquote{get\_interactions\_non\_directed\_0}}}{\emph{**kwargs}}{}
Only the undirected interactions will be considered, if any resource
annotates this interaction as undirected and none as directed, the
interacting pair as a sorted tuple will be returned inside a one
element tuple.

{\color{red}\bfseries{}**}kwargs: see the docs of method \sphinxcode{\sphinxupquote{get\_interactions}}.

\end{fulllineitems}

\index{get\_interactions\_positive() (pypath.core.interaction.Interaction method)@\spxentry{get\_interactions\_positive()}\spxextra{pypath.core.interaction.Interaction method}}

\begin{fulllineitems}
\phantomsection\label{\detokenize{reference:pypath.core.interaction.Interaction.get_interactions_positive}}\pysiglinewithargsret{\sphinxbfcode{\sphinxupquote{get\_interactions\_positive}}}{\emph{**kwargs}}{}
{\color{red}\bfseries{}**}kwargs: see the docs of method \sphinxcode{\sphinxupquote{get\_interactions}}.

\end{fulllineitems}

\index{get\_interactions\_signed() (pypath.core.interaction.Interaction method)@\spxentry{get\_interactions\_signed()}\spxextra{pypath.core.interaction.Interaction method}}

\begin{fulllineitems}
\phantomsection\label{\detokenize{reference:pypath.core.interaction.Interaction.get_interactions_signed}}\pysiglinewithargsret{\sphinxbfcode{\sphinxupquote{get\_interactions\_signed}}}{\emph{**kwargs}}{}
{\color{red}\bfseries{}**}kwargs: see the docs of method \sphinxcode{\sphinxupquote{get\_interactions}}.

\end{fulllineitems}

\index{get\_interactions\_undirected() (pypath.core.interaction.Interaction method)@\spxentry{get\_interactions\_undirected()}\spxextra{pypath.core.interaction.Interaction method}}

\begin{fulllineitems}
\phantomsection\label{\detokenize{reference:pypath.core.interaction.Interaction.get_interactions_undirected}}\pysiglinewithargsret{\sphinxbfcode{\sphinxupquote{get\_interactions\_undirected}}}{\emph{**kwargs}}{}
Only the undirected interactions will be considered, if any resource
annotates this interaction as undirected both directions will be
returned, no matter if certain resources provide direction. However
the \sphinxcode{\sphinxupquote{count\_interactions\_undirected}} method will return \sphinxtitleref{1} in this
case.

{\color{red}\bfseries{}**}kwargs: see the docs of method \sphinxcode{\sphinxupquote{get\_interactions}}.

\end{fulllineitems}

\index{get\_interactions\_undirected\_0() (pypath.core.interaction.Interaction method)@\spxentry{get\_interactions\_undirected\_0()}\spxextra{pypath.core.interaction.Interaction method}}

\begin{fulllineitems}
\phantomsection\label{\detokenize{reference:pypath.core.interaction.Interaction.get_interactions_undirected_0}}\pysiglinewithargsret{\sphinxbfcode{\sphinxupquote{get\_interactions\_undirected\_0}}}{\emph{**kwargs}}{}
Only the undirected interactions will be considered, if any resource
annotates this interaction as undirected the interacting pair as
a sorted tuple will be returned inside a one element tuple.

{\color{red}\bfseries{}**}kwargs: see the docs of method \sphinxcode{\sphinxupquote{get\_interactions}}.

\end{fulllineitems}

\index{get\_labels() (pypath.core.interaction.Interaction method)@\spxentry{get\_labels()}\spxextra{pypath.core.interaction.Interaction method}}

\begin{fulllineitems}
\phantomsection\label{\detokenize{reference:pypath.core.interaction.Interaction.get_labels}}\pysiglinewithargsret{\sphinxbfcode{\sphinxupquote{get\_labels}}}{\emph{entity\_type=None}, \emph{direction=None}, \emph{effect=None}, \emph{resources=None}, \emph{data\_model=None}, \emph{interaction\_type=None}, \emph{via=None}, \emph{references=None}, \emph{return\_type=None}}{}
Retrieves the entities involved in interactions matching the criteria.
It either returns both interacting entities in a \sphinxstyleemphasis{set} or an empty
\sphinxstyleemphasis{set}. This may not sound so useful at the level of this object but
becomes more useful once we want to collect entities having certain
kind of interactions across a series of \sphinxtitleref{Interaction} objects.
\begin{quote}\begin{description}
\item[{Parameters}] \leavevmode\begin{itemize}
\item {} 
\sphinxstyleliteralstrong{\sphinxupquote{entity\_type}} (\sphinxstyleliteralemphasis{\sphinxupquote{str}}) \textendash{} The type of the molecular entity. Possible values: \sphinxtitleref{protein},
\sphinxtitleref{complex}, \sphinxtitleref{mirna}, \sphinxtitleref{small\_molecule}.

\item {} 
\sphinxstyleliteralstrong{\sphinxupquote{return\_type}} (\sphinxstyleliteralemphasis{\sphinxupquote{str}}) \textendash{} The type of values to return. Default is
py:class:\sphinxcode{\sphinxupquote{pypath.entity.Entity}} objects, alternatives are
\sphinxcode{\sphinxupquote{labels}}  \sphinxcode{\sphinxupquote{identifiers}}.

\end{itemize}

\end{description}\end{quote}

\end{fulllineitems}

\index{get\_lncrna\_identifiers() (pypath.core.interaction.Interaction method)@\spxentry{get\_lncrna\_identifiers()}\spxextra{pypath.core.interaction.Interaction method}}

\begin{fulllineitems}
\phantomsection\label{\detokenize{reference:pypath.core.interaction.Interaction.get_lncrna_identifiers}}\pysiglinewithargsret{\sphinxbfcode{\sphinxupquote{get\_lncrna\_identifiers}}}{\emph{entity\_type=None}, \emph{direction=None}, \emph{effect=None}, \emph{resources=None}, \emph{data\_model=None}, \emph{interaction\_type=None}, \emph{via=None}, \emph{references=None}, \emph{return\_type=None}}{}
Retrieves the entities involved in interactions matching the criteria.
It either returns both interacting entities in a \sphinxstyleemphasis{set} or an empty
\sphinxstyleemphasis{set}. This may not sound so useful at the level of this object but
becomes more useful once we want to collect entities having certain
kind of interactions across a series of \sphinxtitleref{Interaction} objects.
\begin{quote}\begin{description}
\item[{Parameters}] \leavevmode\begin{itemize}
\item {} 
\sphinxstyleliteralstrong{\sphinxupquote{entity\_type}} (\sphinxstyleliteralemphasis{\sphinxupquote{str}}) \textendash{} The type of the molecular entity. Possible values: \sphinxtitleref{protein},
\sphinxtitleref{complex}, \sphinxtitleref{mirna}, \sphinxtitleref{small\_molecule}.

\item {} 
\sphinxstyleliteralstrong{\sphinxupquote{return\_type}} (\sphinxstyleliteralemphasis{\sphinxupquote{str}}) \textendash{} The type of values to return. Default is
py:class:\sphinxcode{\sphinxupquote{pypath.entity.Entity}} objects, alternatives are
\sphinxcode{\sphinxupquote{labels}}  \sphinxcode{\sphinxupquote{identifiers}}.

\end{itemize}

\end{description}\end{quote}

\end{fulllineitems}

\index{get\_lncrna\_labels() (pypath.core.interaction.Interaction method)@\spxentry{get\_lncrna\_labels()}\spxextra{pypath.core.interaction.Interaction method}}

\begin{fulllineitems}
\phantomsection\label{\detokenize{reference:pypath.core.interaction.Interaction.get_lncrna_labels}}\pysiglinewithargsret{\sphinxbfcode{\sphinxupquote{get\_lncrna\_labels}}}{\emph{entity\_type=None}, \emph{direction=None}, \emph{effect=None}, \emph{resources=None}, \emph{data\_model=None}, \emph{interaction\_type=None}, \emph{via=None}, \emph{references=None}, \emph{return\_type=None}}{}
Retrieves the entities involved in interactions matching the criteria.
It either returns both interacting entities in a \sphinxstyleemphasis{set} or an empty
\sphinxstyleemphasis{set}. This may not sound so useful at the level of this object but
becomes more useful once we want to collect entities having certain
kind of interactions across a series of \sphinxtitleref{Interaction} objects.
\begin{quote}\begin{description}
\item[{Parameters}] \leavevmode\begin{itemize}
\item {} 
\sphinxstyleliteralstrong{\sphinxupquote{entity\_type}} (\sphinxstyleliteralemphasis{\sphinxupquote{str}}) \textendash{} The type of the molecular entity. Possible values: \sphinxtitleref{protein},
\sphinxtitleref{complex}, \sphinxtitleref{mirna}, \sphinxtitleref{small\_molecule}.

\item {} 
\sphinxstyleliteralstrong{\sphinxupquote{return\_type}} (\sphinxstyleliteralemphasis{\sphinxupquote{str}}) \textendash{} The type of values to return. Default is
py:class:\sphinxcode{\sphinxupquote{pypath.entity.Entity}} objects, alternatives are
\sphinxcode{\sphinxupquote{labels}}  \sphinxcode{\sphinxupquote{identifiers}}.

\end{itemize}

\end{description}\end{quote}

\end{fulllineitems}

\index{get\_lncrnas() (pypath.core.interaction.Interaction method)@\spxentry{get\_lncrnas()}\spxextra{pypath.core.interaction.Interaction method}}

\begin{fulllineitems}
\phantomsection\label{\detokenize{reference:pypath.core.interaction.Interaction.get_lncrnas}}\pysiglinewithargsret{\sphinxbfcode{\sphinxupquote{get\_lncrnas}}}{\emph{entity\_type=None}, \emph{direction=None}, \emph{effect=None}, \emph{resources=None}, \emph{data\_model=None}, \emph{interaction\_type=None}, \emph{via=None}, \emph{references=None}, \emph{return\_type=None}}{}
Retrieves the entities involved in interactions matching the criteria.
It either returns both interacting entities in a \sphinxstyleemphasis{set} or an empty
\sphinxstyleemphasis{set}. This may not sound so useful at the level of this object but
becomes more useful once we want to collect entities having certain
kind of interactions across a series of \sphinxtitleref{Interaction} objects.
\begin{quote}\begin{description}
\item[{Parameters}] \leavevmode\begin{itemize}
\item {} 
\sphinxstyleliteralstrong{\sphinxupquote{entity\_type}} (\sphinxstyleliteralemphasis{\sphinxupquote{str}}) \textendash{} The type of the molecular entity. Possible values: \sphinxtitleref{protein},
\sphinxtitleref{complex}, \sphinxtitleref{mirna}, \sphinxtitleref{small\_molecule}.

\item {} 
\sphinxstyleliteralstrong{\sphinxupquote{return\_type}} (\sphinxstyleliteralemphasis{\sphinxupquote{str}}) \textendash{} The type of values to return. Default is
py:class:\sphinxcode{\sphinxupquote{pypath.entity.Entity}} objects, alternatives are
\sphinxcode{\sphinxupquote{labels}}  \sphinxcode{\sphinxupquote{identifiers}}.

\end{itemize}

\end{description}\end{quote}

\end{fulllineitems}

\index{get\_mirna\_identifiers() (pypath.core.interaction.Interaction method)@\spxentry{get\_mirna\_identifiers()}\spxextra{pypath.core.interaction.Interaction method}}

\begin{fulllineitems}
\phantomsection\label{\detokenize{reference:pypath.core.interaction.Interaction.get_mirna_identifiers}}\pysiglinewithargsret{\sphinxbfcode{\sphinxupquote{get\_mirna\_identifiers}}}{\emph{entity\_type=None}, \emph{direction=None}, \emph{effect=None}, \emph{resources=None}, \emph{data\_model=None}, \emph{interaction\_type=None}, \emph{via=None}, \emph{references=None}, \emph{return\_type=None}}{}
Retrieves the entities involved in interactions matching the criteria.
It either returns both interacting entities in a \sphinxstyleemphasis{set} or an empty
\sphinxstyleemphasis{set}. This may not sound so useful at the level of this object but
becomes more useful once we want to collect entities having certain
kind of interactions across a series of \sphinxtitleref{Interaction} objects.
\begin{quote}\begin{description}
\item[{Parameters}] \leavevmode\begin{itemize}
\item {} 
\sphinxstyleliteralstrong{\sphinxupquote{entity\_type}} (\sphinxstyleliteralemphasis{\sphinxupquote{str}}) \textendash{} The type of the molecular entity. Possible values: \sphinxtitleref{protein},
\sphinxtitleref{complex}, \sphinxtitleref{mirna}, \sphinxtitleref{small\_molecule}.

\item {} 
\sphinxstyleliteralstrong{\sphinxupquote{return\_type}} (\sphinxstyleliteralemphasis{\sphinxupquote{str}}) \textendash{} The type of values to return. Default is
py:class:\sphinxcode{\sphinxupquote{pypath.entity.Entity}} objects, alternatives are
\sphinxcode{\sphinxupquote{labels}}  \sphinxcode{\sphinxupquote{identifiers}}.

\end{itemize}

\end{description}\end{quote}

\end{fulllineitems}

\index{get\_mirna\_labels() (pypath.core.interaction.Interaction method)@\spxentry{get\_mirna\_labels()}\spxextra{pypath.core.interaction.Interaction method}}

\begin{fulllineitems}
\phantomsection\label{\detokenize{reference:pypath.core.interaction.Interaction.get_mirna_labels}}\pysiglinewithargsret{\sphinxbfcode{\sphinxupquote{get\_mirna\_labels}}}{\emph{entity\_type=None}, \emph{direction=None}, \emph{effect=None}, \emph{resources=None}, \emph{data\_model=None}, \emph{interaction\_type=None}, \emph{via=None}, \emph{references=None}, \emph{return\_type=None}}{}
Retrieves the entities involved in interactions matching the criteria.
It either returns both interacting entities in a \sphinxstyleemphasis{set} or an empty
\sphinxstyleemphasis{set}. This may not sound so useful at the level of this object but
becomes more useful once we want to collect entities having certain
kind of interactions across a series of \sphinxtitleref{Interaction} objects.
\begin{quote}\begin{description}
\item[{Parameters}] \leavevmode\begin{itemize}
\item {} 
\sphinxstyleliteralstrong{\sphinxupquote{entity\_type}} (\sphinxstyleliteralemphasis{\sphinxupquote{str}}) \textendash{} The type of the molecular entity. Possible values: \sphinxtitleref{protein},
\sphinxtitleref{complex}, \sphinxtitleref{mirna}, \sphinxtitleref{small\_molecule}.

\item {} 
\sphinxstyleliteralstrong{\sphinxupquote{return\_type}} (\sphinxstyleliteralemphasis{\sphinxupquote{str}}) \textendash{} The type of values to return. Default is
py:class:\sphinxcode{\sphinxupquote{pypath.entity.Entity}} objects, alternatives are
\sphinxcode{\sphinxupquote{labels}}  \sphinxcode{\sphinxupquote{identifiers}}.

\end{itemize}

\end{description}\end{quote}

\end{fulllineitems}

\index{get\_mirnas() (pypath.core.interaction.Interaction method)@\spxentry{get\_mirnas()}\spxextra{pypath.core.interaction.Interaction method}}

\begin{fulllineitems}
\phantomsection\label{\detokenize{reference:pypath.core.interaction.Interaction.get_mirnas}}\pysiglinewithargsret{\sphinxbfcode{\sphinxupquote{get\_mirnas}}}{\emph{entity\_type=None}, \emph{direction=None}, \emph{effect=None}, \emph{resources=None}, \emph{data\_model=None}, \emph{interaction\_type=None}, \emph{via=None}, \emph{references=None}, \emph{return\_type=None}}{}
Retrieves the entities involved in interactions matching the criteria.
It either returns both interacting entities in a \sphinxstyleemphasis{set} or an empty
\sphinxstyleemphasis{set}. This may not sound so useful at the level of this object but
becomes more useful once we want to collect entities having certain
kind of interactions across a series of \sphinxtitleref{Interaction} objects.
\begin{quote}\begin{description}
\item[{Parameters}] \leavevmode\begin{itemize}
\item {} 
\sphinxstyleliteralstrong{\sphinxupquote{entity\_type}} (\sphinxstyleliteralemphasis{\sphinxupquote{str}}) \textendash{} The type of the molecular entity. Possible values: \sphinxtitleref{protein},
\sphinxtitleref{complex}, \sphinxtitleref{mirna}, \sphinxtitleref{small\_molecule}.

\item {} 
\sphinxstyleliteralstrong{\sphinxupquote{return\_type}} (\sphinxstyleliteralemphasis{\sphinxupquote{str}}) \textendash{} The type of values to return. Default is
py:class:\sphinxcode{\sphinxupquote{pypath.entity.Entity}} objects, alternatives are
\sphinxcode{\sphinxupquote{labels}}  \sphinxcode{\sphinxupquote{identifiers}}.

\end{itemize}

\end{description}\end{quote}

\end{fulllineitems}

\index{get\_protein\_identifiers() (pypath.core.interaction.Interaction method)@\spxentry{get\_protein\_identifiers()}\spxextra{pypath.core.interaction.Interaction method}}

\begin{fulllineitems}
\phantomsection\label{\detokenize{reference:pypath.core.interaction.Interaction.get_protein_identifiers}}\pysiglinewithargsret{\sphinxbfcode{\sphinxupquote{get\_protein\_identifiers}}}{\emph{entity\_type=None}, \emph{direction=None}, \emph{effect=None}, \emph{resources=None}, \emph{data\_model=None}, \emph{interaction\_type=None}, \emph{via=None}, \emph{references=None}, \emph{return\_type=None}}{}
Retrieves the entities involved in interactions matching the criteria.
It either returns both interacting entities in a \sphinxstyleemphasis{set} or an empty
\sphinxstyleemphasis{set}. This may not sound so useful at the level of this object but
becomes more useful once we want to collect entities having certain
kind of interactions across a series of \sphinxtitleref{Interaction} objects.
\begin{quote}\begin{description}
\item[{Parameters}] \leavevmode\begin{itemize}
\item {} 
\sphinxstyleliteralstrong{\sphinxupquote{entity\_type}} (\sphinxstyleliteralemphasis{\sphinxupquote{str}}) \textendash{} The type of the molecular entity. Possible values: \sphinxtitleref{protein},
\sphinxtitleref{complex}, \sphinxtitleref{mirna}, \sphinxtitleref{small\_molecule}.

\item {} 
\sphinxstyleliteralstrong{\sphinxupquote{return\_type}} (\sphinxstyleliteralemphasis{\sphinxupquote{str}}) \textendash{} The type of values to return. Default is
py:class:\sphinxcode{\sphinxupquote{pypath.entity.Entity}} objects, alternatives are
\sphinxcode{\sphinxupquote{labels}}  \sphinxcode{\sphinxupquote{identifiers}}.

\end{itemize}

\end{description}\end{quote}

\end{fulllineitems}

\index{get\_protein\_labels() (pypath.core.interaction.Interaction method)@\spxentry{get\_protein\_labels()}\spxextra{pypath.core.interaction.Interaction method}}

\begin{fulllineitems}
\phantomsection\label{\detokenize{reference:pypath.core.interaction.Interaction.get_protein_labels}}\pysiglinewithargsret{\sphinxbfcode{\sphinxupquote{get\_protein\_labels}}}{\emph{entity\_type=None}, \emph{direction=None}, \emph{effect=None}, \emph{resources=None}, \emph{data\_model=None}, \emph{interaction\_type=None}, \emph{via=None}, \emph{references=None}, \emph{return\_type=None}}{}
Retrieves the entities involved in interactions matching the criteria.
It either returns both interacting entities in a \sphinxstyleemphasis{set} or an empty
\sphinxstyleemphasis{set}. This may not sound so useful at the level of this object but
becomes more useful once we want to collect entities having certain
kind of interactions across a series of \sphinxtitleref{Interaction} objects.
\begin{quote}\begin{description}
\item[{Parameters}] \leavevmode\begin{itemize}
\item {} 
\sphinxstyleliteralstrong{\sphinxupquote{entity\_type}} (\sphinxstyleliteralemphasis{\sphinxupquote{str}}) \textendash{} The type of the molecular entity. Possible values: \sphinxtitleref{protein},
\sphinxtitleref{complex}, \sphinxtitleref{mirna}, \sphinxtitleref{small\_molecule}.

\item {} 
\sphinxstyleliteralstrong{\sphinxupquote{return\_type}} (\sphinxstyleliteralemphasis{\sphinxupquote{str}}) \textendash{} The type of values to return. Default is
py:class:\sphinxcode{\sphinxupquote{pypath.entity.Entity}} objects, alternatives are
\sphinxcode{\sphinxupquote{labels}}  \sphinxcode{\sphinxupquote{identifiers}}.

\end{itemize}

\end{description}\end{quote}

\end{fulllineitems}

\index{get\_proteins() (pypath.core.interaction.Interaction method)@\spxentry{get\_proteins()}\spxextra{pypath.core.interaction.Interaction method}}

\begin{fulllineitems}
\phantomsection\label{\detokenize{reference:pypath.core.interaction.Interaction.get_proteins}}\pysiglinewithargsret{\sphinxbfcode{\sphinxupquote{get\_proteins}}}{\emph{entity\_type=None}, \emph{direction=None}, \emph{effect=None}, \emph{resources=None}, \emph{data\_model=None}, \emph{interaction\_type=None}, \emph{via=None}, \emph{references=None}, \emph{return\_type=None}}{}
Retrieves the entities involved in interactions matching the criteria.
It either returns both interacting entities in a \sphinxstyleemphasis{set} or an empty
\sphinxstyleemphasis{set}. This may not sound so useful at the level of this object but
becomes more useful once we want to collect entities having certain
kind of interactions across a series of \sphinxtitleref{Interaction} objects.
\begin{quote}\begin{description}
\item[{Parameters}] \leavevmode\begin{itemize}
\item {} 
\sphinxstyleliteralstrong{\sphinxupquote{entity\_type}} (\sphinxstyleliteralemphasis{\sphinxupquote{str}}) \textendash{} The type of the molecular entity. Possible values: \sphinxtitleref{protein},
\sphinxtitleref{complex}, \sphinxtitleref{mirna}, \sphinxtitleref{small\_molecule}.

\item {} 
\sphinxstyleliteralstrong{\sphinxupquote{return\_type}} (\sphinxstyleliteralemphasis{\sphinxupquote{str}}) \textendash{} The type of values to return. Default is
py:class:\sphinxcode{\sphinxupquote{pypath.entity.Entity}} objects, alternatives are
\sphinxcode{\sphinxupquote{labels}}  \sphinxcode{\sphinxupquote{identifiers}}.

\end{itemize}

\end{description}\end{quote}

\end{fulllineitems}

\index{get\_references() (pypath.core.interaction.Interaction method)@\spxentry{get\_references()}\spxextra{pypath.core.interaction.Interaction method}}

\begin{fulllineitems}
\phantomsection\label{\detokenize{reference:pypath.core.interaction.Interaction.get_references}}\pysiglinewithargsret{\sphinxbfcode{\sphinxupquote{get\_references}}}{\emph{effect=None}, \emph{resources=None}, \emph{data\_model=None}, \emph{interaction\_type=None}, \emph{via=None}, \emph{references=None}}{}
Retrieves references matching the criteria.

\end{fulllineitems}

\index{get\_resource\_names() (pypath.core.interaction.Interaction method)@\spxentry{get\_resource\_names()}\spxextra{pypath.core.interaction.Interaction method}}

\begin{fulllineitems}
\phantomsection\label{\detokenize{reference:pypath.core.interaction.Interaction.get_resource_names}}\pysiglinewithargsret{\sphinxbfcode{\sphinxupquote{get\_resource\_names}}}{\emph{effect=None}, \emph{resources=None}, \emph{data\_model=None}, \emph{interaction\_type=None}, \emph{via=None}, \emph{references=None}}{}
Retrieves resource names matching the criteria.

\end{fulllineitems}

\index{get\_resource\_names\_via() (pypath.core.interaction.Interaction method)@\spxentry{get\_resource\_names\_via()}\spxextra{pypath.core.interaction.Interaction method}}

\begin{fulllineitems}
\phantomsection\label{\detokenize{reference:pypath.core.interaction.Interaction.get_resource_names_via}}\pysiglinewithargsret{\sphinxbfcode{\sphinxupquote{get\_resource\_names\_via}}}{\emph{effect=None}, \emph{resources=None}, \emph{data\_model=None}, \emph{interaction\_type=None}, \emph{via=None}, \emph{references=None}}{}
Retrieves resource names via matching the criteria.

\end{fulllineitems}

\index{get\_resources() (pypath.core.interaction.Interaction method)@\spxentry{get\_resources()}\spxextra{pypath.core.interaction.Interaction method}}

\begin{fulllineitems}
\phantomsection\label{\detokenize{reference:pypath.core.interaction.Interaction.get_resources}}\pysiglinewithargsret{\sphinxbfcode{\sphinxupquote{get\_resources}}}{\emph{effect=None}, \emph{resources=None}, \emph{data\_model=None}, \emph{interaction\_type=None}, \emph{via=None}, \emph{references=None}}{}
Retrieves resources matching the criteria.

\end{fulllineitems}

\index{get\_resources\_via() (pypath.core.interaction.Interaction method)@\spxentry{get\_resources\_via()}\spxextra{pypath.core.interaction.Interaction method}}

\begin{fulllineitems}
\phantomsection\label{\detokenize{reference:pypath.core.interaction.Interaction.get_resources_via}}\pysiglinewithargsret{\sphinxbfcode{\sphinxupquote{get\_resources\_via}}}{\emph{effect=None}, \emph{resources=None}, \emph{data\_model=None}, \emph{interaction\_type=None}, \emph{via=None}, \emph{references=None}}{}
Retrieves resources via matching the criteria.

\end{fulllineitems}

\index{get\_sign() (pypath.core.interaction.Interaction method)@\spxentry{get\_sign()}\spxextra{pypath.core.interaction.Interaction method}}

\begin{fulllineitems}
\phantomsection\label{\detokenize{reference:pypath.core.interaction.Interaction.get_sign}}\pysiglinewithargsret{\sphinxbfcode{\sphinxupquote{get\_sign}}}{\emph{direction}, \emph{sign=None}, \emph{evidences=False}, \emph{resources=False}, \emph{resource\_names=False}, \emph{sources=False}}{}
Retrieves the sign information of the edge in the given
diretion. If specified in \sphinxstyleemphasis{sign}, only that sign’s information
will be retrieved. If specified in \sphinxstyleemphasis{sources}, the sources of
that information will be retrieved instead.
\begin{quote}\begin{description}
\item[{Parameters}] \leavevmode\begin{itemize}
\item {} 
\sphinxstyleliteralstrong{\sphinxupquote{direction}} (\sphinxstyleliteralemphasis{\sphinxupquote{tuple}}) \textendash{} Contains the pair of nodes specifying the directionality of
the edge from which th information is to be retrieved.

\item {} 
\sphinxstyleliteralstrong{\sphinxupquote{sign}} (\sphinxstyleliteralemphasis{\sphinxupquote{str}}) \textendash{} Optional, \sphinxcode{\sphinxupquote{None}} by default. Denotes whether to retrieve
the \sphinxcode{\sphinxupquote{'positive'}} or \sphinxcode{\sphinxupquote{'negative'}} specific information.

\item {} 
\sphinxstyleliteralstrong{\sphinxupquote{resources}} (\sphinxstyleliteralemphasis{\sphinxupquote{bool}}) \textendash{} Optional, \sphinxcode{\sphinxupquote{False}} by default. Specifies whether to return
the resources instead of sign.

\end{itemize}

\item[{Returns}] \leavevmode
(\sphinxstyleemphasis{list}) \textendash{} If \sphinxcode{\sphinxupquote{sign=None}} containing {[}bool{]} values
denoting the presence of positive and negative sign on that
direction, if \sphinxcode{\sphinxupquote{sources=True}} the {[}set{]} of sources for each
of them will be returned instead. If \sphinxstyleemphasis{sign} is specified,
returns {[}bool{]} or {[}set{]} (if \sphinxcode{\sphinxupquote{sources=True}}) of that
specific direction and sign.

\end{description}\end{quote}

\end{fulllineitems}

\index{get\_small\_molecule\_identifiers() (pypath.core.interaction.Interaction method)@\spxentry{get\_small\_molecule\_identifiers()}\spxextra{pypath.core.interaction.Interaction method}}

\begin{fulllineitems}
\phantomsection\label{\detokenize{reference:pypath.core.interaction.Interaction.get_small_molecule_identifiers}}\pysiglinewithargsret{\sphinxbfcode{\sphinxupquote{get\_small\_molecule\_identifiers}}}{\emph{entity\_type=None}, \emph{direction=None}, \emph{effect=None}, \emph{resources=None}, \emph{data\_model=None}, \emph{interaction\_type=None}, \emph{via=None}, \emph{references=None}, \emph{return\_type=None}}{}
Retrieves the entities involved in interactions matching the criteria.
It either returns both interacting entities in a \sphinxstyleemphasis{set} or an empty
\sphinxstyleemphasis{set}. This may not sound so useful at the level of this object but
becomes more useful once we want to collect entities having certain
kind of interactions across a series of \sphinxtitleref{Interaction} objects.
\begin{quote}\begin{description}
\item[{Parameters}] \leavevmode\begin{itemize}
\item {} 
\sphinxstyleliteralstrong{\sphinxupquote{entity\_type}} (\sphinxstyleliteralemphasis{\sphinxupquote{str}}) \textendash{} The type of the molecular entity. Possible values: \sphinxtitleref{protein},
\sphinxtitleref{complex}, \sphinxtitleref{mirna}, \sphinxtitleref{small\_molecule}.

\item {} 
\sphinxstyleliteralstrong{\sphinxupquote{return\_type}} (\sphinxstyleliteralemphasis{\sphinxupquote{str}}) \textendash{} The type of values to return. Default is
py:class:\sphinxcode{\sphinxupquote{pypath.entity.Entity}} objects, alternatives are
\sphinxcode{\sphinxupquote{labels}}  \sphinxcode{\sphinxupquote{identifiers}}.

\end{itemize}

\end{description}\end{quote}

\end{fulllineitems}

\index{get\_small\_molecule\_labels() (pypath.core.interaction.Interaction method)@\spxentry{get\_small\_molecule\_labels()}\spxextra{pypath.core.interaction.Interaction method}}

\begin{fulllineitems}
\phantomsection\label{\detokenize{reference:pypath.core.interaction.Interaction.get_small_molecule_labels}}\pysiglinewithargsret{\sphinxbfcode{\sphinxupquote{get\_small\_molecule\_labels}}}{\emph{entity\_type=None}, \emph{direction=None}, \emph{effect=None}, \emph{resources=None}, \emph{data\_model=None}, \emph{interaction\_type=None}, \emph{via=None}, \emph{references=None}, \emph{return\_type=None}}{}
Retrieves the entities involved in interactions matching the criteria.
It either returns both interacting entities in a \sphinxstyleemphasis{set} or an empty
\sphinxstyleemphasis{set}. This may not sound so useful at the level of this object but
becomes more useful once we want to collect entities having certain
kind of interactions across a series of \sphinxtitleref{Interaction} objects.
\begin{quote}\begin{description}
\item[{Parameters}] \leavevmode\begin{itemize}
\item {} 
\sphinxstyleliteralstrong{\sphinxupquote{entity\_type}} (\sphinxstyleliteralemphasis{\sphinxupquote{str}}) \textendash{} The type of the molecular entity. Possible values: \sphinxtitleref{protein},
\sphinxtitleref{complex}, \sphinxtitleref{mirna}, \sphinxtitleref{small\_molecule}.

\item {} 
\sphinxstyleliteralstrong{\sphinxupquote{return\_type}} (\sphinxstyleliteralemphasis{\sphinxupquote{str}}) \textendash{} The type of values to return. Default is
py:class:\sphinxcode{\sphinxupquote{pypath.entity.Entity}} objects, alternatives are
\sphinxcode{\sphinxupquote{labels}}  \sphinxcode{\sphinxupquote{identifiers}}.

\end{itemize}

\end{description}\end{quote}

\end{fulllineitems}

\index{get\_small\_molecules() (pypath.core.interaction.Interaction method)@\spxentry{get\_small\_molecules()}\spxextra{pypath.core.interaction.Interaction method}}

\begin{fulllineitems}
\phantomsection\label{\detokenize{reference:pypath.core.interaction.Interaction.get_small_molecules}}\pysiglinewithargsret{\sphinxbfcode{\sphinxupquote{get\_small\_molecules}}}{\emph{entity\_type=None}, \emph{direction=None}, \emph{effect=None}, \emph{resources=None}, \emph{data\_model=None}, \emph{interaction\_type=None}, \emph{via=None}, \emph{references=None}, \emph{return\_type=None}}{}
Retrieves the entities involved in interactions matching the criteria.
It either returns both interacting entities in a \sphinxstyleemphasis{set} or an empty
\sphinxstyleemphasis{set}. This may not sound so useful at the level of this object but
becomes more useful once we want to collect entities having certain
kind of interactions across a series of \sphinxtitleref{Interaction} objects.
\begin{quote}\begin{description}
\item[{Parameters}] \leavevmode\begin{itemize}
\item {} 
\sphinxstyleliteralstrong{\sphinxupquote{entity\_type}} (\sphinxstyleliteralemphasis{\sphinxupquote{str}}) \textendash{} The type of the molecular entity. Possible values: \sphinxtitleref{protein},
\sphinxtitleref{complex}, \sphinxtitleref{mirna}, \sphinxtitleref{small\_molecule}.

\item {} 
\sphinxstyleliteralstrong{\sphinxupquote{return\_type}} (\sphinxstyleliteralemphasis{\sphinxupquote{str}}) \textendash{} The type of values to return. Default is
py:class:\sphinxcode{\sphinxupquote{pypath.entity.Entity}} objects, alternatives are
\sphinxcode{\sphinxupquote{labels}}  \sphinxcode{\sphinxupquote{identifiers}}.

\end{itemize}

\end{description}\end{quote}

\end{fulllineitems}

\index{has\_sign() (pypath.core.interaction.Interaction method)@\spxentry{has\_sign()}\spxextra{pypath.core.interaction.Interaction method}}

\begin{fulllineitems}
\phantomsection\label{\detokenize{reference:pypath.core.interaction.Interaction.has_sign}}\pysiglinewithargsret{\sphinxbfcode{\sphinxupquote{has\_sign}}}{\emph{direction=None}, \emph{resources=None}}{}
Checks whether the edge (or for a specific \sphinxstyleemphasis{direction}) has
any signed information (about positive/negative interactions).
\begin{quote}\begin{description}
\item[{Parameters}] \leavevmode
\sphinxstyleliteralstrong{\sphinxupquote{direction}} (\sphinxstyleliteralemphasis{\sphinxupquote{tuple}}) \textendash{} Optional, \sphinxcode{\sphinxupquote{None}} by default. If specified, only the
information of that direction is checked for sign.

\item[{Returns}] \leavevmode
\begin{description}
\item[{(\sphinxstyleemphasis{bool}) \textendash{} \sphinxcode{\sphinxupquote{True}} if there exist any information on the}] \leavevmode
sign of the interaction, \sphinxcode{\sphinxupquote{False}} otherwise.

\end{description}


\end{description}\end{quote}

\end{fulllineitems}

\index{identifiers\_by\_data\_model() (pypath.core.interaction.Interaction method)@\spxentry{identifiers\_by\_data\_model()}\spxextra{pypath.core.interaction.Interaction method}}

\begin{fulllineitems}
\phantomsection\label{\detokenize{reference:pypath.core.interaction.Interaction.identifiers_by_data_model}}\pysiglinewithargsret{\sphinxbfcode{\sphinxupquote{identifiers\_by\_data\_model}}}{\emph{effect=None}, \emph{resources=None}, \emph{data\_model=None}, \emph{interaction\_type=None}, \emph{via=None}, \emph{references=None}}{}
Retrieves the entities involved in interactions matching the criteria.
It either returns both interacting entities in a \sphinxstyleemphasis{set} or an empty
\sphinxstyleemphasis{set}. This may not sound so useful at the level of this object but
becomes more useful once we want to collect entities having certain
kind of interactions across a series of \sphinxtitleref{Interaction} objects.
\begin{quote}\begin{description}
\item[{Parameters}] \leavevmode\begin{itemize}
\item {} 
\sphinxstyleliteralstrong{\sphinxupquote{entity\_type}} (\sphinxstyleliteralemphasis{\sphinxupquote{str}}) \textendash{} The type of the molecular entity. Possible values: \sphinxtitleref{protein},
\sphinxtitleref{complex}, \sphinxtitleref{mirna}, \sphinxtitleref{small\_molecule}.

\item {} 
\sphinxstyleliteralstrong{\sphinxupquote{return\_type}} (\sphinxstyleliteralemphasis{\sphinxupquote{str}}) \textendash{} The type of values to return. Default is
py:class:\sphinxcode{\sphinxupquote{pypath.entity.Entity}} objects, alternatives are
\sphinxcode{\sphinxupquote{labels}}  \sphinxcode{\sphinxupquote{identifiers}}.

\end{itemize}

\end{description}\end{quote}

\end{fulllineitems}

\index{identifiers\_by\_interaction\_type() (pypath.core.interaction.Interaction method)@\spxentry{identifiers\_by\_interaction\_type()}\spxextra{pypath.core.interaction.Interaction method}}

\begin{fulllineitems}
\phantomsection\label{\detokenize{reference:pypath.core.interaction.Interaction.identifiers_by_interaction_type}}\pysiglinewithargsret{\sphinxbfcode{\sphinxupquote{identifiers\_by\_interaction\_type}}}{\emph{effect=None}, \emph{resources=None}, \emph{data\_model=None}, \emph{interaction\_type=None}, \emph{via=None}, \emph{references=None}}{}
Retrieves the entities involved in interactions matching the criteria.
It either returns both interacting entities in a \sphinxstyleemphasis{set} or an empty
\sphinxstyleemphasis{set}. This may not sound so useful at the level of this object but
becomes more useful once we want to collect entities having certain
kind of interactions across a series of \sphinxtitleref{Interaction} objects.
\begin{quote}\begin{description}
\item[{Parameters}] \leavevmode\begin{itemize}
\item {} 
\sphinxstyleliteralstrong{\sphinxupquote{entity\_type}} (\sphinxstyleliteralemphasis{\sphinxupquote{str}}) \textendash{} The type of the molecular entity. Possible values: \sphinxtitleref{protein},
\sphinxtitleref{complex}, \sphinxtitleref{mirna}, \sphinxtitleref{small\_molecule}.

\item {} 
\sphinxstyleliteralstrong{\sphinxupquote{return\_type}} (\sphinxstyleliteralemphasis{\sphinxupquote{str}}) \textendash{} The type of values to return. Default is
py:class:\sphinxcode{\sphinxupquote{pypath.entity.Entity}} objects, alternatives are
\sphinxcode{\sphinxupquote{labels}}  \sphinxcode{\sphinxupquote{identifiers}}.

\end{itemize}

\end{description}\end{quote}

\end{fulllineitems}

\index{identifiers\_by\_interaction\_type\_and\_data\_model() (pypath.core.interaction.Interaction method)@\spxentry{identifiers\_by\_interaction\_type\_and\_data\_model()}\spxextra{pypath.core.interaction.Interaction method}}

\begin{fulllineitems}
\phantomsection\label{\detokenize{reference:pypath.core.interaction.Interaction.identifiers_by_interaction_type_and_data_model}}\pysiglinewithargsret{\sphinxbfcode{\sphinxupquote{identifiers\_by\_interaction\_type\_and\_data\_model}}}{\emph{effect=None}, \emph{resources=None}, \emph{data\_model=None}, \emph{interaction\_type=None}, \emph{via=None}, \emph{references=None}}{}
Retrieves the entities involved in interactions matching the criteria.
It either returns both interacting entities in a \sphinxstyleemphasis{set} or an empty
\sphinxstyleemphasis{set}. This may not sound so useful at the level of this object but
becomes more useful once we want to collect entities having certain
kind of interactions across a series of \sphinxtitleref{Interaction} objects.
\begin{quote}\begin{description}
\item[{Parameters}] \leavevmode\begin{itemize}
\item {} 
\sphinxstyleliteralstrong{\sphinxupquote{entity\_type}} (\sphinxstyleliteralemphasis{\sphinxupquote{str}}) \textendash{} The type of the molecular entity. Possible values: \sphinxtitleref{protein},
\sphinxtitleref{complex}, \sphinxtitleref{mirna}, \sphinxtitleref{small\_molecule}.

\item {} 
\sphinxstyleliteralstrong{\sphinxupquote{return\_type}} (\sphinxstyleliteralemphasis{\sphinxupquote{str}}) \textendash{} The type of values to return. Default is
py:class:\sphinxcode{\sphinxupquote{pypath.entity.Entity}} objects, alternatives are
\sphinxcode{\sphinxupquote{labels}}  \sphinxcode{\sphinxupquote{identifiers}}.

\end{itemize}

\end{description}\end{quote}

\end{fulllineitems}

\index{identifiers\_by\_interaction\_type\_and\_data\_model\_and\_resource() (pypath.core.interaction.Interaction method)@\spxentry{identifiers\_by\_interaction\_type\_and\_data\_model\_and\_resource()}\spxextra{pypath.core.interaction.Interaction method}}

\begin{fulllineitems}
\phantomsection\label{\detokenize{reference:pypath.core.interaction.Interaction.identifiers_by_interaction_type_and_data_model_and_resource}}\pysiglinewithargsret{\sphinxbfcode{\sphinxupquote{identifiers\_by\_interaction\_type\_and\_data\_model\_and\_resource}}}{\emph{effect=None}, \emph{resources=None}, \emph{data\_model=None}, \emph{interaction\_type=None}, \emph{via=None}, \emph{references=None}}{}
Retrieves the entities involved in interactions matching the criteria.
It either returns both interacting entities in a \sphinxstyleemphasis{set} or an empty
\sphinxstyleemphasis{set}. This may not sound so useful at the level of this object but
becomes more useful once we want to collect entities having certain
kind of interactions across a series of \sphinxtitleref{Interaction} objects.
\begin{quote}\begin{description}
\item[{Parameters}] \leavevmode\begin{itemize}
\item {} 
\sphinxstyleliteralstrong{\sphinxupquote{entity\_type}} (\sphinxstyleliteralemphasis{\sphinxupquote{str}}) \textendash{} The type of the molecular entity. Possible values: \sphinxtitleref{protein},
\sphinxtitleref{complex}, \sphinxtitleref{mirna}, \sphinxtitleref{small\_molecule}.

\item {} 
\sphinxstyleliteralstrong{\sphinxupquote{return\_type}} (\sphinxstyleliteralemphasis{\sphinxupquote{str}}) \textendash{} The type of values to return. Default is
py:class:\sphinxcode{\sphinxupquote{pypath.entity.Entity}} objects, alternatives are
\sphinxcode{\sphinxupquote{labels}}  \sphinxcode{\sphinxupquote{identifiers}}.

\end{itemize}

\end{description}\end{quote}

\end{fulllineitems}

\index{identifiers\_by\_reference() (pypath.core.interaction.Interaction method)@\spxentry{identifiers\_by\_reference()}\spxextra{pypath.core.interaction.Interaction method}}

\begin{fulllineitems}
\phantomsection\label{\detokenize{reference:pypath.core.interaction.Interaction.identifiers_by_reference}}\pysiglinewithargsret{\sphinxbfcode{\sphinxupquote{identifiers\_by\_reference}}}{\emph{effect=None}, \emph{resources=None}, \emph{data\_model=None}, \emph{interaction\_type=None}, \emph{via=None}, \emph{references=None}}{}
Retrieves the entities involved in interactions matching the criteria.
It either returns both interacting entities in a \sphinxstyleemphasis{set} or an empty
\sphinxstyleemphasis{set}. This may not sound so useful at the level of this object but
becomes more useful once we want to collect entities having certain
kind of interactions across a series of \sphinxtitleref{Interaction} objects.
\begin{quote}\begin{description}
\item[{Parameters}] \leavevmode\begin{itemize}
\item {} 
\sphinxstyleliteralstrong{\sphinxupquote{entity\_type}} (\sphinxstyleliteralemphasis{\sphinxupquote{str}}) \textendash{} The type of the molecular entity. Possible values: \sphinxtitleref{protein},
\sphinxtitleref{complex}, \sphinxtitleref{mirna}, \sphinxtitleref{small\_molecule}.

\item {} 
\sphinxstyleliteralstrong{\sphinxupquote{return\_type}} (\sphinxstyleliteralemphasis{\sphinxupquote{str}}) \textendash{} The type of values to return. Default is
py:class:\sphinxcode{\sphinxupquote{pypath.entity.Entity}} objects, alternatives are
\sphinxcode{\sphinxupquote{labels}}  \sphinxcode{\sphinxupquote{identifiers}}.

\end{itemize}

\end{description}\end{quote}

\end{fulllineitems}

\index{identifiers\_by\_resource() (pypath.core.interaction.Interaction method)@\spxentry{identifiers\_by\_resource()}\spxextra{pypath.core.interaction.Interaction method}}

\begin{fulllineitems}
\phantomsection\label{\detokenize{reference:pypath.core.interaction.Interaction.identifiers_by_resource}}\pysiglinewithargsret{\sphinxbfcode{\sphinxupquote{identifiers\_by\_resource}}}{\emph{effect=None}, \emph{resources=None}, \emph{data\_model=None}, \emph{interaction\_type=None}, \emph{via=None}, \emph{references=None}}{}
Retrieves the entities involved in interactions matching the criteria.
It either returns both interacting entities in a \sphinxstyleemphasis{set} or an empty
\sphinxstyleemphasis{set}. This may not sound so useful at the level of this object but
becomes more useful once we want to collect entities having certain
kind of interactions across a series of \sphinxtitleref{Interaction} objects.
\begin{quote}\begin{description}
\item[{Parameters}] \leavevmode\begin{itemize}
\item {} 
\sphinxstyleliteralstrong{\sphinxupquote{entity\_type}} (\sphinxstyleliteralemphasis{\sphinxupquote{str}}) \textendash{} The type of the molecular entity. Possible values: \sphinxtitleref{protein},
\sphinxtitleref{complex}, \sphinxtitleref{mirna}, \sphinxtitleref{small\_molecule}.

\item {} 
\sphinxstyleliteralstrong{\sphinxupquote{return\_type}} (\sphinxstyleliteralemphasis{\sphinxupquote{str}}) \textendash{} The type of values to return. Default is
py:class:\sphinxcode{\sphinxupquote{pypath.entity.Entity}} objects, alternatives are
\sphinxcode{\sphinxupquote{labels}}  \sphinxcode{\sphinxupquote{identifiers}}.

\end{itemize}

\end{description}\end{quote}

\end{fulllineitems}

\index{interaction\_types\_by\_data\_model() (pypath.core.interaction.Interaction method)@\spxentry{interaction\_types\_by\_data\_model()}\spxextra{pypath.core.interaction.Interaction method}}

\begin{fulllineitems}
\phantomsection\label{\detokenize{reference:pypath.core.interaction.Interaction.interaction_types_by_data_model}}\pysiglinewithargsret{\sphinxbfcode{\sphinxupquote{interaction\_types\_by\_data\_model}}}{\emph{effect=None}, \emph{resources=None}, \emph{data\_model=None}, \emph{interaction\_type=None}, \emph{via=None}, \emph{references=None}}{}
Retrieves interaction types matching the criteria.

\end{fulllineitems}

\index{interaction\_types\_by\_interaction\_type() (pypath.core.interaction.Interaction method)@\spxentry{interaction\_types\_by\_interaction\_type()}\spxextra{pypath.core.interaction.Interaction method}}

\begin{fulllineitems}
\phantomsection\label{\detokenize{reference:pypath.core.interaction.Interaction.interaction_types_by_interaction_type}}\pysiglinewithargsret{\sphinxbfcode{\sphinxupquote{interaction\_types\_by\_interaction\_type}}}{\emph{effect=None}, \emph{resources=None}, \emph{data\_model=None}, \emph{interaction\_type=None}, \emph{via=None}, \emph{references=None}}{}
Retrieves interaction types matching the criteria.

\end{fulllineitems}

\index{interaction\_types\_by\_interaction\_type\_and\_data\_model() (pypath.core.interaction.Interaction method)@\spxentry{interaction\_types\_by\_interaction\_type\_and\_data\_model()}\spxextra{pypath.core.interaction.Interaction method}}

\begin{fulllineitems}
\phantomsection\label{\detokenize{reference:pypath.core.interaction.Interaction.interaction_types_by_interaction_type_and_data_model}}\pysiglinewithargsret{\sphinxbfcode{\sphinxupquote{interaction\_types\_by\_interaction\_type\_and\_data\_model}}}{\emph{effect=None}, \emph{resources=None}, \emph{data\_model=None}, \emph{interaction\_type=None}, \emph{via=None}, \emph{references=None}}{}
Retrieves interaction types matching the criteria.

\end{fulllineitems}

\index{interaction\_types\_by\_interaction\_type\_and\_data\_model\_and\_resource() (pypath.core.interaction.Interaction method)@\spxentry{interaction\_types\_by\_interaction\_type\_and\_data\_model\_and\_resource()}\spxextra{pypath.core.interaction.Interaction method}}

\begin{fulllineitems}
\phantomsection\label{\detokenize{reference:pypath.core.interaction.Interaction.interaction_types_by_interaction_type_and_data_model_and_resource}}\pysiglinewithargsret{\sphinxbfcode{\sphinxupquote{interaction\_types\_by\_interaction\_type\_and\_data\_model\_and\_resource}}}{\emph{effect=None}, \emph{resources=None}, \emph{data\_model=None}, \emph{interaction\_type=None}, \emph{via=None}, \emph{references=None}}{}
Retrieves interaction types matching the criteria.

\end{fulllineitems}

\index{interaction\_types\_by\_reference() (pypath.core.interaction.Interaction method)@\spxentry{interaction\_types\_by\_reference()}\spxextra{pypath.core.interaction.Interaction method}}

\begin{fulllineitems}
\phantomsection\label{\detokenize{reference:pypath.core.interaction.Interaction.interaction_types_by_reference}}\pysiglinewithargsret{\sphinxbfcode{\sphinxupquote{interaction\_types\_by\_reference}}}{\emph{effect=None}, \emph{resources=None}, \emph{data\_model=None}, \emph{interaction\_type=None}, \emph{via=None}, \emph{references=None}}{}
Retrieves interaction types matching the criteria.

\end{fulllineitems}

\index{interaction\_types\_by\_resource() (pypath.core.interaction.Interaction method)@\spxentry{interaction\_types\_by\_resource()}\spxextra{pypath.core.interaction.Interaction method}}

\begin{fulllineitems}
\phantomsection\label{\detokenize{reference:pypath.core.interaction.Interaction.interaction_types_by_resource}}\pysiglinewithargsret{\sphinxbfcode{\sphinxupquote{interaction\_types\_by\_resource}}}{\emph{effect=None}, \emph{resources=None}, \emph{data\_model=None}, \emph{interaction\_type=None}, \emph{via=None}, \emph{references=None}}{}
Retrieves interaction types matching the criteria.

\end{fulllineitems}

\index{interactions\_0\_by\_data\_model() (pypath.core.interaction.Interaction method)@\spxentry{interactions\_0\_by\_data\_model()}\spxextra{pypath.core.interaction.Interaction method}}

\begin{fulllineitems}
\phantomsection\label{\detokenize{reference:pypath.core.interaction.Interaction.interactions_0_by_data_model}}\pysiglinewithargsret{\sphinxbfcode{\sphinxupquote{interactions\_0\_by\_data\_model}}}{\emph{effect=None}, \emph{resources=None}, \emph{data\_model=None}, \emph{interaction\_type=None}, \emph{via=None}, \emph{references=None}}{}
Returns unique interacting pairs without being aware of the direction.

\end{fulllineitems}

\index{interactions\_0\_by\_interaction\_type() (pypath.core.interaction.Interaction method)@\spxentry{interactions\_0\_by\_interaction\_type()}\spxextra{pypath.core.interaction.Interaction method}}

\begin{fulllineitems}
\phantomsection\label{\detokenize{reference:pypath.core.interaction.Interaction.interactions_0_by_interaction_type}}\pysiglinewithargsret{\sphinxbfcode{\sphinxupquote{interactions\_0\_by\_interaction\_type}}}{\emph{effect=None}, \emph{resources=None}, \emph{data\_model=None}, \emph{interaction\_type=None}, \emph{via=None}, \emph{references=None}}{}
Returns unique interacting pairs without being aware of the direction.

\end{fulllineitems}

\index{interactions\_0\_by\_interaction\_type\_and\_data\_model() (pypath.core.interaction.Interaction method)@\spxentry{interactions\_0\_by\_interaction\_type\_and\_data\_model()}\spxextra{pypath.core.interaction.Interaction method}}

\begin{fulllineitems}
\phantomsection\label{\detokenize{reference:pypath.core.interaction.Interaction.interactions_0_by_interaction_type_and_data_model}}\pysiglinewithargsret{\sphinxbfcode{\sphinxupquote{interactions\_0\_by\_interaction\_type\_and\_data\_model}}}{\emph{effect=None}, \emph{resources=None}, \emph{data\_model=None}, \emph{interaction\_type=None}, \emph{via=None}, \emph{references=None}}{}
Returns unique interacting pairs without being aware of the direction.

\end{fulllineitems}

\index{interactions\_0\_by\_interaction\_type\_and\_data\_model\_and\_resource() (pypath.core.interaction.Interaction method)@\spxentry{interactions\_0\_by\_interaction\_type\_and\_data\_model\_and\_resource()}\spxextra{pypath.core.interaction.Interaction method}}

\begin{fulllineitems}
\phantomsection\label{\detokenize{reference:pypath.core.interaction.Interaction.interactions_0_by_interaction_type_and_data_model_and_resource}}\pysiglinewithargsret{\sphinxbfcode{\sphinxupquote{interactions\_0\_by\_interaction\_type\_and\_data\_model\_and\_resource}}}{\emph{effect=None}, \emph{resources=None}, \emph{data\_model=None}, \emph{interaction\_type=None}, \emph{via=None}, \emph{references=None}}{}
Returns unique interacting pairs without being aware of the direction.

\end{fulllineitems}

\index{interactions\_0\_by\_reference() (pypath.core.interaction.Interaction method)@\spxentry{interactions\_0\_by\_reference()}\spxextra{pypath.core.interaction.Interaction method}}

\begin{fulllineitems}
\phantomsection\label{\detokenize{reference:pypath.core.interaction.Interaction.interactions_0_by_reference}}\pysiglinewithargsret{\sphinxbfcode{\sphinxupquote{interactions\_0\_by\_reference}}}{\emph{effect=None}, \emph{resources=None}, \emph{data\_model=None}, \emph{interaction\_type=None}, \emph{via=None}, \emph{references=None}}{}
Returns unique interacting pairs without being aware of the direction.

\end{fulllineitems}

\index{interactions\_0\_by\_resource() (pypath.core.interaction.Interaction method)@\spxentry{interactions\_0\_by\_resource()}\spxextra{pypath.core.interaction.Interaction method}}

\begin{fulllineitems}
\phantomsection\label{\detokenize{reference:pypath.core.interaction.Interaction.interactions_0_by_resource}}\pysiglinewithargsret{\sphinxbfcode{\sphinxupquote{interactions\_0\_by\_resource}}}{\emph{effect=None}, \emph{resources=None}, \emph{data\_model=None}, \emph{interaction\_type=None}, \emph{via=None}, \emph{references=None}}{}
Returns unique interacting pairs without being aware of the direction.

\end{fulllineitems}

\index{interactions\_by\_data\_model() (pypath.core.interaction.Interaction method)@\spxentry{interactions\_by\_data\_model()}\spxextra{pypath.core.interaction.Interaction method}}

\begin{fulllineitems}
\phantomsection\label{\detokenize{reference:pypath.core.interaction.Interaction.interactions_by_data_model}}\pysiglinewithargsret{\sphinxbfcode{\sphinxupquote{interactions\_by\_data\_model}}}{\emph{effect=None}, \emph{resources=None}, \emph{data\_model=None}, \emph{interaction\_type=None}, \emph{via=None}, \emph{references=None}}{}
Returns one or two tuples of the interacting partners: one if only
one direction, two if both directions match the query criteria.
The tuple will be empty if no evidence matches the criteria.
\begin{quote}\begin{description}
\item[{Parameters}] \leavevmode\begin{itemize}
\item {} 
\sphinxstyleliteralstrong{\sphinxupquote{direction}} (\sphinxstyleliteralemphasis{\sphinxupquote{NontType}}\sphinxstyleliteralemphasis{\sphinxupquote{,}}\sphinxstyleliteralemphasis{\sphinxupquote{bool}}\sphinxstyleliteralemphasis{\sphinxupquote{,}}\sphinxstyleliteralemphasis{\sphinxupquote{tuple}}) \textendash{} If \sphinxtitleref{None} both undirected and directed, if \sphinxtitleref{True} only directed,
if a \sphinxstyleemphasis{tuple} of entities only the interactions with that specific
direction will be considered. Unless you set this parameter to
\sphinxtitleref{True} this method will return both directions if one or more
undirected resources present.
If \sphinxtitleref{False}, only the undirected interactions will be considered,
and if any resource annotates this interaction as undirected
both directions will be returned. However the
\sphinxcode{\sphinxupquote{count\_interactions\_undirected}} method will return \sphinxtitleref{1}
in this case.

\item {} 
\sphinxstyleliteralstrong{\sphinxupquote{effect}} (\sphinxstyleliteralemphasis{\sphinxupquote{NoneType}}\sphinxstyleliteralemphasis{\sphinxupquote{,}}\sphinxstyleliteralemphasis{\sphinxupquote{bool}}\sphinxstyleliteralemphasis{\sphinxupquote{,}}\sphinxstyleliteralemphasis{\sphinxupquote{str}}) \textendash{} If \sphinxtitleref{None} also interactions without effect, if \sphinxtitleref{True} only
the ones with any effect, if a string naming an effect only the
interactions with that specific effect will be considered.

\item {} 
\sphinxstyleliteralstrong{\sphinxupquote{resources}} (\sphinxstyleliteralemphasis{\sphinxupquote{NontType}}\sphinxstyleliteralemphasis{\sphinxupquote{,}}\sphinxstyleliteralemphasis{\sphinxupquote{str}}\sphinxstyleliteralemphasis{\sphinxupquote{,}}\sphinxstyleliteralemphasis{\sphinxupquote{set}}) \textendash{} Optionally limit the query to one or more resources.

\item {} 
\sphinxstyleliteralstrong{\sphinxupquote{data\_model}} (\sphinxstyleliteralemphasis{\sphinxupquote{NontType}}\sphinxstyleliteralemphasis{\sphinxupquote{,}}\sphinxstyleliteralemphasis{\sphinxupquote{str}}\sphinxstyleliteralemphasis{\sphinxupquote{,}}\sphinxstyleliteralemphasis{\sphinxupquote{set}}) \textendash{} Optionally limit the query to one or more data models e.g.
\sphinxtitleref{activity\_flow}.

\item {} 
\sphinxstyleliteralstrong{\sphinxupquote{interaction\_type}} (\sphinxstyleliteralemphasis{\sphinxupquote{NontType}}\sphinxstyleliteralemphasis{\sphinxupquote{,}}\sphinxstyleliteralemphasis{\sphinxupquote{str}}\sphinxstyleliteralemphasis{\sphinxupquote{,}}\sphinxstyleliteralemphasis{\sphinxupquote{set}}) \textendash{} Optionally limit the query to one or more interaction types
e.g. \sphinxtitleref{PPI}.

\item {} 
\sphinxstyleliteralstrong{\sphinxupquote{via}} (\sphinxstyleliteralemphasis{\sphinxupquote{NontType}}\sphinxstyleliteralemphasis{\sphinxupquote{,}}\sphinxstyleliteralemphasis{\sphinxupquote{bool}}\sphinxstyleliteralemphasis{\sphinxupquote{,}}\sphinxstyleliteralemphasis{\sphinxupquote{str}}\sphinxstyleliteralemphasis{\sphinxupquote{,}}\sphinxstyleliteralemphasis{\sphinxupquote{set}}) \textendash{} Optionally limit the query to certain secondary databases or
if \sphinxtitleref{False} consider only data from primary databases.

\item {} 
\sphinxstyleliteralstrong{\sphinxupquote{entity\_type}} (\sphinxstyleliteralemphasis{\sphinxupquote{str}}) \textendash{} Molecule type for both of the entities.

\item {} 
\sphinxstyleliteralstrong{\sphinxupquote{source\_entity\_type}} (\sphinxstyleliteralemphasis{\sphinxupquote{str}}) \textendash{} Molecule type for the source entity.

\item {} 
\sphinxstyleliteralstrong{\sphinxupquote{target\_entity\_type}} (\sphinxstyleliteralemphasis{\sphinxupquote{str}}) \textendash{} Molecule type for the target entity.

\end{itemize}

\end{description}\end{quote}

\end{fulllineitems}

\index{interactions\_by\_interaction\_type() (pypath.core.interaction.Interaction method)@\spxentry{interactions\_by\_interaction\_type()}\spxextra{pypath.core.interaction.Interaction method}}

\begin{fulllineitems}
\phantomsection\label{\detokenize{reference:pypath.core.interaction.Interaction.interactions_by_interaction_type}}\pysiglinewithargsret{\sphinxbfcode{\sphinxupquote{interactions\_by\_interaction\_type}}}{\emph{effect=None}, \emph{resources=None}, \emph{data\_model=None}, \emph{interaction\_type=None}, \emph{via=None}, \emph{references=None}}{}
Returns one or two tuples of the interacting partners: one if only
one direction, two if both directions match the query criteria.
The tuple will be empty if no evidence matches the criteria.
\begin{quote}\begin{description}
\item[{Parameters}] \leavevmode\begin{itemize}
\item {} 
\sphinxstyleliteralstrong{\sphinxupquote{direction}} (\sphinxstyleliteralemphasis{\sphinxupquote{NontType}}\sphinxstyleliteralemphasis{\sphinxupquote{,}}\sphinxstyleliteralemphasis{\sphinxupquote{bool}}\sphinxstyleliteralemphasis{\sphinxupquote{,}}\sphinxstyleliteralemphasis{\sphinxupquote{tuple}}) \textendash{} If \sphinxtitleref{None} both undirected and directed, if \sphinxtitleref{True} only directed,
if a \sphinxstyleemphasis{tuple} of entities only the interactions with that specific
direction will be considered. Unless you set this parameter to
\sphinxtitleref{True} this method will return both directions if one or more
undirected resources present.
If \sphinxtitleref{False}, only the undirected interactions will be considered,
and if any resource annotates this interaction as undirected
both directions will be returned. However the
\sphinxcode{\sphinxupquote{count\_interactions\_undirected}} method will return \sphinxtitleref{1}
in this case.

\item {} 
\sphinxstyleliteralstrong{\sphinxupquote{effect}} (\sphinxstyleliteralemphasis{\sphinxupquote{NoneType}}\sphinxstyleliteralemphasis{\sphinxupquote{,}}\sphinxstyleliteralemphasis{\sphinxupquote{bool}}\sphinxstyleliteralemphasis{\sphinxupquote{,}}\sphinxstyleliteralemphasis{\sphinxupquote{str}}) \textendash{} If \sphinxtitleref{None} also interactions without effect, if \sphinxtitleref{True} only
the ones with any effect, if a string naming an effect only the
interactions with that specific effect will be considered.

\item {} 
\sphinxstyleliteralstrong{\sphinxupquote{resources}} (\sphinxstyleliteralemphasis{\sphinxupquote{NontType}}\sphinxstyleliteralemphasis{\sphinxupquote{,}}\sphinxstyleliteralemphasis{\sphinxupquote{str}}\sphinxstyleliteralemphasis{\sphinxupquote{,}}\sphinxstyleliteralemphasis{\sphinxupquote{set}}) \textendash{} Optionally limit the query to one or more resources.

\item {} 
\sphinxstyleliteralstrong{\sphinxupquote{data\_model}} (\sphinxstyleliteralemphasis{\sphinxupquote{NontType}}\sphinxstyleliteralemphasis{\sphinxupquote{,}}\sphinxstyleliteralemphasis{\sphinxupquote{str}}\sphinxstyleliteralemphasis{\sphinxupquote{,}}\sphinxstyleliteralemphasis{\sphinxupquote{set}}) \textendash{} Optionally limit the query to one or more data models e.g.
\sphinxtitleref{activity\_flow}.

\item {} 
\sphinxstyleliteralstrong{\sphinxupquote{interaction\_type}} (\sphinxstyleliteralemphasis{\sphinxupquote{NontType}}\sphinxstyleliteralemphasis{\sphinxupquote{,}}\sphinxstyleliteralemphasis{\sphinxupquote{str}}\sphinxstyleliteralemphasis{\sphinxupquote{,}}\sphinxstyleliteralemphasis{\sphinxupquote{set}}) \textendash{} Optionally limit the query to one or more interaction types
e.g. \sphinxtitleref{PPI}.

\item {} 
\sphinxstyleliteralstrong{\sphinxupquote{via}} (\sphinxstyleliteralemphasis{\sphinxupquote{NontType}}\sphinxstyleliteralemphasis{\sphinxupquote{,}}\sphinxstyleliteralemphasis{\sphinxupquote{bool}}\sphinxstyleliteralemphasis{\sphinxupquote{,}}\sphinxstyleliteralemphasis{\sphinxupquote{str}}\sphinxstyleliteralemphasis{\sphinxupquote{,}}\sphinxstyleliteralemphasis{\sphinxupquote{set}}) \textendash{} Optionally limit the query to certain secondary databases or
if \sphinxtitleref{False} consider only data from primary databases.

\item {} 
\sphinxstyleliteralstrong{\sphinxupquote{entity\_type}} (\sphinxstyleliteralemphasis{\sphinxupquote{str}}) \textendash{} Molecule type for both of the entities.

\item {} 
\sphinxstyleliteralstrong{\sphinxupquote{source\_entity\_type}} (\sphinxstyleliteralemphasis{\sphinxupquote{str}}) \textendash{} Molecule type for the source entity.

\item {} 
\sphinxstyleliteralstrong{\sphinxupquote{target\_entity\_type}} (\sphinxstyleliteralemphasis{\sphinxupquote{str}}) \textendash{} Molecule type for the target entity.

\end{itemize}

\end{description}\end{quote}

\end{fulllineitems}

\index{interactions\_by\_interaction\_type\_and\_data\_model() (pypath.core.interaction.Interaction method)@\spxentry{interactions\_by\_interaction\_type\_and\_data\_model()}\spxextra{pypath.core.interaction.Interaction method}}

\begin{fulllineitems}
\phantomsection\label{\detokenize{reference:pypath.core.interaction.Interaction.interactions_by_interaction_type_and_data_model}}\pysiglinewithargsret{\sphinxbfcode{\sphinxupquote{interactions\_by\_interaction\_type\_and\_data\_model}}}{\emph{effect=None}, \emph{resources=None}, \emph{data\_model=None}, \emph{interaction\_type=None}, \emph{via=None}, \emph{references=None}}{}
Returns one or two tuples of the interacting partners: one if only
one direction, two if both directions match the query criteria.
The tuple will be empty if no evidence matches the criteria.
\begin{quote}\begin{description}
\item[{Parameters}] \leavevmode\begin{itemize}
\item {} 
\sphinxstyleliteralstrong{\sphinxupquote{direction}} (\sphinxstyleliteralemphasis{\sphinxupquote{NontType}}\sphinxstyleliteralemphasis{\sphinxupquote{,}}\sphinxstyleliteralemphasis{\sphinxupquote{bool}}\sphinxstyleliteralemphasis{\sphinxupquote{,}}\sphinxstyleliteralemphasis{\sphinxupquote{tuple}}) \textendash{} If \sphinxtitleref{None} both undirected and directed, if \sphinxtitleref{True} only directed,
if a \sphinxstyleemphasis{tuple} of entities only the interactions with that specific
direction will be considered. Unless you set this parameter to
\sphinxtitleref{True} this method will return both directions if one or more
undirected resources present.
If \sphinxtitleref{False}, only the undirected interactions will be considered,
and if any resource annotates this interaction as undirected
both directions will be returned. However the
\sphinxcode{\sphinxupquote{count\_interactions\_undirected}} method will return \sphinxtitleref{1}
in this case.

\item {} 
\sphinxstyleliteralstrong{\sphinxupquote{effect}} (\sphinxstyleliteralemphasis{\sphinxupquote{NoneType}}\sphinxstyleliteralemphasis{\sphinxupquote{,}}\sphinxstyleliteralemphasis{\sphinxupquote{bool}}\sphinxstyleliteralemphasis{\sphinxupquote{,}}\sphinxstyleliteralemphasis{\sphinxupquote{str}}) \textendash{} If \sphinxtitleref{None} also interactions without effect, if \sphinxtitleref{True} only
the ones with any effect, if a string naming an effect only the
interactions with that specific effect will be considered.

\item {} 
\sphinxstyleliteralstrong{\sphinxupquote{resources}} (\sphinxstyleliteralemphasis{\sphinxupquote{NontType}}\sphinxstyleliteralemphasis{\sphinxupquote{,}}\sphinxstyleliteralemphasis{\sphinxupquote{str}}\sphinxstyleliteralemphasis{\sphinxupquote{,}}\sphinxstyleliteralemphasis{\sphinxupquote{set}}) \textendash{} Optionally limit the query to one or more resources.

\item {} 
\sphinxstyleliteralstrong{\sphinxupquote{data\_model}} (\sphinxstyleliteralemphasis{\sphinxupquote{NontType}}\sphinxstyleliteralemphasis{\sphinxupquote{,}}\sphinxstyleliteralemphasis{\sphinxupquote{str}}\sphinxstyleliteralemphasis{\sphinxupquote{,}}\sphinxstyleliteralemphasis{\sphinxupquote{set}}) \textendash{} Optionally limit the query to one or more data models e.g.
\sphinxtitleref{activity\_flow}.

\item {} 
\sphinxstyleliteralstrong{\sphinxupquote{interaction\_type}} (\sphinxstyleliteralemphasis{\sphinxupquote{NontType}}\sphinxstyleliteralemphasis{\sphinxupquote{,}}\sphinxstyleliteralemphasis{\sphinxupquote{str}}\sphinxstyleliteralemphasis{\sphinxupquote{,}}\sphinxstyleliteralemphasis{\sphinxupquote{set}}) \textendash{} Optionally limit the query to one or more interaction types
e.g. \sphinxtitleref{PPI}.

\item {} 
\sphinxstyleliteralstrong{\sphinxupquote{via}} (\sphinxstyleliteralemphasis{\sphinxupquote{NontType}}\sphinxstyleliteralemphasis{\sphinxupquote{,}}\sphinxstyleliteralemphasis{\sphinxupquote{bool}}\sphinxstyleliteralemphasis{\sphinxupquote{,}}\sphinxstyleliteralemphasis{\sphinxupquote{str}}\sphinxstyleliteralemphasis{\sphinxupquote{,}}\sphinxstyleliteralemphasis{\sphinxupquote{set}}) \textendash{} Optionally limit the query to certain secondary databases or
if \sphinxtitleref{False} consider only data from primary databases.

\item {} 
\sphinxstyleliteralstrong{\sphinxupquote{entity\_type}} (\sphinxstyleliteralemphasis{\sphinxupquote{str}}) \textendash{} Molecule type for both of the entities.

\item {} 
\sphinxstyleliteralstrong{\sphinxupquote{source\_entity\_type}} (\sphinxstyleliteralemphasis{\sphinxupquote{str}}) \textendash{} Molecule type for the source entity.

\item {} 
\sphinxstyleliteralstrong{\sphinxupquote{target\_entity\_type}} (\sphinxstyleliteralemphasis{\sphinxupquote{str}}) \textendash{} Molecule type for the target entity.

\end{itemize}

\end{description}\end{quote}

\end{fulllineitems}

\index{interactions\_by\_interaction\_type\_and\_data\_model\_and\_resource() (pypath.core.interaction.Interaction method)@\spxentry{interactions\_by\_interaction\_type\_and\_data\_model\_and\_resource()}\spxextra{pypath.core.interaction.Interaction method}}

\begin{fulllineitems}
\phantomsection\label{\detokenize{reference:pypath.core.interaction.Interaction.interactions_by_interaction_type_and_data_model_and_resource}}\pysiglinewithargsret{\sphinxbfcode{\sphinxupquote{interactions\_by\_interaction\_type\_and\_data\_model\_and\_resource}}}{\emph{effect=None}, \emph{resources=None}, \emph{data\_model=None}, \emph{interaction\_type=None}, \emph{via=None}, \emph{references=None}}{}
Returns one or two tuples of the interacting partners: one if only
one direction, two if both directions match the query criteria.
The tuple will be empty if no evidence matches the criteria.
\begin{quote}\begin{description}
\item[{Parameters}] \leavevmode\begin{itemize}
\item {} 
\sphinxstyleliteralstrong{\sphinxupquote{direction}} (\sphinxstyleliteralemphasis{\sphinxupquote{NontType}}\sphinxstyleliteralemphasis{\sphinxupquote{,}}\sphinxstyleliteralemphasis{\sphinxupquote{bool}}\sphinxstyleliteralemphasis{\sphinxupquote{,}}\sphinxstyleliteralemphasis{\sphinxupquote{tuple}}) \textendash{} If \sphinxtitleref{None} both undirected and directed, if \sphinxtitleref{True} only directed,
if a \sphinxstyleemphasis{tuple} of entities only the interactions with that specific
direction will be considered. Unless you set this parameter to
\sphinxtitleref{True} this method will return both directions if one or more
undirected resources present.
If \sphinxtitleref{False}, only the undirected interactions will be considered,
and if any resource annotates this interaction as undirected
both directions will be returned. However the
\sphinxcode{\sphinxupquote{count\_interactions\_undirected}} method will return \sphinxtitleref{1}
in this case.

\item {} 
\sphinxstyleliteralstrong{\sphinxupquote{effect}} (\sphinxstyleliteralemphasis{\sphinxupquote{NoneType}}\sphinxstyleliteralemphasis{\sphinxupquote{,}}\sphinxstyleliteralemphasis{\sphinxupquote{bool}}\sphinxstyleliteralemphasis{\sphinxupquote{,}}\sphinxstyleliteralemphasis{\sphinxupquote{str}}) \textendash{} If \sphinxtitleref{None} also interactions without effect, if \sphinxtitleref{True} only
the ones with any effect, if a string naming an effect only the
interactions with that specific effect will be considered.

\item {} 
\sphinxstyleliteralstrong{\sphinxupquote{resources}} (\sphinxstyleliteralemphasis{\sphinxupquote{NontType}}\sphinxstyleliteralemphasis{\sphinxupquote{,}}\sphinxstyleliteralemphasis{\sphinxupquote{str}}\sphinxstyleliteralemphasis{\sphinxupquote{,}}\sphinxstyleliteralemphasis{\sphinxupquote{set}}) \textendash{} Optionally limit the query to one or more resources.

\item {} 
\sphinxstyleliteralstrong{\sphinxupquote{data\_model}} (\sphinxstyleliteralemphasis{\sphinxupquote{NontType}}\sphinxstyleliteralemphasis{\sphinxupquote{,}}\sphinxstyleliteralemphasis{\sphinxupquote{str}}\sphinxstyleliteralemphasis{\sphinxupquote{,}}\sphinxstyleliteralemphasis{\sphinxupquote{set}}) \textendash{} Optionally limit the query to one or more data models e.g.
\sphinxtitleref{activity\_flow}.

\item {} 
\sphinxstyleliteralstrong{\sphinxupquote{interaction\_type}} (\sphinxstyleliteralemphasis{\sphinxupquote{NontType}}\sphinxstyleliteralemphasis{\sphinxupquote{,}}\sphinxstyleliteralemphasis{\sphinxupquote{str}}\sphinxstyleliteralemphasis{\sphinxupquote{,}}\sphinxstyleliteralemphasis{\sphinxupquote{set}}) \textendash{} Optionally limit the query to one or more interaction types
e.g. \sphinxtitleref{PPI}.

\item {} 
\sphinxstyleliteralstrong{\sphinxupquote{via}} (\sphinxstyleliteralemphasis{\sphinxupquote{NontType}}\sphinxstyleliteralemphasis{\sphinxupquote{,}}\sphinxstyleliteralemphasis{\sphinxupquote{bool}}\sphinxstyleliteralemphasis{\sphinxupquote{,}}\sphinxstyleliteralemphasis{\sphinxupquote{str}}\sphinxstyleliteralemphasis{\sphinxupquote{,}}\sphinxstyleliteralemphasis{\sphinxupquote{set}}) \textendash{} Optionally limit the query to certain secondary databases or
if \sphinxtitleref{False} consider only data from primary databases.

\item {} 
\sphinxstyleliteralstrong{\sphinxupquote{entity\_type}} (\sphinxstyleliteralemphasis{\sphinxupquote{str}}) \textendash{} Molecule type for both of the entities.

\item {} 
\sphinxstyleliteralstrong{\sphinxupquote{source\_entity\_type}} (\sphinxstyleliteralemphasis{\sphinxupquote{str}}) \textendash{} Molecule type for the source entity.

\item {} 
\sphinxstyleliteralstrong{\sphinxupquote{target\_entity\_type}} (\sphinxstyleliteralemphasis{\sphinxupquote{str}}) \textendash{} Molecule type for the target entity.

\end{itemize}

\end{description}\end{quote}

\end{fulllineitems}

\index{interactions\_by\_reference() (pypath.core.interaction.Interaction method)@\spxentry{interactions\_by\_reference()}\spxextra{pypath.core.interaction.Interaction method}}

\begin{fulllineitems}
\phantomsection\label{\detokenize{reference:pypath.core.interaction.Interaction.interactions_by_reference}}\pysiglinewithargsret{\sphinxbfcode{\sphinxupquote{interactions\_by\_reference}}}{\emph{effect=None}, \emph{resources=None}, \emph{data\_model=None}, \emph{interaction\_type=None}, \emph{via=None}, \emph{references=None}}{}
Returns one or two tuples of the interacting partners: one if only
one direction, two if both directions match the query criteria.
The tuple will be empty if no evidence matches the criteria.
\begin{quote}\begin{description}
\item[{Parameters}] \leavevmode\begin{itemize}
\item {} 
\sphinxstyleliteralstrong{\sphinxupquote{direction}} (\sphinxstyleliteralemphasis{\sphinxupquote{NontType}}\sphinxstyleliteralemphasis{\sphinxupquote{,}}\sphinxstyleliteralemphasis{\sphinxupquote{bool}}\sphinxstyleliteralemphasis{\sphinxupquote{,}}\sphinxstyleliteralemphasis{\sphinxupquote{tuple}}) \textendash{} If \sphinxtitleref{None} both undirected and directed, if \sphinxtitleref{True} only directed,
if a \sphinxstyleemphasis{tuple} of entities only the interactions with that specific
direction will be considered. Unless you set this parameter to
\sphinxtitleref{True} this method will return both directions if one or more
undirected resources present.
If \sphinxtitleref{False}, only the undirected interactions will be considered,
and if any resource annotates this interaction as undirected
both directions will be returned. However the
\sphinxcode{\sphinxupquote{count\_interactions\_undirected}} method will return \sphinxtitleref{1}
in this case.

\item {} 
\sphinxstyleliteralstrong{\sphinxupquote{effect}} (\sphinxstyleliteralemphasis{\sphinxupquote{NoneType}}\sphinxstyleliteralemphasis{\sphinxupquote{,}}\sphinxstyleliteralemphasis{\sphinxupquote{bool}}\sphinxstyleliteralemphasis{\sphinxupquote{,}}\sphinxstyleliteralemphasis{\sphinxupquote{str}}) \textendash{} If \sphinxtitleref{None} also interactions without effect, if \sphinxtitleref{True} only
the ones with any effect, if a string naming an effect only the
interactions with that specific effect will be considered.

\item {} 
\sphinxstyleliteralstrong{\sphinxupquote{resources}} (\sphinxstyleliteralemphasis{\sphinxupquote{NontType}}\sphinxstyleliteralemphasis{\sphinxupquote{,}}\sphinxstyleliteralemphasis{\sphinxupquote{str}}\sphinxstyleliteralemphasis{\sphinxupquote{,}}\sphinxstyleliteralemphasis{\sphinxupquote{set}}) \textendash{} Optionally limit the query to one or more resources.

\item {} 
\sphinxstyleliteralstrong{\sphinxupquote{data\_model}} (\sphinxstyleliteralemphasis{\sphinxupquote{NontType}}\sphinxstyleliteralemphasis{\sphinxupquote{,}}\sphinxstyleliteralemphasis{\sphinxupquote{str}}\sphinxstyleliteralemphasis{\sphinxupquote{,}}\sphinxstyleliteralemphasis{\sphinxupquote{set}}) \textendash{} Optionally limit the query to one or more data models e.g.
\sphinxtitleref{activity\_flow}.

\item {} 
\sphinxstyleliteralstrong{\sphinxupquote{interaction\_type}} (\sphinxstyleliteralemphasis{\sphinxupquote{NontType}}\sphinxstyleliteralemphasis{\sphinxupquote{,}}\sphinxstyleliteralemphasis{\sphinxupquote{str}}\sphinxstyleliteralemphasis{\sphinxupquote{,}}\sphinxstyleliteralemphasis{\sphinxupquote{set}}) \textendash{} Optionally limit the query to one or more interaction types
e.g. \sphinxtitleref{PPI}.

\item {} 
\sphinxstyleliteralstrong{\sphinxupquote{via}} (\sphinxstyleliteralemphasis{\sphinxupquote{NontType}}\sphinxstyleliteralemphasis{\sphinxupquote{,}}\sphinxstyleliteralemphasis{\sphinxupquote{bool}}\sphinxstyleliteralemphasis{\sphinxupquote{,}}\sphinxstyleliteralemphasis{\sphinxupquote{str}}\sphinxstyleliteralemphasis{\sphinxupquote{,}}\sphinxstyleliteralemphasis{\sphinxupquote{set}}) \textendash{} Optionally limit the query to certain secondary databases or
if \sphinxtitleref{False} consider only data from primary databases.

\item {} 
\sphinxstyleliteralstrong{\sphinxupquote{entity\_type}} (\sphinxstyleliteralemphasis{\sphinxupquote{str}}) \textendash{} Molecule type for both of the entities.

\item {} 
\sphinxstyleliteralstrong{\sphinxupquote{source\_entity\_type}} (\sphinxstyleliteralemphasis{\sphinxupquote{str}}) \textendash{} Molecule type for the source entity.

\item {} 
\sphinxstyleliteralstrong{\sphinxupquote{target\_entity\_type}} (\sphinxstyleliteralemphasis{\sphinxupquote{str}}) \textendash{} Molecule type for the target entity.

\end{itemize}

\end{description}\end{quote}

\end{fulllineitems}

\index{interactions\_by\_resource() (pypath.core.interaction.Interaction method)@\spxentry{interactions\_by\_resource()}\spxextra{pypath.core.interaction.Interaction method}}

\begin{fulllineitems}
\phantomsection\label{\detokenize{reference:pypath.core.interaction.Interaction.interactions_by_resource}}\pysiglinewithargsret{\sphinxbfcode{\sphinxupquote{interactions\_by\_resource}}}{\emph{effect=None}, \emph{resources=None}, \emph{data\_model=None}, \emph{interaction\_type=None}, \emph{via=None}, \emph{references=None}}{}
Returns one or two tuples of the interacting partners: one if only
one direction, two if both directions match the query criteria.
The tuple will be empty if no evidence matches the criteria.
\begin{quote}\begin{description}
\item[{Parameters}] \leavevmode\begin{itemize}
\item {} 
\sphinxstyleliteralstrong{\sphinxupquote{direction}} (\sphinxstyleliteralemphasis{\sphinxupquote{NontType}}\sphinxstyleliteralemphasis{\sphinxupquote{,}}\sphinxstyleliteralemphasis{\sphinxupquote{bool}}\sphinxstyleliteralemphasis{\sphinxupquote{,}}\sphinxstyleliteralemphasis{\sphinxupquote{tuple}}) \textendash{} If \sphinxtitleref{None} both undirected and directed, if \sphinxtitleref{True} only directed,
if a \sphinxstyleemphasis{tuple} of entities only the interactions with that specific
direction will be considered. Unless you set this parameter to
\sphinxtitleref{True} this method will return both directions if one or more
undirected resources present.
If \sphinxtitleref{False}, only the undirected interactions will be considered,
and if any resource annotates this interaction as undirected
both directions will be returned. However the
\sphinxcode{\sphinxupquote{count\_interactions\_undirected}} method will return \sphinxtitleref{1}
in this case.

\item {} 
\sphinxstyleliteralstrong{\sphinxupquote{effect}} (\sphinxstyleliteralemphasis{\sphinxupquote{NoneType}}\sphinxstyleliteralemphasis{\sphinxupquote{,}}\sphinxstyleliteralemphasis{\sphinxupquote{bool}}\sphinxstyleliteralemphasis{\sphinxupquote{,}}\sphinxstyleliteralemphasis{\sphinxupquote{str}}) \textendash{} If \sphinxtitleref{None} also interactions without effect, if \sphinxtitleref{True} only
the ones with any effect, if a string naming an effect only the
interactions with that specific effect will be considered.

\item {} 
\sphinxstyleliteralstrong{\sphinxupquote{resources}} (\sphinxstyleliteralemphasis{\sphinxupquote{NontType}}\sphinxstyleliteralemphasis{\sphinxupquote{,}}\sphinxstyleliteralemphasis{\sphinxupquote{str}}\sphinxstyleliteralemphasis{\sphinxupquote{,}}\sphinxstyleliteralemphasis{\sphinxupquote{set}}) \textendash{} Optionally limit the query to one or more resources.

\item {} 
\sphinxstyleliteralstrong{\sphinxupquote{data\_model}} (\sphinxstyleliteralemphasis{\sphinxupquote{NontType}}\sphinxstyleliteralemphasis{\sphinxupquote{,}}\sphinxstyleliteralemphasis{\sphinxupquote{str}}\sphinxstyleliteralemphasis{\sphinxupquote{,}}\sphinxstyleliteralemphasis{\sphinxupquote{set}}) \textendash{} Optionally limit the query to one or more data models e.g.
\sphinxtitleref{activity\_flow}.

\item {} 
\sphinxstyleliteralstrong{\sphinxupquote{interaction\_type}} (\sphinxstyleliteralemphasis{\sphinxupquote{NontType}}\sphinxstyleliteralemphasis{\sphinxupquote{,}}\sphinxstyleliteralemphasis{\sphinxupquote{str}}\sphinxstyleliteralemphasis{\sphinxupquote{,}}\sphinxstyleliteralemphasis{\sphinxupquote{set}}) \textendash{} Optionally limit the query to one or more interaction types
e.g. \sphinxtitleref{PPI}.

\item {} 
\sphinxstyleliteralstrong{\sphinxupquote{via}} (\sphinxstyleliteralemphasis{\sphinxupquote{NontType}}\sphinxstyleliteralemphasis{\sphinxupquote{,}}\sphinxstyleliteralemphasis{\sphinxupquote{bool}}\sphinxstyleliteralemphasis{\sphinxupquote{,}}\sphinxstyleliteralemphasis{\sphinxupquote{str}}\sphinxstyleliteralemphasis{\sphinxupquote{,}}\sphinxstyleliteralemphasis{\sphinxupquote{set}}) \textendash{} Optionally limit the query to certain secondary databases or
if \sphinxtitleref{False} consider only data from primary databases.

\item {} 
\sphinxstyleliteralstrong{\sphinxupquote{entity\_type}} (\sphinxstyleliteralemphasis{\sphinxupquote{str}}) \textendash{} Molecule type for both of the entities.

\item {} 
\sphinxstyleliteralstrong{\sphinxupquote{source\_entity\_type}} (\sphinxstyleliteralemphasis{\sphinxupquote{str}}) \textendash{} Molecule type for the source entity.

\item {} 
\sphinxstyleliteralstrong{\sphinxupquote{target\_entity\_type}} (\sphinxstyleliteralemphasis{\sphinxupquote{str}}) \textendash{} Molecule type for the target entity.

\end{itemize}

\end{description}\end{quote}

\end{fulllineitems}

\index{interactions\_directed\_by\_data\_model() (pypath.core.interaction.Interaction method)@\spxentry{interactions\_directed\_by\_data\_model()}\spxextra{pypath.core.interaction.Interaction method}}

\begin{fulllineitems}
\phantomsection\label{\detokenize{reference:pypath.core.interaction.Interaction.interactions_directed_by_data_model}}\pysiglinewithargsret{\sphinxbfcode{\sphinxupquote{interactions\_directed\_by\_data\_model}}}{\emph{effect=None}, \emph{resources=None}, \emph{data\_model=None}, \emph{interaction\_type=None}, \emph{via=None}, \emph{references=None}}{}
{\color{red}\bfseries{}**}kwargs: see the docs of method \sphinxcode{\sphinxupquote{get\_interactions}}.

\end{fulllineitems}

\index{interactions\_directed\_by\_interaction\_type() (pypath.core.interaction.Interaction method)@\spxentry{interactions\_directed\_by\_interaction\_type()}\spxextra{pypath.core.interaction.Interaction method}}

\begin{fulllineitems}
\phantomsection\label{\detokenize{reference:pypath.core.interaction.Interaction.interactions_directed_by_interaction_type}}\pysiglinewithargsret{\sphinxbfcode{\sphinxupquote{interactions\_directed\_by\_interaction\_type}}}{\emph{effect=None}, \emph{resources=None}, \emph{data\_model=None}, \emph{interaction\_type=None}, \emph{via=None}, \emph{references=None}}{}
{\color{red}\bfseries{}**}kwargs: see the docs of method \sphinxcode{\sphinxupquote{get\_interactions}}.

\end{fulllineitems}

\index{interactions\_directed\_by\_interaction\_type\_and\_data\_model() (pypath.core.interaction.Interaction method)@\spxentry{interactions\_directed\_by\_interaction\_type\_and\_data\_model()}\spxextra{pypath.core.interaction.Interaction method}}

\begin{fulllineitems}
\phantomsection\label{\detokenize{reference:pypath.core.interaction.Interaction.interactions_directed_by_interaction_type_and_data_model}}\pysiglinewithargsret{\sphinxbfcode{\sphinxupquote{interactions\_directed\_by\_interaction\_type\_and\_data\_model}}}{\emph{effect=None}, \emph{resources=None}, \emph{data\_model=None}, \emph{interaction\_type=None}, \emph{via=None}, \emph{references=None}}{}
{\color{red}\bfseries{}**}kwargs: see the docs of method \sphinxcode{\sphinxupquote{get\_interactions}}.

\end{fulllineitems}

\index{interactions\_directed\_by\_interaction\_type\_and\_data\_model\_and\_resource() (pypath.core.interaction.Interaction method)@\spxentry{interactions\_directed\_by\_interaction\_type\_and\_data\_model\_and\_resource()}\spxextra{pypath.core.interaction.Interaction method}}

\begin{fulllineitems}
\phantomsection\label{\detokenize{reference:pypath.core.interaction.Interaction.interactions_directed_by_interaction_type_and_data_model_and_resource}}\pysiglinewithargsret{\sphinxbfcode{\sphinxupquote{interactions\_directed\_by\_interaction\_type\_and\_data\_model\_and\_resource}}}{\emph{effect=None}, \emph{resources=None}, \emph{data\_model=None}, \emph{interaction\_type=None}, \emph{via=None}, \emph{references=None}}{}
{\color{red}\bfseries{}**}kwargs: see the docs of method \sphinxcode{\sphinxupquote{get\_interactions}}.

\end{fulllineitems}

\index{interactions\_directed\_by\_reference() (pypath.core.interaction.Interaction method)@\spxentry{interactions\_directed\_by\_reference()}\spxextra{pypath.core.interaction.Interaction method}}

\begin{fulllineitems}
\phantomsection\label{\detokenize{reference:pypath.core.interaction.Interaction.interactions_directed_by_reference}}\pysiglinewithargsret{\sphinxbfcode{\sphinxupquote{interactions\_directed\_by\_reference}}}{\emph{effect=None}, \emph{resources=None}, \emph{data\_model=None}, \emph{interaction\_type=None}, \emph{via=None}, \emph{references=None}}{}
{\color{red}\bfseries{}**}kwargs: see the docs of method \sphinxcode{\sphinxupquote{get\_interactions}}.

\end{fulllineitems}

\index{interactions\_directed\_by\_resource() (pypath.core.interaction.Interaction method)@\spxentry{interactions\_directed\_by\_resource()}\spxextra{pypath.core.interaction.Interaction method}}

\begin{fulllineitems}
\phantomsection\label{\detokenize{reference:pypath.core.interaction.Interaction.interactions_directed_by_resource}}\pysiglinewithargsret{\sphinxbfcode{\sphinxupquote{interactions\_directed\_by\_resource}}}{\emph{effect=None}, \emph{resources=None}, \emph{data\_model=None}, \emph{interaction\_type=None}, \emph{via=None}, \emph{references=None}}{}
{\color{red}\bfseries{}**}kwargs: see the docs of method \sphinxcode{\sphinxupquote{get\_interactions}}.

\end{fulllineitems}

\index{interactions\_mutual\_by\_data\_model() (pypath.core.interaction.Interaction method)@\spxentry{interactions\_mutual\_by\_data\_model()}\spxextra{pypath.core.interaction.Interaction method}}

\begin{fulllineitems}
\phantomsection\label{\detokenize{reference:pypath.core.interaction.Interaction.interactions_mutual_by_data_model}}\pysiglinewithargsret{\sphinxbfcode{\sphinxupquote{interactions\_mutual\_by\_data\_model}}}{\emph{effect=None}, \emph{resources=None}, \emph{data\_model=None}, \emph{interaction\_type=None}, \emph{via=None}, \emph{references=None}}{}
Note: undirected interactions does not count as mutual but only
interactions with explicit direction information for both directions.

{\color{red}\bfseries{}**}kwargs: see the docs of method \sphinxcode{\sphinxupquote{get\_interactions}}.

\end{fulllineitems}

\index{interactions\_mutual\_by\_interaction\_type() (pypath.core.interaction.Interaction method)@\spxentry{interactions\_mutual\_by\_interaction\_type()}\spxextra{pypath.core.interaction.Interaction method}}

\begin{fulllineitems}
\phantomsection\label{\detokenize{reference:pypath.core.interaction.Interaction.interactions_mutual_by_interaction_type}}\pysiglinewithargsret{\sphinxbfcode{\sphinxupquote{interactions\_mutual\_by\_interaction\_type}}}{\emph{effect=None}, \emph{resources=None}, \emph{data\_model=None}, \emph{interaction\_type=None}, \emph{via=None}, \emph{references=None}}{}
Note: undirected interactions does not count as mutual but only
interactions with explicit direction information for both directions.

{\color{red}\bfseries{}**}kwargs: see the docs of method \sphinxcode{\sphinxupquote{get\_interactions}}.

\end{fulllineitems}

\index{interactions\_mutual\_by\_interaction\_type\_and\_data\_model() (pypath.core.interaction.Interaction method)@\spxentry{interactions\_mutual\_by\_interaction\_type\_and\_data\_model()}\spxextra{pypath.core.interaction.Interaction method}}

\begin{fulllineitems}
\phantomsection\label{\detokenize{reference:pypath.core.interaction.Interaction.interactions_mutual_by_interaction_type_and_data_model}}\pysiglinewithargsret{\sphinxbfcode{\sphinxupquote{interactions\_mutual\_by\_interaction\_type\_and\_data\_model}}}{\emph{effect=None}, \emph{resources=None}, \emph{data\_model=None}, \emph{interaction\_type=None}, \emph{via=None}, \emph{references=None}}{}
Note: undirected interactions does not count as mutual but only
interactions with explicit direction information for both directions.

{\color{red}\bfseries{}**}kwargs: see the docs of method \sphinxcode{\sphinxupquote{get\_interactions}}.

\end{fulllineitems}

\index{interactions\_mutual\_by\_interaction\_type\_and\_data\_model\_and\_resource() (pypath.core.interaction.Interaction method)@\spxentry{interactions\_mutual\_by\_interaction\_type\_and\_data\_model\_and\_resource()}\spxextra{pypath.core.interaction.Interaction method}}

\begin{fulllineitems}
\phantomsection\label{\detokenize{reference:pypath.core.interaction.Interaction.interactions_mutual_by_interaction_type_and_data_model_and_resource}}\pysiglinewithargsret{\sphinxbfcode{\sphinxupquote{interactions\_mutual\_by\_interaction\_type\_and\_data\_model\_and\_resource}}}{\emph{effect=None}, \emph{resources=None}, \emph{data\_model=None}, \emph{interaction\_type=None}, \emph{via=None}, \emph{references=None}}{}
Note: undirected interactions does not count as mutual but only
interactions with explicit direction information for both directions.

{\color{red}\bfseries{}**}kwargs: see the docs of method \sphinxcode{\sphinxupquote{get\_interactions}}.

\end{fulllineitems}

\index{interactions\_mutual\_by\_reference() (pypath.core.interaction.Interaction method)@\spxentry{interactions\_mutual\_by\_reference()}\spxextra{pypath.core.interaction.Interaction method}}

\begin{fulllineitems}
\phantomsection\label{\detokenize{reference:pypath.core.interaction.Interaction.interactions_mutual_by_reference}}\pysiglinewithargsret{\sphinxbfcode{\sphinxupquote{interactions\_mutual\_by\_reference}}}{\emph{effect=None}, \emph{resources=None}, \emph{data\_model=None}, \emph{interaction\_type=None}, \emph{via=None}, \emph{references=None}}{}
Note: undirected interactions does not count as mutual but only
interactions with explicit direction information for both directions.

{\color{red}\bfseries{}**}kwargs: see the docs of method \sphinxcode{\sphinxupquote{get\_interactions}}.

\end{fulllineitems}

\index{interactions\_mutual\_by\_resource() (pypath.core.interaction.Interaction method)@\spxentry{interactions\_mutual\_by\_resource()}\spxextra{pypath.core.interaction.Interaction method}}

\begin{fulllineitems}
\phantomsection\label{\detokenize{reference:pypath.core.interaction.Interaction.interactions_mutual_by_resource}}\pysiglinewithargsret{\sphinxbfcode{\sphinxupquote{interactions\_mutual\_by\_resource}}}{\emph{effect=None}, \emph{resources=None}, \emph{data\_model=None}, \emph{interaction\_type=None}, \emph{via=None}, \emph{references=None}}{}
Note: undirected interactions does not count as mutual but only
interactions with explicit direction information for both directions.

{\color{red}\bfseries{}**}kwargs: see the docs of method \sphinxcode{\sphinxupquote{get\_interactions}}.

\end{fulllineitems}

\index{interactions\_negative\_by\_data\_model() (pypath.core.interaction.Interaction method)@\spxentry{interactions\_negative\_by\_data\_model()}\spxextra{pypath.core.interaction.Interaction method}}

\begin{fulllineitems}
\phantomsection\label{\detokenize{reference:pypath.core.interaction.Interaction.interactions_negative_by_data_model}}\pysiglinewithargsret{\sphinxbfcode{\sphinxupquote{interactions\_negative\_by\_data\_model}}}{\emph{effect=None}, \emph{resources=None}, \emph{data\_model=None}, \emph{interaction\_type=None}, \emph{via=None}, \emph{references=None}}{}
{\color{red}\bfseries{}**}kwargs: see the docs of method \sphinxcode{\sphinxupquote{get\_interactions}}.

\end{fulllineitems}

\index{interactions\_negative\_by\_interaction\_type() (pypath.core.interaction.Interaction method)@\spxentry{interactions\_negative\_by\_interaction\_type()}\spxextra{pypath.core.interaction.Interaction method}}

\begin{fulllineitems}
\phantomsection\label{\detokenize{reference:pypath.core.interaction.Interaction.interactions_negative_by_interaction_type}}\pysiglinewithargsret{\sphinxbfcode{\sphinxupquote{interactions\_negative\_by\_interaction\_type}}}{\emph{effect=None}, \emph{resources=None}, \emph{data\_model=None}, \emph{interaction\_type=None}, \emph{via=None}, \emph{references=None}}{}
{\color{red}\bfseries{}**}kwargs: see the docs of method \sphinxcode{\sphinxupquote{get\_interactions}}.

\end{fulllineitems}

\index{interactions\_negative\_by\_interaction\_type\_and\_data\_model() (pypath.core.interaction.Interaction method)@\spxentry{interactions\_negative\_by\_interaction\_type\_and\_data\_model()}\spxextra{pypath.core.interaction.Interaction method}}

\begin{fulllineitems}
\phantomsection\label{\detokenize{reference:pypath.core.interaction.Interaction.interactions_negative_by_interaction_type_and_data_model}}\pysiglinewithargsret{\sphinxbfcode{\sphinxupquote{interactions\_negative\_by\_interaction\_type\_and\_data\_model}}}{\emph{effect=None}, \emph{resources=None}, \emph{data\_model=None}, \emph{interaction\_type=None}, \emph{via=None}, \emph{references=None}}{}
{\color{red}\bfseries{}**}kwargs: see the docs of method \sphinxcode{\sphinxupquote{get\_interactions}}.

\end{fulllineitems}

\index{interactions\_negative\_by\_interaction\_type\_and\_data\_model\_and\_resource() (pypath.core.interaction.Interaction method)@\spxentry{interactions\_negative\_by\_interaction\_type\_and\_data\_model\_and\_resource()}\spxextra{pypath.core.interaction.Interaction method}}

\begin{fulllineitems}
\phantomsection\label{\detokenize{reference:pypath.core.interaction.Interaction.interactions_negative_by_interaction_type_and_data_model_and_resource}}\pysiglinewithargsret{\sphinxbfcode{\sphinxupquote{interactions\_negative\_by\_interaction\_type\_and\_data\_model\_and\_resource}}}{\emph{effect=None}, \emph{resources=None}, \emph{data\_model=None}, \emph{interaction\_type=None}, \emph{via=None}, \emph{references=None}}{}
{\color{red}\bfseries{}**}kwargs: see the docs of method \sphinxcode{\sphinxupquote{get\_interactions}}.

\end{fulllineitems}

\index{interactions\_negative\_by\_reference() (pypath.core.interaction.Interaction method)@\spxentry{interactions\_negative\_by\_reference()}\spxextra{pypath.core.interaction.Interaction method}}

\begin{fulllineitems}
\phantomsection\label{\detokenize{reference:pypath.core.interaction.Interaction.interactions_negative_by_reference}}\pysiglinewithargsret{\sphinxbfcode{\sphinxupquote{interactions\_negative\_by\_reference}}}{\emph{effect=None}, \emph{resources=None}, \emph{data\_model=None}, \emph{interaction\_type=None}, \emph{via=None}, \emph{references=None}}{}
{\color{red}\bfseries{}**}kwargs: see the docs of method \sphinxcode{\sphinxupquote{get\_interactions}}.

\end{fulllineitems}

\index{interactions\_negative\_by\_resource() (pypath.core.interaction.Interaction method)@\spxentry{interactions\_negative\_by\_resource()}\spxextra{pypath.core.interaction.Interaction method}}

\begin{fulllineitems}
\phantomsection\label{\detokenize{reference:pypath.core.interaction.Interaction.interactions_negative_by_resource}}\pysiglinewithargsret{\sphinxbfcode{\sphinxupquote{interactions\_negative\_by\_resource}}}{\emph{effect=None}, \emph{resources=None}, \emph{data\_model=None}, \emph{interaction\_type=None}, \emph{via=None}, \emph{references=None}}{}
{\color{red}\bfseries{}**}kwargs: see the docs of method \sphinxcode{\sphinxupquote{get\_interactions}}.

\end{fulllineitems}

\index{interactions\_non\_directed\_0\_by\_data\_model() (pypath.core.interaction.Interaction method)@\spxentry{interactions\_non\_directed\_0\_by\_data\_model()}\spxextra{pypath.core.interaction.Interaction method}}

\begin{fulllineitems}
\phantomsection\label{\detokenize{reference:pypath.core.interaction.Interaction.interactions_non_directed_0_by_data_model}}\pysiglinewithargsret{\sphinxbfcode{\sphinxupquote{interactions\_non\_directed\_0\_by\_data\_model}}}{\emph{effect=None}, \emph{resources=None}, \emph{data\_model=None}, \emph{interaction\_type=None}, \emph{via=None}, \emph{references=None}}{}
Only the undirected interactions will be considered, if any resource
annotates this interaction as undirected and none as directed, the
interacting pair as a sorted tuple will be returned inside a one
element tuple.

{\color{red}\bfseries{}**}kwargs: see the docs of method \sphinxcode{\sphinxupquote{get\_interactions}}.

\end{fulllineitems}

\index{interactions\_non\_directed\_0\_by\_interaction\_type() (pypath.core.interaction.Interaction method)@\spxentry{interactions\_non\_directed\_0\_by\_interaction\_type()}\spxextra{pypath.core.interaction.Interaction method}}

\begin{fulllineitems}
\phantomsection\label{\detokenize{reference:pypath.core.interaction.Interaction.interactions_non_directed_0_by_interaction_type}}\pysiglinewithargsret{\sphinxbfcode{\sphinxupquote{interactions\_non\_directed\_0\_by\_interaction\_type}}}{\emph{effect=None}, \emph{resources=None}, \emph{data\_model=None}, \emph{interaction\_type=None}, \emph{via=None}, \emph{references=None}}{}
Only the undirected interactions will be considered, if any resource
annotates this interaction as undirected and none as directed, the
interacting pair as a sorted tuple will be returned inside a one
element tuple.

{\color{red}\bfseries{}**}kwargs: see the docs of method \sphinxcode{\sphinxupquote{get\_interactions}}.

\end{fulllineitems}

\index{interactions\_non\_directed\_0\_by\_interaction\_type\_and\_data\_model() (pypath.core.interaction.Interaction method)@\spxentry{interactions\_non\_directed\_0\_by\_interaction\_type\_and\_data\_model()}\spxextra{pypath.core.interaction.Interaction method}}

\begin{fulllineitems}
\phantomsection\label{\detokenize{reference:pypath.core.interaction.Interaction.interactions_non_directed_0_by_interaction_type_and_data_model}}\pysiglinewithargsret{\sphinxbfcode{\sphinxupquote{interactions\_non\_directed\_0\_by\_interaction\_type\_and\_data\_model}}}{\emph{effect=None}, \emph{resources=None}, \emph{data\_model=None}, \emph{interaction\_type=None}, \emph{via=None}, \emph{references=None}}{}
Only the undirected interactions will be considered, if any resource
annotates this interaction as undirected and none as directed, the
interacting pair as a sorted tuple will be returned inside a one
element tuple.

{\color{red}\bfseries{}**}kwargs: see the docs of method \sphinxcode{\sphinxupquote{get\_interactions}}.

\end{fulllineitems}

\index{interactions\_non\_directed\_0\_by\_interaction\_type\_and\_data\_model\_and\_resource() (pypath.core.interaction.Interaction method)@\spxentry{interactions\_non\_directed\_0\_by\_interaction\_type\_and\_data\_model\_and\_resource()}\spxextra{pypath.core.interaction.Interaction method}}

\begin{fulllineitems}
\phantomsection\label{\detokenize{reference:pypath.core.interaction.Interaction.interactions_non_directed_0_by_interaction_type_and_data_model_and_resource}}\pysiglinewithargsret{\sphinxbfcode{\sphinxupquote{interactions\_non\_directed\_0\_by\_interaction\_type\_and\_data\_model\_and\_resource}}}{\emph{effect=None}, \emph{resources=None}, \emph{data\_model=None}, \emph{interaction\_type=None}, \emph{via=None}, \emph{references=None}}{}
Only the undirected interactions will be considered, if any resource
annotates this interaction as undirected and none as directed, the
interacting pair as a sorted tuple will be returned inside a one
element tuple.

{\color{red}\bfseries{}**}kwargs: see the docs of method \sphinxcode{\sphinxupquote{get\_interactions}}.

\end{fulllineitems}

\index{interactions\_non\_directed\_0\_by\_reference() (pypath.core.interaction.Interaction method)@\spxentry{interactions\_non\_directed\_0\_by\_reference()}\spxextra{pypath.core.interaction.Interaction method}}

\begin{fulllineitems}
\phantomsection\label{\detokenize{reference:pypath.core.interaction.Interaction.interactions_non_directed_0_by_reference}}\pysiglinewithargsret{\sphinxbfcode{\sphinxupquote{interactions\_non\_directed\_0\_by\_reference}}}{\emph{effect=None}, \emph{resources=None}, \emph{data\_model=None}, \emph{interaction\_type=None}, \emph{via=None}, \emph{references=None}}{}
Only the undirected interactions will be considered, if any resource
annotates this interaction as undirected and none as directed, the
interacting pair as a sorted tuple will be returned inside a one
element tuple.

{\color{red}\bfseries{}**}kwargs: see the docs of method \sphinxcode{\sphinxupquote{get\_interactions}}.

\end{fulllineitems}

\index{interactions\_non\_directed\_0\_by\_resource() (pypath.core.interaction.Interaction method)@\spxentry{interactions\_non\_directed\_0\_by\_resource()}\spxextra{pypath.core.interaction.Interaction method}}

\begin{fulllineitems}
\phantomsection\label{\detokenize{reference:pypath.core.interaction.Interaction.interactions_non_directed_0_by_resource}}\pysiglinewithargsret{\sphinxbfcode{\sphinxupquote{interactions\_non\_directed\_0\_by\_resource}}}{\emph{effect=None}, \emph{resources=None}, \emph{data\_model=None}, \emph{interaction\_type=None}, \emph{via=None}, \emph{references=None}}{}
Only the undirected interactions will be considered, if any resource
annotates this interaction as undirected and none as directed, the
interacting pair as a sorted tuple will be returned inside a one
element tuple.

{\color{red}\bfseries{}**}kwargs: see the docs of method \sphinxcode{\sphinxupquote{get\_interactions}}.

\end{fulllineitems}

\index{interactions\_non\_directed\_by\_data\_model() (pypath.core.interaction.Interaction method)@\spxentry{interactions\_non\_directed\_by\_data\_model()}\spxextra{pypath.core.interaction.Interaction method}}

\begin{fulllineitems}
\phantomsection\label{\detokenize{reference:pypath.core.interaction.Interaction.interactions_non_directed_by_data_model}}\pysiglinewithargsret{\sphinxbfcode{\sphinxupquote{interactions\_non\_directed\_by\_data\_model}}}{\emph{effect=None}, \emph{resources=None}, \emph{data\_model=None}, \emph{interaction\_type=None}, \emph{via=None}, \emph{references=None}}{}
Only the undirected interactions will be considered, if any resource
annotates this interaction as undirected both directions will be
returned, but only if no resource provide direction. However
the \sphinxcode{\sphinxupquote{count\_interactions\_non\_directed}} method will return \sphinxtitleref{1} in
this case.

{\color{red}\bfseries{}**}kwargs: see the docs of method \sphinxcode{\sphinxupquote{get\_interactions}}.

\end{fulllineitems}

\index{interactions\_non\_directed\_by\_interaction\_type() (pypath.core.interaction.Interaction method)@\spxentry{interactions\_non\_directed\_by\_interaction\_type()}\spxextra{pypath.core.interaction.Interaction method}}

\begin{fulllineitems}
\phantomsection\label{\detokenize{reference:pypath.core.interaction.Interaction.interactions_non_directed_by_interaction_type}}\pysiglinewithargsret{\sphinxbfcode{\sphinxupquote{interactions\_non\_directed\_by\_interaction\_type}}}{\emph{effect=None}, \emph{resources=None}, \emph{data\_model=None}, \emph{interaction\_type=None}, \emph{via=None}, \emph{references=None}}{}
Only the undirected interactions will be considered, if any resource
annotates this interaction as undirected both directions will be
returned, but only if no resource provide direction. However
the \sphinxcode{\sphinxupquote{count\_interactions\_non\_directed}} method will return \sphinxtitleref{1} in
this case.

{\color{red}\bfseries{}**}kwargs: see the docs of method \sphinxcode{\sphinxupquote{get\_interactions}}.

\end{fulllineitems}

\index{interactions\_non\_directed\_by\_interaction\_type\_and\_data\_model() (pypath.core.interaction.Interaction method)@\spxentry{interactions\_non\_directed\_by\_interaction\_type\_and\_data\_model()}\spxextra{pypath.core.interaction.Interaction method}}

\begin{fulllineitems}
\phantomsection\label{\detokenize{reference:pypath.core.interaction.Interaction.interactions_non_directed_by_interaction_type_and_data_model}}\pysiglinewithargsret{\sphinxbfcode{\sphinxupquote{interactions\_non\_directed\_by\_interaction\_type\_and\_data\_model}}}{\emph{effect=None}, \emph{resources=None}, \emph{data\_model=None}, \emph{interaction\_type=None}, \emph{via=None}, \emph{references=None}}{}
Only the undirected interactions will be considered, if any resource
annotates this interaction as undirected both directions will be
returned, but only if no resource provide direction. However
the \sphinxcode{\sphinxupquote{count\_interactions\_non\_directed}} method will return \sphinxtitleref{1} in
this case.

{\color{red}\bfseries{}**}kwargs: see the docs of method \sphinxcode{\sphinxupquote{get\_interactions}}.

\end{fulllineitems}

\index{interactions\_non\_directed\_by\_interaction\_type\_and\_data\_model\_and\_resource() (pypath.core.interaction.Interaction method)@\spxentry{interactions\_non\_directed\_by\_interaction\_type\_and\_data\_model\_and\_resource()}\spxextra{pypath.core.interaction.Interaction method}}

\begin{fulllineitems}
\phantomsection\label{\detokenize{reference:pypath.core.interaction.Interaction.interactions_non_directed_by_interaction_type_and_data_model_and_resource}}\pysiglinewithargsret{\sphinxbfcode{\sphinxupquote{interactions\_non\_directed\_by\_interaction\_type\_and\_data\_model\_and\_resource}}}{\emph{effect=None}, \emph{resources=None}, \emph{data\_model=None}, \emph{interaction\_type=None}, \emph{via=None}, \emph{references=None}}{}
Only the undirected interactions will be considered, if any resource
annotates this interaction as undirected both directions will be
returned, but only if no resource provide direction. However
the \sphinxcode{\sphinxupquote{count\_interactions\_non\_directed}} method will return \sphinxtitleref{1} in
this case.

{\color{red}\bfseries{}**}kwargs: see the docs of method \sphinxcode{\sphinxupquote{get\_interactions}}.

\end{fulllineitems}

\index{interactions\_non\_directed\_by\_reference() (pypath.core.interaction.Interaction method)@\spxentry{interactions\_non\_directed\_by\_reference()}\spxextra{pypath.core.interaction.Interaction method}}

\begin{fulllineitems}
\phantomsection\label{\detokenize{reference:pypath.core.interaction.Interaction.interactions_non_directed_by_reference}}\pysiglinewithargsret{\sphinxbfcode{\sphinxupquote{interactions\_non\_directed\_by\_reference}}}{\emph{effect=None}, \emph{resources=None}, \emph{data\_model=None}, \emph{interaction\_type=None}, \emph{via=None}, \emph{references=None}}{}
Only the undirected interactions will be considered, if any resource
annotates this interaction as undirected both directions will be
returned, but only if no resource provide direction. However
the \sphinxcode{\sphinxupquote{count\_interactions\_non\_directed}} method will return \sphinxtitleref{1} in
this case.

{\color{red}\bfseries{}**}kwargs: see the docs of method \sphinxcode{\sphinxupquote{get\_interactions}}.

\end{fulllineitems}

\index{interactions\_non\_directed\_by\_resource() (pypath.core.interaction.Interaction method)@\spxentry{interactions\_non\_directed\_by\_resource()}\spxextra{pypath.core.interaction.Interaction method}}

\begin{fulllineitems}
\phantomsection\label{\detokenize{reference:pypath.core.interaction.Interaction.interactions_non_directed_by_resource}}\pysiglinewithargsret{\sphinxbfcode{\sphinxupquote{interactions\_non\_directed\_by\_resource}}}{\emph{effect=None}, \emph{resources=None}, \emph{data\_model=None}, \emph{interaction\_type=None}, \emph{via=None}, \emph{references=None}}{}
Only the undirected interactions will be considered, if any resource
annotates this interaction as undirected both directions will be
returned, but only if no resource provide direction. However
the \sphinxcode{\sphinxupquote{count\_interactions\_non\_directed}} method will return \sphinxtitleref{1} in
this case.

{\color{red}\bfseries{}**}kwargs: see the docs of method \sphinxcode{\sphinxupquote{get\_interactions}}.

\end{fulllineitems}

\index{interactions\_positive\_by\_data\_model() (pypath.core.interaction.Interaction method)@\spxentry{interactions\_positive\_by\_data\_model()}\spxextra{pypath.core.interaction.Interaction method}}

\begin{fulllineitems}
\phantomsection\label{\detokenize{reference:pypath.core.interaction.Interaction.interactions_positive_by_data_model}}\pysiglinewithargsret{\sphinxbfcode{\sphinxupquote{interactions\_positive\_by\_data\_model}}}{\emph{effect=None}, \emph{resources=None}, \emph{data\_model=None}, \emph{interaction\_type=None}, \emph{via=None}, \emph{references=None}}{}
{\color{red}\bfseries{}**}kwargs: see the docs of method \sphinxcode{\sphinxupquote{get\_interactions}}.

\end{fulllineitems}

\index{interactions\_positive\_by\_interaction\_type() (pypath.core.interaction.Interaction method)@\spxentry{interactions\_positive\_by\_interaction\_type()}\spxextra{pypath.core.interaction.Interaction method}}

\begin{fulllineitems}
\phantomsection\label{\detokenize{reference:pypath.core.interaction.Interaction.interactions_positive_by_interaction_type}}\pysiglinewithargsret{\sphinxbfcode{\sphinxupquote{interactions\_positive\_by\_interaction\_type}}}{\emph{effect=None}, \emph{resources=None}, \emph{data\_model=None}, \emph{interaction\_type=None}, \emph{via=None}, \emph{references=None}}{}
{\color{red}\bfseries{}**}kwargs: see the docs of method \sphinxcode{\sphinxupquote{get\_interactions}}.

\end{fulllineitems}

\index{interactions\_positive\_by\_interaction\_type\_and\_data\_model() (pypath.core.interaction.Interaction method)@\spxentry{interactions\_positive\_by\_interaction\_type\_and\_data\_model()}\spxextra{pypath.core.interaction.Interaction method}}

\begin{fulllineitems}
\phantomsection\label{\detokenize{reference:pypath.core.interaction.Interaction.interactions_positive_by_interaction_type_and_data_model}}\pysiglinewithargsret{\sphinxbfcode{\sphinxupquote{interactions\_positive\_by\_interaction\_type\_and\_data\_model}}}{\emph{effect=None}, \emph{resources=None}, \emph{data\_model=None}, \emph{interaction\_type=None}, \emph{via=None}, \emph{references=None}}{}
{\color{red}\bfseries{}**}kwargs: see the docs of method \sphinxcode{\sphinxupquote{get\_interactions}}.

\end{fulllineitems}

\index{interactions\_positive\_by\_interaction\_type\_and\_data\_model\_and\_resource() (pypath.core.interaction.Interaction method)@\spxentry{interactions\_positive\_by\_interaction\_type\_and\_data\_model\_and\_resource()}\spxextra{pypath.core.interaction.Interaction method}}

\begin{fulllineitems}
\phantomsection\label{\detokenize{reference:pypath.core.interaction.Interaction.interactions_positive_by_interaction_type_and_data_model_and_resource}}\pysiglinewithargsret{\sphinxbfcode{\sphinxupquote{interactions\_positive\_by\_interaction\_type\_and\_data\_model\_and\_resource}}}{\emph{effect=None}, \emph{resources=None}, \emph{data\_model=None}, \emph{interaction\_type=None}, \emph{via=None}, \emph{references=None}}{}
{\color{red}\bfseries{}**}kwargs: see the docs of method \sphinxcode{\sphinxupquote{get\_interactions}}.

\end{fulllineitems}

\index{interactions\_positive\_by\_reference() (pypath.core.interaction.Interaction method)@\spxentry{interactions\_positive\_by\_reference()}\spxextra{pypath.core.interaction.Interaction method}}

\begin{fulllineitems}
\phantomsection\label{\detokenize{reference:pypath.core.interaction.Interaction.interactions_positive_by_reference}}\pysiglinewithargsret{\sphinxbfcode{\sphinxupquote{interactions\_positive\_by\_reference}}}{\emph{effect=None}, \emph{resources=None}, \emph{data\_model=None}, \emph{interaction\_type=None}, \emph{via=None}, \emph{references=None}}{}
{\color{red}\bfseries{}**}kwargs: see the docs of method \sphinxcode{\sphinxupquote{get\_interactions}}.

\end{fulllineitems}

\index{interactions\_positive\_by\_resource() (pypath.core.interaction.Interaction method)@\spxentry{interactions\_positive\_by\_resource()}\spxextra{pypath.core.interaction.Interaction method}}

\begin{fulllineitems}
\phantomsection\label{\detokenize{reference:pypath.core.interaction.Interaction.interactions_positive_by_resource}}\pysiglinewithargsret{\sphinxbfcode{\sphinxupquote{interactions\_positive\_by\_resource}}}{\emph{effect=None}, \emph{resources=None}, \emph{data\_model=None}, \emph{interaction\_type=None}, \emph{via=None}, \emph{references=None}}{}
{\color{red}\bfseries{}**}kwargs: see the docs of method \sphinxcode{\sphinxupquote{get\_interactions}}.

\end{fulllineitems}

\index{interactions\_signed\_by\_data\_model() (pypath.core.interaction.Interaction method)@\spxentry{interactions\_signed\_by\_data\_model()}\spxextra{pypath.core.interaction.Interaction method}}

\begin{fulllineitems}
\phantomsection\label{\detokenize{reference:pypath.core.interaction.Interaction.interactions_signed_by_data_model}}\pysiglinewithargsret{\sphinxbfcode{\sphinxupquote{interactions\_signed\_by\_data\_model}}}{\emph{effect=None}, \emph{resources=None}, \emph{data\_model=None}, \emph{interaction\_type=None}, \emph{via=None}, \emph{references=None}}{}
{\color{red}\bfseries{}**}kwargs: see the docs of method \sphinxcode{\sphinxupquote{get\_interactions}}.

\end{fulllineitems}

\index{interactions\_signed\_by\_interaction\_type() (pypath.core.interaction.Interaction method)@\spxentry{interactions\_signed\_by\_interaction\_type()}\spxextra{pypath.core.interaction.Interaction method}}

\begin{fulllineitems}
\phantomsection\label{\detokenize{reference:pypath.core.interaction.Interaction.interactions_signed_by_interaction_type}}\pysiglinewithargsret{\sphinxbfcode{\sphinxupquote{interactions\_signed\_by\_interaction\_type}}}{\emph{effect=None}, \emph{resources=None}, \emph{data\_model=None}, \emph{interaction\_type=None}, \emph{via=None}, \emph{references=None}}{}
{\color{red}\bfseries{}**}kwargs: see the docs of method \sphinxcode{\sphinxupquote{get\_interactions}}.

\end{fulllineitems}

\index{interactions\_signed\_by\_interaction\_type\_and\_data\_model() (pypath.core.interaction.Interaction method)@\spxentry{interactions\_signed\_by\_interaction\_type\_and\_data\_model()}\spxextra{pypath.core.interaction.Interaction method}}

\begin{fulllineitems}
\phantomsection\label{\detokenize{reference:pypath.core.interaction.Interaction.interactions_signed_by_interaction_type_and_data_model}}\pysiglinewithargsret{\sphinxbfcode{\sphinxupquote{interactions\_signed\_by\_interaction\_type\_and\_data\_model}}}{\emph{effect=None}, \emph{resources=None}, \emph{data\_model=None}, \emph{interaction\_type=None}, \emph{via=None}, \emph{references=None}}{}
{\color{red}\bfseries{}**}kwargs: see the docs of method \sphinxcode{\sphinxupquote{get\_interactions}}.

\end{fulllineitems}

\index{interactions\_signed\_by\_interaction\_type\_and\_data\_model\_and\_resource() (pypath.core.interaction.Interaction method)@\spxentry{interactions\_signed\_by\_interaction\_type\_and\_data\_model\_and\_resource()}\spxextra{pypath.core.interaction.Interaction method}}

\begin{fulllineitems}
\phantomsection\label{\detokenize{reference:pypath.core.interaction.Interaction.interactions_signed_by_interaction_type_and_data_model_and_resource}}\pysiglinewithargsret{\sphinxbfcode{\sphinxupquote{interactions\_signed\_by\_interaction\_type\_and\_data\_model\_and\_resource}}}{\emph{effect=None}, \emph{resources=None}, \emph{data\_model=None}, \emph{interaction\_type=None}, \emph{via=None}, \emph{references=None}}{}
{\color{red}\bfseries{}**}kwargs: see the docs of method \sphinxcode{\sphinxupquote{get\_interactions}}.

\end{fulllineitems}

\index{interactions\_signed\_by\_reference() (pypath.core.interaction.Interaction method)@\spxentry{interactions\_signed\_by\_reference()}\spxextra{pypath.core.interaction.Interaction method}}

\begin{fulllineitems}
\phantomsection\label{\detokenize{reference:pypath.core.interaction.Interaction.interactions_signed_by_reference}}\pysiglinewithargsret{\sphinxbfcode{\sphinxupquote{interactions\_signed\_by\_reference}}}{\emph{effect=None}, \emph{resources=None}, \emph{data\_model=None}, \emph{interaction\_type=None}, \emph{via=None}, \emph{references=None}}{}
{\color{red}\bfseries{}**}kwargs: see the docs of method \sphinxcode{\sphinxupquote{get\_interactions}}.

\end{fulllineitems}

\index{interactions\_signed\_by\_resource() (pypath.core.interaction.Interaction method)@\spxentry{interactions\_signed\_by\_resource()}\spxextra{pypath.core.interaction.Interaction method}}

\begin{fulllineitems}
\phantomsection\label{\detokenize{reference:pypath.core.interaction.Interaction.interactions_signed_by_resource}}\pysiglinewithargsret{\sphinxbfcode{\sphinxupquote{interactions\_signed\_by\_resource}}}{\emph{effect=None}, \emph{resources=None}, \emph{data\_model=None}, \emph{interaction\_type=None}, \emph{via=None}, \emph{references=None}}{}
{\color{red}\bfseries{}**}kwargs: see the docs of method \sphinxcode{\sphinxupquote{get\_interactions}}.

\end{fulllineitems}

\index{interactions\_undirected\_0\_by\_data\_model() (pypath.core.interaction.Interaction method)@\spxentry{interactions\_undirected\_0\_by\_data\_model()}\spxextra{pypath.core.interaction.Interaction method}}

\begin{fulllineitems}
\phantomsection\label{\detokenize{reference:pypath.core.interaction.Interaction.interactions_undirected_0_by_data_model}}\pysiglinewithargsret{\sphinxbfcode{\sphinxupquote{interactions\_undirected\_0\_by\_data\_model}}}{\emph{effect=None}, \emph{resources=None}, \emph{data\_model=None}, \emph{interaction\_type=None}, \emph{via=None}, \emph{references=None}}{}
Only the undirected interactions will be considered, if any resource
annotates this interaction as undirected the interacting pair as
a sorted tuple will be returned inside a one element tuple.

{\color{red}\bfseries{}**}kwargs: see the docs of method \sphinxcode{\sphinxupquote{get\_interactions}}.

\end{fulllineitems}

\index{interactions\_undirected\_0\_by\_interaction\_type() (pypath.core.interaction.Interaction method)@\spxentry{interactions\_undirected\_0\_by\_interaction\_type()}\spxextra{pypath.core.interaction.Interaction method}}

\begin{fulllineitems}
\phantomsection\label{\detokenize{reference:pypath.core.interaction.Interaction.interactions_undirected_0_by_interaction_type}}\pysiglinewithargsret{\sphinxbfcode{\sphinxupquote{interactions\_undirected\_0\_by\_interaction\_type}}}{\emph{effect=None}, \emph{resources=None}, \emph{data\_model=None}, \emph{interaction\_type=None}, \emph{via=None}, \emph{references=None}}{}
Only the undirected interactions will be considered, if any resource
annotates this interaction as undirected the interacting pair as
a sorted tuple will be returned inside a one element tuple.

{\color{red}\bfseries{}**}kwargs: see the docs of method \sphinxcode{\sphinxupquote{get\_interactions}}.

\end{fulllineitems}

\index{interactions\_undirected\_0\_by\_interaction\_type\_and\_data\_model() (pypath.core.interaction.Interaction method)@\spxentry{interactions\_undirected\_0\_by\_interaction\_type\_and\_data\_model()}\spxextra{pypath.core.interaction.Interaction method}}

\begin{fulllineitems}
\phantomsection\label{\detokenize{reference:pypath.core.interaction.Interaction.interactions_undirected_0_by_interaction_type_and_data_model}}\pysiglinewithargsret{\sphinxbfcode{\sphinxupquote{interactions\_undirected\_0\_by\_interaction\_type\_and\_data\_model}}}{\emph{effect=None}, \emph{resources=None}, \emph{data\_model=None}, \emph{interaction\_type=None}, \emph{via=None}, \emph{references=None}}{}
Only the undirected interactions will be considered, if any resource
annotates this interaction as undirected the interacting pair as
a sorted tuple will be returned inside a one element tuple.

{\color{red}\bfseries{}**}kwargs: see the docs of method \sphinxcode{\sphinxupquote{get\_interactions}}.

\end{fulllineitems}

\index{interactions\_undirected\_0\_by\_interaction\_type\_and\_data\_model\_and\_resource() (pypath.core.interaction.Interaction method)@\spxentry{interactions\_undirected\_0\_by\_interaction\_type\_and\_data\_model\_and\_resource()}\spxextra{pypath.core.interaction.Interaction method}}

\begin{fulllineitems}
\phantomsection\label{\detokenize{reference:pypath.core.interaction.Interaction.interactions_undirected_0_by_interaction_type_and_data_model_and_resource}}\pysiglinewithargsret{\sphinxbfcode{\sphinxupquote{interactions\_undirected\_0\_by\_interaction\_type\_and\_data\_model\_and\_resource}}}{\emph{effect=None}, \emph{resources=None}, \emph{data\_model=None}, \emph{interaction\_type=None}, \emph{via=None}, \emph{references=None}}{}
Only the undirected interactions will be considered, if any resource
annotates this interaction as undirected the interacting pair as
a sorted tuple will be returned inside a one element tuple.

{\color{red}\bfseries{}**}kwargs: see the docs of method \sphinxcode{\sphinxupquote{get\_interactions}}.

\end{fulllineitems}

\index{interactions\_undirected\_0\_by\_reference() (pypath.core.interaction.Interaction method)@\spxentry{interactions\_undirected\_0\_by\_reference()}\spxextra{pypath.core.interaction.Interaction method}}

\begin{fulllineitems}
\phantomsection\label{\detokenize{reference:pypath.core.interaction.Interaction.interactions_undirected_0_by_reference}}\pysiglinewithargsret{\sphinxbfcode{\sphinxupquote{interactions\_undirected\_0\_by\_reference}}}{\emph{effect=None}, \emph{resources=None}, \emph{data\_model=None}, \emph{interaction\_type=None}, \emph{via=None}, \emph{references=None}}{}
Only the undirected interactions will be considered, if any resource
annotates this interaction as undirected the interacting pair as
a sorted tuple will be returned inside a one element tuple.

{\color{red}\bfseries{}**}kwargs: see the docs of method \sphinxcode{\sphinxupquote{get\_interactions}}.

\end{fulllineitems}

\index{interactions\_undirected\_0\_by\_resource() (pypath.core.interaction.Interaction method)@\spxentry{interactions\_undirected\_0\_by\_resource()}\spxextra{pypath.core.interaction.Interaction method}}

\begin{fulllineitems}
\phantomsection\label{\detokenize{reference:pypath.core.interaction.Interaction.interactions_undirected_0_by_resource}}\pysiglinewithargsret{\sphinxbfcode{\sphinxupquote{interactions\_undirected\_0\_by\_resource}}}{\emph{effect=None}, \emph{resources=None}, \emph{data\_model=None}, \emph{interaction\_type=None}, \emph{via=None}, \emph{references=None}}{}
Only the undirected interactions will be considered, if any resource
annotates this interaction as undirected the interacting pair as
a sorted tuple will be returned inside a one element tuple.

{\color{red}\bfseries{}**}kwargs: see the docs of method \sphinxcode{\sphinxupquote{get\_interactions}}.

\end{fulllineitems}

\index{interactions\_undirected\_by\_data\_model() (pypath.core.interaction.Interaction method)@\spxentry{interactions\_undirected\_by\_data\_model()}\spxextra{pypath.core.interaction.Interaction method}}

\begin{fulllineitems}
\phantomsection\label{\detokenize{reference:pypath.core.interaction.Interaction.interactions_undirected_by_data_model}}\pysiglinewithargsret{\sphinxbfcode{\sphinxupquote{interactions\_undirected\_by\_data\_model}}}{\emph{effect=None}, \emph{resources=None}, \emph{data\_model=None}, \emph{interaction\_type=None}, \emph{via=None}, \emph{references=None}}{}
Only the undirected interactions will be considered, if any resource
annotates this interaction as undirected both directions will be
returned, no matter if certain resources provide direction. However
the \sphinxcode{\sphinxupquote{count\_interactions\_undirected}} method will return \sphinxtitleref{1} in this
case.

{\color{red}\bfseries{}**}kwargs: see the docs of method \sphinxcode{\sphinxupquote{get\_interactions}}.

\end{fulllineitems}

\index{interactions\_undirected\_by\_interaction\_type() (pypath.core.interaction.Interaction method)@\spxentry{interactions\_undirected\_by\_interaction\_type()}\spxextra{pypath.core.interaction.Interaction method}}

\begin{fulllineitems}
\phantomsection\label{\detokenize{reference:pypath.core.interaction.Interaction.interactions_undirected_by_interaction_type}}\pysiglinewithargsret{\sphinxbfcode{\sphinxupquote{interactions\_undirected\_by\_interaction\_type}}}{\emph{effect=None}, \emph{resources=None}, \emph{data\_model=None}, \emph{interaction\_type=None}, \emph{via=None}, \emph{references=None}}{}
Only the undirected interactions will be considered, if any resource
annotates this interaction as undirected both directions will be
returned, no matter if certain resources provide direction. However
the \sphinxcode{\sphinxupquote{count\_interactions\_undirected}} method will return \sphinxtitleref{1} in this
case.

{\color{red}\bfseries{}**}kwargs: see the docs of method \sphinxcode{\sphinxupquote{get\_interactions}}.

\end{fulllineitems}

\index{interactions\_undirected\_by\_interaction\_type\_and\_data\_model() (pypath.core.interaction.Interaction method)@\spxentry{interactions\_undirected\_by\_interaction\_type\_and\_data\_model()}\spxextra{pypath.core.interaction.Interaction method}}

\begin{fulllineitems}
\phantomsection\label{\detokenize{reference:pypath.core.interaction.Interaction.interactions_undirected_by_interaction_type_and_data_model}}\pysiglinewithargsret{\sphinxbfcode{\sphinxupquote{interactions\_undirected\_by\_interaction\_type\_and\_data\_model}}}{\emph{effect=None}, \emph{resources=None}, \emph{data\_model=None}, \emph{interaction\_type=None}, \emph{via=None}, \emph{references=None}}{}
Only the undirected interactions will be considered, if any resource
annotates this interaction as undirected both directions will be
returned, no matter if certain resources provide direction. However
the \sphinxcode{\sphinxupquote{count\_interactions\_undirected}} method will return \sphinxtitleref{1} in this
case.

{\color{red}\bfseries{}**}kwargs: see the docs of method \sphinxcode{\sphinxupquote{get\_interactions}}.

\end{fulllineitems}

\index{interactions\_undirected\_by\_interaction\_type\_and\_data\_model\_and\_resource() (pypath.core.interaction.Interaction method)@\spxentry{interactions\_undirected\_by\_interaction\_type\_and\_data\_model\_and\_resource()}\spxextra{pypath.core.interaction.Interaction method}}

\begin{fulllineitems}
\phantomsection\label{\detokenize{reference:pypath.core.interaction.Interaction.interactions_undirected_by_interaction_type_and_data_model_and_resource}}\pysiglinewithargsret{\sphinxbfcode{\sphinxupquote{interactions\_undirected\_by\_interaction\_type\_and\_data\_model\_and\_resource}}}{\emph{effect=None}, \emph{resources=None}, \emph{data\_model=None}, \emph{interaction\_type=None}, \emph{via=None}, \emph{references=None}}{}
Only the undirected interactions will be considered, if any resource
annotates this interaction as undirected both directions will be
returned, no matter if certain resources provide direction. However
the \sphinxcode{\sphinxupquote{count\_interactions\_undirected}} method will return \sphinxtitleref{1} in this
case.

{\color{red}\bfseries{}**}kwargs: see the docs of method \sphinxcode{\sphinxupquote{get\_interactions}}.

\end{fulllineitems}

\index{interactions\_undirected\_by\_reference() (pypath.core.interaction.Interaction method)@\spxentry{interactions\_undirected\_by\_reference()}\spxextra{pypath.core.interaction.Interaction method}}

\begin{fulllineitems}
\phantomsection\label{\detokenize{reference:pypath.core.interaction.Interaction.interactions_undirected_by_reference}}\pysiglinewithargsret{\sphinxbfcode{\sphinxupquote{interactions\_undirected\_by\_reference}}}{\emph{effect=None}, \emph{resources=None}, \emph{data\_model=None}, \emph{interaction\_type=None}, \emph{via=None}, \emph{references=None}}{}
Only the undirected interactions will be considered, if any resource
annotates this interaction as undirected both directions will be
returned, no matter if certain resources provide direction. However
the \sphinxcode{\sphinxupquote{count\_interactions\_undirected}} method will return \sphinxtitleref{1} in this
case.

{\color{red}\bfseries{}**}kwargs: see the docs of method \sphinxcode{\sphinxupquote{get\_interactions}}.

\end{fulllineitems}

\index{interactions\_undirected\_by\_resource() (pypath.core.interaction.Interaction method)@\spxentry{interactions\_undirected\_by\_resource()}\spxextra{pypath.core.interaction.Interaction method}}

\begin{fulllineitems}
\phantomsection\label{\detokenize{reference:pypath.core.interaction.Interaction.interactions_undirected_by_resource}}\pysiglinewithargsret{\sphinxbfcode{\sphinxupquote{interactions\_undirected\_by\_resource}}}{\emph{effect=None}, \emph{resources=None}, \emph{data\_model=None}, \emph{interaction\_type=None}, \emph{via=None}, \emph{references=None}}{}
Only the undirected interactions will be considered, if any resource
annotates this interaction as undirected both directions will be
returned, no matter if certain resources provide direction. However
the \sphinxcode{\sphinxupquote{count\_interactions\_undirected}} method will return \sphinxtitleref{1} in this
case.

{\color{red}\bfseries{}**}kwargs: see the docs of method \sphinxcode{\sphinxupquote{get\_interactions}}.

\end{fulllineitems}

\index{is\_directed() (pypath.core.interaction.Interaction method)@\spxentry{is\_directed()}\spxextra{pypath.core.interaction.Interaction method}}

\begin{fulllineitems}
\phantomsection\label{\detokenize{reference:pypath.core.interaction.Interaction.is_directed}}\pysiglinewithargsret{\sphinxbfcode{\sphinxupquote{is\_directed}}}{}{}
Checks if edge has any directionality information.
\begin{quote}\begin{description}
\item[{Returns}] \leavevmode
(\sphinxstyleemphasis{bool}) \textendash{} Returns \sphinxcode{\sphinxupquote{True}} if any of the \sphinxcode{\sphinxupquote{dirs}}
attribute values is \sphinxcode{\sphinxupquote{True}} (except \sphinxcode{\sphinxupquote{'undirected'}}),
\sphinxcode{\sphinxupquote{False}} otherwise.

\end{description}\end{quote}

\end{fulllineitems}

\index{is\_directed\_by\_resources() (pypath.core.interaction.Interaction method)@\spxentry{is\_directed\_by\_resources()}\spxextra{pypath.core.interaction.Interaction method}}

\begin{fulllineitems}
\phantomsection\label{\detokenize{reference:pypath.core.interaction.Interaction.is_directed_by_resources}}\pysiglinewithargsret{\sphinxbfcode{\sphinxupquote{is\_directed\_by\_resources}}}{\emph{resources=None}}{}
Checks if edge has any directionality information from some
resource(s).
\begin{quote}\begin{description}
\item[{Returns}] \leavevmode
(\sphinxstyleemphasis{bool}) \textendash{} Returns \sphinxcode{\sphinxupquote{True}} if any of the \sphinxcode{\sphinxupquote{dirs}}
attribute values is \sphinxcode{\sphinxupquote{True}} (except \sphinxcode{\sphinxupquote{'undirected'}}),
\sphinxcode{\sphinxupquote{False}} otherwise.

\end{description}\end{quote}

\end{fulllineitems}

\index{is\_inhibition() (pypath.core.interaction.Interaction method)@\spxentry{is\_inhibition()}\spxextra{pypath.core.interaction.Interaction method}}

\begin{fulllineitems}
\phantomsection\label{\detokenize{reference:pypath.core.interaction.Interaction.is_inhibition}}\pysiglinewithargsret{\sphinxbfcode{\sphinxupquote{is\_inhibition}}}{\emph{direction=None}, \emph{resources=None}}{}
Checks if any (or for a specific \sphinxstyleemphasis{direction}) interaction is
inhibition (negative interaction).
\begin{quote}\begin{description}
\item[{Parameters}] \leavevmode
\sphinxstyleliteralstrong{\sphinxupquote{direction}} (\sphinxstyleliteralemphasis{\sphinxupquote{tuple}}) \textendash{} Optional, \sphinxcode{\sphinxupquote{None}} by default. If specified, checks the
\sphinxcode{\sphinxupquote{negative}} attribute of that specific
directionality. If not specified, checks both.

\item[{Returns}] \leavevmode
(\sphinxstyleemphasis{bool}) \textendash{} \sphinxcode{\sphinxupquote{True}} if any interaction (or the specified
\sphinxstyleemphasis{direction}) is inhibitory (negative).

\end{description}\end{quote}

\end{fulllineitems}

\index{is\_loop() (pypath.core.interaction.Interaction method)@\spxentry{is\_loop()}\spxextra{pypath.core.interaction.Interaction method}}

\begin{fulllineitems}
\phantomsection\label{\detokenize{reference:pypath.core.interaction.Interaction.is_loop}}\pysiglinewithargsret{\sphinxbfcode{\sphinxupquote{is\_loop}}}{}{}~\begin{quote}\begin{description}
\item[{Returns}] \leavevmode


\end{description}\end{quote}

\sphinxcode{\sphinxupquote{True}} if the interaction is a loop edge i.e. its endpoints are the
same node.

\end{fulllineitems}

\index{is\_mutual() (pypath.core.interaction.Interaction method)@\spxentry{is\_mutual()}\spxextra{pypath.core.interaction.Interaction method}}

\begin{fulllineitems}
\phantomsection\label{\detokenize{reference:pypath.core.interaction.Interaction.is_mutual}}\pysiglinewithargsret{\sphinxbfcode{\sphinxupquote{is\_mutual}}}{\emph{**kwargs}}{}
Note: undirected interactions does not count as mutual but only
interactions with explicit direction information for both directions.

{\color{red}\bfseries{}**}kwargs: see the docs of method \sphinxcode{\sphinxupquote{get\_interactions}}.

\end{fulllineitems}

\index{is\_mutual\_by\_resources() (pypath.core.interaction.Interaction method)@\spxentry{is\_mutual\_by\_resources()}\spxextra{pypath.core.interaction.Interaction method}}

\begin{fulllineitems}
\phantomsection\label{\detokenize{reference:pypath.core.interaction.Interaction.is_mutual_by_resources}}\pysiglinewithargsret{\sphinxbfcode{\sphinxupquote{is\_mutual\_by\_resources}}}{\emph{resources=None}}{}
Checks if the edge has mutual directions (both A\textendash{}\textgreater{}B and B\textendash{}\textgreater{}A)
according to some resource(s).

\end{fulllineitems}

\index{is\_stimulation() (pypath.core.interaction.Interaction method)@\spxentry{is\_stimulation()}\spxextra{pypath.core.interaction.Interaction method}}

\begin{fulllineitems}
\phantomsection\label{\detokenize{reference:pypath.core.interaction.Interaction.is_stimulation}}\pysiglinewithargsret{\sphinxbfcode{\sphinxupquote{is\_stimulation}}}{\emph{direction=None}, \emph{resources=None}}{}
Checks if any (or for a specific \sphinxstyleemphasis{direction}) interaction is
activation (positive interaction).
\begin{quote}\begin{description}
\item[{Parameters}] \leavevmode
\sphinxstyleliteralstrong{\sphinxupquote{direction}} (\sphinxstyleliteralemphasis{\sphinxupquote{tuple}}) \textendash{} Optional, \sphinxcode{\sphinxupquote{None}} by default. If specified, checks the
\sphinxcode{\sphinxupquote{positive}} attribute of that specific
directionality. If not specified, checks both.

\item[{Returns}] \leavevmode
(\sphinxstyleemphasis{bool}) \textendash{} \sphinxcode{\sphinxupquote{True}} if any interaction (or the specified
\sphinxstyleemphasis{direction}) is activatory (positive).

\end{description}\end{quote}

\end{fulllineitems}

\index{iter\_evidences() (pypath.core.interaction.Interaction method)@\spxentry{iter\_evidences()}\spxextra{pypath.core.interaction.Interaction method}}

\begin{fulllineitems}
\phantomsection\label{\detokenize{reference:pypath.core.interaction.Interaction.iter_evidences}}\pysiglinewithargsret{\sphinxbfcode{\sphinxupquote{iter\_evidences}}}{\emph{this\_direction}, \emph{direction=None}, \emph{effect=None}}{}
Selects and yields evidence collections matching the direction and
effect criteria.

\end{fulllineitems}

\index{iter\_match\_evidences() (pypath.core.interaction.Interaction method)@\spxentry{iter\_match\_evidences()}\spxextra{pypath.core.interaction.Interaction method}}

\begin{fulllineitems}
\phantomsection\label{\detokenize{reference:pypath.core.interaction.Interaction.iter_match_evidences}}\pysiglinewithargsret{\sphinxbfcode{\sphinxupquote{iter\_match\_evidences}}}{\emph{this\_direction}, \emph{direction=None}, \emph{effect=None}, \emph{resources=None}, \emph{data\_model=None}, \emph{interaction\_type=None}, \emph{via=None}, \emph{references=None}}{}
Selects the evidence collections matching the direction and effect
criteria and yields collections matching the evidence criteria.

\end{fulllineitems}

\index{labels\_by\_data\_model() (pypath.core.interaction.Interaction method)@\spxentry{labels\_by\_data\_model()}\spxextra{pypath.core.interaction.Interaction method}}

\begin{fulllineitems}
\phantomsection\label{\detokenize{reference:pypath.core.interaction.Interaction.labels_by_data_model}}\pysiglinewithargsret{\sphinxbfcode{\sphinxupquote{labels\_by\_data\_model}}}{\emph{effect=None}, \emph{resources=None}, \emph{data\_model=None}, \emph{interaction\_type=None}, \emph{via=None}, \emph{references=None}}{}
Retrieves the entities involved in interactions matching the criteria.
It either returns both interacting entities in a \sphinxstyleemphasis{set} or an empty
\sphinxstyleemphasis{set}. This may not sound so useful at the level of this object but
becomes more useful once we want to collect entities having certain
kind of interactions across a series of \sphinxtitleref{Interaction} objects.
\begin{quote}\begin{description}
\item[{Parameters}] \leavevmode\begin{itemize}
\item {} 
\sphinxstyleliteralstrong{\sphinxupquote{entity\_type}} (\sphinxstyleliteralemphasis{\sphinxupquote{str}}) \textendash{} The type of the molecular entity. Possible values: \sphinxtitleref{protein},
\sphinxtitleref{complex}, \sphinxtitleref{mirna}, \sphinxtitleref{small\_molecule}.

\item {} 
\sphinxstyleliteralstrong{\sphinxupquote{return\_type}} (\sphinxstyleliteralemphasis{\sphinxupquote{str}}) \textendash{} The type of values to return. Default is
py:class:\sphinxcode{\sphinxupquote{pypath.entity.Entity}} objects, alternatives are
\sphinxcode{\sphinxupquote{labels}}  \sphinxcode{\sphinxupquote{identifiers}}.

\end{itemize}

\end{description}\end{quote}

\end{fulllineitems}

\index{labels\_by\_interaction\_type() (pypath.core.interaction.Interaction method)@\spxentry{labels\_by\_interaction\_type()}\spxextra{pypath.core.interaction.Interaction method}}

\begin{fulllineitems}
\phantomsection\label{\detokenize{reference:pypath.core.interaction.Interaction.labels_by_interaction_type}}\pysiglinewithargsret{\sphinxbfcode{\sphinxupquote{labels\_by\_interaction\_type}}}{\emph{effect=None}, \emph{resources=None}, \emph{data\_model=None}, \emph{interaction\_type=None}, \emph{via=None}, \emph{references=None}}{}
Retrieves the entities involved in interactions matching the criteria.
It either returns both interacting entities in a \sphinxstyleemphasis{set} or an empty
\sphinxstyleemphasis{set}. This may not sound so useful at the level of this object but
becomes more useful once we want to collect entities having certain
kind of interactions across a series of \sphinxtitleref{Interaction} objects.
\begin{quote}\begin{description}
\item[{Parameters}] \leavevmode\begin{itemize}
\item {} 
\sphinxstyleliteralstrong{\sphinxupquote{entity\_type}} (\sphinxstyleliteralemphasis{\sphinxupquote{str}}) \textendash{} The type of the molecular entity. Possible values: \sphinxtitleref{protein},
\sphinxtitleref{complex}, \sphinxtitleref{mirna}, \sphinxtitleref{small\_molecule}.

\item {} 
\sphinxstyleliteralstrong{\sphinxupquote{return\_type}} (\sphinxstyleliteralemphasis{\sphinxupquote{str}}) \textendash{} The type of values to return. Default is
py:class:\sphinxcode{\sphinxupquote{pypath.entity.Entity}} objects, alternatives are
\sphinxcode{\sphinxupquote{labels}}  \sphinxcode{\sphinxupquote{identifiers}}.

\end{itemize}

\end{description}\end{quote}

\end{fulllineitems}

\index{labels\_by\_interaction\_type\_and\_data\_model() (pypath.core.interaction.Interaction method)@\spxentry{labels\_by\_interaction\_type\_and\_data\_model()}\spxextra{pypath.core.interaction.Interaction method}}

\begin{fulllineitems}
\phantomsection\label{\detokenize{reference:pypath.core.interaction.Interaction.labels_by_interaction_type_and_data_model}}\pysiglinewithargsret{\sphinxbfcode{\sphinxupquote{labels\_by\_interaction\_type\_and\_data\_model}}}{\emph{effect=None}, \emph{resources=None}, \emph{data\_model=None}, \emph{interaction\_type=None}, \emph{via=None}, \emph{references=None}}{}
Retrieves the entities involved in interactions matching the criteria.
It either returns both interacting entities in a \sphinxstyleemphasis{set} or an empty
\sphinxstyleemphasis{set}. This may not sound so useful at the level of this object but
becomes more useful once we want to collect entities having certain
kind of interactions across a series of \sphinxtitleref{Interaction} objects.
\begin{quote}\begin{description}
\item[{Parameters}] \leavevmode\begin{itemize}
\item {} 
\sphinxstyleliteralstrong{\sphinxupquote{entity\_type}} (\sphinxstyleliteralemphasis{\sphinxupquote{str}}) \textendash{} The type of the molecular entity. Possible values: \sphinxtitleref{protein},
\sphinxtitleref{complex}, \sphinxtitleref{mirna}, \sphinxtitleref{small\_molecule}.

\item {} 
\sphinxstyleliteralstrong{\sphinxupquote{return\_type}} (\sphinxstyleliteralemphasis{\sphinxupquote{str}}) \textendash{} The type of values to return. Default is
py:class:\sphinxcode{\sphinxupquote{pypath.entity.Entity}} objects, alternatives are
\sphinxcode{\sphinxupquote{labels}}  \sphinxcode{\sphinxupquote{identifiers}}.

\end{itemize}

\end{description}\end{quote}

\end{fulllineitems}

\index{labels\_by\_interaction\_type\_and\_data\_model\_and\_resource() (pypath.core.interaction.Interaction method)@\spxentry{labels\_by\_interaction\_type\_and\_data\_model\_and\_resource()}\spxextra{pypath.core.interaction.Interaction method}}

\begin{fulllineitems}
\phantomsection\label{\detokenize{reference:pypath.core.interaction.Interaction.labels_by_interaction_type_and_data_model_and_resource}}\pysiglinewithargsret{\sphinxbfcode{\sphinxupquote{labels\_by\_interaction\_type\_and\_data\_model\_and\_resource}}}{\emph{effect=None}, \emph{resources=None}, \emph{data\_model=None}, \emph{interaction\_type=None}, \emph{via=None}, \emph{references=None}}{}
Retrieves the entities involved in interactions matching the criteria.
It either returns both interacting entities in a \sphinxstyleemphasis{set} or an empty
\sphinxstyleemphasis{set}. This may not sound so useful at the level of this object but
becomes more useful once we want to collect entities having certain
kind of interactions across a series of \sphinxtitleref{Interaction} objects.
\begin{quote}\begin{description}
\item[{Parameters}] \leavevmode\begin{itemize}
\item {} 
\sphinxstyleliteralstrong{\sphinxupquote{entity\_type}} (\sphinxstyleliteralemphasis{\sphinxupquote{str}}) \textendash{} The type of the molecular entity. Possible values: \sphinxtitleref{protein},
\sphinxtitleref{complex}, \sphinxtitleref{mirna}, \sphinxtitleref{small\_molecule}.

\item {} 
\sphinxstyleliteralstrong{\sphinxupquote{return\_type}} (\sphinxstyleliteralemphasis{\sphinxupquote{str}}) \textendash{} The type of values to return. Default is
py:class:\sphinxcode{\sphinxupquote{pypath.entity.Entity}} objects, alternatives are
\sphinxcode{\sphinxupquote{labels}}  \sphinxcode{\sphinxupquote{identifiers}}.

\end{itemize}

\end{description}\end{quote}

\end{fulllineitems}

\index{labels\_by\_reference() (pypath.core.interaction.Interaction method)@\spxentry{labels\_by\_reference()}\spxextra{pypath.core.interaction.Interaction method}}

\begin{fulllineitems}
\phantomsection\label{\detokenize{reference:pypath.core.interaction.Interaction.labels_by_reference}}\pysiglinewithargsret{\sphinxbfcode{\sphinxupquote{labels\_by\_reference}}}{\emph{effect=None}, \emph{resources=None}, \emph{data\_model=None}, \emph{interaction\_type=None}, \emph{via=None}, \emph{references=None}}{}
Retrieves the entities involved in interactions matching the criteria.
It either returns both interacting entities in a \sphinxstyleemphasis{set} or an empty
\sphinxstyleemphasis{set}. This may not sound so useful at the level of this object but
becomes more useful once we want to collect entities having certain
kind of interactions across a series of \sphinxtitleref{Interaction} objects.
\begin{quote}\begin{description}
\item[{Parameters}] \leavevmode\begin{itemize}
\item {} 
\sphinxstyleliteralstrong{\sphinxupquote{entity\_type}} (\sphinxstyleliteralemphasis{\sphinxupquote{str}}) \textendash{} The type of the molecular entity. Possible values: \sphinxtitleref{protein},
\sphinxtitleref{complex}, \sphinxtitleref{mirna}, \sphinxtitleref{small\_molecule}.

\item {} 
\sphinxstyleliteralstrong{\sphinxupquote{return\_type}} (\sphinxstyleliteralemphasis{\sphinxupquote{str}}) \textendash{} The type of values to return. Default is
py:class:\sphinxcode{\sphinxupquote{pypath.entity.Entity}} objects, alternatives are
\sphinxcode{\sphinxupquote{labels}}  \sphinxcode{\sphinxupquote{identifiers}}.

\end{itemize}

\end{description}\end{quote}

\end{fulllineitems}

\index{labels\_by\_resource() (pypath.core.interaction.Interaction method)@\spxentry{labels\_by\_resource()}\spxextra{pypath.core.interaction.Interaction method}}

\begin{fulllineitems}
\phantomsection\label{\detokenize{reference:pypath.core.interaction.Interaction.labels_by_resource}}\pysiglinewithargsret{\sphinxbfcode{\sphinxupquote{labels\_by\_resource}}}{\emph{effect=None}, \emph{resources=None}, \emph{data\_model=None}, \emph{interaction\_type=None}, \emph{via=None}, \emph{references=None}}{}
Retrieves the entities involved in interactions matching the criteria.
It either returns both interacting entities in a \sphinxstyleemphasis{set} or an empty
\sphinxstyleemphasis{set}. This may not sound so useful at the level of this object but
becomes more useful once we want to collect entities having certain
kind of interactions across a series of \sphinxtitleref{Interaction} objects.
\begin{quote}\begin{description}
\item[{Parameters}] \leavevmode\begin{itemize}
\item {} 
\sphinxstyleliteralstrong{\sphinxupquote{entity\_type}} (\sphinxstyleliteralemphasis{\sphinxupquote{str}}) \textendash{} The type of the molecular entity. Possible values: \sphinxtitleref{protein},
\sphinxtitleref{complex}, \sphinxtitleref{mirna}, \sphinxtitleref{small\_molecule}.

\item {} 
\sphinxstyleliteralstrong{\sphinxupquote{return\_type}} (\sphinxstyleliteralemphasis{\sphinxupquote{str}}) \textendash{} The type of values to return. Default is
py:class:\sphinxcode{\sphinxupquote{pypath.entity.Entity}} objects, alternatives are
\sphinxcode{\sphinxupquote{labels}}  \sphinxcode{\sphinxupquote{identifiers}}.

\end{itemize}

\end{description}\end{quote}

\end{fulllineitems}

\index{lncrna\_identifiers\_by\_data\_model() (pypath.core.interaction.Interaction method)@\spxentry{lncrna\_identifiers\_by\_data\_model()}\spxextra{pypath.core.interaction.Interaction method}}

\begin{fulllineitems}
\phantomsection\label{\detokenize{reference:pypath.core.interaction.Interaction.lncrna_identifiers_by_data_model}}\pysiglinewithargsret{\sphinxbfcode{\sphinxupquote{lncrna\_identifiers\_by\_data\_model}}}{\emph{effect=None}, \emph{resources=None}, \emph{data\_model=None}, \emph{interaction\_type=None}, \emph{via=None}, \emph{references=None}}{}
Retrieves the entities involved in interactions matching the criteria.
It either returns both interacting entities in a \sphinxstyleemphasis{set} or an empty
\sphinxstyleemphasis{set}. This may not sound so useful at the level of this object but
becomes more useful once we want to collect entities having certain
kind of interactions across a series of \sphinxtitleref{Interaction} objects.
\begin{quote}\begin{description}
\item[{Parameters}] \leavevmode\begin{itemize}
\item {} 
\sphinxstyleliteralstrong{\sphinxupquote{entity\_type}} (\sphinxstyleliteralemphasis{\sphinxupquote{str}}) \textendash{} The type of the molecular entity. Possible values: \sphinxtitleref{protein},
\sphinxtitleref{complex}, \sphinxtitleref{mirna}, \sphinxtitleref{small\_molecule}.

\item {} 
\sphinxstyleliteralstrong{\sphinxupquote{return\_type}} (\sphinxstyleliteralemphasis{\sphinxupquote{str}}) \textendash{} The type of values to return. Default is
py:class:\sphinxcode{\sphinxupquote{pypath.entity.Entity}} objects, alternatives are
\sphinxcode{\sphinxupquote{labels}}  \sphinxcode{\sphinxupquote{identifiers}}.

\end{itemize}

\end{description}\end{quote}

\end{fulllineitems}

\index{lncrna\_identifiers\_by\_interaction\_type() (pypath.core.interaction.Interaction method)@\spxentry{lncrna\_identifiers\_by\_interaction\_type()}\spxextra{pypath.core.interaction.Interaction method}}

\begin{fulllineitems}
\phantomsection\label{\detokenize{reference:pypath.core.interaction.Interaction.lncrna_identifiers_by_interaction_type}}\pysiglinewithargsret{\sphinxbfcode{\sphinxupquote{lncrna\_identifiers\_by\_interaction\_type}}}{\emph{effect=None}, \emph{resources=None}, \emph{data\_model=None}, \emph{interaction\_type=None}, \emph{via=None}, \emph{references=None}}{}
Retrieves the entities involved in interactions matching the criteria.
It either returns both interacting entities in a \sphinxstyleemphasis{set} or an empty
\sphinxstyleemphasis{set}. This may not sound so useful at the level of this object but
becomes more useful once we want to collect entities having certain
kind of interactions across a series of \sphinxtitleref{Interaction} objects.
\begin{quote}\begin{description}
\item[{Parameters}] \leavevmode\begin{itemize}
\item {} 
\sphinxstyleliteralstrong{\sphinxupquote{entity\_type}} (\sphinxstyleliteralemphasis{\sphinxupquote{str}}) \textendash{} The type of the molecular entity. Possible values: \sphinxtitleref{protein},
\sphinxtitleref{complex}, \sphinxtitleref{mirna}, \sphinxtitleref{small\_molecule}.

\item {} 
\sphinxstyleliteralstrong{\sphinxupquote{return\_type}} (\sphinxstyleliteralemphasis{\sphinxupquote{str}}) \textendash{} The type of values to return. Default is
py:class:\sphinxcode{\sphinxupquote{pypath.entity.Entity}} objects, alternatives are
\sphinxcode{\sphinxupquote{labels}}  \sphinxcode{\sphinxupquote{identifiers}}.

\end{itemize}

\end{description}\end{quote}

\end{fulllineitems}

\index{lncrna\_identifiers\_by\_interaction\_type\_and\_data\_model() (pypath.core.interaction.Interaction method)@\spxentry{lncrna\_identifiers\_by\_interaction\_type\_and\_data\_model()}\spxextra{pypath.core.interaction.Interaction method}}

\begin{fulllineitems}
\phantomsection\label{\detokenize{reference:pypath.core.interaction.Interaction.lncrna_identifiers_by_interaction_type_and_data_model}}\pysiglinewithargsret{\sphinxbfcode{\sphinxupquote{lncrna\_identifiers\_by\_interaction\_type\_and\_data\_model}}}{\emph{effect=None}, \emph{resources=None}, \emph{data\_model=None}, \emph{interaction\_type=None}, \emph{via=None}, \emph{references=None}}{}
Retrieves the entities involved in interactions matching the criteria.
It either returns both interacting entities in a \sphinxstyleemphasis{set} or an empty
\sphinxstyleemphasis{set}. This may not sound so useful at the level of this object but
becomes more useful once we want to collect entities having certain
kind of interactions across a series of \sphinxtitleref{Interaction} objects.
\begin{quote}\begin{description}
\item[{Parameters}] \leavevmode\begin{itemize}
\item {} 
\sphinxstyleliteralstrong{\sphinxupquote{entity\_type}} (\sphinxstyleliteralemphasis{\sphinxupquote{str}}) \textendash{} The type of the molecular entity. Possible values: \sphinxtitleref{protein},
\sphinxtitleref{complex}, \sphinxtitleref{mirna}, \sphinxtitleref{small\_molecule}.

\item {} 
\sphinxstyleliteralstrong{\sphinxupquote{return\_type}} (\sphinxstyleliteralemphasis{\sphinxupquote{str}}) \textendash{} The type of values to return. Default is
py:class:\sphinxcode{\sphinxupquote{pypath.entity.Entity}} objects, alternatives are
\sphinxcode{\sphinxupquote{labels}}  \sphinxcode{\sphinxupquote{identifiers}}.

\end{itemize}

\end{description}\end{quote}

\end{fulllineitems}

\index{lncrna\_identifiers\_by\_interaction\_type\_and\_data\_model\_and\_resource() (pypath.core.interaction.Interaction method)@\spxentry{lncrna\_identifiers\_by\_interaction\_type\_and\_data\_model\_and\_resource()}\spxextra{pypath.core.interaction.Interaction method}}

\begin{fulllineitems}
\phantomsection\label{\detokenize{reference:pypath.core.interaction.Interaction.lncrna_identifiers_by_interaction_type_and_data_model_and_resource}}\pysiglinewithargsret{\sphinxbfcode{\sphinxupquote{lncrna\_identifiers\_by\_interaction\_type\_and\_data\_model\_and\_resource}}}{\emph{effect=None}, \emph{resources=None}, \emph{data\_model=None}, \emph{interaction\_type=None}, \emph{via=None}, \emph{references=None}}{}
Retrieves the entities involved in interactions matching the criteria.
It either returns both interacting entities in a \sphinxstyleemphasis{set} or an empty
\sphinxstyleemphasis{set}. This may not sound so useful at the level of this object but
becomes more useful once we want to collect entities having certain
kind of interactions across a series of \sphinxtitleref{Interaction} objects.
\begin{quote}\begin{description}
\item[{Parameters}] \leavevmode\begin{itemize}
\item {} 
\sphinxstyleliteralstrong{\sphinxupquote{entity\_type}} (\sphinxstyleliteralemphasis{\sphinxupquote{str}}) \textendash{} The type of the molecular entity. Possible values: \sphinxtitleref{protein},
\sphinxtitleref{complex}, \sphinxtitleref{mirna}, \sphinxtitleref{small\_molecule}.

\item {} 
\sphinxstyleliteralstrong{\sphinxupquote{return\_type}} (\sphinxstyleliteralemphasis{\sphinxupquote{str}}) \textendash{} The type of values to return. Default is
py:class:\sphinxcode{\sphinxupquote{pypath.entity.Entity}} objects, alternatives are
\sphinxcode{\sphinxupquote{labels}}  \sphinxcode{\sphinxupquote{identifiers}}.

\end{itemize}

\end{description}\end{quote}

\end{fulllineitems}

\index{lncrna\_identifiers\_by\_reference() (pypath.core.interaction.Interaction method)@\spxentry{lncrna\_identifiers\_by\_reference()}\spxextra{pypath.core.interaction.Interaction method}}

\begin{fulllineitems}
\phantomsection\label{\detokenize{reference:pypath.core.interaction.Interaction.lncrna_identifiers_by_reference}}\pysiglinewithargsret{\sphinxbfcode{\sphinxupquote{lncrna\_identifiers\_by\_reference}}}{\emph{effect=None}, \emph{resources=None}, \emph{data\_model=None}, \emph{interaction\_type=None}, \emph{via=None}, \emph{references=None}}{}
Retrieves the entities involved in interactions matching the criteria.
It either returns both interacting entities in a \sphinxstyleemphasis{set} or an empty
\sphinxstyleemphasis{set}. This may not sound so useful at the level of this object but
becomes more useful once we want to collect entities having certain
kind of interactions across a series of \sphinxtitleref{Interaction} objects.
\begin{quote}\begin{description}
\item[{Parameters}] \leavevmode\begin{itemize}
\item {} 
\sphinxstyleliteralstrong{\sphinxupquote{entity\_type}} (\sphinxstyleliteralemphasis{\sphinxupquote{str}}) \textendash{} The type of the molecular entity. Possible values: \sphinxtitleref{protein},
\sphinxtitleref{complex}, \sphinxtitleref{mirna}, \sphinxtitleref{small\_molecule}.

\item {} 
\sphinxstyleliteralstrong{\sphinxupquote{return\_type}} (\sphinxstyleliteralemphasis{\sphinxupquote{str}}) \textendash{} The type of values to return. Default is
py:class:\sphinxcode{\sphinxupquote{pypath.entity.Entity}} objects, alternatives are
\sphinxcode{\sphinxupquote{labels}}  \sphinxcode{\sphinxupquote{identifiers}}.

\end{itemize}

\end{description}\end{quote}

\end{fulllineitems}

\index{lncrna\_identifiers\_by\_resource() (pypath.core.interaction.Interaction method)@\spxentry{lncrna\_identifiers\_by\_resource()}\spxextra{pypath.core.interaction.Interaction method}}

\begin{fulllineitems}
\phantomsection\label{\detokenize{reference:pypath.core.interaction.Interaction.lncrna_identifiers_by_resource}}\pysiglinewithargsret{\sphinxbfcode{\sphinxupquote{lncrna\_identifiers\_by\_resource}}}{\emph{effect=None}, \emph{resources=None}, \emph{data\_model=None}, \emph{interaction\_type=None}, \emph{via=None}, \emph{references=None}}{}
Retrieves the entities involved in interactions matching the criteria.
It either returns both interacting entities in a \sphinxstyleemphasis{set} or an empty
\sphinxstyleemphasis{set}. This may not sound so useful at the level of this object but
becomes more useful once we want to collect entities having certain
kind of interactions across a series of \sphinxtitleref{Interaction} objects.
\begin{quote}\begin{description}
\item[{Parameters}] \leavevmode\begin{itemize}
\item {} 
\sphinxstyleliteralstrong{\sphinxupquote{entity\_type}} (\sphinxstyleliteralemphasis{\sphinxupquote{str}}) \textendash{} The type of the molecular entity. Possible values: \sphinxtitleref{protein},
\sphinxtitleref{complex}, \sphinxtitleref{mirna}, \sphinxtitleref{small\_molecule}.

\item {} 
\sphinxstyleliteralstrong{\sphinxupquote{return\_type}} (\sphinxstyleliteralemphasis{\sphinxupquote{str}}) \textendash{} The type of values to return. Default is
py:class:\sphinxcode{\sphinxupquote{pypath.entity.Entity}} objects, alternatives are
\sphinxcode{\sphinxupquote{labels}}  \sphinxcode{\sphinxupquote{identifiers}}.

\end{itemize}

\end{description}\end{quote}

\end{fulllineitems}

\index{lncrna\_labels\_by\_data\_model() (pypath.core.interaction.Interaction method)@\spxentry{lncrna\_labels\_by\_data\_model()}\spxextra{pypath.core.interaction.Interaction method}}

\begin{fulllineitems}
\phantomsection\label{\detokenize{reference:pypath.core.interaction.Interaction.lncrna_labels_by_data_model}}\pysiglinewithargsret{\sphinxbfcode{\sphinxupquote{lncrna\_labels\_by\_data\_model}}}{\emph{effect=None}, \emph{resources=None}, \emph{data\_model=None}, \emph{interaction\_type=None}, \emph{via=None}, \emph{references=None}}{}
Retrieves the entities involved in interactions matching the criteria.
It either returns both interacting entities in a \sphinxstyleemphasis{set} or an empty
\sphinxstyleemphasis{set}. This may not sound so useful at the level of this object but
becomes more useful once we want to collect entities having certain
kind of interactions across a series of \sphinxtitleref{Interaction} objects.
\begin{quote}\begin{description}
\item[{Parameters}] \leavevmode\begin{itemize}
\item {} 
\sphinxstyleliteralstrong{\sphinxupquote{entity\_type}} (\sphinxstyleliteralemphasis{\sphinxupquote{str}}) \textendash{} The type of the molecular entity. Possible values: \sphinxtitleref{protein},
\sphinxtitleref{complex}, \sphinxtitleref{mirna}, \sphinxtitleref{small\_molecule}.

\item {} 
\sphinxstyleliteralstrong{\sphinxupquote{return\_type}} (\sphinxstyleliteralemphasis{\sphinxupquote{str}}) \textendash{} The type of values to return. Default is
py:class:\sphinxcode{\sphinxupquote{pypath.entity.Entity}} objects, alternatives are
\sphinxcode{\sphinxupquote{labels}}  \sphinxcode{\sphinxupquote{identifiers}}.

\end{itemize}

\end{description}\end{quote}

\end{fulllineitems}

\index{lncrna\_labels\_by\_interaction\_type() (pypath.core.interaction.Interaction method)@\spxentry{lncrna\_labels\_by\_interaction\_type()}\spxextra{pypath.core.interaction.Interaction method}}

\begin{fulllineitems}
\phantomsection\label{\detokenize{reference:pypath.core.interaction.Interaction.lncrna_labels_by_interaction_type}}\pysiglinewithargsret{\sphinxbfcode{\sphinxupquote{lncrna\_labels\_by\_interaction\_type}}}{\emph{effect=None}, \emph{resources=None}, \emph{data\_model=None}, \emph{interaction\_type=None}, \emph{via=None}, \emph{references=None}}{}
Retrieves the entities involved in interactions matching the criteria.
It either returns both interacting entities in a \sphinxstyleemphasis{set} or an empty
\sphinxstyleemphasis{set}. This may not sound so useful at the level of this object but
becomes more useful once we want to collect entities having certain
kind of interactions across a series of \sphinxtitleref{Interaction} objects.
\begin{quote}\begin{description}
\item[{Parameters}] \leavevmode\begin{itemize}
\item {} 
\sphinxstyleliteralstrong{\sphinxupquote{entity\_type}} (\sphinxstyleliteralemphasis{\sphinxupquote{str}}) \textendash{} The type of the molecular entity. Possible values: \sphinxtitleref{protein},
\sphinxtitleref{complex}, \sphinxtitleref{mirna}, \sphinxtitleref{small\_molecule}.

\item {} 
\sphinxstyleliteralstrong{\sphinxupquote{return\_type}} (\sphinxstyleliteralemphasis{\sphinxupquote{str}}) \textendash{} The type of values to return. Default is
py:class:\sphinxcode{\sphinxupquote{pypath.entity.Entity}} objects, alternatives are
\sphinxcode{\sphinxupquote{labels}}  \sphinxcode{\sphinxupquote{identifiers}}.

\end{itemize}

\end{description}\end{quote}

\end{fulllineitems}

\index{lncrna\_labels\_by\_interaction\_type\_and\_data\_model() (pypath.core.interaction.Interaction method)@\spxentry{lncrna\_labels\_by\_interaction\_type\_and\_data\_model()}\spxextra{pypath.core.interaction.Interaction method}}

\begin{fulllineitems}
\phantomsection\label{\detokenize{reference:pypath.core.interaction.Interaction.lncrna_labels_by_interaction_type_and_data_model}}\pysiglinewithargsret{\sphinxbfcode{\sphinxupquote{lncrna\_labels\_by\_interaction\_type\_and\_data\_model}}}{\emph{effect=None}, \emph{resources=None}, \emph{data\_model=None}, \emph{interaction\_type=None}, \emph{via=None}, \emph{references=None}}{}
Retrieves the entities involved in interactions matching the criteria.
It either returns both interacting entities in a \sphinxstyleemphasis{set} or an empty
\sphinxstyleemphasis{set}. This may not sound so useful at the level of this object but
becomes more useful once we want to collect entities having certain
kind of interactions across a series of \sphinxtitleref{Interaction} objects.
\begin{quote}\begin{description}
\item[{Parameters}] \leavevmode\begin{itemize}
\item {} 
\sphinxstyleliteralstrong{\sphinxupquote{entity\_type}} (\sphinxstyleliteralemphasis{\sphinxupquote{str}}) \textendash{} The type of the molecular entity. Possible values: \sphinxtitleref{protein},
\sphinxtitleref{complex}, \sphinxtitleref{mirna}, \sphinxtitleref{small\_molecule}.

\item {} 
\sphinxstyleliteralstrong{\sphinxupquote{return\_type}} (\sphinxstyleliteralemphasis{\sphinxupquote{str}}) \textendash{} The type of values to return. Default is
py:class:\sphinxcode{\sphinxupquote{pypath.entity.Entity}} objects, alternatives are
\sphinxcode{\sphinxupquote{labels}}  \sphinxcode{\sphinxupquote{identifiers}}.

\end{itemize}

\end{description}\end{quote}

\end{fulllineitems}

\index{lncrna\_labels\_by\_interaction\_type\_and\_data\_model\_and\_resource() (pypath.core.interaction.Interaction method)@\spxentry{lncrna\_labels\_by\_interaction\_type\_and\_data\_model\_and\_resource()}\spxextra{pypath.core.interaction.Interaction method}}

\begin{fulllineitems}
\phantomsection\label{\detokenize{reference:pypath.core.interaction.Interaction.lncrna_labels_by_interaction_type_and_data_model_and_resource}}\pysiglinewithargsret{\sphinxbfcode{\sphinxupquote{lncrna\_labels\_by\_interaction\_type\_and\_data\_model\_and\_resource}}}{\emph{effect=None}, \emph{resources=None}, \emph{data\_model=None}, \emph{interaction\_type=None}, \emph{via=None}, \emph{references=None}}{}
Retrieves the entities involved in interactions matching the criteria.
It either returns both interacting entities in a \sphinxstyleemphasis{set} or an empty
\sphinxstyleemphasis{set}. This may not sound so useful at the level of this object but
becomes more useful once we want to collect entities having certain
kind of interactions across a series of \sphinxtitleref{Interaction} objects.
\begin{quote}\begin{description}
\item[{Parameters}] \leavevmode\begin{itemize}
\item {} 
\sphinxstyleliteralstrong{\sphinxupquote{entity\_type}} (\sphinxstyleliteralemphasis{\sphinxupquote{str}}) \textendash{} The type of the molecular entity. Possible values: \sphinxtitleref{protein},
\sphinxtitleref{complex}, \sphinxtitleref{mirna}, \sphinxtitleref{small\_molecule}.

\item {} 
\sphinxstyleliteralstrong{\sphinxupquote{return\_type}} (\sphinxstyleliteralemphasis{\sphinxupquote{str}}) \textendash{} The type of values to return. Default is
py:class:\sphinxcode{\sphinxupquote{pypath.entity.Entity}} objects, alternatives are
\sphinxcode{\sphinxupquote{labels}}  \sphinxcode{\sphinxupquote{identifiers}}.

\end{itemize}

\end{description}\end{quote}

\end{fulllineitems}

\index{lncrna\_labels\_by\_reference() (pypath.core.interaction.Interaction method)@\spxentry{lncrna\_labels\_by\_reference()}\spxextra{pypath.core.interaction.Interaction method}}

\begin{fulllineitems}
\phantomsection\label{\detokenize{reference:pypath.core.interaction.Interaction.lncrna_labels_by_reference}}\pysiglinewithargsret{\sphinxbfcode{\sphinxupquote{lncrna\_labels\_by\_reference}}}{\emph{effect=None}, \emph{resources=None}, \emph{data\_model=None}, \emph{interaction\_type=None}, \emph{via=None}, \emph{references=None}}{}
Retrieves the entities involved in interactions matching the criteria.
It either returns both interacting entities in a \sphinxstyleemphasis{set} or an empty
\sphinxstyleemphasis{set}. This may not sound so useful at the level of this object but
becomes more useful once we want to collect entities having certain
kind of interactions across a series of \sphinxtitleref{Interaction} objects.
\begin{quote}\begin{description}
\item[{Parameters}] \leavevmode\begin{itemize}
\item {} 
\sphinxstyleliteralstrong{\sphinxupquote{entity\_type}} (\sphinxstyleliteralemphasis{\sphinxupquote{str}}) \textendash{} The type of the molecular entity. Possible values: \sphinxtitleref{protein},
\sphinxtitleref{complex}, \sphinxtitleref{mirna}, \sphinxtitleref{small\_molecule}.

\item {} 
\sphinxstyleliteralstrong{\sphinxupquote{return\_type}} (\sphinxstyleliteralemphasis{\sphinxupquote{str}}) \textendash{} The type of values to return. Default is
py:class:\sphinxcode{\sphinxupquote{pypath.entity.Entity}} objects, alternatives are
\sphinxcode{\sphinxupquote{labels}}  \sphinxcode{\sphinxupquote{identifiers}}.

\end{itemize}

\end{description}\end{quote}

\end{fulllineitems}

\index{lncrna\_labels\_by\_resource() (pypath.core.interaction.Interaction method)@\spxentry{lncrna\_labels\_by\_resource()}\spxextra{pypath.core.interaction.Interaction method}}

\begin{fulllineitems}
\phantomsection\label{\detokenize{reference:pypath.core.interaction.Interaction.lncrna_labels_by_resource}}\pysiglinewithargsret{\sphinxbfcode{\sphinxupquote{lncrna\_labels\_by\_resource}}}{\emph{effect=None}, \emph{resources=None}, \emph{data\_model=None}, \emph{interaction\_type=None}, \emph{via=None}, \emph{references=None}}{}
Retrieves the entities involved in interactions matching the criteria.
It either returns both interacting entities in a \sphinxstyleemphasis{set} or an empty
\sphinxstyleemphasis{set}. This may not sound so useful at the level of this object but
becomes more useful once we want to collect entities having certain
kind of interactions across a series of \sphinxtitleref{Interaction} objects.
\begin{quote}\begin{description}
\item[{Parameters}] \leavevmode\begin{itemize}
\item {} 
\sphinxstyleliteralstrong{\sphinxupquote{entity\_type}} (\sphinxstyleliteralemphasis{\sphinxupquote{str}}) \textendash{} The type of the molecular entity. Possible values: \sphinxtitleref{protein},
\sphinxtitleref{complex}, \sphinxtitleref{mirna}, \sphinxtitleref{small\_molecule}.

\item {} 
\sphinxstyleliteralstrong{\sphinxupquote{return\_type}} (\sphinxstyleliteralemphasis{\sphinxupquote{str}}) \textendash{} The type of values to return. Default is
py:class:\sphinxcode{\sphinxupquote{pypath.entity.Entity}} objects, alternatives are
\sphinxcode{\sphinxupquote{labels}}  \sphinxcode{\sphinxupquote{identifiers}}.

\end{itemize}

\end{description}\end{quote}

\end{fulllineitems}

\index{lncrnas\_by\_data\_model() (pypath.core.interaction.Interaction method)@\spxentry{lncrnas\_by\_data\_model()}\spxextra{pypath.core.interaction.Interaction method}}

\begin{fulllineitems}
\phantomsection\label{\detokenize{reference:pypath.core.interaction.Interaction.lncrnas_by_data_model}}\pysiglinewithargsret{\sphinxbfcode{\sphinxupquote{lncrnas\_by\_data\_model}}}{\emph{effect=None}, \emph{resources=None}, \emph{data\_model=None}, \emph{interaction\_type=None}, \emph{via=None}, \emph{references=None}}{}
Retrieves the entities involved in interactions matching the criteria.
It either returns both interacting entities in a \sphinxstyleemphasis{set} or an empty
\sphinxstyleemphasis{set}. This may not sound so useful at the level of this object but
becomes more useful once we want to collect entities having certain
kind of interactions across a series of \sphinxtitleref{Interaction} objects.
\begin{quote}\begin{description}
\item[{Parameters}] \leavevmode\begin{itemize}
\item {} 
\sphinxstyleliteralstrong{\sphinxupquote{entity\_type}} (\sphinxstyleliteralemphasis{\sphinxupquote{str}}) \textendash{} The type of the molecular entity. Possible values: \sphinxtitleref{protein},
\sphinxtitleref{complex}, \sphinxtitleref{mirna}, \sphinxtitleref{small\_molecule}.

\item {} 
\sphinxstyleliteralstrong{\sphinxupquote{return\_type}} (\sphinxstyleliteralemphasis{\sphinxupquote{str}}) \textendash{} The type of values to return. Default is
py:class:\sphinxcode{\sphinxupquote{pypath.entity.Entity}} objects, alternatives are
\sphinxcode{\sphinxupquote{labels}}  \sphinxcode{\sphinxupquote{identifiers}}.

\end{itemize}

\end{description}\end{quote}

\end{fulllineitems}

\index{lncrnas\_by\_interaction\_type() (pypath.core.interaction.Interaction method)@\spxentry{lncrnas\_by\_interaction\_type()}\spxextra{pypath.core.interaction.Interaction method}}

\begin{fulllineitems}
\phantomsection\label{\detokenize{reference:pypath.core.interaction.Interaction.lncrnas_by_interaction_type}}\pysiglinewithargsret{\sphinxbfcode{\sphinxupquote{lncrnas\_by\_interaction\_type}}}{\emph{effect=None}, \emph{resources=None}, \emph{data\_model=None}, \emph{interaction\_type=None}, \emph{via=None}, \emph{references=None}}{}
Retrieves the entities involved in interactions matching the criteria.
It either returns both interacting entities in a \sphinxstyleemphasis{set} or an empty
\sphinxstyleemphasis{set}. This may not sound so useful at the level of this object but
becomes more useful once we want to collect entities having certain
kind of interactions across a series of \sphinxtitleref{Interaction} objects.
\begin{quote}\begin{description}
\item[{Parameters}] \leavevmode\begin{itemize}
\item {} 
\sphinxstyleliteralstrong{\sphinxupquote{entity\_type}} (\sphinxstyleliteralemphasis{\sphinxupquote{str}}) \textendash{} The type of the molecular entity. Possible values: \sphinxtitleref{protein},
\sphinxtitleref{complex}, \sphinxtitleref{mirna}, \sphinxtitleref{small\_molecule}.

\item {} 
\sphinxstyleliteralstrong{\sphinxupquote{return\_type}} (\sphinxstyleliteralemphasis{\sphinxupquote{str}}) \textendash{} The type of values to return. Default is
py:class:\sphinxcode{\sphinxupquote{pypath.entity.Entity}} objects, alternatives are
\sphinxcode{\sphinxupquote{labels}}  \sphinxcode{\sphinxupquote{identifiers}}.

\end{itemize}

\end{description}\end{quote}

\end{fulllineitems}

\index{lncrnas\_by\_interaction\_type\_and\_data\_model() (pypath.core.interaction.Interaction method)@\spxentry{lncrnas\_by\_interaction\_type\_and\_data\_model()}\spxextra{pypath.core.interaction.Interaction method}}

\begin{fulllineitems}
\phantomsection\label{\detokenize{reference:pypath.core.interaction.Interaction.lncrnas_by_interaction_type_and_data_model}}\pysiglinewithargsret{\sphinxbfcode{\sphinxupquote{lncrnas\_by\_interaction\_type\_and\_data\_model}}}{\emph{effect=None}, \emph{resources=None}, \emph{data\_model=None}, \emph{interaction\_type=None}, \emph{via=None}, \emph{references=None}}{}
Retrieves the entities involved in interactions matching the criteria.
It either returns both interacting entities in a \sphinxstyleemphasis{set} or an empty
\sphinxstyleemphasis{set}. This may not sound so useful at the level of this object but
becomes more useful once we want to collect entities having certain
kind of interactions across a series of \sphinxtitleref{Interaction} objects.
\begin{quote}\begin{description}
\item[{Parameters}] \leavevmode\begin{itemize}
\item {} 
\sphinxstyleliteralstrong{\sphinxupquote{entity\_type}} (\sphinxstyleliteralemphasis{\sphinxupquote{str}}) \textendash{} The type of the molecular entity. Possible values: \sphinxtitleref{protein},
\sphinxtitleref{complex}, \sphinxtitleref{mirna}, \sphinxtitleref{small\_molecule}.

\item {} 
\sphinxstyleliteralstrong{\sphinxupquote{return\_type}} (\sphinxstyleliteralemphasis{\sphinxupquote{str}}) \textendash{} The type of values to return. Default is
py:class:\sphinxcode{\sphinxupquote{pypath.entity.Entity}} objects, alternatives are
\sphinxcode{\sphinxupquote{labels}}  \sphinxcode{\sphinxupquote{identifiers}}.

\end{itemize}

\end{description}\end{quote}

\end{fulllineitems}

\index{lncrnas\_by\_interaction\_type\_and\_data\_model\_and\_resource() (pypath.core.interaction.Interaction method)@\spxentry{lncrnas\_by\_interaction\_type\_and\_data\_model\_and\_resource()}\spxextra{pypath.core.interaction.Interaction method}}

\begin{fulllineitems}
\phantomsection\label{\detokenize{reference:pypath.core.interaction.Interaction.lncrnas_by_interaction_type_and_data_model_and_resource}}\pysiglinewithargsret{\sphinxbfcode{\sphinxupquote{lncrnas\_by\_interaction\_type\_and\_data\_model\_and\_resource}}}{\emph{effect=None}, \emph{resources=None}, \emph{data\_model=None}, \emph{interaction\_type=None}, \emph{via=None}, \emph{references=None}}{}
Retrieves the entities involved in interactions matching the criteria.
It either returns both interacting entities in a \sphinxstyleemphasis{set} or an empty
\sphinxstyleemphasis{set}. This may not sound so useful at the level of this object but
becomes more useful once we want to collect entities having certain
kind of interactions across a series of \sphinxtitleref{Interaction} objects.
\begin{quote}\begin{description}
\item[{Parameters}] \leavevmode\begin{itemize}
\item {} 
\sphinxstyleliteralstrong{\sphinxupquote{entity\_type}} (\sphinxstyleliteralemphasis{\sphinxupquote{str}}) \textendash{} The type of the molecular entity. Possible values: \sphinxtitleref{protein},
\sphinxtitleref{complex}, \sphinxtitleref{mirna}, \sphinxtitleref{small\_molecule}.

\item {} 
\sphinxstyleliteralstrong{\sphinxupquote{return\_type}} (\sphinxstyleliteralemphasis{\sphinxupquote{str}}) \textendash{} The type of values to return. Default is
py:class:\sphinxcode{\sphinxupquote{pypath.entity.Entity}} objects, alternatives are
\sphinxcode{\sphinxupquote{labels}}  \sphinxcode{\sphinxupquote{identifiers}}.

\end{itemize}

\end{description}\end{quote}

\end{fulllineitems}

\index{lncrnas\_by\_reference() (pypath.core.interaction.Interaction method)@\spxentry{lncrnas\_by\_reference()}\spxextra{pypath.core.interaction.Interaction method}}

\begin{fulllineitems}
\phantomsection\label{\detokenize{reference:pypath.core.interaction.Interaction.lncrnas_by_reference}}\pysiglinewithargsret{\sphinxbfcode{\sphinxupquote{lncrnas\_by\_reference}}}{\emph{effect=None}, \emph{resources=None}, \emph{data\_model=None}, \emph{interaction\_type=None}, \emph{via=None}, \emph{references=None}}{}
Retrieves the entities involved in interactions matching the criteria.
It either returns both interacting entities in a \sphinxstyleemphasis{set} or an empty
\sphinxstyleemphasis{set}. This may not sound so useful at the level of this object but
becomes more useful once we want to collect entities having certain
kind of interactions across a series of \sphinxtitleref{Interaction} objects.
\begin{quote}\begin{description}
\item[{Parameters}] \leavevmode\begin{itemize}
\item {} 
\sphinxstyleliteralstrong{\sphinxupquote{entity\_type}} (\sphinxstyleliteralemphasis{\sphinxupquote{str}}) \textendash{} The type of the molecular entity. Possible values: \sphinxtitleref{protein},
\sphinxtitleref{complex}, \sphinxtitleref{mirna}, \sphinxtitleref{small\_molecule}.

\item {} 
\sphinxstyleliteralstrong{\sphinxupquote{return\_type}} (\sphinxstyleliteralemphasis{\sphinxupquote{str}}) \textendash{} The type of values to return. Default is
py:class:\sphinxcode{\sphinxupquote{pypath.entity.Entity}} objects, alternatives are
\sphinxcode{\sphinxupquote{labels}}  \sphinxcode{\sphinxupquote{identifiers}}.

\end{itemize}

\end{description}\end{quote}

\end{fulllineitems}

\index{lncrnas\_by\_resource() (pypath.core.interaction.Interaction method)@\spxentry{lncrnas\_by\_resource()}\spxextra{pypath.core.interaction.Interaction method}}

\begin{fulllineitems}
\phantomsection\label{\detokenize{reference:pypath.core.interaction.Interaction.lncrnas_by_resource}}\pysiglinewithargsret{\sphinxbfcode{\sphinxupquote{lncrnas\_by\_resource}}}{\emph{effect=None}, \emph{resources=None}, \emph{data\_model=None}, \emph{interaction\_type=None}, \emph{via=None}, \emph{references=None}}{}
Retrieves the entities involved in interactions matching the criteria.
It either returns both interacting entities in a \sphinxstyleemphasis{set} or an empty
\sphinxstyleemphasis{set}. This may not sound so useful at the level of this object but
becomes more useful once we want to collect entities having certain
kind of interactions across a series of \sphinxtitleref{Interaction} objects.
\begin{quote}\begin{description}
\item[{Parameters}] \leavevmode\begin{itemize}
\item {} 
\sphinxstyleliteralstrong{\sphinxupquote{entity\_type}} (\sphinxstyleliteralemphasis{\sphinxupquote{str}}) \textendash{} The type of the molecular entity. Possible values: \sphinxtitleref{protein},
\sphinxtitleref{complex}, \sphinxtitleref{mirna}, \sphinxtitleref{small\_molecule}.

\item {} 
\sphinxstyleliteralstrong{\sphinxupquote{return\_type}} (\sphinxstyleliteralemphasis{\sphinxupquote{str}}) \textendash{} The type of values to return. Default is
py:class:\sphinxcode{\sphinxupquote{pypath.entity.Entity}} objects, alternatives are
\sphinxcode{\sphinxupquote{labels}}  \sphinxcode{\sphinxupquote{identifiers}}.

\end{itemize}

\end{description}\end{quote}

\end{fulllineitems}

\index{majority\_dir() (pypath.core.interaction.Interaction method)@\spxentry{majority\_dir()}\spxextra{pypath.core.interaction.Interaction method}}

\begin{fulllineitems}
\phantomsection\label{\detokenize{reference:pypath.core.interaction.Interaction.majority_dir}}\pysiglinewithargsret{\sphinxbfcode{\sphinxupquote{majority\_dir}}}{\emph{only\_interaction\_type=None}, \emph{only\_primary=True}, \emph{by\_references=False}, \emph{by\_reference\_resource\_pairs=True}}{}
Infers which is the major directionality of the edge by number
of supporting sources.
\begin{quote}\begin{description}
\item[{Returns}] \leavevmode
(\sphinxstyleemphasis{tuple}) \textendash{} Contains the pair of nodes denoting the
consensus directionality. If the number of sources on both
directions is equal, \sphinxcode{\sphinxupquote{None}} is returned. If there is no
directionality information, \sphinxcode{\sphinxupquote{'undirected'{}`}} will be
returned.

\end{description}\end{quote}

\end{fulllineitems}

\index{majority\_sign() (pypath.core.interaction.Interaction method)@\spxentry{majority\_sign()}\spxextra{pypath.core.interaction.Interaction method}}

\begin{fulllineitems}
\phantomsection\label{\detokenize{reference:pypath.core.interaction.Interaction.majority_sign}}\pysiglinewithargsret{\sphinxbfcode{\sphinxupquote{majority\_sign}}}{\emph{only\_interaction\_type=None}, \emph{only\_primary=True}, \emph{by\_references=False}, \emph{by\_reference\_resource\_pairs=True}}{}
Infers which is the major sign (activation/inhibition) of the
edge by number of supporting sources on both directions.
\begin{quote}\begin{description}
\item[{Returns}] \leavevmode
(\sphinxstyleemphasis{dict}) \textendash{} Keys are the node tuples on both directions
(\sphinxcode{\sphinxupquote{straight}}/\sphinxcode{\sphinxupquote{reverse}}) and values can be
either \sphinxcode{\sphinxupquote{None}} if that direction has no sign information or
a list of two {[}bool{]} elements corresponding to majority of
positive and majority of negative support. In case both
elements of the list are \sphinxcode{\sphinxupquote{True}}, this means the number of
supporting sources for both signs in that direction is
equal.

\end{description}\end{quote}

\end{fulllineitems}

\index{merge() (pypath.core.interaction.Interaction method)@\spxentry{merge()}\spxextra{pypath.core.interaction.Interaction method}}

\begin{fulllineitems}
\phantomsection\label{\detokenize{reference:pypath.core.interaction.Interaction.merge}}\pysiglinewithargsret{\sphinxbfcode{\sphinxupquote{merge}}}{\emph{other}}{}
Merges current Interaction with another (if and only if they are the
same class and contain the same nodes). Updates the attributes
\sphinxcode{\sphinxupquote{direction}}, \sphinxcode{\sphinxupquote{positive}} and \sphinxcode{\sphinxupquote{negative}}.
\begin{quote}\begin{description}
\item[{Parameters}] \leavevmode
\sphinxstyleliteralstrong{\sphinxupquote{other}} (\sphinxstyleliteralemphasis{\sphinxupquote{pypath.interaction.Interaction}}) \textendash{} The new Interaction object to be merged with the current one.

\end{description}\end{quote}

\end{fulllineitems}

\index{mirna\_identifiers\_by\_data\_model() (pypath.core.interaction.Interaction method)@\spxentry{mirna\_identifiers\_by\_data\_model()}\spxextra{pypath.core.interaction.Interaction method}}

\begin{fulllineitems}
\phantomsection\label{\detokenize{reference:pypath.core.interaction.Interaction.mirna_identifiers_by_data_model}}\pysiglinewithargsret{\sphinxbfcode{\sphinxupquote{mirna\_identifiers\_by\_data\_model}}}{\emph{effect=None}, \emph{resources=None}, \emph{data\_model=None}, \emph{interaction\_type=None}, \emph{via=None}, \emph{references=None}}{}
Retrieves the entities involved in interactions matching the criteria.
It either returns both interacting entities in a \sphinxstyleemphasis{set} or an empty
\sphinxstyleemphasis{set}. This may not sound so useful at the level of this object but
becomes more useful once we want to collect entities having certain
kind of interactions across a series of \sphinxtitleref{Interaction} objects.
\begin{quote}\begin{description}
\item[{Parameters}] \leavevmode\begin{itemize}
\item {} 
\sphinxstyleliteralstrong{\sphinxupquote{entity\_type}} (\sphinxstyleliteralemphasis{\sphinxupquote{str}}) \textendash{} The type of the molecular entity. Possible values: \sphinxtitleref{protein},
\sphinxtitleref{complex}, \sphinxtitleref{mirna}, \sphinxtitleref{small\_molecule}.

\item {} 
\sphinxstyleliteralstrong{\sphinxupquote{return\_type}} (\sphinxstyleliteralemphasis{\sphinxupquote{str}}) \textendash{} The type of values to return. Default is
py:class:\sphinxcode{\sphinxupquote{pypath.entity.Entity}} objects, alternatives are
\sphinxcode{\sphinxupquote{labels}}  \sphinxcode{\sphinxupquote{identifiers}}.

\end{itemize}

\end{description}\end{quote}

\end{fulllineitems}

\index{mirna\_identifiers\_by\_interaction\_type() (pypath.core.interaction.Interaction method)@\spxentry{mirna\_identifiers\_by\_interaction\_type()}\spxextra{pypath.core.interaction.Interaction method}}

\begin{fulllineitems}
\phantomsection\label{\detokenize{reference:pypath.core.interaction.Interaction.mirna_identifiers_by_interaction_type}}\pysiglinewithargsret{\sphinxbfcode{\sphinxupquote{mirna\_identifiers\_by\_interaction\_type}}}{\emph{effect=None}, \emph{resources=None}, \emph{data\_model=None}, \emph{interaction\_type=None}, \emph{via=None}, \emph{references=None}}{}
Retrieves the entities involved in interactions matching the criteria.
It either returns both interacting entities in a \sphinxstyleemphasis{set} or an empty
\sphinxstyleemphasis{set}. This may not sound so useful at the level of this object but
becomes more useful once we want to collect entities having certain
kind of interactions across a series of \sphinxtitleref{Interaction} objects.
\begin{quote}\begin{description}
\item[{Parameters}] \leavevmode\begin{itemize}
\item {} 
\sphinxstyleliteralstrong{\sphinxupquote{entity\_type}} (\sphinxstyleliteralemphasis{\sphinxupquote{str}}) \textendash{} The type of the molecular entity. Possible values: \sphinxtitleref{protein},
\sphinxtitleref{complex}, \sphinxtitleref{mirna}, \sphinxtitleref{small\_molecule}.

\item {} 
\sphinxstyleliteralstrong{\sphinxupquote{return\_type}} (\sphinxstyleliteralemphasis{\sphinxupquote{str}}) \textendash{} The type of values to return. Default is
py:class:\sphinxcode{\sphinxupquote{pypath.entity.Entity}} objects, alternatives are
\sphinxcode{\sphinxupquote{labels}}  \sphinxcode{\sphinxupquote{identifiers}}.

\end{itemize}

\end{description}\end{quote}

\end{fulllineitems}

\index{mirna\_identifiers\_by\_interaction\_type\_and\_data\_model() (pypath.core.interaction.Interaction method)@\spxentry{mirna\_identifiers\_by\_interaction\_type\_and\_data\_model()}\spxextra{pypath.core.interaction.Interaction method}}

\begin{fulllineitems}
\phantomsection\label{\detokenize{reference:pypath.core.interaction.Interaction.mirna_identifiers_by_interaction_type_and_data_model}}\pysiglinewithargsret{\sphinxbfcode{\sphinxupquote{mirna\_identifiers\_by\_interaction\_type\_and\_data\_model}}}{\emph{effect=None}, \emph{resources=None}, \emph{data\_model=None}, \emph{interaction\_type=None}, \emph{via=None}, \emph{references=None}}{}
Retrieves the entities involved in interactions matching the criteria.
It either returns both interacting entities in a \sphinxstyleemphasis{set} or an empty
\sphinxstyleemphasis{set}. This may not sound so useful at the level of this object but
becomes more useful once we want to collect entities having certain
kind of interactions across a series of \sphinxtitleref{Interaction} objects.
\begin{quote}\begin{description}
\item[{Parameters}] \leavevmode\begin{itemize}
\item {} 
\sphinxstyleliteralstrong{\sphinxupquote{entity\_type}} (\sphinxstyleliteralemphasis{\sphinxupquote{str}}) \textendash{} The type of the molecular entity. Possible values: \sphinxtitleref{protein},
\sphinxtitleref{complex}, \sphinxtitleref{mirna}, \sphinxtitleref{small\_molecule}.

\item {} 
\sphinxstyleliteralstrong{\sphinxupquote{return\_type}} (\sphinxstyleliteralemphasis{\sphinxupquote{str}}) \textendash{} The type of values to return. Default is
py:class:\sphinxcode{\sphinxupquote{pypath.entity.Entity}} objects, alternatives are
\sphinxcode{\sphinxupquote{labels}}  \sphinxcode{\sphinxupquote{identifiers}}.

\end{itemize}

\end{description}\end{quote}

\end{fulllineitems}

\index{mirna\_identifiers\_by\_interaction\_type\_and\_data\_model\_and\_resource() (pypath.core.interaction.Interaction method)@\spxentry{mirna\_identifiers\_by\_interaction\_type\_and\_data\_model\_and\_resource()}\spxextra{pypath.core.interaction.Interaction method}}

\begin{fulllineitems}
\phantomsection\label{\detokenize{reference:pypath.core.interaction.Interaction.mirna_identifiers_by_interaction_type_and_data_model_and_resource}}\pysiglinewithargsret{\sphinxbfcode{\sphinxupquote{mirna\_identifiers\_by\_interaction\_type\_and\_data\_model\_and\_resource}}}{\emph{effect=None}, \emph{resources=None}, \emph{data\_model=None}, \emph{interaction\_type=None}, \emph{via=None}, \emph{references=None}}{}
Retrieves the entities involved in interactions matching the criteria.
It either returns both interacting entities in a \sphinxstyleemphasis{set} or an empty
\sphinxstyleemphasis{set}. This may not sound so useful at the level of this object but
becomes more useful once we want to collect entities having certain
kind of interactions across a series of \sphinxtitleref{Interaction} objects.
\begin{quote}\begin{description}
\item[{Parameters}] \leavevmode\begin{itemize}
\item {} 
\sphinxstyleliteralstrong{\sphinxupquote{entity\_type}} (\sphinxstyleliteralemphasis{\sphinxupquote{str}}) \textendash{} The type of the molecular entity. Possible values: \sphinxtitleref{protein},
\sphinxtitleref{complex}, \sphinxtitleref{mirna}, \sphinxtitleref{small\_molecule}.

\item {} 
\sphinxstyleliteralstrong{\sphinxupquote{return\_type}} (\sphinxstyleliteralemphasis{\sphinxupquote{str}}) \textendash{} The type of values to return. Default is
py:class:\sphinxcode{\sphinxupquote{pypath.entity.Entity}} objects, alternatives are
\sphinxcode{\sphinxupquote{labels}}  \sphinxcode{\sphinxupquote{identifiers}}.

\end{itemize}

\end{description}\end{quote}

\end{fulllineitems}

\index{mirna\_identifiers\_by\_reference() (pypath.core.interaction.Interaction method)@\spxentry{mirna\_identifiers\_by\_reference()}\spxextra{pypath.core.interaction.Interaction method}}

\begin{fulllineitems}
\phantomsection\label{\detokenize{reference:pypath.core.interaction.Interaction.mirna_identifiers_by_reference}}\pysiglinewithargsret{\sphinxbfcode{\sphinxupquote{mirna\_identifiers\_by\_reference}}}{\emph{effect=None}, \emph{resources=None}, \emph{data\_model=None}, \emph{interaction\_type=None}, \emph{via=None}, \emph{references=None}}{}
Retrieves the entities involved in interactions matching the criteria.
It either returns both interacting entities in a \sphinxstyleemphasis{set} or an empty
\sphinxstyleemphasis{set}. This may not sound so useful at the level of this object but
becomes more useful once we want to collect entities having certain
kind of interactions across a series of \sphinxtitleref{Interaction} objects.
\begin{quote}\begin{description}
\item[{Parameters}] \leavevmode\begin{itemize}
\item {} 
\sphinxstyleliteralstrong{\sphinxupquote{entity\_type}} (\sphinxstyleliteralemphasis{\sphinxupquote{str}}) \textendash{} The type of the molecular entity. Possible values: \sphinxtitleref{protein},
\sphinxtitleref{complex}, \sphinxtitleref{mirna}, \sphinxtitleref{small\_molecule}.

\item {} 
\sphinxstyleliteralstrong{\sphinxupquote{return\_type}} (\sphinxstyleliteralemphasis{\sphinxupquote{str}}) \textendash{} The type of values to return. Default is
py:class:\sphinxcode{\sphinxupquote{pypath.entity.Entity}} objects, alternatives are
\sphinxcode{\sphinxupquote{labels}}  \sphinxcode{\sphinxupquote{identifiers}}.

\end{itemize}

\end{description}\end{quote}

\end{fulllineitems}

\index{mirna\_identifiers\_by\_resource() (pypath.core.interaction.Interaction method)@\spxentry{mirna\_identifiers\_by\_resource()}\spxextra{pypath.core.interaction.Interaction method}}

\begin{fulllineitems}
\phantomsection\label{\detokenize{reference:pypath.core.interaction.Interaction.mirna_identifiers_by_resource}}\pysiglinewithargsret{\sphinxbfcode{\sphinxupquote{mirna\_identifiers\_by\_resource}}}{\emph{effect=None}, \emph{resources=None}, \emph{data\_model=None}, \emph{interaction\_type=None}, \emph{via=None}, \emph{references=None}}{}
Retrieves the entities involved in interactions matching the criteria.
It either returns both interacting entities in a \sphinxstyleemphasis{set} or an empty
\sphinxstyleemphasis{set}. This may not sound so useful at the level of this object but
becomes more useful once we want to collect entities having certain
kind of interactions across a series of \sphinxtitleref{Interaction} objects.
\begin{quote}\begin{description}
\item[{Parameters}] \leavevmode\begin{itemize}
\item {} 
\sphinxstyleliteralstrong{\sphinxupquote{entity\_type}} (\sphinxstyleliteralemphasis{\sphinxupquote{str}}) \textendash{} The type of the molecular entity. Possible values: \sphinxtitleref{protein},
\sphinxtitleref{complex}, \sphinxtitleref{mirna}, \sphinxtitleref{small\_molecule}.

\item {} 
\sphinxstyleliteralstrong{\sphinxupquote{return\_type}} (\sphinxstyleliteralemphasis{\sphinxupquote{str}}) \textendash{} The type of values to return. Default is
py:class:\sphinxcode{\sphinxupquote{pypath.entity.Entity}} objects, alternatives are
\sphinxcode{\sphinxupquote{labels}}  \sphinxcode{\sphinxupquote{identifiers}}.

\end{itemize}

\end{description}\end{quote}

\end{fulllineitems}

\index{mirna\_labels\_by\_data\_model() (pypath.core.interaction.Interaction method)@\spxentry{mirna\_labels\_by\_data\_model()}\spxextra{pypath.core.interaction.Interaction method}}

\begin{fulllineitems}
\phantomsection\label{\detokenize{reference:pypath.core.interaction.Interaction.mirna_labels_by_data_model}}\pysiglinewithargsret{\sphinxbfcode{\sphinxupquote{mirna\_labels\_by\_data\_model}}}{\emph{effect=None}, \emph{resources=None}, \emph{data\_model=None}, \emph{interaction\_type=None}, \emph{via=None}, \emph{references=None}}{}
Retrieves the entities involved in interactions matching the criteria.
It either returns both interacting entities in a \sphinxstyleemphasis{set} or an empty
\sphinxstyleemphasis{set}. This may not sound so useful at the level of this object but
becomes more useful once we want to collect entities having certain
kind of interactions across a series of \sphinxtitleref{Interaction} objects.
\begin{quote}\begin{description}
\item[{Parameters}] \leavevmode\begin{itemize}
\item {} 
\sphinxstyleliteralstrong{\sphinxupquote{entity\_type}} (\sphinxstyleliteralemphasis{\sphinxupquote{str}}) \textendash{} The type of the molecular entity. Possible values: \sphinxtitleref{protein},
\sphinxtitleref{complex}, \sphinxtitleref{mirna}, \sphinxtitleref{small\_molecule}.

\item {} 
\sphinxstyleliteralstrong{\sphinxupquote{return\_type}} (\sphinxstyleliteralemphasis{\sphinxupquote{str}}) \textendash{} The type of values to return. Default is
py:class:\sphinxcode{\sphinxupquote{pypath.entity.Entity}} objects, alternatives are
\sphinxcode{\sphinxupquote{labels}}  \sphinxcode{\sphinxupquote{identifiers}}.

\end{itemize}

\end{description}\end{quote}

\end{fulllineitems}

\index{mirna\_labels\_by\_interaction\_type() (pypath.core.interaction.Interaction method)@\spxentry{mirna\_labels\_by\_interaction\_type()}\spxextra{pypath.core.interaction.Interaction method}}

\begin{fulllineitems}
\phantomsection\label{\detokenize{reference:pypath.core.interaction.Interaction.mirna_labels_by_interaction_type}}\pysiglinewithargsret{\sphinxbfcode{\sphinxupquote{mirna\_labels\_by\_interaction\_type}}}{\emph{effect=None}, \emph{resources=None}, \emph{data\_model=None}, \emph{interaction\_type=None}, \emph{via=None}, \emph{references=None}}{}
Retrieves the entities involved in interactions matching the criteria.
It either returns both interacting entities in a \sphinxstyleemphasis{set} or an empty
\sphinxstyleemphasis{set}. This may not sound so useful at the level of this object but
becomes more useful once we want to collect entities having certain
kind of interactions across a series of \sphinxtitleref{Interaction} objects.
\begin{quote}\begin{description}
\item[{Parameters}] \leavevmode\begin{itemize}
\item {} 
\sphinxstyleliteralstrong{\sphinxupquote{entity\_type}} (\sphinxstyleliteralemphasis{\sphinxupquote{str}}) \textendash{} The type of the molecular entity. Possible values: \sphinxtitleref{protein},
\sphinxtitleref{complex}, \sphinxtitleref{mirna}, \sphinxtitleref{small\_molecule}.

\item {} 
\sphinxstyleliteralstrong{\sphinxupquote{return\_type}} (\sphinxstyleliteralemphasis{\sphinxupquote{str}}) \textendash{} The type of values to return. Default is
py:class:\sphinxcode{\sphinxupquote{pypath.entity.Entity}} objects, alternatives are
\sphinxcode{\sphinxupquote{labels}}  \sphinxcode{\sphinxupquote{identifiers}}.

\end{itemize}

\end{description}\end{quote}

\end{fulllineitems}

\index{mirna\_labels\_by\_interaction\_type\_and\_data\_model() (pypath.core.interaction.Interaction method)@\spxentry{mirna\_labels\_by\_interaction\_type\_and\_data\_model()}\spxextra{pypath.core.interaction.Interaction method}}

\begin{fulllineitems}
\phantomsection\label{\detokenize{reference:pypath.core.interaction.Interaction.mirna_labels_by_interaction_type_and_data_model}}\pysiglinewithargsret{\sphinxbfcode{\sphinxupquote{mirna\_labels\_by\_interaction\_type\_and\_data\_model}}}{\emph{effect=None}, \emph{resources=None}, \emph{data\_model=None}, \emph{interaction\_type=None}, \emph{via=None}, \emph{references=None}}{}
Retrieves the entities involved in interactions matching the criteria.
It either returns both interacting entities in a \sphinxstyleemphasis{set} or an empty
\sphinxstyleemphasis{set}. This may not sound so useful at the level of this object but
becomes more useful once we want to collect entities having certain
kind of interactions across a series of \sphinxtitleref{Interaction} objects.
\begin{quote}\begin{description}
\item[{Parameters}] \leavevmode\begin{itemize}
\item {} 
\sphinxstyleliteralstrong{\sphinxupquote{entity\_type}} (\sphinxstyleliteralemphasis{\sphinxupquote{str}}) \textendash{} The type of the molecular entity. Possible values: \sphinxtitleref{protein},
\sphinxtitleref{complex}, \sphinxtitleref{mirna}, \sphinxtitleref{small\_molecule}.

\item {} 
\sphinxstyleliteralstrong{\sphinxupquote{return\_type}} (\sphinxstyleliteralemphasis{\sphinxupquote{str}}) \textendash{} The type of values to return. Default is
py:class:\sphinxcode{\sphinxupquote{pypath.entity.Entity}} objects, alternatives are
\sphinxcode{\sphinxupquote{labels}}  \sphinxcode{\sphinxupquote{identifiers}}.

\end{itemize}

\end{description}\end{quote}

\end{fulllineitems}

\index{mirna\_labels\_by\_interaction\_type\_and\_data\_model\_and\_resource() (pypath.core.interaction.Interaction method)@\spxentry{mirna\_labels\_by\_interaction\_type\_and\_data\_model\_and\_resource()}\spxextra{pypath.core.interaction.Interaction method}}

\begin{fulllineitems}
\phantomsection\label{\detokenize{reference:pypath.core.interaction.Interaction.mirna_labels_by_interaction_type_and_data_model_and_resource}}\pysiglinewithargsret{\sphinxbfcode{\sphinxupquote{mirna\_labels\_by\_interaction\_type\_and\_data\_model\_and\_resource}}}{\emph{effect=None}, \emph{resources=None}, \emph{data\_model=None}, \emph{interaction\_type=None}, \emph{via=None}, \emph{references=None}}{}
Retrieves the entities involved in interactions matching the criteria.
It either returns both interacting entities in a \sphinxstyleemphasis{set} or an empty
\sphinxstyleemphasis{set}. This may not sound so useful at the level of this object but
becomes more useful once we want to collect entities having certain
kind of interactions across a series of \sphinxtitleref{Interaction} objects.
\begin{quote}\begin{description}
\item[{Parameters}] \leavevmode\begin{itemize}
\item {} 
\sphinxstyleliteralstrong{\sphinxupquote{entity\_type}} (\sphinxstyleliteralemphasis{\sphinxupquote{str}}) \textendash{} The type of the molecular entity. Possible values: \sphinxtitleref{protein},
\sphinxtitleref{complex}, \sphinxtitleref{mirna}, \sphinxtitleref{small\_molecule}.

\item {} 
\sphinxstyleliteralstrong{\sphinxupquote{return\_type}} (\sphinxstyleliteralemphasis{\sphinxupquote{str}}) \textendash{} The type of values to return. Default is
py:class:\sphinxcode{\sphinxupquote{pypath.entity.Entity}} objects, alternatives are
\sphinxcode{\sphinxupquote{labels}}  \sphinxcode{\sphinxupquote{identifiers}}.

\end{itemize}

\end{description}\end{quote}

\end{fulllineitems}

\index{mirna\_labels\_by\_reference() (pypath.core.interaction.Interaction method)@\spxentry{mirna\_labels\_by\_reference()}\spxextra{pypath.core.interaction.Interaction method}}

\begin{fulllineitems}
\phantomsection\label{\detokenize{reference:pypath.core.interaction.Interaction.mirna_labels_by_reference}}\pysiglinewithargsret{\sphinxbfcode{\sphinxupquote{mirna\_labels\_by\_reference}}}{\emph{effect=None}, \emph{resources=None}, \emph{data\_model=None}, \emph{interaction\_type=None}, \emph{via=None}, \emph{references=None}}{}
Retrieves the entities involved in interactions matching the criteria.
It either returns both interacting entities in a \sphinxstyleemphasis{set} or an empty
\sphinxstyleemphasis{set}. This may not sound so useful at the level of this object but
becomes more useful once we want to collect entities having certain
kind of interactions across a series of \sphinxtitleref{Interaction} objects.
\begin{quote}\begin{description}
\item[{Parameters}] \leavevmode\begin{itemize}
\item {} 
\sphinxstyleliteralstrong{\sphinxupquote{entity\_type}} (\sphinxstyleliteralemphasis{\sphinxupquote{str}}) \textendash{} The type of the molecular entity. Possible values: \sphinxtitleref{protein},
\sphinxtitleref{complex}, \sphinxtitleref{mirna}, \sphinxtitleref{small\_molecule}.

\item {} 
\sphinxstyleliteralstrong{\sphinxupquote{return\_type}} (\sphinxstyleliteralemphasis{\sphinxupquote{str}}) \textendash{} The type of values to return. Default is
py:class:\sphinxcode{\sphinxupquote{pypath.entity.Entity}} objects, alternatives are
\sphinxcode{\sphinxupquote{labels}}  \sphinxcode{\sphinxupquote{identifiers}}.

\end{itemize}

\end{description}\end{quote}

\end{fulllineitems}

\index{mirna\_labels\_by\_resource() (pypath.core.interaction.Interaction method)@\spxentry{mirna\_labels\_by\_resource()}\spxextra{pypath.core.interaction.Interaction method}}

\begin{fulllineitems}
\phantomsection\label{\detokenize{reference:pypath.core.interaction.Interaction.mirna_labels_by_resource}}\pysiglinewithargsret{\sphinxbfcode{\sphinxupquote{mirna\_labels\_by\_resource}}}{\emph{effect=None}, \emph{resources=None}, \emph{data\_model=None}, \emph{interaction\_type=None}, \emph{via=None}, \emph{references=None}}{}
Retrieves the entities involved in interactions matching the criteria.
It either returns both interacting entities in a \sphinxstyleemphasis{set} or an empty
\sphinxstyleemphasis{set}. This may not sound so useful at the level of this object but
becomes more useful once we want to collect entities having certain
kind of interactions across a series of \sphinxtitleref{Interaction} objects.
\begin{quote}\begin{description}
\item[{Parameters}] \leavevmode\begin{itemize}
\item {} 
\sphinxstyleliteralstrong{\sphinxupquote{entity\_type}} (\sphinxstyleliteralemphasis{\sphinxupquote{str}}) \textendash{} The type of the molecular entity. Possible values: \sphinxtitleref{protein},
\sphinxtitleref{complex}, \sphinxtitleref{mirna}, \sphinxtitleref{small\_molecule}.

\item {} 
\sphinxstyleliteralstrong{\sphinxupquote{return\_type}} (\sphinxstyleliteralemphasis{\sphinxupquote{str}}) \textendash{} The type of values to return. Default is
py:class:\sphinxcode{\sphinxupquote{pypath.entity.Entity}} objects, alternatives are
\sphinxcode{\sphinxupquote{labels}}  \sphinxcode{\sphinxupquote{identifiers}}.

\end{itemize}

\end{description}\end{quote}

\end{fulllineitems}

\index{mirnas\_by\_data\_model() (pypath.core.interaction.Interaction method)@\spxentry{mirnas\_by\_data\_model()}\spxextra{pypath.core.interaction.Interaction method}}

\begin{fulllineitems}
\phantomsection\label{\detokenize{reference:pypath.core.interaction.Interaction.mirnas_by_data_model}}\pysiglinewithargsret{\sphinxbfcode{\sphinxupquote{mirnas\_by\_data\_model}}}{\emph{effect=None}, \emph{resources=None}, \emph{data\_model=None}, \emph{interaction\_type=None}, \emph{via=None}, \emph{references=None}}{}
Retrieves the entities involved in interactions matching the criteria.
It either returns both interacting entities in a \sphinxstyleemphasis{set} or an empty
\sphinxstyleemphasis{set}. This may not sound so useful at the level of this object but
becomes more useful once we want to collect entities having certain
kind of interactions across a series of \sphinxtitleref{Interaction} objects.
\begin{quote}\begin{description}
\item[{Parameters}] \leavevmode\begin{itemize}
\item {} 
\sphinxstyleliteralstrong{\sphinxupquote{entity\_type}} (\sphinxstyleliteralemphasis{\sphinxupquote{str}}) \textendash{} The type of the molecular entity. Possible values: \sphinxtitleref{protein},
\sphinxtitleref{complex}, \sphinxtitleref{mirna}, \sphinxtitleref{small\_molecule}.

\item {} 
\sphinxstyleliteralstrong{\sphinxupquote{return\_type}} (\sphinxstyleliteralemphasis{\sphinxupquote{str}}) \textendash{} The type of values to return. Default is
py:class:\sphinxcode{\sphinxupquote{pypath.entity.Entity}} objects, alternatives are
\sphinxcode{\sphinxupquote{labels}}  \sphinxcode{\sphinxupquote{identifiers}}.

\end{itemize}

\end{description}\end{quote}

\end{fulllineitems}

\index{mirnas\_by\_interaction\_type() (pypath.core.interaction.Interaction method)@\spxentry{mirnas\_by\_interaction\_type()}\spxextra{pypath.core.interaction.Interaction method}}

\begin{fulllineitems}
\phantomsection\label{\detokenize{reference:pypath.core.interaction.Interaction.mirnas_by_interaction_type}}\pysiglinewithargsret{\sphinxbfcode{\sphinxupquote{mirnas\_by\_interaction\_type}}}{\emph{effect=None}, \emph{resources=None}, \emph{data\_model=None}, \emph{interaction\_type=None}, \emph{via=None}, \emph{references=None}}{}
Retrieves the entities involved in interactions matching the criteria.
It either returns both interacting entities in a \sphinxstyleemphasis{set} or an empty
\sphinxstyleemphasis{set}. This may not sound so useful at the level of this object but
becomes more useful once we want to collect entities having certain
kind of interactions across a series of \sphinxtitleref{Interaction} objects.
\begin{quote}\begin{description}
\item[{Parameters}] \leavevmode\begin{itemize}
\item {} 
\sphinxstyleliteralstrong{\sphinxupquote{entity\_type}} (\sphinxstyleliteralemphasis{\sphinxupquote{str}}) \textendash{} The type of the molecular entity. Possible values: \sphinxtitleref{protein},
\sphinxtitleref{complex}, \sphinxtitleref{mirna}, \sphinxtitleref{small\_molecule}.

\item {} 
\sphinxstyleliteralstrong{\sphinxupquote{return\_type}} (\sphinxstyleliteralemphasis{\sphinxupquote{str}}) \textendash{} The type of values to return. Default is
py:class:\sphinxcode{\sphinxupquote{pypath.entity.Entity}} objects, alternatives are
\sphinxcode{\sphinxupquote{labels}}  \sphinxcode{\sphinxupquote{identifiers}}.

\end{itemize}

\end{description}\end{quote}

\end{fulllineitems}

\index{mirnas\_by\_interaction\_type\_and\_data\_model() (pypath.core.interaction.Interaction method)@\spxentry{mirnas\_by\_interaction\_type\_and\_data\_model()}\spxextra{pypath.core.interaction.Interaction method}}

\begin{fulllineitems}
\phantomsection\label{\detokenize{reference:pypath.core.interaction.Interaction.mirnas_by_interaction_type_and_data_model}}\pysiglinewithargsret{\sphinxbfcode{\sphinxupquote{mirnas\_by\_interaction\_type\_and\_data\_model}}}{\emph{effect=None}, \emph{resources=None}, \emph{data\_model=None}, \emph{interaction\_type=None}, \emph{via=None}, \emph{references=None}}{}
Retrieves the entities involved in interactions matching the criteria.
It either returns both interacting entities in a \sphinxstyleemphasis{set} or an empty
\sphinxstyleemphasis{set}. This may not sound so useful at the level of this object but
becomes more useful once we want to collect entities having certain
kind of interactions across a series of \sphinxtitleref{Interaction} objects.
\begin{quote}\begin{description}
\item[{Parameters}] \leavevmode\begin{itemize}
\item {} 
\sphinxstyleliteralstrong{\sphinxupquote{entity\_type}} (\sphinxstyleliteralemphasis{\sphinxupquote{str}}) \textendash{} The type of the molecular entity. Possible values: \sphinxtitleref{protein},
\sphinxtitleref{complex}, \sphinxtitleref{mirna}, \sphinxtitleref{small\_molecule}.

\item {} 
\sphinxstyleliteralstrong{\sphinxupquote{return\_type}} (\sphinxstyleliteralemphasis{\sphinxupquote{str}}) \textendash{} The type of values to return. Default is
py:class:\sphinxcode{\sphinxupquote{pypath.entity.Entity}} objects, alternatives are
\sphinxcode{\sphinxupquote{labels}}  \sphinxcode{\sphinxupquote{identifiers}}.

\end{itemize}

\end{description}\end{quote}

\end{fulllineitems}

\index{mirnas\_by\_interaction\_type\_and\_data\_model\_and\_resource() (pypath.core.interaction.Interaction method)@\spxentry{mirnas\_by\_interaction\_type\_and\_data\_model\_and\_resource()}\spxextra{pypath.core.interaction.Interaction method}}

\begin{fulllineitems}
\phantomsection\label{\detokenize{reference:pypath.core.interaction.Interaction.mirnas_by_interaction_type_and_data_model_and_resource}}\pysiglinewithargsret{\sphinxbfcode{\sphinxupquote{mirnas\_by\_interaction\_type\_and\_data\_model\_and\_resource}}}{\emph{effect=None}, \emph{resources=None}, \emph{data\_model=None}, \emph{interaction\_type=None}, \emph{via=None}, \emph{references=None}}{}
Retrieves the entities involved in interactions matching the criteria.
It either returns both interacting entities in a \sphinxstyleemphasis{set} or an empty
\sphinxstyleemphasis{set}. This may not sound so useful at the level of this object but
becomes more useful once we want to collect entities having certain
kind of interactions across a series of \sphinxtitleref{Interaction} objects.
\begin{quote}\begin{description}
\item[{Parameters}] \leavevmode\begin{itemize}
\item {} 
\sphinxstyleliteralstrong{\sphinxupquote{entity\_type}} (\sphinxstyleliteralemphasis{\sphinxupquote{str}}) \textendash{} The type of the molecular entity. Possible values: \sphinxtitleref{protein},
\sphinxtitleref{complex}, \sphinxtitleref{mirna}, \sphinxtitleref{small\_molecule}.

\item {} 
\sphinxstyleliteralstrong{\sphinxupquote{return\_type}} (\sphinxstyleliteralemphasis{\sphinxupquote{str}}) \textendash{} The type of values to return. Default is
py:class:\sphinxcode{\sphinxupquote{pypath.entity.Entity}} objects, alternatives are
\sphinxcode{\sphinxupquote{labels}}  \sphinxcode{\sphinxupquote{identifiers}}.

\end{itemize}

\end{description}\end{quote}

\end{fulllineitems}

\index{mirnas\_by\_reference() (pypath.core.interaction.Interaction method)@\spxentry{mirnas\_by\_reference()}\spxextra{pypath.core.interaction.Interaction method}}

\begin{fulllineitems}
\phantomsection\label{\detokenize{reference:pypath.core.interaction.Interaction.mirnas_by_reference}}\pysiglinewithargsret{\sphinxbfcode{\sphinxupquote{mirnas\_by\_reference}}}{\emph{effect=None}, \emph{resources=None}, \emph{data\_model=None}, \emph{interaction\_type=None}, \emph{via=None}, \emph{references=None}}{}
Retrieves the entities involved in interactions matching the criteria.
It either returns both interacting entities in a \sphinxstyleemphasis{set} or an empty
\sphinxstyleemphasis{set}. This may not sound so useful at the level of this object but
becomes more useful once we want to collect entities having certain
kind of interactions across a series of \sphinxtitleref{Interaction} objects.
\begin{quote}\begin{description}
\item[{Parameters}] \leavevmode\begin{itemize}
\item {} 
\sphinxstyleliteralstrong{\sphinxupquote{entity\_type}} (\sphinxstyleliteralemphasis{\sphinxupquote{str}}) \textendash{} The type of the molecular entity. Possible values: \sphinxtitleref{protein},
\sphinxtitleref{complex}, \sphinxtitleref{mirna}, \sphinxtitleref{small\_molecule}.

\item {} 
\sphinxstyleliteralstrong{\sphinxupquote{return\_type}} (\sphinxstyleliteralemphasis{\sphinxupquote{str}}) \textendash{} The type of values to return. Default is
py:class:\sphinxcode{\sphinxupquote{pypath.entity.Entity}} objects, alternatives are
\sphinxcode{\sphinxupquote{labels}}  \sphinxcode{\sphinxupquote{identifiers}}.

\end{itemize}

\end{description}\end{quote}

\end{fulllineitems}

\index{mirnas\_by\_resource() (pypath.core.interaction.Interaction method)@\spxentry{mirnas\_by\_resource()}\spxextra{pypath.core.interaction.Interaction method}}

\begin{fulllineitems}
\phantomsection\label{\detokenize{reference:pypath.core.interaction.Interaction.mirnas_by_resource}}\pysiglinewithargsret{\sphinxbfcode{\sphinxupquote{mirnas\_by\_resource}}}{\emph{effect=None}, \emph{resources=None}, \emph{data\_model=None}, \emph{interaction\_type=None}, \emph{via=None}, \emph{references=None}}{}
Retrieves the entities involved in interactions matching the criteria.
It either returns both interacting entities in a \sphinxstyleemphasis{set} or an empty
\sphinxstyleemphasis{set}. This may not sound so useful at the level of this object but
becomes more useful once we want to collect entities having certain
kind of interactions across a series of \sphinxtitleref{Interaction} objects.
\begin{quote}\begin{description}
\item[{Parameters}] \leavevmode\begin{itemize}
\item {} 
\sphinxstyleliteralstrong{\sphinxupquote{entity\_type}} (\sphinxstyleliteralemphasis{\sphinxupquote{str}}) \textendash{} The type of the molecular entity. Possible values: \sphinxtitleref{protein},
\sphinxtitleref{complex}, \sphinxtitleref{mirna}, \sphinxtitleref{small\_molecule}.

\item {} 
\sphinxstyleliteralstrong{\sphinxupquote{return\_type}} (\sphinxstyleliteralemphasis{\sphinxupquote{str}}) \textendash{} The type of values to return. Default is
py:class:\sphinxcode{\sphinxupquote{pypath.entity.Entity}} objects, alternatives are
\sphinxcode{\sphinxupquote{labels}}  \sphinxcode{\sphinxupquote{identifiers}}.

\end{itemize}

\end{description}\end{quote}

\end{fulllineitems}

\index{negative\_a\_b() (pypath.core.interaction.Interaction method)@\spxentry{negative\_a\_b()}\spxextra{pypath.core.interaction.Interaction method}}

\begin{fulllineitems}
\phantomsection\label{\detokenize{reference:pypath.core.interaction.Interaction.negative_a_b}}\pysiglinewithargsret{\sphinxbfcode{\sphinxupquote{negative\_a\_b}}}{}{}
Checks if the \sphinxcode{\sphinxupquote{a\_b}} directionality is a negative
interaction.
\begin{quote}\begin{description}
\item[{Returns}] \leavevmode
(\sphinxstyleemphasis{bool}) \textendash{} \sphinxcode{\sphinxupquote{True}} if there is supporting information on
the \sphinxcode{\sphinxupquote{a\_b}} direction of the edge as inhibition.
\sphinxcode{\sphinxupquote{False}} otherwise.

\end{description}\end{quote}

\end{fulllineitems}

\index{negative\_b\_a() (pypath.core.interaction.Interaction method)@\spxentry{negative\_b\_a()}\spxextra{pypath.core.interaction.Interaction method}}

\begin{fulllineitems}
\phantomsection\label{\detokenize{reference:pypath.core.interaction.Interaction.negative_b_a}}\pysiglinewithargsret{\sphinxbfcode{\sphinxupquote{negative\_b\_a}}}{}{}
Checks if the \sphinxcode{\sphinxupquote{b\_a}} directionality is a negative
interaction.
\begin{quote}\begin{description}
\item[{Returns}] \leavevmode
(\sphinxstyleemphasis{bool}) \textendash{} \sphinxcode{\sphinxupquote{True}} if there is supporting information on
the \sphinxcode{\sphinxupquote{b\_a}} direction of the edge as inhibition.
\sphinxcode{\sphinxupquote{False}} otherwise.

\end{description}\end{quote}

\end{fulllineitems}

\index{negative\_resources\_a\_b() (pypath.core.interaction.Interaction method)@\spxentry{negative\_resources\_a\_b()}\spxextra{pypath.core.interaction.Interaction method}}

\begin{fulllineitems}
\phantomsection\label{\detokenize{reference:pypath.core.interaction.Interaction.negative_resources_a_b}}\pysiglinewithargsret{\sphinxbfcode{\sphinxupquote{negative\_resources\_a\_b}}}{\emph{**kwargs}}{}
Retrieves the list of resources for the \sphinxcode{\sphinxupquote{a\_b}}
direction and negative sign.
\begin{quote}\begin{description}
\item[{Returns}] \leavevmode
(\sphinxstyleemphasis{set}) \textendash{} Contains the names of the resources supporting the
\sphinxcode{\sphinxupquote{a\_b}} directionality of the edge with a
negative sign.

\end{description}\end{quote}

\end{fulllineitems}

\index{negative\_resources\_b\_a() (pypath.core.interaction.Interaction method)@\spxentry{negative\_resources\_b\_a()}\spxextra{pypath.core.interaction.Interaction method}}

\begin{fulllineitems}
\phantomsection\label{\detokenize{reference:pypath.core.interaction.Interaction.negative_resources_b_a}}\pysiglinewithargsret{\sphinxbfcode{\sphinxupquote{negative\_resources\_b\_a}}}{\emph{**kwargs}}{}
Retrieves the list of resources for the \sphinxcode{\sphinxupquote{b\_a}}
direction and negative sign.
\begin{quote}\begin{description}
\item[{Returns}] \leavevmode
(\sphinxstyleemphasis{set}) \textendash{} Contains the names of the resources supporting the
\sphinxcode{\sphinxupquote{b\_a}} directionality of the edge with a
negative sign.

\end{description}\end{quote}

\end{fulllineitems}

\index{negative\_reverse() (pypath.core.interaction.Interaction method)@\spxentry{negative\_reverse()}\spxextra{pypath.core.interaction.Interaction method}}

\begin{fulllineitems}
\phantomsection\label{\detokenize{reference:pypath.core.interaction.Interaction.negative_reverse}}\pysiglinewithargsret{\sphinxbfcode{\sphinxupquote{negative\_reverse}}}{}{}
Checks if the \sphinxcode{\sphinxupquote{b\_a}} directionality is a negative
interaction.
\begin{quote}\begin{description}
\item[{Returns}] \leavevmode
(\sphinxstyleemphasis{bool}) \textendash{} \sphinxcode{\sphinxupquote{True}} if there is supporting information on
the \sphinxcode{\sphinxupquote{b\_a}} direction of the edge as inhibition.
\sphinxcode{\sphinxupquote{False}} otherwise.

\end{description}\end{quote}

\end{fulllineitems}

\index{negative\_straight() (pypath.core.interaction.Interaction method)@\spxentry{negative\_straight()}\spxextra{pypath.core.interaction.Interaction method}}

\begin{fulllineitems}
\phantomsection\label{\detokenize{reference:pypath.core.interaction.Interaction.negative_straight}}\pysiglinewithargsret{\sphinxbfcode{\sphinxupquote{negative\_straight}}}{}{}
Checks if the \sphinxcode{\sphinxupquote{a\_b}} directionality is a negative
interaction.
\begin{quote}\begin{description}
\item[{Returns}] \leavevmode
(\sphinxstyleemphasis{bool}) \textendash{} \sphinxcode{\sphinxupquote{True}} if there is supporting information on
the \sphinxcode{\sphinxupquote{a\_b}} direction of the edge as inhibition.
\sphinxcode{\sphinxupquote{False}} otherwise.

\end{description}\end{quote}

\end{fulllineitems}

\index{positive\_a\_b() (pypath.core.interaction.Interaction method)@\spxentry{positive\_a\_b()}\spxextra{pypath.core.interaction.Interaction method}}

\begin{fulllineitems}
\phantomsection\label{\detokenize{reference:pypath.core.interaction.Interaction.positive_a_b}}\pysiglinewithargsret{\sphinxbfcode{\sphinxupquote{positive\_a\_b}}}{}{}
Checks if the \sphinxcode{\sphinxupquote{a\_b}} directionality is a positive
interaction.
\begin{quote}\begin{description}
\item[{Returns}] \leavevmode
(\sphinxstyleemphasis{bool}) \textendash{} \sphinxcode{\sphinxupquote{True}} if there is supporting information on
the \sphinxcode{\sphinxupquote{a\_b}} direction of the edge as activation.
\sphinxcode{\sphinxupquote{False}} otherwise.

\end{description}\end{quote}

\end{fulllineitems}

\index{positive\_b\_a() (pypath.core.interaction.Interaction method)@\spxentry{positive\_b\_a()}\spxextra{pypath.core.interaction.Interaction method}}

\begin{fulllineitems}
\phantomsection\label{\detokenize{reference:pypath.core.interaction.Interaction.positive_b_a}}\pysiglinewithargsret{\sphinxbfcode{\sphinxupquote{positive\_b\_a}}}{}{}
Checks if the \sphinxcode{\sphinxupquote{b\_a}} directionality is a positive
interaction.
\begin{quote}\begin{description}
\item[{Returns}] \leavevmode
(\sphinxstyleemphasis{bool}) \textendash{} \sphinxcode{\sphinxupquote{True}} if there is supporting information on
the \sphinxcode{\sphinxupquote{b\_a}} direction of the edge as activation.
\sphinxcode{\sphinxupquote{False}} otherwise.

\end{description}\end{quote}

\end{fulllineitems}

\index{positive\_resources\_a\_b() (pypath.core.interaction.Interaction method)@\spxentry{positive\_resources\_a\_b()}\spxextra{pypath.core.interaction.Interaction method}}

\begin{fulllineitems}
\phantomsection\label{\detokenize{reference:pypath.core.interaction.Interaction.positive_resources_a_b}}\pysiglinewithargsret{\sphinxbfcode{\sphinxupquote{positive\_resources\_a\_b}}}{\emph{**kwargs}}{}
Retrieves the list of resources for the \sphinxcode{\sphinxupquote{a\_b}}
direction and positive sign.
\begin{quote}\begin{description}
\item[{Returns}] \leavevmode
(\sphinxstyleemphasis{set}) \textendash{} Contains the names of the resources supporting the
\sphinxcode{\sphinxupquote{a\_b}} directionality of the edge with a
positive sign.

\end{description}\end{quote}

\end{fulllineitems}

\index{positive\_resources\_b\_a() (pypath.core.interaction.Interaction method)@\spxentry{positive\_resources\_b\_a()}\spxextra{pypath.core.interaction.Interaction method}}

\begin{fulllineitems}
\phantomsection\label{\detokenize{reference:pypath.core.interaction.Interaction.positive_resources_b_a}}\pysiglinewithargsret{\sphinxbfcode{\sphinxupquote{positive\_resources\_b\_a}}}{\emph{**kwargs}}{}
Retrieves the list of resources for the \sphinxcode{\sphinxupquote{b\_a}}
direction and positive sign.
\begin{quote}\begin{description}
\item[{Returns}] \leavevmode
(\sphinxstyleemphasis{set}) \textendash{} Contains the names of the resources supporting the
\sphinxcode{\sphinxupquote{b\_a}} directionality of the edge with a
positive sign.

\end{description}\end{quote}

\end{fulllineitems}

\index{positive\_reverse() (pypath.core.interaction.Interaction method)@\spxentry{positive\_reverse()}\spxextra{pypath.core.interaction.Interaction method}}

\begin{fulllineitems}
\phantomsection\label{\detokenize{reference:pypath.core.interaction.Interaction.positive_reverse}}\pysiglinewithargsret{\sphinxbfcode{\sphinxupquote{positive\_reverse}}}{}{}
Checks if the \sphinxcode{\sphinxupquote{b\_a}} directionality is a positive
interaction.
\begin{quote}\begin{description}
\item[{Returns}] \leavevmode
(\sphinxstyleemphasis{bool}) \textendash{} \sphinxcode{\sphinxupquote{True}} if there is supporting information on
the \sphinxcode{\sphinxupquote{b\_a}} direction of the edge as activation.
\sphinxcode{\sphinxupquote{False}} otherwise.

\end{description}\end{quote}

\end{fulllineitems}

\index{positive\_straight() (pypath.core.interaction.Interaction method)@\spxentry{positive\_straight()}\spxextra{pypath.core.interaction.Interaction method}}

\begin{fulllineitems}
\phantomsection\label{\detokenize{reference:pypath.core.interaction.Interaction.positive_straight}}\pysiglinewithargsret{\sphinxbfcode{\sphinxupquote{positive\_straight}}}{}{}
Checks if the \sphinxcode{\sphinxupquote{a\_b}} directionality is a positive
interaction.
\begin{quote}\begin{description}
\item[{Returns}] \leavevmode
(\sphinxstyleemphasis{bool}) \textendash{} \sphinxcode{\sphinxupquote{True}} if there is supporting information on
the \sphinxcode{\sphinxupquote{a\_b}} direction of the edge as activation.
\sphinxcode{\sphinxupquote{False}} otherwise.

\end{description}\end{quote}

\end{fulllineitems}

\index{protein\_identifiers\_by\_data\_model() (pypath.core.interaction.Interaction method)@\spxentry{protein\_identifiers\_by\_data\_model()}\spxextra{pypath.core.interaction.Interaction method}}

\begin{fulllineitems}
\phantomsection\label{\detokenize{reference:pypath.core.interaction.Interaction.protein_identifiers_by_data_model}}\pysiglinewithargsret{\sphinxbfcode{\sphinxupquote{protein\_identifiers\_by\_data\_model}}}{\emph{effect=None}, \emph{resources=None}, \emph{data\_model=None}, \emph{interaction\_type=None}, \emph{via=None}, \emph{references=None}}{}
Retrieves the entities involved in interactions matching the criteria.
It either returns both interacting entities in a \sphinxstyleemphasis{set} or an empty
\sphinxstyleemphasis{set}. This may not sound so useful at the level of this object but
becomes more useful once we want to collect entities having certain
kind of interactions across a series of \sphinxtitleref{Interaction} objects.
\begin{quote}\begin{description}
\item[{Parameters}] \leavevmode\begin{itemize}
\item {} 
\sphinxstyleliteralstrong{\sphinxupquote{entity\_type}} (\sphinxstyleliteralemphasis{\sphinxupquote{str}}) \textendash{} The type of the molecular entity. Possible values: \sphinxtitleref{protein},
\sphinxtitleref{complex}, \sphinxtitleref{mirna}, \sphinxtitleref{small\_molecule}.

\item {} 
\sphinxstyleliteralstrong{\sphinxupquote{return\_type}} (\sphinxstyleliteralemphasis{\sphinxupquote{str}}) \textendash{} The type of values to return. Default is
py:class:\sphinxcode{\sphinxupquote{pypath.entity.Entity}} objects, alternatives are
\sphinxcode{\sphinxupquote{labels}}  \sphinxcode{\sphinxupquote{identifiers}}.

\end{itemize}

\end{description}\end{quote}

\end{fulllineitems}

\index{protein\_identifiers\_by\_interaction\_type() (pypath.core.interaction.Interaction method)@\spxentry{protein\_identifiers\_by\_interaction\_type()}\spxextra{pypath.core.interaction.Interaction method}}

\begin{fulllineitems}
\phantomsection\label{\detokenize{reference:pypath.core.interaction.Interaction.protein_identifiers_by_interaction_type}}\pysiglinewithargsret{\sphinxbfcode{\sphinxupquote{protein\_identifiers\_by\_interaction\_type}}}{\emph{effect=None}, \emph{resources=None}, \emph{data\_model=None}, \emph{interaction\_type=None}, \emph{via=None}, \emph{references=None}}{}
Retrieves the entities involved in interactions matching the criteria.
It either returns both interacting entities in a \sphinxstyleemphasis{set} or an empty
\sphinxstyleemphasis{set}. This may not sound so useful at the level of this object but
becomes more useful once we want to collect entities having certain
kind of interactions across a series of \sphinxtitleref{Interaction} objects.
\begin{quote}\begin{description}
\item[{Parameters}] \leavevmode\begin{itemize}
\item {} 
\sphinxstyleliteralstrong{\sphinxupquote{entity\_type}} (\sphinxstyleliteralemphasis{\sphinxupquote{str}}) \textendash{} The type of the molecular entity. Possible values: \sphinxtitleref{protein},
\sphinxtitleref{complex}, \sphinxtitleref{mirna}, \sphinxtitleref{small\_molecule}.

\item {} 
\sphinxstyleliteralstrong{\sphinxupquote{return\_type}} (\sphinxstyleliteralemphasis{\sphinxupquote{str}}) \textendash{} The type of values to return. Default is
py:class:\sphinxcode{\sphinxupquote{pypath.entity.Entity}} objects, alternatives are
\sphinxcode{\sphinxupquote{labels}}  \sphinxcode{\sphinxupquote{identifiers}}.

\end{itemize}

\end{description}\end{quote}

\end{fulllineitems}

\index{protein\_identifiers\_by\_interaction\_type\_and\_data\_model() (pypath.core.interaction.Interaction method)@\spxentry{protein\_identifiers\_by\_interaction\_type\_and\_data\_model()}\spxextra{pypath.core.interaction.Interaction method}}

\begin{fulllineitems}
\phantomsection\label{\detokenize{reference:pypath.core.interaction.Interaction.protein_identifiers_by_interaction_type_and_data_model}}\pysiglinewithargsret{\sphinxbfcode{\sphinxupquote{protein\_identifiers\_by\_interaction\_type\_and\_data\_model}}}{\emph{effect=None}, \emph{resources=None}, \emph{data\_model=None}, \emph{interaction\_type=None}, \emph{via=None}, \emph{references=None}}{}
Retrieves the entities involved in interactions matching the criteria.
It either returns both interacting entities in a \sphinxstyleemphasis{set} or an empty
\sphinxstyleemphasis{set}. This may not sound so useful at the level of this object but
becomes more useful once we want to collect entities having certain
kind of interactions across a series of \sphinxtitleref{Interaction} objects.
\begin{quote}\begin{description}
\item[{Parameters}] \leavevmode\begin{itemize}
\item {} 
\sphinxstyleliteralstrong{\sphinxupquote{entity\_type}} (\sphinxstyleliteralemphasis{\sphinxupquote{str}}) \textendash{} The type of the molecular entity. Possible values: \sphinxtitleref{protein},
\sphinxtitleref{complex}, \sphinxtitleref{mirna}, \sphinxtitleref{small\_molecule}.

\item {} 
\sphinxstyleliteralstrong{\sphinxupquote{return\_type}} (\sphinxstyleliteralemphasis{\sphinxupquote{str}}) \textendash{} The type of values to return. Default is
py:class:\sphinxcode{\sphinxupquote{pypath.entity.Entity}} objects, alternatives are
\sphinxcode{\sphinxupquote{labels}}  \sphinxcode{\sphinxupquote{identifiers}}.

\end{itemize}

\end{description}\end{quote}

\end{fulllineitems}

\index{protein\_identifiers\_by\_interaction\_type\_and\_data\_model\_and\_resource() (pypath.core.interaction.Interaction method)@\spxentry{protein\_identifiers\_by\_interaction\_type\_and\_data\_model\_and\_resource()}\spxextra{pypath.core.interaction.Interaction method}}

\begin{fulllineitems}
\phantomsection\label{\detokenize{reference:pypath.core.interaction.Interaction.protein_identifiers_by_interaction_type_and_data_model_and_resource}}\pysiglinewithargsret{\sphinxbfcode{\sphinxupquote{protein\_identifiers\_by\_interaction\_type\_and\_data\_model\_and\_resource}}}{\emph{effect=None}, \emph{resources=None}, \emph{data\_model=None}, \emph{interaction\_type=None}, \emph{via=None}, \emph{references=None}}{}
Retrieves the entities involved in interactions matching the criteria.
It either returns both interacting entities in a \sphinxstyleemphasis{set} or an empty
\sphinxstyleemphasis{set}. This may not sound so useful at the level of this object but
becomes more useful once we want to collect entities having certain
kind of interactions across a series of \sphinxtitleref{Interaction} objects.
\begin{quote}\begin{description}
\item[{Parameters}] \leavevmode\begin{itemize}
\item {} 
\sphinxstyleliteralstrong{\sphinxupquote{entity\_type}} (\sphinxstyleliteralemphasis{\sphinxupquote{str}}) \textendash{} The type of the molecular entity. Possible values: \sphinxtitleref{protein},
\sphinxtitleref{complex}, \sphinxtitleref{mirna}, \sphinxtitleref{small\_molecule}.

\item {} 
\sphinxstyleliteralstrong{\sphinxupquote{return\_type}} (\sphinxstyleliteralemphasis{\sphinxupquote{str}}) \textendash{} The type of values to return. Default is
py:class:\sphinxcode{\sphinxupquote{pypath.entity.Entity}} objects, alternatives are
\sphinxcode{\sphinxupquote{labels}}  \sphinxcode{\sphinxupquote{identifiers}}.

\end{itemize}

\end{description}\end{quote}

\end{fulllineitems}

\index{protein\_identifiers\_by\_reference() (pypath.core.interaction.Interaction method)@\spxentry{protein\_identifiers\_by\_reference()}\spxextra{pypath.core.interaction.Interaction method}}

\begin{fulllineitems}
\phantomsection\label{\detokenize{reference:pypath.core.interaction.Interaction.protein_identifiers_by_reference}}\pysiglinewithargsret{\sphinxbfcode{\sphinxupquote{protein\_identifiers\_by\_reference}}}{\emph{effect=None}, \emph{resources=None}, \emph{data\_model=None}, \emph{interaction\_type=None}, \emph{via=None}, \emph{references=None}}{}
Retrieves the entities involved in interactions matching the criteria.
It either returns both interacting entities in a \sphinxstyleemphasis{set} or an empty
\sphinxstyleemphasis{set}. This may not sound so useful at the level of this object but
becomes more useful once we want to collect entities having certain
kind of interactions across a series of \sphinxtitleref{Interaction} objects.
\begin{quote}\begin{description}
\item[{Parameters}] \leavevmode\begin{itemize}
\item {} 
\sphinxstyleliteralstrong{\sphinxupquote{entity\_type}} (\sphinxstyleliteralemphasis{\sphinxupquote{str}}) \textendash{} The type of the molecular entity. Possible values: \sphinxtitleref{protein},
\sphinxtitleref{complex}, \sphinxtitleref{mirna}, \sphinxtitleref{small\_molecule}.

\item {} 
\sphinxstyleliteralstrong{\sphinxupquote{return\_type}} (\sphinxstyleliteralemphasis{\sphinxupquote{str}}) \textendash{} The type of values to return. Default is
py:class:\sphinxcode{\sphinxupquote{pypath.entity.Entity}} objects, alternatives are
\sphinxcode{\sphinxupquote{labels}}  \sphinxcode{\sphinxupquote{identifiers}}.

\end{itemize}

\end{description}\end{quote}

\end{fulllineitems}

\index{protein\_identifiers\_by\_resource() (pypath.core.interaction.Interaction method)@\spxentry{protein\_identifiers\_by\_resource()}\spxextra{pypath.core.interaction.Interaction method}}

\begin{fulllineitems}
\phantomsection\label{\detokenize{reference:pypath.core.interaction.Interaction.protein_identifiers_by_resource}}\pysiglinewithargsret{\sphinxbfcode{\sphinxupquote{protein\_identifiers\_by\_resource}}}{\emph{effect=None}, \emph{resources=None}, \emph{data\_model=None}, \emph{interaction\_type=None}, \emph{via=None}, \emph{references=None}}{}
Retrieves the entities involved in interactions matching the criteria.
It either returns both interacting entities in a \sphinxstyleemphasis{set} or an empty
\sphinxstyleemphasis{set}. This may not sound so useful at the level of this object but
becomes more useful once we want to collect entities having certain
kind of interactions across a series of \sphinxtitleref{Interaction} objects.
\begin{quote}\begin{description}
\item[{Parameters}] \leavevmode\begin{itemize}
\item {} 
\sphinxstyleliteralstrong{\sphinxupquote{entity\_type}} (\sphinxstyleliteralemphasis{\sphinxupquote{str}}) \textendash{} The type of the molecular entity. Possible values: \sphinxtitleref{protein},
\sphinxtitleref{complex}, \sphinxtitleref{mirna}, \sphinxtitleref{small\_molecule}.

\item {} 
\sphinxstyleliteralstrong{\sphinxupquote{return\_type}} (\sphinxstyleliteralemphasis{\sphinxupquote{str}}) \textendash{} The type of values to return. Default is
py:class:\sphinxcode{\sphinxupquote{pypath.entity.Entity}} objects, alternatives are
\sphinxcode{\sphinxupquote{labels}}  \sphinxcode{\sphinxupquote{identifiers}}.

\end{itemize}

\end{description}\end{quote}

\end{fulllineitems}

\index{protein\_labels\_by\_data\_model() (pypath.core.interaction.Interaction method)@\spxentry{protein\_labels\_by\_data\_model()}\spxextra{pypath.core.interaction.Interaction method}}

\begin{fulllineitems}
\phantomsection\label{\detokenize{reference:pypath.core.interaction.Interaction.protein_labels_by_data_model}}\pysiglinewithargsret{\sphinxbfcode{\sphinxupquote{protein\_labels\_by\_data\_model}}}{\emph{effect=None}, \emph{resources=None}, \emph{data\_model=None}, \emph{interaction\_type=None}, \emph{via=None}, \emph{references=None}}{}
Retrieves the entities involved in interactions matching the criteria.
It either returns both interacting entities in a \sphinxstyleemphasis{set} or an empty
\sphinxstyleemphasis{set}. This may not sound so useful at the level of this object but
becomes more useful once we want to collect entities having certain
kind of interactions across a series of \sphinxtitleref{Interaction} objects.
\begin{quote}\begin{description}
\item[{Parameters}] \leavevmode\begin{itemize}
\item {} 
\sphinxstyleliteralstrong{\sphinxupquote{entity\_type}} (\sphinxstyleliteralemphasis{\sphinxupquote{str}}) \textendash{} The type of the molecular entity. Possible values: \sphinxtitleref{protein},
\sphinxtitleref{complex}, \sphinxtitleref{mirna}, \sphinxtitleref{small\_molecule}.

\item {} 
\sphinxstyleliteralstrong{\sphinxupquote{return\_type}} (\sphinxstyleliteralemphasis{\sphinxupquote{str}}) \textendash{} The type of values to return. Default is
py:class:\sphinxcode{\sphinxupquote{pypath.entity.Entity}} objects, alternatives are
\sphinxcode{\sphinxupquote{labels}}  \sphinxcode{\sphinxupquote{identifiers}}.

\end{itemize}

\end{description}\end{quote}

\end{fulllineitems}

\index{protein\_labels\_by\_interaction\_type() (pypath.core.interaction.Interaction method)@\spxentry{protein\_labels\_by\_interaction\_type()}\spxextra{pypath.core.interaction.Interaction method}}

\begin{fulllineitems}
\phantomsection\label{\detokenize{reference:pypath.core.interaction.Interaction.protein_labels_by_interaction_type}}\pysiglinewithargsret{\sphinxbfcode{\sphinxupquote{protein\_labels\_by\_interaction\_type}}}{\emph{effect=None}, \emph{resources=None}, \emph{data\_model=None}, \emph{interaction\_type=None}, \emph{via=None}, \emph{references=None}}{}
Retrieves the entities involved in interactions matching the criteria.
It either returns both interacting entities in a \sphinxstyleemphasis{set} or an empty
\sphinxstyleemphasis{set}. This may not sound so useful at the level of this object but
becomes more useful once we want to collect entities having certain
kind of interactions across a series of \sphinxtitleref{Interaction} objects.
\begin{quote}\begin{description}
\item[{Parameters}] \leavevmode\begin{itemize}
\item {} 
\sphinxstyleliteralstrong{\sphinxupquote{entity\_type}} (\sphinxstyleliteralemphasis{\sphinxupquote{str}}) \textendash{} The type of the molecular entity. Possible values: \sphinxtitleref{protein},
\sphinxtitleref{complex}, \sphinxtitleref{mirna}, \sphinxtitleref{small\_molecule}.

\item {} 
\sphinxstyleliteralstrong{\sphinxupquote{return\_type}} (\sphinxstyleliteralemphasis{\sphinxupquote{str}}) \textendash{} The type of values to return. Default is
py:class:\sphinxcode{\sphinxupquote{pypath.entity.Entity}} objects, alternatives are
\sphinxcode{\sphinxupquote{labels}}  \sphinxcode{\sphinxupquote{identifiers}}.

\end{itemize}

\end{description}\end{quote}

\end{fulllineitems}

\index{protein\_labels\_by\_interaction\_type\_and\_data\_model() (pypath.core.interaction.Interaction method)@\spxentry{protein\_labels\_by\_interaction\_type\_and\_data\_model()}\spxextra{pypath.core.interaction.Interaction method}}

\begin{fulllineitems}
\phantomsection\label{\detokenize{reference:pypath.core.interaction.Interaction.protein_labels_by_interaction_type_and_data_model}}\pysiglinewithargsret{\sphinxbfcode{\sphinxupquote{protein\_labels\_by\_interaction\_type\_and\_data\_model}}}{\emph{effect=None}, \emph{resources=None}, \emph{data\_model=None}, \emph{interaction\_type=None}, \emph{via=None}, \emph{references=None}}{}
Retrieves the entities involved in interactions matching the criteria.
It either returns both interacting entities in a \sphinxstyleemphasis{set} or an empty
\sphinxstyleemphasis{set}. This may not sound so useful at the level of this object but
becomes more useful once we want to collect entities having certain
kind of interactions across a series of \sphinxtitleref{Interaction} objects.
\begin{quote}\begin{description}
\item[{Parameters}] \leavevmode\begin{itemize}
\item {} 
\sphinxstyleliteralstrong{\sphinxupquote{entity\_type}} (\sphinxstyleliteralemphasis{\sphinxupquote{str}}) \textendash{} The type of the molecular entity. Possible values: \sphinxtitleref{protein},
\sphinxtitleref{complex}, \sphinxtitleref{mirna}, \sphinxtitleref{small\_molecule}.

\item {} 
\sphinxstyleliteralstrong{\sphinxupquote{return\_type}} (\sphinxstyleliteralemphasis{\sphinxupquote{str}}) \textendash{} The type of values to return. Default is
py:class:\sphinxcode{\sphinxupquote{pypath.entity.Entity}} objects, alternatives are
\sphinxcode{\sphinxupquote{labels}}  \sphinxcode{\sphinxupquote{identifiers}}.

\end{itemize}

\end{description}\end{quote}

\end{fulllineitems}

\index{protein\_labels\_by\_interaction\_type\_and\_data\_model\_and\_resource() (pypath.core.interaction.Interaction method)@\spxentry{protein\_labels\_by\_interaction\_type\_and\_data\_model\_and\_resource()}\spxextra{pypath.core.interaction.Interaction method}}

\begin{fulllineitems}
\phantomsection\label{\detokenize{reference:pypath.core.interaction.Interaction.protein_labels_by_interaction_type_and_data_model_and_resource}}\pysiglinewithargsret{\sphinxbfcode{\sphinxupquote{protein\_labels\_by\_interaction\_type\_and\_data\_model\_and\_resource}}}{\emph{effect=None}, \emph{resources=None}, \emph{data\_model=None}, \emph{interaction\_type=None}, \emph{via=None}, \emph{references=None}}{}
Retrieves the entities involved in interactions matching the criteria.
It either returns both interacting entities in a \sphinxstyleemphasis{set} or an empty
\sphinxstyleemphasis{set}. This may not sound so useful at the level of this object but
becomes more useful once we want to collect entities having certain
kind of interactions across a series of \sphinxtitleref{Interaction} objects.
\begin{quote}\begin{description}
\item[{Parameters}] \leavevmode\begin{itemize}
\item {} 
\sphinxstyleliteralstrong{\sphinxupquote{entity\_type}} (\sphinxstyleliteralemphasis{\sphinxupquote{str}}) \textendash{} The type of the molecular entity. Possible values: \sphinxtitleref{protein},
\sphinxtitleref{complex}, \sphinxtitleref{mirna}, \sphinxtitleref{small\_molecule}.

\item {} 
\sphinxstyleliteralstrong{\sphinxupquote{return\_type}} (\sphinxstyleliteralemphasis{\sphinxupquote{str}}) \textendash{} The type of values to return. Default is
py:class:\sphinxcode{\sphinxupquote{pypath.entity.Entity}} objects, alternatives are
\sphinxcode{\sphinxupquote{labels}}  \sphinxcode{\sphinxupquote{identifiers}}.

\end{itemize}

\end{description}\end{quote}

\end{fulllineitems}

\index{protein\_labels\_by\_reference() (pypath.core.interaction.Interaction method)@\spxentry{protein\_labels\_by\_reference()}\spxextra{pypath.core.interaction.Interaction method}}

\begin{fulllineitems}
\phantomsection\label{\detokenize{reference:pypath.core.interaction.Interaction.protein_labels_by_reference}}\pysiglinewithargsret{\sphinxbfcode{\sphinxupquote{protein\_labels\_by\_reference}}}{\emph{effect=None}, \emph{resources=None}, \emph{data\_model=None}, \emph{interaction\_type=None}, \emph{via=None}, \emph{references=None}}{}
Retrieves the entities involved in interactions matching the criteria.
It either returns both interacting entities in a \sphinxstyleemphasis{set} or an empty
\sphinxstyleemphasis{set}. This may not sound so useful at the level of this object but
becomes more useful once we want to collect entities having certain
kind of interactions across a series of \sphinxtitleref{Interaction} objects.
\begin{quote}\begin{description}
\item[{Parameters}] \leavevmode\begin{itemize}
\item {} 
\sphinxstyleliteralstrong{\sphinxupquote{entity\_type}} (\sphinxstyleliteralemphasis{\sphinxupquote{str}}) \textendash{} The type of the molecular entity. Possible values: \sphinxtitleref{protein},
\sphinxtitleref{complex}, \sphinxtitleref{mirna}, \sphinxtitleref{small\_molecule}.

\item {} 
\sphinxstyleliteralstrong{\sphinxupquote{return\_type}} (\sphinxstyleliteralemphasis{\sphinxupquote{str}}) \textendash{} The type of values to return. Default is
py:class:\sphinxcode{\sphinxupquote{pypath.entity.Entity}} objects, alternatives are
\sphinxcode{\sphinxupquote{labels}}  \sphinxcode{\sphinxupquote{identifiers}}.

\end{itemize}

\end{description}\end{quote}

\end{fulllineitems}

\index{protein\_labels\_by\_resource() (pypath.core.interaction.Interaction method)@\spxentry{protein\_labels\_by\_resource()}\spxextra{pypath.core.interaction.Interaction method}}

\begin{fulllineitems}
\phantomsection\label{\detokenize{reference:pypath.core.interaction.Interaction.protein_labels_by_resource}}\pysiglinewithargsret{\sphinxbfcode{\sphinxupquote{protein\_labels\_by\_resource}}}{\emph{effect=None}, \emph{resources=None}, \emph{data\_model=None}, \emph{interaction\_type=None}, \emph{via=None}, \emph{references=None}}{}
Retrieves the entities involved in interactions matching the criteria.
It either returns both interacting entities in a \sphinxstyleemphasis{set} or an empty
\sphinxstyleemphasis{set}. This may not sound so useful at the level of this object but
becomes more useful once we want to collect entities having certain
kind of interactions across a series of \sphinxtitleref{Interaction} objects.
\begin{quote}\begin{description}
\item[{Parameters}] \leavevmode\begin{itemize}
\item {} 
\sphinxstyleliteralstrong{\sphinxupquote{entity\_type}} (\sphinxstyleliteralemphasis{\sphinxupquote{str}}) \textendash{} The type of the molecular entity. Possible values: \sphinxtitleref{protein},
\sphinxtitleref{complex}, \sphinxtitleref{mirna}, \sphinxtitleref{small\_molecule}.

\item {} 
\sphinxstyleliteralstrong{\sphinxupquote{return\_type}} (\sphinxstyleliteralemphasis{\sphinxupquote{str}}) \textendash{} The type of values to return. Default is
py:class:\sphinxcode{\sphinxupquote{pypath.entity.Entity}} objects, alternatives are
\sphinxcode{\sphinxupquote{labels}}  \sphinxcode{\sphinxupquote{identifiers}}.

\end{itemize}

\end{description}\end{quote}

\end{fulllineitems}

\index{proteins\_by\_data\_model() (pypath.core.interaction.Interaction method)@\spxentry{proteins\_by\_data\_model()}\spxextra{pypath.core.interaction.Interaction method}}

\begin{fulllineitems}
\phantomsection\label{\detokenize{reference:pypath.core.interaction.Interaction.proteins_by_data_model}}\pysiglinewithargsret{\sphinxbfcode{\sphinxupquote{proteins\_by\_data\_model}}}{\emph{effect=None}, \emph{resources=None}, \emph{data\_model=None}, \emph{interaction\_type=None}, \emph{via=None}, \emph{references=None}}{}
Retrieves the entities involved in interactions matching the criteria.
It either returns both interacting entities in a \sphinxstyleemphasis{set} or an empty
\sphinxstyleemphasis{set}. This may not sound so useful at the level of this object but
becomes more useful once we want to collect entities having certain
kind of interactions across a series of \sphinxtitleref{Interaction} objects.
\begin{quote}\begin{description}
\item[{Parameters}] \leavevmode\begin{itemize}
\item {} 
\sphinxstyleliteralstrong{\sphinxupquote{entity\_type}} (\sphinxstyleliteralemphasis{\sphinxupquote{str}}) \textendash{} The type of the molecular entity. Possible values: \sphinxtitleref{protein},
\sphinxtitleref{complex}, \sphinxtitleref{mirna}, \sphinxtitleref{small\_molecule}.

\item {} 
\sphinxstyleliteralstrong{\sphinxupquote{return\_type}} (\sphinxstyleliteralemphasis{\sphinxupquote{str}}) \textendash{} The type of values to return. Default is
py:class:\sphinxcode{\sphinxupquote{pypath.entity.Entity}} objects, alternatives are
\sphinxcode{\sphinxupquote{labels}}  \sphinxcode{\sphinxupquote{identifiers}}.

\end{itemize}

\end{description}\end{quote}

\end{fulllineitems}

\index{proteins\_by\_interaction\_type() (pypath.core.interaction.Interaction method)@\spxentry{proteins\_by\_interaction\_type()}\spxextra{pypath.core.interaction.Interaction method}}

\begin{fulllineitems}
\phantomsection\label{\detokenize{reference:pypath.core.interaction.Interaction.proteins_by_interaction_type}}\pysiglinewithargsret{\sphinxbfcode{\sphinxupquote{proteins\_by\_interaction\_type}}}{\emph{effect=None}, \emph{resources=None}, \emph{data\_model=None}, \emph{interaction\_type=None}, \emph{via=None}, \emph{references=None}}{}
Retrieves the entities involved in interactions matching the criteria.
It either returns both interacting entities in a \sphinxstyleemphasis{set} or an empty
\sphinxstyleemphasis{set}. This may not sound so useful at the level of this object but
becomes more useful once we want to collect entities having certain
kind of interactions across a series of \sphinxtitleref{Interaction} objects.
\begin{quote}\begin{description}
\item[{Parameters}] \leavevmode\begin{itemize}
\item {} 
\sphinxstyleliteralstrong{\sphinxupquote{entity\_type}} (\sphinxstyleliteralemphasis{\sphinxupquote{str}}) \textendash{} The type of the molecular entity. Possible values: \sphinxtitleref{protein},
\sphinxtitleref{complex}, \sphinxtitleref{mirna}, \sphinxtitleref{small\_molecule}.

\item {} 
\sphinxstyleliteralstrong{\sphinxupquote{return\_type}} (\sphinxstyleliteralemphasis{\sphinxupquote{str}}) \textendash{} The type of values to return. Default is
py:class:\sphinxcode{\sphinxupquote{pypath.entity.Entity}} objects, alternatives are
\sphinxcode{\sphinxupquote{labels}}  \sphinxcode{\sphinxupquote{identifiers}}.

\end{itemize}

\end{description}\end{quote}

\end{fulllineitems}

\index{proteins\_by\_interaction\_type\_and\_data\_model() (pypath.core.interaction.Interaction method)@\spxentry{proteins\_by\_interaction\_type\_and\_data\_model()}\spxextra{pypath.core.interaction.Interaction method}}

\begin{fulllineitems}
\phantomsection\label{\detokenize{reference:pypath.core.interaction.Interaction.proteins_by_interaction_type_and_data_model}}\pysiglinewithargsret{\sphinxbfcode{\sphinxupquote{proteins\_by\_interaction\_type\_and\_data\_model}}}{\emph{effect=None}, \emph{resources=None}, \emph{data\_model=None}, \emph{interaction\_type=None}, \emph{via=None}, \emph{references=None}}{}
Retrieves the entities involved in interactions matching the criteria.
It either returns both interacting entities in a \sphinxstyleemphasis{set} or an empty
\sphinxstyleemphasis{set}. This may not sound so useful at the level of this object but
becomes more useful once we want to collect entities having certain
kind of interactions across a series of \sphinxtitleref{Interaction} objects.
\begin{quote}\begin{description}
\item[{Parameters}] \leavevmode\begin{itemize}
\item {} 
\sphinxstyleliteralstrong{\sphinxupquote{entity\_type}} (\sphinxstyleliteralemphasis{\sphinxupquote{str}}) \textendash{} The type of the molecular entity. Possible values: \sphinxtitleref{protein},
\sphinxtitleref{complex}, \sphinxtitleref{mirna}, \sphinxtitleref{small\_molecule}.

\item {} 
\sphinxstyleliteralstrong{\sphinxupquote{return\_type}} (\sphinxstyleliteralemphasis{\sphinxupquote{str}}) \textendash{} The type of values to return. Default is
py:class:\sphinxcode{\sphinxupquote{pypath.entity.Entity}} objects, alternatives are
\sphinxcode{\sphinxupquote{labels}}  \sphinxcode{\sphinxupquote{identifiers}}.

\end{itemize}

\end{description}\end{quote}

\end{fulllineitems}

\index{proteins\_by\_interaction\_type\_and\_data\_model\_and\_resource() (pypath.core.interaction.Interaction method)@\spxentry{proteins\_by\_interaction\_type\_and\_data\_model\_and\_resource()}\spxextra{pypath.core.interaction.Interaction method}}

\begin{fulllineitems}
\phantomsection\label{\detokenize{reference:pypath.core.interaction.Interaction.proteins_by_interaction_type_and_data_model_and_resource}}\pysiglinewithargsret{\sphinxbfcode{\sphinxupquote{proteins\_by\_interaction\_type\_and\_data\_model\_and\_resource}}}{\emph{effect=None}, \emph{resources=None}, \emph{data\_model=None}, \emph{interaction\_type=None}, \emph{via=None}, \emph{references=None}}{}
Retrieves the entities involved in interactions matching the criteria.
It either returns both interacting entities in a \sphinxstyleemphasis{set} or an empty
\sphinxstyleemphasis{set}. This may not sound so useful at the level of this object but
becomes more useful once we want to collect entities having certain
kind of interactions across a series of \sphinxtitleref{Interaction} objects.
\begin{quote}\begin{description}
\item[{Parameters}] \leavevmode\begin{itemize}
\item {} 
\sphinxstyleliteralstrong{\sphinxupquote{entity\_type}} (\sphinxstyleliteralemphasis{\sphinxupquote{str}}) \textendash{} The type of the molecular entity. Possible values: \sphinxtitleref{protein},
\sphinxtitleref{complex}, \sphinxtitleref{mirna}, \sphinxtitleref{small\_molecule}.

\item {} 
\sphinxstyleliteralstrong{\sphinxupquote{return\_type}} (\sphinxstyleliteralemphasis{\sphinxupquote{str}}) \textendash{} The type of values to return. Default is
py:class:\sphinxcode{\sphinxupquote{pypath.entity.Entity}} objects, alternatives are
\sphinxcode{\sphinxupquote{labels}}  \sphinxcode{\sphinxupquote{identifiers}}.

\end{itemize}

\end{description}\end{quote}

\end{fulllineitems}

\index{proteins\_by\_reference() (pypath.core.interaction.Interaction method)@\spxentry{proteins\_by\_reference()}\spxextra{pypath.core.interaction.Interaction method}}

\begin{fulllineitems}
\phantomsection\label{\detokenize{reference:pypath.core.interaction.Interaction.proteins_by_reference}}\pysiglinewithargsret{\sphinxbfcode{\sphinxupquote{proteins\_by\_reference}}}{\emph{effect=None}, \emph{resources=None}, \emph{data\_model=None}, \emph{interaction\_type=None}, \emph{via=None}, \emph{references=None}}{}
Retrieves the entities involved in interactions matching the criteria.
It either returns both interacting entities in a \sphinxstyleemphasis{set} or an empty
\sphinxstyleemphasis{set}. This may not sound so useful at the level of this object but
becomes more useful once we want to collect entities having certain
kind of interactions across a series of \sphinxtitleref{Interaction} objects.
\begin{quote}\begin{description}
\item[{Parameters}] \leavevmode\begin{itemize}
\item {} 
\sphinxstyleliteralstrong{\sphinxupquote{entity\_type}} (\sphinxstyleliteralemphasis{\sphinxupquote{str}}) \textendash{} The type of the molecular entity. Possible values: \sphinxtitleref{protein},
\sphinxtitleref{complex}, \sphinxtitleref{mirna}, \sphinxtitleref{small\_molecule}.

\item {} 
\sphinxstyleliteralstrong{\sphinxupquote{return\_type}} (\sphinxstyleliteralemphasis{\sphinxupquote{str}}) \textendash{} The type of values to return. Default is
py:class:\sphinxcode{\sphinxupquote{pypath.entity.Entity}} objects, alternatives are
\sphinxcode{\sphinxupquote{labels}}  \sphinxcode{\sphinxupquote{identifiers}}.

\end{itemize}

\end{description}\end{quote}

\end{fulllineitems}

\index{proteins\_by\_resource() (pypath.core.interaction.Interaction method)@\spxentry{proteins\_by\_resource()}\spxextra{pypath.core.interaction.Interaction method}}

\begin{fulllineitems}
\phantomsection\label{\detokenize{reference:pypath.core.interaction.Interaction.proteins_by_resource}}\pysiglinewithargsret{\sphinxbfcode{\sphinxupquote{proteins\_by\_resource}}}{\emph{effect=None}, \emph{resources=None}, \emph{data\_model=None}, \emph{interaction\_type=None}, \emph{via=None}, \emph{references=None}}{}
Retrieves the entities involved in interactions matching the criteria.
It either returns both interacting entities in a \sphinxstyleemphasis{set} or an empty
\sphinxstyleemphasis{set}. This may not sound so useful at the level of this object but
becomes more useful once we want to collect entities having certain
kind of interactions across a series of \sphinxtitleref{Interaction} objects.
\begin{quote}\begin{description}
\item[{Parameters}] \leavevmode\begin{itemize}
\item {} 
\sphinxstyleliteralstrong{\sphinxupquote{entity\_type}} (\sphinxstyleliteralemphasis{\sphinxupquote{str}}) \textendash{} The type of the molecular entity. Possible values: \sphinxtitleref{protein},
\sphinxtitleref{complex}, \sphinxtitleref{mirna}, \sphinxtitleref{small\_molecule}.

\item {} 
\sphinxstyleliteralstrong{\sphinxupquote{return\_type}} (\sphinxstyleliteralemphasis{\sphinxupquote{str}}) \textendash{} The type of values to return. Default is
py:class:\sphinxcode{\sphinxupquote{pypath.entity.Entity}} objects, alternatives are
\sphinxcode{\sphinxupquote{labels}}  \sphinxcode{\sphinxupquote{identifiers}}.

\end{itemize}

\end{description}\end{quote}

\end{fulllineitems}

\index{references\_by\_data\_model() (pypath.core.interaction.Interaction method)@\spxentry{references\_by\_data\_model()}\spxextra{pypath.core.interaction.Interaction method}}

\begin{fulllineitems}
\phantomsection\label{\detokenize{reference:pypath.core.interaction.Interaction.references_by_data_model}}\pysiglinewithargsret{\sphinxbfcode{\sphinxupquote{references\_by\_data\_model}}}{\emph{effect=None}, \emph{resources=None}, \emph{data\_model=None}, \emph{interaction\_type=None}, \emph{via=None}, \emph{references=None}}{}
Retrieves references matching the criteria.

\end{fulllineitems}

\index{references\_by\_interaction\_type() (pypath.core.interaction.Interaction method)@\spxentry{references\_by\_interaction\_type()}\spxextra{pypath.core.interaction.Interaction method}}

\begin{fulllineitems}
\phantomsection\label{\detokenize{reference:pypath.core.interaction.Interaction.references_by_interaction_type}}\pysiglinewithargsret{\sphinxbfcode{\sphinxupquote{references\_by\_interaction\_type}}}{\emph{effect=None}, \emph{resources=None}, \emph{data\_model=None}, \emph{interaction\_type=None}, \emph{via=None}, \emph{references=None}}{}
Retrieves references matching the criteria.

\end{fulllineitems}

\index{references\_by\_interaction\_type\_and\_data\_model() (pypath.core.interaction.Interaction method)@\spxentry{references\_by\_interaction\_type\_and\_data\_model()}\spxextra{pypath.core.interaction.Interaction method}}

\begin{fulllineitems}
\phantomsection\label{\detokenize{reference:pypath.core.interaction.Interaction.references_by_interaction_type_and_data_model}}\pysiglinewithargsret{\sphinxbfcode{\sphinxupquote{references\_by\_interaction\_type\_and\_data\_model}}}{\emph{effect=None}, \emph{resources=None}, \emph{data\_model=None}, \emph{interaction\_type=None}, \emph{via=None}, \emph{references=None}}{}
Retrieves references matching the criteria.

\end{fulllineitems}

\index{references\_by\_interaction\_type\_and\_data\_model\_and\_resource() (pypath.core.interaction.Interaction method)@\spxentry{references\_by\_interaction\_type\_and\_data\_model\_and\_resource()}\spxextra{pypath.core.interaction.Interaction method}}

\begin{fulllineitems}
\phantomsection\label{\detokenize{reference:pypath.core.interaction.Interaction.references_by_interaction_type_and_data_model_and_resource}}\pysiglinewithargsret{\sphinxbfcode{\sphinxupquote{references\_by\_interaction\_type\_and\_data\_model\_and\_resource}}}{\emph{effect=None}, \emph{resources=None}, \emph{data\_model=None}, \emph{interaction\_type=None}, \emph{via=None}, \emph{references=None}}{}
Retrieves references matching the criteria.

\end{fulllineitems}

\index{references\_by\_reference() (pypath.core.interaction.Interaction method)@\spxentry{references\_by\_reference()}\spxextra{pypath.core.interaction.Interaction method}}

\begin{fulllineitems}
\phantomsection\label{\detokenize{reference:pypath.core.interaction.Interaction.references_by_reference}}\pysiglinewithargsret{\sphinxbfcode{\sphinxupquote{references\_by\_reference}}}{\emph{effect=None}, \emph{resources=None}, \emph{data\_model=None}, \emph{interaction\_type=None}, \emph{via=None}, \emph{references=None}}{}
Retrieves references matching the criteria.

\end{fulllineitems}

\index{references\_by\_resource() (pypath.core.interaction.Interaction method)@\spxentry{references\_by\_resource()}\spxextra{pypath.core.interaction.Interaction method}}

\begin{fulllineitems}
\phantomsection\label{\detokenize{reference:pypath.core.interaction.Interaction.references_by_resource}}\pysiglinewithargsret{\sphinxbfcode{\sphinxupquote{references\_by\_resource}}}{\emph{effect=None}, \emph{resources=None}, \emph{data\_model=None}, \emph{interaction\_type=None}, \emph{via=None}, \emph{references=None}}{}
Retrieves references matching the criteria.

\end{fulllineitems}

\index{reload() (pypath.core.interaction.Interaction method)@\spxentry{reload()}\spxextra{pypath.core.interaction.Interaction method}}

\begin{fulllineitems}
\phantomsection\label{\detokenize{reference:pypath.core.interaction.Interaction.reload}}\pysiglinewithargsret{\sphinxbfcode{\sphinxupquote{reload}}}{}{}
Reloads the object from the module level.

\end{fulllineitems}

\index{resource\_names\_by\_data\_model() (pypath.core.interaction.Interaction method)@\spxentry{resource\_names\_by\_data\_model()}\spxextra{pypath.core.interaction.Interaction method}}

\begin{fulllineitems}
\phantomsection\label{\detokenize{reference:pypath.core.interaction.Interaction.resource_names_by_data_model}}\pysiglinewithargsret{\sphinxbfcode{\sphinxupquote{resource\_names\_by\_data\_model}}}{\emph{effect=None}, \emph{resources=None}, \emph{data\_model=None}, \emph{interaction\_type=None}, \emph{via=None}, \emph{references=None}}{}
Retrieves resource names matching the criteria.

\end{fulllineitems}

\index{resource\_names\_by\_interaction\_type() (pypath.core.interaction.Interaction method)@\spxentry{resource\_names\_by\_interaction\_type()}\spxextra{pypath.core.interaction.Interaction method}}

\begin{fulllineitems}
\phantomsection\label{\detokenize{reference:pypath.core.interaction.Interaction.resource_names_by_interaction_type}}\pysiglinewithargsret{\sphinxbfcode{\sphinxupquote{resource\_names\_by\_interaction\_type}}}{\emph{effect=None}, \emph{resources=None}, \emph{data\_model=None}, \emph{interaction\_type=None}, \emph{via=None}, \emph{references=None}}{}
Retrieves resource names matching the criteria.

\end{fulllineitems}

\index{resource\_names\_by\_interaction\_type\_and\_data\_model() (pypath.core.interaction.Interaction method)@\spxentry{resource\_names\_by\_interaction\_type\_and\_data\_model()}\spxextra{pypath.core.interaction.Interaction method}}

\begin{fulllineitems}
\phantomsection\label{\detokenize{reference:pypath.core.interaction.Interaction.resource_names_by_interaction_type_and_data_model}}\pysiglinewithargsret{\sphinxbfcode{\sphinxupquote{resource\_names\_by\_interaction\_type\_and\_data\_model}}}{\emph{effect=None}, \emph{resources=None}, \emph{data\_model=None}, \emph{interaction\_type=None}, \emph{via=None}, \emph{references=None}}{}
Retrieves resource names matching the criteria.

\end{fulllineitems}

\index{resource\_names\_by\_interaction\_type\_and\_data\_model\_and\_resource() (pypath.core.interaction.Interaction method)@\spxentry{resource\_names\_by\_interaction\_type\_and\_data\_model\_and\_resource()}\spxextra{pypath.core.interaction.Interaction method}}

\begin{fulllineitems}
\phantomsection\label{\detokenize{reference:pypath.core.interaction.Interaction.resource_names_by_interaction_type_and_data_model_and_resource}}\pysiglinewithargsret{\sphinxbfcode{\sphinxupquote{resource\_names\_by\_interaction\_type\_and\_data\_model\_and\_resource}}}{\emph{effect=None}, \emph{resources=None}, \emph{data\_model=None}, \emph{interaction\_type=None}, \emph{via=None}, \emph{references=None}}{}
Retrieves resource names matching the criteria.

\end{fulllineitems}

\index{resource\_names\_by\_reference() (pypath.core.interaction.Interaction method)@\spxentry{resource\_names\_by\_reference()}\spxextra{pypath.core.interaction.Interaction method}}

\begin{fulllineitems}
\phantomsection\label{\detokenize{reference:pypath.core.interaction.Interaction.resource_names_by_reference}}\pysiglinewithargsret{\sphinxbfcode{\sphinxupquote{resource\_names\_by\_reference}}}{\emph{effect=None}, \emph{resources=None}, \emph{data\_model=None}, \emph{interaction\_type=None}, \emph{via=None}, \emph{references=None}}{}
Retrieves resource names matching the criteria.

\end{fulllineitems}

\index{resource\_names\_by\_resource() (pypath.core.interaction.Interaction method)@\spxentry{resource\_names\_by\_resource()}\spxextra{pypath.core.interaction.Interaction method}}

\begin{fulllineitems}
\phantomsection\label{\detokenize{reference:pypath.core.interaction.Interaction.resource_names_by_resource}}\pysiglinewithargsret{\sphinxbfcode{\sphinxupquote{resource\_names\_by\_resource}}}{\emph{effect=None}, \emph{resources=None}, \emph{data\_model=None}, \emph{interaction\_type=None}, \emph{via=None}, \emph{references=None}}{}
Retrieves resource names matching the criteria.

\end{fulllineitems}

\index{resource\_names\_via\_by\_data\_model() (pypath.core.interaction.Interaction method)@\spxentry{resource\_names\_via\_by\_data\_model()}\spxextra{pypath.core.interaction.Interaction method}}

\begin{fulllineitems}
\phantomsection\label{\detokenize{reference:pypath.core.interaction.Interaction.resource_names_via_by_data_model}}\pysiglinewithargsret{\sphinxbfcode{\sphinxupquote{resource\_names\_via\_by\_data\_model}}}{\emph{effect=None}, \emph{resources=None}, \emph{data\_model=None}, \emph{interaction\_type=None}, \emph{via=None}, \emph{references=None}}{}
Retrieves resource names via matching the criteria.

\end{fulllineitems}

\index{resource\_names\_via\_by\_interaction\_type() (pypath.core.interaction.Interaction method)@\spxentry{resource\_names\_via\_by\_interaction\_type()}\spxextra{pypath.core.interaction.Interaction method}}

\begin{fulllineitems}
\phantomsection\label{\detokenize{reference:pypath.core.interaction.Interaction.resource_names_via_by_interaction_type}}\pysiglinewithargsret{\sphinxbfcode{\sphinxupquote{resource\_names\_via\_by\_interaction\_type}}}{\emph{effect=None}, \emph{resources=None}, \emph{data\_model=None}, \emph{interaction\_type=None}, \emph{via=None}, \emph{references=None}}{}
Retrieves resource names via matching the criteria.

\end{fulllineitems}

\index{resource\_names\_via\_by\_interaction\_type\_and\_data\_model() (pypath.core.interaction.Interaction method)@\spxentry{resource\_names\_via\_by\_interaction\_type\_and\_data\_model()}\spxextra{pypath.core.interaction.Interaction method}}

\begin{fulllineitems}
\phantomsection\label{\detokenize{reference:pypath.core.interaction.Interaction.resource_names_via_by_interaction_type_and_data_model}}\pysiglinewithargsret{\sphinxbfcode{\sphinxupquote{resource\_names\_via\_by\_interaction\_type\_and\_data\_model}}}{\emph{effect=None}, \emph{resources=None}, \emph{data\_model=None}, \emph{interaction\_type=None}, \emph{via=None}, \emph{references=None}}{}
Retrieves resource names via matching the criteria.

\end{fulllineitems}

\index{resource\_names\_via\_by\_interaction\_type\_and\_data\_model\_and\_resource() (pypath.core.interaction.Interaction method)@\spxentry{resource\_names\_via\_by\_interaction\_type\_and\_data\_model\_and\_resource()}\spxextra{pypath.core.interaction.Interaction method}}

\begin{fulllineitems}
\phantomsection\label{\detokenize{reference:pypath.core.interaction.Interaction.resource_names_via_by_interaction_type_and_data_model_and_resource}}\pysiglinewithargsret{\sphinxbfcode{\sphinxupquote{resource\_names\_via\_by\_interaction\_type\_and\_data\_model\_and\_resource}}}{\emph{effect=None}, \emph{resources=None}, \emph{data\_model=None}, \emph{interaction\_type=None}, \emph{via=None}, \emph{references=None}}{}
Retrieves resource names via matching the criteria.

\end{fulllineitems}

\index{resource\_names\_via\_by\_reference() (pypath.core.interaction.Interaction method)@\spxentry{resource\_names\_via\_by\_reference()}\spxextra{pypath.core.interaction.Interaction method}}

\begin{fulllineitems}
\phantomsection\label{\detokenize{reference:pypath.core.interaction.Interaction.resource_names_via_by_reference}}\pysiglinewithargsret{\sphinxbfcode{\sphinxupquote{resource\_names\_via\_by\_reference}}}{\emph{effect=None}, \emph{resources=None}, \emph{data\_model=None}, \emph{interaction\_type=None}, \emph{via=None}, \emph{references=None}}{}
Retrieves resource names via matching the criteria.

\end{fulllineitems}

\index{resource\_names\_via\_by\_resource() (pypath.core.interaction.Interaction method)@\spxentry{resource\_names\_via\_by\_resource()}\spxextra{pypath.core.interaction.Interaction method}}

\begin{fulllineitems}
\phantomsection\label{\detokenize{reference:pypath.core.interaction.Interaction.resource_names_via_by_resource}}\pysiglinewithargsret{\sphinxbfcode{\sphinxupquote{resource\_names\_via\_by\_resource}}}{\emph{effect=None}, \emph{resources=None}, \emph{data\_model=None}, \emph{interaction\_type=None}, \emph{via=None}, \emph{references=None}}{}
Retrieves resource names via matching the criteria.

\end{fulllineitems}

\index{resources\_a\_b() (pypath.core.interaction.Interaction method)@\spxentry{resources\_a\_b()}\spxextra{pypath.core.interaction.Interaction method}}

\begin{fulllineitems}
\phantomsection\label{\detokenize{reference:pypath.core.interaction.Interaction.resources_a_b}}\pysiglinewithargsret{\sphinxbfcode{\sphinxupquote{resources\_a\_b}}}{\emph{resources=False}, \emph{evidences=False}, \emph{resource\_names=False}, \emph{sources=False}}{}
Retrieves the list of resources for the \sphinxcode{\sphinxupquote{a\_b}}
direction.
\begin{quote}\begin{description}
\item[{Returns}] \leavevmode
(\sphinxstyleemphasis{set}) \textendash{} Contains the names of the sources supporting the
\sphinxcode{\sphinxupquote{a\_b}} directionality of the edge.

\end{description}\end{quote}

\end{fulllineitems}

\index{resources\_b\_a() (pypath.core.interaction.Interaction method)@\spxentry{resources\_b\_a()}\spxextra{pypath.core.interaction.Interaction method}}

\begin{fulllineitems}
\phantomsection\label{\detokenize{reference:pypath.core.interaction.Interaction.resources_b_a}}\pysiglinewithargsret{\sphinxbfcode{\sphinxupquote{resources\_b\_a}}}{\emph{resources=False}, \emph{evidences=False}, \emph{resource\_names=False}, \emph{sources=False}}{}
Retrieves the list of sources for the \sphinxcode{\sphinxupquote{b\_a}} direction.
\begin{quote}\begin{description}
\item[{Returns}] \leavevmode
(\sphinxstyleemphasis{set}) \textendash{} Contains the names of the sources supporting the
\sphinxcode{\sphinxupquote{b\_a}} directionality of the edge.

\end{description}\end{quote}

\end{fulllineitems}

\index{resources\_by\_data\_model() (pypath.core.interaction.Interaction method)@\spxentry{resources\_by\_data\_model()}\spxextra{pypath.core.interaction.Interaction method}}

\begin{fulllineitems}
\phantomsection\label{\detokenize{reference:pypath.core.interaction.Interaction.resources_by_data_model}}\pysiglinewithargsret{\sphinxbfcode{\sphinxupquote{resources\_by\_data\_model}}}{\emph{effect=None}, \emph{resources=None}, \emph{data\_model=None}, \emph{interaction\_type=None}, \emph{via=None}, \emph{references=None}}{}
Retrieves resources matching the criteria.

\end{fulllineitems}

\index{resources\_by\_interaction\_type() (pypath.core.interaction.Interaction method)@\spxentry{resources\_by\_interaction\_type()}\spxextra{pypath.core.interaction.Interaction method}}

\begin{fulllineitems}
\phantomsection\label{\detokenize{reference:pypath.core.interaction.Interaction.resources_by_interaction_type}}\pysiglinewithargsret{\sphinxbfcode{\sphinxupquote{resources\_by\_interaction\_type}}}{\emph{effect=None}, \emph{resources=None}, \emph{data\_model=None}, \emph{interaction\_type=None}, \emph{via=None}, \emph{references=None}}{}
Retrieves resources matching the criteria.

\end{fulllineitems}

\index{resources\_by\_interaction\_type\_and\_data\_model() (pypath.core.interaction.Interaction method)@\spxentry{resources\_by\_interaction\_type\_and\_data\_model()}\spxextra{pypath.core.interaction.Interaction method}}

\begin{fulllineitems}
\phantomsection\label{\detokenize{reference:pypath.core.interaction.Interaction.resources_by_interaction_type_and_data_model}}\pysiglinewithargsret{\sphinxbfcode{\sphinxupquote{resources\_by\_interaction\_type\_and\_data\_model}}}{\emph{effect=None}, \emph{resources=None}, \emph{data\_model=None}, \emph{interaction\_type=None}, \emph{via=None}, \emph{references=None}}{}
Retrieves resources matching the criteria.

\end{fulllineitems}

\index{resources\_by\_interaction\_type\_and\_data\_model\_and\_resource() (pypath.core.interaction.Interaction method)@\spxentry{resources\_by\_interaction\_type\_and\_data\_model\_and\_resource()}\spxextra{pypath.core.interaction.Interaction method}}

\begin{fulllineitems}
\phantomsection\label{\detokenize{reference:pypath.core.interaction.Interaction.resources_by_interaction_type_and_data_model_and_resource}}\pysiglinewithargsret{\sphinxbfcode{\sphinxupquote{resources\_by\_interaction\_type\_and\_data\_model\_and\_resource}}}{\emph{effect=None}, \emph{resources=None}, \emph{data\_model=None}, \emph{interaction\_type=None}, \emph{via=None}, \emph{references=None}}{}
Retrieves resources matching the criteria.

\end{fulllineitems}

\index{resources\_by\_reference() (pypath.core.interaction.Interaction method)@\spxentry{resources\_by\_reference()}\spxextra{pypath.core.interaction.Interaction method}}

\begin{fulllineitems}
\phantomsection\label{\detokenize{reference:pypath.core.interaction.Interaction.resources_by_reference}}\pysiglinewithargsret{\sphinxbfcode{\sphinxupquote{resources\_by\_reference}}}{\emph{effect=None}, \emph{resources=None}, \emph{data\_model=None}, \emph{interaction\_type=None}, \emph{via=None}, \emph{references=None}}{}
Retrieves resources matching the criteria.

\end{fulllineitems}

\index{resources\_by\_resource() (pypath.core.interaction.Interaction method)@\spxentry{resources\_by\_resource()}\spxextra{pypath.core.interaction.Interaction method}}

\begin{fulllineitems}
\phantomsection\label{\detokenize{reference:pypath.core.interaction.Interaction.resources_by_resource}}\pysiglinewithargsret{\sphinxbfcode{\sphinxupquote{resources\_by\_resource}}}{\emph{effect=None}, \emph{resources=None}, \emph{data\_model=None}, \emph{interaction\_type=None}, \emph{via=None}, \emph{references=None}}{}
Retrieves resources matching the criteria.

\end{fulllineitems}

\index{resources\_undirected() (pypath.core.interaction.Interaction method)@\spxentry{resources\_undirected()}\spxextra{pypath.core.interaction.Interaction method}}

\begin{fulllineitems}
\phantomsection\label{\detokenize{reference:pypath.core.interaction.Interaction.resources_undirected}}\pysiglinewithargsret{\sphinxbfcode{\sphinxupquote{resources\_undirected}}}{\emph{resources=False}, \emph{evidences=False}, \emph{resource\_names=False}, \emph{sources=False}}{}
Retrieves the list of resources without directed information.
\begin{quote}\begin{description}
\item[{Returns}] \leavevmode
(\sphinxstyleemphasis{set}) \textendash{} Contains the names of the sources supporting the
edge presence but without specific directionality
information.

\end{description}\end{quote}

\end{fulllineitems}

\index{resources\_via\_by\_data\_model() (pypath.core.interaction.Interaction method)@\spxentry{resources\_via\_by\_data\_model()}\spxextra{pypath.core.interaction.Interaction method}}

\begin{fulllineitems}
\phantomsection\label{\detokenize{reference:pypath.core.interaction.Interaction.resources_via_by_data_model}}\pysiglinewithargsret{\sphinxbfcode{\sphinxupquote{resources\_via\_by\_data\_model}}}{\emph{effect=None}, \emph{resources=None}, \emph{data\_model=None}, \emph{interaction\_type=None}, \emph{via=None}, \emph{references=None}}{}
Retrieves resources via matching the criteria.

\end{fulllineitems}

\index{resources\_via\_by\_interaction\_type() (pypath.core.interaction.Interaction method)@\spxentry{resources\_via\_by\_interaction\_type()}\spxextra{pypath.core.interaction.Interaction method}}

\begin{fulllineitems}
\phantomsection\label{\detokenize{reference:pypath.core.interaction.Interaction.resources_via_by_interaction_type}}\pysiglinewithargsret{\sphinxbfcode{\sphinxupquote{resources\_via\_by\_interaction\_type}}}{\emph{effect=None}, \emph{resources=None}, \emph{data\_model=None}, \emph{interaction\_type=None}, \emph{via=None}, \emph{references=None}}{}
Retrieves resources via matching the criteria.

\end{fulllineitems}

\index{resources\_via\_by\_interaction\_type\_and\_data\_model() (pypath.core.interaction.Interaction method)@\spxentry{resources\_via\_by\_interaction\_type\_and\_data\_model()}\spxextra{pypath.core.interaction.Interaction method}}

\begin{fulllineitems}
\phantomsection\label{\detokenize{reference:pypath.core.interaction.Interaction.resources_via_by_interaction_type_and_data_model}}\pysiglinewithargsret{\sphinxbfcode{\sphinxupquote{resources\_via\_by\_interaction\_type\_and\_data\_model}}}{\emph{effect=None}, \emph{resources=None}, \emph{data\_model=None}, \emph{interaction\_type=None}, \emph{via=None}, \emph{references=None}}{}
Retrieves resources via matching the criteria.

\end{fulllineitems}

\index{resources\_via\_by\_interaction\_type\_and\_data\_model\_and\_resource() (pypath.core.interaction.Interaction method)@\spxentry{resources\_via\_by\_interaction\_type\_and\_data\_model\_and\_resource()}\spxextra{pypath.core.interaction.Interaction method}}

\begin{fulllineitems}
\phantomsection\label{\detokenize{reference:pypath.core.interaction.Interaction.resources_via_by_interaction_type_and_data_model_and_resource}}\pysiglinewithargsret{\sphinxbfcode{\sphinxupquote{resources\_via\_by\_interaction\_type\_and\_data\_model\_and\_resource}}}{\emph{effect=None}, \emph{resources=None}, \emph{data\_model=None}, \emph{interaction\_type=None}, \emph{via=None}, \emph{references=None}}{}
Retrieves resources via matching the criteria.

\end{fulllineitems}

\index{resources\_via\_by\_reference() (pypath.core.interaction.Interaction method)@\spxentry{resources\_via\_by\_reference()}\spxextra{pypath.core.interaction.Interaction method}}

\begin{fulllineitems}
\phantomsection\label{\detokenize{reference:pypath.core.interaction.Interaction.resources_via_by_reference}}\pysiglinewithargsret{\sphinxbfcode{\sphinxupquote{resources\_via\_by\_reference}}}{\emph{effect=None}, \emph{resources=None}, \emph{data\_model=None}, \emph{interaction\_type=None}, \emph{via=None}, \emph{references=None}}{}
Retrieves resources via matching the criteria.

\end{fulllineitems}

\index{resources\_via\_by\_resource() (pypath.core.interaction.Interaction method)@\spxentry{resources\_via\_by\_resource()}\spxextra{pypath.core.interaction.Interaction method}}

\begin{fulllineitems}
\phantomsection\label{\detokenize{reference:pypath.core.interaction.Interaction.resources_via_by_resource}}\pysiglinewithargsret{\sphinxbfcode{\sphinxupquote{resources\_via\_by\_resource}}}{\emph{effect=None}, \emph{resources=None}, \emph{data\_model=None}, \emph{interaction\_type=None}, \emph{via=None}, \emph{references=None}}{}
Retrieves resources via matching the criteria.

\end{fulllineitems}

\index{small\_molecule\_identifiers\_by\_data\_model() (pypath.core.interaction.Interaction method)@\spxentry{small\_molecule\_identifiers\_by\_data\_model()}\spxextra{pypath.core.interaction.Interaction method}}

\begin{fulllineitems}
\phantomsection\label{\detokenize{reference:pypath.core.interaction.Interaction.small_molecule_identifiers_by_data_model}}\pysiglinewithargsret{\sphinxbfcode{\sphinxupquote{small\_molecule\_identifiers\_by\_data\_model}}}{\emph{effect=None}, \emph{resources=None}, \emph{data\_model=None}, \emph{interaction\_type=None}, \emph{via=None}, \emph{references=None}}{}
Retrieves the entities involved in interactions matching the criteria.
It either returns both interacting entities in a \sphinxstyleemphasis{set} or an empty
\sphinxstyleemphasis{set}. This may not sound so useful at the level of this object but
becomes more useful once we want to collect entities having certain
kind of interactions across a series of \sphinxtitleref{Interaction} objects.
\begin{quote}\begin{description}
\item[{Parameters}] \leavevmode\begin{itemize}
\item {} 
\sphinxstyleliteralstrong{\sphinxupquote{entity\_type}} (\sphinxstyleliteralemphasis{\sphinxupquote{str}}) \textendash{} The type of the molecular entity. Possible values: \sphinxtitleref{protein},
\sphinxtitleref{complex}, \sphinxtitleref{mirna}, \sphinxtitleref{small\_molecule}.

\item {} 
\sphinxstyleliteralstrong{\sphinxupquote{return\_type}} (\sphinxstyleliteralemphasis{\sphinxupquote{str}}) \textendash{} The type of values to return. Default is
py:class:\sphinxcode{\sphinxupquote{pypath.entity.Entity}} objects, alternatives are
\sphinxcode{\sphinxupquote{labels}}  \sphinxcode{\sphinxupquote{identifiers}}.

\end{itemize}

\end{description}\end{quote}

\end{fulllineitems}

\index{small\_molecule\_identifiers\_by\_interaction\_type() (pypath.core.interaction.Interaction method)@\spxentry{small\_molecule\_identifiers\_by\_interaction\_type()}\spxextra{pypath.core.interaction.Interaction method}}

\begin{fulllineitems}
\phantomsection\label{\detokenize{reference:pypath.core.interaction.Interaction.small_molecule_identifiers_by_interaction_type}}\pysiglinewithargsret{\sphinxbfcode{\sphinxupquote{small\_molecule\_identifiers\_by\_interaction\_type}}}{\emph{effect=None}, \emph{resources=None}, \emph{data\_model=None}, \emph{interaction\_type=None}, \emph{via=None}, \emph{references=None}}{}
Retrieves the entities involved in interactions matching the criteria.
It either returns both interacting entities in a \sphinxstyleemphasis{set} or an empty
\sphinxstyleemphasis{set}. This may not sound so useful at the level of this object but
becomes more useful once we want to collect entities having certain
kind of interactions across a series of \sphinxtitleref{Interaction} objects.
\begin{quote}\begin{description}
\item[{Parameters}] \leavevmode\begin{itemize}
\item {} 
\sphinxstyleliteralstrong{\sphinxupquote{entity\_type}} (\sphinxstyleliteralemphasis{\sphinxupquote{str}}) \textendash{} The type of the molecular entity. Possible values: \sphinxtitleref{protein},
\sphinxtitleref{complex}, \sphinxtitleref{mirna}, \sphinxtitleref{small\_molecule}.

\item {} 
\sphinxstyleliteralstrong{\sphinxupquote{return\_type}} (\sphinxstyleliteralemphasis{\sphinxupquote{str}}) \textendash{} The type of values to return. Default is
py:class:\sphinxcode{\sphinxupquote{pypath.entity.Entity}} objects, alternatives are
\sphinxcode{\sphinxupquote{labels}}  \sphinxcode{\sphinxupquote{identifiers}}.

\end{itemize}

\end{description}\end{quote}

\end{fulllineitems}

\index{small\_molecule\_identifiers\_by\_interaction\_type\_and\_data\_model() (pypath.core.interaction.Interaction method)@\spxentry{small\_molecule\_identifiers\_by\_interaction\_type\_and\_data\_model()}\spxextra{pypath.core.interaction.Interaction method}}

\begin{fulllineitems}
\phantomsection\label{\detokenize{reference:pypath.core.interaction.Interaction.small_molecule_identifiers_by_interaction_type_and_data_model}}\pysiglinewithargsret{\sphinxbfcode{\sphinxupquote{small\_molecule\_identifiers\_by\_interaction\_type\_and\_data\_model}}}{\emph{effect=None}, \emph{resources=None}, \emph{data\_model=None}, \emph{interaction\_type=None}, \emph{via=None}, \emph{references=None}}{}
Retrieves the entities involved in interactions matching the criteria.
It either returns both interacting entities in a \sphinxstyleemphasis{set} or an empty
\sphinxstyleemphasis{set}. This may not sound so useful at the level of this object but
becomes more useful once we want to collect entities having certain
kind of interactions across a series of \sphinxtitleref{Interaction} objects.
\begin{quote}\begin{description}
\item[{Parameters}] \leavevmode\begin{itemize}
\item {} 
\sphinxstyleliteralstrong{\sphinxupquote{entity\_type}} (\sphinxstyleliteralemphasis{\sphinxupquote{str}}) \textendash{} The type of the molecular entity. Possible values: \sphinxtitleref{protein},
\sphinxtitleref{complex}, \sphinxtitleref{mirna}, \sphinxtitleref{small\_molecule}.

\item {} 
\sphinxstyleliteralstrong{\sphinxupquote{return\_type}} (\sphinxstyleliteralemphasis{\sphinxupquote{str}}) \textendash{} The type of values to return. Default is
py:class:\sphinxcode{\sphinxupquote{pypath.entity.Entity}} objects, alternatives are
\sphinxcode{\sphinxupquote{labels}}  \sphinxcode{\sphinxupquote{identifiers}}.

\end{itemize}

\end{description}\end{quote}

\end{fulllineitems}

\index{small\_molecule\_identifiers\_by\_interaction\_type\_and\_data\_model\_and\_resource() (pypath.core.interaction.Interaction method)@\spxentry{small\_molecule\_identifiers\_by\_interaction\_type\_and\_data\_model\_and\_resource()}\spxextra{pypath.core.interaction.Interaction method}}

\begin{fulllineitems}
\phantomsection\label{\detokenize{reference:pypath.core.interaction.Interaction.small_molecule_identifiers_by_interaction_type_and_data_model_and_resource}}\pysiglinewithargsret{\sphinxbfcode{\sphinxupquote{small\_molecule\_identifiers\_by\_interaction\_type\_and\_data\_model\_and\_resource}}}{\emph{effect=None}, \emph{resources=None}, \emph{data\_model=None}, \emph{interaction\_type=None}, \emph{via=None}, \emph{references=None}}{}
Retrieves the entities involved in interactions matching the criteria.
It either returns both interacting entities in a \sphinxstyleemphasis{set} or an empty
\sphinxstyleemphasis{set}. This may not sound so useful at the level of this object but
becomes more useful once we want to collect entities having certain
kind of interactions across a series of \sphinxtitleref{Interaction} objects.
\begin{quote}\begin{description}
\item[{Parameters}] \leavevmode\begin{itemize}
\item {} 
\sphinxstyleliteralstrong{\sphinxupquote{entity\_type}} (\sphinxstyleliteralemphasis{\sphinxupquote{str}}) \textendash{} The type of the molecular entity. Possible values: \sphinxtitleref{protein},
\sphinxtitleref{complex}, \sphinxtitleref{mirna}, \sphinxtitleref{small\_molecule}.

\item {} 
\sphinxstyleliteralstrong{\sphinxupquote{return\_type}} (\sphinxstyleliteralemphasis{\sphinxupquote{str}}) \textendash{} The type of values to return. Default is
py:class:\sphinxcode{\sphinxupquote{pypath.entity.Entity}} objects, alternatives are
\sphinxcode{\sphinxupquote{labels}}  \sphinxcode{\sphinxupquote{identifiers}}.

\end{itemize}

\end{description}\end{quote}

\end{fulllineitems}

\index{small\_molecule\_identifiers\_by\_reference() (pypath.core.interaction.Interaction method)@\spxentry{small\_molecule\_identifiers\_by\_reference()}\spxextra{pypath.core.interaction.Interaction method}}

\begin{fulllineitems}
\phantomsection\label{\detokenize{reference:pypath.core.interaction.Interaction.small_molecule_identifiers_by_reference}}\pysiglinewithargsret{\sphinxbfcode{\sphinxupquote{small\_molecule\_identifiers\_by\_reference}}}{\emph{effect=None}, \emph{resources=None}, \emph{data\_model=None}, \emph{interaction\_type=None}, \emph{via=None}, \emph{references=None}}{}
Retrieves the entities involved in interactions matching the criteria.
It either returns both interacting entities in a \sphinxstyleemphasis{set} or an empty
\sphinxstyleemphasis{set}. This may not sound so useful at the level of this object but
becomes more useful once we want to collect entities having certain
kind of interactions across a series of \sphinxtitleref{Interaction} objects.
\begin{quote}\begin{description}
\item[{Parameters}] \leavevmode\begin{itemize}
\item {} 
\sphinxstyleliteralstrong{\sphinxupquote{entity\_type}} (\sphinxstyleliteralemphasis{\sphinxupquote{str}}) \textendash{} The type of the molecular entity. Possible values: \sphinxtitleref{protein},
\sphinxtitleref{complex}, \sphinxtitleref{mirna}, \sphinxtitleref{small\_molecule}.

\item {} 
\sphinxstyleliteralstrong{\sphinxupquote{return\_type}} (\sphinxstyleliteralemphasis{\sphinxupquote{str}}) \textendash{} The type of values to return. Default is
py:class:\sphinxcode{\sphinxupquote{pypath.entity.Entity}} objects, alternatives are
\sphinxcode{\sphinxupquote{labels}}  \sphinxcode{\sphinxupquote{identifiers}}.

\end{itemize}

\end{description}\end{quote}

\end{fulllineitems}

\index{small\_molecule\_identifiers\_by\_resource() (pypath.core.interaction.Interaction method)@\spxentry{small\_molecule\_identifiers\_by\_resource()}\spxextra{pypath.core.interaction.Interaction method}}

\begin{fulllineitems}
\phantomsection\label{\detokenize{reference:pypath.core.interaction.Interaction.small_molecule_identifiers_by_resource}}\pysiglinewithargsret{\sphinxbfcode{\sphinxupquote{small\_molecule\_identifiers\_by\_resource}}}{\emph{effect=None}, \emph{resources=None}, \emph{data\_model=None}, \emph{interaction\_type=None}, \emph{via=None}, \emph{references=None}}{}
Retrieves the entities involved in interactions matching the criteria.
It either returns both interacting entities in a \sphinxstyleemphasis{set} or an empty
\sphinxstyleemphasis{set}. This may not sound so useful at the level of this object but
becomes more useful once we want to collect entities having certain
kind of interactions across a series of \sphinxtitleref{Interaction} objects.
\begin{quote}\begin{description}
\item[{Parameters}] \leavevmode\begin{itemize}
\item {} 
\sphinxstyleliteralstrong{\sphinxupquote{entity\_type}} (\sphinxstyleliteralemphasis{\sphinxupquote{str}}) \textendash{} The type of the molecular entity. Possible values: \sphinxtitleref{protein},
\sphinxtitleref{complex}, \sphinxtitleref{mirna}, \sphinxtitleref{small\_molecule}.

\item {} 
\sphinxstyleliteralstrong{\sphinxupquote{return\_type}} (\sphinxstyleliteralemphasis{\sphinxupquote{str}}) \textendash{} The type of values to return. Default is
py:class:\sphinxcode{\sphinxupquote{pypath.entity.Entity}} objects, alternatives are
\sphinxcode{\sphinxupquote{labels}}  \sphinxcode{\sphinxupquote{identifiers}}.

\end{itemize}

\end{description}\end{quote}

\end{fulllineitems}

\index{small\_molecule\_labels\_by\_data\_model() (pypath.core.interaction.Interaction method)@\spxentry{small\_molecule\_labels\_by\_data\_model()}\spxextra{pypath.core.interaction.Interaction method}}

\begin{fulllineitems}
\phantomsection\label{\detokenize{reference:pypath.core.interaction.Interaction.small_molecule_labels_by_data_model}}\pysiglinewithargsret{\sphinxbfcode{\sphinxupquote{small\_molecule\_labels\_by\_data\_model}}}{\emph{effect=None}, \emph{resources=None}, \emph{data\_model=None}, \emph{interaction\_type=None}, \emph{via=None}, \emph{references=None}}{}
Retrieves the entities involved in interactions matching the criteria.
It either returns both interacting entities in a \sphinxstyleemphasis{set} or an empty
\sphinxstyleemphasis{set}. This may not sound so useful at the level of this object but
becomes more useful once we want to collect entities having certain
kind of interactions across a series of \sphinxtitleref{Interaction} objects.
\begin{quote}\begin{description}
\item[{Parameters}] \leavevmode\begin{itemize}
\item {} 
\sphinxstyleliteralstrong{\sphinxupquote{entity\_type}} (\sphinxstyleliteralemphasis{\sphinxupquote{str}}) \textendash{} The type of the molecular entity. Possible values: \sphinxtitleref{protein},
\sphinxtitleref{complex}, \sphinxtitleref{mirna}, \sphinxtitleref{small\_molecule}.

\item {} 
\sphinxstyleliteralstrong{\sphinxupquote{return\_type}} (\sphinxstyleliteralemphasis{\sphinxupquote{str}}) \textendash{} The type of values to return. Default is
py:class:\sphinxcode{\sphinxupquote{pypath.entity.Entity}} objects, alternatives are
\sphinxcode{\sphinxupquote{labels}}  \sphinxcode{\sphinxupquote{identifiers}}.

\end{itemize}

\end{description}\end{quote}

\end{fulllineitems}

\index{small\_molecule\_labels\_by\_interaction\_type() (pypath.core.interaction.Interaction method)@\spxentry{small\_molecule\_labels\_by\_interaction\_type()}\spxextra{pypath.core.interaction.Interaction method}}

\begin{fulllineitems}
\phantomsection\label{\detokenize{reference:pypath.core.interaction.Interaction.small_molecule_labels_by_interaction_type}}\pysiglinewithargsret{\sphinxbfcode{\sphinxupquote{small\_molecule\_labels\_by\_interaction\_type}}}{\emph{effect=None}, \emph{resources=None}, \emph{data\_model=None}, \emph{interaction\_type=None}, \emph{via=None}, \emph{references=None}}{}
Retrieves the entities involved in interactions matching the criteria.
It either returns both interacting entities in a \sphinxstyleemphasis{set} or an empty
\sphinxstyleemphasis{set}. This may not sound so useful at the level of this object but
becomes more useful once we want to collect entities having certain
kind of interactions across a series of \sphinxtitleref{Interaction} objects.
\begin{quote}\begin{description}
\item[{Parameters}] \leavevmode\begin{itemize}
\item {} 
\sphinxstyleliteralstrong{\sphinxupquote{entity\_type}} (\sphinxstyleliteralemphasis{\sphinxupquote{str}}) \textendash{} The type of the molecular entity. Possible values: \sphinxtitleref{protein},
\sphinxtitleref{complex}, \sphinxtitleref{mirna}, \sphinxtitleref{small\_molecule}.

\item {} 
\sphinxstyleliteralstrong{\sphinxupquote{return\_type}} (\sphinxstyleliteralemphasis{\sphinxupquote{str}}) \textendash{} The type of values to return. Default is
py:class:\sphinxcode{\sphinxupquote{pypath.entity.Entity}} objects, alternatives are
\sphinxcode{\sphinxupquote{labels}}  \sphinxcode{\sphinxupquote{identifiers}}.

\end{itemize}

\end{description}\end{quote}

\end{fulllineitems}

\index{small\_molecule\_labels\_by\_interaction\_type\_and\_data\_model() (pypath.core.interaction.Interaction method)@\spxentry{small\_molecule\_labels\_by\_interaction\_type\_and\_data\_model()}\spxextra{pypath.core.interaction.Interaction method}}

\begin{fulllineitems}
\phantomsection\label{\detokenize{reference:pypath.core.interaction.Interaction.small_molecule_labels_by_interaction_type_and_data_model}}\pysiglinewithargsret{\sphinxbfcode{\sphinxupquote{small\_molecule\_labels\_by\_interaction\_type\_and\_data\_model}}}{\emph{effect=None}, \emph{resources=None}, \emph{data\_model=None}, \emph{interaction\_type=None}, \emph{via=None}, \emph{references=None}}{}
Retrieves the entities involved in interactions matching the criteria.
It either returns both interacting entities in a \sphinxstyleemphasis{set} or an empty
\sphinxstyleemphasis{set}. This may not sound so useful at the level of this object but
becomes more useful once we want to collect entities having certain
kind of interactions across a series of \sphinxtitleref{Interaction} objects.
\begin{quote}\begin{description}
\item[{Parameters}] \leavevmode\begin{itemize}
\item {} 
\sphinxstyleliteralstrong{\sphinxupquote{entity\_type}} (\sphinxstyleliteralemphasis{\sphinxupquote{str}}) \textendash{} The type of the molecular entity. Possible values: \sphinxtitleref{protein},
\sphinxtitleref{complex}, \sphinxtitleref{mirna}, \sphinxtitleref{small\_molecule}.

\item {} 
\sphinxstyleliteralstrong{\sphinxupquote{return\_type}} (\sphinxstyleliteralemphasis{\sphinxupquote{str}}) \textendash{} The type of values to return. Default is
py:class:\sphinxcode{\sphinxupquote{pypath.entity.Entity}} objects, alternatives are
\sphinxcode{\sphinxupquote{labels}}  \sphinxcode{\sphinxupquote{identifiers}}.

\end{itemize}

\end{description}\end{quote}

\end{fulllineitems}

\index{small\_molecule\_labels\_by\_interaction\_type\_and\_data\_model\_and\_resource() (pypath.core.interaction.Interaction method)@\spxentry{small\_molecule\_labels\_by\_interaction\_type\_and\_data\_model\_and\_resource()}\spxextra{pypath.core.interaction.Interaction method}}

\begin{fulllineitems}
\phantomsection\label{\detokenize{reference:pypath.core.interaction.Interaction.small_molecule_labels_by_interaction_type_and_data_model_and_resource}}\pysiglinewithargsret{\sphinxbfcode{\sphinxupquote{small\_molecule\_labels\_by\_interaction\_type\_and\_data\_model\_and\_resource}}}{\emph{effect=None}, \emph{resources=None}, \emph{data\_model=None}, \emph{interaction\_type=None}, \emph{via=None}, \emph{references=None}}{}
Retrieves the entities involved in interactions matching the criteria.
It either returns both interacting entities in a \sphinxstyleemphasis{set} or an empty
\sphinxstyleemphasis{set}. This may not sound so useful at the level of this object but
becomes more useful once we want to collect entities having certain
kind of interactions across a series of \sphinxtitleref{Interaction} objects.
\begin{quote}\begin{description}
\item[{Parameters}] \leavevmode\begin{itemize}
\item {} 
\sphinxstyleliteralstrong{\sphinxupquote{entity\_type}} (\sphinxstyleliteralemphasis{\sphinxupquote{str}}) \textendash{} The type of the molecular entity. Possible values: \sphinxtitleref{protein},
\sphinxtitleref{complex}, \sphinxtitleref{mirna}, \sphinxtitleref{small\_molecule}.

\item {} 
\sphinxstyleliteralstrong{\sphinxupquote{return\_type}} (\sphinxstyleliteralemphasis{\sphinxupquote{str}}) \textendash{} The type of values to return. Default is
py:class:\sphinxcode{\sphinxupquote{pypath.entity.Entity}} objects, alternatives are
\sphinxcode{\sphinxupquote{labels}}  \sphinxcode{\sphinxupquote{identifiers}}.

\end{itemize}

\end{description}\end{quote}

\end{fulllineitems}

\index{small\_molecule\_labels\_by\_reference() (pypath.core.interaction.Interaction method)@\spxentry{small\_molecule\_labels\_by\_reference()}\spxextra{pypath.core.interaction.Interaction method}}

\begin{fulllineitems}
\phantomsection\label{\detokenize{reference:pypath.core.interaction.Interaction.small_molecule_labels_by_reference}}\pysiglinewithargsret{\sphinxbfcode{\sphinxupquote{small\_molecule\_labels\_by\_reference}}}{\emph{effect=None}, \emph{resources=None}, \emph{data\_model=None}, \emph{interaction\_type=None}, \emph{via=None}, \emph{references=None}}{}
Retrieves the entities involved in interactions matching the criteria.
It either returns both interacting entities in a \sphinxstyleemphasis{set} or an empty
\sphinxstyleemphasis{set}. This may not sound so useful at the level of this object but
becomes more useful once we want to collect entities having certain
kind of interactions across a series of \sphinxtitleref{Interaction} objects.
\begin{quote}\begin{description}
\item[{Parameters}] \leavevmode\begin{itemize}
\item {} 
\sphinxstyleliteralstrong{\sphinxupquote{entity\_type}} (\sphinxstyleliteralemphasis{\sphinxupquote{str}}) \textendash{} The type of the molecular entity. Possible values: \sphinxtitleref{protein},
\sphinxtitleref{complex}, \sphinxtitleref{mirna}, \sphinxtitleref{small\_molecule}.

\item {} 
\sphinxstyleliteralstrong{\sphinxupquote{return\_type}} (\sphinxstyleliteralemphasis{\sphinxupquote{str}}) \textendash{} The type of values to return. Default is
py:class:\sphinxcode{\sphinxupquote{pypath.entity.Entity}} objects, alternatives are
\sphinxcode{\sphinxupquote{labels}}  \sphinxcode{\sphinxupquote{identifiers}}.

\end{itemize}

\end{description}\end{quote}

\end{fulllineitems}

\index{small\_molecule\_labels\_by\_resource() (pypath.core.interaction.Interaction method)@\spxentry{small\_molecule\_labels\_by\_resource()}\spxextra{pypath.core.interaction.Interaction method}}

\begin{fulllineitems}
\phantomsection\label{\detokenize{reference:pypath.core.interaction.Interaction.small_molecule_labels_by_resource}}\pysiglinewithargsret{\sphinxbfcode{\sphinxupquote{small\_molecule\_labels\_by\_resource}}}{\emph{effect=None}, \emph{resources=None}, \emph{data\_model=None}, \emph{interaction\_type=None}, \emph{via=None}, \emph{references=None}}{}
Retrieves the entities involved in interactions matching the criteria.
It either returns both interacting entities in a \sphinxstyleemphasis{set} or an empty
\sphinxstyleemphasis{set}. This may not sound so useful at the level of this object but
becomes more useful once we want to collect entities having certain
kind of interactions across a series of \sphinxtitleref{Interaction} objects.
\begin{quote}\begin{description}
\item[{Parameters}] \leavevmode\begin{itemize}
\item {} 
\sphinxstyleliteralstrong{\sphinxupquote{entity\_type}} (\sphinxstyleliteralemphasis{\sphinxupquote{str}}) \textendash{} The type of the molecular entity. Possible values: \sphinxtitleref{protein},
\sphinxtitleref{complex}, \sphinxtitleref{mirna}, \sphinxtitleref{small\_molecule}.

\item {} 
\sphinxstyleliteralstrong{\sphinxupquote{return\_type}} (\sphinxstyleliteralemphasis{\sphinxupquote{str}}) \textendash{} The type of values to return. Default is
py:class:\sphinxcode{\sphinxupquote{pypath.entity.Entity}} objects, alternatives are
\sphinxcode{\sphinxupquote{labels}}  \sphinxcode{\sphinxupquote{identifiers}}.

\end{itemize}

\end{description}\end{quote}

\end{fulllineitems}

\index{small\_molecules\_by\_data\_model() (pypath.core.interaction.Interaction method)@\spxentry{small\_molecules\_by\_data\_model()}\spxextra{pypath.core.interaction.Interaction method}}

\begin{fulllineitems}
\phantomsection\label{\detokenize{reference:pypath.core.interaction.Interaction.small_molecules_by_data_model}}\pysiglinewithargsret{\sphinxbfcode{\sphinxupquote{small\_molecules\_by\_data\_model}}}{\emph{effect=None}, \emph{resources=None}, \emph{data\_model=None}, \emph{interaction\_type=None}, \emph{via=None}, \emph{references=None}}{}
Retrieves the entities involved in interactions matching the criteria.
It either returns both interacting entities in a \sphinxstyleemphasis{set} or an empty
\sphinxstyleemphasis{set}. This may not sound so useful at the level of this object but
becomes more useful once we want to collect entities having certain
kind of interactions across a series of \sphinxtitleref{Interaction} objects.
\begin{quote}\begin{description}
\item[{Parameters}] \leavevmode\begin{itemize}
\item {} 
\sphinxstyleliteralstrong{\sphinxupquote{entity\_type}} (\sphinxstyleliteralemphasis{\sphinxupquote{str}}) \textendash{} The type of the molecular entity. Possible values: \sphinxtitleref{protein},
\sphinxtitleref{complex}, \sphinxtitleref{mirna}, \sphinxtitleref{small\_molecule}.

\item {} 
\sphinxstyleliteralstrong{\sphinxupquote{return\_type}} (\sphinxstyleliteralemphasis{\sphinxupquote{str}}) \textendash{} The type of values to return. Default is
py:class:\sphinxcode{\sphinxupquote{pypath.entity.Entity}} objects, alternatives are
\sphinxcode{\sphinxupquote{labels}}  \sphinxcode{\sphinxupquote{identifiers}}.

\end{itemize}

\end{description}\end{quote}

\end{fulllineitems}

\index{small\_molecules\_by\_interaction\_type() (pypath.core.interaction.Interaction method)@\spxentry{small\_molecules\_by\_interaction\_type()}\spxextra{pypath.core.interaction.Interaction method}}

\begin{fulllineitems}
\phantomsection\label{\detokenize{reference:pypath.core.interaction.Interaction.small_molecules_by_interaction_type}}\pysiglinewithargsret{\sphinxbfcode{\sphinxupquote{small\_molecules\_by\_interaction\_type}}}{\emph{effect=None}, \emph{resources=None}, \emph{data\_model=None}, \emph{interaction\_type=None}, \emph{via=None}, \emph{references=None}}{}
Retrieves the entities involved in interactions matching the criteria.
It either returns both interacting entities in a \sphinxstyleemphasis{set} or an empty
\sphinxstyleemphasis{set}. This may not sound so useful at the level of this object but
becomes more useful once we want to collect entities having certain
kind of interactions across a series of \sphinxtitleref{Interaction} objects.
\begin{quote}\begin{description}
\item[{Parameters}] \leavevmode\begin{itemize}
\item {} 
\sphinxstyleliteralstrong{\sphinxupquote{entity\_type}} (\sphinxstyleliteralemphasis{\sphinxupquote{str}}) \textendash{} The type of the molecular entity. Possible values: \sphinxtitleref{protein},
\sphinxtitleref{complex}, \sphinxtitleref{mirna}, \sphinxtitleref{small\_molecule}.

\item {} 
\sphinxstyleliteralstrong{\sphinxupquote{return\_type}} (\sphinxstyleliteralemphasis{\sphinxupquote{str}}) \textendash{} The type of values to return. Default is
py:class:\sphinxcode{\sphinxupquote{pypath.entity.Entity}} objects, alternatives are
\sphinxcode{\sphinxupquote{labels}}  \sphinxcode{\sphinxupquote{identifiers}}.

\end{itemize}

\end{description}\end{quote}

\end{fulllineitems}

\index{small\_molecules\_by\_interaction\_type\_and\_data\_model() (pypath.core.interaction.Interaction method)@\spxentry{small\_molecules\_by\_interaction\_type\_and\_data\_model()}\spxextra{pypath.core.interaction.Interaction method}}

\begin{fulllineitems}
\phantomsection\label{\detokenize{reference:pypath.core.interaction.Interaction.small_molecules_by_interaction_type_and_data_model}}\pysiglinewithargsret{\sphinxbfcode{\sphinxupquote{small\_molecules\_by\_interaction\_type\_and\_data\_model}}}{\emph{effect=None}, \emph{resources=None}, \emph{data\_model=None}, \emph{interaction\_type=None}, \emph{via=None}, \emph{references=None}}{}
Retrieves the entities involved in interactions matching the criteria.
It either returns both interacting entities in a \sphinxstyleemphasis{set} or an empty
\sphinxstyleemphasis{set}. This may not sound so useful at the level of this object but
becomes more useful once we want to collect entities having certain
kind of interactions across a series of \sphinxtitleref{Interaction} objects.
\begin{quote}\begin{description}
\item[{Parameters}] \leavevmode\begin{itemize}
\item {} 
\sphinxstyleliteralstrong{\sphinxupquote{entity\_type}} (\sphinxstyleliteralemphasis{\sphinxupquote{str}}) \textendash{} The type of the molecular entity. Possible values: \sphinxtitleref{protein},
\sphinxtitleref{complex}, \sphinxtitleref{mirna}, \sphinxtitleref{small\_molecule}.

\item {} 
\sphinxstyleliteralstrong{\sphinxupquote{return\_type}} (\sphinxstyleliteralemphasis{\sphinxupquote{str}}) \textendash{} The type of values to return. Default is
py:class:\sphinxcode{\sphinxupquote{pypath.entity.Entity}} objects, alternatives are
\sphinxcode{\sphinxupquote{labels}}  \sphinxcode{\sphinxupquote{identifiers}}.

\end{itemize}

\end{description}\end{quote}

\end{fulllineitems}

\index{small\_molecules\_by\_interaction\_type\_and\_data\_model\_and\_resource() (pypath.core.interaction.Interaction method)@\spxentry{small\_molecules\_by\_interaction\_type\_and\_data\_model\_and\_resource()}\spxextra{pypath.core.interaction.Interaction method}}

\begin{fulllineitems}
\phantomsection\label{\detokenize{reference:pypath.core.interaction.Interaction.small_molecules_by_interaction_type_and_data_model_and_resource}}\pysiglinewithargsret{\sphinxbfcode{\sphinxupquote{small\_molecules\_by\_interaction\_type\_and\_data\_model\_and\_resource}}}{\emph{effect=None}, \emph{resources=None}, \emph{data\_model=None}, \emph{interaction\_type=None}, \emph{via=None}, \emph{references=None}}{}
Retrieves the entities involved in interactions matching the criteria.
It either returns both interacting entities in a \sphinxstyleemphasis{set} or an empty
\sphinxstyleemphasis{set}. This may not sound so useful at the level of this object but
becomes more useful once we want to collect entities having certain
kind of interactions across a series of \sphinxtitleref{Interaction} objects.
\begin{quote}\begin{description}
\item[{Parameters}] \leavevmode\begin{itemize}
\item {} 
\sphinxstyleliteralstrong{\sphinxupquote{entity\_type}} (\sphinxstyleliteralemphasis{\sphinxupquote{str}}) \textendash{} The type of the molecular entity. Possible values: \sphinxtitleref{protein},
\sphinxtitleref{complex}, \sphinxtitleref{mirna}, \sphinxtitleref{small\_molecule}.

\item {} 
\sphinxstyleliteralstrong{\sphinxupquote{return\_type}} (\sphinxstyleliteralemphasis{\sphinxupquote{str}}) \textendash{} The type of values to return. Default is
py:class:\sphinxcode{\sphinxupquote{pypath.entity.Entity}} objects, alternatives are
\sphinxcode{\sphinxupquote{labels}}  \sphinxcode{\sphinxupquote{identifiers}}.

\end{itemize}

\end{description}\end{quote}

\end{fulllineitems}

\index{small\_molecules\_by\_reference() (pypath.core.interaction.Interaction method)@\spxentry{small\_molecules\_by\_reference()}\spxextra{pypath.core.interaction.Interaction method}}

\begin{fulllineitems}
\phantomsection\label{\detokenize{reference:pypath.core.interaction.Interaction.small_molecules_by_reference}}\pysiglinewithargsret{\sphinxbfcode{\sphinxupquote{small\_molecules\_by\_reference}}}{\emph{effect=None}, \emph{resources=None}, \emph{data\_model=None}, \emph{interaction\_type=None}, \emph{via=None}, \emph{references=None}}{}
Retrieves the entities involved in interactions matching the criteria.
It either returns both interacting entities in a \sphinxstyleemphasis{set} or an empty
\sphinxstyleemphasis{set}. This may not sound so useful at the level of this object but
becomes more useful once we want to collect entities having certain
kind of interactions across a series of \sphinxtitleref{Interaction} objects.
\begin{quote}\begin{description}
\item[{Parameters}] \leavevmode\begin{itemize}
\item {} 
\sphinxstyleliteralstrong{\sphinxupquote{entity\_type}} (\sphinxstyleliteralemphasis{\sphinxupquote{str}}) \textendash{} The type of the molecular entity. Possible values: \sphinxtitleref{protein},
\sphinxtitleref{complex}, \sphinxtitleref{mirna}, \sphinxtitleref{small\_molecule}.

\item {} 
\sphinxstyleliteralstrong{\sphinxupquote{return\_type}} (\sphinxstyleliteralemphasis{\sphinxupquote{str}}) \textendash{} The type of values to return. Default is
py:class:\sphinxcode{\sphinxupquote{pypath.entity.Entity}} objects, alternatives are
\sphinxcode{\sphinxupquote{labels}}  \sphinxcode{\sphinxupquote{identifiers}}.

\end{itemize}

\end{description}\end{quote}

\end{fulllineitems}

\index{small\_molecules\_by\_resource() (pypath.core.interaction.Interaction method)@\spxentry{small\_molecules\_by\_resource()}\spxextra{pypath.core.interaction.Interaction method}}

\begin{fulllineitems}
\phantomsection\label{\detokenize{reference:pypath.core.interaction.Interaction.small_molecules_by_resource}}\pysiglinewithargsret{\sphinxbfcode{\sphinxupquote{small\_molecules\_by\_resource}}}{\emph{effect=None}, \emph{resources=None}, \emph{data\_model=None}, \emph{interaction\_type=None}, \emph{via=None}, \emph{references=None}}{}
Retrieves the entities involved in interactions matching the criteria.
It either returns both interacting entities in a \sphinxstyleemphasis{set} or an empty
\sphinxstyleemphasis{set}. This may not sound so useful at the level of this object but
becomes more useful once we want to collect entities having certain
kind of interactions across a series of \sphinxtitleref{Interaction} objects.
\begin{quote}\begin{description}
\item[{Parameters}] \leavevmode\begin{itemize}
\item {} 
\sphinxstyleliteralstrong{\sphinxupquote{entity\_type}} (\sphinxstyleliteralemphasis{\sphinxupquote{str}}) \textendash{} The type of the molecular entity. Possible values: \sphinxtitleref{protein},
\sphinxtitleref{complex}, \sphinxtitleref{mirna}, \sphinxtitleref{small\_molecule}.

\item {} 
\sphinxstyleliteralstrong{\sphinxupquote{return\_type}} (\sphinxstyleliteralemphasis{\sphinxupquote{str}}) \textendash{} The type of values to return. Default is
py:class:\sphinxcode{\sphinxupquote{pypath.entity.Entity}} objects, alternatives are
\sphinxcode{\sphinxupquote{labels}}  \sphinxcode{\sphinxupquote{identifiers}}.

\end{itemize}

\end{description}\end{quote}

\end{fulllineitems}

\index{source() (pypath.core.interaction.Interaction method)@\spxentry{source()}\spxextra{pypath.core.interaction.Interaction method}}

\begin{fulllineitems}
\phantomsection\label{\detokenize{reference:pypath.core.interaction.Interaction.source}}\pysiglinewithargsret{\sphinxbfcode{\sphinxupquote{source}}}{\emph{undirected=False}, \emph{resources=None}, \emph{**kwargs}}{}
Returns the name(s) of the source node(s) for each existing
direction on the interaction.
\begin{quote}\begin{description}
\item[{Parameters}] \leavevmode
\sphinxstyleliteralstrong{\sphinxupquote{undirected}} (\sphinxstyleliteralemphasis{\sphinxupquote{bool}}) \textendash{} Optional, \sphinxcode{\sphinxupquote{False}} by default.

\item[{Returns}] \leavevmode
(\sphinxstyleemphasis{list}) \textendash{} Contains the name(s) for the source node(s).
This means if the interaction is bidirectional, the list
will contain both identifiers on the edge. If the
interaction is undirected, an empty list will be returned.

\end{description}\end{quote}

\end{fulllineitems}

\index{sources\_reverse() (pypath.core.interaction.Interaction method)@\spxentry{sources\_reverse()}\spxextra{pypath.core.interaction.Interaction method}}

\begin{fulllineitems}
\phantomsection\label{\detokenize{reference:pypath.core.interaction.Interaction.sources_reverse}}\pysiglinewithargsret{\sphinxbfcode{\sphinxupquote{sources\_reverse}}}{\emph{resources=False}, \emph{evidences=False}, \emph{resource\_names=False}, \emph{sources=False}}{}
Retrieves the list of sources for the \sphinxcode{\sphinxupquote{b\_a}} direction.
\begin{quote}\begin{description}
\item[{Returns}] \leavevmode
(\sphinxstyleemphasis{set}) \textendash{} Contains the names of the sources supporting the
\sphinxcode{\sphinxupquote{b\_a}} directionality of the edge.

\end{description}\end{quote}

\end{fulllineitems}

\index{sources\_straight() (pypath.core.interaction.Interaction method)@\spxentry{sources\_straight()}\spxextra{pypath.core.interaction.Interaction method}}

\begin{fulllineitems}
\phantomsection\label{\detokenize{reference:pypath.core.interaction.Interaction.sources_straight}}\pysiglinewithargsret{\sphinxbfcode{\sphinxupquote{sources\_straight}}}{\emph{resources=False}, \emph{evidences=False}, \emph{resource\_names=False}, \emph{sources=False}}{}
Retrieves the list of resources for the \sphinxcode{\sphinxupquote{a\_b}}
direction.
\begin{quote}\begin{description}
\item[{Returns}] \leavevmode
(\sphinxstyleemphasis{set}) \textendash{} Contains the names of the sources supporting the
\sphinxcode{\sphinxupquote{a\_b}} directionality of the edge.

\end{description}\end{quote}

\end{fulllineitems}

\index{sources\_undirected() (pypath.core.interaction.Interaction method)@\spxentry{sources\_undirected()}\spxextra{pypath.core.interaction.Interaction method}}

\begin{fulllineitems}
\phantomsection\label{\detokenize{reference:pypath.core.interaction.Interaction.sources_undirected}}\pysiglinewithargsret{\sphinxbfcode{\sphinxupquote{sources\_undirected}}}{\emph{resources=False}, \emph{evidences=False}, \emph{resource\_names=False}, \emph{sources=False}}{}
Retrieves the list of resources without directed information.
\begin{quote}\begin{description}
\item[{Returns}] \leavevmode
(\sphinxstyleemphasis{set}) \textendash{} Contains the names of the sources supporting the
edge presence but without specific directionality
information.

\end{description}\end{quote}

\end{fulllineitems}

\index{src() (pypath.core.interaction.Interaction method)@\spxentry{src()}\spxextra{pypath.core.interaction.Interaction method}}

\begin{fulllineitems}
\phantomsection\label{\detokenize{reference:pypath.core.interaction.Interaction.src}}\pysiglinewithargsret{\sphinxbfcode{\sphinxupquote{src}}}{\emph{undirected=False}, \emph{resources=None}, \emph{**kwargs}}{}
Returns the name(s) of the source node(s) for each existing
direction on the interaction.
\begin{quote}\begin{description}
\item[{Parameters}] \leavevmode
\sphinxstyleliteralstrong{\sphinxupquote{undirected}} (\sphinxstyleliteralemphasis{\sphinxupquote{bool}}) \textendash{} Optional, \sphinxcode{\sphinxupquote{False}} by default.

\item[{Returns}] \leavevmode
(\sphinxstyleemphasis{list}) \textendash{} Contains the name(s) for the source node(s).
This means if the interaction is bidirectional, the list
will contain both identifiers on the edge. If the
interaction is undirected, an empty list will be returned.

\end{description}\end{quote}

\end{fulllineitems}

\index{src\_by\_resource() (pypath.core.interaction.Interaction method)@\spxentry{src\_by\_resource()}\spxextra{pypath.core.interaction.Interaction method}}

\begin{fulllineitems}
\phantomsection\label{\detokenize{reference:pypath.core.interaction.Interaction.src_by_resource}}\pysiglinewithargsret{\sphinxbfcode{\sphinxupquote{src\_by\_resource}}}{\emph{resource}}{}
Returns the name(s) of the source node(s) for each existing
direction on the interaction for a specific \sphinxstyleemphasis{resource}.
\begin{quote}\begin{description}
\item[{Parameters}] \leavevmode
\sphinxstyleliteralstrong{\sphinxupquote{resource}} (\sphinxstyleliteralemphasis{\sphinxupquote{str}}) \textendash{} Name of the resource according to which the information is to
be retrieved.

\item[{Returns}] \leavevmode
(\sphinxstyleemphasis{list}) \textendash{} Contains the name(s) for the source node(s)
according to the specified \sphinxstyleemphasis{resource}. This means if the
interaction is bidirectional, the list will contain both
identifiers on the edge. If the specified \sphinxstyleemphasis{source} is not
found or invalid, an empty list will be returned.

\end{description}\end{quote}

\end{fulllineitems}

\index{target() (pypath.core.interaction.Interaction method)@\spxentry{target()}\spxextra{pypath.core.interaction.Interaction method}}

\begin{fulllineitems}
\phantomsection\label{\detokenize{reference:pypath.core.interaction.Interaction.target}}\pysiglinewithargsret{\sphinxbfcode{\sphinxupquote{target}}}{\emph{undirected=False}, \emph{resources=None}, \emph{**kwargs}}{}
Returns the name(s) of the target node(s) for each existing
direction on the interaction.
\begin{quote}\begin{description}
\item[{Parameters}] \leavevmode
\sphinxstyleliteralstrong{\sphinxupquote{undirected}} (\sphinxstyleliteralemphasis{\sphinxupquote{bool}}) \textendash{} Optional, \sphinxcode{\sphinxupquote{False}} by default.

\item[{Returns}] \leavevmode
(\sphinxstyleemphasis{list}) \textendash{} Contains the name(s) for the target node(s).
This means if the interaction is bidirectional, the list
will contain both identifiers on the edge. If the
interaction is undirected, an empty list will be returned.

\end{description}\end{quote}

\end{fulllineitems}

\index{tgt() (pypath.core.interaction.Interaction method)@\spxentry{tgt()}\spxextra{pypath.core.interaction.Interaction method}}

\begin{fulllineitems}
\phantomsection\label{\detokenize{reference:pypath.core.interaction.Interaction.tgt}}\pysiglinewithargsret{\sphinxbfcode{\sphinxupquote{tgt}}}{\emph{undirected=False}, \emph{resources=None}, \emph{**kwargs}}{}
Returns the name(s) of the target node(s) for each existing
direction on the interaction.
\begin{quote}\begin{description}
\item[{Parameters}] \leavevmode
\sphinxstyleliteralstrong{\sphinxupquote{undirected}} (\sphinxstyleliteralemphasis{\sphinxupquote{bool}}) \textendash{} Optional, \sphinxcode{\sphinxupquote{False}} by default.

\item[{Returns}] \leavevmode
(\sphinxstyleemphasis{list}) \textendash{} Contains the name(s) for the target node(s).
This means if the interaction is bidirectional, the list
will contain both identifiers on the edge. If the
interaction is undirected, an empty list will be returned.

\end{description}\end{quote}

\end{fulllineitems}

\index{tgt\_by\_resource() (pypath.core.interaction.Interaction method)@\spxentry{tgt\_by\_resource()}\spxextra{pypath.core.interaction.Interaction method}}

\begin{fulllineitems}
\phantomsection\label{\detokenize{reference:pypath.core.interaction.Interaction.tgt_by_resource}}\pysiglinewithargsret{\sphinxbfcode{\sphinxupquote{tgt\_by\_resource}}}{\emph{resource}}{}
Returns the name(s) of the target node(s) for each existing
direction on the interaction for a specific \sphinxstyleemphasis{resource}.
\begin{quote}\begin{description}
\item[{Parameters}] \leavevmode
\sphinxstyleliteralstrong{\sphinxupquote{resource}} (\sphinxstyleliteralemphasis{\sphinxupquote{str}}) \textendash{} Name of the resource according to which the information is to
be retrieved.

\item[{Returns}] \leavevmode
(\sphinxstyleemphasis{list}) \textendash{} Contains the name(s) for the target node(s)
according to the specified \sphinxstyleemphasis{resource}. This means if the
interaction is bidirectional, the list will contain both
identifiers on the edge. If the specified \sphinxstyleemphasis{source} is not
found or invalid, an empty list will be returned.

\end{description}\end{quote}

\end{fulllineitems}

\index{translate() (pypath.core.interaction.Interaction method)@\spxentry{translate()}\spxextra{pypath.core.interaction.Interaction method}}

\begin{fulllineitems}
\phantomsection\label{\detokenize{reference:pypath.core.interaction.Interaction.translate}}\pysiglinewithargsret{\sphinxbfcode{\sphinxupquote{translate}}}{\emph{ids}, \emph{new\_attrs=None}}{}
Translates the node names/identifiers according to the
dictionary \sphinxstyleemphasis{ids}. Also is able to change attributes like \sphinxtitleref{id\_type},
\sphinxtitleref{taxon} and \sphinxtitleref{entity\_type}.
\begin{quote}\begin{description}
\item[{Parameters}] \leavevmode\begin{itemize}
\item {} 
\sphinxstyleliteralstrong{\sphinxupquote{ids}} (\sphinxstyleliteralemphasis{\sphinxupquote{dict}}) \textendash{} Dictionary containing (at least) the current names of the
nodes as keys and their translation as values.

\item {} 
\sphinxstyleliteralstrong{\sphinxupquote{new\_attrs}} (\sphinxstyleliteralemphasis{\sphinxupquote{dict}}) \textendash{} Dictionary with new IDs as keys and their dicts of their new
attributes as values. For any attribute not provided here
the attributes from the original instance will be used.
E.g. you can provide {\color{red}\bfseries{}{}`}\{‘1956’: \{‘id\_type’: ‘entrez’\}\}’ if the
new ID type for protein EGFR is Entrez Gene ID.

\end{itemize}

\item[{Returns}] \leavevmode
(\sphinxstyleemphasis{pypath.main.Direction}) \textendash{} The copy of current edge object
with translated node names.

\end{description}\end{quote}

\end{fulllineitems}

\index{unset\_dir() (pypath.core.interaction.Interaction method)@\spxentry{unset\_dir()}\spxextra{pypath.core.interaction.Interaction method}}

\begin{fulllineitems}
\phantomsection\label{\detokenize{reference:pypath.core.interaction.Interaction.unset_dir}}\pysiglinewithargsret{\sphinxbfcode{\sphinxupquote{unset\_dir}}}{\emph{direction}, \emph{only\_sign=False}, \emph{resource=None}, \emph{interaction\_type=None}, \emph{via=False}, \emph{source=None}}{}
Removes directionality and/or source information of the
specified \sphinxstyleemphasis{direction}. Modifies attribute \sphinxcode{\sphinxupquote{dirs}} and
\sphinxcode{\sphinxupquote{sources}}.
\begin{quote}\begin{description}
\item[{Parameters}] \leavevmode\begin{itemize}
\item {} 
\sphinxstyleliteralstrong{\sphinxupquote{direction}} (\sphinxstyleliteralemphasis{\sphinxupquote{tuple}}) \textendash{} Or {[}str{]} (if \sphinxcode{\sphinxupquote{'undirected'}}) the pair of nodes specifying
the directionality from which the information is to be
removed.

\item {} 
\sphinxstyleliteralstrong{\sphinxupquote{resource}} (\sphinxstyleliteralemphasis{\sphinxupquote{set}}) \textendash{} Optional, \sphinxcode{\sphinxupquote{None}} by default. If specified, determines
which specific source(s) is(are) to be removed from
\sphinxcode{\sphinxupquote{sources}} attribute in the specified \sphinxstyleemphasis{direction}.

\end{itemize}

\end{description}\end{quote}

\end{fulllineitems}

\index{unset\_direction() (pypath.core.interaction.Interaction method)@\spxentry{unset\_direction()}\spxextra{pypath.core.interaction.Interaction method}}

\begin{fulllineitems}
\phantomsection\label{\detokenize{reference:pypath.core.interaction.Interaction.unset_direction}}\pysiglinewithargsret{\sphinxbfcode{\sphinxupquote{unset\_direction}}}{\emph{direction}, \emph{only\_sign=False}, \emph{resource=None}, \emph{interaction\_type=None}, \emph{via=False}, \emph{source=None}}{}
Removes directionality and/or source information of the
specified \sphinxstyleemphasis{direction}. Modifies attribute \sphinxcode{\sphinxupquote{dirs}} and
\sphinxcode{\sphinxupquote{sources}}.
\begin{quote}\begin{description}
\item[{Parameters}] \leavevmode\begin{itemize}
\item {} 
\sphinxstyleliteralstrong{\sphinxupquote{direction}} (\sphinxstyleliteralemphasis{\sphinxupquote{tuple}}) \textendash{} Or {[}str{]} (if \sphinxcode{\sphinxupquote{'undirected'}}) the pair of nodes specifying
the directionality from which the information is to be
removed.

\item {} 
\sphinxstyleliteralstrong{\sphinxupquote{resource}} (\sphinxstyleliteralemphasis{\sphinxupquote{set}}) \textendash{} Optional, \sphinxcode{\sphinxupquote{None}} by default. If specified, determines
which specific source(s) is(are) to be removed from
\sphinxcode{\sphinxupquote{sources}} attribute in the specified \sphinxstyleemphasis{direction}.

\end{itemize}

\end{description}\end{quote}

\end{fulllineitems}

\index{unset\_interaction\_type() (pypath.core.interaction.Interaction method)@\spxentry{unset\_interaction\_type()}\spxextra{pypath.core.interaction.Interaction method}}

\begin{fulllineitems}
\phantomsection\label{\detokenize{reference:pypath.core.interaction.Interaction.unset_interaction_type}}\pysiglinewithargsret{\sphinxbfcode{\sphinxupquote{unset\_interaction\_type}}}{\emph{interaction\_type}}{}
Removes all evidences with a certain \sphinxcode{\sphinxupquote{interaction\_type}}.

\end{fulllineitems}

\index{unset\_sign() (pypath.core.interaction.Interaction method)@\spxentry{unset\_sign()}\spxextra{pypath.core.interaction.Interaction method}}

\begin{fulllineitems}
\phantomsection\label{\detokenize{reference:pypath.core.interaction.Interaction.unset_sign}}\pysiglinewithargsret{\sphinxbfcode{\sphinxupquote{unset\_sign}}}{\emph{direction}, \emph{sign}, \emph{resource=None}, \emph{interaction\_type=None}, \emph{via=False}, \emph{source=None}}{}
Removes sign and/or source information of the specified
\sphinxstyleemphasis{direction} and \sphinxstyleemphasis{sign}. Modifies attribute \sphinxcode{\sphinxupquote{positive}}
and \sphinxcode{\sphinxupquote{positive\_sources}} or \sphinxcode{\sphinxupquote{negative}} and
\sphinxcode{\sphinxupquote{negative\_sources}} (or
\sphinxcode{\sphinxupquote{positive\_attributes}}/\sphinxcode{\sphinxupquote{negative\_sources}}
only if \sphinxcode{\sphinxupquote{source=True}}).
\begin{quote}\begin{description}
\item[{Parameters}] \leavevmode\begin{itemize}
\item {} 
\sphinxstyleliteralstrong{\sphinxupquote{direction}} (\sphinxstyleliteralemphasis{\sphinxupquote{tuple}}) \textendash{} The pair of nodes specifying the directionality from which
the information is to be removed.

\item {} 
\sphinxstyleliteralstrong{\sphinxupquote{sign}} (\sphinxstyleliteralemphasis{\sphinxupquote{str}}) \textendash{} Sign from which the information is to be removed. Must be
either \sphinxcode{\sphinxupquote{'positive'}} or \sphinxcode{\sphinxupquote{'negative'}}.

\item {} 
\sphinxstyleliteralstrong{\sphinxupquote{source}} (\sphinxstyleliteralemphasis{\sphinxupquote{set}}) \textendash{} Optional, \sphinxcode{\sphinxupquote{None}} by default. If specified, determines
which source(s) is(are) to be removed from the sources in
the specified \sphinxstyleemphasis{direction} and \sphinxstyleemphasis{sign}.

\end{itemize}

\end{description}\end{quote}

\end{fulllineitems}

\index{which\_directions() (pypath.core.interaction.Interaction method)@\spxentry{which\_directions()}\spxextra{pypath.core.interaction.Interaction method}}

\begin{fulllineitems}
\phantomsection\label{\detokenize{reference:pypath.core.interaction.Interaction.which_directions}}\pysiglinewithargsret{\sphinxbfcode{\sphinxupquote{which\_directions}}}{\emph{resources=None}, \emph{effect=None}}{}
Returns the pair(s) of nodes for which there is information
about their directionality.
\begin{quote}\begin{description}
\item[{Parameters}] \leavevmode\begin{itemize}
\item {} 
\sphinxstyleliteralstrong{\sphinxupquote{effect}} (\sphinxstyleliteralemphasis{\sphinxupquote{str}}) \textendash{} Either \sphinxstyleemphasis{positive} or \sphinxstyleemphasis{negative}.

\item {} 
\sphinxstyleliteralstrong{\sphinxupquote{resources}} (\sphinxstyleliteralemphasis{\sphinxupquote{str}}\sphinxstyleliteralemphasis{\sphinxupquote{,}}\sphinxstyleliteralemphasis{\sphinxupquote{set}}) \textendash{} Limits the query to one or more resources. Optional.

\end{itemize}

\item[{Returns}] \leavevmode
(\sphinxstyleemphasis{tuple}) \textendash{} Tuple of tuples with pairs of nodes where the
first element is the source and the second is the target
entity, according to the given resources and limited to the
effect.

\end{description}\end{quote}

\end{fulllineitems}

\index{which\_dirs() (pypath.core.interaction.Interaction method)@\spxentry{which\_dirs()}\spxextra{pypath.core.interaction.Interaction method}}

\begin{fulllineitems}
\phantomsection\label{\detokenize{reference:pypath.core.interaction.Interaction.which_dirs}}\pysiglinewithargsret{\sphinxbfcode{\sphinxupquote{which\_dirs}}}{\emph{resources=None}, \emph{effect=None}}{}
Returns the pair(s) of nodes for which there is information
about their directionality.
\begin{quote}\begin{description}
\item[{Parameters}] \leavevmode\begin{itemize}
\item {} 
\sphinxstyleliteralstrong{\sphinxupquote{effect}} (\sphinxstyleliteralemphasis{\sphinxupquote{str}}) \textendash{} Either \sphinxstyleemphasis{positive} or \sphinxstyleemphasis{negative}.

\item {} 
\sphinxstyleliteralstrong{\sphinxupquote{resources}} (\sphinxstyleliteralemphasis{\sphinxupquote{str}}\sphinxstyleliteralemphasis{\sphinxupquote{,}}\sphinxstyleliteralemphasis{\sphinxupquote{set}}) \textendash{} Limits the query to one or more resources. Optional.

\end{itemize}

\item[{Returns}] \leavevmode
(\sphinxstyleemphasis{tuple}) \textendash{} Tuple of tuples with pairs of nodes where the
first element is the source and the second is the target
entity, according to the given resources and limited to the
effect.

\end{description}\end{quote}

\end{fulllineitems}

\index{which\_signs() (pypath.core.interaction.Interaction method)@\spxentry{which\_signs()}\spxextra{pypath.core.interaction.Interaction method}}

\begin{fulllineitems}
\phantomsection\label{\detokenize{reference:pypath.core.interaction.Interaction.which_signs}}\pysiglinewithargsret{\sphinxbfcode{\sphinxupquote{which\_signs}}}{\emph{resources=None}, \emph{effect=None}}{}
Returns the pair(s) of nodes for which there is information
about their effect signs.
\begin{quote}\begin{description}
\item[{Parameters}] \leavevmode\begin{itemize}
\item {} 
\sphinxstyleliteralstrong{\sphinxupquote{resources}} (\sphinxstyleliteralemphasis{\sphinxupquote{str}}\sphinxstyleliteralemphasis{\sphinxupquote{,}}\sphinxstyleliteralemphasis{\sphinxupquote{set}}) \textendash{} Limits the query to one or more resources. Optional.

\item {} 
\sphinxstyleliteralstrong{\sphinxupquote{effect}} (\sphinxstyleliteralemphasis{\sphinxupquote{str}}) \textendash{} Either \sphinxstyleemphasis{positive} or \sphinxstyleemphasis{negative}, limiting the query to positive
or negative effects; for any other values effects of both
signs will be returned.

\end{itemize}

\item[{Returns}] \leavevmode
(\sphinxstyleemphasis{tuple}) \textendash{} Tuple of tuples with pairs of nodes where the
first element is a tuple of the source and the target entity,
while the second element is the effect sign, according to
the given resources. E.g. (((‘A’, ‘B’), ‘positive’),)

\end{description}\end{quote}

\end{fulllineitems}


\end{fulllineitems}

\index{InteractionDataFrameRecord (class in pypath.core.interaction)@\spxentry{InteractionDataFrameRecord}\spxextra{class in pypath.core.interaction}}

\begin{fulllineitems}
\phantomsection\label{\detokenize{reference:pypath.core.interaction.InteractionDataFrameRecord}}\pysiglinewithargsret{\sphinxbfcode{\sphinxupquote{class }}\sphinxcode{\sphinxupquote{pypath.core.interaction.}}\sphinxbfcode{\sphinxupquote{InteractionDataFrameRecord}}}{\emph{id\_a}, \emph{id\_b}, \emph{type\_a}, \emph{type\_b}, \emph{directed}, \emph{effect}, \emph{type}, \emph{dmodel}, \emph{sources}, \emph{references}}{}~\index{directed (pypath.core.interaction.InteractionDataFrameRecord attribute)@\spxentry{directed}\spxextra{pypath.core.interaction.InteractionDataFrameRecord attribute}}

\begin{fulllineitems}
\phantomsection\label{\detokenize{reference:pypath.core.interaction.InteractionDataFrameRecord.directed}}\pysigline{\sphinxbfcode{\sphinxupquote{directed}}}
Alias for field number 4

\end{fulllineitems}

\index{dmodel (pypath.core.interaction.InteractionDataFrameRecord attribute)@\spxentry{dmodel}\spxextra{pypath.core.interaction.InteractionDataFrameRecord attribute}}

\begin{fulllineitems}
\phantomsection\label{\detokenize{reference:pypath.core.interaction.InteractionDataFrameRecord.dmodel}}\pysigline{\sphinxbfcode{\sphinxupquote{dmodel}}}
Alias for field number 7

\end{fulllineitems}

\index{effect (pypath.core.interaction.InteractionDataFrameRecord attribute)@\spxentry{effect}\spxextra{pypath.core.interaction.InteractionDataFrameRecord attribute}}

\begin{fulllineitems}
\phantomsection\label{\detokenize{reference:pypath.core.interaction.InteractionDataFrameRecord.effect}}\pysigline{\sphinxbfcode{\sphinxupquote{effect}}}
Alias for field number 5

\end{fulllineitems}

\index{id\_a (pypath.core.interaction.InteractionDataFrameRecord attribute)@\spxentry{id\_a}\spxextra{pypath.core.interaction.InteractionDataFrameRecord attribute}}

\begin{fulllineitems}
\phantomsection\label{\detokenize{reference:pypath.core.interaction.InteractionDataFrameRecord.id_a}}\pysigline{\sphinxbfcode{\sphinxupquote{id\_a}}}
Alias for field number 0

\end{fulllineitems}

\index{id\_b (pypath.core.interaction.InteractionDataFrameRecord attribute)@\spxentry{id\_b}\spxextra{pypath.core.interaction.InteractionDataFrameRecord attribute}}

\begin{fulllineitems}
\phantomsection\label{\detokenize{reference:pypath.core.interaction.InteractionDataFrameRecord.id_b}}\pysigline{\sphinxbfcode{\sphinxupquote{id\_b}}}
Alias for field number 1

\end{fulllineitems}

\index{references (pypath.core.interaction.InteractionDataFrameRecord attribute)@\spxentry{references}\spxextra{pypath.core.interaction.InteractionDataFrameRecord attribute}}

\begin{fulllineitems}
\phantomsection\label{\detokenize{reference:pypath.core.interaction.InteractionDataFrameRecord.references}}\pysigline{\sphinxbfcode{\sphinxupquote{references}}}
Alias for field number 9

\end{fulllineitems}

\index{sources (pypath.core.interaction.InteractionDataFrameRecord attribute)@\spxentry{sources}\spxextra{pypath.core.interaction.InteractionDataFrameRecord attribute}}

\begin{fulllineitems}
\phantomsection\label{\detokenize{reference:pypath.core.interaction.InteractionDataFrameRecord.sources}}\pysigline{\sphinxbfcode{\sphinxupquote{sources}}}
Alias for field number 8

\end{fulllineitems}

\index{type (pypath.core.interaction.InteractionDataFrameRecord attribute)@\spxentry{type}\spxextra{pypath.core.interaction.InteractionDataFrameRecord attribute}}

\begin{fulllineitems}
\phantomsection\label{\detokenize{reference:pypath.core.interaction.InteractionDataFrameRecord.type}}\pysigline{\sphinxbfcode{\sphinxupquote{type}}}
Alias for field number 6

\end{fulllineitems}

\index{type\_a (pypath.core.interaction.InteractionDataFrameRecord attribute)@\spxentry{type\_a}\spxextra{pypath.core.interaction.InteractionDataFrameRecord attribute}}

\begin{fulllineitems}
\phantomsection\label{\detokenize{reference:pypath.core.interaction.InteractionDataFrameRecord.type_a}}\pysigline{\sphinxbfcode{\sphinxupquote{type\_a}}}
Alias for field number 2

\end{fulllineitems}

\index{type\_b (pypath.core.interaction.InteractionDataFrameRecord attribute)@\spxentry{type\_b}\spxextra{pypath.core.interaction.InteractionDataFrameRecord attribute}}

\begin{fulllineitems}
\phantomsection\label{\detokenize{reference:pypath.core.interaction.InteractionDataFrameRecord.type_b}}\pysigline{\sphinxbfcode{\sphinxupquote{type\_b}}}
Alias for field number 3

\end{fulllineitems}


\end{fulllineitems}

\index{InteractionKey (class in pypath.core.interaction)@\spxentry{InteractionKey}\spxextra{class in pypath.core.interaction}}

\begin{fulllineitems}
\phantomsection\label{\detokenize{reference:pypath.core.interaction.InteractionKey}}\pysiglinewithargsret{\sphinxbfcode{\sphinxupquote{class }}\sphinxcode{\sphinxupquote{pypath.core.interaction.}}\sphinxbfcode{\sphinxupquote{InteractionKey}}}{\emph{entity\_a}, \emph{entity\_b}}{}~\index{entity\_a (pypath.core.interaction.InteractionKey attribute)@\spxentry{entity\_a}\spxextra{pypath.core.interaction.InteractionKey attribute}}

\begin{fulllineitems}
\phantomsection\label{\detokenize{reference:pypath.core.interaction.InteractionKey.entity_a}}\pysigline{\sphinxbfcode{\sphinxupquote{entity\_a}}}
Alias for field number 0

\end{fulllineitems}

\index{entity\_b (pypath.core.interaction.InteractionKey attribute)@\spxentry{entity\_b}\spxextra{pypath.core.interaction.InteractionKey attribute}}

\begin{fulllineitems}
\phantomsection\label{\detokenize{reference:pypath.core.interaction.InteractionKey.entity_b}}\pysigline{\sphinxbfcode{\sphinxupquote{entity\_b}}}
Alias for field number 1

\end{fulllineitems}


\end{fulllineitems}



\section{intera}
\label{\detokenize{reference:module-pypath.internals.intera}}\label{\detokenize{reference:intera}}\index{pypath.internals.intera (module)@\spxentry{pypath.internals.intera}\spxextra{module}}
This module provides classes to represent and handle
structural details of protein interactions
i.e. residues, post-translational modifications,
short motifs, domains, domain-motifs and
domain-motif interactions, binding interfaces.


\section{intercell\_annot}
\label{\detokenize{reference:module-pypath.core.intercell_annot}}\label{\detokenize{reference:intercell-annot}}\index{pypath.core.intercell\_annot (module)@\spxentry{pypath.core.intercell\_annot}\spxextra{module}}\index{go\_single\_terms (in module pypath.core.intercell\_annot)@\spxentry{go\_single\_terms}\spxextra{in module pypath.core.intercell\_annot}}

\begin{fulllineitems}
\phantomsection\label{\detokenize{reference:pypath.core.intercell_annot.go_single_terms}}\pysigline{\sphinxcode{\sphinxupquote{pypath.core.intercell\_annot.}}\sphinxbfcode{\sphinxupquote{go\_single\_terms}}\sphinxbfcode{\sphinxupquote{ = \{'C': \{'axolemma', 'banded collagen fibril', 'cell junction', 'cell surface', 'clathrin-coated pit', 'collagen beaded filament', 'collagen network', 'complex of collagen trimers', 'cytoplasmic side of plasma membrane', 'elastic fiber', 'external side of plasma membrane', 'extracellular matrix', 'extracellular matrix of synaptic cleft', 'extracellular region', 'extracellular region part', 'extracellular vesicle', 'extrinsic component of plasma membrane', 'extrinsic component of postsynaptic density membrane', 'extrinsic component of presynaptic membrane', 'fibronectin fibril', 'immunological synapse', 'intrinsic component of plasma membrane', 'intrinsic component of postsynaptic density membrane', 'intrinsic component of presynaptic membrane', 'intrinsic component of synaptic vesicle membrane', 'neuron projection membrane', 'neuronal cell body membrane', 'photoreceptor inner segment membrane', 'photoreceptor outer segment membrane', 'plasma membrane', 'plasma membrane raft', 'postsynaptic density membrane', 'presynaptic active zone membrane', 'presynaptic endocytic zone', 'presynaptic endocytic zone membrane', 'presynaptic membrane', 'stereocilia coupling link', 'stereocilium membrane', 'synaptic vesicle', 'synaptic vesicle lumen', 'synaptic vesicle membrane'\}, 'F': \{'antigen binding', 'antioxidant activity', 'binding', 'cargo adaptor activity', 'cargo receptor activity', 'catalytic activity', 'catalytic activity, acting on a protein', 'cell adhesion mediator activity', 'channel activator activity', 'channel inhibitor activity', 'channel regulator activity', 'drug transmembrane transporter activity', 'enzyme activator activity', 'enzyme inhibitor activity', 'enzyme regulator activity', 'extracellular matrix binding', 'extracellular matrix structural constituent', 'hormone binding', 'hydroxyapatite binding', 'ion channel inhibitor activity', 'ion channel regulator activity', 'ion transmembrane transporter activity', 'molecular carrier activity', 'negative regulation of binding', 'negative regulation of catalytic activity', 'negative regulation of ion transmembrane transporter activity', 'negative regulation of molecular function', 'negative regulation of signaling receptor activity', 'negative regulation of transporter activity', 'neurotransmitter binding', 'neurotransmitter receptor regulator activity', 'peptidase activator activity', 'peptidase activity', 'peptidase inhibitor activity', 'peptidase regulator activity', 'positive regulation of binding', 'positive regulation of catalytic activity', 'positive regulation of ion transmembrane transporter activity', 'positive regulation of molecular function', 'positive regulation of signaling receptor activity', 'positive regulation of transporter activity', 'protein folding chaperone', 'receptor complex', 'receptor inhibitor activity', 'receptor ligand activity', 'receptor regulator activity', 'regulation of binding', 'regulation of catalytic activity', 'regulation of ion transmembrane transporter activity', 'regulation of molecular function', 'regulation of peptidase activity', 'regulation of signaling receptor activity', 'regulation of transmembrane transporter activity', 'regulation of transporter activity', 'signaling receptor activator activity', 'signaling receptor activity', 'structural constituent of bone', 'structural molecule activity', 'transforming growth factor beta receptor,cytoplasmic mediator activity', 'transmembrane transporter activity', 'transporter activity'\}, 'P': \{'autocrine signaling', 'cell activation', 'cell adhesion', 'cell adhesion molecule production', 'cell chemotaxis to fibroblast growth factor', 'cell communication', 'cell communication by chemical coupling', 'cell communication by electrical coupling', 'cell junction assembly', 'cell junction organization', 'cell-cell adhesion', 'cell-cell adhesion in response to extracellular stimulus', 'cell-cell recognition', 'cell-cell signaling', 'cell-cell signaling via exosome', 'cell-matrix recognition', 'cell-substrate adhesion', 'cellular response to cell-matrix adhesion', 'cellular response to extracellular stimulus', 'collagen metabolic process', 'connective tissue replacement', 'contact inhibition', 'cytokine production', 'cytokine secretion', 'endothelial cell activation', 'endothelial cell chemotaxis', 'endothelial cell migration', 'epithelial cell apoptotic process', 'epithelial cell migration', 'epithelial structure maintenance', 'epithelial-mesenchymal cell signaling', 'establishment or maintenance of cell polaritymyofibroblast cell apoptotic process', 'exocytic process', 'extracellular exosome assembly', 'extracellular matrix assembly', 'extracellular matrix constituent secretion', 'extracellular matrix organization', 'extracellular matrix-cell signaling', 'extracellular vesicle biogenesis', 'fibroblast activation', 'fibroblast apoptotic process', 'fibroblast chemotaxis', 'fibroblast migration', 'gap junction-mediated intercellular transport', 'hormone metabolic process', 'hormone secretion', 'inner medulla of kidney', 'intercellular bridge organization', 'ion channel activity', 'kidney pyramid', 'leukocyte activation', 'leukocyte chemotaxis', 'leukocyte migration', 'membrane docking', 'membrane raft localization', 'membrane to membrane docking', 'nephron', 'outer medulla of kidney', 'paracrine signaling', 'protein to membrane docking', 'receptor clustering', 'receptor diffusion trapping', 'regulation of receptor recycling', 'secretion by cell', 'signal release', 'substrate-dependent cell migration', 'synaptic signaling', 'vesicle-mediated intercellular transport'\}\}}}}
Higher level classes of intercellular communication roles.

\end{fulllineitems}



\section{intercell}
\label{\detokenize{reference:module-pypath.core.intercell}}\label{\detokenize{reference:intercell}}\index{pypath.core.intercell (module)@\spxentry{pypath.core.intercell}\spxextra{module}}\index{IntercellRole (class in pypath.core.intercell)@\spxentry{IntercellRole}\spxextra{class in pypath.core.intercell}}

\begin{fulllineitems}
\phantomsection\label{\detokenize{reference:pypath.core.intercell.IntercellRole}}\pysiglinewithargsret{\sphinxbfcode{\sphinxupquote{class }}\sphinxcode{\sphinxupquote{pypath.core.intercell.}}\sphinxbfcode{\sphinxupquote{IntercellRole}}}{\emph{source}, \emph{role}}{}~\index{role (pypath.core.intercell.IntercellRole attribute)@\spxentry{role}\spxextra{pypath.core.intercell.IntercellRole attribute}}

\begin{fulllineitems}
\phantomsection\label{\detokenize{reference:pypath.core.intercell.IntercellRole.role}}\pysigline{\sphinxbfcode{\sphinxupquote{role}}}
Alias for field number 1

\end{fulllineitems}

\index{source (pypath.core.intercell.IntercellRole attribute)@\spxentry{source}\spxextra{pypath.core.intercell.IntercellRole attribute}}

\begin{fulllineitems}
\phantomsection\label{\detokenize{reference:pypath.core.intercell.IntercellRole.source}}\pysigline{\sphinxbfcode{\sphinxupquote{source}}}
Alias for field number 0

\end{fulllineitems}


\end{fulllineitems}



\section{log}
\label{\detokenize{reference:module-pypath.share.log}}\label{\detokenize{reference:log}}\index{pypath.share.log (module)@\spxentry{pypath.share.log}\spxextra{module}}\index{new\_logger() (in module pypath.share.log)@\spxentry{new\_logger()}\spxextra{in module pypath.share.log}}

\begin{fulllineitems}
\phantomsection\label{\detokenize{reference:pypath.share.log.new_logger}}\pysiglinewithargsret{\sphinxcode{\sphinxupquote{pypath.share.log.}}\sphinxbfcode{\sphinxupquote{new\_logger}}}{\emph{name=None}, \emph{logdir=None}, \emph{verbosity=None}, \emph{**kwargs}}{}
Returns a new logger with default settings (can be customized).
\begin{description}
\item[{name}] \leavevmode{[}str{]}
Custom name for the log.

\item[{logdir}] \leavevmode{[}str{]}
Path to the directoty to store log files.

\item[{verbosity}] \leavevmode{[}int{]}
Verbosity level, lowest is 0. Messages from levels above this
won’t be written to the log..

\end{description}

\sphinxcode{\sphinxupquote{log.Logger}} instance.

\end{fulllineitems}



\section{main}
\label{\detokenize{reference:module-pypath.legacy.main}}\label{\detokenize{reference:main}}\index{pypath.legacy.main (module)@\spxentry{pypath.legacy.main}\spxextra{module}}\index{PyPath (class in pypath.legacy.main)@\spxentry{PyPath}\spxextra{class in pypath.legacy.main}}

\begin{fulllineitems}
\phantomsection\label{\detokenize{reference:pypath.legacy.main.PyPath}}\pysiglinewithargsret{\sphinxbfcode{\sphinxupquote{class }}\sphinxcode{\sphinxupquote{pypath.legacy.main.}}\sphinxbfcode{\sphinxupquote{PyPath}}}{\emph{ncbi\_tax\_id=None}, \emph{copy=None}, \emph{name='unnamed'}, \emph{cache\_dir=None}, \emph{outdir='results'}, \emph{loglevel=0}, \emph{loops=False}}{}
This is the a \sphinxstyleemphasis{legacy} object representing a molecular interaction
network. At some point it will be removed, we don’t recommend to rely
on it when you build your applications.
The \sphinxcode{\sphinxupquote{pypath.network.Network}} object offers a much clearer and
more versatile API. As of end of 2019 not all functionalities have been
migrated to the new API. For this reason we offer an intermediate
solution: in this \sphinxtitleref{igraph} based object the \sphinxtitleref{attrs} edge attribute
holds instances of \sphinxcode{\sphinxupquote{pypath.interaction.Interaction}} objects,
the same type of object we use to represent interactions in the new
\sphinxcode{\sphinxupquote{pypath.network.Network}}.
At the same time we will keep supporting \sphinxtitleref{igraph} with a method for
converting \sphinxcode{\sphinxupquote{pypath.network.Network}} to a
\sphinxcode{\sphinxupquote{igraph.Graph}} object, however this won’t provide all the
methods available here but will serve only the purpose to make it
possible to use the graph theory methods from the \sphinxtitleref{igraph} library
on networks built with \sphinxtitleref{pypath}.

An object representing a molecular interaction network.
\begin{quote}\begin{description}
\item[{Parameters}] \leavevmode\begin{itemize}
\item {} 
\sphinxstyleliteralstrong{\sphinxupquote{ncbi\_tax\_id}} (\sphinxstyleliteralemphasis{\sphinxupquote{int}}) \textendash{} Optional, \sphinxcode{\sphinxupquote{9606}} (Homo sapiens) by default. NCBI Taxonomic
identifier of the organism from which the data will be
downloaded.

\item {} 
\sphinxstyleliteralstrong{\sphinxupquote{default\_name\_type}} (\sphinxstyleliteralemphasis{\sphinxupquote{dict}}) \textendash{} Optional, \sphinxcode{\sphinxupquote{\{'protein': 'uniprot', 'mirna': 'mirbase', 'drug':
'chembl', 'lncrna': 'lncrna-genesymbol'\}}} by default. Contains
the default identifier types to which the downloaded data will
be converted. If others are used, user may need to provide the
format definitions for the conversion tables.

\item {} 
\sphinxstyleliteralstrong{\sphinxupquote{copy}} (\sphinxstyleliteralemphasis{\sphinxupquote{pypath.main.PyPath}}) \textendash{} Optional, \sphinxcode{\sphinxupquote{None}} by default. Other
\sphinxcode{\sphinxupquote{pypath.main.PyPath}} instance from which the data will
be copied.

\item {} 
\sphinxstyleliteralstrong{\sphinxupquote{name}} (\sphinxstyleliteralemphasis{\sphinxupquote{str}}) \textendash{} Optional, \sphinxcode{\sphinxupquote{'unnamed'}} by default. Session or project name
(custom).

\item {} 
\sphinxstyleliteralstrong{\sphinxupquote{outdir}} (\sphinxstyleliteralemphasis{\sphinxupquote{str}}) \textendash{} Optional, \sphinxcode{\sphinxupquote{'results'}} by default. Output directory where to
store all output files.

\item {} 
\sphinxstyleliteralstrong{\sphinxupquote{loglevel}} (\sphinxstyleliteralemphasis{\sphinxupquote{int}}) \textendash{} Optional, 0 by default. Sets the level of the logger.
The higher the level the more messages will be written to the log.

\item {} 
\sphinxstyleliteralstrong{\sphinxupquote{loops}} (\sphinxstyleliteralemphasis{\sphinxupquote{bool}}) \textendash{} Optional, \sphinxcode{\sphinxupquote{False}} by default. Determines if self-loop edges
are allowed in the graph.

\end{itemize}

\item[{Variables}] \leavevmode\begin{itemize}
\item {} 
\sphinxstyleliteralstrong{\sphinxupquote{adjlist}} (\sphinxstyleliteralemphasis{\sphinxupquote{list}}) \textendash{} List of {[}set{]} containing the adjacency of each node. See
{\hyperref[\detokenize{reference:pypath.legacy.main.PyPath.update_adjlist}]{\sphinxcrossref{\sphinxcode{\sphinxupquote{PyPath.update\_adjlist()}}}}} method for more information.

\item {} 
\sphinxstyleliteralstrong{\sphinxupquote{chembl}} (\sphinxstyleliteralemphasis{\sphinxupquote{pypath.chembl.Chembl}}) \textendash{} Contains the ChEMBL data. See \sphinxcode{\sphinxupquote{pypath.chembl}} module
documentation for more information.

\item {} 
\sphinxstyleliteralstrong{\sphinxupquote{chembl\_mysql}} (\sphinxstyleliteralemphasis{\sphinxupquote{tuple}}) \textendash{} DEPRECATED Contains the MySQL parameters used by the
\sphinxcode{\sphinxupquote{pypath.mapping}} module to load the ChEMBL ID conversion
tables.

\item {} 
\sphinxstyleliteralstrong{\sphinxupquote{data}} (\sphinxstyleliteralemphasis{\sphinxupquote{dict}}) \textendash{} Stores the loaded interaction and attribute table. See
\sphinxcode{\sphinxupquote{PyPath.read\_data\_file()}} method for more information.

\item {} 
\sphinxstyleliteralstrong{\sphinxupquote{db\_dict}} (\sphinxstyleliteralemphasis{\sphinxupquote{dict}}) \textendash{} Dictionary of dictionaries. Outer-level keys are \sphinxcode{\sphinxupquote{'nodes'}} and
\sphinxcode{\sphinxupquote{'edges'}}, corresponding values are {[}dict{]} whose keys are the
database sources with values of type {[}set{]} containing the
edge/node indexes for which that database provided some
information.

\item {} 
\sphinxstyleliteralstrong{\sphinxupquote{dgraph}} (\sphinxstyleliteralemphasis{\sphinxupquote{igraph.Graph}}) \textendash{} Directed network graph object.

\item {} 
\sphinxstyleliteralstrong{\sphinxupquote{disclaimer}} (\sphinxstyleliteralemphasis{\sphinxupquote{str}}) \textendash{} Disclaimer text.

\item {} 
\sphinxstyleliteralstrong{\sphinxupquote{dlabDct}} (\sphinxstyleliteralemphasis{\sphinxupquote{dict}}) \textendash{} Maps the directed graph node labels {[}str{]} (keys) to their
indices {[}int{]} (values).

\item {} 
\sphinxstyleliteralstrong{\sphinxupquote{dnodDct}} (\sphinxstyleliteralemphasis{\sphinxupquote{dict}}) \textendash{} Maps the directed graph node names {[}str{]} (keys) to their indices
{[}int{]} (values).

\item {} 
\sphinxstyleliteralstrong{\sphinxupquote{dnodInd}} (\sphinxstyleliteralemphasis{\sphinxupquote{set}}) \textendash{} Stores the directed graph node names {[}str{]}.

\item {} 
\sphinxstyleliteralstrong{\sphinxupquote{dnodLab}} (\sphinxstyleliteralemphasis{\sphinxupquote{dict}}) \textendash{} Maps the directed graph node indices {[}int{]} (keys) to their
labels {[}str{]} (values).

\item {} 
\sphinxstyleliteralstrong{\sphinxupquote{dnodNam}} (\sphinxstyleliteralemphasis{\sphinxupquote{dict}}) \textendash{} Maps the directed graph node indices {[}int{]} (keys) to their names
{[}str{]} (values).

\item {} 
\sphinxstyleliteralstrong{\sphinxupquote{edgeAttrs}} (\sphinxstyleliteralemphasis{\sphinxupquote{dict}}) \textendash{} Stores the edge attribute names {[}str{]} as keys and their
corresponding types (e.g.: \sphinxcode{\sphinxupquote{set}}, \sphinxcode{\sphinxupquote{list}}, \sphinxcode{\sphinxupquote{str}}, …) as
values.

\item {} 
\sphinxstyleliteralstrong{\sphinxupquote{exp}} (\sphinxstyleliteralemphasis{\sphinxupquote{pandas.DataFrame}}) \textendash{} Stores the expression data for the nodes (if loaded).

\item {} 
\sphinxstyleliteralstrong{\sphinxupquote{exp\_prod}} (\sphinxstyleliteralemphasis{\sphinxupquote{pandas.DataFrame}}) \textendash{} Stores the edge expression data (as the product of the
normalized expression between the pair of nodes by default). For
more details see \sphinxcode{\sphinxupquote{pypath.main.PyPath.edges\_expression()}}.

\item {} 
\sphinxstyleliteralstrong{\sphinxupquote{exp\_samples}} (\sphinxstyleliteralemphasis{\sphinxupquote{set}}) \textendash{} Contains a list of tissues as downloaded by ProteomicsDB. See
{\hyperref[\detokenize{reference:pypath.legacy.main.PyPath.get_proteomicsdb}]{\sphinxcrossref{\sphinxcode{\sphinxupquote{PyPath.get\_proteomicsdb()}}}}} for more information.

\item {} 
\sphinxstyleliteralstrong{\sphinxupquote{failed\_edges}} (\sphinxstyleliteralemphasis{\sphinxupquote{list}}) \textendash{} List of lists containing information about the failed edges.
Each failed edge sublist contains (in this order): {[}tuple{]} with
the node IDs, {[}str{]} names of nodes A and B, {[}int{]} IDs of nodes
A and B and {[}int{]} IDs of the edges in both directions.

\item {} 
{\hyperref[\detokenize{reference:module-pypath.utils.go}]{\sphinxcrossref{\sphinxstyleliteralstrong{\sphinxupquote{go}}}}} (\sphinxstyleliteralemphasis{\sphinxupquote{dict}}) \textendash{} Contains the organism(s) NCBI taxonomy ID as key {[}int{]} and
\sphinxcode{\sphinxupquote{pypath.go.GOAnnotation}} object as value, which
contains the GO annotations for the nodes in the graph. See
\sphinxcode{\sphinxupquote{pypath.go.GOAnnotation}} for more information.

\item {} 
\sphinxstyleliteralstrong{\sphinxupquote{graph}} (\sphinxstyleliteralemphasis{\sphinxupquote{igraph.Graph}}) \textendash{} Undirected network graph object.

\item {} 
\sphinxstyleliteralstrong{\sphinxupquote{gsea}} (\sphinxstyleliteralemphasis{\sphinxupquote{pypath.gsea.GSEA}}) \textendash{} Contains the loaded gene-sets from MSigDB. See
\sphinxcode{\sphinxupquote{pypath.gsea.GSEA}} for more information.

\item {} 
\sphinxstyleliteralstrong{\sphinxupquote{has\_cats}} (\sphinxstyleliteralemphasis{\sphinxupquote{set}}) \textendash{} Contains the categories (e.g.: resource types) {[}str{]} loaded in
the current network. Possible categories are: \sphinxcode{\sphinxupquote{'m'}} for
PTM/enzyme-substrate resources, \sphinxcode{\sphinxupquote{'p'}} for pathway/activity
flow resources, \sphinxcode{\sphinxupquote{'i'}} for undirected/PPI resources, \sphinxcode{\sphinxupquote{'r'}}
for process description/reaction resources and \sphinxcode{\sphinxupquote{'t'}} for
transcription resources.

\item {} 
\sphinxstyleliteralstrong{\sphinxupquote{htp}} (\sphinxstyleliteralemphasis{\sphinxupquote{dict}}) \textendash{} Contains information about high-throughput data of the network
for different thresholds {[}int{]} (keys). Values are {[}dict{]}
containing the number of references (\sphinxcode{\sphinxupquote{'rnum'}}) {[}int{]}, number
of edges (\sphinxcode{\sphinxupquote{'enum'}}) {[}int{]}, number of sources (\sphinxcode{\sphinxupquote{'snum'}})
{[}int{]} and list of PMIDs of the most common references above the
given threshold (\sphinxcode{\sphinxupquote{'htrefs'}}) {[}set{]}.

\item {} 
\sphinxstyleliteralstrong{\sphinxupquote{labDct}} (\sphinxstyleliteralemphasis{\sphinxupquote{dict}}) \textendash{} Maps the undirected graph node labels {[}str{]} (keys) to their
indices {[}int{]} (values).

\item {} 
\sphinxstyleliteralstrong{\sphinxupquote{lists}} (\sphinxstyleliteralemphasis{\sphinxupquote{dict}}) \textendash{} Contains specific lists of nodes (values) for different
categories {[}str{]} (keys). These can to be loaded from a file or
a resource. Some methods include
\sphinxcode{\sphinxupquote{pypath.main.PyPath.receptor\_list()}} (\sphinxcode{\sphinxupquote{'rec'}}),
\sphinxcode{\sphinxupquote{pypath.main.PyPath.druggability\_list()}} (\sphinxcode{\sphinxupquote{'dgb'}}),
\sphinxcode{\sphinxupquote{pypath.main.PyPath.kinases\_list()}} (\sphinxcode{\sphinxupquote{'kin'}}),
\sphinxcode{\sphinxupquote{pypath.main.PyPath.tfs\_list()}} (\sphinxcode{\sphinxupquote{'tf'}}),
\sphinxcode{\sphinxupquote{pypath.main.PyPath.disease\_genes\_list()}} (\sphinxcode{\sphinxupquote{'dis'}}),
\sphinxcode{\sphinxupquote{pypath.main.PyPath.signaling\_proteins\_list()}}
(\sphinxcode{\sphinxupquote{'sig'}}), \sphinxcode{\sphinxupquote{pypath.main.PyPath.proteome\_list()}}
(\sphinxcode{\sphinxupquote{'proteome'}}) and
\sphinxcode{\sphinxupquote{pypath.main.PyPath.cancer\_drivers\_list()}} (\sphinxcode{\sphinxupquote{'cdv'}}).

\item {} 
\sphinxstyleliteralstrong{\sphinxupquote{loglevel}} (\sphinxstyleliteralemphasis{\sphinxupquote{str}}) \textendash{} The level of the logger.

\item {} 
\sphinxstyleliteralstrong{\sphinxupquote{loops}} (\sphinxstyleliteralemphasis{\sphinxupquote{bool}}) \textendash{} Whether if self-loop edges are allowed in the graph.

\item {} 
\sphinxstyleliteralstrong{\sphinxupquote{mapper}} (\sphinxstyleliteralemphasis{\sphinxupquote{pypath.mapping.Mapper}}) \textendash{} \sphinxcode{\sphinxupquote{pypath.mapper.Mapper}} object for ID conversion and
other ID-related operations across resources.

\item {} 
\sphinxstyleliteralstrong{\sphinxupquote{mutation\_samples}} (\sphinxstyleliteralemphasis{\sphinxupquote{list}}) \textendash{} DEPRECATED

\item {} 
\sphinxstyleliteralstrong{\sphinxupquote{mysql\_conf}} (\sphinxstyleliteralemphasis{\sphinxupquote{tuple}}) \textendash{} DEPRECATED Contains the MySQL parameters used by the
\sphinxcode{\sphinxupquote{pypath.mapping}} module to load the ID conversion
tables.

\item {} 
{\hyperref[\detokenize{reference:pypath.internals.resource.EnzymeSubstrateResourceKey.name}]{\sphinxcrossref{\sphinxstyleliteralstrong{\sphinxupquote{name}}}}} (\sphinxstyleliteralemphasis{\sphinxupquote{str}}) \textendash{} Session or project name (custom).

\item {} 
\sphinxstyleliteralstrong{\sphinxupquote{ncbi\_tax\_id}} (\sphinxstyleliteralemphasis{\sphinxupquote{int}}) \textendash{} NCBI Taxonomic identifier of the organism from which the data
will be downloaded.

\item {} 
\sphinxstyleliteralstrong{\sphinxupquote{negatives}} (\sphinxstyleliteralemphasis{\sphinxupquote{dict}}) \textendash{} Contains a list of negative interactions according to a given
source (e.g.: Negatome database). See
\sphinxcode{\sphinxupquote{pypath.main.PyPath.apply\_negative()}} for more
information.

\item {} 
\sphinxstyleliteralstrong{\sphinxupquote{nodDct}} (\sphinxstyleliteralemphasis{\sphinxupquote{dict}}) \textendash{} Maps the undirected graph node names {[}str{]} (keys) to their
indices {[}int{]} (values).

\item {} 
\sphinxstyleliteralstrong{\sphinxupquote{nodInd}} (\sphinxstyleliteralemphasis{\sphinxupquote{set}}) \textendash{} Stores the undirected graph node names {[}str{]}.

\item {} 
\sphinxstyleliteralstrong{\sphinxupquote{nodLab}} (\sphinxstyleliteralemphasis{\sphinxupquote{dict}}) \textendash{} Maps the undirected graph node indices {[}int{]} (keys) to their
labels {[}str{]} (values).

\item {} 
\sphinxstyleliteralstrong{\sphinxupquote{nodNam}} (\sphinxstyleliteralemphasis{\sphinxupquote{dict}}) \textendash{} Maps the directed graph node indices {[}int{]} (keys) to their names
{[}str{]} (values).

\item {} 
\sphinxstyleliteralstrong{\sphinxupquote{outdir}} (\sphinxstyleliteralemphasis{\sphinxupquote{str}}) \textendash{} Output directory where to store all output files.

\item {} 
\sphinxstyleliteralstrong{\sphinxupquote{palette}} (\sphinxstyleliteralemphasis{\sphinxupquote{list}}) \textendash{} Contains a list of hexadecimal {[}str{]} of colors. Used for
plotting purposes.

\item {} 
\sphinxstyleliteralstrong{\sphinxupquote{pathway\_types}} (\sphinxstyleliteralemphasis{\sphinxupquote{list}}) \textendash{} Contains the names of all the loaded pathway resources {[}str{]}.

\item {} 
\sphinxstyleliteralstrong{\sphinxupquote{pathways}} (\sphinxstyleliteralemphasis{\sphinxupquote{dict}}) \textendash{} Contains the list of pathways (values) for each resource (keys)
loaded in the network.

\item {} 
\sphinxstyleliteralstrong{\sphinxupquote{plots}} (\sphinxstyleliteralemphasis{\sphinxupquote{dict}}) \textendash{} DEPRECATED (?)

\item {} 
\sphinxstyleliteralstrong{\sphinxupquote{proteomicsdb}} (\sphinxstyleliteralemphasis{\sphinxupquote{pypath.proteomicsdb.ProteomicsDB}}) \textendash{} Contains a \sphinxcode{\sphinxupquote{pypath.proteomicsdb.ProteomicsDB}}
instance, see the class documentation for more information.

\item {} 
\sphinxstyleliteralstrong{\sphinxupquote{raw\_data}} (\sphinxstyleliteralemphasis{\sphinxupquote{list}}) \textendash{} Contains a list of loaded edges {[}dict{]} from a data file. See
\sphinxcode{\sphinxupquote{PyPath.read\_data\_file()}} for more information.

\item {} 
{\hyperref[\detokenize{reference:module-pypath.utils.seq}]{\sphinxcrossref{\sphinxstyleliteralstrong{\sphinxupquote{seq}}}}} (\sphinxstyleliteralemphasis{\sphinxupquote{dict}}) \textendash{} (?)

\item {} 
{\hyperref[\detokenize{reference:module-pypath.share.session}]{\sphinxcrossref{\sphinxstyleliteralstrong{\sphinxupquote{session}}}}} (\sphinxstyleliteralemphasis{\sphinxupquote{str}}) \textendash{} Session ID, a five random alphanumeric characters.

\item {} 
\sphinxstyleliteralstrong{\sphinxupquote{session\_name}} (\sphinxstyleliteralemphasis{\sphinxupquote{str}}) \textendash{} Session name and ID (e.g. \sphinxcode{\sphinxupquote{'unnamed-abc12'}}).

\item {} 
\sphinxstyleliteralstrong{\sphinxupquote{sourceNetEdges}} (\sphinxstyleliteralemphasis{\sphinxupquote{igraph.Graph}}) \textendash{} (?)

\item {} 
\sphinxstyleliteralstrong{\sphinxupquote{sourceNetNodes}} (\sphinxstyleliteralemphasis{\sphinxupquote{igraph.Graph}}) \textendash{} (?)

\item {} 
{\hyperref[\detokenize{reference:pypath.core.interaction.InteractionDataFrameRecord.sources}]{\sphinxcrossref{\sphinxstyleliteralstrong{\sphinxupquote{sources}}}}} (\sphinxstyleliteralemphasis{\sphinxupquote{list}}) \textendash{} List contianing the names of the loaded resources {[}str{]}.

\item {} 
\sphinxstyleliteralstrong{\sphinxupquote{u\_pfam}} (\sphinxstyleliteralemphasis{\sphinxupquote{dict}}) \textendash{} Dictionary of dictionaries, contains the mapping of UniProt IDs
to their respective protein families and other information.

\item {} 
\sphinxstyleliteralstrong{\sphinxupquote{uniprot\_mapped}} (\sphinxstyleliteralemphasis{\sphinxupquote{list}}) \textendash{} DEPRECATED (?)

\item {} 
\sphinxstyleliteralstrong{\sphinxupquote{unmapped}} (\sphinxstyleliteralemphasis{\sphinxupquote{list}}) \textendash{} Contains the names of unmapped items {[}str{]}. See
\sphinxcode{\sphinxupquote{pypath.main.PyPath.map\_item()}} for more information.

\item {} 
\sphinxstyleliteralstrong{\sphinxupquote{vertexAttrs}} (\sphinxstyleliteralemphasis{\sphinxupquote{dict}}) \textendash{} Stores the node attribute names {[}str{]} as keys and their
corresponding types (e.g.: \sphinxcode{\sphinxupquote{set}}, \sphinxcode{\sphinxupquote{list}}, \sphinxcode{\sphinxupquote{str}}, …) as
values.

\end{itemize}

\end{description}\end{quote}
\index{acsn\_effects() (pypath.legacy.main.PyPath method)@\spxentry{acsn\_effects()}\spxextra{pypath.legacy.main.PyPath method}}

\begin{fulllineitems}
\phantomsection\label{\detokenize{reference:pypath.legacy.main.PyPath.acsn_effects}}\pysiglinewithargsret{\sphinxbfcode{\sphinxupquote{acsn\_effects}}}{\emph{graph=None}}{}
\end{fulllineitems}

\index{add\_genesets() (pypath.legacy.main.PyPath method)@\spxentry{add\_genesets()}\spxextra{pypath.legacy.main.PyPath method}}

\begin{fulllineitems}
\phantomsection\label{\detokenize{reference:pypath.legacy.main.PyPath.add_genesets}}\pysiglinewithargsret{\sphinxbfcode{\sphinxupquote{add\_genesets}}}{\emph{genesets}}{}
\end{fulllineitems}

\index{add\_grouped\_eattr() (pypath.legacy.main.PyPath method)@\spxentry{add\_grouped\_eattr()}\spxextra{pypath.legacy.main.PyPath method}}

\begin{fulllineitems}
\phantomsection\label{\detokenize{reference:pypath.legacy.main.PyPath.add_grouped_eattr}}\pysiglinewithargsret{\sphinxbfcode{\sphinxupquote{add\_grouped\_eattr}}}{\emph{edge}, \emph{attr}, \emph{group}, \emph{value}}{}
Merges (or creates) a given edge attribute as {[}dict{]} of {[}list{]}
values.
\begin{quote}\begin{description}
\item[{Parameters}] \leavevmode\begin{itemize}
\item {} 
\sphinxstyleliteralstrong{\sphinxupquote{edge}} (\sphinxstyleliteralemphasis{\sphinxupquote{int}}) \textendash{} Edge index where the given attribute value is to be merged
or created.

\item {} 
\sphinxstyleliteralstrong{\sphinxupquote{attr}} (\sphinxstyleliteralemphasis{\sphinxupquote{str}}) \textendash{} The name of the attribute. If such attribute does not exist
in the network edges, it will be created on all edges (as an
empty {[}dict{]}, \sphinxstyleemphasis{value} will only be assigned to the given
\sphinxstyleemphasis{edge} and \sphinxstyleemphasis{group}).

\item {} 
\sphinxstyleliteralstrong{\sphinxupquote{group}} (\sphinxstyleliteralemphasis{\sphinxupquote{str}}) \textendash{} The key of the attribute dictionary where \sphinxstyleemphasis{value} is to be
assigned.

\item {} 
\sphinxstyleliteralstrong{\sphinxupquote{value}} (\sphinxstyleliteralemphasis{\sphinxupquote{list}}) \textendash{} The value of the attribute to be assigned/merged.

\end{itemize}

\end{description}\end{quote}

\end{fulllineitems}

\index{add\_grouped\_set\_eattr() (pypath.legacy.main.PyPath method)@\spxentry{add\_grouped\_set\_eattr()}\spxextra{pypath.legacy.main.PyPath method}}

\begin{fulllineitems}
\phantomsection\label{\detokenize{reference:pypath.legacy.main.PyPath.add_grouped_set_eattr}}\pysiglinewithargsret{\sphinxbfcode{\sphinxupquote{add\_grouped\_set\_eattr}}}{\emph{edge}, \emph{attr}, \emph{group}, \emph{value}}{}
Merges (or creates) a given edge attribute as {[}dict{]} of {[}set{]}
values.
\begin{quote}\begin{description}
\item[{Parameters}] \leavevmode\begin{itemize}
\item {} 
\sphinxstyleliteralstrong{\sphinxupquote{edge}} (\sphinxstyleliteralemphasis{\sphinxupquote{int}}) \textendash{} Edge index where the given attribute value is to be merged
or created.

\item {} 
\sphinxstyleliteralstrong{\sphinxupquote{attr}} (\sphinxstyleliteralemphasis{\sphinxupquote{str}}) \textendash{} The name of the attribute. If such attribute does not exist
in the network edges, it will be created on all edges (as an
empty {[}dict{]}, \sphinxstyleemphasis{value} will only be assigned to the given
\sphinxstyleemphasis{edge} and \sphinxstyleemphasis{group}).

\item {} 
\sphinxstyleliteralstrong{\sphinxupquote{group}} (\sphinxstyleliteralemphasis{\sphinxupquote{str}}) \textendash{} The key of the attribute dictionary where \sphinxstyleemphasis{value} is to be
assigned.

\item {} 
\sphinxstyleliteralstrong{\sphinxupquote{value}} (\sphinxstyleliteralemphasis{\sphinxupquote{set}}) \textendash{} The value of the attribute to be assigned/merged.

\end{itemize}

\end{description}\end{quote}

\end{fulllineitems}

\index{add\_list\_eattr() (pypath.legacy.main.PyPath method)@\spxentry{add\_list\_eattr()}\spxextra{pypath.legacy.main.PyPath method}}

\begin{fulllineitems}
\phantomsection\label{\detokenize{reference:pypath.legacy.main.PyPath.add_list_eattr}}\pysiglinewithargsret{\sphinxbfcode{\sphinxupquote{add\_list\_eattr}}}{\emph{edge}, \emph{attr}, \emph{value}}{}
Merges (or creates) a given edge attribute as {[}list{]}.
\begin{quote}\begin{description}
\item[{Parameters}] \leavevmode\begin{itemize}
\item {} 
\sphinxstyleliteralstrong{\sphinxupquote{edge}} (\sphinxstyleliteralemphasis{\sphinxupquote{int}}) \textendash{} Edge index where the given attribute value is to be merged
or created.

\item {} 
\sphinxstyleliteralstrong{\sphinxupquote{attr}} (\sphinxstyleliteralemphasis{\sphinxupquote{str}}) \textendash{} The name of the attribute. If such attribute does not exist
in the network edges, it will be created on all edges (as an
empty {[}list{]}, \sphinxstyleemphasis{value} will only be assigned to the given
\sphinxstyleemphasis{edge}).

\item {} 
\sphinxstyleliteralstrong{\sphinxupquote{value}} (\sphinxstyleliteralemphasis{\sphinxupquote{list}}) \textendash{} The value of the attribute to be assigned/merged.

\end{itemize}

\end{description}\end{quote}

\end{fulllineitems}

\index{add\_set\_eattr() (pypath.legacy.main.PyPath method)@\spxentry{add\_set\_eattr()}\spxextra{pypath.legacy.main.PyPath method}}

\begin{fulllineitems}
\phantomsection\label{\detokenize{reference:pypath.legacy.main.PyPath.add_set_eattr}}\pysiglinewithargsret{\sphinxbfcode{\sphinxupquote{add\_set\_eattr}}}{\emph{edge}, \emph{attr}, \emph{value}}{}
Merges (or creates) a given edge attribute as {[}set{]}.
\begin{quote}\begin{description}
\item[{Parameters}] \leavevmode\begin{itemize}
\item {} 
\sphinxstyleliteralstrong{\sphinxupquote{edge}} (\sphinxstyleliteralemphasis{\sphinxupquote{int}}) \textendash{} Edge index where the given attribute value is to be merged
or created.

\item {} 
\sphinxstyleliteralstrong{\sphinxupquote{attr}} (\sphinxstyleliteralemphasis{\sphinxupquote{str}}) \textendash{} The name of the attribute. If such attribute does not exist
in the network edges, it will be created on all edges (as an
empty {[}set{]}, \sphinxstyleemphasis{value} will only be assigned to the given
\sphinxstyleemphasis{edge}).

\item {} 
\sphinxstyleliteralstrong{\sphinxupquote{value}} (\sphinxstyleliteralemphasis{\sphinxupquote{set}}) \textendash{} The value of the attribute to be assigned/merged.

\end{itemize}

\end{description}\end{quote}

\end{fulllineitems}

\index{affects() (pypath.legacy.main.PyPath method)@\spxentry{affects()}\spxextra{pypath.legacy.main.PyPath method}}

\begin{fulllineitems}
\phantomsection\label{\detokenize{reference:pypath.legacy.main.PyPath.affects}}\pysiglinewithargsret{\sphinxbfcode{\sphinxupquote{affects}}}{\emph{identifier}}{}
\end{fulllineitems}

\index{all\_between() (pypath.legacy.main.PyPath method)@\spxentry{all\_between()}\spxextra{pypath.legacy.main.PyPath method}}

\begin{fulllineitems}
\phantomsection\label{\detokenize{reference:pypath.legacy.main.PyPath.all_between}}\pysiglinewithargsret{\sphinxbfcode{\sphinxupquote{all\_between}}}{\emph{id\_a}, \emph{id\_b}}{}
Checks for any edges (in any direction) between the provided
nodes.
\begin{quote}\begin{description}
\item[{Parameters}] \leavevmode\begin{itemize}
\item {} 
\sphinxstyleliteralstrong{\sphinxupquote{id\_a}} (\sphinxstyleliteralemphasis{\sphinxupquote{str}}) \textendash{} The name of the source node.

\item {} 
\sphinxstyleliteralstrong{\sphinxupquote{id\_b}} (\sphinxstyleliteralemphasis{\sphinxupquote{str}}) \textendash{} The name of the target node.

\end{itemize}

\item[{Returns}] \leavevmode
(\sphinxstyleemphasis{dict}) \textendash{} Contains information on the directionality of
the requested edge. Keys are \sphinxcode{\sphinxupquote{'ab'}} and \sphinxcode{\sphinxupquote{'ba'}}, denoting
the straight/reverse directionalities respectively. Values
are {[}list{]} whose elements are the edge ID or \sphinxcode{\sphinxupquote{None}}
according to the existance of that edge in the following
categories: undirected, straight and reverse (in that
order).

\end{description}\end{quote}

\end{fulllineitems}

\index{all\_neighbours() (pypath.legacy.main.PyPath method)@\spxentry{all\_neighbours()}\spxextra{pypath.legacy.main.PyPath method}}

\begin{fulllineitems}
\phantomsection\label{\detokenize{reference:pypath.legacy.main.PyPath.all_neighbours}}\pysiglinewithargsret{\sphinxbfcode{\sphinxupquote{all\_neighbours}}}{\emph{indices=False}}{}
Looks for the first neighbours of all the nodes and creates an
attribute (\sphinxcode{\sphinxupquote{'neighbours'}}) on each one of them containing a
list of their UniProt IDs.
\begin{quote}\begin{description}
\item[{Parameters}] \leavevmode
\sphinxstyleliteralstrong{\sphinxupquote{indices}} (\sphinxstyleliteralemphasis{\sphinxupquote{bool}}) \textendash{} Optional, \sphinxcode{\sphinxupquote{False}} by default. Whether to list the
neighbour nodes indices or their UniProt IDs.

\end{description}\end{quote}

\end{fulllineitems}

\index{apply\_list() (pypath.legacy.main.PyPath method)@\spxentry{apply\_list()}\spxextra{pypath.legacy.main.PyPath method}}

\begin{fulllineitems}
\phantomsection\label{\detokenize{reference:pypath.legacy.main.PyPath.apply_list}}\pysiglinewithargsret{\sphinxbfcode{\sphinxupquote{apply\_list}}}{\emph{name}, \emph{node\_or\_edge='node'}}{}
Creates vertex or edge attribute based on a list.
\begin{quote}\begin{description}
\item[{Parameters}] \leavevmode\begin{itemize}
\item {} 
\sphinxstyleliteralstrong{\sphinxupquote{name}} (\sphinxstyleliteralemphasis{\sphinxupquote{str}}) \textendash{} The name of the list to be added as attribute. Must have
been previously loaded with
\sphinxcode{\sphinxupquote{pypath.main.PyPath.load\_list()}} or other methods.
See description of \sphinxcode{\sphinxupquote{pypath.main.PyPath.lists}}
attribute for more information.

\item {} 
\sphinxstyleliteralstrong{\sphinxupquote{node\_or\_edge}} (\sphinxstyleliteralemphasis{\sphinxupquote{str}}) \textendash{} Optional, \sphinxcode{\sphinxupquote{'node'}} by default. Whether the attribute list
is to be added to the nodes or to the edges.

\end{itemize}

\end{description}\end{quote}

\end{fulllineitems}

\index{apply\_negative() (pypath.legacy.main.PyPath method)@\spxentry{apply\_negative()}\spxextra{pypath.legacy.main.PyPath method}}

\begin{fulllineitems}
\phantomsection\label{\detokenize{reference:pypath.legacy.main.PyPath.apply_negative}}\pysiglinewithargsret{\sphinxbfcode{\sphinxupquote{apply\_negative}}}{\emph{settings}}{}
Loads a negative interaction source (e.g.: Negatome) into the
current network.
\begin{quote}\begin{description}
\item[{Parameters}] \leavevmode
\sphinxstyleliteralstrong{\sphinxupquote{settings}} (\sphinxstyleliteralemphasis{\sphinxupquote{pypath.input\_formats.NetworkInput}}) \textendash{} \sphinxcode{\sphinxupquote{pypath.input\_formats.NetworkInput}} instance
containing the detailed definition of the input format to
the downloaded file. For instance
\sphinxcode{\sphinxupquote{pypath.data\_formats.negative{[}'negatome'{]}}}

\end{description}\end{quote}

\end{fulllineitems}

\index{basic\_stats() (pypath.legacy.main.PyPath method)@\spxentry{basic\_stats()}\spxextra{pypath.legacy.main.PyPath method}}

\begin{fulllineitems}
\phantomsection\label{\detokenize{reference:pypath.legacy.main.PyPath.basic_stats}}\pysiglinewithargsret{\sphinxbfcode{\sphinxupquote{basic\_stats}}}{\emph{latex=False, caption='', latex\_hdr=True, fontsize=8, font='HelveticaNeueLTStd-LtCn', fname=None, header\_format='\%s', row\_order=None, by\_category=True, use\_cats={[}'p', 'm', 'i', 'r'{]}, urls=True, annots=False}}{}
Returns basic numbers about the network resources, e.g. edge and
node counts.
\begin{description}
\item[{latex}] \leavevmode
Return table in a LaTeX document. This can be compiled by
PDFLaTeX:
latex stats.tex

\end{description}

\end{fulllineitems}

\index{basic\_stats\_intergroup() (pypath.legacy.main.PyPath method)@\spxentry{basic\_stats\_intergroup()}\spxextra{pypath.legacy.main.PyPath method}}

\begin{fulllineitems}
\phantomsection\label{\detokenize{reference:pypath.legacy.main.PyPath.basic_stats_intergroup}}\pysiglinewithargsret{\sphinxbfcode{\sphinxupquote{basic\_stats\_intergroup}}}{\emph{groupA}, \emph{groupB}, \emph{header=None}}{}
\end{fulllineitems}

\index{cancer\_drivers\_list() (pypath.legacy.main.PyPath method)@\spxentry{cancer\_drivers\_list()}\spxextra{pypath.legacy.main.PyPath method}}

\begin{fulllineitems}
\phantomsection\label{\detokenize{reference:pypath.legacy.main.PyPath.cancer_drivers_list}}\pysiglinewithargsret{\sphinxbfcode{\sphinxupquote{cancer\_drivers\_list}}}{\emph{intogen\_file=None}}{}
Loads the list of cancer drivers. Contains information from
COSMIC (needs user log in credentials) and IntOGen (if provided)
and adds the attribute to the undirected network nodes.
\begin{quote}\begin{description}
\item[{Parameters}] \leavevmode
\sphinxstyleliteralstrong{\sphinxupquote{intogen\_file}} (\sphinxstyleliteralemphasis{\sphinxupquote{str}}) \textendash{} Optional, \sphinxcode{\sphinxupquote{None}} by default. Path to the data file. Can
also be {[}function{]} that provides the data. In general,
anything accepted by
\sphinxcode{\sphinxupquote{pypath.input\_formats.NetworkInput.input}}.

\end{description}\end{quote}

\end{fulllineitems}

\index{cancer\_gene\_census\_list() (pypath.legacy.main.PyPath method)@\spxentry{cancer\_gene\_census\_list()}\spxextra{pypath.legacy.main.PyPath method}}

\begin{fulllineitems}
\phantomsection\label{\detokenize{reference:pypath.legacy.main.PyPath.cancer_gene_census_list}}\pysiglinewithargsret{\sphinxbfcode{\sphinxupquote{cancer\_gene\_census\_list}}}{}{}
Loads the list of cancer driver proteins from the COSMIC Cancer
Gene Census.

\end{fulllineitems}

\index{clean\_graph() (pypath.legacy.main.PyPath method)@\spxentry{clean\_graph()}\spxextra{pypath.legacy.main.PyPath method}}

\begin{fulllineitems}
\phantomsection\label{\detokenize{reference:pypath.legacy.main.PyPath.clean_graph}}\pysiglinewithargsret{\sphinxbfcode{\sphinxupquote{clean\_graph}}}{\emph{organisms\_allowed=None}}{}
Removes multiple edges, unknown molecules and those from wrong
taxon. Multiple edges will be combined by
\sphinxcode{\sphinxupquote{pypath.main.PyPath.combine\_attr()}} method.
Loops will be deleted unless the attribute
\sphinxcode{\sphinxupquote{pypath.main.PyPath.loops}} is set to \sphinxcode{\sphinxupquote{True}}.
\begin{quote}\begin{description}
\item[{Parameters}] \leavevmode
\sphinxstyleliteralstrong{\sphinxupquote{organisms\_allowed}} (\sphinxstyleliteralemphasis{\sphinxupquote{set}}) \textendash{} NCBI Taxonomy identifiers {[}int{]} of the organisms allowed
in the network.

\end{description}\end{quote}

\end{fulllineitems}

\index{collapse\_by\_name() (pypath.legacy.main.PyPath method)@\spxentry{collapse\_by\_name()}\spxextra{pypath.legacy.main.PyPath method}}

\begin{fulllineitems}
\phantomsection\label{\detokenize{reference:pypath.legacy.main.PyPath.collapse_by_name}}\pysiglinewithargsret{\sphinxbfcode{\sphinxupquote{collapse\_by\_name}}}{\emph{graph=None}}{}
Collapses nodes with the same name by copying and merging
all edges and attributes. Operates directly on the provided
network object.
\begin{quote}\begin{description}
\item[{Parameters}] \leavevmode
\sphinxstyleliteralstrong{\sphinxupquote{graph}} (\sphinxstyleliteralemphasis{\sphinxupquote{igraph.Graph}}) \textendash{} Optional, \sphinxcode{\sphinxupquote{None}} by default. The network for which the
nodes are to be collapsed. If none is provided, takes
\sphinxcode{\sphinxupquote{pypath.main.PyPath.graph}} (undirected network) by
default.

\end{description}\end{quote}

\end{fulllineitems}

\index{collect() (pypath.legacy.main.PyPath method)@\spxentry{collect()}\spxextra{pypath.legacy.main.PyPath method}}

\begin{fulllineitems}
\phantomsection\label{\detokenize{reference:pypath.legacy.main.PyPath.collect}}\pysiglinewithargsret{\sphinxbfcode{\sphinxupquote{collect}}}{\emph{method}, \emph{**kwargs}}{}
Collects various entities over the network according to \sphinxcode{\sphinxupquote{method}}.

\end{fulllineitems}

\index{combine\_attr() (pypath.legacy.main.PyPath method)@\spxentry{combine\_attr()}\spxextra{pypath.legacy.main.PyPath method}}

\begin{fulllineitems}
\phantomsection\label{\detokenize{reference:pypath.legacy.main.PyPath.combine_attr}}\pysiglinewithargsret{\sphinxbfcode{\sphinxupquote{combine\_attr}}}{\emph{lst}, \emph{num\_method=\textless{}built-in function max\textgreater{}}}{}
Combines multiple attributes into one. This method attempts
to find out which is the best way to combine attributes.
\begin{itemize}
\item {} 
If there is only one value or one of them is None, then
returns the one available.

\item {} 
Lists: concatenates unique values of lists.

\item {} 
Numbers: returns the greater by default or calls
\sphinxstyleemphasis{num\_method} if given.

\item {} 
Sets: returns the union.

\item {} 
Dictionaries: calls \sphinxcode{\sphinxupquote{pypath.common.merge\_dicts()}}.

\item {} 
Direction: calls their special
\sphinxcode{\sphinxupquote{pypath.main.Direction.merge()}} method.

\end{itemize}

Works on more than 2 attributes recursively.
\begin{quote}\begin{description}
\item[{Parameters}] \leavevmode\begin{itemize}
\item {} 
\sphinxstyleliteralstrong{\sphinxupquote{lst}} (\sphinxstyleliteralemphasis{\sphinxupquote{list}}) \textendash{} List of one or two attribute values.

\item {} 
\sphinxstyleliteralstrong{\sphinxupquote{num\_method}} (\sphinxstyleliteralemphasis{\sphinxupquote{function}}) \textendash{} Optional, \sphinxcode{\sphinxupquote{max}} by default. Method to merge numeric
attributes.

\end{itemize}

\end{description}\end{quote}

\end{fulllineitems}

\index{communities() (pypath.legacy.main.PyPath method)@\spxentry{communities()}\spxextra{pypath.legacy.main.PyPath method}}

\begin{fulllineitems}
\phantomsection\label{\detokenize{reference:pypath.legacy.main.PyPath.communities}}\pysiglinewithargsret{\sphinxbfcode{\sphinxupquote{communities}}}{\emph{method}, \emph{**kwargs}}{}
\end{fulllineitems}

\index{complex\_comembership\_network() (pypath.legacy.main.PyPath method)@\spxentry{complex\_comembership\_network()}\spxextra{pypath.legacy.main.PyPath method}}

\begin{fulllineitems}
\phantomsection\label{\detokenize{reference:pypath.legacy.main.PyPath.complex_comembership_network}}\pysiglinewithargsret{\sphinxbfcode{\sphinxupquote{complex\_comembership\_network}}}{\emph{graph=None}, \emph{resources=None}}{}
\end{fulllineitems}

\index{complexes() (pypath.legacy.main.PyPath method)@\spxentry{complexes()}\spxextra{pypath.legacy.main.PyPath method}}

\begin{fulllineitems}
\phantomsection\label{\detokenize{reference:pypath.legacy.main.PyPath.complexes}}\pysiglinewithargsret{\sphinxbfcode{\sphinxupquote{complexes}}}{\emph{methods={[}'3dcomplexes', 'havugimana', 'corum', 'complexportal', 'compleat'{]}}}{}
\end{fulllineitems}

\index{complexes\_in\_network() (pypath.legacy.main.PyPath method)@\spxentry{complexes\_in\_network()}\spxextra{pypath.legacy.main.PyPath method}}

\begin{fulllineitems}
\phantomsection\label{\detokenize{reference:pypath.legacy.main.PyPath.complexes_in_network}}\pysiglinewithargsret{\sphinxbfcode{\sphinxupquote{complexes\_in\_network}}}{\emph{csource='corum'}, \emph{graph=None}}{}
\end{fulllineitems}

\index{compounds\_from\_chembl() (pypath.legacy.main.PyPath method)@\spxentry{compounds\_from\_chembl()}\spxextra{pypath.legacy.main.PyPath method}}

\begin{fulllineitems}
\phantomsection\label{\detokenize{reference:pypath.legacy.main.PyPath.compounds_from_chembl}}\pysiglinewithargsret{\sphinxbfcode{\sphinxupquote{compounds\_from\_chembl}}}{\emph{chembl\_mysql=None, nodes=None, crit=None, andor='or', assay\_types={[}'B', 'F'{]}, relationship\_types={[}'D', 'H'{]}, multi\_query=False, **kwargs}}{}
Loads compound data from ChEMBL to the network.
\begin{quote}\begin{description}
\item[{Parameters}] \leavevmode\begin{itemize}
\item {} 
\sphinxstyleliteralstrong{\sphinxupquote{chebl\_mysql}} (\sphinxstyleliteralemphasis{\sphinxupquote{tuple}}) \textendash{} Optional, \sphinxcode{\sphinxupquote{None}} by default. Contains the MySQL parameters
used by the \sphinxcode{\sphinxupquote{pypath.mapping}} module to loadthe
ChEMBL ID conversion tables. If none is passed, takes the
current instance \sphinxcode{\sphinxupquote{pypath.main.PyPath.chembl\_mysql}}
attribute.

\item {} 
\sphinxstyleliteralstrong{\sphinxupquote{nodes}} (\sphinxstyleliteralemphasis{\sphinxupquote{list}}) \textendash{} Optional, \sphinxcode{\sphinxupquote{None}} by default. List of node indices for
which the information is to be loaded. If none is provided
calls the method \sphinxcode{\sphinxupquote{pypath.main.PyPath.get\_sub()}} with
the provided \sphinxstyleemphasis{crit} parameter.

\item {} 
\sphinxstyleliteralstrong{\sphinxupquote{crit}} (\sphinxstyleliteralemphasis{\sphinxupquote{dict}}) \textendash{} Optional, \sphinxcode{\sphinxupquote{None}} by default. Defines the critical
attributes to generate a subnetwork to extract the nodes in
case \sphinxstyleemphasis{nodes} is not provided. Keys are \sphinxcode{\sphinxupquote{'edge'}} and
\sphinxcode{\sphinxupquote{'node'}} and values are {[}dict{]} containing the critical
attribute names {[}str{]} and values are {[}set{]} containing those
attributes of the nodes/edges that are to be kept in the
subnetwork. If none is provided, takes the whole network.

\item {} 
\sphinxstyleliteralstrong{\sphinxupquote{andor}} (\sphinxstyleliteralemphasis{\sphinxupquote{str}}) \textendash{} Optional, \sphinxcode{\sphinxupquote{'or'}} by default. Determines the search mode
for the subnetwork generation (if \sphinxcode{\sphinxupquote{nodes=None}}). See
\sphinxcode{\sphinxupquote{pypath.main.PyPath.search\_attr\_or()}} and
\sphinxcode{\sphinxupquote{pypath.main.PyPath.search\_attr\_and()}} for more
details.

\item {} 
\sphinxstyleliteralstrong{\sphinxupquote{assay\_types}} (\sphinxstyleliteralemphasis{\sphinxupquote{list}}) \textendash{} Optional, \sphinxcode{\sphinxupquote{{[}'B', 'F'{]}}} by default. Types of assay to query
Options are: \sphinxcode{\sphinxupquote{'A'}} (ADME), \sphinxcode{\sphinxupquote{'B'}} (Binding),{}`{}`’F’{}`{}`
(Functional), \sphinxcode{\sphinxupquote{'P'}} (Physicochemical), \sphinxcode{\sphinxupquote{'T'}} (Toxicity)
and/or \sphinxcode{\sphinxupquote{'U'}} (Unassigned).

\item {} 
\sphinxstyleliteralstrong{\sphinxupquote{relationship\_types}} (\sphinxstyleliteralemphasis{\sphinxupquote{list}}) \textendash{} Optional, \sphinxcode{\sphinxupquote{{[}'D', 'H'{]}}} by default. Assay relationship
types to query. Possible values are: \sphinxcode{\sphinxupquote{'D'}} (Direct protein
target assigned), \sphinxcode{\sphinxupquote{'H'}} (Homologous protein target
assigned), \sphinxcode{\sphinxupquote{'M'}} (Molecular target other than protein
assigned), \sphinxcode{\sphinxupquote{'N'}} (Non-molecular target assigned), \sphinxcode{\sphinxupquote{'S'}}
(Subcellular target assigned) and/or \sphinxcode{\sphinxupquote{'U'}} (Default value,
target has yet to be curated).

\item {} 
\sphinxstyleliteralstrong{\sphinxupquote{multi\_query}} (\sphinxstyleliteralemphasis{\sphinxupquote{bool}}) \textendash{} Optional, \sphinxcode{\sphinxupquote{False}} by default. Not used.

\item {} 
\sphinxstyleliteralstrong{\sphinxupquote{**kwargs}} \textendash{} Additional keyword arguments for
\sphinxcode{\sphinxupquote{pypath.chembl.Chembl.compounds\_targets()}}.

\end{itemize}

\end{description}\end{quote}

\end{fulllineitems}

\index{consistency() (pypath.legacy.main.PyPath method)@\spxentry{consistency()}\spxextra{pypath.legacy.main.PyPath method}}

\begin{fulllineitems}
\phantomsection\label{\detokenize{reference:pypath.legacy.main.PyPath.consistency}}\pysiglinewithargsret{\sphinxbfcode{\sphinxupquote{consistency}}}{}{}
\end{fulllineitems}

\index{copy() (pypath.legacy.main.PyPath method)@\spxentry{copy()}\spxextra{pypath.legacy.main.PyPath method}}

\begin{fulllineitems}
\phantomsection\label{\detokenize{reference:pypath.legacy.main.PyPath.copy}}\pysiglinewithargsret{\sphinxbfcode{\sphinxupquote{copy}}}{\emph{other}}{}
Copies another \sphinxcode{\sphinxupquote{pypath.main.PyPath}} instance into the
current one.
\begin{quote}\begin{description}
\item[{Parameters}] \leavevmode
\sphinxstyleliteralstrong{\sphinxupquote{other}} (\sphinxstyleliteralemphasis{\sphinxupquote{pypath.main.PyPath}}) \textendash{} The instance to be copied from.

\end{description}\end{quote}

\end{fulllineitems}

\index{copy\_edges() (pypath.legacy.main.PyPath method)@\spxentry{copy\_edges()}\spxextra{pypath.legacy.main.PyPath method}}

\begin{fulllineitems}
\phantomsection\label{\detokenize{reference:pypath.legacy.main.PyPath.copy_edges}}\pysiglinewithargsret{\sphinxbfcode{\sphinxupquote{copy\_edges}}}{\emph{sources}, \emph{target}, \emph{move=False}, \emph{graph=None}}{}
Copies edges from \sphinxstyleemphasis{sources} node(s) to another one (\sphinxstyleemphasis{target}),
keeping attributes and directions.
\begin{quote}\begin{description}
\item[{Parameters}] \leavevmode\begin{itemize}
\item {} 
\sphinxstyleliteralstrong{\sphinxupquote{sources}} (\sphinxstyleliteralemphasis{\sphinxupquote{list}}) \textendash{} Contains the vertex index(es) {[}int{]} of the node(s) to be
copied or moved.

\item {} 
\sphinxstyleliteralstrong{\sphinxupquote{target}} (\sphinxstyleliteralemphasis{\sphinxupquote{int}}) \textendash{} Vertex index where edges and attributes are to be copied to.

\item {} 
\sphinxstyleliteralstrong{\sphinxupquote{move}} (\sphinxstyleliteralemphasis{\sphinxupquote{bool}}) \textendash{} Optional, \sphinxcode{\sphinxupquote{False}} by default. Whether to perform copy or
move (remove or keep the source edges).

\item {} 
\sphinxstyleliteralstrong{\sphinxupquote{graph}} (\sphinxstyleliteralemphasis{\sphinxupquote{igraph.Graph}}) \textendash{} Optional, \sphinxcode{\sphinxupquote{None}} by default. The network graph object from
which the nodes are to be merged. If none is passed, takes
the undirected network graph.

\end{itemize}

\end{description}\end{quote}

\end{fulllineitems}

\index{count\_sol() (pypath.legacy.main.PyPath method)@\spxentry{count\_sol()}\spxextra{pypath.legacy.main.PyPath method}}

\begin{fulllineitems}
\phantomsection\label{\detokenize{reference:pypath.legacy.main.PyPath.count_sol}}\pysiglinewithargsret{\sphinxbfcode{\sphinxupquote{count\_sol}}}{}{}
Counts the number of nodes with zero degree.
\begin{quote}\begin{description}
\item[{Returns}] \leavevmode
(\sphinxstyleemphasis{int}) \textendash{} The number of nodes with zero degree.

\end{description}\end{quote}

\end{fulllineitems}

\index{counts() (pypath.legacy.main.PyPath method)@\spxentry{counts()}\spxextra{pypath.legacy.main.PyPath method}}

\begin{fulllineitems}
\phantomsection\label{\detokenize{reference:pypath.legacy.main.PyPath.counts}}\pysiglinewithargsret{\sphinxbfcode{\sphinxupquote{counts}}}{\emph{collection\_method}, \emph{add\_total=True}, \emph{add\_percent=True}, \emph{add\_cat\_total=True}, \emph{**kwargs}}{}
Collects various entities over the network according to \sphinxcode{\sphinxupquote{method}}
and counts them in total and by resources.

\end{fulllineitems}

\index{coverage() (pypath.legacy.main.PyPath method)@\spxentry{coverage()}\spxextra{pypath.legacy.main.PyPath method}}

\begin{fulllineitems}
\phantomsection\label{\detokenize{reference:pypath.legacy.main.PyPath.coverage}}\pysiglinewithargsret{\sphinxbfcode{\sphinxupquote{coverage}}}{\emph{lst}}{}
Computes the coverage (range {[}0, 1{]}) of a list of nodes against
the current (undirected) network.
\begin{quote}\begin{description}
\item[{Parameters}] \leavevmode
\sphinxstyleliteralstrong{\sphinxupquote{lst}} (\sphinxstyleliteralemphasis{\sphinxupquote{set}}) \textendash{} Can also be {[}list{]} (will be converted to {[}set{]}) or {[}str{]}. In
the latter case it will retrieve the list with that name (if
such list exists in \sphinxcode{\sphinxupquote{pypath.main.PyPath.lists}}).

\end{description}\end{quote}

\end{fulllineitems}

\index{curation\_effort() (pypath.legacy.main.PyPath method)@\spxentry{curation\_effort()}\spxextra{pypath.legacy.main.PyPath method}}

\begin{fulllineitems}
\phantomsection\label{\detokenize{reference:pypath.legacy.main.PyPath.curation_effort}}\pysiglinewithargsret{\sphinxbfcode{\sphinxupquote{curation\_effort}}}{\emph{resources=None}, \emph{**kwargs}}{}
Returns a \sphinxstyleemphasis{set} of reference-interactions pairs.

\end{fulllineitems}

\index{curation\_effort\_by\_resource() (pypath.legacy.main.PyPath method)@\spxentry{curation\_effort\_by\_resource()}\spxextra{pypath.legacy.main.PyPath method}}

\begin{fulllineitems}
\phantomsection\label{\detokenize{reference:pypath.legacy.main.PyPath.curation_effort_by_resource}}\pysiglinewithargsret{\sphinxbfcode{\sphinxupquote{curation\_effort\_by\_resource}}}{\emph{resources=None}, \emph{**kwargs}}{}
A \sphinxstyleemphasis{dict} with resources as keys and {\color{red}\bfseries{}*}set*s of curation items
(interaction-reference pairs) as values.

\end{fulllineitems}

\index{curation\_stats() (pypath.legacy.main.PyPath method)@\spxentry{curation\_stats()}\spxextra{pypath.legacy.main.PyPath method}}

\begin{fulllineitems}
\phantomsection\label{\detokenize{reference:pypath.legacy.main.PyPath.curation_stats}}\pysiglinewithargsret{\sphinxbfcode{\sphinxupquote{curation\_stats}}}{\emph{by\_category=True}}{}
\end{fulllineitems}

\index{curation\_tab() (pypath.legacy.main.PyPath method)@\spxentry{curation\_tab()}\spxextra{pypath.legacy.main.PyPath method}}

\begin{fulllineitems}
\phantomsection\label{\detokenize{reference:pypath.legacy.main.PyPath.curation_tab}}\pysiglinewithargsret{\sphinxbfcode{\sphinxupquote{curation\_tab}}}{\emph{fname='curation\_stats.tex', by\_category=True, use\_cats={[}'p', 'm', 'i', 'r'{]}, header\_size='normalsize', **kwargs}}{}
\end{fulllineitems}

\index{curators\_work() (pypath.legacy.main.PyPath method)@\spxentry{curators\_work()}\spxextra{pypath.legacy.main.PyPath method}}

\begin{fulllineitems}
\phantomsection\label{\detokenize{reference:pypath.legacy.main.PyPath.curators_work}}\pysiglinewithargsret{\sphinxbfcode{\sphinxupquote{curators\_work}}}{}{}
Computes and prints an estimation of how many years of curation
took to achieve the amount of information on the network.

\end{fulllineitems}

\index{databases\_similarity() (pypath.legacy.main.PyPath method)@\spxentry{databases\_similarity()}\spxextra{pypath.legacy.main.PyPath method}}

\begin{fulllineitems}
\phantomsection\label{\detokenize{reference:pypath.legacy.main.PyPath.databases_similarity}}\pysiglinewithargsret{\sphinxbfcode{\sphinxupquote{databases\_similarity}}}{\emph{index='simpson'}}{}
Computes the similarity across databases according to a given
index metric. Computes the similarity across the loaded
resources (listed in \sphinxcode{\sphinxupquote{pypath.main.PyPath.sources}} in
terms of nodes and edges separately.
\begin{quote}\begin{description}
\item[{Parameters}] \leavevmode
\sphinxstyleliteralstrong{\sphinxupquote{index}} (\sphinxstyleliteralemphasis{\sphinxupquote{str}}) \textendash{} Optional, \sphinxcode{\sphinxupquote{'simpson'}} by default. The type of index metric
to use to compute the similarity. Options are \sphinxcode{\sphinxupquote{'simpson'}},
\sphinxcode{\sphinxupquote{'sorensen'}} and \sphinxcode{\sphinxupquote{'jaccard'}}.

\item[{Returns}] \leavevmode
(\sphinxstyleemphasis{dict}) \textendash{} Nested dictionaries (three levels). First-level
keys are \sphinxcode{\sphinxupquote{'nodes'}} and \sphinxcode{\sphinxupquote{'edges'}}, then second and third
levels correspond to sources names which map to the
similarity index between those sources {[}float{]}.

\end{description}\end{quote}

\end{fulllineitems}

\index{degree\_dist() (pypath.legacy.main.PyPath method)@\spxentry{degree\_dist()}\spxextra{pypath.legacy.main.PyPath method}}

\begin{fulllineitems}
\phantomsection\label{\detokenize{reference:pypath.legacy.main.PyPath.degree_dist}}\pysiglinewithargsret{\sphinxbfcode{\sphinxupquote{degree\_dist}}}{\emph{prefix}, \emph{g=None}, \emph{group=None}}{}
Computes the degree distribution over all nodes of the network.
If \sphinxstyleemphasis{group} is provided, also across nodes of that group(s).
\begin{quote}\begin{description}
\item[{Parameters}] \leavevmode\begin{itemize}
\item {} 
\sphinxstyleliteralstrong{\sphinxupquote{prefix}} (\sphinxstyleliteralemphasis{\sphinxupquote{str}}) \textendash{} Prefix for the file name(s).

\item {} 
\sphinxstyleliteralstrong{\sphinxupquote{g}} (\sphinxstyleliteralemphasis{\sphinxupquote{igraph.Graph}}) \textendash{} Optional, \sphinxcode{\sphinxupquote{None}} by default. The network over which to
compute the degree distribution. If none is passed, takes
the undirected network of the current instance.

\item {} 
\sphinxstyleliteralstrong{\sphinxupquote{group}} (\sphinxstyleliteralemphasis{\sphinxupquote{list}}) \textendash{} Optional, \sphinxcode{\sphinxupquote{None}} by default. Additional group(s) name(s)
{[}str{]} of node attributes to subset the network and compute
its degree distribution.

\end{itemize}

\end{description}\end{quote}

\end{fulllineitems}

\index{degree\_dists() (pypath.legacy.main.PyPath method)@\spxentry{degree\_dists()}\spxextra{pypath.legacy.main.PyPath method}}

\begin{fulllineitems}
\phantomsection\label{\detokenize{reference:pypath.legacy.main.PyPath.degree_dists}}\pysiglinewithargsret{\sphinxbfcode{\sphinxupquote{degree\_dists}}}{}{}
Computes the degree distribution for all the different network
sources. This is, for each source, the subnetwork comprising all
interactions coming from it is extracted and the degree
distribution information is computed and saved into a file.
A file is created for each resource under the name
\sphinxcode{\sphinxupquote{'pwnet-\textless{}session\_id\textgreater{}-degdist-\textless{}resource\textgreater{}'{'}}}. Files are stored
in \sphinxcode{\sphinxupquote{pypath.main.PyPath.outdir}} (\sphinxcode{\sphinxupquote{'results'}} by
default).

\end{fulllineitems}

\index{delete\_by\_organism() (pypath.legacy.main.PyPath method)@\spxentry{delete\_by\_organism()}\spxextra{pypath.legacy.main.PyPath method}}

\begin{fulllineitems}
\phantomsection\label{\detokenize{reference:pypath.legacy.main.PyPath.delete_by_organism}}\pysiglinewithargsret{\sphinxbfcode{\sphinxupquote{delete\_by\_organism}}}{\emph{organisms\_allowed=None}}{}
Removes the proteins of all organisms which are not given in
\sphinxstyleemphasis{tax}.
\begin{quote}\begin{description}
\item[{Parameters}] \leavevmode
\sphinxstyleliteralstrong{\sphinxupquote{organisms\_allowed}} (\sphinxstyleliteralemphasis{\sphinxupquote{list}}\sphinxstyleliteralemphasis{\sphinxupquote{,}}\sphinxstyleliteralemphasis{\sphinxupquote{set}}) \textendash{} List of NCBI Taxonomy IDs {[}int{]} of the organism(s) that are
to be kept.

\end{description}\end{quote}

\end{fulllineitems}

\index{delete\_by\_source() (pypath.legacy.main.PyPath method)@\spxentry{delete\_by\_source()}\spxextra{pypath.legacy.main.PyPath method}}

\begin{fulllineitems}
\phantomsection\label{\detokenize{reference:pypath.legacy.main.PyPath.delete_by_source}}\pysiglinewithargsret{\sphinxbfcode{\sphinxupquote{delete\_by\_source}}}{\emph{source}, \emph{vertexAttrsToDel=None}, \emph{edgeAttrsToDel=None}}{}
Deletes nodes and edges from the network according to a provided
source name. Optionally can also remove the given list of
attributes from nodes and/or edges.
\begin{quote}\begin{description}
\item[{Parameters}] \leavevmode\begin{itemize}
\item {} 
\sphinxstyleliteralstrong{\sphinxupquote{source}} (\sphinxstyleliteralemphasis{\sphinxupquote{str}}) \textendash{} Name of the source from which the nodes and edges have to be
removed.

\item {} 
\sphinxstyleliteralstrong{\sphinxupquote{vertexAttrsToDel}} (\sphinxstyleliteralemphasis{\sphinxupquote{list}}) \textendash{} Optional, \sphinxcode{\sphinxupquote{None}} by default. Contains the names {[}str{]} of
the attributes to be removed from the nodes.

\item {} 
\sphinxstyleliteralstrong{\sphinxupquote{edgeAttrsToDel}} (\sphinxstyleliteralemphasis{\sphinxupquote{list}}) \textendash{} Optional, \sphinxcode{\sphinxupquote{None}} by default. Contains the names {[}str{]} of
the attributes to be removed from the edges.

\end{itemize}

\end{description}\end{quote}

\end{fulllineitems}

\index{delete\_unknown() (pypath.legacy.main.PyPath method)@\spxentry{delete\_unknown()}\spxextra{pypath.legacy.main.PyPath method}}

\begin{fulllineitems}
\phantomsection\label{\detokenize{reference:pypath.legacy.main.PyPath.delete_unknown}}\pysiglinewithargsret{\sphinxbfcode{\sphinxupquote{delete\_unknown}}}{\emph{organisms\_allowed=None}, \emph{entity\_type='protein'}, \emph{default\_name\_type=None}}{}
Removes those items which are not in the list of all default
IDs of the organisms. By default, it means to remove all protein
nodes not having a human UniProt ID.
\begin{quote}\begin{description}
\item[{Parameters}] \leavevmode\begin{itemize}
\item {} 
\sphinxstyleliteralstrong{\sphinxupquote{typ}} (\sphinxstyleliteralemphasis{\sphinxupquote{str}}) \textendash{} Optional, \sphinxcode{\sphinxupquote{'protein'}} by default. Determines the molecule
type. These can be \sphinxcode{\sphinxupquote{'protein'}}, \sphinxcode{\sphinxupquote{'drug'}}, \sphinxcode{\sphinxupquote{'lncrna'}},
\sphinxcode{\sphinxupquote{'mirna'}} or any other type defined in
\sphinxcode{\sphinxupquote{pypath.main.PyPath.default\_name\_type}}.

\item {} 
\sphinxstyleliteralstrong{\sphinxupquote{default\_name\_type}} (\sphinxstyleliteralemphasis{\sphinxupquote{str}}) \textendash{} Optional, \sphinxcode{\sphinxupquote{None}} by default. The default name type for the
given molecular species. If none is specified takes it from
\sphinxcode{\sphinxupquote{pypath.main.PyPath.default\_name\_type}} (e.g.: for
\sphinxcode{\sphinxupquote{'protein'}}, default is \sphinxcode{\sphinxupquote{'uniprot'}}).

\item {} 
\sphinxstyleliteralstrong{\sphinxupquote{organisms\_allowed}} (\sphinxstyleliteralemphasis{\sphinxupquote{set}}) \textendash{} NCBI Taxonomy identifiers {[}int{]} of the organisms allowed in
the network.

\end{itemize}

\end{description}\end{quote}

\end{fulllineitems}

\index{delete\_unmapped() (pypath.legacy.main.PyPath method)@\spxentry{delete\_unmapped()}\spxextra{pypath.legacy.main.PyPath method}}

\begin{fulllineitems}
\phantomsection\label{\detokenize{reference:pypath.legacy.main.PyPath.delete_unmapped}}\pysiglinewithargsret{\sphinxbfcode{\sphinxupquote{delete\_unmapped}}}{}{}
Checks the network for any existing unmapped node and removes
it.

\end{fulllineitems}

\index{dgenesymbol() (pypath.legacy.main.PyPath method)@\spxentry{dgenesymbol()}\spxextra{pypath.legacy.main.PyPath method}}

\begin{fulllineitems}
\phantomsection\label{\detokenize{reference:pypath.legacy.main.PyPath.dgenesymbol}}\pysiglinewithargsret{\sphinxbfcode{\sphinxupquote{dgenesymbol}}}{\emph{genesymbol}}{}
Returns \sphinxcode{\sphinxupquote{igraph.Vertex()}} object if the GeneSymbol
can be found in the default directed network,
otherwise \sphinxcode{\sphinxupquote{None}}.
\begin{description}
\item[{@genesymbol}] \leavevmode{[}str{]}
GeneSymbol.

\end{description}

\end{fulllineitems}

\index{dgenesymbols() (pypath.legacy.main.PyPath method)@\spxentry{dgenesymbols()}\spxextra{pypath.legacy.main.PyPath method}}

\begin{fulllineitems}
\phantomsection\label{\detokenize{reference:pypath.legacy.main.PyPath.dgenesymbols}}\pysiglinewithargsret{\sphinxbfcode{\sphinxupquote{dgenesymbols}}}{\emph{genesymbols}}{}
\end{fulllineitems}

\index{dgs() (pypath.legacy.main.PyPath method)@\spxentry{dgs()}\spxextra{pypath.legacy.main.PyPath method}}

\begin{fulllineitems}
\phantomsection\label{\detokenize{reference:pypath.legacy.main.PyPath.dgs}}\pysiglinewithargsret{\sphinxbfcode{\sphinxupquote{dgs}}}{\emph{genesymbol}}{}
Returns \sphinxcode{\sphinxupquote{igraph.Vertex()}} object if the GeneSymbol
can be found in the default directed network,
otherwise \sphinxcode{\sphinxupquote{None}}.
\begin{description}
\item[{@genesymbol}] \leavevmode{[}str{]}
GeneSymbol.

\end{description}

\end{fulllineitems}

\index{dgss() (pypath.legacy.main.PyPath method)@\spxentry{dgss()}\spxextra{pypath.legacy.main.PyPath method}}

\begin{fulllineitems}
\phantomsection\label{\detokenize{reference:pypath.legacy.main.PyPath.dgss}}\pysiglinewithargsret{\sphinxbfcode{\sphinxupquote{dgss}}}{\emph{genesymbols}}{}
\end{fulllineitems}

\index{dneighbors() (pypath.legacy.main.PyPath method)@\spxentry{dneighbors()}\spxextra{pypath.legacy.main.PyPath method}}

\begin{fulllineitems}
\phantomsection\label{\detokenize{reference:pypath.legacy.main.PyPath.dneighbors}}\pysiglinewithargsret{\sphinxbfcode{\sphinxupquote{dneighbors}}}{\emph{identifier}, \emph{mode='ALL'}}{}
\end{fulllineitems}

\index{dp() (pypath.legacy.main.PyPath method)@\spxentry{dp()}\spxextra{pypath.legacy.main.PyPath method}}

\begin{fulllineitems}
\phantomsection\label{\detokenize{reference:pypath.legacy.main.PyPath.dp}}\pysiglinewithargsret{\sphinxbfcode{\sphinxupquote{dp}}}{\emph{identifier}}{}
Same as \sphinxcode{\sphinxupquote{PyPath.get\_node}}, just for the directed graph.
Returns \sphinxcode{\sphinxupquote{igraph.Vertex()}} object if the identifier
is a valid vertex index in the default directed graph,
or a UniProt ID or GeneSymbol which can be found in the
default directed network, otherwise \sphinxcode{\sphinxupquote{None}}.
\begin{description}
\item[{@identifier}] \leavevmode{[}int, str{]}
Vertex index (int) or GeneSymbol (str) or UniProt ID (str) or
\sphinxcode{\sphinxupquote{igraph.Vertex}} object.

\end{description}

\end{fulllineitems}

\index{dproteins() (pypath.legacy.main.PyPath method)@\spxentry{dproteins()}\spxextra{pypath.legacy.main.PyPath method}}

\begin{fulllineitems}
\phantomsection\label{\detokenize{reference:pypath.legacy.main.PyPath.dproteins}}\pysiglinewithargsret{\sphinxbfcode{\sphinxupquote{dproteins}}}{\emph{identifiers}}{}
\end{fulllineitems}

\index{dps() (pypath.legacy.main.PyPath method)@\spxentry{dps()}\spxextra{pypath.legacy.main.PyPath method}}

\begin{fulllineitems}
\phantomsection\label{\detokenize{reference:pypath.legacy.main.PyPath.dps}}\pysiglinewithargsret{\sphinxbfcode{\sphinxupquote{dps}}}{\emph{identifiers}}{}
\end{fulllineitems}

\index{duniprot() (pypath.legacy.main.PyPath method)@\spxentry{duniprot()}\spxextra{pypath.legacy.main.PyPath method}}

\begin{fulllineitems}
\phantomsection\label{\detokenize{reference:pypath.legacy.main.PyPath.duniprot}}\pysiglinewithargsret{\sphinxbfcode{\sphinxupquote{duniprot}}}{\emph{uniprot}}{}
Same as \sphinxcode{\sphinxupquote{PyPath.uniprot(), just for directed graph.
Returns {}`{}`igraph.Vertex()}} object if the UniProt
can be found in the default directed network,
otherwise \sphinxcode{\sphinxupquote{None}}.
\begin{description}
\item[{@uniprot}] \leavevmode{[}str{]}
UniProt ID.

\end{description}

\end{fulllineitems}

\index{duniprots() (pypath.legacy.main.PyPath method)@\spxentry{duniprots()}\spxextra{pypath.legacy.main.PyPath method}}

\begin{fulllineitems}
\phantomsection\label{\detokenize{reference:pypath.legacy.main.PyPath.duniprots}}\pysiglinewithargsret{\sphinxbfcode{\sphinxupquote{duniprots}}}{\emph{uniprots}}{}
Returns list of \sphinxcode{\sphinxupquote{igraph.Vertex()}} object
for a list of UniProt IDs omitting those
could not be found in the default
directed graph.

\end{fulllineitems}

\index{dup() (pypath.legacy.main.PyPath method)@\spxentry{dup()}\spxextra{pypath.legacy.main.PyPath method}}

\begin{fulllineitems}
\phantomsection\label{\detokenize{reference:pypath.legacy.main.PyPath.dup}}\pysiglinewithargsret{\sphinxbfcode{\sphinxupquote{dup}}}{\emph{uniprot}}{}
Same as \sphinxcode{\sphinxupquote{PyPath.uniprot(), just for directed graph.
Returns {}`{}`igraph.Vertex()}} object if the UniProt
can be found in the default directed network,
otherwise \sphinxcode{\sphinxupquote{None}}.
\begin{description}
\item[{@uniprot}] \leavevmode{[}str{]}
UniProt ID.

\end{description}

\end{fulllineitems}

\index{dups() (pypath.legacy.main.PyPath method)@\spxentry{dups()}\spxextra{pypath.legacy.main.PyPath method}}

\begin{fulllineitems}
\phantomsection\label{\detokenize{reference:pypath.legacy.main.PyPath.dups}}\pysiglinewithargsret{\sphinxbfcode{\sphinxupquote{dups}}}{\emph{uniprots}}{}
Returns list of \sphinxcode{\sphinxupquote{igraph.Vertex()}} object
for a list of UniProt IDs omitting those
could not be found in the default
directed graph.

\end{fulllineitems}

\index{dv() (pypath.legacy.main.PyPath method)@\spxentry{dv()}\spxextra{pypath.legacy.main.PyPath method}}

\begin{fulllineitems}
\phantomsection\label{\detokenize{reference:pypath.legacy.main.PyPath.dv}}\pysiglinewithargsret{\sphinxbfcode{\sphinxupquote{dv}}}{\emph{identifier}}{}
Same as \sphinxcode{\sphinxupquote{PyPath.get\_node}}, just for the directed graph.
Returns \sphinxcode{\sphinxupquote{igraph.Vertex()}} object if the identifier
is a valid vertex index in the default directed graph,
or a UniProt ID or GeneSymbol which can be found in the
default directed network, otherwise \sphinxcode{\sphinxupquote{None}}.
\begin{description}
\item[{@identifier}] \leavevmode{[}int, str{]}
Vertex index (int) or GeneSymbol (str) or UniProt ID (str) or
\sphinxcode{\sphinxupquote{igraph.Vertex}} object.

\end{description}

\end{fulllineitems}

\index{dvs() (pypath.legacy.main.PyPath method)@\spxentry{dvs()}\spxextra{pypath.legacy.main.PyPath method}}

\begin{fulllineitems}
\phantomsection\label{\detokenize{reference:pypath.legacy.main.PyPath.dvs}}\pysiglinewithargsret{\sphinxbfcode{\sphinxupquote{dvs}}}{\emph{identifiers}}{}
\end{fulllineitems}

\index{edge\_exists() (pypath.legacy.main.PyPath method)@\spxentry{edge\_exists()}\spxextra{pypath.legacy.main.PyPath method}}

\begin{fulllineitems}
\phantomsection\label{\detokenize{reference:pypath.legacy.main.PyPath.edge_exists}}\pysiglinewithargsret{\sphinxbfcode{\sphinxupquote{edge\_exists}}}{\emph{id\_a}, \emph{id\_b}}{}
Returns a tuple of vertex indices if edge doesn’t exist,
otherwise, the edge ID. Not sensitive to direction.
\begin{quote}\begin{description}
\item[{Parameters}] \leavevmode\begin{itemize}
\item {} 
\sphinxstyleliteralstrong{\sphinxupquote{id\_a}} (\sphinxstyleliteralemphasis{\sphinxupquote{str}}) \textendash{} Name of the source node.

\item {} 
\sphinxstyleliteralstrong{\sphinxupquote{id\_b}} (\sphinxstyleliteralemphasis{\sphinxupquote{str}}) \textendash{} Name of the target node.

\end{itemize}

\item[{Returns}] \leavevmode
(\sphinxstyleemphasis{int}) \textendash{} The edge index, if exists such edge. Otherwise,
{[}tuple{]} of {[}int{]} corresponding to the node IDs.

\end{description}\end{quote}

\end{fulllineitems}

\index{edge\_loc() (pypath.legacy.main.PyPath method)@\spxentry{edge\_loc()}\spxextra{pypath.legacy.main.PyPath method}}

\begin{fulllineitems}
\phantomsection\label{\detokenize{reference:pypath.legacy.main.PyPath.edge_loc}}\pysiglinewithargsret{\sphinxbfcode{\sphinxupquote{edge\_loc}}}{\emph{graph=None}, \emph{topn=2}}{}
\end{fulllineitems}

\index{edge\_names() (pypath.legacy.main.PyPath method)@\spxentry{edge\_names()}\spxextra{pypath.legacy.main.PyPath method}}

\begin{fulllineitems}
\phantomsection\label{\detokenize{reference:pypath.legacy.main.PyPath.edge_names}}\pysiglinewithargsret{\sphinxbfcode{\sphinxupquote{edge\_names}}}{\emph{e}}{}
Returns the node names of a given edge.
\begin{quote}\begin{description}
\item[{Parameters}] \leavevmode
\sphinxstyleliteralstrong{\sphinxupquote{e}} (\sphinxstyleliteralemphasis{\sphinxupquote{int}}) \textendash{} The edge index.

\item[{Returns}] \leavevmode
(\sphinxstyleemphasis{tuple}) \textendash{} Contains the source and target node names of
the edge {[}str{]}.

\end{description}\end{quote}

\end{fulllineitems}

\index{edges\_3d() (pypath.legacy.main.PyPath method)@\spxentry{edges\_3d()}\spxextra{pypath.legacy.main.PyPath method}}

\begin{fulllineitems}
\phantomsection\label{\detokenize{reference:pypath.legacy.main.PyPath.edges_3d}}\pysiglinewithargsret{\sphinxbfcode{\sphinxupquote{edges\_3d}}}{\emph{methods={[}'dataio.get\_instruct', 'dataio.get\_i3d'{]}}}{}
\end{fulllineitems}

\index{edges\_between() (pypath.legacy.main.PyPath method)@\spxentry{edges\_between()}\spxextra{pypath.legacy.main.PyPath method}}

\begin{fulllineitems}
\phantomsection\label{\detokenize{reference:pypath.legacy.main.PyPath.edges_between}}\pysiglinewithargsret{\sphinxbfcode{\sphinxupquote{edges\_between}}}{\emph{group1}, \emph{group2}, \emph{directed=True}, \emph{strict=False}}{}
Selects edges between two groups of vertex IDs.
Returns set of edge IDs.
\begin{quote}\begin{description}
\item[{Parameters}] \leavevmode\begin{itemize}
\item {} 
\sphinxstyleliteralstrong{\sphinxupquote{group1}}\sphinxstyleliteralstrong{\sphinxupquote{,}}\sphinxstyleliteralstrong{\sphinxupquote{group2}} (\sphinxstyleliteralemphasis{\sphinxupquote{set}}) \textendash{} List, set or tuple of vertex IDs.

\item {} 
\sphinxstyleliteralstrong{\sphinxupquote{directed}} (\sphinxstyleliteralemphasis{\sphinxupquote{bool}}) \textendash{} Only edges with direction \sphinxtitleref{group1 -\textgreater{} group2} selected.

\item {} 
\sphinxstyleliteralstrong{\sphinxupquote{strict}} (\sphinxstyleliteralemphasis{\sphinxupquote{bool}}) \textendash{} Edges with no direction information still selected even if
\sphinxcode{\sphinxupquote{directed}} is \sphinxtitleref{False}.

\end{itemize}

\end{description}\end{quote}

\end{fulllineitems}

\index{edges\_expression() (pypath.legacy.main.PyPath method)@\spxentry{edges\_expression()}\spxextra{pypath.legacy.main.PyPath method}}

\begin{fulllineitems}
\phantomsection\label{\detokenize{reference:pypath.legacy.main.PyPath.edges_expression}}\pysiglinewithargsret{\sphinxbfcode{\sphinxupquote{edges\_expression}}}{\emph{func=\textless{}function PyPath.\textless{}lambda\textgreater{}\textgreater{}}}{}
Executes function \sphinxtitleref{func} for each pairs of connected proteins in the
network, for every expression dataset. By default, \sphinxtitleref{func} simply
gives the product the (normalized) expression values.
\begin{description}
\item[{func}] \leavevmode{[}callable{]}
Function to handle 2 vectors (pandas.Series() objects), should
return one vector of the same length.

\end{description}

\end{fulllineitems}

\index{edges\_in\_complexes() (pypath.legacy.main.PyPath method)@\spxentry{edges\_in\_complexes()}\spxextra{pypath.legacy.main.PyPath method}}

\begin{fulllineitems}
\phantomsection\label{\detokenize{reference:pypath.legacy.main.PyPath.edges_in_complexes}}\pysiglinewithargsret{\sphinxbfcode{\sphinxupquote{edges\_in\_complexes}}}{\emph{csources={[}'corum'{]}, graph=None}}{}
Creates edge attributes \sphinxcode{\sphinxupquote{complexes}} and \sphinxcode{\sphinxupquote{in\_complex}}.
These are both dicts where the keys are complex resources.
The values in \sphinxcode{\sphinxupquote{complexes}} are the list of complex names
both the source and the target vertices belong to.
The values \sphinxcode{\sphinxupquote{in\_complex}} are boolean values whether there
is at least one complex in the given resources both the
source and the target vertex of the edge belong to.
\begin{description}
\item[{@csources}] \leavevmode{[}list{]}
List of complex resources. Should be already loaded.

\item[{@graph}] \leavevmode{[}igraph.Graph(){]}
The graph object to do the calculations on.

\end{description}

\end{fulllineitems}

\index{edges\_ptms() (pypath.legacy.main.PyPath method)@\spxentry{edges\_ptms()}\spxextra{pypath.legacy.main.PyPath method}}

\begin{fulllineitems}
\phantomsection\label{\detokenize{reference:pypath.legacy.main.PyPath.edges_ptms}}\pysiglinewithargsret{\sphinxbfcode{\sphinxupquote{edges\_ptms}}}{}{}
\end{fulllineitems}

\index{edgeseq\_inverse() (pypath.legacy.main.PyPath method)@\spxentry{edgeseq\_inverse()}\spxextra{pypath.legacy.main.PyPath method}}

\begin{fulllineitems}
\phantomsection\label{\detokenize{reference:pypath.legacy.main.PyPath.edgeseq_inverse}}\pysiglinewithargsret{\sphinxbfcode{\sphinxupquote{edgeseq\_inverse}}}{\emph{edges}}{}
Returns the sequence of all edge indexes that are not in
the argument \sphinxstyleemphasis{edges}.
\begin{quote}\begin{description}
\item[{Parameters}] \leavevmode
\sphinxstyleliteralstrong{\sphinxupquote{edges}} (\sphinxstyleliteralemphasis{\sphinxupquote{set}}) \textendash{} Sequence of edge indices {[}int{]} that will not be returned.

\item[{Returns}] \leavevmode
(\sphinxstyleemphasis{list}) \textendash{} Contains all edge indices {[}int{]} of the
undirected network except the ones on \sphinxstyleemphasis{edges} argument.

\end{description}\end{quote}

\end{fulllineitems}

\index{entities\_by\_resource() (pypath.legacy.main.PyPath method)@\spxentry{entities\_by\_resource()}\spxextra{pypath.legacy.main.PyPath method}}

\begin{fulllineitems}
\phantomsection\label{\detokenize{reference:pypath.legacy.main.PyPath.entities_by_resource}}\pysiglinewithargsret{\sphinxbfcode{\sphinxupquote{entities\_by\_resource}}}{\emph{resources=None}, \emph{entity\_type=None}, \emph{**kwargs}}{}
Returns a \sphinxstyleemphasis{dict} of {\color{red}\bfseries{}*}set*s with resources as keys and sets of
entities as values.

\end{fulllineitems}

\index{entities\_by\_resources() (pypath.legacy.main.PyPath method)@\spxentry{entities\_by\_resources()}\spxextra{pypath.legacy.main.PyPath method}}

\begin{fulllineitems}
\phantomsection\label{\detokenize{reference:pypath.legacy.main.PyPath.entities_by_resources}}\pysiglinewithargsret{\sphinxbfcode{\sphinxupquote{entities\_by\_resources}}}{}{}
Returns a dict of sets with resources as keys and sets of entity IDs
as values.

\end{fulllineitems}

\index{export\_dot() (pypath.legacy.main.PyPath method)@\spxentry{export\_dot()}\spxextra{pypath.legacy.main.PyPath method}}

\begin{fulllineitems}
\phantomsection\label{\detokenize{reference:pypath.legacy.main.PyPath.export_dot}}\pysiglinewithargsret{\sphinxbfcode{\sphinxupquote{export\_dot}}}{\emph{nodes=None}, \emph{edges=None}, \emph{directed=True}, \emph{labels='genesymbol'}, \emph{edges\_filter=\textless{}function PyPath.\textless{}lambda\textgreater{}\textgreater{}}, \emph{nodes\_filter=\textless{}function PyPath.\textless{}lambda\textgreater{}\textgreater{}}, \emph{edge\_sources=None}, \emph{dir\_sources=None}, \emph{graph=None}, \emph{return\_object=False}, \emph{save\_dot=None}, \emph{save\_graphics=None}, \emph{prog='neato'}, \emph{format=None}, \emph{hide=False}, \emph{font=None}, \emph{auto\_edges=False}, \emph{hide\_nodes={[}{]}}, \emph{defaults=\{\}}, \emph{**kwargs}}{}
Builds a pygraphviz.AGraph() object with filtering the edges
and vertices along arbitrary criteria.
Returns the Agraph object if requesred, or exports the dot
file, or saves the graphics.

@nodes : list
List of vertex ids to be included.
@edges : list
List of edge ids to be included.
@directed : bool
Create a directed or undirected graph.
@labels : str
Name type to be used as id/label in the dot format.
@edges\_filter : function
Function to filter edges, accepting igraph.Edge as argument.
@nodes\_filter : function
Function to filter vertices, accepting igraph.Vertex as argument.
@edge\_sources : list
Sources to be included.
@dir\_sources : list
Direction and effect sources to be included.
@graph : igraph.Graph
The graph object to export.
@return\_object : bool
Whether to return the pygraphviz.AGraph object.
@save\_dot : str
Filename to export the dot file to.
@save\_graphics : str
Filename to export the graphics, the extension defines the format.
@prog : str
The graphviz layout algorithm to use.
@format : str
The graphics format passed to pygraphviz.AGrapg().draw().
@hide : bool
Hide filtered edges instead of omit them.
@hide nodes : list
Nodes to hide. List of vertex ids.
@auto\_edges : str
Automatic, built-in style for edges.
‘DIRECTIONS’ or ‘RESOURCE\_CATEGORIES’ are supported.
@font : str
Font to use for labels.
For using more than one fonts refer to graphviz attributes with constant values
or define callbacks or mapping dictionaries.
@defaults : dict
Default values for graphviz attributes, labeled with the entity, e.g.
\sphinxtitleref{\{‘edge\_penwidth’: 0.2\}}.
@**kwargs : constant, callable or dict
Graphviz attributes, labeled by the target entity. E.g. \sphinxtitleref{edge\_penwidth},
‘vertex\_shape{}` or \sphinxtitleref{graph\_label}.
If the value is constant, this value will be used.
If the value is dict, and has \sphinxtitleref{\_name} as key, for every instance of the
given entity, the value of the attribute defined by \sphinxtitleref{\_name} will be looked
up in the dict, and the corresponding value will be given to the graphviz
attribute. If the key \sphinxtitleref{\_name} is missing from the dict, igraph vertex and
edge indices will be looked up among the keys.
If the value is callable, it will be called with the current instance of
the entity and the returned value will be used for the graphviz attribute.
E.g. \sphinxtitleref{edge\_arrowhead(edge)} or \sphinxtitleref{vertex\_fillcolor(vertex)}
Example:
\begin{quote}

import pypath
from pypath import data\_formats
net = pypath.PyPath()
net.init\_network(pfile = ‘cache/default.pickle’)
\#net.init\_network(\{‘arn’: data\_formats.omnipath{[}‘arn’{]}\})
tgf = {[}v.index for v in net.graph.vs if ‘TGF’ in v{[}‘slk\_pathways’{]}{]}
dot = net.export\_dot(nodes = tgf, save\_graphics = ‘tgf\_slk.pdf’, prog = ‘dot’,
\begin{quote}

main\_title = ‘TGF-beta pathway’, return\_object = True,
label\_font = ‘HelveticaNeueLTStd Med Cn’,
edge\_sources = {[}‘SignaLink3’{]},
dir\_sources = {[}‘SignaLink3’{]}, hide = True)
\end{quote}
\end{quote}

\end{fulllineitems}

\index{export\_edgelist() (pypath.legacy.main.PyPath method)@\spxentry{export\_edgelist()}\spxextra{pypath.legacy.main.PyPath method}}

\begin{fulllineitems}
\phantomsection\label{\detokenize{reference:pypath.legacy.main.PyPath.export_edgelist}}\pysiglinewithargsret{\sphinxbfcode{\sphinxupquote{export\_edgelist}}}{\emph{fname, graph=None, names={[}'name'{]}, edge\_attributes={[}{]}, sep='\textbackslash{}t'}}{}
Write edge list to text file with attributes
\begin{quote}\begin{description}
\item[{Parameters}] \leavevmode\begin{itemize}
\item {} 
\sphinxstyleliteralstrong{\sphinxupquote{fname}} \textendash{} the name of the file or a stream to read from.

\item {} 
\sphinxstyleliteralstrong{\sphinxupquote{graph}} \textendash{} the igraph object containing the network

\item {} 
\sphinxstyleliteralstrong{\sphinxupquote{names}} \textendash{} list with the vertex attribute names to be printed
for source and target vertices

\item {} 
\sphinxstyleliteralstrong{\sphinxupquote{edge\_attributes}} \textendash{} list with the edge attribute names
to be printed

\item {} 
\sphinxstyleliteralstrong{\sphinxupquote{sep}} \textendash{} string used to separate columns

\end{itemize}

\end{description}\end{quote}

\end{fulllineitems}

\index{export\_graphml() (pypath.legacy.main.PyPath method)@\spxentry{export\_graphml()}\spxextra{pypath.legacy.main.PyPath method}}

\begin{fulllineitems}
\phantomsection\label{\detokenize{reference:pypath.legacy.main.PyPath.export_graphml}}\pysiglinewithargsret{\sphinxbfcode{\sphinxupquote{export\_graphml}}}{\emph{outfile=None}, \emph{graph=None}, \emph{name='main'}}{}
Saves the network in a \sphinxcode{\sphinxupquote{.graphml}} file.
\begin{quote}\begin{description}
\item[{Parameters}] \leavevmode\begin{itemize}
\item {} 
\sphinxstyleliteralstrong{\sphinxupquote{outfile}} (\sphinxstyleliteralemphasis{\sphinxupquote{str}}) \textendash{} Optional, \sphinxcode{\sphinxupquote{None}} by default. Name/path of the output file.
If none is passed,
\sphinxcode{\sphinxupquote{'results/netrowk-\textless{}session\_id\textgreater{}.graphml'}} is used.

\item {} 
\sphinxstyleliteralstrong{\sphinxupquote{graph}} (\sphinxstyleliteralemphasis{\sphinxupquote{igraph.Graph}}) \textendash{} Optional, \sphinxcode{\sphinxupquote{None}} by default. The network object to be
saved. If none is passed, takes the undirected network of
the current instance.

\item {} 
\sphinxstyleliteralstrong{\sphinxupquote{name}} (\sphinxstyleliteralemphasis{\sphinxupquote{str}}) \textendash{} Optional, \sphinxcode{\sphinxupquote{'main'}} by default. The graph name for the
output file.

\end{itemize}

\end{description}\end{quote}

\end{fulllineitems}

\index{export\_ptms\_tab() (pypath.legacy.main.PyPath method)@\spxentry{export\_ptms\_tab()}\spxextra{pypath.legacy.main.PyPath method}}

\begin{fulllineitems}
\phantomsection\label{\detokenize{reference:pypath.legacy.main.PyPath.export_ptms_tab}}\pysiglinewithargsret{\sphinxbfcode{\sphinxupquote{export\_ptms\_tab}}}{\emph{outfile=None}}{}
Exports a tab file containing the PTM interaction information
loaded in the network.
\begin{quote}\begin{description}
\item[{Parameters}] \leavevmode
\sphinxstyleliteralstrong{\sphinxupquote{outfile}} (\sphinxstyleliteralemphasis{\sphinxupquote{str}}) \textendash{} Optional, \sphinxcode{\sphinxupquote{None}} by default. The output file nama/path to
store the PTM information. If none is provided, the default
is \sphinxcode{\sphinxupquote{'results/network-\textless{}session\_id\textgreater{}.tab'}}.

\item[{Returns}] \leavevmode
(\sphinxstyleemphasis{list}) \textendash{} Contains the edge indices {[}int{]} of all PTM
interactions.

\end{description}\end{quote}

\end{fulllineitems}

\index{export\_sif() (pypath.legacy.main.PyPath method)@\spxentry{export\_sif()}\spxextra{pypath.legacy.main.PyPath method}}

\begin{fulllineitems}
\phantomsection\label{\detokenize{reference:pypath.legacy.main.PyPath.export_sif}}\pysiglinewithargsret{\sphinxbfcode{\sphinxupquote{export\_sif}}}{\emph{outfile=None}}{}
Exports the network interactions in \sphinxcode{\sphinxupquote{.sif}} format (Simple
Interaction Format).
\begin{quote}\begin{description}
\item[{Parameters}] \leavevmode
\sphinxstyleliteralstrong{\sphinxupquote{outfile}} (\sphinxstyleliteralemphasis{\sphinxupquote{str}}) \textendash{} Optional, \sphinxcode{\sphinxupquote{None}} by default. Name/path of the output file.
If none is passed, \sphinxcode{\sphinxupquote{'results/netrowk-\textless{}session\_id\textgreater{}.sif'}}
is used.

\end{description}\end{quote}

\end{fulllineitems}

\index{export\_struct\_tab() (pypath.legacy.main.PyPath method)@\spxentry{export\_struct\_tab()}\spxextra{pypath.legacy.main.PyPath method}}

\begin{fulllineitems}
\phantomsection\label{\detokenize{reference:pypath.legacy.main.PyPath.export_struct_tab}}\pysiglinewithargsret{\sphinxbfcode{\sphinxupquote{export\_struct\_tab}}}{\emph{outfile=None}}{}
Exports a tab file containing the domain interaction information
and PTM regulation loaded in the network.
\begin{quote}\begin{description}
\item[{Parameters}] \leavevmode
\sphinxstyleliteralstrong{\sphinxupquote{outfile}} (\sphinxstyleliteralemphasis{\sphinxupquote{str}}) \textendash{} Optional, \sphinxcode{\sphinxupquote{None}} by default. The output file nama/path to
store the PTM information. If none is provided, the default
is \sphinxcode{\sphinxupquote{'results/network-\textless{}session\_id\textgreater{}.tab'}}.

\item[{Returns}] \leavevmode
(\sphinxstyleemphasis{list}) \textendash{} Contains the edge indices {[}int{]} of all PTM
interactions.

\end{description}\end{quote}

\end{fulllineitems}

\index{export\_tab() (pypath.legacy.main.PyPath method)@\spxentry{export\_tab()}\spxextra{pypath.legacy.main.PyPath method}}

\begin{fulllineitems}
\phantomsection\label{\detokenize{reference:pypath.legacy.main.PyPath.export_tab}}\pysiglinewithargsret{\sphinxbfcode{\sphinxupquote{export\_tab}}}{\emph{outfile=None}, \emph{extra\_node\_attrs=\{\}}, \emph{extra\_edge\_attrs=\{\}}, \emph{unique\_pairs=True}, \emph{**kwargs}}{}
Exports the network in a tabular format. By default UniProt IDs,
Gene Symbols, source databases, literature references,
directionality and sign information and interaction type are
included.
\begin{quote}\begin{description}
\item[{Parameters}] \leavevmode\begin{itemize}
\item {} 
\sphinxstyleliteralstrong{\sphinxupquote{outfile}} (\sphinxstyleliteralemphasis{\sphinxupquote{str}}) \textendash{} Optional, \sphinxcode{\sphinxupquote{None}} by default. Name/path of the output file.
If none is passed, \sphinxcode{\sphinxupquote{'results/netrowk-\textless{}session\_id\textgreater{}.tab'}}
is used.

\item {} 
\sphinxstyleliteralstrong{\sphinxupquote{extra\_node\_attrs}} (\sphinxstyleliteralemphasis{\sphinxupquote{dict}}) \textendash{} Optional, \sphinxcode{\sphinxupquote{\{\}}} by default. Additional node attributes to
be included in the exported table. Keys are column names
used in the header while values are names of vertex
attributes. In the header \sphinxcode{\sphinxupquote{'\_A'}} and \sphinxcode{\sphinxupquote{'\_B'}} suffixes
will be appended to the column names so the values can be
assigned to A and B side interaction partners.

\item {} 
\sphinxstyleliteralstrong{\sphinxupquote{extra\_edge\_attrs}} (\sphinxstyleliteralemphasis{\sphinxupquote{dict}}) \textendash{} Optional, \sphinxcode{\sphinxupquote{\{\}}} by default. Additional edge attributes to
be included in the exported table. Keys are column names
used in the header while values are names of edge
attributes.

\item {} 
\sphinxstyleliteralstrong{\sphinxupquote{unique\_pairs}} (\sphinxstyleliteralemphasis{\sphinxupquote{bool}}) \textendash{} Optional, \sphinxcode{\sphinxupquote{True}} by default. If set to \sphinxcode{\sphinxupquote{True}} each line
corresponds to a unique pair of molecules, all
directionality and sign information are covered in other
columns. If \sphinxcode{\sphinxupquote{False}}, order of \sphinxcode{\sphinxupquote{'A'}} and \sphinxcode{\sphinxupquote{'B'}} IDs
corresponds to the direction while sign covered in further
columns.

\item {} 
\sphinxstyleliteralstrong{\sphinxupquote{kwargs}} (\sphinxstyleliteralemphasis{\sphinxupquote{**}}) \textendash{} Additional keyword arguments passed to
\sphinxcode{\sphinxupquote{pypath.export.Export}}.

\end{itemize}

\end{description}\end{quote}

\end{fulllineitems}

\index{find\_all\_paths() (pypath.legacy.main.PyPath method)@\spxentry{find\_all\_paths()}\spxextra{pypath.legacy.main.PyPath method}}

\begin{fulllineitems}
\phantomsection\label{\detokenize{reference:pypath.legacy.main.PyPath.find_all_paths}}\pysiglinewithargsret{\sphinxbfcode{\sphinxupquote{find\_all\_paths}}}{\emph{start}, \emph{end}, \emph{attr=None}, \emph{mode='OUT'}, \emph{maxlen=2}, \emph{graph=None}, \emph{silent=False}, \emph{update\_adjlist=True}}{}
Finds all paths up to length \sphinxtitleref{maxlen} between groups of
vertices. This function is needed only becaues igraph{}`s
get\_all\_shortest\_paths() finds only the shortest, not any
path up to a defined length.
\begin{description}
\item[{start}] \leavevmode{[}int or list{]}
Indices of the starting node(s) of the paths.

\item[{end}] \leavevmode{[}int or list{]}
Indices of the target node(s) of the paths.

\item[{attr}] \leavevmode{[}str{]}
Name of the vertex attribute to identify the vertices by.
Necessary if \sphinxcode{\sphinxupquote{start}} and \sphinxcode{\sphinxupquote{end}} are not igraph vertex ids
but for example vertex names or labels.

\item[{mode}] \leavevmode{[}‘IN’, ‘OUT’, ‘ALL’{]}
Passed to igraph.Graph.neighbors()

\item[{maxlen}] \leavevmode{[}int{]}
Maximum length of paths in steps, i.e. if maxlen = 3, then
the longest path may consist of 3 edges and 4 nodes.

\item[{graph}] \leavevmode{[}igraph.Graph object{]}
The graph you want to find paths in. self.graph by default.

\end{description}

\end{fulllineitems}

\index{find\_all\_paths2() (pypath.legacy.main.PyPath method)@\spxentry{find\_all\_paths2()}\spxextra{pypath.legacy.main.PyPath method}}

\begin{fulllineitems}
\phantomsection\label{\detokenize{reference:pypath.legacy.main.PyPath.find_all_paths2}}\pysiglinewithargsret{\sphinxbfcode{\sphinxupquote{find\_all\_paths2}}}{\emph{graph}, \emph{start}, \emph{end}, \emph{mode='OUT'}, \emph{maxlen=2}, \emph{psize=100}, \emph{update\_adjlist=True}}{}
\end{fulllineitems}

\index{find\_complex() (pypath.legacy.main.PyPath method)@\spxentry{find\_complex()}\spxextra{pypath.legacy.main.PyPath method}}

\begin{fulllineitems}
\phantomsection\label{\detokenize{reference:pypath.legacy.main.PyPath.find_complex}}\pysiglinewithargsret{\sphinxbfcode{\sphinxupquote{find\_complex}}}{\emph{search}}{}
Finds complexes by their non standard names.
E.g. to find DNA polymerases you can use the search
term \sphinxtitleref{DNA pol} which will be tested against complex names
in CORUM.

\end{fulllineitems}

\index{first\_neighbours() (pypath.legacy.main.PyPath method)@\spxentry{first\_neighbours()}\spxextra{pypath.legacy.main.PyPath method}}

\begin{fulllineitems}
\phantomsection\label{\detokenize{reference:pypath.legacy.main.PyPath.first_neighbours}}\pysiglinewithargsret{\sphinxbfcode{\sphinxupquote{first\_neighbours}}}{\emph{node}, \emph{indices=False}}{}
Looks for the first neighbours of a given node and returns a
list of their UniProt IDs.
\begin{quote}\begin{description}
\item[{Parameters}] \leavevmode\begin{itemize}
\item {} 
\sphinxstyleliteralstrong{\sphinxupquote{node}} (\sphinxstyleliteralemphasis{\sphinxupquote{str}}) \textendash{} The UniProt ID of the node of interest. Can also be the
index of such node {[}int{]}.

\item {} 
\sphinxstyleliteralstrong{\sphinxupquote{indices}} (\sphinxstyleliteralemphasis{\sphinxupquote{bool}}) \textendash{} Optional, \sphinxcode{\sphinxupquote{False}} by default. Whether to return the
neighbour nodes indices or their UniProt IDs.

\end{itemize}

\item[{Returns}] \leavevmode
(\sphinxstyleemphasis{list}) \textendash{} The list containing the first neighbours of the
queried node.

\end{description}\end{quote}

\end{fulllineitems}

\index{fisher\_enrichment() (pypath.legacy.main.PyPath method)@\spxentry{fisher\_enrichment()}\spxextra{pypath.legacy.main.PyPath method}}

\begin{fulllineitems}
\phantomsection\label{\detokenize{reference:pypath.legacy.main.PyPath.fisher_enrichment}}\pysiglinewithargsret{\sphinxbfcode{\sphinxupquote{fisher\_enrichment}}}{\emph{lst}, \emph{attr}, \emph{ref='proteome'}}{}
Computes an enrichment analysis using Fisher’s exact test. The
contingency table is built as follows:
First row contains the number of nodes in the \sphinxstyleemphasis{ref} list (such
list is considered to be loaded in
\sphinxcode{\sphinxupquote{pypath.main.PyPath.lists}}) and the number of nodes in
the current (undirected) network. Second row contains the number
of nodes in \sphinxstyleemphasis{lst} list (also considered to be already loaded)
and the number of nodes in the network with a non-empty
attribute \sphinxstyleemphasis{attr}. Uses \sphinxcode{\sphinxupquote{scipy.stats.fisher\_exact()}}, see
the documentation of the corresponding package for more
information.
\begin{quote}\begin{description}
\item[{Parameters}] \leavevmode\begin{itemize}
\item {} 
\sphinxstyleliteralstrong{\sphinxupquote{lst}} (\sphinxstyleliteralemphasis{\sphinxupquote{str}}) \textendash{} Name of the list in \sphinxcode{\sphinxupquote{pypath.main.PyPath.lists}}
whose number of elements will be the first element in the
second row of the contingency table.

\item {} 
\sphinxstyleliteralstrong{\sphinxupquote{attr}} (\sphinxstyleliteralemphasis{\sphinxupquote{str}}) \textendash{} The node attribute name for which the number of nodes in the
network with such attribute will be the second element of
the second row of the contingency table.

\item {} 
\sphinxstyleliteralstrong{\sphinxupquote{ref}} (\sphinxstyleliteralemphasis{\sphinxupquote{str}}) \textendash{} Optional, \sphinxcode{\sphinxupquote{'proteome'}} by default. The name of the list in
\sphinxcode{\sphinxupquote{pypath.main.PyPath.lists}} whose number of elements
will be the first element of the first row of the
contingency table.

\end{itemize}

\item[{Returns}] \leavevmode
\begin{itemize}
\item {} 
(\sphinxstyleemphasis{float}) \textendash{} Prior odds ratio.

\item {} 
(\sphinxstyleemphasis{float}) \textendash{} P-value or probability of obtaining a
distribution as extreme as the observed, assuming that the
null hypothesis is true.

\end{itemize}


\end{description}\end{quote}

\end{fulllineitems}

\index{geneset\_enrichment() (pypath.legacy.main.PyPath method)@\spxentry{geneset\_enrichment()}\spxextra{pypath.legacy.main.PyPath method}}

\begin{fulllineitems}
\phantomsection\label{\detokenize{reference:pypath.legacy.main.PyPath.geneset_enrichment}}\pysiglinewithargsret{\sphinxbfcode{\sphinxupquote{geneset\_enrichment}}}{\emph{proteins}, \emph{all\_proteins=None}, \emph{geneset\_ids=None}, \emph{alpha=0.05}, \emph{correction\_method='hommel'}}{}
Does not work at the moment because cfisher module should be
replaced with scipy.

\end{fulllineitems}

\index{genesymbol() (pypath.legacy.main.PyPath method)@\spxentry{genesymbol()}\spxextra{pypath.legacy.main.PyPath method}}

\begin{fulllineitems}
\phantomsection\label{\detokenize{reference:pypath.legacy.main.PyPath.genesymbol}}\pysiglinewithargsret{\sphinxbfcode{\sphinxupquote{genesymbol}}}{\emph{genesymbol}}{}
Returns \sphinxcode{\sphinxupquote{igraph.Vertex()}} object if the GeneSymbol
can be found in the default undirected network,
otherwise \sphinxcode{\sphinxupquote{None}}.
\begin{description}
\item[{@genesymbol}] \leavevmode{[}str{]}
GeneSymbol.

\end{description}

\end{fulllineitems}

\index{genesymbol\_labels() (pypath.legacy.main.PyPath method)@\spxentry{genesymbol\_labels()}\spxextra{pypath.legacy.main.PyPath method}}

\begin{fulllineitems}
\phantomsection\label{\detokenize{reference:pypath.legacy.main.PyPath.genesymbol_labels}}\pysiglinewithargsret{\sphinxbfcode{\sphinxupquote{genesymbol\_labels}}}{\emph{graph=None}, \emph{remap\_all=False}}{}
Creats vertex attribute \sphinxcode{\sphinxupquote{'label'}} and fills up with the
corresponding GeneSymbols of all proteins where the GeneSymbol
can be looked up based on the default name of the protein
vertex (UniProt ID by default). If the attribute \sphinxcode{\sphinxupquote{'label'}} has
been already initialized, updates this attribute or recreates if
\sphinxstyleemphasis{remap\_all} is set to \sphinxcode{\sphinxupquote{True}}.
\begin{quote}\begin{description}
\item[{Parameters}] \leavevmode\begin{itemize}
\item {} 
\sphinxstyleliteralstrong{\sphinxupquote{graph}} (\sphinxstyleliteralemphasis{\sphinxupquote{igraph.Graph}}) \textendash{} Optional, \sphinxcode{\sphinxupquote{None}} by default. The network graph object
where the GeneSymbol labels are to be set/updated. If none
is passed, takes the current network undirected graph by
default (\sphinxcode{\sphinxupquote{pypath.main.PyPath.graph}}).

\item {} 
\sphinxstyleliteralstrong{\sphinxupquote{remap\_all}} (\sphinxstyleliteralemphasis{\sphinxupquote{bool}}) \textendash{} Optional, \sphinxcode{\sphinxupquote{False}} by default. Whether to map anew the
GeneSymbol labels if those were already initialized.

\end{itemize}

\end{description}\end{quote}

\end{fulllineitems}

\index{genesymbols() (pypath.legacy.main.PyPath method)@\spxentry{genesymbols()}\spxextra{pypath.legacy.main.PyPath method}}

\begin{fulllineitems}
\phantomsection\label{\detokenize{reference:pypath.legacy.main.PyPath.genesymbols}}\pysiglinewithargsret{\sphinxbfcode{\sphinxupquote{genesymbols}}}{\emph{genesymbols}}{}
\end{fulllineitems}

\index{get\_attrs() (pypath.legacy.main.PyPath method)@\spxentry{get\_attrs()}\spxextra{pypath.legacy.main.PyPath method}}

\begin{fulllineitems}
\phantomsection\label{\detokenize{reference:pypath.legacy.main.PyPath.get_attrs}}\pysiglinewithargsret{\sphinxbfcode{\sphinxupquote{get\_attrs}}}{\emph{line}, \emph{spec}, \emph{lnum}}{}
\end{fulllineitems}

\index{get\_directed() (pypath.legacy.main.PyPath method)@\spxentry{get\_directed()}\spxextra{pypath.legacy.main.PyPath method}}

\begin{fulllineitems}
\phantomsection\label{\detokenize{reference:pypath.legacy.main.PyPath.get_directed}}\pysiglinewithargsret{\sphinxbfcode{\sphinxupquote{get\_directed}}}{\emph{graph=False}, \emph{conv\_edges=False}, \emph{mutual=False}, \emph{ret=False}}{}
Converts a copy of \sphinxstyleemphasis{graph} undirected \sphinxstyleemphasis{igraph.Graph} object to a
directed one. By default it converts the current network
instance in \sphinxcode{\sphinxupquote{pypath.main.PyPath.graph}} and places the
copy of the directed instance in
\sphinxcode{\sphinxupquote{pypath.main.PyPath.dgraph}}.
\begin{quote}\begin{description}
\item[{Parameters}] \leavevmode\begin{itemize}
\item {} 
\sphinxstyleliteralstrong{\sphinxupquote{graph}} (\sphinxstyleliteralemphasis{\sphinxupquote{igraph.Graph}}) \textendash{} Optional, \sphinxcode{\sphinxupquote{None}} by default. Undirected graph object. If
none is passed, takes the current undirected network
instance and saves the directed network under the attribute
\sphinxcode{\sphinxupquote{pypath.main.PyPath.dgraph}}. Otherwise, the
directed graph will be returned instead.

\item {} 
\sphinxstyleliteralstrong{\sphinxupquote{conv\_edges}} (\sphinxstyleliteralemphasis{\sphinxupquote{bool}}) \textendash{} Optional, \sphinxcode{\sphinxupquote{False}} by default. Whether to convert
undirected edges (those without explicit direction
information) to an arbitrary direction edge or
a pair of opposite edges. Otherwise those will be deleted.

\item {} 
\sphinxstyleliteralstrong{\sphinxupquote{mutual}} (\sphinxstyleliteralemphasis{\sphinxupquote{bool}}) \textendash{} Optional, \sphinxcode{\sphinxupquote{False}} by default. If \sphinxstyleemphasis{conv\_edges} is \sphinxcode{\sphinxupquote{True}},
whether to convert the undirected edges to a single,
arbitrary directed edge, or a pair of opposite directed
edges.

\item {} 
\sphinxstyleliteralstrong{\sphinxupquote{ret}} (\sphinxstyleliteralemphasis{\sphinxupquote{bool}}) \textendash{} Optional, \sphinxcode{\sphinxupquote{False}} by default. Whether to return the
directed graph instance, or not. If a \sphinxstyleemphasis{graph} is provided,
its directed version will be returned anyway.

\end{itemize}

\item[{Returns}] \leavevmode
(\sphinxstyleemphasis{igraph.Graph}) \textendash{} If \sphinxstyleemphasis{graph} is passed or \sphinxstyleemphasis{ret} is
\sphinxcode{\sphinxupquote{True}}, returns the copy of the directed graph. otherwise
returns \sphinxcode{\sphinxupquote{None}}.

\end{description}\end{quote}

\end{fulllineitems}

\index{get\_dirs\_signs() (pypath.legacy.main.PyPath method)@\spxentry{get\_dirs\_signs()}\spxextra{pypath.legacy.main.PyPath method}}

\begin{fulllineitems}
\phantomsection\label{\detokenize{reference:pypath.legacy.main.PyPath.get_dirs_signs}}\pysiglinewithargsret{\sphinxbfcode{\sphinxupquote{get\_dirs\_signs}}}{}{}
\end{fulllineitems}

\index{get\_edge() (pypath.legacy.main.PyPath method)@\spxentry{get\_edge()}\spxextra{pypath.legacy.main.PyPath method}}

\begin{fulllineitems}
\phantomsection\label{\detokenize{reference:pypath.legacy.main.PyPath.get_edge}}\pysiglinewithargsret{\sphinxbfcode{\sphinxupquote{get\_edge}}}{\emph{source}, \emph{target}, \emph{directed=True}}{}
Returns \sphinxcode{\sphinxupquote{igraph.Edge}} object if an edge exist between
the 2 proteins, otherwise \sphinxcode{\sphinxupquote{None}}.
\begin{quote}\begin{description}
\item[{Parameters}] \leavevmode\begin{itemize}
\item {} 
\sphinxstyleliteralstrong{\sphinxupquote{source}} (\sphinxstyleliteralemphasis{\sphinxupquote{int}}\sphinxstyleliteralemphasis{\sphinxupquote{,}}\sphinxstyleliteralemphasis{\sphinxupquote{str}}) \textendash{} Vertex index or UniProt ID or GeneSymbol or \sphinxcode{\sphinxupquote{igraph.Vertex}}
object.

\item {} 
\sphinxstyleliteralstrong{\sphinxupquote{target}} (\sphinxstyleliteralemphasis{\sphinxupquote{int}}\sphinxstyleliteralemphasis{\sphinxupquote{,}}\sphinxstyleliteralemphasis{\sphinxupquote{str}}) \textendash{} Vertex index or UniProt ID or GeneSymbol or \sphinxcode{\sphinxupquote{igraph.Vertex}}
object.

\item {} 
\sphinxstyleliteralstrong{\sphinxupquote{directed}} (\sphinxstyleliteralemphasis{\sphinxupquote{bool}}) \textendash{} To be passed to igraph.Graph.get\_eid()

\end{itemize}

\end{description}\end{quote}

\end{fulllineitems}

\index{get\_edges() (pypath.legacy.main.PyPath method)@\spxentry{get\_edges()}\spxextra{pypath.legacy.main.PyPath method}}

\begin{fulllineitems}
\phantomsection\label{\detokenize{reference:pypath.legacy.main.PyPath.get_edges}}\pysiglinewithargsret{\sphinxbfcode{\sphinxupquote{get\_edges}}}{\emph{sources}, \emph{targets}, \emph{directed=True}}{}
Returns a generator with all edges between source and target vertices.
\begin{quote}\begin{description}
\item[{Parameters}] \leavevmode\begin{itemize}
\item {} 
\sphinxstyleliteralstrong{\sphinxupquote{sources}} (\sphinxstyleliteralemphasis{\sphinxupquote{iterable}}) \textendash{} Source vertex IDs, names, labels, or any iterable yielding
\sphinxcode{\sphinxupquote{igraph.Vertex}} objects.

\item {} 
\sphinxstyleliteralstrong{\sphinxupquote{targets}} (\sphinxstyleliteralemphasis{\sphinxupquote{iterable}}) \textendash{} Target vertec IDs, names, labels, or any iterable yielding
\sphinxcode{\sphinxupquote{igraph.Vertex}} objects.

\item {} 
\sphinxstyleliteralstrong{\sphinxupquote{directed}} (\sphinxstyleliteralemphasis{\sphinxupquote{bool}}) \textendash{} Passed to \sphinxcode{\sphinxupquote{igraph.get\_eid()}}.

\end{itemize}

\end{description}\end{quote}

\end{fulllineitems}

\index{get\_function() (pypath.legacy.main.PyPath method)@\spxentry{get\_function()}\spxextra{pypath.legacy.main.PyPath method}}

\begin{fulllineitems}
\phantomsection\label{\detokenize{reference:pypath.legacy.main.PyPath.get_function}}\pysiglinewithargsret{\sphinxbfcode{\sphinxupquote{get\_function}}}{\emph{fun}}{}
\end{fulllineitems}

\index{get\_giant() (pypath.legacy.main.PyPath method)@\spxentry{get\_giant()}\spxextra{pypath.legacy.main.PyPath method}}

\begin{fulllineitems}
\phantomsection\label{\detokenize{reference:pypath.legacy.main.PyPath.get_giant}}\pysiglinewithargsret{\sphinxbfcode{\sphinxupquote{get\_giant}}}{\emph{replace=False}, \emph{graph=None}}{}
Returns the giant component of the \sphinxstyleemphasis{graph}, or replaces the
\sphinxcode{\sphinxupquote{igraph.Graph}} instance with only the giant component
if specified.
\begin{quote}\begin{description}
\item[{Parameters}] \leavevmode\begin{itemize}
\item {} 
\sphinxstyleliteralstrong{\sphinxupquote{replace}} (\sphinxstyleliteralemphasis{\sphinxupquote{bool}}) \textendash{} Optional, \sphinxcode{\sphinxupquote{False}} by default. Specifies whether to replace
the \sphinxcode{\sphinxupquote{igraph.Graph}} instance. This can be either
the undirected network of the current
\sphinxcode{\sphinxupquote{pypath.main.PyPath}} instance (default) or the one
passed under the keyword argument \sphinxstyleemphasis{graph}.

\item {} 
\sphinxstyleliteralstrong{\sphinxupquote{graph}} (\sphinxstyleliteralemphasis{\sphinxupquote{igraph.Graph}}) \textendash{} Optional, \sphinxcode{\sphinxupquote{None}} by default. The graph object from which
the giant component is to be computed. If none is specified,
takes the undirected network of the current
\sphinxcode{\sphinxupquote{pypath.main.PyPath}} instance.

\end{itemize}

\item[{Returns}] \leavevmode
(\sphinxstyleemphasis{igraph.Graph}) \textendash{} If \sphinxcode{\sphinxupquote{replace=False}}, returns a copy of
the giant component graph.

\end{description}\end{quote}

\end{fulllineitems}

\index{get\_go() (pypath.legacy.main.PyPath method)@\spxentry{get\_go()}\spxextra{pypath.legacy.main.PyPath method}}

\begin{fulllineitems}
\phantomsection\label{\detokenize{reference:pypath.legacy.main.PyPath.get_go}}\pysiglinewithargsret{\sphinxbfcode{\sphinxupquote{get\_go}}}{\emph{organism=None}}{}
Returns the \sphinxcode{\sphinxupquote{GOAnnotation}} object for the organism requested
(or the default one).

\end{fulllineitems}

\index{get\_max() (pypath.legacy.main.PyPath method)@\spxentry{get\_max()}\spxextra{pypath.legacy.main.PyPath method}}

\begin{fulllineitems}
\phantomsection\label{\detokenize{reference:pypath.legacy.main.PyPath.get_max}}\pysiglinewithargsret{\sphinxbfcode{\sphinxupquote{get\_max}}}{\emph{attrList}}{}
\end{fulllineitems}

\index{get\_network() (pypath.legacy.main.PyPath method)@\spxentry{get\_network()}\spxextra{pypath.legacy.main.PyPath method}}

\begin{fulllineitems}
\phantomsection\label{\detokenize{reference:pypath.legacy.main.PyPath.get_network}}\pysiglinewithargsret{\sphinxbfcode{\sphinxupquote{get\_network}}}{\emph{crit}, \emph{andor='or'}, \emph{graph=None}}{}
Retrieves a subnetwork according to a set of user-defined
attributes. Basically applies
\sphinxcode{\sphinxupquote{pypath.main.PyPath.get\_sub()}} on a given \sphinxstyleemphasis{graph}.
\begin{quote}\begin{description}
\item[{Parameters}] \leavevmode\begin{itemize}
\item {} 
\sphinxstyleliteralstrong{\sphinxupquote{crit}} (\sphinxstyleliteralemphasis{\sphinxupquote{dict}}) \textendash{} Defines the critical attributes to generate the subnetwork.
Keys are \sphinxcode{\sphinxupquote{'edge'}} and \sphinxcode{\sphinxupquote{'node'}} and values are {[}dict{]}
containing the critical attribute names {[}str{]} and values
are {[}set{]} containing those attributes of the nodes/edges
that are to be kept.

\item {} 
\sphinxstyleliteralstrong{\sphinxupquote{andor}} (\sphinxstyleliteralemphasis{\sphinxupquote{str}}) \textendash{} Optional, \sphinxcode{\sphinxupquote{'or'}} by default. Determines the search mode.
See \sphinxcode{\sphinxupquote{pypath.main.PyPath.search\_attr\_or()}} and
\sphinxcode{\sphinxupquote{pypath.main.PyPath.search\_attr\_and()}} for more
details.

\item {} 
\sphinxstyleliteralstrong{\sphinxupquote{graph}} (\sphinxstyleliteralemphasis{\sphinxupquote{igraph.Graph}}) \textendash{} Optional, \sphinxcode{\sphinxupquote{None}} by default. The graph object where to
extract the subnetwork. If none is passed, takes the current
network (undirected) graph
(\sphinxcode{\sphinxupquote{pypath.main.PyPath.graph}}).

\end{itemize}

\item[{Returns}] \leavevmode
(\sphinxstyleemphasis{igraph.Graph}) \textendash{} The subgraph obtained from filtering
according to the attributes defined in \sphinxstyleemphasis{crit}.

\end{description}\end{quote}

\end{fulllineitems}

\index{get\_node() (pypath.legacy.main.PyPath method)@\spxentry{get\_node()}\spxextra{pypath.legacy.main.PyPath method}}

\begin{fulllineitems}
\phantomsection\label{\detokenize{reference:pypath.legacy.main.PyPath.get_node}}\pysiglinewithargsret{\sphinxbfcode{\sphinxupquote{get\_node}}}{\emph{identifier}}{}
Returns \sphinxcode{\sphinxupquote{igraph.Vertex()}} object if the identifier
is a valid vertex index in the default undirected graph,
or a UniProt ID or GeneSymbol which can be found in the
default undirected network, otherwise \sphinxcode{\sphinxupquote{None}}.
\begin{description}
\item[{@identifier}] \leavevmode{[}int, str{]}
Vertex index (int) or GeneSymbol (str) or UniProt ID (str) or
\sphinxcode{\sphinxupquote{igraph.Vertex}} object.

\end{description}

\end{fulllineitems}

\index{get\_node\_d() (pypath.legacy.main.PyPath method)@\spxentry{get\_node\_d()}\spxextra{pypath.legacy.main.PyPath method}}

\begin{fulllineitems}
\phantomsection\label{\detokenize{reference:pypath.legacy.main.PyPath.get_node_d}}\pysiglinewithargsret{\sphinxbfcode{\sphinxupquote{get\_node\_d}}}{\emph{identifier}}{}
Same as \sphinxcode{\sphinxupquote{PyPath.get\_node}}, just for the directed graph.
Returns \sphinxcode{\sphinxupquote{igraph.Vertex()}} object if the identifier
is a valid vertex index in the default directed graph,
or a UniProt ID or GeneSymbol which can be found in the
default directed network, otherwise \sphinxcode{\sphinxupquote{None}}.
\begin{description}
\item[{@identifier}] \leavevmode{[}int, str{]}
Vertex index (int) or GeneSymbol (str) or UniProt ID (str) or
\sphinxcode{\sphinxupquote{igraph.Vertex}} object.

\end{description}

\end{fulllineitems}

\index{get\_node\_pair() (pypath.legacy.main.PyPath method)@\spxentry{get\_node\_pair()}\spxextra{pypath.legacy.main.PyPath method}}

\begin{fulllineitems}
\phantomsection\label{\detokenize{reference:pypath.legacy.main.PyPath.get_node_pair}}\pysiglinewithargsret{\sphinxbfcode{\sphinxupquote{get\_node\_pair}}}{\emph{id\_a}, \emph{id\_b}, \emph{directed=False}}{}
Retrieves the node IDs from a pair of node names.
\begin{quote}\begin{description}
\item[{Parameters}] \leavevmode\begin{itemize}
\item {} 
\sphinxstyleliteralstrong{\sphinxupquote{id\_a}} (\sphinxstyleliteralemphasis{\sphinxupquote{str}}) \textendash{} Name of the source node.

\item {} 
\sphinxstyleliteralstrong{\sphinxupquote{id\_b}} (\sphinxstyleliteralemphasis{\sphinxupquote{str}}) \textendash{} Name of the target node.

\item {} 
\sphinxstyleliteralstrong{\sphinxupquote{directed}} (\sphinxstyleliteralemphasis{\sphinxupquote{bool}}) \textendash{} Optional, \sphinxcode{\sphinxupquote{False}} by default. Whether to return the node
indices from the directed or undirected graph.

\end{itemize}

\item[{Returns}] \leavevmode
(\sphinxstyleemphasis{tuple}) \textendash{} The pair of node IDs of the selected graph.
If not found, returns \sphinxcode{\sphinxupquote{False}}.

\end{description}\end{quote}

\end{fulllineitems}

\index{get\_nodes() (pypath.legacy.main.PyPath method)@\spxentry{get\_nodes()}\spxextra{pypath.legacy.main.PyPath method}}

\begin{fulllineitems}
\phantomsection\label{\detokenize{reference:pypath.legacy.main.PyPath.get_nodes}}\pysiglinewithargsret{\sphinxbfcode{\sphinxupquote{get\_nodes}}}{\emph{identifiers}}{}
\end{fulllineitems}

\index{get\_nodes\_d() (pypath.legacy.main.PyPath method)@\spxentry{get\_nodes\_d()}\spxextra{pypath.legacy.main.PyPath method}}

\begin{fulllineitems}
\phantomsection\label{\detokenize{reference:pypath.legacy.main.PyPath.get_nodes_d}}\pysiglinewithargsret{\sphinxbfcode{\sphinxupquote{get\_nodes\_d}}}{\emph{identifiers}}{}
\end{fulllineitems}

\index{get\_pathways() (pypath.legacy.main.PyPath method)@\spxentry{get\_pathways()}\spxextra{pypath.legacy.main.PyPath method}}

\begin{fulllineitems}
\phantomsection\label{\detokenize{reference:pypath.legacy.main.PyPath.get_pathways}}\pysiglinewithargsret{\sphinxbfcode{\sphinxupquote{get\_pathways}}}{\emph{source}}{}
\end{fulllineitems}

\index{get\_proteomicsdb() (pypath.legacy.main.PyPath method)@\spxentry{get\_proteomicsdb()}\spxextra{pypath.legacy.main.PyPath method}}

\begin{fulllineitems}
\phantomsection\label{\detokenize{reference:pypath.legacy.main.PyPath.get_proteomicsdb}}\pysiglinewithargsret{\sphinxbfcode{\sphinxupquote{get\_proteomicsdb}}}{\emph{user}, \emph{passwd}, \emph{tissues=None}, \emph{pickle=None}}{}
\end{fulllineitems}

\index{get\_sub() (pypath.legacy.main.PyPath method)@\spxentry{get\_sub()}\spxextra{pypath.legacy.main.PyPath method}}

\begin{fulllineitems}
\phantomsection\label{\detokenize{reference:pypath.legacy.main.PyPath.get_sub}}\pysiglinewithargsret{\sphinxbfcode{\sphinxupquote{get\_sub}}}{\emph{crit}, \emph{andor='or'}, \emph{graph=None}}{}
Selects the nodes from \sphinxstyleemphasis{graph} (and edges to be removed)
according to a set of user-defined attributes.
\begin{quote}\begin{description}
\item[{Parameters}] \leavevmode\begin{itemize}
\item {} 
\sphinxstyleliteralstrong{\sphinxupquote{crit}} (\sphinxstyleliteralemphasis{\sphinxupquote{dict}}) \textendash{} Defines the critical attributes to generate the subnetwork.
Keys are \sphinxcode{\sphinxupquote{'edge'}} and \sphinxcode{\sphinxupquote{'node'}} and values are {[}dict{]}
containing the critical attribute names {[}str{]} and values
are {[}set{]} containing those attributes of the nodes/edges
that are to be kept.

\item {} 
\sphinxstyleliteralstrong{\sphinxupquote{andor}} (\sphinxstyleliteralemphasis{\sphinxupquote{str}}) \textendash{} Optional, \sphinxcode{\sphinxupquote{'or'}} by default. Determines the search mode.
See \sphinxcode{\sphinxupquote{pypath.main.PyPath.search\_attr\_or()}} and
\sphinxcode{\sphinxupquote{pypath.main.PyPath.search\_attr\_and()}} for more
details.

\item {} 
\sphinxstyleliteralstrong{\sphinxupquote{graph}} (\sphinxstyleliteralemphasis{\sphinxupquote{igraph.Graph}}) \textendash{} Optional, \sphinxcode{\sphinxupquote{None}} by default. The graph object where to
extract the subnetwork. If none is passed, takes the current
network (undirected) graph
(\sphinxcode{\sphinxupquote{pypath.main.PyPath.graph}}).

\end{itemize}

\item[{Returns}] \leavevmode
(\sphinxstyleemphasis{dict}) \textendash{} Keys are \sphinxcode{\sphinxupquote{'nodes'}} and \sphinxcode{\sphinxupquote{'edges'}} whose
values are {[}lst{]} of elements (as indexes {[}int{]}). Nodes are
those to be kept and edges to be removed on the extracted
subnetwork.

\end{description}\end{quote}

\end{fulllineitems}

\index{get\_taxon() (pypath.legacy.main.PyPath method)@\spxentry{get\_taxon()}\spxextra{pypath.legacy.main.PyPath method}}

\begin{fulllineitems}
\phantomsection\label{\detokenize{reference:pypath.legacy.main.PyPath.get_taxon}}\pysiglinewithargsret{\sphinxbfcode{\sphinxupquote{get\_taxon}}}{\emph{tax\_dict}, \emph{fields}}{}
\end{fulllineitems}

\index{go\_annotate\_graph() (pypath.legacy.main.PyPath method)@\spxentry{go\_annotate\_graph()}\spxextra{pypath.legacy.main.PyPath method}}

\begin{fulllineitems}
\phantomsection\label{\detokenize{reference:pypath.legacy.main.PyPath.go_annotate_graph}}\pysiglinewithargsret{\sphinxbfcode{\sphinxupquote{go\_annotate\_graph}}}{\emph{aspects=('C'}, \emph{'F'}, \emph{'P')}}{}
Annotates protein nodes with GO terms. In the \sphinxcode{\sphinxupquote{go}} vertex
attribute each node is annotated by a dict of sets where keys are
one letter codes of GO aspects and values are sets of GO accessions.

\end{fulllineitems}

\index{go\_enrichment() (pypath.legacy.main.PyPath method)@\spxentry{go\_enrichment()}\spxextra{pypath.legacy.main.PyPath method}}

\begin{fulllineitems}
\phantomsection\label{\detokenize{reference:pypath.legacy.main.PyPath.go_enrichment}}\pysiglinewithargsret{\sphinxbfcode{\sphinxupquote{go\_enrichment}}}{\emph{proteins=None}, \emph{aspect='P'}, \emph{alpha=0.05}, \emph{correction\_method='hommel'}, \emph{all\_proteins=None}}{}
Does not work at the moment because cfisher module should be
replaced with scipy.

\end{fulllineitems}

\index{gs() (pypath.legacy.main.PyPath method)@\spxentry{gs()}\spxextra{pypath.legacy.main.PyPath method}}

\begin{fulllineitems}
\phantomsection\label{\detokenize{reference:pypath.legacy.main.PyPath.gs}}\pysiglinewithargsret{\sphinxbfcode{\sphinxupquote{gs}}}{\emph{genesymbol}}{}
Returns \sphinxcode{\sphinxupquote{igraph.Vertex()}} object if the GeneSymbol
can be found in the default undirected network,
otherwise \sphinxcode{\sphinxupquote{None}}.
\begin{description}
\item[{@genesymbol}] \leavevmode{[}str{]}
GeneSymbol.

\end{description}

\end{fulllineitems}

\index{gs\_affected\_by() (pypath.legacy.main.PyPath method)@\spxentry{gs\_affected\_by()}\spxextra{pypath.legacy.main.PyPath method}}

\begin{fulllineitems}
\phantomsection\label{\detokenize{reference:pypath.legacy.main.PyPath.gs_affected_by}}\pysiglinewithargsret{\sphinxbfcode{\sphinxupquote{gs\_affected\_by}}}{\emph{genesymbol}}{}
\end{fulllineitems}

\index{gs\_affects() (pypath.legacy.main.PyPath method)@\spxentry{gs\_affects()}\spxextra{pypath.legacy.main.PyPath method}}

\begin{fulllineitems}
\phantomsection\label{\detokenize{reference:pypath.legacy.main.PyPath.gs_affects}}\pysiglinewithargsret{\sphinxbfcode{\sphinxupquote{gs\_affects}}}{\emph{genesymbol}}{}
\end{fulllineitems}

\index{gs\_edge() (pypath.legacy.main.PyPath method)@\spxentry{gs\_edge()}\spxextra{pypath.legacy.main.PyPath method}}

\begin{fulllineitems}
\phantomsection\label{\detokenize{reference:pypath.legacy.main.PyPath.gs_edge}}\pysiglinewithargsret{\sphinxbfcode{\sphinxupquote{gs\_edge}}}{\emph{source}, \emph{target}, \emph{directed=True}}{}
Returns \sphinxcode{\sphinxupquote{igraph.Edge}} object if an edge exist between
the 2 proteins, otherwise \sphinxcode{\sphinxupquote{None}}.
\begin{description}
\item[{@source}] \leavevmode{[}str{]}
GeneSymbol

\item[{@target}] \leavevmode{[}str{]}
GeneSymbol

\item[{@directed}] \leavevmode{[}bool{]}
To be passed to igraph.Graph.get\_eid()

\end{description}

\end{fulllineitems}

\index{gs\_in\_directed() (pypath.legacy.main.PyPath method)@\spxentry{gs\_in\_directed()}\spxextra{pypath.legacy.main.PyPath method}}

\begin{fulllineitems}
\phantomsection\label{\detokenize{reference:pypath.legacy.main.PyPath.gs_in_directed}}\pysiglinewithargsret{\sphinxbfcode{\sphinxupquote{gs\_in\_directed}}}{\emph{genesymbol}}{}
\end{fulllineitems}

\index{gs\_in\_undirected() (pypath.legacy.main.PyPath method)@\spxentry{gs\_in\_undirected()}\spxextra{pypath.legacy.main.PyPath method}}

\begin{fulllineitems}
\phantomsection\label{\detokenize{reference:pypath.legacy.main.PyPath.gs_in_undirected}}\pysiglinewithargsret{\sphinxbfcode{\sphinxupquote{gs\_in\_undirected}}}{\emph{genesymbol}}{}
\end{fulllineitems}

\index{gs\_inhibited\_by() (pypath.legacy.main.PyPath method)@\spxentry{gs\_inhibited\_by()}\spxextra{pypath.legacy.main.PyPath method}}

\begin{fulllineitems}
\phantomsection\label{\detokenize{reference:pypath.legacy.main.PyPath.gs_inhibited_by}}\pysiglinewithargsret{\sphinxbfcode{\sphinxupquote{gs\_inhibited\_by}}}{\emph{genesymbol}}{}
\end{fulllineitems}

\index{gs\_inhibits() (pypath.legacy.main.PyPath method)@\spxentry{gs\_inhibits()}\spxextra{pypath.legacy.main.PyPath method}}

\begin{fulllineitems}
\phantomsection\label{\detokenize{reference:pypath.legacy.main.PyPath.gs_inhibits}}\pysiglinewithargsret{\sphinxbfcode{\sphinxupquote{gs\_inhibits}}}{\emph{genesymbol}}{}
\end{fulllineitems}

\index{gs\_neighborhood() (pypath.legacy.main.PyPath method)@\spxentry{gs\_neighborhood()}\spxextra{pypath.legacy.main.PyPath method}}

\begin{fulllineitems}
\phantomsection\label{\detokenize{reference:pypath.legacy.main.PyPath.gs_neighborhood}}\pysiglinewithargsret{\sphinxbfcode{\sphinxupquote{gs\_neighborhood}}}{\emph{genesymbols}, \emph{order=1}, \emph{mode='ALL'}}{}
\end{fulllineitems}

\index{gs\_neighbors() (pypath.legacy.main.PyPath method)@\spxentry{gs\_neighbors()}\spxextra{pypath.legacy.main.PyPath method}}

\begin{fulllineitems}
\phantomsection\label{\detokenize{reference:pypath.legacy.main.PyPath.gs_neighbors}}\pysiglinewithargsret{\sphinxbfcode{\sphinxupquote{gs\_neighbors}}}{\emph{genesymbol}, \emph{mode='ALL'}}{}
\end{fulllineitems}

\index{gs\_stimulated\_by() (pypath.legacy.main.PyPath method)@\spxentry{gs\_stimulated\_by()}\spxextra{pypath.legacy.main.PyPath method}}

\begin{fulllineitems}
\phantomsection\label{\detokenize{reference:pypath.legacy.main.PyPath.gs_stimulated_by}}\pysiglinewithargsret{\sphinxbfcode{\sphinxupquote{gs\_stimulated\_by}}}{\emph{genesymbol}}{}
\end{fulllineitems}

\index{gs\_stimulates() (pypath.legacy.main.PyPath method)@\spxentry{gs\_stimulates()}\spxextra{pypath.legacy.main.PyPath method}}

\begin{fulllineitems}
\phantomsection\label{\detokenize{reference:pypath.legacy.main.PyPath.gs_stimulates}}\pysiglinewithargsret{\sphinxbfcode{\sphinxupquote{gs\_stimulates}}}{\emph{genesymbol}}{}
\end{fulllineitems}

\index{gss() (pypath.legacy.main.PyPath method)@\spxentry{gss()}\spxextra{pypath.legacy.main.PyPath method}}

\begin{fulllineitems}
\phantomsection\label{\detokenize{reference:pypath.legacy.main.PyPath.gss}}\pysiglinewithargsret{\sphinxbfcode{\sphinxupquote{gss}}}{\emph{genesymbols}}{}
\end{fulllineitems}

\index{guide2pharma() (pypath.legacy.main.PyPath method)@\spxentry{guide2pharma()}\spxextra{pypath.legacy.main.PyPath method}}

\begin{fulllineitems}
\phantomsection\label{\detokenize{reference:pypath.legacy.main.PyPath.guide2pharma}}\pysiglinewithargsret{\sphinxbfcode{\sphinxupquote{guide2pharma}}}{}{}
\end{fulllineitems}

\index{having\_attr() (pypath.legacy.main.PyPath method)@\spxentry{having\_attr()}\spxextra{pypath.legacy.main.PyPath method}}

\begin{fulllineitems}
\phantomsection\label{\detokenize{reference:pypath.legacy.main.PyPath.having_attr}}\pysiglinewithargsret{\sphinxbfcode{\sphinxupquote{having\_attr}}}{\emph{attr}, \emph{graph=None}, \emph{index=True}, \emph{edges=True}}{}
Checks if edges or nodes of the network have a specific
attribute and returns an iterator of the indices (or the
edge/node instances) of edges/nodes having such attribute.
\begin{quote}\begin{description}
\item[{Parameters}] \leavevmode\begin{itemize}
\item {} 
\sphinxstyleliteralstrong{\sphinxupquote{attr}} (\sphinxstyleliteralemphasis{\sphinxupquote{str}}) \textendash{} The name of the attribute to look for.

\item {} 
\sphinxstyleliteralstrong{\sphinxupquote{graph}} (\sphinxstyleliteralemphasis{\sphinxupquote{igraph.Graph}}) \textendash{} Optional, \sphinxcode{\sphinxupquote{None}} by default. The graph object where the
edge/node attribute is to be searched. If none is passed,
takes the undirected network of the current instance.

\item {} 
\sphinxstyleliteralstrong{\sphinxupquote{index}} (\sphinxstyleliteralemphasis{\sphinxupquote{bool}}) \textendash{} Optional, \sphinxcode{\sphinxupquote{True}} by default. Whether to return the
iterator of the indices or the node/edge instances.

\item {} 
\sphinxstyleliteralstrong{\sphinxupquote{edges}} (\sphinxstyleliteralemphasis{\sphinxupquote{bool}}) \textendash{} Optional, \sphinxcode{\sphinxupquote{True}} by default. Whether to look for the
attribute in the networks edges or nodes instead.

\end{itemize}

\item[{Returns}] \leavevmode
(\sphinxstyleemphasis{generator}) \textendash{} Generator object containing the edge/node
indices (or instances) having the specified attribute.

\end{description}\end{quote}

\end{fulllineitems}

\index{having\_eattr() (pypath.legacy.main.PyPath method)@\spxentry{having\_eattr()}\spxextra{pypath.legacy.main.PyPath method}}

\begin{fulllineitems}
\phantomsection\label{\detokenize{reference:pypath.legacy.main.PyPath.having_eattr}}\pysiglinewithargsret{\sphinxbfcode{\sphinxupquote{having\_eattr}}}{\emph{attr}, \emph{graph=None}, \emph{index=True}}{}
Checks if edges of the network have a specific attribute and
returns an iterator of the indices (or the edge instances) of
edges having such attribute.
\begin{quote}\begin{description}
\item[{Parameters}] \leavevmode\begin{itemize}
\item {} 
\sphinxstyleliteralstrong{\sphinxupquote{attr}} (\sphinxstyleliteralemphasis{\sphinxupquote{str}}) \textendash{} The name of the attribute to look for.

\item {} 
\sphinxstyleliteralstrong{\sphinxupquote{graph}} (\sphinxstyleliteralemphasis{\sphinxupquote{igraph.Graph}}) \textendash{} Optional, \sphinxcode{\sphinxupquote{None}} by default. The graph object where the
edge/node attribute is to be searched. If none is passed,
takes the undirected network of the current instance.

\item {} 
\sphinxstyleliteralstrong{\sphinxupquote{index}} (\sphinxstyleliteralemphasis{\sphinxupquote{bool}}) \textendash{} Optional, \sphinxcode{\sphinxupquote{True}} by default. Whether to return the
iterator of the indices or the node/edge instances.

\end{itemize}

\item[{Returns}] \leavevmode
(\sphinxstyleemphasis{generator}) \textendash{} Generator object containing the edge
indices (or instances) having the specified attribute.

\end{description}\end{quote}

\end{fulllineitems}

\index{having\_ptm() (pypath.legacy.main.PyPath method)@\spxentry{having\_ptm()}\spxextra{pypath.legacy.main.PyPath method}}

\begin{fulllineitems}
\phantomsection\label{\detokenize{reference:pypath.legacy.main.PyPath.having_ptm}}\pysiglinewithargsret{\sphinxbfcode{\sphinxupquote{having\_ptm}}}{\emph{index=True}, \emph{graph=None}}{}
Checks if edges of the network have the \sphinxcode{\sphinxupquote{'ptm'}} attribute and
returns an iterator of the indices (or the edge instances) of
edges having such attribute.
\begin{quote}\begin{description}
\item[{Parameters}] \leavevmode\begin{itemize}
\item {} 
\sphinxstyleliteralstrong{\sphinxupquote{index}} (\sphinxstyleliteralemphasis{\sphinxupquote{bool}}) \textendash{} Optional, \sphinxcode{\sphinxupquote{True}} by default. Whether to return the
iterator of the indices or the node/edge instances.

\item {} 
\sphinxstyleliteralstrong{\sphinxupquote{graph}} (\sphinxstyleliteralemphasis{\sphinxupquote{igraph.Graph}}) \textendash{} Optional, \sphinxcode{\sphinxupquote{None}} by default. The graph object where the
edge/node attribute is to be searched. If none is passed,
takes the undirected network of the current instance.

\end{itemize}

\item[{Returns}] \leavevmode
(\sphinxstyleemphasis{generator}) \textendash{} Generator object containing the edge
indices (or instances) having the \sphinxcode{\sphinxupquote{ptm'{'}}} attribute.

\end{description}\end{quote}

\end{fulllineitems}

\index{having\_vattr() (pypath.legacy.main.PyPath method)@\spxentry{having\_vattr()}\spxextra{pypath.legacy.main.PyPath method}}

\begin{fulllineitems}
\phantomsection\label{\detokenize{reference:pypath.legacy.main.PyPath.having_vattr}}\pysiglinewithargsret{\sphinxbfcode{\sphinxupquote{having\_vattr}}}{\emph{attr}, \emph{graph=None}, \emph{index=True}}{}
Checks if nodes of the network have a specific attribute and
returns an iterator of the indices (or the node instances) of
nodes having such attribute.
\begin{quote}\begin{description}
\item[{Parameters}] \leavevmode\begin{itemize}
\item {} 
\sphinxstyleliteralstrong{\sphinxupquote{attr}} (\sphinxstyleliteralemphasis{\sphinxupquote{str}}) \textendash{} The name of the attribute to look for.

\item {} 
\sphinxstyleliteralstrong{\sphinxupquote{graph}} (\sphinxstyleliteralemphasis{\sphinxupquote{igraph.Graph}}) \textendash{} Optional, \sphinxcode{\sphinxupquote{None}} by default. The graph object where the
edge/node attribute is to be searched. If none is passed,
takes the undirected network of the current instance.

\item {} 
\sphinxstyleliteralstrong{\sphinxupquote{index}} (\sphinxstyleliteralemphasis{\sphinxupquote{bool}}) \textendash{} Optional, \sphinxcode{\sphinxupquote{True}} by default. Whether to return the
iterator of the indices or the node/edge instances.

\end{itemize}

\item[{Returns}] \leavevmode
(\sphinxstyleemphasis{generator}) \textendash{} Generator object containing the node
indices (or instances) having the specified attribute.

\end{description}\end{quote}

\end{fulllineitems}

\index{homology\_translation() (pypath.legacy.main.PyPath method)@\spxentry{homology\_translation()}\spxextra{pypath.legacy.main.PyPath method}}

\begin{fulllineitems}
\phantomsection\label{\detokenize{reference:pypath.legacy.main.PyPath.homology_translation}}\pysiglinewithargsret{\sphinxbfcode{\sphinxupquote{homology\_translation}}}{\emph{target}, \emph{source=None}, \emph{only\_swissprot=True}, \emph{graph=None}}{}
Translates the current object to another organism by orthology.
Proteins without known ortholog will be deleted.
\begin{quote}\begin{description}
\item[{Parameters}] \leavevmode
\sphinxstyleliteralstrong{\sphinxupquote{target}} (\sphinxstyleliteralemphasis{\sphinxupquote{int}}) \textendash{} NCBI Taxonomy ID of the target organism. E.g. 10090 for mouse.

\end{description}\end{quote}

\end{fulllineitems}

\index{htp\_stats() (pypath.legacy.main.PyPath method)@\spxentry{htp\_stats()}\spxextra{pypath.legacy.main.PyPath method}}

\begin{fulllineitems}
\phantomsection\label{\detokenize{reference:pypath.legacy.main.PyPath.htp_stats}}\pysiglinewithargsret{\sphinxbfcode{\sphinxupquote{htp\_stats}}}{}{}
\end{fulllineitems}

\index{in\_complex() (pypath.legacy.main.PyPath method)@\spxentry{in\_complex()}\spxextra{pypath.legacy.main.PyPath method}}

\begin{fulllineitems}
\phantomsection\label{\detokenize{reference:pypath.legacy.main.PyPath.in_complex}}\pysiglinewithargsret{\sphinxbfcode{\sphinxupquote{in\_complex}}}{\emph{csources={[}'corum'{]}}}{}
Deprecated, will be removed.

\end{fulllineitems}

\index{in\_directed() (pypath.legacy.main.PyPath method)@\spxentry{in\_directed()}\spxextra{pypath.legacy.main.PyPath method}}

\begin{fulllineitems}
\phantomsection\label{\detokenize{reference:pypath.legacy.main.PyPath.in_directed}}\pysiglinewithargsret{\sphinxbfcode{\sphinxupquote{in\_directed}}}{\emph{vertex}}{}
\end{fulllineitems}

\index{in\_undirected() (pypath.legacy.main.PyPath method)@\spxentry{in\_undirected()}\spxextra{pypath.legacy.main.PyPath method}}

\begin{fulllineitems}
\phantomsection\label{\detokenize{reference:pypath.legacy.main.PyPath.in_undirected}}\pysiglinewithargsret{\sphinxbfcode{\sphinxupquote{in\_undirected}}}{\emph{vertex}}{}
\end{fulllineitems}

\index{info() (pypath.legacy.main.PyPath method)@\spxentry{info()}\spxextra{pypath.legacy.main.PyPath method}}

\begin{fulllineitems}
\phantomsection\label{\detokenize{reference:pypath.legacy.main.PyPath.info}}\pysiglinewithargsret{\sphinxbfcode{\sphinxupquote{info}}}{\emph{name}}{}
Given the name of a resource, prints out the information about
that source/database. You can check the list of available
resource descriptions in
\sphinxcode{\sphinxupquote{ypath.descriptions.descriptions.keys()}}.
\begin{quote}\begin{description}
\item[{Parameters}] \leavevmode
\sphinxstyleliteralstrong{\sphinxupquote{name}} (\sphinxstyleliteralemphasis{\sphinxupquote{str}}) \textendash{} The name of the resource from which to print the
information.

\end{description}\end{quote}

\end{fulllineitems}

\index{init\_complex\_attr() (pypath.legacy.main.PyPath method)@\spxentry{init\_complex\_attr()}\spxextra{pypath.legacy.main.PyPath method}}

\begin{fulllineitems}
\phantomsection\label{\detokenize{reference:pypath.legacy.main.PyPath.init_complex_attr}}\pysiglinewithargsret{\sphinxbfcode{\sphinxupquote{init\_complex\_attr}}}{\emph{graph}, \emph{name}}{}
\end{fulllineitems}

\index{init\_edge\_attr() (pypath.legacy.main.PyPath method)@\spxentry{init\_edge\_attr()}\spxextra{pypath.legacy.main.PyPath method}}

\begin{fulllineitems}
\phantomsection\label{\detokenize{reference:pypath.legacy.main.PyPath.init_edge_attr}}\pysiglinewithargsret{\sphinxbfcode{\sphinxupquote{init\_edge\_attr}}}{\emph{attr}}{}
Fills all edges attribute \sphinxstyleemphasis{attr} with its default type (if
such attribute value is \sphinxcode{\sphinxupquote{None}}), creates {[}list{]} if in
\sphinxcode{\sphinxupquote{pypath.main.PyPath.edgeAttrs}} such attribute is
registered as {[}list{]}.
\begin{quote}\begin{description}
\item[{Parameters}] \leavevmode
\sphinxstyleliteralstrong{\sphinxupquote{attr}} (\sphinxstyleliteralemphasis{\sphinxupquote{str}}) \textendash{} The attribute name to be initialized on the network edges.

\end{description}\end{quote}

\end{fulllineitems}

\index{init\_gsea() (pypath.legacy.main.PyPath method)@\spxentry{init\_gsea()}\spxextra{pypath.legacy.main.PyPath method}}

\begin{fulllineitems}
\phantomsection\label{\detokenize{reference:pypath.legacy.main.PyPath.init_gsea}}\pysiglinewithargsret{\sphinxbfcode{\sphinxupquote{init\_gsea}}}{\emph{user}}{}
Initializes a \sphinxcode{\sphinxupquote{pypath.gsea.GSEA}} object and shows the list of the
collections in MSigDB.

\end{fulllineitems}

\index{init\_network() (pypath.legacy.main.PyPath method)@\spxentry{init\_network()}\spxextra{pypath.legacy.main.PyPath method}}

\begin{fulllineitems}
\phantomsection\label{\detokenize{reference:pypath.legacy.main.PyPath.init_network}}\pysiglinewithargsret{\sphinxbfcode{\sphinxupquote{init\_network}}}{\emph{lst=None}, \emph{exclude={[}{]}}, \emph{cache\_files=\{\}}, \emph{pickle\_file=None}, \emph{pfile=False}, \emph{save=False}, \emph{reread=None}, \emph{redownload=None}, \emph{keep\_raw=False}, \emph{**kwargs}}{}
Loads the network data.

This is a lazy way to start the module, load data and build the
high confidence, literature curated part of the signaling
network.
\begin{quote}\begin{description}
\item[{Parameters}] \leavevmode\begin{itemize}
\item {} 
\sphinxstyleliteralstrong{\sphinxupquote{lst}} (\sphinxstyleliteralemphasis{\sphinxupquote{dict}}) \textendash{} Optional, \sphinxcode{\sphinxupquote{None}} by default. Specifies the data input
formats for the different resources (keys) {[}str{]}. Values
are \sphinxcode{\sphinxupquote{pypath.input\_formats.NetworkInput}} instances
containing the information. By default uses the set of
resources of OmniPath.

\item {} 
\sphinxstyleliteralstrong{\sphinxupquote{exclude}} (\sphinxstyleliteralemphasis{\sphinxupquote{list}}) \textendash{} Optional, \sphinxcode{\sphinxupquote{{[}{]}}} by default. List of resources {[}str{]} to
exclude from the network.

\item {} 
\sphinxstyleliteralstrong{\sphinxupquote{cache\_files}} (\sphinxstyleliteralemphasis{\sphinxupquote{dict}}) \textendash{} Optional, \sphinxcode{\sphinxupquote{\{\}}} by default. Contains the resource name(s)
{[}str{]} (keys) and the corresponding cached file name {[}str{]}.
If provided (and file exists) bypasses the download of the
data for that resource and uses the cache file instead.

\item {} 
\sphinxstyleliteralstrong{\sphinxupquote{pfile}} (\sphinxstyleliteralemphasis{\sphinxupquote{str}}) \textendash{} Optional, \sphinxcode{\sphinxupquote{False}} by default. If any, provides the file
name or path to a previously saved network pickle file.
If \sphinxcode{\sphinxupquote{True}} is passed, takes the default path from
{\hyperref[\detokenize{reference:pypath.legacy.main.PyPath.save_network}]{\sphinxcrossref{\sphinxcode{\sphinxupquote{PyPath.save\_network()}}}}}
(\sphinxcode{\sphinxupquote{'cache/default\_network.pickle'}}).

\item {} 
\sphinxstyleliteralstrong{\sphinxupquote{save}} (\sphinxstyleliteralemphasis{\sphinxupquote{bool}}) \textendash{} Optional, \sphinxcode{\sphinxupquote{False}} by default. If set to \sphinxcode{\sphinxupquote{True}}, saves
the loaded network to its default location
(\sphinxcode{\sphinxupquote{'cache/default\_network.pickle'}}).

\item {} 
\sphinxstyleliteralstrong{\sphinxupquote{reread}} (\sphinxstyleliteralemphasis{\sphinxupquote{bool}}) \textendash{} Optional, \sphinxcode{\sphinxupquote{False}} by default. Specifies whether to reread
the data files from the cache or omit them (similar to
\sphinxstyleemphasis{redownload}).

\item {} 
\sphinxstyleliteralstrong{\sphinxupquote{redownload}} (\sphinxstyleliteralemphasis{\sphinxupquote{bool}}) \textendash{} Optional, \sphinxcode{\sphinxupquote{False}} by default. Specifies whether to
re-download the data and ignore the cache.

\item {} 
\sphinxstyleliteralstrong{\sphinxupquote{**kwargs}} \textendash{} Not used.

\end{itemize}

\end{description}\end{quote}

\end{fulllineitems}

\index{init\_vertex\_attr() (pypath.legacy.main.PyPath method)@\spxentry{init\_vertex\_attr()}\spxextra{pypath.legacy.main.PyPath method}}

\begin{fulllineitems}
\phantomsection\label{\detokenize{reference:pypath.legacy.main.PyPath.init_vertex_attr}}\pysiglinewithargsret{\sphinxbfcode{\sphinxupquote{init\_vertex\_attr}}}{\emph{attr}}{}
Fills all vertices attribute \sphinxstyleemphasis{attr} with its default type (if
such attribute value is \sphinxcode{\sphinxupquote{None}}), creates {[}list{]} if in
\sphinxcode{\sphinxupquote{pypath.main.PyPath.vertexAttrs}} such attribute is
registered as {[}list{]}.
\begin{quote}\begin{description}
\item[{Parameters}] \leavevmode
\sphinxstyleliteralstrong{\sphinxupquote{attr}} (\sphinxstyleliteralemphasis{\sphinxupquote{str}}) \textendash{} The attribute name to be initialized on the network
vertices.

\end{description}\end{quote}

\end{fulllineitems}

\index{interactions\_all() (pypath.legacy.main.PyPath method)@\spxentry{interactions\_all()}\spxextra{pypath.legacy.main.PyPath method}}

\begin{fulllineitems}
\phantomsection\label{\detokenize{reference:pypath.legacy.main.PyPath.interactions_all}}\pysiglinewithargsret{\sphinxbfcode{\sphinxupquote{interactions\_all}}}{\emph{resources=None}, \emph{**kwargs}}{}
Returns a \sphinxstyleemphasis{set} of tuples of node name pairs representing
interactions, both directed and undirected. Directed interactions
will be present according to their direction, mutual directed
interactions are represented by two tuples. The directed and
undirected interactions are not distinguished here.

\end{fulllineitems}

\index{interactions\_directed() (pypath.legacy.main.PyPath method)@\spxentry{interactions\_directed()}\spxextra{pypath.legacy.main.PyPath method}}

\begin{fulllineitems}
\phantomsection\label{\detokenize{reference:pypath.legacy.main.PyPath.interactions_directed}}\pysiglinewithargsret{\sphinxbfcode{\sphinxupquote{interactions\_directed}}}{\emph{resources=None}, \emph{**kwargs}}{}
Returns a \sphinxstyleemphasis{set} of tuples of node name pairs with being aware
of the directions.
Undirected interactions will be discarded.
Pairs of node names represent the directions: first is the source,
second is the target.

\end{fulllineitems}

\index{interactions\_directed\_by\_resource() (pypath.legacy.main.PyPath method)@\spxentry{interactions\_directed\_by\_resource()}\spxextra{pypath.legacy.main.PyPath method}}

\begin{fulllineitems}
\phantomsection\label{\detokenize{reference:pypath.legacy.main.PyPath.interactions_directed_by_resource}}\pysiglinewithargsret{\sphinxbfcode{\sphinxupquote{interactions\_directed\_by\_resource}}}{\emph{resources=None}, \emph{effect=None}, \emph{**kwargs}}{}
Returns a \sphinxstyleemphasis{dict} of {\color{red}\bfseries{}*}set*s of tuples of node name pairs with being
aware of the directions.
Undirected interactions will be discarded.
Pairs of node names represent the directions: first is the source,
second is the target.

\end{fulllineitems}

\index{interactions\_inhibitory() (pypath.legacy.main.PyPath method)@\spxentry{interactions\_inhibitory()}\spxextra{pypath.legacy.main.PyPath method}}

\begin{fulllineitems}
\phantomsection\label{\detokenize{reference:pypath.legacy.main.PyPath.interactions_inhibitory}}\pysiglinewithargsret{\sphinxbfcode{\sphinxupquote{interactions\_inhibitory}}}{\emph{resources=None}, \emph{**kwargs}}{}
Returns a \sphinxstyleemphasis{set} of tuples of node name pairs only for inhibitory
interactions.
Pairs of node names represent the directions: first is the source,
second is the target.

\end{fulllineitems}

\index{interactions\_inhibitory\_by\_resource() (pypath.legacy.main.PyPath method)@\spxentry{interactions\_inhibitory\_by\_resource()}\spxextra{pypath.legacy.main.PyPath method}}

\begin{fulllineitems}
\phantomsection\label{\detokenize{reference:pypath.legacy.main.PyPath.interactions_inhibitory_by_resource}}\pysiglinewithargsret{\sphinxbfcode{\sphinxupquote{interactions\_inhibitory\_by\_resource}}}{\emph{resources=None}, \emph{**kwargs}}{}
Returns a \sphinxstyleemphasis{dict} of {\color{red}\bfseries{}*}set*s of tuples of node name pairs with being
aware of the directions.
Undirected interactions will be discarded.
Pairs of node names represent the directions: first is the source,
second is the target.

\end{fulllineitems}

\index{interactions\_mutual() (pypath.legacy.main.PyPath method)@\spxentry{interactions\_mutual()}\spxextra{pypath.legacy.main.PyPath method}}

\begin{fulllineitems}
\phantomsection\label{\detokenize{reference:pypath.legacy.main.PyPath.interactions_mutual}}\pysiglinewithargsret{\sphinxbfcode{\sphinxupquote{interactions\_mutual}}}{\emph{resources=None}, \emph{**kwargs}}{}
Returns a \sphinxstyleemphasis{set} of tuples of node name pairs representing
mutual interactions (i.e. A\textendash{}\textgreater{}B and B\textendash{}\textgreater{}A).
Pairs of node names will be sorted alphabetically.

\end{fulllineitems}

\index{interactions\_mutual\_by\_resource() (pypath.legacy.main.PyPath method)@\spxentry{interactions\_mutual\_by\_resource()}\spxextra{pypath.legacy.main.PyPath method}}

\begin{fulllineitems}
\phantomsection\label{\detokenize{reference:pypath.legacy.main.PyPath.interactions_mutual_by_resource}}\pysiglinewithargsret{\sphinxbfcode{\sphinxupquote{interactions\_mutual\_by\_resource}}}{\emph{resources=None}, \emph{**kwargs}}{}
Returns a \sphinxstyleemphasis{dict} of {\color{red}\bfseries{}*}set*s of tuples of node name pairs representing
mutual interactions (i.e. A\textendash{}\textgreater{}B and B\textendash{}\textgreater{}A).
Pairs of node names will be sorted alphabetically.

\end{fulllineitems}

\index{interactions\_signed() (pypath.legacy.main.PyPath method)@\spxentry{interactions\_signed()}\spxextra{pypath.legacy.main.PyPath method}}

\begin{fulllineitems}
\phantomsection\label{\detokenize{reference:pypath.legacy.main.PyPath.interactions_signed}}\pysiglinewithargsret{\sphinxbfcode{\sphinxupquote{interactions\_signed}}}{\emph{resources=None}, \emph{**kwargs}}{}
Returns a \sphinxstyleemphasis{set} of tuples of node name pairs only for signed
interactions.
Pairs of node names represent the directions: first is the source,
second is the target.

\end{fulllineitems}

\index{interactions\_signed\_by\_resource() (pypath.legacy.main.PyPath method)@\spxentry{interactions\_signed\_by\_resource()}\spxextra{pypath.legacy.main.PyPath method}}

\begin{fulllineitems}
\phantomsection\label{\detokenize{reference:pypath.legacy.main.PyPath.interactions_signed_by_resource}}\pysiglinewithargsret{\sphinxbfcode{\sphinxupquote{interactions\_signed\_by\_resource}}}{\emph{resources=None}, \emph{**kwargs}}{}
Returns a \sphinxstyleemphasis{dict} of {\color{red}\bfseries{}*}set*s of tuples of node name pairs with being
aware of the directions.
Undirected interactions will be discarded.
Pairs of node names represent the directions: first is the source,
second is the target.

\end{fulllineitems}

\index{interactions\_stimulatory() (pypath.legacy.main.PyPath method)@\spxentry{interactions\_stimulatory()}\spxextra{pypath.legacy.main.PyPath method}}

\begin{fulllineitems}
\phantomsection\label{\detokenize{reference:pypath.legacy.main.PyPath.interactions_stimulatory}}\pysiglinewithargsret{\sphinxbfcode{\sphinxupquote{interactions\_stimulatory}}}{\emph{resources=None}, \emph{**kwargs}}{}
Returns a \sphinxstyleemphasis{set} of tuples of node name pairs only for stimulatory
interactions.
Pairs of node names represent the directions: first is the source,
second is the target.

\end{fulllineitems}

\index{interactions\_stimulatory\_by\_resource() (pypath.legacy.main.PyPath method)@\spxentry{interactions\_stimulatory\_by\_resource()}\spxextra{pypath.legacy.main.PyPath method}}

\begin{fulllineitems}
\phantomsection\label{\detokenize{reference:pypath.legacy.main.PyPath.interactions_stimulatory_by_resource}}\pysiglinewithargsret{\sphinxbfcode{\sphinxupquote{interactions\_stimulatory\_by\_resource}}}{\emph{resources=None}, \emph{**kwargs}}{}
Returns a \sphinxstyleemphasis{dict} of {\color{red}\bfseries{}*}set*s of tuples of node name pairs with being
aware of the directions.
Undirected interactions will be discarded.
Pairs of node names represent the directions: first is the source,
second is the target.

\end{fulllineitems}

\index{interactions\_undirected() (pypath.legacy.main.PyPath method)@\spxentry{interactions\_undirected()}\spxextra{pypath.legacy.main.PyPath method}}

\begin{fulllineitems}
\phantomsection\label{\detokenize{reference:pypath.legacy.main.PyPath.interactions_undirected}}\pysiglinewithargsret{\sphinxbfcode{\sphinxupquote{interactions\_undirected}}}{\emph{resources=None}, \emph{**kwargs}}{}
Returns a \sphinxstyleemphasis{set} of tuples of node name pairs without being aware
of the directions.
Pairs of node names will be sorted alphabetically.

\end{fulllineitems}

\index{interactions\_undirected\_by\_resource() (pypath.legacy.main.PyPath method)@\spxentry{interactions\_undirected\_by\_resource()}\spxextra{pypath.legacy.main.PyPath method}}

\begin{fulllineitems}
\phantomsection\label{\detokenize{reference:pypath.legacy.main.PyPath.interactions_undirected_by_resource}}\pysiglinewithargsret{\sphinxbfcode{\sphinxupquote{interactions\_undirected\_by\_resource}}}{\emph{resources=None}, \emph{**kwargs}}{}
Returns a \sphinxstyleemphasis{dict} of {\color{red}\bfseries{}*}set*s of tuples of node name pairs without
being aware of the directions.
Pairs of node names will be sorted alphabetically.

\end{fulllineitems}

\index{intergroup\_shortest\_paths() (pypath.legacy.main.PyPath method)@\spxentry{intergroup\_shortest\_paths()}\spxextra{pypath.legacy.main.PyPath method}}

\begin{fulllineitems}
\phantomsection\label{\detokenize{reference:pypath.legacy.main.PyPath.intergroup_shortest_paths}}\pysiglinewithargsret{\sphinxbfcode{\sphinxupquote{intergroup\_shortest\_paths}}}{\emph{groupA}, \emph{groupB}, \emph{random=False}}{}
\end{fulllineitems}

\index{intogen\_cancer\_drivers\_list() (pypath.legacy.main.PyPath method)@\spxentry{intogen\_cancer\_drivers\_list()}\spxextra{pypath.legacy.main.PyPath method}}

\begin{fulllineitems}
\phantomsection\label{\detokenize{reference:pypath.legacy.main.PyPath.intogen_cancer_drivers_list}}\pysiglinewithargsret{\sphinxbfcode{\sphinxupquote{intogen\_cancer\_drivers\_list}}}{\emph{intogen\_file}}{}
Loads the list of cancer driver proteins from IntOGen data.
\begin{quote}\begin{description}
\item[{Parameters}] \leavevmode
\sphinxstyleliteralstrong{\sphinxupquote{intogen\_file}} (\sphinxstyleliteralemphasis{\sphinxupquote{str}}) \textendash{} Path to the data file. Can also be {[}function{]} that provides
the data. In general, anything accepted by
\sphinxcode{\sphinxupquote{pypath.input\_formats.NetworkInput.input}}.

\end{description}\end{quote}

\end{fulllineitems}

\index{iter\_edges() (pypath.legacy.main.PyPath method)@\spxentry{iter\_edges()}\spxextra{pypath.legacy.main.PyPath method}}

\begin{fulllineitems}
\phantomsection\label{\detokenize{reference:pypath.legacy.main.PyPath.iter_edges}}\pysiglinewithargsret{\sphinxbfcode{\sphinxupquote{iter\_edges}}}{\emph{resources=None}}{}
Iterates the edges in the graph optionally limited to certain
resources. Yields \sphinxcode{\sphinxupquote{igraph.Edge}} objects.

\end{fulllineitems}

\index{iter\_interactions() (pypath.legacy.main.PyPath method)@\spxentry{iter\_interactions()}\spxextra{pypath.legacy.main.PyPath method}}

\begin{fulllineitems}
\phantomsection\label{\detokenize{reference:pypath.legacy.main.PyPath.iter_interactions}}\pysiglinewithargsret{\sphinxbfcode{\sphinxupquote{iter\_interactions}}}{\emph{signs=True}, \emph{all\_undirected=True}, \emph{by\_source=False}, \emph{with\_references=False}}{}
Iterates over edges and yields interaction records.
\begin{quote}\begin{description}
\item[{Parameters}] \leavevmode\begin{itemize}
\item {} 
\sphinxstyleliteralstrong{\sphinxupquote{signs}} (\sphinxstyleliteralemphasis{\sphinxupquote{bool}}) \textendash{} Ignoring signs if \sphinxcode{\sphinxupquote{False}}. This way each directed interaction
will yield a single record even if it’s ambiguously labeled
both positive and negative. The default behaviour is to yield
two records in this case, one with positive and one with negative
sign.

\item {} 
\sphinxstyleliteralstrong{\sphinxupquote{all\_undirected}} (\sphinxstyleliteralemphasis{\sphinxupquote{bool}}) \textendash{} Yield records for undirected interactions even if certain sources
provide direction. If \sphinxcode{\sphinxupquote{False}} only the directed records will
be provided and the undirected sources and references will be
added to the directed records.

\item {} 
\sphinxstyleliteralstrong{\sphinxupquote{by\_source}} (\sphinxstyleliteralemphasis{\sphinxupquote{bool}}) \textendash{} Yield separate records by resources. This way the node pairs
will be redundant and you need to group later if you want
unique interacting pairs. By default is \sphinxcode{\sphinxupquote{False}} because for
most applications unique interactions are preferred.
If \sphinxcode{\sphinxupquote{False}} the \sphinxstyleemphasis{refrences} field will still be present
but with \sphinxcode{\sphinxupquote{None}} values.

\item {} 
\sphinxstyleliteralstrong{\sphinxupquote{with\_references}} (\sphinxstyleliteralemphasis{\sphinxupquote{bool}}) \textendash{} Include the literature references. By default is \sphinxcode{\sphinxupquote{False}}
because you rarely need these and they increase the data size
significantly.

\end{itemize}

\end{description}\end{quote}

\end{fulllineitems}

\index{jaccard\_edges() (pypath.legacy.main.PyPath method)@\spxentry{jaccard\_edges()}\spxextra{pypath.legacy.main.PyPath method}}

\begin{fulllineitems}
\phantomsection\label{\detokenize{reference:pypath.legacy.main.PyPath.jaccard_edges}}\pysiglinewithargsret{\sphinxbfcode{\sphinxupquote{jaccard\_edges}}}{}{}
Computes the Jaccard similarity index between the sets of first
neighbours of all node pairs. \sphinxstylestrong{NOTE:} this method can take a
while to compute, e.g.: if the network has 10K nodes, the total
number of possible pairs to compute is:
\begin{equation*}
\begin{split}\binom{10^4}{2} = 49995000\end{split}
\end{equation*}\begin{quote}\begin{description}
\item[{Returns}] \leavevmode
(\sphinxstyleemphasis{list}) \textendash{} Large list of {[}tuple{]} elements containing the
node pair names {[}str{]} and their corresponding first
neighbours Jaccard index {[}float{]}.

\end{description}\end{quote}

\end{fulllineitems}

\index{jaccard\_meta() (pypath.legacy.main.PyPath method)@\spxentry{jaccard\_meta()}\spxextra{pypath.legacy.main.PyPath method}}

\begin{fulllineitems}
\phantomsection\label{\detokenize{reference:pypath.legacy.main.PyPath.jaccard_meta}}\pysiglinewithargsret{\sphinxbfcode{\sphinxupquote{jaccard\_meta}}}{\emph{jedges}, \emph{critical}}{}
Creates a (undirected) graph from a list of edges filtering by
their Jaccard index.
\begin{quote}\begin{description}
\item[{Parameters}] \leavevmode\begin{itemize}
\item {} 
\sphinxstyleliteralstrong{\sphinxupquote{jedges}} (\sphinxstyleliteralemphasis{\sphinxupquote{list}}) \textendash{} List of {[}tuple{]} containing the edges node names {[}str{]} and
their Jaccard index. Basically, the output of
\sphinxcode{\sphinxupquote{pypath.main.PyPath.jaccard\_edges()}}.

\item {} 
\sphinxstyleliteralstrong{\sphinxupquote{critical}} (\sphinxstyleliteralemphasis{\sphinxupquote{float}}) \textendash{} Specifies the threshold of the Jaccard index from above
which an edge will be included in the graph.

\end{itemize}

\item[{Returns}] \leavevmode
(\sphinxstyleemphasis{igraph.Graph}) \textendash{} The Undirected graph instance containing
only the edges whose Jaccard similarity index is above the
threshold specified by \sphinxstyleemphasis{critical}.

\end{description}\end{quote}

\end{fulllineitems}

\index{kegg\_directions() (pypath.legacy.main.PyPath method)@\spxentry{kegg\_directions()}\spxextra{pypath.legacy.main.PyPath method}}

\begin{fulllineitems}
\phantomsection\label{\detokenize{reference:pypath.legacy.main.PyPath.kegg_directions}}\pysiglinewithargsret{\sphinxbfcode{\sphinxupquote{kegg\_directions}}}{\emph{graph=None}}{}
\end{fulllineitems}

\index{kegg\_pathways() (pypath.legacy.main.PyPath method)@\spxentry{kegg\_pathways()}\spxextra{pypath.legacy.main.PyPath method}}

\begin{fulllineitems}
\phantomsection\label{\detokenize{reference:pypath.legacy.main.PyPath.kegg_pathways}}\pysiglinewithargsret{\sphinxbfcode{\sphinxupquote{kegg\_pathways}}}{\emph{graph=None}}{}
\end{fulllineitems}

\index{kinase\_stats() (pypath.legacy.main.PyPath method)@\spxentry{kinase\_stats()}\spxextra{pypath.legacy.main.PyPath method}}

\begin{fulllineitems}
\phantomsection\label{\detokenize{reference:pypath.legacy.main.PyPath.kinase_stats}}\pysiglinewithargsret{\sphinxbfcode{\sphinxupquote{kinase\_stats}}}{}{}
\end{fulllineitems}

\index{label() (pypath.legacy.main.PyPath method)@\spxentry{label()}\spxextra{pypath.legacy.main.PyPath method}}

\begin{fulllineitems}
\phantomsection\label{\detokenize{reference:pypath.legacy.main.PyPath.label}}\pysiglinewithargsret{\sphinxbfcode{\sphinxupquote{label}}}{\emph{label}, \emph{idx}, \emph{what='vertices'}}{}
Creates a boolean attribute \sphinxcode{\sphinxupquote{label}} True for the
vertex or edge IDs in the set \sphinxcode{\sphinxupquote{idx}}.

\end{fulllineitems}

\index{label\_by\_go() (pypath.legacy.main.PyPath method)@\spxentry{label\_by\_go()}\spxextra{pypath.legacy.main.PyPath method}}

\begin{fulllineitems}
\phantomsection\label{\detokenize{reference:pypath.legacy.main.PyPath.label_by_go}}\pysiglinewithargsret{\sphinxbfcode{\sphinxupquote{label\_by\_go}}}{\emph{label}, \emph{go\_terms}, \emph{method='ANY'}}{}
Assigns a boolean vertex attribute to nodes which tells whether
the node is annotated by all or any of the GO terms.

\end{fulllineitems}

\index{label\_edges() (pypath.legacy.main.PyPath method)@\spxentry{label\_edges()}\spxextra{pypath.legacy.main.PyPath method}}

\begin{fulllineitems}
\phantomsection\label{\detokenize{reference:pypath.legacy.main.PyPath.label_edges}}\pysiglinewithargsret{\sphinxbfcode{\sphinxupquote{label\_edges}}}{\emph{label}, \emph{edges}}{}
Creates a boolean edge attribute \sphinxcode{\sphinxupquote{label}} True for the
edge IDs in the set \sphinxcode{\sphinxupquote{edges}}.

\end{fulllineitems}

\index{label\_vertices() (pypath.legacy.main.PyPath method)@\spxentry{label\_vertices()}\spxextra{pypath.legacy.main.PyPath method}}

\begin{fulllineitems}
\phantomsection\label{\detokenize{reference:pypath.legacy.main.PyPath.label_vertices}}\pysiglinewithargsret{\sphinxbfcode{\sphinxupquote{label\_vertices}}}{\emph{label}, \emph{vertices}}{}
Creates a boolean vertex attribute \sphinxcode{\sphinxupquote{label}} True for the
vertex IDs in the set \sphinxcode{\sphinxupquote{vertices}}.

\end{fulllineitems}

\index{laudanna\_directions() (pypath.legacy.main.PyPath method)@\spxentry{laudanna\_directions()}\spxextra{pypath.legacy.main.PyPath method}}

\begin{fulllineitems}
\phantomsection\label{\detokenize{reference:pypath.legacy.main.PyPath.laudanna_directions}}\pysiglinewithargsret{\sphinxbfcode{\sphinxupquote{laudanna\_directions}}}{\emph{graph=None}}{}
\end{fulllineitems}

\index{laudanna\_effects() (pypath.legacy.main.PyPath method)@\spxentry{laudanna\_effects()}\spxextra{pypath.legacy.main.PyPath method}}

\begin{fulllineitems}
\phantomsection\label{\detokenize{reference:pypath.legacy.main.PyPath.laudanna_effects}}\pysiglinewithargsret{\sphinxbfcode{\sphinxupquote{laudanna\_effects}}}{\emph{graph=None}}{}
\end{fulllineitems}

\index{license() (pypath.legacy.main.PyPath static method)@\spxentry{license()}\spxextra{pypath.legacy.main.PyPath static method}}

\begin{fulllineitems}
\phantomsection\label{\detokenize{reference:pypath.legacy.main.PyPath.license}}\pysiglinewithargsret{\sphinxbfcode{\sphinxupquote{static }}\sphinxbfcode{\sphinxupquote{license}}}{\emph{self}}{}
Prints information about data licences.

\end{fulllineitems}

\index{list\_resources() (pypath.legacy.main.PyPath static method)@\spxentry{list\_resources()}\spxextra{pypath.legacy.main.PyPath static method}}

\begin{fulllineitems}
\phantomsection\label{\detokenize{reference:pypath.legacy.main.PyPath.list_resources}}\pysiglinewithargsret{\sphinxbfcode{\sphinxupquote{static }}\sphinxbfcode{\sphinxupquote{list\_resources}}}{}{}
Prints the list of resources through the standard output.

\end{fulllineitems}

\index{load\_3dcomplexes() (pypath.legacy.main.PyPath method)@\spxentry{load\_3dcomplexes()}\spxextra{pypath.legacy.main.PyPath method}}

\begin{fulllineitems}
\phantomsection\label{\detokenize{reference:pypath.legacy.main.PyPath.load_3dcomplexes}}\pysiglinewithargsret{\sphinxbfcode{\sphinxupquote{load\_3dcomplexes}}}{\emph{graph=None}}{}
\end{fulllineitems}

\index{load\_3did\_ddi() (pypath.legacy.main.PyPath method)@\spxentry{load\_3did\_ddi()}\spxextra{pypath.legacy.main.PyPath method}}

\begin{fulllineitems}
\phantomsection\label{\detokenize{reference:pypath.legacy.main.PyPath.load_3did_ddi}}\pysiglinewithargsret{\sphinxbfcode{\sphinxupquote{load\_3did\_ddi}}}{}{}
\end{fulllineitems}

\index{load\_3did\_ddi2() (pypath.legacy.main.PyPath method)@\spxentry{load\_3did\_ddi2()}\spxextra{pypath.legacy.main.PyPath method}}

\begin{fulllineitems}
\phantomsection\label{\detokenize{reference:pypath.legacy.main.PyPath.load_3did_ddi2}}\pysiglinewithargsret{\sphinxbfcode{\sphinxupquote{load\_3did\_ddi2}}}{\emph{ddi=True}, \emph{interfaces=False}}{}
\end{fulllineitems}

\index{load\_3did\_dmi() (pypath.legacy.main.PyPath method)@\spxentry{load\_3did\_dmi()}\spxextra{pypath.legacy.main.PyPath method}}

\begin{fulllineitems}
\phantomsection\label{\detokenize{reference:pypath.legacy.main.PyPath.load_3did_dmi}}\pysiglinewithargsret{\sphinxbfcode{\sphinxupquote{load\_3did\_dmi}}}{}{}
\end{fulllineitems}

\index{load\_3did\_interfaces() (pypath.legacy.main.PyPath method)@\spxentry{load\_3did\_interfaces()}\spxextra{pypath.legacy.main.PyPath method}}

\begin{fulllineitems}
\phantomsection\label{\detokenize{reference:pypath.legacy.main.PyPath.load_3did_interfaces}}\pysiglinewithargsret{\sphinxbfcode{\sphinxupquote{load\_3did\_interfaces}}}{}{}
\end{fulllineitems}

\index{load\_all\_pathways() (pypath.legacy.main.PyPath method)@\spxentry{load\_all\_pathways()}\spxextra{pypath.legacy.main.PyPath method}}

\begin{fulllineitems}
\phantomsection\label{\detokenize{reference:pypath.legacy.main.PyPath.load_all_pathways}}\pysiglinewithargsret{\sphinxbfcode{\sphinxupquote{load\_all\_pathways}}}{\emph{graph=None}}{}
\end{fulllineitems}

\index{load\_compleat() (pypath.legacy.main.PyPath method)@\spxentry{load\_compleat()}\spxextra{pypath.legacy.main.PyPath method}}

\begin{fulllineitems}
\phantomsection\label{\detokenize{reference:pypath.legacy.main.PyPath.load_compleat}}\pysiglinewithargsret{\sphinxbfcode{\sphinxupquote{load\_compleat}}}{\emph{graph=None}}{}
Loads complexes from Compleat. Loads data into vertex attribute
\sphinxtitleref{graph.vs{[}‘complexes’{]}{[}‘compleat’{]}}.
This resource is human only.

\end{fulllineitems}

\index{load\_complexportal() (pypath.legacy.main.PyPath method)@\spxentry{load\_complexportal()}\spxextra{pypath.legacy.main.PyPath method}}

\begin{fulllineitems}
\phantomsection\label{\detokenize{reference:pypath.legacy.main.PyPath.load_complexportal}}\pysiglinewithargsret{\sphinxbfcode{\sphinxupquote{load\_complexportal}}}{\emph{graph=None}}{}
Loads complexes from ComplexPortal. Loads data into vertex attribute
\sphinxtitleref{graph.vs{[}‘complexes’{]}{[}‘complexportal’{]}}.
This resource is human only.

\end{fulllineitems}

\index{load\_comppi() (pypath.legacy.main.PyPath method)@\spxentry{load\_comppi()}\spxextra{pypath.legacy.main.PyPath method}}

\begin{fulllineitems}
\phantomsection\label{\detokenize{reference:pypath.legacy.main.PyPath.load_comppi}}\pysiglinewithargsret{\sphinxbfcode{\sphinxupquote{load\_comppi}}}{\emph{graph=None}}{}
\end{fulllineitems}

\index{load\_corum() (pypath.legacy.main.PyPath method)@\spxentry{load\_corum()}\spxextra{pypath.legacy.main.PyPath method}}

\begin{fulllineitems}
\phantomsection\label{\detokenize{reference:pypath.legacy.main.PyPath.load_corum}}\pysiglinewithargsret{\sphinxbfcode{\sphinxupquote{load\_corum}}}{\emph{graph=None}}{}
Loads complexes from CORUM database. Loads data into vertex attribute
\sphinxtitleref{graph.vs{[}‘complexes’{]}{[}‘corum’{]}}.
This resource is human only.

\end{fulllineitems}

\index{load\_dbptm() (pypath.legacy.main.PyPath method)@\spxentry{load\_dbptm()}\spxextra{pypath.legacy.main.PyPath method}}

\begin{fulllineitems}
\phantomsection\label{\detokenize{reference:pypath.legacy.main.PyPath.load_dbptm}}\pysiglinewithargsret{\sphinxbfcode{\sphinxupquote{load\_dbptm}}}{\emph{non\_matching=False}, \emph{trace=False}, \emph{**kwargs}}{}
\end{fulllineitems}

\index{load\_ddi() (pypath.legacy.main.PyPath method)@\spxentry{load\_ddi()}\spxextra{pypath.legacy.main.PyPath method}}

\begin{fulllineitems}
\phantomsection\label{\detokenize{reference:pypath.legacy.main.PyPath.load_ddi}}\pysiglinewithargsret{\sphinxbfcode{\sphinxupquote{load\_ddi}}}{\emph{ddi}}{}
ddi is either a list of intera.DomainDomain objects,
or a function resulting this list

\end{fulllineitems}

\index{load\_ddis() (pypath.legacy.main.PyPath method)@\spxentry{load\_ddis()}\spxextra{pypath.legacy.main.PyPath method}}

\begin{fulllineitems}
\phantomsection\label{\detokenize{reference:pypath.legacy.main.PyPath.load_ddis}}\pysiglinewithargsret{\sphinxbfcode{\sphinxupquote{load\_ddis}}}{\emph{methods={[}'dataio.get\_3dc\_ddi', 'dataio.get\_domino\_ddi', 'self.load\_3did\_ddi2'{]}}}{}
\end{fulllineitems}

\index{load\_depod\_dmi() (pypath.legacy.main.PyPath method)@\spxentry{load\_depod\_dmi()}\spxextra{pypath.legacy.main.PyPath method}}

\begin{fulllineitems}
\phantomsection\label{\detokenize{reference:pypath.legacy.main.PyPath.load_depod_dmi}}\pysiglinewithargsret{\sphinxbfcode{\sphinxupquote{load\_depod\_dmi}}}{}{}
\end{fulllineitems}

\index{load\_disgenet() (pypath.legacy.main.PyPath method)@\spxentry{load\_disgenet()}\spxextra{pypath.legacy.main.PyPath method}}

\begin{fulllineitems}
\phantomsection\label{\detokenize{reference:pypath.legacy.main.PyPath.load_disgenet}}\pysiglinewithargsret{\sphinxbfcode{\sphinxupquote{load\_disgenet}}}{\emph{dataset='curated'}, \emph{score=0.0}, \emph{umls=False}, \emph{full\_data=False}}{}
Assigns DisGeNet disease-gene associations to the proteins
in the network. Disease annotations will be added to the \sphinxtitleref{dis}
vertex attribute.
\begin{quote}\begin{description}
\item[{Parameters}] \leavevmode\begin{itemize}
\item {} 
\sphinxstyleliteralstrong{\sphinxupquote{score}} (\sphinxstyleliteralemphasis{\sphinxupquote{float}}) \textendash{} Confidence score from DisGeNet. Only associations
above the score provided will be considered.

\item {} 
\sphinxstyleliteralstrong{\sphinxupquote{ulms}} (\sphinxstyleliteralemphasis{\sphinxupquote{bool}}) \textendash{} By default we assign a list of disease names to
each protein. To use Unified Medical Language System IDs instead
set this to \sphinxtitleref{True}.

\item {} 
\sphinxstyleliteralstrong{\sphinxupquote{full\_data}} (\sphinxstyleliteralemphasis{\sphinxupquote{bool}}) \textendash{} By default we load only disease names. Set this
to \sphinxtitleref{True} if you wish to load additional annotations like number
of PubMed IDs, number of SNPs and original sources.

\end{itemize}

\end{description}\end{quote}

\end{fulllineitems}

\index{load\_dmi() (pypath.legacy.main.PyPath method)@\spxentry{load\_dmi()}\spxextra{pypath.legacy.main.PyPath method}}

\begin{fulllineitems}
\phantomsection\label{\detokenize{reference:pypath.legacy.main.PyPath.load_dmi}}\pysiglinewithargsret{\sphinxbfcode{\sphinxupquote{load\_dmi}}}{\emph{dmi}}{}
dmi is either a list of intera.DomainMotif objects,
or a function resulting this list

\end{fulllineitems}

\index{load\_dmis() (pypath.legacy.main.PyPath method)@\spxentry{load\_dmis()}\spxextra{pypath.legacy.main.PyPath method}}

\begin{fulllineitems}
\phantomsection\label{\detokenize{reference:pypath.legacy.main.PyPath.load_dmis}}\pysiglinewithargsret{\sphinxbfcode{\sphinxupquote{load\_dmis}}}{\emph{methods={[}'self.pfam\_regions', 'self.load\_depod\_dmi', 'self.load\_dbptm', 'self.load\_mimp\_dmi', 'self.load\_pnetworks\_dmi', 'self.load\_domino\_dmi', 'self.load\_pepcyber', 'self.load\_psite\_reg', 'self.load\_psite\_phos', 'self.load\_ielm', 'self.load\_phosphoelm', 'self.load\_elm', 'self.load\_3did\_dmi'{]}}}{}
\end{fulllineitems}

\index{load\_domino\_dmi() (pypath.legacy.main.PyPath method)@\spxentry{load\_domino\_dmi()}\spxextra{pypath.legacy.main.PyPath method}}

\begin{fulllineitems}
\phantomsection\label{\detokenize{reference:pypath.legacy.main.PyPath.load_domino_dmi}}\pysiglinewithargsret{\sphinxbfcode{\sphinxupquote{load\_domino\_dmi}}}{\emph{organism=None}}{}
\end{fulllineitems}

\index{load\_dorothea() (pypath.legacy.main.PyPath method)@\spxentry{load\_dorothea()}\spxextra{pypath.legacy.main.PyPath method}}

\begin{fulllineitems}
\phantomsection\label{\detokenize{reference:pypath.legacy.main.PyPath.load_dorothea}}\pysiglinewithargsret{\sphinxbfcode{\sphinxupquote{load\_dorothea}}}{\emph{levels=\{'A'}, \emph{'B'\}}, \emph{only\_curated=False}}{}
Adds TF-target interactions from TF regulons to the network.
DoRothEA is a comprehensive resource of TF-target
interactions combining multiple lines of evidences: literature
curated databases, ChIP-Seq data, PWM based prediction using
HOCOMOCO and JASPAR matrices and prediction from GTEx expression
data by ARACNe.

For details see \sphinxurl{https://github.com/saezlab/DoRothEA}.
\begin{quote}\begin{description}
\item[{Parameters}] \leavevmode\begin{itemize}
\item {} 
\sphinxstyleliteralstrong{\sphinxupquote{levels}} (\sphinxstyleliteralemphasis{\sphinxupquote{set}}) \textendash{} Optional, \sphinxcode{\sphinxupquote{\{'A', 'B'\}}} by default. Confidence levels to be
loaded (from A to E) {[}str{]}.

\item {} 
\sphinxstyleliteralstrong{\sphinxupquote{only\_curated}} (\sphinxstyleliteralemphasis{\sphinxupquote{bool}}) \textendash{} Optional, \sphinxcode{\sphinxupquote{False}} by default. Whether to retrieve only the
literature curated interactions or not.

\end{itemize}

\end{description}\end{quote}

\end{fulllineitems}

\index{load\_elm() (pypath.legacy.main.PyPath method)@\spxentry{load\_elm()}\spxextra{pypath.legacy.main.PyPath method}}

\begin{fulllineitems}
\phantomsection\label{\detokenize{reference:pypath.legacy.main.PyPath.load_elm}}\pysiglinewithargsret{\sphinxbfcode{\sphinxupquote{load\_elm}}}{}{}
\end{fulllineitems}

\index{load\_exocarta\_attrs() (pypath.legacy.main.PyPath method)@\spxentry{load\_exocarta\_attrs()}\spxextra{pypath.legacy.main.PyPath method}}

\begin{fulllineitems}
\phantomsection\label{\detokenize{reference:pypath.legacy.main.PyPath.load_exocarta_attrs}}\pysiglinewithargsret{\sphinxbfcode{\sphinxupquote{load\_exocarta\_attrs}}}{\emph{load\_samples=False}, \emph{load\_refs=False}}{}
Creates vertex attributes from ExoCarta data. Creates a boolean
attribute \sphinxcode{\sphinxupquote{exocarts\_exosomal}} which tells whether a protein is
in ExoCarta i.e. has been found in exosomes. Optionally creates
attributes \sphinxcode{\sphinxupquote{exocarta\_samples}} and \sphinxcode{\sphinxupquote{exocarta\_refs}} listing the
sample tissue and the PubMed references, respectively.

\end{fulllineitems}

\index{load\_expression() (pypath.legacy.main.PyPath method)@\spxentry{load\_expression()}\spxextra{pypath.legacy.main.PyPath method}}

\begin{fulllineitems}
\phantomsection\label{\detokenize{reference:pypath.legacy.main.PyPath.load_expression}}\pysiglinewithargsret{\sphinxbfcode{\sphinxupquote{load\_expression}}}{\emph{array=False}}{}
Expression data can be loaded into vertex attributes,
or into a pandas DataFrame \textendash{} the latter offers faster
ways to process and use these huge matrices.

\end{fulllineitems}

\index{load\_from\_pickle() (pypath.legacy.main.PyPath method)@\spxentry{load\_from\_pickle()}\spxextra{pypath.legacy.main.PyPath method}}

\begin{fulllineitems}
\phantomsection\label{\detokenize{reference:pypath.legacy.main.PyPath.load_from_pickle}}\pysiglinewithargsret{\sphinxbfcode{\sphinxupquote{load\_from\_pickle}}}{\emph{pickle\_file}}{}
Shortcut for loading a network from a pickle dump.

\end{fulllineitems}

\index{load\_go() (pypath.legacy.main.PyPath method)@\spxentry{load\_go()}\spxextra{pypath.legacy.main.PyPath method}}

\begin{fulllineitems}
\phantomsection\label{\detokenize{reference:pypath.legacy.main.PyPath.load_go}}\pysiglinewithargsret{\sphinxbfcode{\sphinxupquote{load\_go}}}{\emph{organism=None}}{}
Creates a \sphinxcode{\sphinxupquote{pypath.go.GOAnnotation}} object for one organism in the
dict under \sphinxcode{\sphinxupquote{go}} attribute.
\begin{quote}\begin{description}
\item[{Parameters}] \leavevmode
\sphinxstyleliteralstrong{\sphinxupquote{organism}} (\sphinxstyleliteralemphasis{\sphinxupquote{int}}) \textendash{} NCBI Taxonomy ID of the organism.

\end{description}\end{quote}

\end{fulllineitems}

\index{load\_havugimana() (pypath.legacy.main.PyPath method)@\spxentry{load\_havugimana()}\spxextra{pypath.legacy.main.PyPath method}}

\begin{fulllineitems}
\phantomsection\label{\detokenize{reference:pypath.legacy.main.PyPath.load_havugimana}}\pysiglinewithargsret{\sphinxbfcode{\sphinxupquote{load\_havugimana}}}{\emph{graph=None}}{}
Loads complexes from Havugimana 2012. Loads data into vertex attribute
\sphinxtitleref{graph.vs{[}‘complexes’{]}{[}‘havugimana’{]}}.
This resource is human only.

\end{fulllineitems}

\index{load\_hpa() (pypath.legacy.main.PyPath method)@\spxentry{load\_hpa()}\spxextra{pypath.legacy.main.PyPath method}}

\begin{fulllineitems}
\phantomsection\label{\detokenize{reference:pypath.legacy.main.PyPath.load_hpa}}\pysiglinewithargsret{\sphinxbfcode{\sphinxupquote{load\_hpa}}}{\emph{normal=True}, \emph{pathology=True}, \emph{cancer=True}, \emph{summarize\_pathology=True}, \emph{tissues=None}, \emph{quality=\{'Approved'}, \emph{'Supported'\}}, \emph{levels=\{'High': 3}, \emph{'Low': 1}, \emph{'Medium': 2}, \emph{'Not detected': 0\}}, \emph{graph=None}, \emph{na\_value=0}}{}
Loads Human Protein Atlas data into vertex attributes.

\end{fulllineitems}

\index{load\_hprd\_ptms() (pypath.legacy.main.PyPath method)@\spxentry{load\_hprd\_ptms()}\spxextra{pypath.legacy.main.PyPath method}}

\begin{fulllineitems}
\phantomsection\label{\detokenize{reference:pypath.legacy.main.PyPath.load_hprd_ptms}}\pysiglinewithargsret{\sphinxbfcode{\sphinxupquote{load\_hprd\_ptms}}}{\emph{non\_matching=False}, \emph{trace=False}, \emph{**kwargs}}{}
\end{fulllineitems}

\index{load\_ielm() (pypath.legacy.main.PyPath method)@\spxentry{load\_ielm()}\spxextra{pypath.legacy.main.PyPath method}}

\begin{fulllineitems}
\phantomsection\label{\detokenize{reference:pypath.legacy.main.PyPath.load_ielm}}\pysiglinewithargsret{\sphinxbfcode{\sphinxupquote{load\_ielm}}}{}{}
\end{fulllineitems}

\index{load\_interfaces() (pypath.legacy.main.PyPath method)@\spxentry{load\_interfaces()}\spxextra{pypath.legacy.main.PyPath method}}

\begin{fulllineitems}
\phantomsection\label{\detokenize{reference:pypath.legacy.main.PyPath.load_interfaces}}\pysiglinewithargsret{\sphinxbfcode{\sphinxupquote{load\_interfaces}}}{}{}
\end{fulllineitems}

\index{load\_li2012\_ptms() (pypath.legacy.main.PyPath method)@\spxentry{load\_li2012\_ptms()}\spxextra{pypath.legacy.main.PyPath method}}

\begin{fulllineitems}
\phantomsection\label{\detokenize{reference:pypath.legacy.main.PyPath.load_li2012_ptms}}\pysiglinewithargsret{\sphinxbfcode{\sphinxupquote{load\_li2012\_ptms}}}{\emph{non\_matching=False}, \emph{trace=False}, \emph{**kwargs}}{}
\end{fulllineitems}

\index{load\_ligand\_receptor\_network() (pypath.legacy.main.PyPath method)@\spxentry{load\_ligand\_receptor\_network()}\spxextra{pypath.legacy.main.PyPath method}}

\begin{fulllineitems}
\phantomsection\label{\detokenize{reference:pypath.legacy.main.PyPath.load_ligand_receptor_network}}\pysiglinewithargsret{\sphinxbfcode{\sphinxupquote{load\_ligand\_receptor\_network}}}{\emph{lig\_rec\_resources=True}, \emph{inference\_from\_go=True}, \emph{sources=None}, \emph{keep\_undirected=False}, \emph{keep\_rec\_rec=False}, \emph{keep\_lig\_lig=False}}{}
Initializes a ligand-receptor network.

\end{fulllineitems}

\index{load\_lmpid() (pypath.legacy.main.PyPath method)@\spxentry{load\_lmpid()}\spxextra{pypath.legacy.main.PyPath method}}

\begin{fulllineitems}
\phantomsection\label{\detokenize{reference:pypath.legacy.main.PyPath.load_lmpid}}\pysiglinewithargsret{\sphinxbfcode{\sphinxupquote{load\_lmpid}}}{\emph{method}}{}
\end{fulllineitems}

\index{load\_matrisome\_attrs() (pypath.legacy.main.PyPath method)@\spxentry{load\_matrisome\_attrs()}\spxextra{pypath.legacy.main.PyPath method}}

\begin{fulllineitems}
\phantomsection\label{\detokenize{reference:pypath.legacy.main.PyPath.load_matrisome_attrs}}\pysiglinewithargsret{\sphinxbfcode{\sphinxupquote{load\_matrisome\_attrs}}}{\emph{organism=None}}{}
Loads vertex attributes from MatrisomeDB 2.0. Attributes are
\sphinxcode{\sphinxupquote{matrisome\_class}}, \sphinxcode{\sphinxupquote{matrisome\_subclass}} and \sphinxcode{\sphinxupquote{matrisome\_notes}}.

\end{fulllineitems}

\index{load\_membranome\_attrs() (pypath.legacy.main.PyPath method)@\spxentry{load\_membranome\_attrs()}\spxextra{pypath.legacy.main.PyPath method}}

\begin{fulllineitems}
\phantomsection\label{\detokenize{reference:pypath.legacy.main.PyPath.load_membranome_attrs}}\pysiglinewithargsret{\sphinxbfcode{\sphinxupquote{load\_membranome\_attrs}}}{}{}
Loads attributes from Membranome, a database of single-helix
transmembrane proteins.

\end{fulllineitems}

\index{load\_mimp\_dmi() (pypath.legacy.main.PyPath method)@\spxentry{load\_mimp\_dmi()}\spxextra{pypath.legacy.main.PyPath method}}

\begin{fulllineitems}
\phantomsection\label{\detokenize{reference:pypath.legacy.main.PyPath.load_mimp_dmi}}\pysiglinewithargsret{\sphinxbfcode{\sphinxupquote{load\_mimp\_dmi}}}{\emph{non\_matching=False}, \emph{trace=False}, \emph{**kwargs}}{}
\end{fulllineitems}

\index{load\_mutations() (pypath.legacy.main.PyPath method)@\spxentry{load\_mutations()}\spxextra{pypath.legacy.main.PyPath method}}

\begin{fulllineitems}
\phantomsection\label{\detokenize{reference:pypath.legacy.main.PyPath.load_mutations}}\pysiglinewithargsret{\sphinxbfcode{\sphinxupquote{load\_mutations}}}{\emph{attributes=None}, \emph{gdsc\_datadir=None}, \emph{mutation\_file=None}}{}
Mutations are listed in vertex attributes. Mutation() objects
offers methods to identify residues and look up in Ptm(), Motif()
and Domain() objects, to check if those residues are
modified, or are in some short motif or domain.

\end{fulllineitems}

\index{load\_negatives() (pypath.legacy.main.PyPath method)@\spxentry{load\_negatives()}\spxextra{pypath.legacy.main.PyPath method}}

\begin{fulllineitems}
\phantomsection\label{\detokenize{reference:pypath.legacy.main.PyPath.load_negatives}}\pysiglinewithargsret{\sphinxbfcode{\sphinxupquote{load\_negatives}}}{}{}
\end{fulllineitems}

\index{load\_old\_omnipath() (pypath.legacy.main.PyPath method)@\spxentry{load\_old\_omnipath()}\spxextra{pypath.legacy.main.PyPath method}}

\begin{fulllineitems}
\phantomsection\label{\detokenize{reference:pypath.legacy.main.PyPath.load_old_omnipath}}\pysiglinewithargsret{\sphinxbfcode{\sphinxupquote{load\_old\_omnipath}}}{\emph{kinase\_substrate\_extra=False}, \emph{remove\_htp=False}, \emph{htp\_threshold=1}, \emph{keep\_directed=False}, \emph{min\_refs\_undirected=2}}{}
Loads the OmniPath network as it was before August 2016.
Furthermore it gives some more options.

\end{fulllineitems}

\index{load\_omnipath() (pypath.legacy.main.PyPath method)@\spxentry{load\_omnipath()}\spxextra{pypath.legacy.main.PyPath method}}

\begin{fulllineitems}
\phantomsection\label{\detokenize{reference:pypath.legacy.main.PyPath.load_omnipath}}\pysiglinewithargsret{\sphinxbfcode{\sphinxupquote{load\_omnipath}}}{\emph{omnipath=None}, \emph{kinase\_substrate\_extra=False}, \emph{ligand\_receptor\_extra=False}, \emph{pathway\_extra=False}, \emph{remove\_htp=True}, \emph{htp\_threshold=1}, \emph{keep\_directed=True}, \emph{min\_refs\_undirected=2}, \emph{old\_omnipath\_resources=False}, \emph{exclude=None}, \emph{pickle\_file=None}}{}
Loads the OmniPath network.
Note, if \sphinxcode{\sphinxupquote{pickle\_file}} provided the network will be loaded directly
from there regardless of its content.

\end{fulllineitems}

\index{load\_pathways() (pypath.legacy.main.PyPath method)@\spxentry{load\_pathways()}\spxextra{pypath.legacy.main.PyPath method}}

\begin{fulllineitems}
\phantomsection\label{\detokenize{reference:pypath.legacy.main.PyPath.load_pathways}}\pysiglinewithargsret{\sphinxbfcode{\sphinxupquote{load\_pathways}}}{\emph{source}, \emph{graph=None}}{}
Generic method to load pathway annotations from a resource.
We don’t recommend calling this method but either specific
methods for a single source e.g. \sphinxtitleref{kegg\_pathways()}
or \sphinxtitleref{sirnor\_pathways()} or call \sphinxtitleref{load\_all\_pathways()} to
load all resources.
\begin{quote}\begin{description}
\item[{Parameters}] \leavevmode\begin{itemize}
\item {} 
\sphinxstyleliteralstrong{\sphinxupquote{source}} (\sphinxstyleliteralemphasis{\sphinxupquote{str}}) \textendash{} Name of the source, this need to match a method in the dict
in \sphinxtitleref{get\_pathways()} method and the edge and vertex attributes
with pathway annotations will be called “\textless{}source\textgreater{}\_pathways”.

\item {} 
\sphinxstyleliteralstrong{\sphinxupquote{graph}} (\sphinxstyleliteralemphasis{\sphinxupquote{igraph.Graph}}) \textendash{} A graph, by default the default the \sphinxtitleref{graph} attribute of the
current instance.

\end{itemize}

\end{description}\end{quote}

\end{fulllineitems}

\index{load\_pdb() (pypath.legacy.main.PyPath method)@\spxentry{load\_pdb()}\spxextra{pypath.legacy.main.PyPath method}}

\begin{fulllineitems}
\phantomsection\label{\detokenize{reference:pypath.legacy.main.PyPath.load_pdb}}\pysiglinewithargsret{\sphinxbfcode{\sphinxupquote{load\_pdb}}}{\emph{graph=None}}{}
Loads the 3D structure information from PDB into the network.
Creates the node attribute \sphinxcode{\sphinxupquote{'pdb'}} containing a {[}dict{]} whose
keys are the PDB identifier {[}str{]} and values are {[}tuple{]} of two
elements denoting the experimental method {[}str{]} (e.g.:
\sphinxcode{\sphinxupquote{'X-ray'}}, \sphinxcode{\sphinxupquote{'NMR'}}, …) and the resolution {[}float{]} (if
applicable).
\begin{quote}\begin{description}
\item[{Parameters}] \leavevmode
\sphinxstyleliteralstrong{\sphinxupquote{graph}} (\sphinxstyleliteralemphasis{\sphinxupquote{igraph.Graph}}) \textendash{} Optional, \sphinxcode{\sphinxupquote{None}} by default. The network object for which
the information is to be loaded. If none is passed, takes
the undirected network of the current instance.

\end{description}\end{quote}

\end{fulllineitems}

\index{load\_pepcyber() (pypath.legacy.main.PyPath method)@\spxentry{load\_pepcyber()}\spxextra{pypath.legacy.main.PyPath method}}

\begin{fulllineitems}
\phantomsection\label{\detokenize{reference:pypath.legacy.main.PyPath.load_pepcyber}}\pysiglinewithargsret{\sphinxbfcode{\sphinxupquote{load\_pepcyber}}}{}{}
\end{fulllineitems}

\index{load\_pfam() (pypath.legacy.main.PyPath method)@\spxentry{load\_pfam()}\spxextra{pypath.legacy.main.PyPath method}}

\begin{fulllineitems}
\phantomsection\label{\detokenize{reference:pypath.legacy.main.PyPath.load_pfam}}\pysiglinewithargsret{\sphinxbfcode{\sphinxupquote{load\_pfam}}}{\emph{graph=None}}{}
Loads the protein family information from UniProt into the
network. Creates the node attribute \sphinxcode{\sphinxupquote{'pfam'}} containing a
{[}list{]} of protein family identifier(s) {[}str{]}.
\begin{quote}\begin{description}
\item[{Parameters}] \leavevmode
\sphinxstyleliteralstrong{\sphinxupquote{graph}} (\sphinxstyleliteralemphasis{\sphinxupquote{igraph.Graph}}) \textendash{} Optional, \sphinxcode{\sphinxupquote{None}} by default. The network object for which
the information is to be loaded. If none is passed, takes
the undirected network of the current instance.

\end{description}\end{quote}

\end{fulllineitems}

\index{load\_pfam2() (pypath.legacy.main.PyPath method)@\spxentry{load\_pfam2()}\spxextra{pypath.legacy.main.PyPath method}}

\begin{fulllineitems}
\phantomsection\label{\detokenize{reference:pypath.legacy.main.PyPath.load_pfam2}}\pysiglinewithargsret{\sphinxbfcode{\sphinxupquote{load\_pfam2}}}{}{}
Loads the protein family information from Pfam into the network.
Creates the node attribute \sphinxcode{\sphinxupquote{'pfam'}} containing a {[}list{]} of
{[}dict{]} whose keys are protein family identifier(s) {[}str{]} and
corresponding values are {[}list{]} of {[}dict{]} containing detailed
information about the protein family(ies) for regions and
isoforms of the protein.
\begin{quote}\begin{description}
\item[{Parameters}] \leavevmode
\sphinxstyleliteralstrong{\sphinxupquote{graph}} (\sphinxstyleliteralemphasis{\sphinxupquote{igraph.Graph}}) \textendash{} Optional, \sphinxcode{\sphinxupquote{None}} by default. The network object for which
the information is to be loaded. If none is passed, takes
the undirected network of the current instance.

\end{description}\end{quote}

\end{fulllineitems}

\index{load\_pfam3() (pypath.legacy.main.PyPath method)@\spxentry{load\_pfam3()}\spxextra{pypath.legacy.main.PyPath method}}

\begin{fulllineitems}
\phantomsection\label{\detokenize{reference:pypath.legacy.main.PyPath.load_pfam3}}\pysiglinewithargsret{\sphinxbfcode{\sphinxupquote{load\_pfam3}}}{}{}
Loads the protein domain information from Pfam into the network.
Creates the node attribute \sphinxcode{\sphinxupquote{'doms'}} containing a {[}list{]} of
\sphinxcode{\sphinxupquote{pypath.intera.Domain}} instances with information
about each domain of the protein (see the corresponding class
documentation for more information).

\end{fulllineitems}

\index{load\_phospho\_dmi() (pypath.legacy.main.PyPath method)@\spxentry{load\_phospho\_dmi()}\spxextra{pypath.legacy.main.PyPath method}}

\begin{fulllineitems}
\phantomsection\label{\detokenize{reference:pypath.legacy.main.PyPath.load_phospho_dmi}}\pysiglinewithargsret{\sphinxbfcode{\sphinxupquote{load\_phospho\_dmi}}}{\emph{source}, \emph{trace=False}, \emph{return\_raw=False}, \emph{**kwargs}}{}
\end{fulllineitems}

\index{load\_phosphoelm() (pypath.legacy.main.PyPath method)@\spxentry{load\_phosphoelm()}\spxextra{pypath.legacy.main.PyPath method}}

\begin{fulllineitems}
\phantomsection\label{\detokenize{reference:pypath.legacy.main.PyPath.load_phosphoelm}}\pysiglinewithargsret{\sphinxbfcode{\sphinxupquote{load\_phosphoelm}}}{\emph{trace=False}, \emph{**kwargs}}{}
\end{fulllineitems}

\index{load\_pisa() (pypath.legacy.main.PyPath method)@\spxentry{load\_pisa()}\spxextra{pypath.legacy.main.PyPath method}}

\begin{fulllineitems}
\phantomsection\label{\detokenize{reference:pypath.legacy.main.PyPath.load_pisa}}\pysiglinewithargsret{\sphinxbfcode{\sphinxupquote{load\_pisa}}}{\emph{graph=None}}{}
\end{fulllineitems}

\index{load\_pnetworks\_dmi() (pypath.legacy.main.PyPath method)@\spxentry{load\_pnetworks\_dmi()}\spxextra{pypath.legacy.main.PyPath method}}

\begin{fulllineitems}
\phantomsection\label{\detokenize{reference:pypath.legacy.main.PyPath.load_pnetworks_dmi}}\pysiglinewithargsret{\sphinxbfcode{\sphinxupquote{load\_pnetworks\_dmi}}}{\emph{trace=False}, \emph{**kwargs}}{}
\end{fulllineitems}

\index{load\_psite\_phos() (pypath.legacy.main.PyPath method)@\spxentry{load\_psite\_phos()}\spxextra{pypath.legacy.main.PyPath method}}

\begin{fulllineitems}
\phantomsection\label{\detokenize{reference:pypath.legacy.main.PyPath.load_psite_phos}}\pysiglinewithargsret{\sphinxbfcode{\sphinxupquote{load\_psite\_phos}}}{\emph{trace=False}, \emph{**kwargs}}{}
\end{fulllineitems}

\index{load\_psite\_reg() (pypath.legacy.main.PyPath method)@\spxentry{load\_psite\_reg()}\spxextra{pypath.legacy.main.PyPath method}}

\begin{fulllineitems}
\phantomsection\label{\detokenize{reference:pypath.legacy.main.PyPath.load_psite_reg}}\pysiglinewithargsret{\sphinxbfcode{\sphinxupquote{load\_psite\_reg}}}{}{}
\end{fulllineitems}

\index{load\_ptms() (pypath.legacy.main.PyPath method)@\spxentry{load\_ptms()}\spxextra{pypath.legacy.main.PyPath method}}

\begin{fulllineitems}
\phantomsection\label{\detokenize{reference:pypath.legacy.main.PyPath.load_ptms}}\pysiglinewithargsret{\sphinxbfcode{\sphinxupquote{load\_ptms}}}{}{}
\end{fulllineitems}

\index{load\_ptms2() (pypath.legacy.main.PyPath method)@\spxentry{load\_ptms2()}\spxextra{pypath.legacy.main.PyPath method}}

\begin{fulllineitems}
\phantomsection\label{\detokenize{reference:pypath.legacy.main.PyPath.load_ptms2}}\pysiglinewithargsret{\sphinxbfcode{\sphinxupquote{load\_ptms2}}}{\emph{input\_methods=None, map\_by\_homology\_from={[}9606{]}, homology\_only\_swissprot=True, ptm\_homology\_strict=False, nonhuman\_direct\_lookup=True, inputargs=\{\}, database=None, force\_load=False}}{}
This is a new method which will replace \sphinxtitleref{load\_ptms}.
It uses \sphinxtitleref{pypath.enz\_sub.EnzymeSubstrateAggregator}, a newly
introduced module for combining enzyme-substrate data from multiple
resources using homology translation on users demand.
\begin{quote}\begin{description}
\item[{Parameters}] \leavevmode\begin{itemize}
\item {} 
\sphinxstyleliteralstrong{\sphinxupquote{input\_methods}} (\sphinxstyleliteralemphasis{\sphinxupquote{list}}) \textendash{} Resources to collect enzyme-substrate
interactions from. E.g. \sphinxtitleref{{[}‘Signor’, ‘phosphoELM’{]}}. By default
it contains Signor, PhosphoSitePlus, HPRD, phosphoELM, dbPTM,
PhosphoNetworks, Li2012 and MIMP.

\item {} 
\sphinxstyleliteralstrong{\sphinxupquote{map\_by\_homology\_from}} (\sphinxstyleliteralemphasis{\sphinxupquote{list}}) \textendash{} List of NCBI Taxonomy IDs of
source taxons used for homology translation of enzyme-substrate
interactions. If you have a human network and you add here
\sphinxtitleref{{[}10090, 10116{]}} then mouse and rat interactions from the source
databases will be translated to human.

\item {} 
\sphinxstyleliteralstrong{\sphinxupquote{homology\_only\_swissprot}} (\sphinxstyleliteralemphasis{\sphinxupquote{bool}}) \textendash{} \sphinxtitleref{True} by default which means
only SwissProt IDs are accepted at homology translateion, Trembl
IDs will be dropped.

\item {} 
\sphinxstyleliteralstrong{\sphinxupquote{ptm\_homology\_strict}} (\sphinxstyleliteralemphasis{\sphinxupquote{bool}}) \textendash{} For homology translation use
PhosphoSite’s PTM homology table. This guarantees that only
truely homologous sites will be included. Otherwise we only
check if at the same numeric offset in the homologous sequence
the appropriate residue can be find.

\item {} 
\sphinxstyleliteralstrong{\sphinxupquote{nonhuman\_direct\_lookup}} (\sphinxstyleliteralemphasis{\sphinxupquote{bool}}) \textendash{} Fetch also directly nonhuman
data from the resources whereever it’s available. PhosphoSite
contains mouse enzyme-substrate interactions and it is possible
to extract these directly beside translating the human ones
to mouse.

\item {} 
\sphinxstyleliteralstrong{\sphinxupquote{inputargs}} (\sphinxstyleliteralemphasis{\sphinxupquote{dict}}) \textendash{} Additional arguments passed to \sphinxtitleref{PtmProcessor}.
A \sphinxtitleref{dict} can be supplied for each resource, e.g.
\sphinxtitleref{\{‘Signor’: \{…\}, ‘PhosphoSite’: \{…\}, …\}}.
Those not used by \sphinxtitleref{PtmProcessor} are forwarded to the
\sphinxtitleref{pypath.dataio} methods.

\item {} 
\sphinxstyleliteralstrong{\sphinxupquote{database}} \textendash{} A \sphinxcode{\sphinxupquote{PtmAggregator}} object. If provided no new database will be
created.

\item {} 
\sphinxstyleliteralstrong{\sphinxupquote{force\_load}} (\sphinxstyleliteralemphasis{\sphinxupquote{bool}}) \textendash{} If \sphinxcode{\sphinxupquote{True}} the database will be loaded with the parameters
provided here; otherwise if the \sphinxcode{\sphinxupquote{ptm}} module already has a
database no new database will be created. This means the
parameters specified in other arguments might have no effect.

\end{itemize}

\end{description}\end{quote}

\end{fulllineitems}

\index{load\_resource() (pypath.legacy.main.PyPath method)@\spxentry{load\_resource()}\spxextra{pypath.legacy.main.PyPath method}}

\begin{fulllineitems}
\phantomsection\label{\detokenize{reference:pypath.legacy.main.PyPath.load_resource}}\pysiglinewithargsret{\sphinxbfcode{\sphinxupquote{load\_resource}}}{\emph{settings}, \emph{clean=True}, \emph{cache\_files=\{\}}, \emph{reread=None}, \emph{redownload=None}, \emph{keep\_raw=False}}{}
Loads the data from a single resource and attaches it to the
network
\begin{quote}\begin{description}
\item[{Parameters}] \leavevmode\begin{itemize}
\item {} 
\sphinxstyleliteralstrong{\sphinxupquote{settings}} (\sphinxstyleliteralemphasis{\sphinxupquote{pypath.input\_formats.NetworkInput}}) \textendash{} \sphinxcode{\sphinxupquote{pypath.input\_formats.NetworkInput}} instance
containing the detailed definition of the input format to
the downloaded file.

\item {} 
\sphinxstyleliteralstrong{\sphinxupquote{clean}} (\sphinxstyleliteralemphasis{\sphinxupquote{bool}}) \textendash{} Optional, \sphinxcode{\sphinxupquote{True}} by default. Whether to clean the graph
after importing the data or not. See
\sphinxcode{\sphinxupquote{pypath.main.PyPath.clean\_graph()}} for more
information.

\item {} 
\sphinxstyleliteralstrong{\sphinxupquote{cache\_files}} (\sphinxstyleliteralemphasis{\sphinxupquote{dict}}) \textendash{} Optional, \sphinxcode{\sphinxupquote{\{\}}} by default. Contains the resource name(s)
{[}str{]} (keys) and the corresponding cached file name {[}str{]}.
If provided (and file exists) bypasses the download of the
data for that resource and uses the cache file instead.

\item {} 
\sphinxstyleliteralstrong{\sphinxupquote{reread}} (\sphinxstyleliteralemphasis{\sphinxupquote{bool}}) \textendash{} Optional, \sphinxcode{\sphinxupquote{False}} by default. Specifies whether to reread
the data files from the cache or omit them (similar to
\sphinxstyleemphasis{redownload}).

\item {} 
\sphinxstyleliteralstrong{\sphinxupquote{redownload}} (\sphinxstyleliteralemphasis{\sphinxupquote{bool}}) \textendash{} Optional, \sphinxcode{\sphinxupquote{False}} by default. Specifies whether to
re-download the data and ignore the cache.

\end{itemize}

\end{description}\end{quote}

\end{fulllineitems}

\index{load\_resources() (pypath.legacy.main.PyPath method)@\spxentry{load\_resources()}\spxextra{pypath.legacy.main.PyPath method}}

\begin{fulllineitems}
\phantomsection\label{\detokenize{reference:pypath.legacy.main.PyPath.load_resources}}\pysiglinewithargsret{\sphinxbfcode{\sphinxupquote{load\_resources}}}{\emph{lst=None}, \emph{exclude={[}{]}}, \emph{cache\_files=\{\}}, \emph{reread=False}, \emph{redownload=None}, \emph{keep\_raw=False}}{}
Loads multiple resources, and cleans up after. Looks up ID
types, and loads all ID conversion tables from UniProt if
necessary. This is much faster than loading the ID conversion
and the resources one by one.
\begin{quote}\begin{description}
\item[{Parameters}] \leavevmode\begin{itemize}
\item {} 
\sphinxstyleliteralstrong{\sphinxupquote{lst}} (\sphinxstyleliteralemphasis{\sphinxupquote{dict}}) \textendash{} Optional, \sphinxcode{\sphinxupquote{None}} by default. Specifies the data input
formats for the different resources (keys) {[}str{]}. Values
are \sphinxcode{\sphinxupquote{pypath.input\_formats.NetworkInput}} instances
containing the information. By default uses the set of
resources of OmniPath.

\item {} 
\sphinxstyleliteralstrong{\sphinxupquote{exclude}} (\sphinxstyleliteralemphasis{\sphinxupquote{list}}) \textendash{} Optional, \sphinxcode{\sphinxupquote{{[}{]}}} by default. List of resources {[}str{]} to
exclude from the network.

\item {} 
\sphinxstyleliteralstrong{\sphinxupquote{cache\_files}} (\sphinxstyleliteralemphasis{\sphinxupquote{dict}}) \textendash{} Optional, \sphinxcode{\sphinxupquote{\{\}}} by default. Contains the resource name(s)
{[}str{]} (keys) and the corresponding cached file name {[}str{]}.
If provided (and file exists) bypasses the download of the
data for that resource and uses the cache file instead.

\item {} 
\sphinxstyleliteralstrong{\sphinxupquote{reread}} (\sphinxstyleliteralemphasis{\sphinxupquote{bool}}) \textendash{} Optional, \sphinxcode{\sphinxupquote{False}} by default. Specifies whether to reread
the data files from the cache or omit them (similar to
\sphinxstyleemphasis{redownload}).

\item {} 
\sphinxstyleliteralstrong{\sphinxupquote{redownload}} (\sphinxstyleliteralemphasis{\sphinxupquote{bool}}) \textendash{} Optional, \sphinxcode{\sphinxupquote{False}} by default. Specifies whether to
re-download the data and ignore the cache.

\end{itemize}

\end{description}\end{quote}

\end{fulllineitems}

\index{load\_signor\_ptms() (pypath.legacy.main.PyPath method)@\spxentry{load\_signor\_ptms()}\spxextra{pypath.legacy.main.PyPath method}}

\begin{fulllineitems}
\phantomsection\label{\detokenize{reference:pypath.legacy.main.PyPath.load_signor_ptms}}\pysiglinewithargsret{\sphinxbfcode{\sphinxupquote{load\_signor\_ptms}}}{\emph{non\_matching=False}, \emph{trace=False}, \emph{**kwargs}}{}
\end{fulllineitems}

\index{load\_surfaceome\_attrs() (pypath.legacy.main.PyPath method)@\spxentry{load\_surfaceome\_attrs()}\spxextra{pypath.legacy.main.PyPath method}}

\begin{fulllineitems}
\phantomsection\label{\detokenize{reference:pypath.legacy.main.PyPath.load_surfaceome_attrs}}\pysiglinewithargsret{\sphinxbfcode{\sphinxupquote{load\_surfaceome\_attrs}}}{}{}
Loads vertex attributes from the In Silico Human Surfaceome.
Attributes are \sphinxcode{\sphinxupquote{surfaceome\_score}}, \sphinxcode{\sphinxupquote{surfaceome\_class}} and
\sphinxcode{\sphinxupquote{surfaceome\_subclass}}.

\end{fulllineitems}

\index{load\_tfregulons() (pypath.legacy.main.PyPath method)@\spxentry{load\_tfregulons()}\spxextra{pypath.legacy.main.PyPath method}}

\begin{fulllineitems}
\phantomsection\label{\detokenize{reference:pypath.legacy.main.PyPath.load_tfregulons}}\pysiglinewithargsret{\sphinxbfcode{\sphinxupquote{load\_tfregulons}}}{\emph{levels=\{'A'}, \emph{'B'\}}, \emph{only\_curated=False}}{}
Adds TF-target interactions from TF regulons to the network.
DoRothEA is a comprehensive resource of TF-target
interactions combining multiple lines of evidences: literature
curated databases, ChIP-Seq data, PWM based prediction using
HOCOMOCO and JASPAR matrices and prediction from GTEx expression
data by ARACNe.

For details see \sphinxurl{https://github.com/saezlab/DoRothEA}.
\begin{quote}\begin{description}
\item[{Parameters}] \leavevmode\begin{itemize}
\item {} 
\sphinxstyleliteralstrong{\sphinxupquote{levels}} (\sphinxstyleliteralemphasis{\sphinxupquote{set}}) \textendash{} Optional, \sphinxcode{\sphinxupquote{\{'A', 'B'\}}} by default. Confidence levels to be
loaded (from A to E) {[}str{]}.

\item {} 
\sphinxstyleliteralstrong{\sphinxupquote{only\_curated}} (\sphinxstyleliteralemphasis{\sphinxupquote{bool}}) \textendash{} Optional, \sphinxcode{\sphinxupquote{False}} by default. Whether to retrieve only the
literature curated interactions or not.

\end{itemize}

\end{description}\end{quote}

\end{fulllineitems}

\index{load\_vesiclepedia\_attrs() (pypath.legacy.main.PyPath method)@\spxentry{load\_vesiclepedia\_attrs()}\spxextra{pypath.legacy.main.PyPath method}}

\begin{fulllineitems}
\phantomsection\label{\detokenize{reference:pypath.legacy.main.PyPath.load_vesiclepedia_attrs}}\pysiglinewithargsret{\sphinxbfcode{\sphinxupquote{load\_vesiclepedia\_attrs}}}{\emph{load\_samples=False}, \emph{load\_refs=False}, \emph{load\_vesicle\_type=False}}{}
Creates vertex attributes from Vesiclepedia data. Creates a boolean
attribute \sphinxcode{\sphinxupquote{vesiclepedia\_in\_vesicle}} which tells whether a protein is
in ExoCarta i.e. has been found in exosomes. Optionally creates
attributes \sphinxcode{\sphinxupquote{vesiclepedia\_samples}}, \sphinxcode{\sphinxupquote{vesiclepedia\_refs}} and
\sphinxcode{\sphinxupquote{vesiclepedia\_vesicles}} listing the sample tissue, the PubMed
references and the vesicle types, respectively.

\end{fulllineitems}

\index{lookup\_cache() (pypath.legacy.main.PyPath method)@\spxentry{lookup\_cache()}\spxextra{pypath.legacy.main.PyPath method}}

\begin{fulllineitems}
\phantomsection\label{\detokenize{reference:pypath.legacy.main.PyPath.lookup_cache}}\pysiglinewithargsret{\sphinxbfcode{\sphinxupquote{lookup\_cache}}}{\emph{name}, \emph{cache\_files}, \emph{int\_cache}, \emph{edges\_cache}}{}
Checks up the cache folder for the files of a given resource.
First checks if \sphinxstyleemphasis{name} is on the \sphinxstyleemphasis{cache\_files} dictionary.
If so, loads either the interactions or edges otherwise. If
not, checks \sphinxstyleemphasis{edges\_cache} or \sphinxstyleemphasis{int\_cache} otherwise.
\begin{quote}\begin{description}
\item[{Parameters}] \leavevmode\begin{itemize}
\item {} 
\sphinxstyleliteralstrong{\sphinxupquote{name}} (\sphinxstyleliteralemphasis{\sphinxupquote{str}}) \textendash{} Name of the resource (lower-case).

\item {} 
\sphinxstyleliteralstrong{\sphinxupquote{cache\_files}} (\sphinxstyleliteralemphasis{\sphinxupquote{dict}}) \textendash{} Contains the resource name(s) {[}str{]} (keys) and the
corresponding cached file name {[}str{]} (values).

\item {} 
\sphinxstyleliteralstrong{\sphinxupquote{int\_cache}} (\sphinxstyleliteralemphasis{\sphinxupquote{str}}) \textendash{} Path to the interactions cache file of the resource.

\item {} 
\sphinxstyleliteralstrong{\sphinxupquote{edges\_cache}} (\sphinxstyleliteralemphasis{\sphinxupquote{str}}) \textendash{} Path to the edges cache file of the resource.

\end{itemize}

\item[{Returns}] \leavevmode
\begin{itemize}
\item {} 
(\sphinxstyleemphasis{file}) \textendash{} The loaded pickle file from the cache if the
file is contains the interactions. \sphinxcode{\sphinxupquote{None}} otherwise.

\item {} 
(\sphinxstyleemphasis{list}) \textendash{} List of mapped edges if the file contains the
information from the edges. \sphinxcode{\sphinxupquote{{[}{]}}} otherwise.

\end{itemize}


\end{description}\end{quote}

\end{fulllineitems}

\index{loop\_edges() (pypath.legacy.main.PyPath method)@\spxentry{loop\_edges()}\spxextra{pypath.legacy.main.PyPath method}}

\begin{fulllineitems}
\phantomsection\label{\detokenize{reference:pypath.legacy.main.PyPath.loop_edges}}\pysiglinewithargsret{\sphinxbfcode{\sphinxupquote{loop\_edges}}}{\emph{index=True}, \emph{graph=None}}{}
Returns an iterator of the indices (or the edge instances) of
the edges which represent a loop (whose source and target node
are the same).
\begin{quote}\begin{description}
\item[{Parameters}] \leavevmode\begin{itemize}
\item {} 
\sphinxstyleliteralstrong{\sphinxupquote{index}} (\sphinxstyleliteralemphasis{\sphinxupquote{bool}}) \textendash{} Optional, \sphinxcode{\sphinxupquote{True}} by default. Whether to return the
iterator of the indices or the edge instances.

\item {} 
\sphinxstyleliteralstrong{\sphinxupquote{graph}} (\sphinxstyleliteralemphasis{\sphinxupquote{igraph.Graph}}) \textendash{} Optional, \sphinxcode{\sphinxupquote{None}} by default. The graph object where the
edge loops are to be searched. If none is passed, takes the
undirected network of the current instance.

\end{itemize}

\item[{Returns}] \leavevmode
(\sphinxstyleemphasis{generator}) \textendash{} Generator object containing the edge
indices (or instances) containing loops.

\end{description}\end{quote}

\end{fulllineitems}

\index{mean\_reference\_per\_interaction() (pypath.legacy.main.PyPath method)@\spxentry{mean\_reference\_per\_interaction()}\spxextra{pypath.legacy.main.PyPath method}}

\begin{fulllineitems}
\phantomsection\label{\detokenize{reference:pypath.legacy.main.PyPath.mean_reference_per_interaction}}\pysiglinewithargsret{\sphinxbfcode{\sphinxupquote{mean\_reference\_per\_interaction}}}{\emph{resources=None}}{}
Computes the mean number of references per interaction of the
network.
\begin{quote}\begin{description}
\item[{Returns}] \leavevmode
(\sphinxstyleemphasis{float}) \textendash{} Mean number of interactions per edge.

\end{description}\end{quote}

\end{fulllineitems}

\index{mean\_reference\_per\_interaction\_by\_resource() (pypath.legacy.main.PyPath method)@\spxentry{mean\_reference\_per\_interaction\_by\_resource()}\spxextra{pypath.legacy.main.PyPath method}}

\begin{fulllineitems}
\phantomsection\label{\detokenize{reference:pypath.legacy.main.PyPath.mean_reference_per_interaction_by_resource}}\pysiglinewithargsret{\sphinxbfcode{\sphinxupquote{mean\_reference\_per\_interaction\_by\_resource}}}{\emph{resources=None}}{}
Computes the mean number of references per interaction of the
network.
\begin{quote}\begin{description}
\item[{Returns}] \leavevmode
(\sphinxstyleemphasis{float}) \textendash{} Mean number of interactions per edge.

\end{description}\end{quote}

\end{fulllineitems}

\index{merge\_lists() (pypath.legacy.main.PyPath method)@\spxentry{merge\_lists()}\spxextra{pypath.legacy.main.PyPath method}}

\begin{fulllineitems}
\phantomsection\label{\detokenize{reference:pypath.legacy.main.PyPath.merge_lists}}\pysiglinewithargsret{\sphinxbfcode{\sphinxupquote{merge\_lists}}}{\emph{id\_a}, \emph{id\_b}, \emph{name=None}, \emph{and\_or='and'}, \emph{delete=False}, \emph{func='max'}}{}
Merges two lists from \sphinxcode{\sphinxupquote{pypat.main.PyPath.lists}}.
\begin{quote}\begin{description}
\item[{Parameters}] \leavevmode\begin{itemize}
\item {} 
\sphinxstyleliteralstrong{\sphinxupquote{id\_a}} (\sphinxstyleliteralemphasis{\sphinxupquote{str}}) \textendash{} Name of the first list to be merged.

\item {} 
\sphinxstyleliteralstrong{\sphinxupquote{id\_b}} (\sphinxstyleliteralemphasis{\sphinxupquote{str}}) \textendash{} Name of the second list to be merged.

\item {} 
\sphinxstyleliteralstrong{\sphinxupquote{name}} (\sphinxstyleliteralemphasis{\sphinxupquote{str}}) \textendash{} Optional, \sphinxcode{\sphinxupquote{None}} by default. Specifies a new name for the
merged list. If none is passed, name will be set to
\sphinxstyleemphasis{id\_a*\_*id\_b}.

\item {} 
\sphinxstyleliteralstrong{\sphinxupquote{and\_or}} (\sphinxstyleliteralemphasis{\sphinxupquote{str}}) \textendash{} Optional, \sphinxcode{\sphinxupquote{'and'}} by default. The logic operation perfomed
in the merging: \sphinxcode{\sphinxupquote{'and'}} performs an union, \sphinxcode{\sphinxupquote{'or'}} for
the intersection.

\item {} 
\sphinxstyleliteralstrong{\sphinxupquote{delete}} (\sphinxstyleliteralemphasis{\sphinxupquote{bool}}) \textendash{} Optional, \sphinxcode{\sphinxupquote{False}} by default. Whether to delete the
former lists or not.

\item {} 
\sphinxstyleliteralstrong{\sphinxupquote{func}} (\sphinxstyleliteralemphasis{\sphinxupquote{str}}) \textendash{} Optional, \sphinxcode{\sphinxupquote{'max'}} by default. Not used.

\end{itemize}

\end{description}\end{quote}

\end{fulllineitems}

\index{merge\_nodes() (pypath.legacy.main.PyPath method)@\spxentry{merge\_nodes()}\spxextra{pypath.legacy.main.PyPath method}}

\begin{fulllineitems}
\phantomsection\label{\detokenize{reference:pypath.legacy.main.PyPath.merge_nodes}}\pysiglinewithargsret{\sphinxbfcode{\sphinxupquote{merge\_nodes}}}{\emph{nodes}, \emph{primary=None}, \emph{graph=None}}{}
Merges all attributes and edges of selected nodes and assigns
them to the primary node (by default the one with lowest index).
\begin{quote}\begin{description}
\item[{Parameters}] \leavevmode\begin{itemize}
\item {} 
\sphinxstyleliteralstrong{\sphinxupquote{nodes}} (\sphinxstyleliteralemphasis{\sphinxupquote{list}}) \textendash{} List of node indexes {[}int{]} that are to be collapsed.

\item {} 
\sphinxstyleliteralstrong{\sphinxupquote{primary}} (\sphinxstyleliteralemphasis{\sphinxupquote{int}}) \textendash{} Optional, \sphinxcode{\sphinxupquote{None}} by default. ID of the primary edge, if
none is passed, the node with lowest index on \sphinxstyleemphasis{nodes} is
selected.

\item {} 
\sphinxstyleliteralstrong{\sphinxupquote{graph}} (\sphinxstyleliteralemphasis{\sphinxupquote{igraph.Graph}}) \textendash{} Optional, \sphinxcode{\sphinxupquote{None}} by default. The network graph object from
which the nodes are to be merged. If none is passed, takes
the undirected network graph.

\end{itemize}

\end{description}\end{quote}

\end{fulllineitems}

\index{mimp\_directions() (pypath.legacy.main.PyPath method)@\spxentry{mimp\_directions()}\spxextra{pypath.legacy.main.PyPath method}}

\begin{fulllineitems}
\phantomsection\label{\detokenize{reference:pypath.legacy.main.PyPath.mimp_directions}}\pysiglinewithargsret{\sphinxbfcode{\sphinxupquote{mimp\_directions}}}{\emph{graph=None}}{}
\end{fulllineitems}

\index{mutated\_edges() (pypath.legacy.main.PyPath method)@\spxentry{mutated\_edges()}\spxextra{pypath.legacy.main.PyPath method}}

\begin{fulllineitems}
\phantomsection\label{\detokenize{reference:pypath.legacy.main.PyPath.mutated_edges}}\pysiglinewithargsret{\sphinxbfcode{\sphinxupquote{mutated\_edges}}}{\emph{sample}}{}
Compares the mutated residues and the modified residues in PTMs.
Interactions are marked as mutated if the target residue in the
underlying PTM is mutated.

\end{fulllineitems}

\index{name\_edgelist() (pypath.legacy.main.PyPath method)@\spxentry{name\_edgelist()}\spxextra{pypath.legacy.main.PyPath method}}

\begin{fulllineitems}
\phantomsection\label{\detokenize{reference:pypath.legacy.main.PyPath.name_edgelist}}\pysiglinewithargsret{\sphinxbfcode{\sphinxupquote{name\_edgelist}}}{\emph{graph=None}}{}
Returns an edge list, i.e. a list with tuples of vertex names.

\end{fulllineitems}

\index{names2vids() (pypath.legacy.main.PyPath method)@\spxentry{names2vids()}\spxextra{pypath.legacy.main.PyPath method}}

\begin{fulllineitems}
\phantomsection\label{\detokenize{reference:pypath.legacy.main.PyPath.names2vids}}\pysiglinewithargsret{\sphinxbfcode{\sphinxupquote{names2vids}}}{\emph{names}}{}
From a list of node names, returns their corresponding indices.
\begin{quote}\begin{description}
\item[{Parameters}] \leavevmode
\sphinxstyleliteralstrong{\sphinxupquote{names}} (\sphinxstyleliteralemphasis{\sphinxupquote{list}}) \textendash{} Contains the node names {[}str{]} for which the IDs are to be
searched.

\item[{Returns}] \leavevmode
(\sphinxstyleemphasis{list}) \textendash{} The queried node IDs {[}int{]}.

\end{description}\end{quote}

\end{fulllineitems}

\index{negative\_report() (pypath.legacy.main.PyPath method)@\spxentry{negative\_report()}\spxextra{pypath.legacy.main.PyPath method}}

\begin{fulllineitems}
\phantomsection\label{\detokenize{reference:pypath.legacy.main.PyPath.negative_report}}\pysiglinewithargsret{\sphinxbfcode{\sphinxupquote{negative\_report}}}{\emph{lst=True}, \emph{outFile=None}}{}
Generates a report file with the negative interactions (assumed
to be already loaded).
\begin{quote}\begin{description}
\item[{Parameters}] \leavevmode\begin{itemize}
\item {} 
\sphinxstyleliteralstrong{\sphinxupquote{lst}} (\sphinxstyleliteralemphasis{\sphinxupquote{bool}}) \textendash{} Optional, \sphinxcode{\sphinxupquote{True}} by default. Whether to retun a list of
edges containing the edge instances which have negative
references.

\item {} 
\sphinxstyleliteralstrong{\sphinxupquote{outFile}} (\sphinxstyleliteralemphasis{\sphinxupquote{str}}) \textendash{} Optional, \sphinxcode{\sphinxupquote{None}} by default. The output file name/path. If
none is passed, the default is
\sphinxcode{\sphinxupquote{'results/\textless{}session\_id\textgreater{}-negatives'}}

\end{itemize}

\item[{Returns}] \leavevmode
(\sphinxstyleemphasis{list}) \textendash{} If \sphinxstyleemphasis{lst} is set to \sphinxcode{\sphinxupquote{True}}, returns a {[}list{]}
is returned with the \sphinxcode{\sphinxupquote{igraph.Edge}} instances that
contain at least a negative reference.

\end{description}\end{quote}

\end{fulllineitems}

\index{neighborhood() (pypath.legacy.main.PyPath method)@\spxentry{neighborhood()}\spxextra{pypath.legacy.main.PyPath method}}

\begin{fulllineitems}
\phantomsection\label{\detokenize{reference:pypath.legacy.main.PyPath.neighborhood}}\pysiglinewithargsret{\sphinxbfcode{\sphinxupquote{neighborhood}}}{\emph{identifiers}, \emph{order=1}, \emph{mode='ALL'}}{}
\end{fulllineitems}

\index{neighbors() (pypath.legacy.main.PyPath method)@\spxentry{neighbors()}\spxextra{pypath.legacy.main.PyPath method}}

\begin{fulllineitems}
\phantomsection\label{\detokenize{reference:pypath.legacy.main.PyPath.neighbors}}\pysiglinewithargsret{\sphinxbfcode{\sphinxupquote{neighbors}}}{\emph{identifier}, \emph{mode='ALL'}}{}
\end{fulllineitems}

\index{neighbourhood\_network() (pypath.legacy.main.PyPath method)@\spxentry{neighbourhood\_network()}\spxextra{pypath.legacy.main.PyPath method}}

\begin{fulllineitems}
\phantomsection\label{\detokenize{reference:pypath.legacy.main.PyPath.neighbourhood_network}}\pysiglinewithargsret{\sphinxbfcode{\sphinxupquote{neighbourhood\_network}}}{\emph{center}, \emph{second=False}}{}
\end{fulllineitems}

\index{network\_by\_go() (pypath.legacy.main.PyPath method)@\spxentry{network\_by\_go()}\spxextra{pypath.legacy.main.PyPath method}}

\begin{fulllineitems}
\phantomsection\label{\detokenize{reference:pypath.legacy.main.PyPath.network_by_go}}\pysiglinewithargsret{\sphinxbfcode{\sphinxupquote{network\_by\_go}}}{\emph{node\_categories}, \emph{network\_sources=None}, \emph{include=None}, \emph{exclude=None}, \emph{directed=False}, \emph{keep\_undirected=False}, \emph{prefix='GO'}, \emph{delete=True}, \emph{copy=False}, \emph{vertex\_attrs=True}, \emph{edge\_attrs=True}}{}
Creates or filters a network based on Gene Ontology annotations.
\begin{quote}\begin{description}
\item[{Parameters}] \leavevmode\begin{itemize}
\item {} 
\sphinxstyleliteralstrong{\sphinxupquote{node\_categories}} (\sphinxstyleliteralemphasis{\sphinxupquote{dict}}) \textendash{} 
A dict with custom category labels as keys and expressions of
GO terms as values. E.g.
{\color{red}\bfseries{}{}`{}`}\{‘extracell’: ‘\sphinxurl{GO:0005576} and not \sphinxurl{GO:0070062}’,
\begin{quote}

’plasmamem’: ‘\sphinxurl{GO:0005887}’\}{}`{}`.
\end{quote}


\item {} 
\sphinxstyleliteralstrong{\sphinxupquote{network\_sources}} (\sphinxstyleliteralemphasis{\sphinxupquote{dict}}) \textendash{} A dict with anything as keys and network input format definintions
(\sphinxcode{\sphinxupquote{input\_formats.NetworkInput}} instances) as values.

\item {} 
\sphinxstyleliteralstrong{\sphinxupquote{include}} (\sphinxstyleliteralemphasis{\sphinxupquote{list}}) \textendash{} A list of tuples of category label pairs. By default we keep all
edges connecting proteins annotated with any of the defined
categories. If \sphinxcode{\sphinxupquote{include}} is defined then only edges between
category pairs defined here will be kept and all others deleted.

\item {} 
\sphinxstyleliteralstrong{\sphinxupquote{exclude}} (\sphinxstyleliteralemphasis{\sphinxupquote{list}}) \textendash{} Similarly to include, all edges will be kept but the ones listed
in \sphinxcode{\sphinxupquote{exclude}} will be deleted.

\item {} 
\sphinxstyleliteralstrong{\sphinxupquote{directed}} (\sphinxstyleliteralemphasis{\sphinxupquote{bool}}) \textendash{} If True \sphinxcode{\sphinxupquote{include}} and \sphinxcode{\sphinxupquote{exclude}} relations will be processed
with directed (source, target) else direction won’t be considered.

\item {} 
\sphinxstyleliteralstrong{\sphinxupquote{keep\_undirected}} (\sphinxstyleliteralemphasis{\sphinxupquote{bool}}) \textendash{} If True the interactions without direction information will be
kept even if \sphinxcode{\sphinxupquote{directed}} is True. Passed to \sphinxcode{\sphinxupquote{edges\_between}}
as \sphinxcode{\sphinxupquote{strict}} argument.

\item {} 
\sphinxstyleliteralstrong{\sphinxupquote{prefix}} (\sphinxstyleliteralemphasis{\sphinxupquote{str}}) \textendash{} Prefix for all vertex and edge attributes created in this
operation. E.g. if you have a category label ‘bar’ and prefix
is ‘foo’ then you will have a new vertex attribute ‘foo\_\_bar’.

\item {} 
\sphinxstyleliteralstrong{\sphinxupquote{delete}} (\sphinxstyleliteralemphasis{\sphinxupquote{bool}}) \textendash{} Delete the vertices and edges which don’t belong to any of the
categories.

\item {} 
\sphinxstyleliteralstrong{\sphinxupquote{copy}} (\sphinxstyleliteralemphasis{\sphinxupquote{bool}}) \textendash{} Return a copy of the entire \sphinxcode{\sphinxupquote{PyPath}} object with the graph
filtered by GO terms. By default the object is modified in place
and \sphinxcode{\sphinxupquote{None}} is returned.

\item {} 
\sphinxstyleliteralstrong{\sphinxupquote{vertex\_attrs}} (\sphinxstyleliteralemphasis{\sphinxupquote{bool}}) \textendash{} Create vertex attributes.

\item {} 
\sphinxstyleliteralstrong{\sphinxupquote{edge\_attrs}} (\sphinxstyleliteralemphasis{\sphinxupquote{bool}}) \textendash{} Create edge attributes.

\end{itemize}

\end{description}\end{quote}

\end{fulllineitems}

\index{network\_filter() (pypath.legacy.main.PyPath method)@\spxentry{network\_filter()}\spxextra{pypath.legacy.main.PyPath method}}

\begin{fulllineitems}
\phantomsection\label{\detokenize{reference:pypath.legacy.main.PyPath.network_filter}}\pysiglinewithargsret{\sphinxbfcode{\sphinxupquote{network\_filter}}}{\emph{p=2.0}}{}
This function aims to cut the number of edges in the network,
without losing nodes, to make the network less connected,
less hairball-like, more usable for analysis.

\end{fulllineitems}

\index{network\_stats() (pypath.legacy.main.PyPath method)@\spxentry{network\_stats()}\spxextra{pypath.legacy.main.PyPath method}}

\begin{fulllineitems}
\phantomsection\label{\detokenize{reference:pypath.legacy.main.PyPath.network_stats}}\pysiglinewithargsret{\sphinxbfcode{\sphinxupquote{network\_stats}}}{\emph{outfile=None}}{}
Calculates basic statistics for the whole network and each of
sources (node and edge counts, average node degree, graph
diameter, transitivity, adhesion and cohesion). Writes the
results in a tab file. File is stored in
\sphinxcode{\sphinxupquote{pypath.main.PyPath.outdir}} (\sphinxcode{\sphinxupquote{'results'}} by default).
\begin{quote}\begin{description}
\item[{Parameters}] \leavevmode
\sphinxstyleliteralstrong{\sphinxupquote{outfile}} (\sphinxstyleliteralemphasis{\sphinxupquote{str}}) \textendash{} Optional, \sphinxcode{\sphinxupquote{None}} by default. Specifies the file name. If
none is specified, this will be
\sphinxcode{\sphinxupquote{'pwnet-\textless{}session\_id\textgreater{}-stats'}}.

\end{description}\end{quote}

\end{fulllineitems}

\index{new\_edges() (pypath.legacy.main.PyPath method)@\spxentry{new\_edges()}\spxextra{pypath.legacy.main.PyPath method}}

\begin{fulllineitems}
\phantomsection\label{\detokenize{reference:pypath.legacy.main.PyPath.new_edges}}\pysiglinewithargsret{\sphinxbfcode{\sphinxupquote{new\_edges}}}{\emph{edges}}{}
Adds new edges from any iterable of edges to the undirected
graph. Basically, calls \sphinxcode{\sphinxupquote{igraph.Graph.add\_edges()}}.
\begin{quote}\begin{description}
\item[{Parameters}] \leavevmode
\sphinxstyleliteralstrong{\sphinxupquote{edges}} (\sphinxstyleliteralemphasis{\sphinxupquote{list}}) \textendash{} Contains the edges that are to be added to the network.

\end{description}\end{quote}

\end{fulllineitems}

\index{new\_nodes() (pypath.legacy.main.PyPath method)@\spxentry{new\_nodes()}\spxextra{pypath.legacy.main.PyPath method}}

\begin{fulllineitems}
\phantomsection\label{\detokenize{reference:pypath.legacy.main.PyPath.new_nodes}}\pysiglinewithargsret{\sphinxbfcode{\sphinxupquote{new\_nodes}}}{\emph{nodes}}{}
Adds new nodes from any iterable of nodes to the undirected
graph. Basically, calls \sphinxcode{\sphinxupquote{igraph.Graph.add\_vertices()}}.
\begin{quote}\begin{description}
\item[{Parameters}] \leavevmode
\sphinxstyleliteralstrong{\sphinxupquote{nodes}} (\sphinxstyleliteralemphasis{\sphinxupquote{list}}) \textendash{} Contains the nodes that are to be added to the network.

\end{description}\end{quote}

\end{fulllineitems}

\index{node\_exists() (pypath.legacy.main.PyPath method)@\spxentry{node\_exists()}\spxextra{pypath.legacy.main.PyPath method}}

\begin{fulllineitems}
\phantomsection\label{\detokenize{reference:pypath.legacy.main.PyPath.node_exists}}\pysiglinewithargsret{\sphinxbfcode{\sphinxupquote{node\_exists}}}{\emph{name}}{}
Checks if a node exists in the (undirected) network.
\begin{quote}\begin{description}
\item[{Parameters}] \leavevmode
\sphinxstyleliteralstrong{\sphinxupquote{name}} (\sphinxstyleliteralemphasis{\sphinxupquote{str}}) \textendash{} The name of the node to be searched.

\item[{Returns}] \leavevmode
(\sphinxstyleemphasis{bool}) \textendash{} Whether the node exists in the network or not.

\end{description}\end{quote}

\end{fulllineitems}

\index{numof\_directed\_edges() (pypath.legacy.main.PyPath method)@\spxentry{numof\_directed\_edges()}\spxextra{pypath.legacy.main.PyPath method}}

\begin{fulllineitems}
\phantomsection\label{\detokenize{reference:pypath.legacy.main.PyPath.numof_directed_edges}}\pysiglinewithargsret{\sphinxbfcode{\sphinxupquote{numof\_directed\_edges}}}{}{}
\end{fulllineitems}

\index{numof\_edges() (pypath.legacy.main.PyPath method)@\spxentry{numof\_edges()}\spxextra{pypath.legacy.main.PyPath method}}

\begin{fulllineitems}
\phantomsection\label{\detokenize{reference:pypath.legacy.main.PyPath.numof_edges}}\pysiglinewithargsret{\sphinxbfcode{\sphinxupquote{numof\_edges}}}{\emph{resources=None}}{}
Number of edges optionally limited to certain resources.

\end{fulllineitems}

\index{numof\_reference\_interaction\_pairs() (pypath.legacy.main.PyPath method)@\spxentry{numof\_reference\_interaction\_pairs()}\spxextra{pypath.legacy.main.PyPath method}}

\begin{fulllineitems}
\phantomsection\label{\detokenize{reference:pypath.legacy.main.PyPath.numof_reference_interaction_pairs}}\pysiglinewithargsret{\sphinxbfcode{\sphinxupquote{numof\_reference\_interaction\_pairs}}}{}{}
Returns the total of unique references per interaction.
\begin{quote}\begin{description}
\item[{Returns}] \leavevmode
(\sphinxstyleemphasis{int}) \textendash{} Total number of unique references per
interaction.

\end{description}\end{quote}

\end{fulllineitems}

\index{numof\_references\_by\_resource() (pypath.legacy.main.PyPath method)@\spxentry{numof\_references\_by\_resource()}\spxextra{pypath.legacy.main.PyPath method}}

\begin{fulllineitems}
\phantomsection\label{\detokenize{reference:pypath.legacy.main.PyPath.numof_references_by_resource}}\pysiglinewithargsret{\sphinxbfcode{\sphinxupquote{numof\_references\_by\_resource}}}{\emph{resources=None}, \emph{**kwargs}}{}
Counts the references for each resource, optionally limited
to certain resources.

\end{fulllineitems}

\index{numof\_undirected\_edges() (pypath.legacy.main.PyPath method)@\spxentry{numof\_undirected\_edges()}\spxextra{pypath.legacy.main.PyPath method}}

\begin{fulllineitems}
\phantomsection\label{\detokenize{reference:pypath.legacy.main.PyPath.numof_undirected_edges}}\pysiglinewithargsret{\sphinxbfcode{\sphinxupquote{numof\_undirected\_edges}}}{}{}
\end{fulllineitems}

\index{orthology\_translation() (pypath.legacy.main.PyPath method)@\spxentry{orthology\_translation()}\spxextra{pypath.legacy.main.PyPath method}}

\begin{fulllineitems}
\phantomsection\label{\detokenize{reference:pypath.legacy.main.PyPath.orthology_translation}}\pysiglinewithargsret{\sphinxbfcode{\sphinxupquote{orthology\_translation}}}{\emph{target}, \emph{source=None}, \emph{only\_swissprot=True}, \emph{graph=None}}{}
Translates the current object to another organism by orthology.
Proteins without known ortholog will be deleted.
\begin{quote}\begin{description}
\item[{Parameters}] \leavevmode
\sphinxstyleliteralstrong{\sphinxupquote{target}} (\sphinxstyleliteralemphasis{\sphinxupquote{int}}) \textendash{} NCBI Taxonomy ID of the target organism. E.g. 10090 for mouse.

\end{description}\end{quote}

\end{fulllineitems}

\index{p() (pypath.legacy.main.PyPath method)@\spxentry{p()}\spxextra{pypath.legacy.main.PyPath method}}

\begin{fulllineitems}
\phantomsection\label{\detokenize{reference:pypath.legacy.main.PyPath.p}}\pysiglinewithargsret{\sphinxbfcode{\sphinxupquote{p}}}{\emph{identifier}}{}
Returns \sphinxcode{\sphinxupquote{igraph.Vertex()}} object if the identifier
is a valid vertex index in the default undirected graph,
or a UniProt ID or GeneSymbol which can be found in the
default undirected network, otherwise \sphinxcode{\sphinxupquote{None}}.
\begin{description}
\item[{@identifier}] \leavevmode{[}int, str{]}
Vertex index (int) or GeneSymbol (str) or UniProt ID (str) or
\sphinxcode{\sphinxupquote{igraph.Vertex}} object.

\end{description}

\end{fulllineitems}

\index{pathway\_attributes() (pypath.legacy.main.PyPath method)@\spxentry{pathway\_attributes()}\spxextra{pypath.legacy.main.PyPath method}}

\begin{fulllineitems}
\phantomsection\label{\detokenize{reference:pypath.legacy.main.PyPath.pathway_attributes}}\pysiglinewithargsret{\sphinxbfcode{\sphinxupquote{pathway\_attributes}}}{\emph{graph=None}}{}
\end{fulllineitems}

\index{pathway\_members() (pypath.legacy.main.PyPath method)@\spxentry{pathway\_members()}\spxextra{pypath.legacy.main.PyPath method}}

\begin{fulllineitems}
\phantomsection\label{\detokenize{reference:pypath.legacy.main.PyPath.pathway_members}}\pysiglinewithargsret{\sphinxbfcode{\sphinxupquote{pathway\_members}}}{\emph{pathway}, \emph{source}}{}
Returns an iterator with the members of a single pathway.
Apart from the pathway name you need to supply its source
database too.

\end{fulllineitems}

\index{pathway\_names() (pypath.legacy.main.PyPath method)@\spxentry{pathway\_names()}\spxextra{pypath.legacy.main.PyPath method}}

\begin{fulllineitems}
\phantomsection\label{\detokenize{reference:pypath.legacy.main.PyPath.pathway_names}}\pysiglinewithargsret{\sphinxbfcode{\sphinxupquote{pathway\_names}}}{\emph{source}, \emph{graph=None}}{}
Returns the names of all pathways having at least one member
in the current graph.

\end{fulllineitems}

\index{pathway\_similarity() (pypath.legacy.main.PyPath method)@\spxentry{pathway\_similarity()}\spxextra{pypath.legacy.main.PyPath method}}

\begin{fulllineitems}
\phantomsection\label{\detokenize{reference:pypath.legacy.main.PyPath.pathway_similarity}}\pysiglinewithargsret{\sphinxbfcode{\sphinxupquote{pathway\_similarity}}}{\emph{outfile=None}}{}
Computes the Sorensen’s similarity index across nodes and edges
for all the available pathway sources (already loaded in the
network) and saves them into table files. Files are stored in
\sphinxcode{\sphinxupquote{pypath.main.PyPath.outdir}} (\sphinxcode{\sphinxupquote{'results'}} by default).
See \sphinxcode{\sphinxupquote{pypath.main.PyPath.sorensen\_pathways()}} for more
information..
\begin{quote}\begin{description}
\item[{Parameters}] \leavevmode
\sphinxstyleliteralstrong{\sphinxupquote{outfile}} (\sphinxstyleliteralemphasis{\sphinxupquote{str}}) \textendash{} Optional, \sphinxcode{\sphinxupquote{None}} by default. Specifies the file name
prefix (suffixes will be \sphinxcode{\sphinxupquote{'-nodes'}} and \sphinxcode{\sphinxupquote{'-edges'}}). If
none is specified, this will be
\sphinxcode{\sphinxupquote{'pwnet-\textless{}session\_id\textgreater{}-sim-pw'}}.

\end{description}\end{quote}

\end{fulllineitems}

\index{pathways\_table() (pypath.legacy.main.PyPath method)@\spxentry{pathways\_table()}\spxextra{pypath.legacy.main.PyPath method}}

\begin{fulllineitems}
\phantomsection\label{\detokenize{reference:pypath.legacy.main.PyPath.pathways_table}}\pysiglinewithargsret{\sphinxbfcode{\sphinxupquote{pathways\_table}}}{\emph{filename='genes\_pathways.list', pw\_sources={[}'signalink', 'signor', 'netpath', 'kegg'{]}, graph=None}}{}
\end{fulllineitems}

\index{pfam\_regions() (pypath.legacy.main.PyPath method)@\spxentry{pfam\_regions()}\spxextra{pypath.legacy.main.PyPath method}}

\begin{fulllineitems}
\phantomsection\label{\detokenize{reference:pypath.legacy.main.PyPath.pfam_regions}}\pysiglinewithargsret{\sphinxbfcode{\sphinxupquote{pfam\_regions}}}{}{}
\end{fulllineitems}

\index{phosphonetworks\_directions() (pypath.legacy.main.PyPath method)@\spxentry{phosphonetworks\_directions()}\spxextra{pypath.legacy.main.PyPath method}}

\begin{fulllineitems}
\phantomsection\label{\detokenize{reference:pypath.legacy.main.PyPath.phosphonetworks_directions}}\pysiglinewithargsret{\sphinxbfcode{\sphinxupquote{phosphonetworks\_directions}}}{\emph{graph=None}}{}
\end{fulllineitems}

\index{phosphopoint\_directions() (pypath.legacy.main.PyPath method)@\spxentry{phosphopoint\_directions()}\spxextra{pypath.legacy.main.PyPath method}}

\begin{fulllineitems}
\phantomsection\label{\detokenize{reference:pypath.legacy.main.PyPath.phosphopoint_directions}}\pysiglinewithargsret{\sphinxbfcode{\sphinxupquote{phosphopoint\_directions}}}{\emph{graph=None}}{}
\end{fulllineitems}

\index{phosphorylation\_directions() (pypath.legacy.main.PyPath method)@\spxentry{phosphorylation\_directions()}\spxextra{pypath.legacy.main.PyPath method}}

\begin{fulllineitems}
\phantomsection\label{\detokenize{reference:pypath.legacy.main.PyPath.phosphorylation_directions}}\pysiglinewithargsret{\sphinxbfcode{\sphinxupquote{phosphorylation\_directions}}}{}{}
\end{fulllineitems}

\index{phosphorylation\_signs() (pypath.legacy.main.PyPath method)@\spxentry{phosphorylation\_signs()}\spxextra{pypath.legacy.main.PyPath method}}

\begin{fulllineitems}
\phantomsection\label{\detokenize{reference:pypath.legacy.main.PyPath.phosphorylation_signs}}\pysiglinewithargsret{\sphinxbfcode{\sphinxupquote{phosphorylation\_signs}}}{}{}
\end{fulllineitems}

\index{phosphosite\_directions() (pypath.legacy.main.PyPath method)@\spxentry{phosphosite\_directions()}\spxextra{pypath.legacy.main.PyPath method}}

\begin{fulllineitems}
\phantomsection\label{\detokenize{reference:pypath.legacy.main.PyPath.phosphosite_directions}}\pysiglinewithargsret{\sphinxbfcode{\sphinxupquote{phosphosite\_directions}}}{\emph{graph=None}}{}
\end{fulllineitems}

\index{prdb\_tissue\_expr() (pypath.legacy.main.PyPath method)@\spxentry{prdb\_tissue\_expr()}\spxextra{pypath.legacy.main.PyPath method}}

\begin{fulllineitems}
\phantomsection\label{\detokenize{reference:pypath.legacy.main.PyPath.prdb_tissue_expr}}\pysiglinewithargsret{\sphinxbfcode{\sphinxupquote{prdb\_tissue\_expr}}}{\emph{tissue}, \emph{prdb=None}, \emph{graph=None}, \emph{occurrence=1}, \emph{group\_function=\textless{}function PyPath.\textless{}lambda\textgreater{}\textgreater{}}, \emph{na\_value=0.0}}{}
\end{fulllineitems}

\index{process\_directions() (pypath.legacy.main.PyPath method)@\spxentry{process\_directions()}\spxextra{pypath.legacy.main.PyPath method}}

\begin{fulllineitems}
\phantomsection\label{\detokenize{reference:pypath.legacy.main.PyPath.process_directions}}\pysiglinewithargsret{\sphinxbfcode{\sphinxupquote{process\_directions}}}{\emph{dirs}, \emph{name}, \emph{directed=None}, \emph{stimulation=None}, \emph{inhibition=None}, \emph{graph=None}, \emph{id\_type=None}, \emph{dirs\_only=False}}{}
\end{fulllineitems}

\index{process\_dmi() (pypath.legacy.main.PyPath method)@\spxentry{process\_dmi()}\spxextra{pypath.legacy.main.PyPath method}}

\begin{fulllineitems}
\phantomsection\label{\detokenize{reference:pypath.legacy.main.PyPath.process_dmi}}\pysiglinewithargsret{\sphinxbfcode{\sphinxupquote{process\_dmi}}}{\emph{source}, \emph{**kwargs}}{}
This is an universal function
for loading domain-motif objects
like load\_phospho\_dmi() for PTMs.
TODO this will replace load\_elm, load\_ielm, etc

\end{fulllineitems}

\index{protein() (pypath.legacy.main.PyPath method)@\spxentry{protein()}\spxextra{pypath.legacy.main.PyPath method}}

\begin{fulllineitems}
\phantomsection\label{\detokenize{reference:pypath.legacy.main.PyPath.protein}}\pysiglinewithargsret{\sphinxbfcode{\sphinxupquote{protein}}}{\emph{identifier}}{}
Same as \sphinxcode{\sphinxupquote{PyPath.get\_node}}, just for the directed graph.
Returns \sphinxcode{\sphinxupquote{igraph.Vertex()}} object if the identifier
is a valid vertex index in the default directed graph,
or a UniProt ID or GeneSymbol which can be found in the
default directed network, otherwise \sphinxcode{\sphinxupquote{None}}.
\begin{description}
\item[{@identifier}] \leavevmode{[}int, str{]}
Vertex index (int) or GeneSymbol (str) or UniProt ID (str) or
\sphinxcode{\sphinxupquote{igraph.Vertex}} object.

\end{description}

\end{fulllineitems}

\index{protein\_edge() (pypath.legacy.main.PyPath method)@\spxentry{protein\_edge()}\spxextra{pypath.legacy.main.PyPath method}}

\begin{fulllineitems}
\phantomsection\label{\detokenize{reference:pypath.legacy.main.PyPath.protein_edge}}\pysiglinewithargsret{\sphinxbfcode{\sphinxupquote{protein\_edge}}}{\emph{source}, \emph{target}, \emph{directed=True}}{}
Returns \sphinxcode{\sphinxupquote{igraph.Edge}} object if an edge exist between
the 2 proteins, otherwise \sphinxcode{\sphinxupquote{None}}.
\begin{quote}\begin{description}
\item[{Parameters}] \leavevmode\begin{itemize}
\item {} 
\sphinxstyleliteralstrong{\sphinxupquote{source}} (\sphinxstyleliteralemphasis{\sphinxupquote{int}}\sphinxstyleliteralemphasis{\sphinxupquote{,}}\sphinxstyleliteralemphasis{\sphinxupquote{str}}) \textendash{} Vertex index or UniProt ID or GeneSymbol or \sphinxcode{\sphinxupquote{igraph.Vertex}}
object.

\item {} 
\sphinxstyleliteralstrong{\sphinxupquote{target}} (\sphinxstyleliteralemphasis{\sphinxupquote{int}}\sphinxstyleliteralemphasis{\sphinxupquote{,}}\sphinxstyleliteralemphasis{\sphinxupquote{str}}) \textendash{} Vertex index or UniProt ID or GeneSymbol or \sphinxcode{\sphinxupquote{igraph.Vertex}}
object.

\item {} 
\sphinxstyleliteralstrong{\sphinxupquote{directed}} (\sphinxstyleliteralemphasis{\sphinxupquote{bool}}) \textendash{} To be passed to igraph.Graph.get\_eid()

\end{itemize}

\end{description}\end{quote}

\end{fulllineitems}

\index{proteins() (pypath.legacy.main.PyPath method)@\spxentry{proteins()}\spxextra{pypath.legacy.main.PyPath method}}

\begin{fulllineitems}
\phantomsection\label{\detokenize{reference:pypath.legacy.main.PyPath.proteins}}\pysiglinewithargsret{\sphinxbfcode{\sphinxupquote{proteins}}}{\emph{identifiers}}{}
\end{fulllineitems}

\index{ps() (pypath.legacy.main.PyPath method)@\spxentry{ps()}\spxextra{pypath.legacy.main.PyPath method}}

\begin{fulllineitems}
\phantomsection\label{\detokenize{reference:pypath.legacy.main.PyPath.ps}}\pysiglinewithargsret{\sphinxbfcode{\sphinxupquote{ps}}}{\emph{identifiers}}{}
\end{fulllineitems}

\index{random\_walk\_with\_return() (pypath.legacy.main.PyPath method)@\spxentry{random\_walk\_with\_return()}\spxextra{pypath.legacy.main.PyPath method}}

\begin{fulllineitems}
\phantomsection\label{\detokenize{reference:pypath.legacy.main.PyPath.random_walk_with_return}}\pysiglinewithargsret{\sphinxbfcode{\sphinxupquote{random\_walk\_with\_return}}}{\emph{q}, \emph{graph=None}, \emph{c=0.5}, \emph{niter=1000}}{}
Random walk with return (RWR) starting from one or more query nodes.
Returns affinity (probability) vector of all nodes in the graph.
\begin{quote}
\begin{quote}\begin{description}
\item[{param int,list q}] \leavevmode
Vertex IDs of query nodes.

\item[{param igraph.Graph graph}] \leavevmode
An \sphinxtitleref{igraph.Graph} object.

\item[{param float c}] \leavevmode
Probability of restart.

\item[{param int niter}] \leavevmode
Number of iterations.

\end{description}\end{quote}
\end{quote}

\begin{sphinxVerbatim}[commandchars=\\\{\}]
\PYG{g+gp}{\PYGZgt{}\PYGZgt{}\PYGZgt{} }\PYG{k+kn}{import} \PYG{n+nn}{igraph}
\PYG{g+gp}{\PYGZgt{}\PYGZgt{}\PYGZgt{} }\PYG{k+kn}{import} \PYG{n+nn}{pypath}
\PYG{g+gp}{\PYGZgt{}\PYGZgt{}\PYGZgt{} }\PYG{n}{pa} \PYG{o}{=} \PYG{n}{pypath}\PYG{o}{.}\PYG{n}{PyPath}\PYG{p}{(}\PYG{p}{)}
\PYG{g+gp}{\PYGZgt{}\PYGZgt{}\PYGZgt{} }\PYG{n}{pa}\PYG{o}{.}\PYG{n}{init\PYGZus{}network}\PYG{p}{(}\PYG{p}{\PYGZob{}}
\PYG{g+go}{        \PYGZsq{}signor\PYGZsq{}: pypath.data\PYGZus{}formats.pathway[\PYGZsq{}signor\PYGZsq{}]}
\PYG{g+go}{    \PYGZcb{})}
\PYG{g+gp}{\PYGZgt{}\PYGZgt{}\PYGZgt{} }\PYG{n}{q} \PYG{o}{=} \PYG{p}{[}
\PYG{g+go}{        pa.gs(\PYGZsq{}EGFR\PYGZsq{}).index,}
\PYG{g+go}{        pa.gs(\PYGZsq{}ATG4B\PYGZsq{}).index}
\PYG{g+go}{    ]}
\PYG{g+gp}{\PYGZgt{}\PYGZgt{}\PYGZgt{} }\PYG{n}{rwr} \PYG{o}{=} \PYG{n}{pa}\PYG{o}{.}\PYG{n}{random\PYGZus{}walk\PYGZus{}with\PYGZus{}return}\PYG{p}{(}\PYG{n}{q} \PYG{o}{=} \PYG{n}{q}\PYG{p}{)}
\PYG{g+gp}{\PYGZgt{}\PYGZgt{}\PYGZgt{} }\PYG{n}{palette} \PYG{o}{=} \PYG{n}{igraph}\PYG{o}{.}\PYG{n}{RainbowPalette}\PYG{p}{(}\PYG{n}{n} \PYG{o}{=} \PYG{l+m+mi}{100}\PYG{p}{)}
\PYG{g+gp}{\PYGZgt{}\PYGZgt{}\PYGZgt{} }\PYG{n}{colors}  \PYG{o}{=} \PYG{p}{[}\PYG{n}{palette}\PYG{o}{.}\PYG{n}{get}\PYG{p}{(}\PYG{n+nb}{int}\PYG{p}{(}\PYG{n+nb}{round}\PYG{p}{(}\PYG{n}{i}\PYG{p}{)}\PYG{p}{)}\PYG{p}{)} \PYG{k}{for} \PYG{n}{i} \PYG{o+ow}{in} \PYG{n}{rwr} \PYG{o}{/} \PYG{n+nb}{max}\PYG{p}{(}\PYG{n}{rwr}\PYG{p}{)} \PYG{o}{*} \PYG{l+m+mi}{99}\PYG{p}{]}
\PYG{g+gp}{\PYGZgt{}\PYGZgt{}\PYGZgt{} }\PYG{n}{igraph}\PYG{o}{.}\PYG{n}{plot}\PYG{p}{(}\PYG{n}{pa}\PYG{o}{.}\PYG{n}{graph}\PYG{p}{,} \PYG{n}{vertex\PYGZus{}color} \PYG{o}{=} \PYG{n}{colors}\PYG{p}{)}
\end{sphinxVerbatim}

\end{fulllineitems}

\index{random\_walk\_with\_return2() (pypath.legacy.main.PyPath method)@\spxentry{random\_walk\_with\_return2()}\spxextra{pypath.legacy.main.PyPath method}}

\begin{fulllineitems}
\phantomsection\label{\detokenize{reference:pypath.legacy.main.PyPath.random_walk_with_return2}}\pysiglinewithargsret{\sphinxbfcode{\sphinxupquote{random\_walk\_with\_return2}}}{\emph{q}, \emph{c=0.5}, \emph{niter=1000}}{}
Literally does random walks.
Only for testing of the other method, to be deleted later.

\end{fulllineitems}

\index{read\_from\_cache() (pypath.legacy.main.PyPath method)@\spxentry{read\_from\_cache()}\spxextra{pypath.legacy.main.PyPath method}}

\begin{fulllineitems}
\phantomsection\label{\detokenize{reference:pypath.legacy.main.PyPath.read_from_cache}}\pysiglinewithargsret{\sphinxbfcode{\sphinxupquote{read\_from\_cache}}}{\emph{cache\_file}}{}
Reads a pickle file from the cache and returns it. It is assumed
that the subfolder \sphinxcode{\sphinxupquote{cache/}} is on the supplied path.
\begin{quote}\begin{description}
\item[{Parameters}] \leavevmode
\sphinxstyleliteralstrong{\sphinxupquote{cache\_file}} (\sphinxstyleliteralemphasis{\sphinxupquote{str}}) \textendash{} Path to the cache file that is to be loaded.

\item[{Returns}] \leavevmode
(\sphinxstyleemphasis{file}) \textendash{} The loaded pickle file from the cache. Type will
depend on the file itself (e.g.: if the pickle was saved
from a dictionary, the type will be {[}dict{]}).

\end{description}\end{quote}

\end{fulllineitems}

\index{read\_list\_file() (pypath.legacy.main.PyPath method)@\spxentry{read\_list\_file()}\spxextra{pypath.legacy.main.PyPath method}}

\begin{fulllineitems}
\phantomsection\label{\detokenize{reference:pypath.legacy.main.PyPath.read_list_file}}\pysiglinewithargsret{\sphinxbfcode{\sphinxupquote{read\_list\_file}}}{\emph{settings}, \emph{**kwargs}}{}
Reads a list from a file and adds it to
\sphinxcode{\sphinxupquote{pypath.main.PyPath.lists}}.
\begin{quote}\begin{description}
\item[{Parameters}] \leavevmode\begin{itemize}
\item {} 
\sphinxstyleliteralstrong{\sphinxupquote{settings}} (\sphinxstyleliteralemphasis{\sphinxupquote{pypath.input\_formats.ReadList}}) \textendash{} \sphinxcode{\sphinxupquote{python.data\_formats.ReadList}} instance specifying
the settings of the file to be read. See the class
documentation for more details.

\item {} 
\sphinxstyleliteralstrong{\sphinxupquote{**kwargs}} \textendash{} Extra arguments passed to the file reading function. Such
function name is outlined in the
\sphinxcode{\sphinxupquote{python.data\_formats.ReadList.input}} attribute and
defined in \sphinxcode{\sphinxupquote{pypath.dataio}}.

\end{itemize}

\end{description}\end{quote}

\end{fulllineitems}

\index{reference\_edge\_ratio() (pypath.legacy.main.PyPath method)@\spxentry{reference\_edge\_ratio()}\spxextra{pypath.legacy.main.PyPath method}}

\begin{fulllineitems}
\phantomsection\label{\detokenize{reference:pypath.legacy.main.PyPath.reference_edge_ratio}}\pysiglinewithargsret{\sphinxbfcode{\sphinxupquote{reference\_edge\_ratio}}}{}{}
Computes the average number of references per edge (as in the
undirected graph).
\begin{quote}\begin{description}
\item[{Returns}] \leavevmode
(\sphinxstyleemphasis{float}) \textendash{} Average number of references per edge.

\end{description}\end{quote}

\end{fulllineitems}

\index{reference\_hist() (pypath.legacy.main.PyPath method)@\spxentry{reference\_hist()}\spxextra{pypath.legacy.main.PyPath method}}

\begin{fulllineitems}
\phantomsection\label{\detokenize{reference:pypath.legacy.main.PyPath.reference_hist}}\pysiglinewithargsret{\sphinxbfcode{\sphinxupquote{reference\_hist}}}{\emph{filename=None}}{}
Generates a file containing a table with information about the
network’s edges. First column contains the source node ID,
followed by the target’s ID, third column contains the number of
references for that interaction and finally the number of
sources. Writes the results in a tab file.
\begin{quote}\begin{description}
\item[{Parameters}] \leavevmode
\sphinxstyleliteralstrong{\sphinxupquote{filename}} (\sphinxstyleliteralemphasis{\sphinxupquote{str}}) \textendash{} Optional, \sphinxcode{\sphinxupquote{None}} by default. Specifies the file name and
path to save the table. If none is passed, file will be
saved in \sphinxcode{\sphinxupquote{pypath.main.PyPath.outdir}} (\sphinxcode{\sphinxupquote{'results'}}
by default) with the name \sphinxcode{\sphinxupquote{'\textless{}session\_id\textgreater{}-refs-hist'}}.

\end{description}\end{quote}

\end{fulllineitems}

\index{references() (pypath.legacy.main.PyPath method)@\spxentry{references()}\spxextra{pypath.legacy.main.PyPath method}}

\begin{fulllineitems}
\phantomsection\label{\detokenize{reference:pypath.legacy.main.PyPath.references}}\pysiglinewithargsret{\sphinxbfcode{\sphinxupquote{references}}}{\emph{resources=None}, \emph{**kwargs}}{}
Returns a set of references for all edges.
\begin{description}
\item[{resources}] \leavevmode{[}None,str,set{]}
Limits the query to one or more resources.

\end{description}

\end{fulllineitems}

\index{references\_by\_resource() (pypath.legacy.main.PyPath method)@\spxentry{references\_by\_resource()}\spxextra{pypath.legacy.main.PyPath method}}

\begin{fulllineitems}
\phantomsection\label{\detokenize{reference:pypath.legacy.main.PyPath.references_by_resource}}\pysiglinewithargsret{\sphinxbfcode{\sphinxupquote{references\_by\_resource}}}{\emph{resources=None}, \emph{**kwargs}}{}
Creates a dict with resources as keys and sets of references
as values.

\end{fulllineitems}

\index{reload() (pypath.legacy.main.PyPath method)@\spxentry{reload()}\spxextra{pypath.legacy.main.PyPath method}}

\begin{fulllineitems}
\phantomsection\label{\detokenize{reference:pypath.legacy.main.PyPath.reload}}\pysiglinewithargsret{\sphinxbfcode{\sphinxupquote{reload}}}{}{}
Reloads the object from the module level.

\end{fulllineitems}

\index{remove\_htp() (pypath.legacy.main.PyPath method)@\spxentry{remove\_htp()}\spxextra{pypath.legacy.main.PyPath method}}

\begin{fulllineitems}
\phantomsection\label{\detokenize{reference:pypath.legacy.main.PyPath.remove_htp}}\pysiglinewithargsret{\sphinxbfcode{\sphinxupquote{remove\_htp}}}{\emph{threshold=50}, \emph{keep\_directed=False}}{}
\end{fulllineitems}

\index{remove\_undirected() (pypath.legacy.main.PyPath method)@\spxentry{remove\_undirected()}\spxextra{pypath.legacy.main.PyPath method}}

\begin{fulllineitems}
\phantomsection\label{\detokenize{reference:pypath.legacy.main.PyPath.remove_undirected}}\pysiglinewithargsret{\sphinxbfcode{\sphinxupquote{remove\_undirected}}}{\emph{min\_refs=None}}{}
\end{fulllineitems}

\index{resources (pypath.legacy.main.PyPath attribute)@\spxentry{resources}\spxextra{pypath.legacy.main.PyPath attribute}}

\begin{fulllineitems}
\phantomsection\label{\detokenize{reference:pypath.legacy.main.PyPath.resources}}\pysigline{\sphinxbfcode{\sphinxupquote{resources}}}
All network resources. Returns \sphinxstyleemphasis{set} of strings.

\end{fulllineitems}

\index{run\_batch() (pypath.legacy.main.PyPath method)@\spxentry{run\_batch()}\spxextra{pypath.legacy.main.PyPath method}}

\begin{fulllineitems}
\phantomsection\label{\detokenize{reference:pypath.legacy.main.PyPath.run_batch}}\pysiglinewithargsret{\sphinxbfcode{\sphinxupquote{run\_batch}}}{\emph{methods}, \emph{toCall=None}}{}
\end{fulllineitems}

\index{save\_network() (pypath.legacy.main.PyPath method)@\spxentry{save\_network()}\spxextra{pypath.legacy.main.PyPath method}}

\begin{fulllineitems}
\phantomsection\label{\detokenize{reference:pypath.legacy.main.PyPath.save_network}}\pysiglinewithargsret{\sphinxbfcode{\sphinxupquote{save\_network}}}{\emph{pickle\_file=None}, \emph{pfile=None}}{}
Saves the network object.

Stores the instance into a pickle (binary) file which can be
reloaded in the future.
\begin{quote}\begin{description}
\item[{Parameters}] \leavevmode
\sphinxstyleliteralstrong{\sphinxupquote{pickle\_file}} (\sphinxstyleliteralemphasis{\sphinxupquote{str}}) \textendash{} Optional, \sphinxcode{\sphinxupquote{None}} by default. The path/file name where to
store the pcikle file. If not specified, saves the network
to its default location
(\sphinxcode{\sphinxupquote{'cache/default\_network.pickle'}}).

\end{description}\end{quote}

\end{fulllineitems}

\index{save\_session() (pypath.legacy.main.PyPath method)@\spxentry{save\_session()}\spxextra{pypath.legacy.main.PyPath method}}

\begin{fulllineitems}
\phantomsection\label{\detokenize{reference:pypath.legacy.main.PyPath.save_session}}\pysiglinewithargsret{\sphinxbfcode{\sphinxupquote{save\_session}}}{}{}
Save the current session state into pickle dump. The file will
be saved in the current working directory as
\sphinxcode{\sphinxupquote{'pypath-\textless{}session\_id\textgreater{}.pickle'}}.

\end{fulllineitems}

\index{save\_to\_pickle() (pypath.legacy.main.PyPath method)@\spxentry{save\_to\_pickle()}\spxextra{pypath.legacy.main.PyPath method}}

\begin{fulllineitems}
\phantomsection\label{\detokenize{reference:pypath.legacy.main.PyPath.save_to_pickle}}\pysiglinewithargsret{\sphinxbfcode{\sphinxupquote{save\_to\_pickle}}}{\emph{pickle\_file=None}, \emph{pfile=None}}{}
Saves the network object.

Stores the instance into a pickle (binary) file which can be
reloaded in the future.
\begin{quote}\begin{description}
\item[{Parameters}] \leavevmode
\sphinxstyleliteralstrong{\sphinxupquote{pickle\_file}} (\sphinxstyleliteralemphasis{\sphinxupquote{str}}) \textendash{} Optional, \sphinxcode{\sphinxupquote{None}} by default. The path/file name where to
store the pcikle file. If not specified, saves the network
to its default location
(\sphinxcode{\sphinxupquote{'cache/default\_network.pickle'}}).

\end{description}\end{quote}

\end{fulllineitems}

\index{search\_attr\_and() (pypath.legacy.main.PyPath method)@\spxentry{search\_attr\_and()}\spxextra{pypath.legacy.main.PyPath method}}

\begin{fulllineitems}
\phantomsection\label{\detokenize{reference:pypath.legacy.main.PyPath.search_attr_and}}\pysiglinewithargsret{\sphinxbfcode{\sphinxupquote{search\_attr\_and}}}{\emph{obj}, \emph{lst}}{}
Searches a given collection of attributes in a given object.
Only returns \sphinxcode{\sphinxupquote{True}}, if all elements of \sphinxstyleemphasis{lst} can be found in
\sphinxstyleemphasis{obj}.
\begin{quote}\begin{description}
\item[{Parameters}] \leavevmode\begin{itemize}
\item {} 
\sphinxstyleliteralstrong{\sphinxupquote{obj}} (\sphinxstyleliteralemphasis{\sphinxupquote{object}}) \textendash{} Object (dictionary-like) where to search for elements of
\sphinxstyleemphasis{lst}.

\item {} 
\sphinxstyleliteralstrong{\sphinxupquote{lst}} (\sphinxstyleliteralemphasis{\sphinxupquote{dict}}) \textendash{} Keys are the attribute names {[}str{]} and values the collection
of elements to be searched in such attribute {[}set{]}.

\end{itemize}

\item[{Returns}] \leavevmode
(\sphinxstyleemphasis{bool}) \textendash{} \sphinxcode{\sphinxupquote{True}} only if \sphinxstyleemphasis{lst} is empty or all of its
elements are found in \sphinxstyleemphasis{obj}. Returns \sphinxcode{\sphinxupquote{False}} otherwise (as
soon as one element of \sphinxstyleemphasis{lst} is not found).

\end{description}\end{quote}

\end{fulllineitems}

\index{search\_attr\_or() (pypath.legacy.main.PyPath method)@\spxentry{search\_attr\_or()}\spxextra{pypath.legacy.main.PyPath method}}

\begin{fulllineitems}
\phantomsection\label{\detokenize{reference:pypath.legacy.main.PyPath.search_attr_or}}\pysiglinewithargsret{\sphinxbfcode{\sphinxupquote{search\_attr\_or}}}{\emph{obj}, \emph{lst}}{}
Searches a given collection of attributes in a given object. As
soon as one item is found, returns \sphinxcode{\sphinxupquote{True}}, if none could be
found then returns \sphinxcode{\sphinxupquote{False}}.
\begin{quote}\begin{description}
\item[{Parameters}] \leavevmode\begin{itemize}
\item {} 
\sphinxstyleliteralstrong{\sphinxupquote{obj}} (\sphinxstyleliteralemphasis{\sphinxupquote{object}}) \textendash{} Object (dictionary-like) where to search for elements of
\sphinxstyleemphasis{lst}.

\item {} 
\sphinxstyleliteralstrong{\sphinxupquote{lst}} (\sphinxstyleliteralemphasis{\sphinxupquote{dict}}) \textendash{} Keys are the attribute names {[}str{]} and values the collection
of elements to be searched in such attribute {[}set{]}.

\end{itemize}

\item[{Returns}] \leavevmode
(\sphinxstyleemphasis{bool}) \textendash{} \sphinxcode{\sphinxupquote{True}} if \sphinxstyleemphasis{lst} is empty or any of its
elements is found in \sphinxstyleemphasis{obj}. Returns only \sphinxcode{\sphinxupquote{False}} if cannot
find anything.

\end{description}\end{quote}

\end{fulllineitems}

\index{second\_neighbours() (pypath.legacy.main.PyPath method)@\spxentry{second\_neighbours()}\spxextra{pypath.legacy.main.PyPath method}}

\begin{fulllineitems}
\phantomsection\label{\detokenize{reference:pypath.legacy.main.PyPath.second_neighbours}}\pysiglinewithargsret{\sphinxbfcode{\sphinxupquote{second\_neighbours}}}{\emph{node}, \emph{indices=False}, \emph{with\_first=False}}{}
Looks for the (first and) second neighbours of a given node and
returns a list of their UniProt IDs.
\begin{quote}\begin{description}
\item[{Parameters}] \leavevmode\begin{itemize}
\item {} 
\sphinxstyleliteralstrong{\sphinxupquote{node}} (\sphinxstyleliteralemphasis{\sphinxupquote{str}}) \textendash{} The UniProt ID of the node of interest. Can also be the
index of such node {[}int{]}.

\item {} 
\sphinxstyleliteralstrong{\sphinxupquote{indices}} (\sphinxstyleliteralemphasis{\sphinxupquote{bool}}) \textendash{} Optional, \sphinxcode{\sphinxupquote{False}} by default. Whether to return the
neighbour nodes indices or their UniProt IDs.

\item {} 
\sphinxstyleliteralstrong{\sphinxupquote{wit\_first}} (\sphinxstyleliteralemphasis{\sphinxupquote{bool}}) \textendash{} Optional, \sphinxcode{\sphinxupquote{False}} by default. Whether to return also the
first neighbours or not.

\end{itemize}

\item[{Returns}] \leavevmode
(\sphinxstyleemphasis{list}) \textendash{} The list containing the second neighbours of the
queried node (including the first ones if specified).

\end{description}\end{quote}

\end{fulllineitems}

\index{select\_by\_go() (pypath.legacy.main.PyPath method)@\spxentry{select\_by\_go()}\spxextra{pypath.legacy.main.PyPath method}}

\begin{fulllineitems}
\phantomsection\label{\detokenize{reference:pypath.legacy.main.PyPath.select_by_go}}\pysiglinewithargsret{\sphinxbfcode{\sphinxupquote{select\_by\_go}}}{\emph{go\_terms}}{}
Retrieves the vertex IDs of all vertices annotated with any
Gene Ontology terms or their descendants, or evaluates string
expression (see \sphinxcode{\sphinxupquote{select\_by\_go\_expr}}).
\begin{quote}\begin{description}
\item[{Parameters}] \leavevmode
\sphinxstyleliteralstrong{\sphinxupquote{go\_terms}} (\sphinxstyleliteralemphasis{\sphinxupquote{str}}\sphinxstyleliteralemphasis{\sphinxupquote{,}}\sphinxstyleliteralemphasis{\sphinxupquote{set}}) \textendash{} A single GO term, a set of GO terms or an expression with
GO terms.

\end{description}\end{quote}

\end{fulllineitems}

\index{select\_by\_go\_all() (pypath.legacy.main.PyPath method)@\spxentry{select\_by\_go\_all()}\spxextra{pypath.legacy.main.PyPath method}}

\begin{fulllineitems}
\phantomsection\label{\detokenize{reference:pypath.legacy.main.PyPath.select_by_go_all}}\pysiglinewithargsret{\sphinxbfcode{\sphinxupquote{select\_by\_go\_all}}}{\emph{go\_terms}}{}
Selects the nodes annotated by all GO terms in \sphinxcode{\sphinxupquote{go\_terms}}.

Returns set of vertex IDs.
\begin{quote}\begin{description}
\item[{Parameters}] \leavevmode
\sphinxstyleliteralstrong{\sphinxupquote{go\_terms}} (\sphinxstyleliteralemphasis{\sphinxupquote{list}}) \textendash{} List, set or tuple of GO terms.

\end{description}\end{quote}

\end{fulllineitems}

\index{select\_by\_go\_expr() (pypath.legacy.main.PyPath method)@\spxentry{select\_by\_go\_expr()}\spxextra{pypath.legacy.main.PyPath method}}

\begin{fulllineitems}
\phantomsection\label{\detokenize{reference:pypath.legacy.main.PyPath.select_by_go_expr}}\pysiglinewithargsret{\sphinxbfcode{\sphinxupquote{select\_by\_go\_expr}}}{\emph{go\_expr}}{}
Selects vertices based on an expression of Gene Ontology terms.
Operator precedence not considered, please use parentheses.
\begin{quote}\begin{description}
\item[{Parameters}] \leavevmode
\sphinxstyleliteralstrong{\sphinxupquote{go\_expr}} (\sphinxstyleliteralemphasis{\sphinxupquote{str}}) \textendash{} An expression of Gene Ontology terms. E.g.
\sphinxcode{\sphinxupquote{'(GO:0005576 and not GO:0070062) or GO:0005887'}}. Parentheses
and operators \sphinxcode{\sphinxupquote{and}}, \sphinxcode{\sphinxupquote{or}} and \sphinxcode{\sphinxupquote{not}} can be used.

\end{description}\end{quote}

\end{fulllineitems}

\index{separate() (pypath.legacy.main.PyPath method)@\spxentry{separate()}\spxextra{pypath.legacy.main.PyPath method}}

\begin{fulllineitems}
\phantomsection\label{\detokenize{reference:pypath.legacy.main.PyPath.separate}}\pysiglinewithargsret{\sphinxbfcode{\sphinxupquote{separate}}}{}{}
Separates the undirected network according to the different
sources. Basically applies
\sphinxcode{\sphinxupquote{pypath.main.PyPath.get\_network()}} for each resource.
\begin{quote}\begin{description}
\item[{Returns}] \leavevmode
(\sphinxstyleemphasis{dict}) \textendash{} Keys are resource names {[}str{]} whose values are
the subnetwork {[}igraph.Graph{]} containing the elements of
that source.

\end{description}\end{quote}

\end{fulllineitems}

\index{separate\_by\_category() (pypath.legacy.main.PyPath method)@\spxentry{separate\_by\_category()}\spxextra{pypath.legacy.main.PyPath method}}

\begin{fulllineitems}
\phantomsection\label{\detokenize{reference:pypath.legacy.main.PyPath.separate_by_category}}\pysiglinewithargsret{\sphinxbfcode{\sphinxupquote{separate\_by\_category}}}{}{}
Separates the undirected network according to resource
categories. Possible categories are:
\begin{itemize}
\item {} 
\sphinxcode{\sphinxupquote{'m'}}: PTM/enzyme-substrate resources.

\item {} 
\sphinxcode{\sphinxupquote{'p'}}: Pathway/activity flow resources.

\item {} 
\sphinxcode{\sphinxupquote{'i'}}: Undirected/PPI resources.

\item {} 
\sphinxcode{\sphinxupquote{'r'}}: Process description/reaction resources.

\item {} 
\sphinxcode{\sphinxupquote{'t'}}: Transcription resources.

\end{itemize}

Works in the same way as \sphinxcode{\sphinxupquote{pypath.main.PyPath.separate()}}.
\begin{quote}\begin{description}
\item[{Returns}] \leavevmode
(\sphinxstyleemphasis{dict}) \textendash{} Keys are category names {[}str{]} whose values are
the subnetwork {[}igraph.Graph{]} containing the elements of
those resources corresponding to that category.

\end{description}\end{quote}

\end{fulllineitems}

\index{sequences() (pypath.legacy.main.PyPath method)@\spxentry{sequences()}\spxextra{pypath.legacy.main.PyPath method}}

\begin{fulllineitems}
\phantomsection\label{\detokenize{reference:pypath.legacy.main.PyPath.sequences}}\pysiglinewithargsret{\sphinxbfcode{\sphinxupquote{sequences}}}{\emph{isoforms=True}, \emph{update=False}}{}
\end{fulllineitems}

\index{set\_boolean\_vattr() (pypath.legacy.main.PyPath method)@\spxentry{set\_boolean\_vattr()}\spxextra{pypath.legacy.main.PyPath method}}

\begin{fulllineitems}
\phantomsection\label{\detokenize{reference:pypath.legacy.main.PyPath.set_boolean_vattr}}\pysiglinewithargsret{\sphinxbfcode{\sphinxupquote{set\_boolean\_vattr}}}{\emph{attr}, \emph{vids}, \emph{negate=False}}{}
\end{fulllineitems}

\index{set\_categories() (pypath.legacy.main.PyPath method)@\spxentry{set\_categories()}\spxextra{pypath.legacy.main.PyPath method}}

\begin{fulllineitems}
\phantomsection\label{\detokenize{reference:pypath.legacy.main.PyPath.set_categories}}\pysiglinewithargsret{\sphinxbfcode{\sphinxupquote{set\_categories}}}{}{}
Sets the category attribute on the network nodes and edges
(\sphinxcode{\sphinxupquote{'cat'}}) as well the edge attribute coercing the references
by category (\sphinxcode{\sphinxupquote{'refs\_by\_cat'}}). The possible categories are
as follows:
\begin{itemize}
\item {} 
\sphinxcode{\sphinxupquote{'m'}}: PTM/enzyme-substrate resources.

\item {} 
\sphinxcode{\sphinxupquote{'p'}}: Pathway/activity flow resources.

\item {} 
\sphinxcode{\sphinxupquote{'i'}}: Undirected/PPI resources.

\item {} 
\sphinxcode{\sphinxupquote{'r'}}: Process description/reaction resources.

\item {} 
\sphinxcode{\sphinxupquote{'t'}}: Transcription resources.

\end{itemize}

\end{fulllineitems}

\index{set\_chembl\_mysql() (pypath.legacy.main.PyPath method)@\spxentry{set\_chembl\_mysql()}\spxextra{pypath.legacy.main.PyPath method}}

\begin{fulllineitems}
\phantomsection\label{\detokenize{reference:pypath.legacy.main.PyPath.set_chembl_mysql}}\pysiglinewithargsret{\sphinxbfcode{\sphinxupquote{set\_chembl\_mysql}}}{\emph{title}, \emph{config\_file=None}}{}
Sets the ChEMBL MySQL configuration according to the \sphinxstyleemphasis{title}
section in \sphinxstyleemphasis{config\_file} ini file configuration.
\begin{quote}\begin{description}
\item[{Parameters}] \leavevmode\begin{itemize}
\item {} 
\sphinxstyleliteralstrong{\sphinxupquote{title}} (\sphinxstyleliteralemphasis{\sphinxupquote{str}}) \textendash{} Section title of the ini file.

\item {} 
\sphinxstyleliteralstrong{\sphinxupquote{config\_file}} (\sphinxstyleliteralemphasis{\sphinxupquote{str}}) \textendash{} Optional, \sphinxcode{\sphinxupquote{None}} by default. Specifies the configuration
file name if none is passed, \sphinxcode{\sphinxupquote{mysql\_config/defaults.mysql}}
will be used.

\end{itemize}

\end{description}\end{quote}

\end{fulllineitems}

\index{set\_disease\_genes() (pypath.legacy.main.PyPath method)@\spxentry{set\_disease\_genes()}\spxextra{pypath.legacy.main.PyPath method}}

\begin{fulllineitems}
\phantomsection\label{\detokenize{reference:pypath.legacy.main.PyPath.set_disease_genes}}\pysiglinewithargsret{\sphinxbfcode{\sphinxupquote{set\_disease\_genes}}}{\emph{dataset='curated'}}{}
Creates a vertex attribute named \sphinxtitleref{dis} with boolean values \sphinxstyleemphasis{True}
if the protein encoded by a disease related gene according to
DisGeNet.
\begin{quote}\begin{description}
\item[{Parameters}] \leavevmode
\sphinxstyleliteralstrong{\sphinxupquote{dataset}} (\sphinxstyleliteralemphasis{\sphinxupquote{str}}) \textendash{} Which dataset to use from DisGeNet. Default is \sphinxtitleref{curated}.

\end{description}\end{quote}

\end{fulllineitems}

\index{set\_druggability() (pypath.legacy.main.PyPath method)@\spxentry{set\_druggability()}\spxextra{pypath.legacy.main.PyPath method}}

\begin{fulllineitems}
\phantomsection\label{\detokenize{reference:pypath.legacy.main.PyPath.set_druggability}}\pysiglinewithargsret{\sphinxbfcode{\sphinxupquote{set\_druggability}}}{}{}
Creates a vertex attribute \sphinxtitleref{dgb} with value \sphinxstyleemphasis{True} if
the protein is druggable, otherwise \sphinxstyleemphasis{False}.

\end{fulllineitems}

\index{set\_drugtargets() (pypath.legacy.main.PyPath method)@\spxentry{set\_drugtargets()}\spxextra{pypath.legacy.main.PyPath method}}

\begin{fulllineitems}
\phantomsection\label{\detokenize{reference:pypath.legacy.main.PyPath.set_drugtargets}}\pysiglinewithargsret{\sphinxbfcode{\sphinxupquote{set\_drugtargets}}}{\emph{pchembl=5.0}}{}
Creates a vertex attribute \sphinxtitleref{dtg} with value \sphinxstyleemphasis{True} if
the protein has at least one compound binding with
affinity higher than \sphinxtitleref{pchembl}, otherwise \sphinxstyleemphasis{False}.
\begin{quote}\begin{description}
\item[{Parameters}] \leavevmode
\sphinxstyleliteralstrong{\sphinxupquote{pchembl}} (\sphinxstyleliteralemphasis{\sphinxupquote{float}}) \textendash{} Pchembl threshold.

\end{description}\end{quote}

\end{fulllineitems}

\index{set\_kinases() (pypath.legacy.main.PyPath method)@\spxentry{set\_kinases()}\spxextra{pypath.legacy.main.PyPath method}}

\begin{fulllineitems}
\phantomsection\label{\detokenize{reference:pypath.legacy.main.PyPath.set_kinases}}\pysiglinewithargsret{\sphinxbfcode{\sphinxupquote{set\_kinases}}}{}{}
Creates a vertex attribute \sphinxtitleref{kin} with value \sphinxstyleemphasis{True} if
the protein is a kinase, otherwise \sphinxstyleemphasis{False}.

\end{fulllineitems}

\index{set\_plasma\_membrane\_proteins\_cspa() (pypath.legacy.main.PyPath method)@\spxentry{set\_plasma\_membrane\_proteins\_cspa()}\spxextra{pypath.legacy.main.PyPath method}}

\begin{fulllineitems}
\phantomsection\label{\detokenize{reference:pypath.legacy.main.PyPath.set_plasma_membrane_proteins_cspa}}\pysiglinewithargsret{\sphinxbfcode{\sphinxupquote{set\_plasma\_membrane\_proteins\_cspa}}}{}{}
Creates a vertex attribute \sphinxtitleref{cspa} with value \sphinxstyleemphasis{True} if
the protein is a plasma membrane protein according to CPSA,
otherwise \sphinxstyleemphasis{False}.

\end{fulllineitems}

\index{set\_plasma\_membrane\_proteins\_cspa\_surfaceome() (pypath.legacy.main.PyPath method)@\spxentry{set\_plasma\_membrane\_proteins\_cspa\_surfaceome()}\spxextra{pypath.legacy.main.PyPath method}}

\begin{fulllineitems}
\phantomsection\label{\detokenize{reference:pypath.legacy.main.PyPath.set_plasma_membrane_proteins_cspa_surfaceome}}\pysiglinewithargsret{\sphinxbfcode{\sphinxupquote{set\_plasma\_membrane\_proteins\_cspa\_surfaceome}}}{\emph{score\_threshold=0.0}}{}
Creates a vertex attribute \sphinxcode{\sphinxupquote{surf}} with value \sphinxstyleemphasis{True} if
the protein is a plasma membrane protein according either to the
Cell Surface Protein Atlas or the In Silico Human Surfaceome.

\end{fulllineitems}

\index{set\_plasma\_membrane\_proteins\_surfaceome() (pypath.legacy.main.PyPath method)@\spxentry{set\_plasma\_membrane\_proteins\_surfaceome()}\spxextra{pypath.legacy.main.PyPath method}}

\begin{fulllineitems}
\phantomsection\label{\detokenize{reference:pypath.legacy.main.PyPath.set_plasma_membrane_proteins_surfaceome}}\pysiglinewithargsret{\sphinxbfcode{\sphinxupquote{set\_plasma\_membrane\_proteins\_surfaceome}}}{\emph{score\_threshold=0.0}}{}
Creates a vertex attribute \sphinxtitleref{ishs} with value \sphinxstyleemphasis{True} if
the protein is a plasma membrane protein according to the In Silico
Human Surfaceome, otherwise \sphinxstyleemphasis{False}.

\end{fulllineitems}

\index{set\_receptors() (pypath.legacy.main.PyPath method)@\spxentry{set\_receptors()}\spxextra{pypath.legacy.main.PyPath method}}

\begin{fulllineitems}
\phantomsection\label{\detokenize{reference:pypath.legacy.main.PyPath.set_receptors}}\pysiglinewithargsret{\sphinxbfcode{\sphinxupquote{set\_receptors}}}{}{}
Creates a vertex attribute \sphinxtitleref{rec} with value \sphinxstyleemphasis{True} if
the protein is a receptor, otherwise \sphinxstyleemphasis{False}.

\end{fulllineitems}

\index{set\_signaling\_proteins() (pypath.legacy.main.PyPath method)@\spxentry{set\_signaling\_proteins()}\spxextra{pypath.legacy.main.PyPath method}}

\begin{fulllineitems}
\phantomsection\label{\detokenize{reference:pypath.legacy.main.PyPath.set_signaling_proteins}}\pysiglinewithargsret{\sphinxbfcode{\sphinxupquote{set\_signaling\_proteins}}}{}{}
Creates a vertex attribute \sphinxtitleref{kin} with value \sphinxstyleemphasis{True} if
the protein is a kinase, otherwise \sphinxstyleemphasis{False}.

\end{fulllineitems}

\index{set\_tfs() (pypath.legacy.main.PyPath method)@\spxentry{set\_tfs()}\spxextra{pypath.legacy.main.PyPath method}}

\begin{fulllineitems}
\phantomsection\label{\detokenize{reference:pypath.legacy.main.PyPath.set_tfs}}\pysiglinewithargsret{\sphinxbfcode{\sphinxupquote{set\_tfs}}}{\emph{classes={[}'a', 'b', 'other'{]}}}{}
\end{fulllineitems}

\index{set\_transcription\_factors() (pypath.legacy.main.PyPath method)@\spxentry{set\_transcription\_factors()}\spxextra{pypath.legacy.main.PyPath method}}

\begin{fulllineitems}
\phantomsection\label{\detokenize{reference:pypath.legacy.main.PyPath.set_transcription_factors}}\pysiglinewithargsret{\sphinxbfcode{\sphinxupquote{set\_transcription\_factors}}}{\emph{classes={[}'a', 'b', 'other'{]}}}{}
Creates a vertex attribute \sphinxtitleref{tf} with value \sphinxstyleemphasis{True} if
the protein is a transcription factor, otherwise \sphinxstyleemphasis{False}.
\begin{quote}\begin{description}
\item[{Parameters}] \leavevmode
\sphinxstyleliteralstrong{\sphinxupquote{classes}} (\sphinxstyleliteralemphasis{\sphinxupquote{list}}) \textendash{} Classes to use from TF Census. Default is \sphinxtitleref{{[}‘a’, ‘b’, ‘other’{]}}.

\end{description}\end{quote}

\end{fulllineitems}

\index{shortest\_path\_dist() (pypath.legacy.main.PyPath method)@\spxentry{shortest\_path\_dist()}\spxextra{pypath.legacy.main.PyPath method}}

\begin{fulllineitems}
\phantomsection\label{\detokenize{reference:pypath.legacy.main.PyPath.shortest_path_dist}}\pysiglinewithargsret{\sphinxbfcode{\sphinxupquote{shortest\_path\_dist}}}{\emph{graph=None}, \emph{subset=None}, \emph{outfile=None}, \emph{**kwargs}}{}
Computes the distribution of shortest paths for each pair of
nodes in the network (or between group(s) of nodes if \sphinxstyleemphasis{subset}
is provided). \sphinxstylestrong{NOTE:} this method can take a while to compute,
e.g.: if the network has 10K nodes, the total number of possible
pairs to compute is:
\begin{equation*}
\begin{split}\binom{10^4}{2} = 49995000\end{split}
\end{equation*}\begin{quote}\begin{description}
\item[{Parameters}] \leavevmode\begin{itemize}
\item {} 
\sphinxstyleliteralstrong{\sphinxupquote{graph}} (\sphinxstyleliteralemphasis{\sphinxupquote{igraph.Graph}}) \textendash{} Optional, \sphinxcode{\sphinxupquote{None}} by default. The network object for which
the shortest path distribution is to be computed. If none is
passed, takes the undirected network of the current
instance.

\item {} 
\sphinxstyleliteralstrong{\sphinxupquote{susbet}} (\sphinxstyleliteralemphasis{\sphinxupquote{tuple}}) \textendash{} Optional, \sphinxcode{\sphinxupquote{None}} by default. Contains two lists of node
indices defining two groups between which the distribution
is to be computed. Can also be {[}list{]} if the shortest paths
are to be searched whithin the group. If none is passed, the
whole network is taken by default.

\item {} 
\sphinxstyleliteralstrong{\sphinxupquote{outfile}} (\sphinxstyleliteralemphasis{\sphinxupquote{str}}) \textendash{} Optional, \sphinxcode{\sphinxupquote{None}} by default. File name/path to save the
shortest path distribution. If none is passed, no file is
generated.

\item {} 
\sphinxstyleliteralstrong{\sphinxupquote{**kwargs}} \textendash{} Additional keyword arguments passed to
\sphinxcode{\sphinxupquote{igraph.Graph.get\_shortest\_paths()}}.

\end{itemize}

\item[{Returns}] \leavevmode
(\sphinxstyleemphasis{list}) \textendash{} The length of the shortest paths for each pair
of nodes of the network (or whithin/between group/s if
\sphinxstyleemphasis{subset} is provided).

\end{description}\end{quote}

\end{fulllineitems}

\index{signaling\_proteins\_list() (pypath.legacy.main.PyPath method)@\spxentry{signaling\_proteins\_list()}\spxextra{pypath.legacy.main.PyPath method}}

\begin{fulllineitems}
\phantomsection\label{\detokenize{reference:pypath.legacy.main.PyPath.signaling_proteins_list}}\pysiglinewithargsret{\sphinxbfcode{\sphinxupquote{signaling\_proteins\_list}}}{}{}
Compiles a list of signaling proteins (as opposed to other
proteins like metabolic enzymes, matrix proteins, etc), by
looking up a few simple keywords in short description of GO
terms.

\end{fulllineitems}

\index{signor\_pathways() (pypath.legacy.main.PyPath method)@\spxentry{signor\_pathways()}\spxextra{pypath.legacy.main.PyPath method}}

\begin{fulllineitems}
\phantomsection\label{\detokenize{reference:pypath.legacy.main.PyPath.signor_pathways}}\pysiglinewithargsret{\sphinxbfcode{\sphinxupquote{signor\_pathways}}}{\emph{graph=None}}{}
\end{fulllineitems}

\index{similarity\_groups() (pypath.legacy.main.PyPath method)@\spxentry{similarity\_groups()}\spxextra{pypath.legacy.main.PyPath method}}

\begin{fulllineitems}
\phantomsection\label{\detokenize{reference:pypath.legacy.main.PyPath.similarity_groups}}\pysiglinewithargsret{\sphinxbfcode{\sphinxupquote{similarity\_groups}}}{\emph{groups}, \emph{index='simpson'}}{}
Computes the similarity index across the given \sphinxstyleemphasis{groups}.
\begin{quote}\begin{description}
\item[{Parameters}] \leavevmode\begin{itemize}
\item {} 
\sphinxstyleliteralstrong{\sphinxupquote{groups}} (\sphinxstyleliteralemphasis{\sphinxupquote{dict}}) \textendash{} Contains the different group names {[}str{]} as keys and their
corresponding elements {[}set{]}.

\item {} 
\sphinxstyleliteralstrong{\sphinxupquote{index}} (\sphinxstyleliteralemphasis{\sphinxupquote{str}}) \textendash{} Optional, \sphinxcode{\sphinxupquote{'simpson'}} by default. The type of index metric
to use to compute the similarity. Options are \sphinxcode{\sphinxupquote{'simpson'}},
\sphinxcode{\sphinxupquote{'sorensen'}} and \sphinxcode{\sphinxupquote{'jaccard'}}.

\end{itemize}

\item[{Returns}] \leavevmode
(\sphinxstyleemphasis{dict}) \textendash{} Dictionary of dictionaries containing the groups
names {[}str{]} as keys (for both inner and outer dictionaries)
and the index metric as inner value {[}float{]} between those
groups.

\end{description}\end{quote}

\end{fulllineitems}

\index{small\_plot() (pypath.legacy.main.PyPath method)@\spxentry{small\_plot()}\spxextra{pypath.legacy.main.PyPath method}}

\begin{fulllineitems}
\phantomsection\label{\detokenize{reference:pypath.legacy.main.PyPath.small_plot}}\pysiglinewithargsret{\sphinxbfcode{\sphinxupquote{small\_plot}}}{\emph{graph}, \emph{**kwargs}}{}
This method is deprecated, do not use it.

\end{fulllineitems}

\index{sorensen\_pathways() (pypath.legacy.main.PyPath method)@\spxentry{sorensen\_pathways()}\spxextra{pypath.legacy.main.PyPath method}}

\begin{fulllineitems}
\phantomsection\label{\detokenize{reference:pypath.legacy.main.PyPath.sorensen_pathways}}\pysiglinewithargsret{\sphinxbfcode{\sphinxupquote{sorensen\_pathways}}}{\emph{pwlist=None}}{}
Computes the Sorensen’s similarity index across nodes and edges
for the given list of pathway sources (all loaded pathway
sources by default).
\begin{quote}\begin{description}
\item[{Parameters}] \leavevmode
\sphinxstyleliteralstrong{\sphinxupquote{pwlist}} (\sphinxstyleliteralemphasis{\sphinxupquote{list}}) \textendash{} Optional, \sphinxcode{\sphinxupquote{None}} by default. The list of pathway sources
to be compared.

\item[{Returns}] \leavevmode
(\sphinxstyleemphasis{dict}) \textendash{} Nested dictionaries (three levels). First-level
keys are \sphinxcode{\sphinxupquote{'nodes'}} and \sphinxcode{\sphinxupquote{'edges'}}, then second and third
levels correspond to \sphinxcode{\sphinxupquote{\textless{}source\textgreater{}\_\_\textless{}patwhay\textgreater{}}} names which map
to the similarity index between those pathways {[}float{]}.

\end{description}\end{quote}

\end{fulllineitems}

\index{source\_diagram() (pypath.legacy.main.PyPath method)@\spxentry{source\_diagram()}\spxextra{pypath.legacy.main.PyPath method}}

\begin{fulllineitems}
\phantomsection\label{\detokenize{reference:pypath.legacy.main.PyPath.source_diagram}}\pysiglinewithargsret{\sphinxbfcode{\sphinxupquote{source\_diagram}}}{\emph{outf=None}, \emph{**kwargs}}{}
\end{fulllineitems}

\index{source\_network() (pypath.legacy.main.PyPath method)@\spxentry{source\_network()}\spxextra{pypath.legacy.main.PyPath method}}

\begin{fulllineitems}
\phantomsection\label{\detokenize{reference:pypath.legacy.main.PyPath.source_network}}\pysiglinewithargsret{\sphinxbfcode{\sphinxupquote{source\_network}}}{\emph{font='HelveticaNeueLTStd'}}{}
For EMBL branding, use Helvetica Neue Linotype Standard light

\end{fulllineitems}

\index{source\_similarity() (pypath.legacy.main.PyPath method)@\spxentry{source\_similarity()}\spxextra{pypath.legacy.main.PyPath method}}

\begin{fulllineitems}
\phantomsection\label{\detokenize{reference:pypath.legacy.main.PyPath.source_similarity}}\pysiglinewithargsret{\sphinxbfcode{\sphinxupquote{source\_similarity}}}{\emph{outfile=None}}{}
Computes the Sorensen’s similarity index across nodes and edges
for all the sources available (already loaded in the network)
and saves them into table files. Files are stored in
\sphinxcode{\sphinxupquote{pypath.main.PyPath.outdir}} (\sphinxcode{\sphinxupquote{'results'}} by default).
See \sphinxcode{\sphinxupquote{pypath.main.PyPath.databases\_similarity()}} for more
information.
\begin{quote}\begin{description}
\item[{Parameters}] \leavevmode
\sphinxstyleliteralstrong{\sphinxupquote{outfile}} (\sphinxstyleliteralemphasis{\sphinxupquote{str}}) \textendash{} Optional, \sphinxcode{\sphinxupquote{None}} by default. Specifies the file name
prefix (suffixes will be \sphinxcode{\sphinxupquote{'-nodes'}} and \sphinxcode{\sphinxupquote{'-edges'}}). If
none is specified, this will be
\sphinxcode{\sphinxupquote{'pwnet-\textless{}session\_id\textgreater{}-sim-src'}}.

\end{description}\end{quote}

\end{fulllineitems}

\index{source\_stats() (pypath.legacy.main.PyPath method)@\spxentry{source\_stats()}\spxextra{pypath.legacy.main.PyPath method}}

\begin{fulllineitems}
\phantomsection\label{\detokenize{reference:pypath.legacy.main.PyPath.source_stats}}\pysiglinewithargsret{\sphinxbfcode{\sphinxupquote{source\_stats}}}{}{}
\end{fulllineitems}

\index{sources\_hist() (pypath.legacy.main.PyPath method)@\spxentry{sources\_hist()}\spxextra{pypath.legacy.main.PyPath method}}

\begin{fulllineitems}
\phantomsection\label{\detokenize{reference:pypath.legacy.main.PyPath.sources_hist}}\pysiglinewithargsret{\sphinxbfcode{\sphinxupquote{sources\_hist}}}{}{}
Counts the number of sources per interaction in the graph and
saves them into a file named \sphinxcode{\sphinxupquote{source\_num}}. File is stored in
\sphinxcode{\sphinxupquote{pypath.main.PyPath.outdir}} (\sphinxcode{\sphinxupquote{'results'}} by
default).

\end{fulllineitems}

\index{sources\_overlap() (pypath.legacy.main.PyPath method)@\spxentry{sources\_overlap()}\spxextra{pypath.legacy.main.PyPath method}}

\begin{fulllineitems}
\phantomsection\label{\detokenize{reference:pypath.legacy.main.PyPath.sources_overlap}}\pysiglinewithargsret{\sphinxbfcode{\sphinxupquote{sources\_overlap}}}{\emph{diagonal=False}}{}
\end{fulllineitems}

\index{sources\_venn\_data() (pypath.legacy.main.PyPath method)@\spxentry{sources\_venn\_data()}\spxextra{pypath.legacy.main.PyPath method}}

\begin{fulllineitems}
\phantomsection\label{\detokenize{reference:pypath.legacy.main.PyPath.sources_venn_data}}\pysiglinewithargsret{\sphinxbfcode{\sphinxupquote{sources\_venn\_data}}}{\emph{fname=None}, \emph{return\_data=False}}{}
Computes the overlap in number of interactions for all pairs of
sources.
\begin{quote}\begin{description}
\item[{Parameters}] \leavevmode\begin{itemize}
\item {} 
\sphinxstyleliteralstrong{\sphinxupquote{fname}} (\sphinxstyleliteralemphasis{\sphinxupquote{str}}) \textendash{} Optional, \sphinxcode{\sphinxupquote{None}} by default. If provided, saves the
results into a table file. File is stored in
\sphinxcode{\sphinxupquote{pypath.main.PyPath.outdir}} (\sphinxcode{\sphinxupquote{'results'}} by
default).

\item {} 
\sphinxstyleliteralstrong{\sphinxupquote{return\_data}} (\sphinxstyleliteralemphasis{\sphinxupquote{bool}}) \textendash{} Optional, \sphinxcode{\sphinxupquote{False}} by default. Whether to return the
results as a {[}list{]}.

\end{itemize}

\item[{Returns}] \leavevmode
(\sphinxstyleemphasis{list}) \textendash{} Only if \sphinxstyleemphasis{return\_data} is set to \sphinxcode{\sphinxupquote{True}}. List
of lists containing the counts for each pair of resources.
This is, for instance, number of interactions only in
resource A, number of interactions only in resource B and
number of common interactions between A and B.

\end{description}\end{quote}

\end{fulllineitems}

\index{stats() (pypath.legacy.main.PyPath method)@\spxentry{stats()}\spxextra{pypath.legacy.main.PyPath method}}

\begin{fulllineitems}
\phantomsection\label{\detokenize{reference:pypath.legacy.main.PyPath.stats}}\pysiglinewithargsret{\sphinxbfcode{\sphinxupquote{stats}}}{\emph{method}, \emph{keep\_collection=False}, \emph{**kwargs}}{}
Creates a collection of entities over the network according to
\sphinxcode{\sphinxupquote{method}} and counts them. By default the collection won’t be
returned but only the counts.

\end{fulllineitems}

\index{straight\_between() (pypath.legacy.main.PyPath method)@\spxentry{straight\_between()}\spxextra{pypath.legacy.main.PyPath method}}

\begin{fulllineitems}
\phantomsection\label{\detokenize{reference:pypath.legacy.main.PyPath.straight_between}}\pysiglinewithargsret{\sphinxbfcode{\sphinxupquote{straight\_between}}}{\emph{id\_a}, \emph{id\_b}}{}
Finds an edge between the provided node names.
\begin{quote}\begin{description}
\item[{Parameters}] \leavevmode\begin{itemize}
\item {} 
\sphinxstyleliteralstrong{\sphinxupquote{id\_a}} (\sphinxstyleliteralemphasis{\sphinxupquote{str}}) \textendash{} The name of the source node.

\item {} 
\sphinxstyleliteralstrong{\sphinxupquote{id\_b}} (\sphinxstyleliteralemphasis{\sphinxupquote{str}}) \textendash{} The name of the target node.

\end{itemize}

\item[{Returns}] \leavevmode
(\sphinxstyleemphasis{int}) \textendash{} The edge ID. If the edge doesn’t exist, returns
{[}list{]} with the node indices {[}int{]}.

\end{description}\end{quote}

\end{fulllineitems}

\index{string\_effects() (pypath.legacy.main.PyPath method)@\spxentry{string\_effects()}\spxextra{pypath.legacy.main.PyPath method}}

\begin{fulllineitems}
\phantomsection\label{\detokenize{reference:pypath.legacy.main.PyPath.string_effects}}\pysiglinewithargsret{\sphinxbfcode{\sphinxupquote{string\_effects}}}{\emph{graph=None}}{}
\end{fulllineitems}

\index{sum\_in\_complex() (pypath.legacy.main.PyPath method)@\spxentry{sum\_in\_complex()}\spxextra{pypath.legacy.main.PyPath method}}

\begin{fulllineitems}
\phantomsection\label{\detokenize{reference:pypath.legacy.main.PyPath.sum_in_complex}}\pysiglinewithargsret{\sphinxbfcode{\sphinxupquote{sum\_in\_complex}}}{\emph{csources={[}'corum'{]}, graph=None}}{}
Returns the total number of edges in the network falling
between two members of the same complex.
Returns as a dict by complex resources.
Calls :py:func:pypath.pypath.Pypath.edges\_in\_comlexes()
to do the calculations.
\begin{description}
\item[{@csources}] \leavevmode{[}list{]}
List of complex resources. Should be already loaded.

\item[{@graph}] \leavevmode{[}igraph.Graph(){]}
The graph object to do the calculations on.

\end{description}

\end{fulllineitems}

\index{summaries\_tab() (pypath.legacy.main.PyPath method)@\spxentry{summaries\_tab()}\spxextra{pypath.legacy.main.PyPath method}}

\begin{fulllineitems}
\phantomsection\label{\detokenize{reference:pypath.legacy.main.PyPath.summaries_tab}}\pysiglinewithargsret{\sphinxbfcode{\sphinxupquote{summaries\_tab}}}{\emph{outfile=None}, \emph{return\_table=False}}{}
Creates a table from resource vs. entity counts and optionally
writes it to \sphinxcode{\sphinxupquote{outfile}} and returns it.

\end{fulllineitems}

\index{table\_latex() (pypath.legacy.main.PyPath method)@\spxentry{table\_latex()}\spxextra{pypath.legacy.main.PyPath method}}

\begin{fulllineitems}
\phantomsection\label{\detokenize{reference:pypath.legacy.main.PyPath.table_latex}}\pysiglinewithargsret{\sphinxbfcode{\sphinxupquote{table\_latex}}}{\emph{fname}, \emph{header}, \emph{data}, \emph{sum\_row=True}, \emph{row\_order=None}, \emph{latex\_hdr=True}, \emph{caption=''}, \emph{font='HelveticaNeueLTStd-LtCn'}, \emph{fontsize=8}, \emph{sum\_label='Total'}, \emph{sum\_cols=None}, \emph{header\_format='\%s'}, \emph{by\_category=True}}{}
\end{fulllineitems}

\index{third\_source\_directions() (pypath.legacy.main.PyPath method)@\spxentry{third\_source\_directions()}\spxextra{pypath.legacy.main.PyPath method}}

\begin{fulllineitems}
\phantomsection\label{\detokenize{reference:pypath.legacy.main.PyPath.third_source_directions}}\pysiglinewithargsret{\sphinxbfcode{\sphinxupquote{third\_source\_directions}}}{\emph{graph=None}, \emph{use\_string\_effects=False}, \emph{use\_laudanna\_data=False}}{}
This method calls a series of methods to get
additional direction \& effect information
from sources having no literature curated references,
but giving sufficient evidence about the directionality
for interactions already supported by literature
evidences from other sources.

\end{fulllineitems}

\index{tissue\_network() (pypath.legacy.main.PyPath method)@\spxentry{tissue\_network()}\spxextra{pypath.legacy.main.PyPath method}}

\begin{fulllineitems}
\phantomsection\label{\detokenize{reference:pypath.legacy.main.PyPath.tissue_network}}\pysiglinewithargsret{\sphinxbfcode{\sphinxupquote{tissue\_network}}}{\emph{tissue}, \emph{graph=None}}{}
Returns a network which includes the proteins expressed in
certain tissue according to ProteomicsDB.
\begin{quote}\begin{description}
\item[{Parameters}] \leavevmode\begin{itemize}
\item {} 
\sphinxstyleliteralstrong{\sphinxupquote{tissue}} (\sphinxstyleliteralemphasis{\sphinxupquote{str}}) \textendash{} Tissue name as used in ProteomicsDB.

\item {} 
\sphinxstyleliteralstrong{\sphinxupquote{graph}} (\sphinxstyleliteralemphasis{\sphinxupquote{igraph.Graph}}) \textendash{} A graph object, by default the \sphinxtitleref{graph} attribute of
the current instance.

\end{itemize}

\end{description}\end{quote}

\end{fulllineitems}

\index{transcription\_factors() (pypath.legacy.main.PyPath method)@\spxentry{transcription\_factors()}\spxextra{pypath.legacy.main.PyPath method}}

\begin{fulllineitems}
\phantomsection\label{\detokenize{reference:pypath.legacy.main.PyPath.transcription_factors}}\pysiglinewithargsret{\sphinxbfcode{\sphinxupquote{transcription\_factors}}}{}{}
\end{fulllineitems}

\index{uniprot() (pypath.legacy.main.PyPath method)@\spxentry{uniprot()}\spxextra{pypath.legacy.main.PyPath method}}

\begin{fulllineitems}
\phantomsection\label{\detokenize{reference:pypath.legacy.main.PyPath.uniprot}}\pysiglinewithargsret{\sphinxbfcode{\sphinxupquote{uniprot}}}{\emph{uniprot}}{}
Returns \sphinxcode{\sphinxupquote{igraph.Vertex()}} object if the UniProt
can be found in the default undirected network,
otherwise \sphinxcode{\sphinxupquote{None}}.
\begin{description}
\item[{@uniprot}] \leavevmode{[}str{]}
UniProt ID.

\end{description}

\end{fulllineitems}

\index{uniprots() (pypath.legacy.main.PyPath method)@\spxentry{uniprots()}\spxextra{pypath.legacy.main.PyPath method}}

\begin{fulllineitems}
\phantomsection\label{\detokenize{reference:pypath.legacy.main.PyPath.uniprots}}\pysiglinewithargsret{\sphinxbfcode{\sphinxupquote{uniprots}}}{\emph{uniprots}}{}
Returns list of \sphinxcode{\sphinxupquote{igraph.Vertex()}} object
for a list of UniProt IDs omitting those
could not be found in the default
undirected graph.

\end{fulllineitems}

\index{uniq\_node\_list() (pypath.legacy.main.PyPath method)@\spxentry{uniq\_node\_list()}\spxextra{pypath.legacy.main.PyPath method}}

\begin{fulllineitems}
\phantomsection\label{\detokenize{reference:pypath.legacy.main.PyPath.uniq_node_list}}\pysiglinewithargsret{\sphinxbfcode{\sphinxupquote{uniq\_node\_list}}}{\emph{lst}}{}
Returns a given list of nodes containing only the unique
elements.
\begin{quote}\begin{description}
\item[{Parameters}] \leavevmode
\sphinxstyleliteralstrong{\sphinxupquote{lst}} (\sphinxstyleliteralemphasis{\sphinxupquote{list}}) \textendash{} List of nodes.

\item[{Returns}] \leavevmode
(\sphinxstyleemphasis{list}) \textendash{} Copy of \sphinxstyleemphasis{lst} containing only unique nodes.

\end{description}\end{quote}

\end{fulllineitems}

\index{uniq\_ptm() (pypath.legacy.main.PyPath method)@\spxentry{uniq\_ptm()}\spxextra{pypath.legacy.main.PyPath method}}

\begin{fulllineitems}
\phantomsection\label{\detokenize{reference:pypath.legacy.main.PyPath.uniq_ptm}}\pysiglinewithargsret{\sphinxbfcode{\sphinxupquote{uniq\_ptm}}}{\emph{ptms}}{}
\end{fulllineitems}

\index{uniq\_ptms() (pypath.legacy.main.PyPath method)@\spxentry{uniq\_ptms()}\spxextra{pypath.legacy.main.PyPath method}}

\begin{fulllineitems}
\phantomsection\label{\detokenize{reference:pypath.legacy.main.PyPath.uniq_ptms}}\pysiglinewithargsret{\sphinxbfcode{\sphinxupquote{uniq\_ptms}}}{}{}
\end{fulllineitems}

\index{up() (pypath.legacy.main.PyPath method)@\spxentry{up()}\spxextra{pypath.legacy.main.PyPath method}}

\begin{fulllineitems}
\phantomsection\label{\detokenize{reference:pypath.legacy.main.PyPath.up}}\pysiglinewithargsret{\sphinxbfcode{\sphinxupquote{up}}}{\emph{uniprot}}{}
Returns \sphinxcode{\sphinxupquote{igraph.Vertex()}} object if the UniProt
can be found in the default undirected network,
otherwise \sphinxcode{\sphinxupquote{None}}.
\begin{description}
\item[{@uniprot}] \leavevmode{[}str{]}
UniProt ID.

\end{description}

\end{fulllineitems}

\index{up\_affected\_by() (pypath.legacy.main.PyPath method)@\spxentry{up\_affected\_by()}\spxextra{pypath.legacy.main.PyPath method}}

\begin{fulllineitems}
\phantomsection\label{\detokenize{reference:pypath.legacy.main.PyPath.up_affected_by}}\pysiglinewithargsret{\sphinxbfcode{\sphinxupquote{up\_affected\_by}}}{\emph{uniprot}}{}
\end{fulllineitems}

\index{up\_affects() (pypath.legacy.main.PyPath method)@\spxentry{up\_affects()}\spxextra{pypath.legacy.main.PyPath method}}

\begin{fulllineitems}
\phantomsection\label{\detokenize{reference:pypath.legacy.main.PyPath.up_affects}}\pysiglinewithargsret{\sphinxbfcode{\sphinxupquote{up\_affects}}}{\emph{uniprot}}{}
\end{fulllineitems}

\index{up\_edge() (pypath.legacy.main.PyPath method)@\spxentry{up\_edge()}\spxextra{pypath.legacy.main.PyPath method}}

\begin{fulllineitems}
\phantomsection\label{\detokenize{reference:pypath.legacy.main.PyPath.up_edge}}\pysiglinewithargsret{\sphinxbfcode{\sphinxupquote{up\_edge}}}{\emph{source}, \emph{target}, \emph{directed=True}}{}
Returns \sphinxcode{\sphinxupquote{igraph.Edge}} object if an edge exist between
the 2 proteins, otherwise \sphinxcode{\sphinxupquote{None}}.
\begin{description}
\item[{@source}] \leavevmode{[}str{]}
UniProt ID

\item[{@target}] \leavevmode{[}str{]}
UniProt ID

\item[{@directed}] \leavevmode{[}bool{]}
To be passed to igraph.Graph.get\_eid()

\end{description}

\end{fulllineitems}

\index{up\_in\_directed() (pypath.legacy.main.PyPath method)@\spxentry{up\_in\_directed()}\spxextra{pypath.legacy.main.PyPath method}}

\begin{fulllineitems}
\phantomsection\label{\detokenize{reference:pypath.legacy.main.PyPath.up_in_directed}}\pysiglinewithargsret{\sphinxbfcode{\sphinxupquote{up\_in\_directed}}}{\emph{uniprot}}{}
\end{fulllineitems}

\index{up\_in\_undirected() (pypath.legacy.main.PyPath method)@\spxentry{up\_in\_undirected()}\spxextra{pypath.legacy.main.PyPath method}}

\begin{fulllineitems}
\phantomsection\label{\detokenize{reference:pypath.legacy.main.PyPath.up_in_undirected}}\pysiglinewithargsret{\sphinxbfcode{\sphinxupquote{up\_in\_undirected}}}{\emph{uniprot}}{}
\end{fulllineitems}

\index{up\_inhibited\_by() (pypath.legacy.main.PyPath method)@\spxentry{up\_inhibited\_by()}\spxextra{pypath.legacy.main.PyPath method}}

\begin{fulllineitems}
\phantomsection\label{\detokenize{reference:pypath.legacy.main.PyPath.up_inhibited_by}}\pysiglinewithargsret{\sphinxbfcode{\sphinxupquote{up\_inhibited\_by}}}{\emph{uniprot}}{}
\end{fulllineitems}

\index{up\_inhibits() (pypath.legacy.main.PyPath method)@\spxentry{up\_inhibits()}\spxextra{pypath.legacy.main.PyPath method}}

\begin{fulllineitems}
\phantomsection\label{\detokenize{reference:pypath.legacy.main.PyPath.up_inhibits}}\pysiglinewithargsret{\sphinxbfcode{\sphinxupquote{up\_inhibits}}}{\emph{uniprot}}{}
\end{fulllineitems}

\index{up\_neighborhood() (pypath.legacy.main.PyPath method)@\spxentry{up\_neighborhood()}\spxextra{pypath.legacy.main.PyPath method}}

\begin{fulllineitems}
\phantomsection\label{\detokenize{reference:pypath.legacy.main.PyPath.up_neighborhood}}\pysiglinewithargsret{\sphinxbfcode{\sphinxupquote{up\_neighborhood}}}{\emph{uniprots}, \emph{order=1}, \emph{mode='ALL'}}{}
\end{fulllineitems}

\index{up\_neighbors() (pypath.legacy.main.PyPath method)@\spxentry{up\_neighbors()}\spxextra{pypath.legacy.main.PyPath method}}

\begin{fulllineitems}
\phantomsection\label{\detokenize{reference:pypath.legacy.main.PyPath.up_neighbors}}\pysiglinewithargsret{\sphinxbfcode{\sphinxupquote{up\_neighbors}}}{\emph{uniprot}, \emph{mode='ALL'}}{}
\end{fulllineitems}

\index{up\_stimulated\_by() (pypath.legacy.main.PyPath method)@\spxentry{up\_stimulated\_by()}\spxextra{pypath.legacy.main.PyPath method}}

\begin{fulllineitems}
\phantomsection\label{\detokenize{reference:pypath.legacy.main.PyPath.up_stimulated_by}}\pysiglinewithargsret{\sphinxbfcode{\sphinxupquote{up\_stimulated\_by}}}{\emph{uniprot}}{}
\end{fulllineitems}

\index{up\_stimulates() (pypath.legacy.main.PyPath method)@\spxentry{up\_stimulates()}\spxextra{pypath.legacy.main.PyPath method}}

\begin{fulllineitems}
\phantomsection\label{\detokenize{reference:pypath.legacy.main.PyPath.up_stimulates}}\pysiglinewithargsret{\sphinxbfcode{\sphinxupquote{up\_stimulates}}}{\emph{uniprot}}{}
\end{fulllineitems}

\index{update\_adjlist() (pypath.legacy.main.PyPath method)@\spxentry{update\_adjlist()}\spxextra{pypath.legacy.main.PyPath method}}

\begin{fulllineitems}
\phantomsection\label{\detokenize{reference:pypath.legacy.main.PyPath.update_adjlist}}\pysiglinewithargsret{\sphinxbfcode{\sphinxupquote{update\_adjlist}}}{\emph{graph=None}, \emph{mode='ALL'}}{}
Creates an adjacency list in a list of sets format.

\end{fulllineitems}

\index{update\_attrs() (pypath.legacy.main.PyPath method)@\spxentry{update\_attrs()}\spxextra{pypath.legacy.main.PyPath method}}

\begin{fulllineitems}
\phantomsection\label{\detokenize{reference:pypath.legacy.main.PyPath.update_attrs}}\pysiglinewithargsret{\sphinxbfcode{\sphinxupquote{update\_attrs}}}{}{}
Updates the node and edge attributes. Note that no data is
donwloaded, mainly updates the dictionaries of attributes
\sphinxcode{\sphinxupquote{pypath.main.PyPath.edgeAttrs}} and
\sphinxcode{\sphinxupquote{pypath.main.PyPath.vertexAttrs}} containing the
attributes names and their correspoding types and initializes
such attributes in the network nodes/edges if they weren’t.

\end{fulllineitems}

\index{update\_cats() (pypath.legacy.main.PyPath method)@\spxentry{update\_cats()}\spxextra{pypath.legacy.main.PyPath method}}

\begin{fulllineitems}
\phantomsection\label{\detokenize{reference:pypath.legacy.main.PyPath.update_cats}}\pysiglinewithargsret{\sphinxbfcode{\sphinxupquote{update\_cats}}}{}{}
Makes sure that the \sphinxcode{\sphinxupquote{pypath.main.PyPath.has\_cats}}
attribute is an up to date {[}set{]} of all categories in the
current network.

\end{fulllineitems}

\index{update\_db\_dict() (pypath.legacy.main.PyPath method)@\spxentry{update\_db\_dict()}\spxextra{pypath.legacy.main.PyPath method}}

\begin{fulllineitems}
\phantomsection\label{\detokenize{reference:pypath.legacy.main.PyPath.update_db_dict}}\pysiglinewithargsret{\sphinxbfcode{\sphinxupquote{update\_db\_dict}}}{}{}
\end{fulllineitems}

\index{update\_pathway\_types() (pypath.legacy.main.PyPath method)@\spxentry{update\_pathway\_types()}\spxextra{pypath.legacy.main.PyPath method}}

\begin{fulllineitems}
\phantomsection\label{\detokenize{reference:pypath.legacy.main.PyPath.update_pathway_types}}\pysiglinewithargsret{\sphinxbfcode{\sphinxupquote{update\_pathway\_types}}}{}{}
Updates the pathway types attribute
(\sphinxcode{\sphinxupquote{pypath.main.PyPath.pathway\_types}}) according to the
loaded resources of the undirected network.

\end{fulllineitems}

\index{update\_pathways() (pypath.legacy.main.PyPath method)@\spxentry{update\_pathways()}\spxextra{pypath.legacy.main.PyPath method}}

\begin{fulllineitems}
\phantomsection\label{\detokenize{reference:pypath.legacy.main.PyPath.update_pathways}}\pysiglinewithargsret{\sphinxbfcode{\sphinxupquote{update\_pathways}}}{}{}
Makes sure that the \sphinxcode{\sphinxupquote{pypath.main.PyPath.pathways}}
attribute is an up to date {[}dict{]} of all pathways and their
sources in the current network.

\end{fulllineitems}

\index{update\_sources() (pypath.legacy.main.PyPath method)@\spxentry{update\_sources()}\spxextra{pypath.legacy.main.PyPath method}}

\begin{fulllineitems}
\phantomsection\label{\detokenize{reference:pypath.legacy.main.PyPath.update_sources}}\pysiglinewithargsret{\sphinxbfcode{\sphinxupquote{update\_sources}}}{}{}
Makes sure that the \sphinxcode{\sphinxupquote{pypath.main.PyPath.sources}}
attribute is an up to date {[}list{]} of all sources in the current
network.

\end{fulllineitems}

\index{update\_summaries() (pypath.legacy.main.PyPath method)@\spxentry{update\_summaries()}\spxextra{pypath.legacy.main.PyPath method}}

\begin{fulllineitems}
\phantomsection\label{\detokenize{reference:pypath.legacy.main.PyPath.update_summaries}}\pysiglinewithargsret{\sphinxbfcode{\sphinxupquote{update\_summaries}}}{}{}
Creates a dict with many summarizing and comparative statistics
about the resources in the current network.
The result will be assigned to the attribute \sphinxcode{\sphinxupquote{summaries}}.

\end{fulllineitems}

\index{update\_vertex\_sources() (pypath.legacy.main.PyPath method)@\spxentry{update\_vertex\_sources()}\spxextra{pypath.legacy.main.PyPath method}}

\begin{fulllineitems}
\phantomsection\label{\detokenize{reference:pypath.legacy.main.PyPath.update_vertex_sources}}\pysiglinewithargsret{\sphinxbfcode{\sphinxupquote{update\_vertex\_sources}}}{}{}
Updates the all the vertex attributes \sphinxcode{\sphinxupquote{'sources'}} and
\sphinxcode{\sphinxupquote{'references'}} according to their related edges (on the
undirected graph).

\end{fulllineitems}

\index{update\_vindex() (pypath.legacy.main.PyPath method)@\spxentry{update\_vindex()}\spxextra{pypath.legacy.main.PyPath method}}

\begin{fulllineitems}
\phantomsection\label{\detokenize{reference:pypath.legacy.main.PyPath.update_vindex}}\pysiglinewithargsret{\sphinxbfcode{\sphinxupquote{update\_vindex}}}{}{}
This is deprecated.

\end{fulllineitems}

\index{update\_vname() (pypath.legacy.main.PyPath method)@\spxentry{update\_vname()}\spxextra{pypath.legacy.main.PyPath method}}

\begin{fulllineitems}
\phantomsection\label{\detokenize{reference:pypath.legacy.main.PyPath.update_vname}}\pysiglinewithargsret{\sphinxbfcode{\sphinxupquote{update\_vname}}}{}{}
Fast lookup of node names and indexes, these are hold in a
{[}list{]} and a {[}dict{]} as well. However, every time new nodes are
added, these should be updated. This function is automatically
called after all operations affecting node indices.

\end{fulllineitems}

\index{ups() (pypath.legacy.main.PyPath method)@\spxentry{ups()}\spxextra{pypath.legacy.main.PyPath method}}

\begin{fulllineitems}
\phantomsection\label{\detokenize{reference:pypath.legacy.main.PyPath.ups}}\pysiglinewithargsret{\sphinxbfcode{\sphinxupquote{ups}}}{\emph{uniprots}}{}
Returns list of \sphinxcode{\sphinxupquote{igraph.Vertex()}} object
for a list of UniProt IDs omitting those
could not be found in the default
undirected graph.

\end{fulllineitems}

\index{v() (pypath.legacy.main.PyPath method)@\spxentry{v()}\spxextra{pypath.legacy.main.PyPath method}}

\begin{fulllineitems}
\phantomsection\label{\detokenize{reference:pypath.legacy.main.PyPath.v}}\pysiglinewithargsret{\sphinxbfcode{\sphinxupquote{v}}}{\emph{identifier}}{}
Returns \sphinxcode{\sphinxupquote{igraph.Vertex()}} object if the identifier
is a valid vertex index in the default undirected graph,
or a UniProt ID or GeneSymbol which can be found in the
default undirected network, otherwise \sphinxcode{\sphinxupquote{None}}.
\begin{description}
\item[{@identifier}] \leavevmode{[}int, str{]}
Vertex index (int) or GeneSymbol (str) or UniProt ID (str) or
\sphinxcode{\sphinxupquote{igraph.Vertex}} object.

\end{description}

\end{fulllineitems}

\index{vertex\_pathways() (pypath.legacy.main.PyPath method)@\spxentry{vertex\_pathways()}\spxextra{pypath.legacy.main.PyPath method}}

\begin{fulllineitems}
\phantomsection\label{\detokenize{reference:pypath.legacy.main.PyPath.vertex_pathways}}\pysiglinewithargsret{\sphinxbfcode{\sphinxupquote{vertex\_pathways}}}{}{}
Some resources assignes interactions some others proteins to
pathways. This function copies pathway annotations from edge
attributes to vertex attributes.

\end{fulllineitems}

\index{vsgs() (pypath.legacy.main.PyPath method)@\spxentry{vsgs()}\spxextra{pypath.legacy.main.PyPath method}}

\begin{fulllineitems}
\phantomsection\label{\detokenize{reference:pypath.legacy.main.PyPath.vsgs}}\pysiglinewithargsret{\sphinxbfcode{\sphinxupquote{vsgs}}}{}{}
Returns a generator sequence of the node names as GeneSymbols
{[}str{]} (from the undirected graph).
\begin{quote}\begin{description}
\item[{Returns}] \leavevmode
(\sphinxstyleemphasis{generator}) \textendash{} Sequence containing the node names as
GeneSymbols {[}str{]}.

\end{description}\end{quote}

\end{fulllineitems}

\index{vsup() (pypath.legacy.main.PyPath method)@\spxentry{vsup()}\spxextra{pypath.legacy.main.PyPath method}}

\begin{fulllineitems}
\phantomsection\label{\detokenize{reference:pypath.legacy.main.PyPath.vsup}}\pysiglinewithargsret{\sphinxbfcode{\sphinxupquote{vsup}}}{}{}
Returns a generator sequence of the node names as UniProt IDs
{[}str{]} (from the undirected graph).
\begin{quote}\begin{description}
\item[{Returns}] \leavevmode
(\sphinxstyleemphasis{generator}) \textendash{} Sequence containing the node names as
UniProt IDs {[}str{]}.

\end{description}\end{quote}

\end{fulllineitems}

\index{wang\_effects() (pypath.legacy.main.PyPath method)@\spxentry{wang\_effects()}\spxextra{pypath.legacy.main.PyPath method}}

\begin{fulllineitems}
\phantomsection\label{\detokenize{reference:pypath.legacy.main.PyPath.wang_effects}}\pysiglinewithargsret{\sphinxbfcode{\sphinxupquote{wang\_effects}}}{\emph{graph=None}}{}
\end{fulllineitems}

\index{write\_table() (pypath.legacy.main.PyPath method)@\spxentry{write\_table()}\spxextra{pypath.legacy.main.PyPath method}}

\begin{fulllineitems}
\phantomsection\label{\detokenize{reference:pypath.legacy.main.PyPath.write_table}}\pysiglinewithargsret{\sphinxbfcode{\sphinxupquote{write\_table}}}{\emph{tbl}, \emph{outfile}, \emph{sep='\textbackslash{}t'}, \emph{cut=None}, \emph{colnames=True}, \emph{rownames=True}}{}
Writes a given table to a file.
\begin{quote}\begin{description}
\item[{Parameters}] \leavevmode\begin{itemize}
\item {} 
\sphinxstyleliteralstrong{\sphinxupquote{tbl}} (\sphinxstyleliteralemphasis{\sphinxupquote{dict}}) \textendash{} Contains the data of the table. It is assumed that keys are
the row names {[}str{]} and the values, well, values. Column
names (if any) are defined with the key \sphinxcode{\sphinxupquote{'header'}}.

\item {} 
\sphinxstyleliteralstrong{\sphinxupquote{outfile}} (\sphinxstyleliteralemphasis{\sphinxupquote{str}}) \textendash{} File name where to save the table. The file will be saved
under the object’s \sphinxcode{\sphinxupquote{pypath.main.PyPath.outdir}}
(\sphinxcode{\sphinxupquote{'results'}} by default).

\item {} 
\sphinxstyleliteralstrong{\sphinxupquote{sep}} (\sphinxstyleliteralemphasis{\sphinxupquote{str}}) \textendash{} Optional, \sphinxcode{\sphinxupquote{'       '}} (tab) by default. Specifies the separator
for the file.

\item {} 
\sphinxstyleliteralstrong{\sphinxupquote{cut}} (\sphinxstyleliteralemphasis{\sphinxupquote{int}}) \textendash{} Optional, \sphinxcode{\sphinxupquote{None}} by default. Specifies the maximum number
of characters for the row names.

\item {} 
\sphinxstyleliteralstrong{\sphinxupquote{colnames}} (\sphinxstyleliteralemphasis{\sphinxupquote{bool}}) \textendash{} Optional, \sphinxcode{\sphinxupquote{True}} by default. Specifies whether to write
the column names in the file or not.

\item {} 
\sphinxstyleliteralstrong{\sphinxupquote{rownames}} (\sphinxstyleliteralemphasis{\sphinxupquote{bool}}) \textendash{} Optional, \sphinxcode{\sphinxupquote{True}} by default. Specifies whether to write
the row names in the file or not.

\end{itemize}

\end{description}\end{quote}

\end{fulllineitems}


\end{fulllineitems}

\index{Direction (class in pypath.legacy.main)@\spxentry{Direction}\spxextra{class in pypath.legacy.main}}

\begin{fulllineitems}
\phantomsection\label{\detokenize{reference:pypath.legacy.main.Direction}}\pysiglinewithargsret{\sphinxbfcode{\sphinxupquote{class }}\sphinxcode{\sphinxupquote{pypath.legacy.main.}}\sphinxbfcode{\sphinxupquote{Direction}}}{\emph{id\_a}, \emph{id\_b}}{}
This is a \sphinxstyleemphasis{legacy} object for handling directionality information
associated with unique pairs of interacting molecular entities.
The py:class:\sphinxtitleref{pypath.interaction.Interaction} available in the \sphinxcode{\sphinxupquote{attrs}}
edge attribute of the legacy \sphinxcode{\sphinxupquote{pypath.main.PyPath{}`}} object
provides a clearer and much more versatile interface. This object will
be removed at some point, we don’t recommend to build applications by
using it.

Object storing directionality information of an edge. Also includes
information about the reverse direction, mode of regulation and
sources of that information.
\begin{quote}\begin{description}
\item[{Parameters}] \leavevmode\begin{itemize}
\item {} 
\sphinxstyleliteralstrong{\sphinxupquote{id\_a}} (\sphinxstyleliteralemphasis{\sphinxupquote{str}}) \textendash{} Name of the source node.

\item {} 
\sphinxstyleliteralstrong{\sphinxupquote{id\_b}} (\sphinxstyleliteralemphasis{\sphinxupquote{str}}) \textendash{} Name of the target node.

\end{itemize}

\item[{Variables}] \leavevmode\begin{itemize}
\item {} 
\sphinxstyleliteralstrong{\sphinxupquote{dirs}} (\sphinxstyleliteralemphasis{\sphinxupquote{dict}}) \textendash{} Dictionary containing the presence of directionality of the
given edge. Keys are \sphinxstyleemphasis{straight}, \sphinxstyleemphasis{reverse} and \sphinxcode{\sphinxupquote{'undirected'}}
and their values denote the presence/absence {[}bool{]}.

\item {} 
\sphinxstyleliteralstrong{\sphinxupquote{negative}} (\sphinxstyleliteralemphasis{\sphinxupquote{dict}}) \textendash{} Dictionary contianing the presence/absence {[}bool{]} of negative
interactions for both \sphinxcode{\sphinxupquote{straight}} and \sphinxcode{\sphinxupquote{reverse}}
directions.

\item {} 
\sphinxstyleliteralstrong{\sphinxupquote{negative\_sources}} (\sphinxstyleliteralemphasis{\sphinxupquote{dict}}) \textendash{} Contains the resource names {[}str{]} supporting a negative
interaction on \sphinxcode{\sphinxupquote{straight}} and \sphinxcode{\sphinxupquote{reverse}}
directions.

\item {} 
\sphinxstyleliteralstrong{\sphinxupquote{nodes}} (\sphinxstyleliteralemphasis{\sphinxupquote{list}}) \textendash{} Contains the node names {[}str{]} sorted alphabetically (\sphinxstyleemphasis{id\_a},
\sphinxstyleemphasis{id\_b}).

\item {} 
\sphinxstyleliteralstrong{\sphinxupquote{positive}} (\sphinxstyleliteralemphasis{\sphinxupquote{dict}}) \textendash{} Dictionary contianing the presence/absence {[}bool{]} of positive
interactions for both \sphinxcode{\sphinxupquote{straight}} and \sphinxcode{\sphinxupquote{reverse}}
directions.

\item {} 
\sphinxstyleliteralstrong{\sphinxupquote{positive\_sources}} (\sphinxstyleliteralemphasis{\sphinxupquote{dict}}) \textendash{} Contains the resource names {[}str{]} supporting a positive
interaction on \sphinxcode{\sphinxupquote{straight}} and \sphinxcode{\sphinxupquote{reverse}}
directions.

\item {} 
\sphinxstyleliteralstrong{\sphinxupquote{reverse}} (\sphinxstyleliteralemphasis{\sphinxupquote{tuple}}) \textendash{} Contains the node names {[}str{]} in reverse order e.g. (\sphinxstyleemphasis{id\_b},
\sphinxstyleemphasis{id\_a}).

\item {} 
{\hyperref[\detokenize{reference:pypath.core.interaction.InteractionDataFrameRecord.sources}]{\sphinxcrossref{\sphinxstyleliteralstrong{\sphinxupquote{sources}}}}} (\sphinxstyleliteralemphasis{\sphinxupquote{dict}}) \textendash{} Contains the resource names {[}str{]} of a given edge for each
directionality (\sphinxcode{\sphinxupquote{straight}}, \sphinxcode{\sphinxupquote{reverse}} and
\sphinxcode{\sphinxupquote{'undirected'}}). Values are sets containing the names of those
resources supporting such directionality.

\item {} 
\sphinxstyleliteralstrong{\sphinxupquote{straight}} (\sphinxstyleliteralemphasis{\sphinxupquote{tuple}}) \textendash{} Contains the node names {[}str{]} in the original order e.g.
(\sphinxstyleemphasis{id\_a}, \sphinxstyleemphasis{id\_b}).

\end{itemize}

\end{description}\end{quote}
\index{check\_nodes() (pypath.legacy.main.Direction method)@\spxentry{check\_nodes()}\spxextra{pypath.legacy.main.Direction method}}

\begin{fulllineitems}
\phantomsection\label{\detokenize{reference:pypath.legacy.main.Direction.check_nodes}}\pysiglinewithargsret{\sphinxbfcode{\sphinxupquote{check\_nodes}}}{\emph{nodes}}{}
Checks if \sphinxstyleemphasis{nodes} is contained in the edge.
\begin{quote}\begin{description}
\item[{Parameters}] \leavevmode
\sphinxstyleliteralstrong{\sphinxupquote{nodes}} (\sphinxstyleliteralemphasis{\sphinxupquote{list}}) \textendash{} Or {[}tuple{]}, contains the names of the nodes to be checked.

\item[{Returns}] \leavevmode
(\sphinxstyleemphasis{bool}) \textendash{} \sphinxcode{\sphinxupquote{True}} if all elements in \sphinxstyleemphasis{nodes} are
contained in the object \sphinxcode{\sphinxupquote{nodes}} list.

\end{description}\end{quote}

\end{fulllineitems}

\index{check\_param() (pypath.legacy.main.Direction method)@\spxentry{check\_param()}\spxextra{pypath.legacy.main.Direction method}}

\begin{fulllineitems}
\phantomsection\label{\detokenize{reference:pypath.legacy.main.Direction.check_param}}\pysiglinewithargsret{\sphinxbfcode{\sphinxupquote{check\_param}}}{\emph{di}}{}
Checks if \sphinxstyleemphasis{di} is \sphinxcode{\sphinxupquote{'undirected'}} or contains the nodes of
the current edge. Used internally to check that \sphinxstyleemphasis{di} is a valid
key for the object attributes declared on dictionaries.
\begin{quote}\begin{description}
\item[{Parameters}] \leavevmode
\sphinxstyleliteralstrong{\sphinxupquote{di}} (\sphinxstyleliteralemphasis{\sphinxupquote{tuple}}) \textendash{} Or {[}str{]}, key to be tested for validity.

\item[{Returns}] \leavevmode
\begin{description}
\item[{(\sphinxstyleemphasis{bool}) \textendash{} \sphinxcode{\sphinxupquote{True}} if \sphinxstyleemphasis{di} is \sphinxcode{\sphinxupquote{'undirected'}} or a tuple}] \leavevmode
of node names contained in the edge, \sphinxcode{\sphinxupquote{False}} otherwise.

\end{description}


\end{description}\end{quote}

\end{fulllineitems}

\index{consensus\_edges() (pypath.legacy.main.Direction method)@\spxentry{consensus\_edges()}\spxextra{pypath.legacy.main.Direction method}}

\begin{fulllineitems}
\phantomsection\label{\detokenize{reference:pypath.legacy.main.Direction.consensus_edges}}\pysiglinewithargsret{\sphinxbfcode{\sphinxupquote{consensus\_edges}}}{}{}
Infers the consensus edge(s) according to the number of
supporting sources. This includes direction and sign.
\begin{quote}\begin{description}
\item[{Returns}] \leavevmode
(\sphinxstyleemphasis{list}) \textendash{} Contains the consensus edge(s) along with the
consensus sign. If there is no major directionality, both
are returned. The structure is as follows:
\sphinxcode{\sphinxupquote{{[}'\textless{}source\textgreater{}', '\textless{}target\textgreater{}', '\textless{}(un)directed\textgreater{}', '\textless{}sign\textgreater{}'{]}}}

\end{description}\end{quote}

\end{fulllineitems}

\index{get\_dir() (pypath.legacy.main.Direction method)@\spxentry{get\_dir()}\spxextra{pypath.legacy.main.Direction method}}

\begin{fulllineitems}
\phantomsection\label{\detokenize{reference:pypath.legacy.main.Direction.get_dir}}\pysiglinewithargsret{\sphinxbfcode{\sphinxupquote{get\_dir}}}{\emph{direction}, \emph{sources=False}}{}
Returns the state (or \sphinxstyleemphasis{sources} if specified) of the given
\sphinxstyleemphasis{direction}.
\begin{quote}\begin{description}
\item[{Parameters}] \leavevmode\begin{itemize}
\item {} 
\sphinxstyleliteralstrong{\sphinxupquote{direction}} (\sphinxstyleliteralemphasis{\sphinxupquote{tuple}}) \textendash{} Or {[}str{]} (if \sphinxcode{\sphinxupquote{'undirected'}}). Pair of nodes from which
direction information is to be retrieved.

\item {} 
\sphinxstyleliteralstrong{\sphinxupquote{sources}} (\sphinxstyleliteralemphasis{\sphinxupquote{bool}}) \textendash{} Optional, \sphinxcode{\sphinxupquote{'False'}} by default. Specifies if the
\sphinxcode{\sphinxupquote{sources}} information of the given direction is to
be retrieved instead.

\end{itemize}

\item[{Returns}] \leavevmode
(\sphinxstyleemphasis{bool} or \sphinxstyleemphasis{set}) \textendash{} (if \sphinxcode{\sphinxupquote{sources=True}}). Presence/absence
of the requested direction (or the list of sources if
specified). Returns \sphinxcode{\sphinxupquote{None}} if \sphinxstyleemphasis{direction} is not valid.

\end{description}\end{quote}

\end{fulllineitems}

\index{get\_direction() (pypath.legacy.main.Direction method)@\spxentry{get\_direction()}\spxextra{pypath.legacy.main.Direction method}}

\begin{fulllineitems}
\phantomsection\label{\detokenize{reference:pypath.legacy.main.Direction.get_direction}}\pysiglinewithargsret{\sphinxbfcode{\sphinxupquote{get\_direction}}}{\emph{direction}, \emph{sources=False}}{}
Returns the state (or \sphinxstyleemphasis{sources} if specified) of the given
\sphinxstyleemphasis{direction}.
\begin{quote}\begin{description}
\item[{Parameters}] \leavevmode\begin{itemize}
\item {} 
\sphinxstyleliteralstrong{\sphinxupquote{direction}} (\sphinxstyleliteralemphasis{\sphinxupquote{tuple}}) \textendash{} Or {[}str{]} (if \sphinxcode{\sphinxupquote{'undirected'}}). Pair of nodes from which
direction information is to be retrieved.

\item {} 
\sphinxstyleliteralstrong{\sphinxupquote{sources}} (\sphinxstyleliteralemphasis{\sphinxupquote{bool}}) \textendash{} Optional, \sphinxcode{\sphinxupquote{'False'}} by default. Specifies if the
\sphinxcode{\sphinxupquote{sources}} information of the given direction is to
be retrieved instead.

\end{itemize}

\item[{Returns}] \leavevmode
(\sphinxstyleemphasis{bool} or \sphinxstyleemphasis{set}) \textendash{} (if \sphinxcode{\sphinxupquote{sources=True}}). Presence/absence
of the requested direction (or the list of sources if
specified). Returns \sphinxcode{\sphinxupquote{None}} if \sphinxstyleemphasis{direction} is not valid.

\end{description}\end{quote}

\end{fulllineitems}

\index{get\_directions() (pypath.legacy.main.Direction method)@\spxentry{get\_directions()}\spxextra{pypath.legacy.main.Direction method}}

\begin{fulllineitems}
\phantomsection\label{\detokenize{reference:pypath.legacy.main.Direction.get_directions}}\pysiglinewithargsret{\sphinxbfcode{\sphinxupquote{get\_directions}}}{\emph{src}, \emph{tgt}, \emph{sources=False}}{}
Returns all directions with boolean values or list of sources.
\begin{quote}\begin{description}
\item[{Parameters}] \leavevmode\begin{itemize}
\item {} 
\sphinxstyleliteralstrong{\sphinxupquote{src}} (\sphinxstyleliteralemphasis{\sphinxupquote{str}}) \textendash{} Source node.

\item {} 
\sphinxstyleliteralstrong{\sphinxupquote{tgt}} (\sphinxstyleliteralemphasis{\sphinxupquote{str}}) \textendash{} Target node.

\item {} 
\sphinxstyleliteralstrong{\sphinxupquote{sources}} (\sphinxstyleliteralemphasis{\sphinxupquote{bool}}) \textendash{} Optional, \sphinxcode{\sphinxupquote{False}} by default. Specifies whether to return
the \sphinxcode{\sphinxupquote{sources}} attribute instead of \sphinxcode{\sphinxupquote{dirs}}.

\end{itemize}

\item[{Returns}] \leavevmode
Contains the \sphinxcode{\sphinxupquote{dirs}} (or \sphinxcode{\sphinxupquote{sources}} if
specified) of the given edge.

\end{description}\end{quote}

\end{fulllineitems}

\index{get\_dirs() (pypath.legacy.main.Direction method)@\spxentry{get\_dirs()}\spxextra{pypath.legacy.main.Direction method}}

\begin{fulllineitems}
\phantomsection\label{\detokenize{reference:pypath.legacy.main.Direction.get_dirs}}\pysiglinewithargsret{\sphinxbfcode{\sphinxupquote{get\_dirs}}}{\emph{src}, \emph{tgt}, \emph{sources=False}}{}
Returns all directions with boolean values or list of sources.
\begin{quote}\begin{description}
\item[{Parameters}] \leavevmode\begin{itemize}
\item {} 
\sphinxstyleliteralstrong{\sphinxupquote{src}} (\sphinxstyleliteralemphasis{\sphinxupquote{str}}) \textendash{} Source node.

\item {} 
\sphinxstyleliteralstrong{\sphinxupquote{tgt}} (\sphinxstyleliteralemphasis{\sphinxupquote{str}}) \textendash{} Target node.

\item {} 
\sphinxstyleliteralstrong{\sphinxupquote{sources}} (\sphinxstyleliteralemphasis{\sphinxupquote{bool}}) \textendash{} Optional, \sphinxcode{\sphinxupquote{False}} by default. Specifies whether to return
the \sphinxcode{\sphinxupquote{sources}} attribute instead of \sphinxcode{\sphinxupquote{dirs}}.

\end{itemize}

\item[{Returns}] \leavevmode
Contains the \sphinxcode{\sphinxupquote{dirs}} (or \sphinxcode{\sphinxupquote{sources}} if
specified) of the given edge.

\end{description}\end{quote}

\end{fulllineitems}

\index{get\_sign() (pypath.legacy.main.Direction method)@\spxentry{get\_sign()}\spxextra{pypath.legacy.main.Direction method}}

\begin{fulllineitems}
\phantomsection\label{\detokenize{reference:pypath.legacy.main.Direction.get_sign}}\pysiglinewithargsret{\sphinxbfcode{\sphinxupquote{get\_sign}}}{\emph{direction}, \emph{sign=None}, \emph{sources=False}}{}
Retrieves the sign information of the edge in the given
diretion. If specified in \sphinxstyleemphasis{sign}, only that sign’s information
will be retrieved. If specified in \sphinxstyleemphasis{sources}, the sources of
that information will be retrieved instead.
\begin{quote}\begin{description}
\item[{Parameters}] \leavevmode\begin{itemize}
\item {} 
\sphinxstyleliteralstrong{\sphinxupquote{direction}} (\sphinxstyleliteralemphasis{\sphinxupquote{tuple}}) \textendash{} Contains the pair of nodes specifying the directionality of
the edge from which th information is to be retrieved.

\item {} 
\sphinxstyleliteralstrong{\sphinxupquote{sign}} (\sphinxstyleliteralemphasis{\sphinxupquote{str}}) \textendash{} Optional, \sphinxcode{\sphinxupquote{None}} by default. Denotes whether to retrieve
the \sphinxcode{\sphinxupquote{'positive'}} or \sphinxcode{\sphinxupquote{'negative'}} specific information.

\item {} 
\sphinxstyleliteralstrong{\sphinxupquote{sources}} (\sphinxstyleliteralemphasis{\sphinxupquote{bool}}) \textendash{} Optional, \sphinxcode{\sphinxupquote{False}} by default. Specifies whether to return
the sources instead of sign.

\end{itemize}

\item[{Returns}] \leavevmode
(\sphinxstyleemphasis{list}) \textendash{} If \sphinxcode{\sphinxupquote{sign=None}} containing {[}bool{]} values
denoting the presence of positive and negative sign on that
direction, if \sphinxcode{\sphinxupquote{sources=True}} the {[}set{]} of sources for each
of them will be returned instead. If \sphinxstyleemphasis{sign} is specified,
returns {[}bool{]} or {[}set{]} (if \sphinxcode{\sphinxupquote{sources=True}}) of that
specific direction and sign.

\end{description}\end{quote}

\end{fulllineitems}

\index{has\_sign() (pypath.legacy.main.Direction method)@\spxentry{has\_sign()}\spxextra{pypath.legacy.main.Direction method}}

\begin{fulllineitems}
\phantomsection\label{\detokenize{reference:pypath.legacy.main.Direction.has_sign}}\pysiglinewithargsret{\sphinxbfcode{\sphinxupquote{has\_sign}}}{\emph{direction=None}, \emph{resources=None}}{}
Checks whether the edge (or for a specific \sphinxstyleemphasis{direction}) has
any signed information (about positive/negative interactions).
\begin{quote}\begin{description}
\item[{Parameters}] \leavevmode
\sphinxstyleliteralstrong{\sphinxupquote{direction}} (\sphinxstyleliteralemphasis{\sphinxupquote{tuple}}) \textendash{} Optional, \sphinxcode{\sphinxupquote{None}} by default. If specified, only the
information of that direction is checked for sign.

\item[{Returns}] \leavevmode
\begin{description}
\item[{(\sphinxstyleemphasis{bool}) \textendash{} \sphinxcode{\sphinxupquote{True}} if there exist any information on the}] \leavevmode
sign of the interaction, \sphinxcode{\sphinxupquote{False}} otherwise.

\end{description}


\end{description}\end{quote}

\end{fulllineitems}

\index{is\_directed() (pypath.legacy.main.Direction method)@\spxentry{is\_directed()}\spxextra{pypath.legacy.main.Direction method}}

\begin{fulllineitems}
\phantomsection\label{\detokenize{reference:pypath.legacy.main.Direction.is_directed}}\pysiglinewithargsret{\sphinxbfcode{\sphinxupquote{is\_directed}}}{}{}
Checks if edge has any directionality information.
\begin{quote}\begin{description}
\item[{Returns}] \leavevmode
(\sphinxstyleemphasis{bool}) \textendash{} Returns \sphinxcode{\sphinxupquote{True}} if any of the \sphinxcode{\sphinxupquote{dirs}}
attribute values is \sphinxcode{\sphinxupquote{True}} (except \sphinxcode{\sphinxupquote{'undirected'}}),
\sphinxcode{\sphinxupquote{False}} otherwise.

\end{description}\end{quote}

\end{fulllineitems}

\index{is\_directed\_by\_resources() (pypath.legacy.main.Direction method)@\spxentry{is\_directed\_by\_resources()}\spxextra{pypath.legacy.main.Direction method}}

\begin{fulllineitems}
\phantomsection\label{\detokenize{reference:pypath.legacy.main.Direction.is_directed_by_resources}}\pysiglinewithargsret{\sphinxbfcode{\sphinxupquote{is\_directed\_by\_resources}}}{\emph{resources=None}}{}
Checks if edge has any directionality information from some
resource(s).
\begin{quote}\begin{description}
\item[{Returns}] \leavevmode
(\sphinxstyleemphasis{bool}) \textendash{} Returns \sphinxcode{\sphinxupquote{True}} if any of the \sphinxcode{\sphinxupquote{dirs}}
attribute values is \sphinxcode{\sphinxupquote{True}} (except \sphinxcode{\sphinxupquote{'undirected'}}),
\sphinxcode{\sphinxupquote{False}} otherwise.

\end{description}\end{quote}

\end{fulllineitems}

\index{is\_inhibition() (pypath.legacy.main.Direction method)@\spxentry{is\_inhibition()}\spxextra{pypath.legacy.main.Direction method}}

\begin{fulllineitems}
\phantomsection\label{\detokenize{reference:pypath.legacy.main.Direction.is_inhibition}}\pysiglinewithargsret{\sphinxbfcode{\sphinxupquote{is\_inhibition}}}{\emph{direction=None}, \emph{resources=None}}{}
Checks if any (or for a specific \sphinxstyleemphasis{direction}) interaction is
inhibition (negative interaction).
\begin{quote}\begin{description}
\item[{Parameters}] \leavevmode
\sphinxstyleliteralstrong{\sphinxupquote{direction}} (\sphinxstyleliteralemphasis{\sphinxupquote{tuple}}) \textendash{} Optional, \sphinxcode{\sphinxupquote{None}} by default. If specified, checks the
\sphinxcode{\sphinxupquote{negative}} attribute of that specific
directionality. If not specified, checks both.

\item[{Returns}] \leavevmode
(\sphinxstyleemphasis{bool}) \textendash{} \sphinxcode{\sphinxupquote{True}} if any interaction (or the specified
\sphinxstyleemphasis{direction}) is inhibitory (negative).

\end{description}\end{quote}

\end{fulllineitems}

\index{is\_mutual() (pypath.legacy.main.Direction method)@\spxentry{is\_mutual()}\spxextra{pypath.legacy.main.Direction method}}

\begin{fulllineitems}
\phantomsection\label{\detokenize{reference:pypath.legacy.main.Direction.is_mutual}}\pysiglinewithargsret{\sphinxbfcode{\sphinxupquote{is\_mutual}}}{\emph{resources=None}}{}
Checks if the edge has mutual directions (both A\textendash{}\textgreater{}B and B\textendash{}\textgreater{}A).

\end{fulllineitems}

\index{is\_mutual\_by\_resources() (pypath.legacy.main.Direction method)@\spxentry{is\_mutual\_by\_resources()}\spxextra{pypath.legacy.main.Direction method}}

\begin{fulllineitems}
\phantomsection\label{\detokenize{reference:pypath.legacy.main.Direction.is_mutual_by_resources}}\pysiglinewithargsret{\sphinxbfcode{\sphinxupquote{is\_mutual\_by\_resources}}}{\emph{resources=None}}{}
Checks if the edge has mutual directions (both A\textendash{}\textgreater{}B and B\textendash{}\textgreater{}A)
according to some resource(s).

\end{fulllineitems}

\index{is\_stimulation() (pypath.legacy.main.Direction method)@\spxentry{is\_stimulation()}\spxextra{pypath.legacy.main.Direction method}}

\begin{fulllineitems}
\phantomsection\label{\detokenize{reference:pypath.legacy.main.Direction.is_stimulation}}\pysiglinewithargsret{\sphinxbfcode{\sphinxupquote{is\_stimulation}}}{\emph{direction=None}, \emph{resources=None}}{}
Checks if any (or for a specific \sphinxstyleemphasis{direction}) interaction is
activation (positive interaction).
\begin{quote}\begin{description}
\item[{Parameters}] \leavevmode
\sphinxstyleliteralstrong{\sphinxupquote{direction}} (\sphinxstyleliteralemphasis{\sphinxupquote{tuple}}) \textendash{} Optional, \sphinxcode{\sphinxupquote{None}} by default. If specified, checks the
\sphinxcode{\sphinxupquote{positive}} attribute of that specific
directionality. If not specified, checks both.

\item[{Returns}] \leavevmode
(\sphinxstyleemphasis{bool}) \textendash{} \sphinxcode{\sphinxupquote{True}} if any interaction (or the specified
\sphinxstyleemphasis{direction}) is activatory (positive).

\end{description}\end{quote}

\end{fulllineitems}

\index{majority\_dir() (pypath.legacy.main.Direction method)@\spxentry{majority\_dir()}\spxextra{pypath.legacy.main.Direction method}}

\begin{fulllineitems}
\phantomsection\label{\detokenize{reference:pypath.legacy.main.Direction.majority_dir}}\pysiglinewithargsret{\sphinxbfcode{\sphinxupquote{majority\_dir}}}{}{}
Infers which is the major directionality of the edge by number
of supporting sources.
\begin{quote}\begin{description}
\item[{Returns}] \leavevmode
(\sphinxstyleemphasis{tuple}) \textendash{} Contains the pair of nodes denoting the
consensus directionality. If the number of sources on both
directions is equal, \sphinxcode{\sphinxupquote{None}} is returned. If there is no
directionality information, \sphinxcode{\sphinxupquote{'undirected'{}`}} will be
returned.

\end{description}\end{quote}

\end{fulllineitems}

\index{majority\_sign() (pypath.legacy.main.Direction method)@\spxentry{majority\_sign()}\spxextra{pypath.legacy.main.Direction method}}

\begin{fulllineitems}
\phantomsection\label{\detokenize{reference:pypath.legacy.main.Direction.majority_sign}}\pysiglinewithargsret{\sphinxbfcode{\sphinxupquote{majority\_sign}}}{}{}
Infers which is the major sign (activation/inhibition) of the
edge by number of supporting sources on both directions.
\begin{quote}\begin{description}
\item[{Returns}] \leavevmode
(\sphinxstyleemphasis{dict}) \textendash{} Keys are the node tuples on both directions
(\sphinxcode{\sphinxupquote{straight}}/\sphinxcode{\sphinxupquote{reverse}}) and values can be
either \sphinxcode{\sphinxupquote{None}} if that direction has no sign information or
a list of two {[}bool{]} elements corresponding to majority of
positive and majority of negative support. In case both
elements of the list are \sphinxcode{\sphinxupquote{True}}, this means the number of
supporting sources for both signs in that direction is
equal.

\end{description}\end{quote}

\end{fulllineitems}

\index{merge() (pypath.legacy.main.Direction method)@\spxentry{merge()}\spxextra{pypath.legacy.main.Direction method}}

\begin{fulllineitems}
\phantomsection\label{\detokenize{reference:pypath.legacy.main.Direction.merge}}\pysiglinewithargsret{\sphinxbfcode{\sphinxupquote{merge}}}{\emph{other}}{}
Merges current edge with another (if and only if they are the
same class and contain the same nodes). Updates the attributes
\sphinxcode{\sphinxupquote{dirs}}, \sphinxcode{\sphinxupquote{sources}}, \sphinxcode{\sphinxupquote{positive}},
\sphinxcode{\sphinxupquote{negative}}, \sphinxcode{\sphinxupquote{positive\_sources}} and
\sphinxcode{\sphinxupquote{negative\_sources}}.
\begin{quote}\begin{description}
\item[{Parameters}] \leavevmode
\sphinxstyleliteralstrong{\sphinxupquote{other}} (\sphinxstyleliteralemphasis{\sphinxupquote{pypath.main.Direction}}) \textendash{} The new edge object to be merged with the current one.

\end{description}\end{quote}

\end{fulllineitems}

\index{negative\_reverse() (pypath.legacy.main.Direction method)@\spxentry{negative\_reverse()}\spxextra{pypath.legacy.main.Direction method}}

\begin{fulllineitems}
\phantomsection\label{\detokenize{reference:pypath.legacy.main.Direction.negative_reverse}}\pysiglinewithargsret{\sphinxbfcode{\sphinxupquote{negative\_reverse}}}{}{}
Checks if the \sphinxcode{\sphinxupquote{reverse}} directionality is a negative
interaction.
\begin{quote}\begin{description}
\item[{Returns}] \leavevmode
(\sphinxstyleemphasis{bool}) \textendash{} \sphinxcode{\sphinxupquote{True}} if there is supporting information on
the \sphinxcode{\sphinxupquote{reverse}} direction of the edge as inhibition.
\sphinxcode{\sphinxupquote{False}} otherwise.

\end{description}\end{quote}

\end{fulllineitems}

\index{negative\_sources\_reverse() (pypath.legacy.main.Direction method)@\spxentry{negative\_sources\_reverse()}\spxextra{pypath.legacy.main.Direction method}}

\begin{fulllineitems}
\phantomsection\label{\detokenize{reference:pypath.legacy.main.Direction.negative_sources_reverse}}\pysiglinewithargsret{\sphinxbfcode{\sphinxupquote{negative\_sources\_reverse}}}{}{}
Retrieves the list of sources for the \sphinxcode{\sphinxupquote{reverse}}
direction and negative sign.
\begin{quote}\begin{description}
\item[{Returns}] \leavevmode
(\sphinxstyleemphasis{set}) \textendash{} Contains the names of the sources supporting the
\sphinxcode{\sphinxupquote{reverse}} directionality of the edge with a
negative sign.

\end{description}\end{quote}

\end{fulllineitems}

\index{negative\_sources\_straight() (pypath.legacy.main.Direction method)@\spxentry{negative\_sources\_straight()}\spxextra{pypath.legacy.main.Direction method}}

\begin{fulllineitems}
\phantomsection\label{\detokenize{reference:pypath.legacy.main.Direction.negative_sources_straight}}\pysiglinewithargsret{\sphinxbfcode{\sphinxupquote{negative\_sources\_straight}}}{}{}
Retrieves the list of sources for the \sphinxcode{\sphinxupquote{straight}}
direction and negative sign.
\begin{quote}\begin{description}
\item[{Returns}] \leavevmode
(\sphinxstyleemphasis{set}) \textendash{} Contains the names of the sources supporting the
\sphinxcode{\sphinxupquote{straight}} directionality of the edge with a
negative sign.

\end{description}\end{quote}

\end{fulllineitems}

\index{negative\_straight() (pypath.legacy.main.Direction method)@\spxentry{negative\_straight()}\spxextra{pypath.legacy.main.Direction method}}

\begin{fulllineitems}
\phantomsection\label{\detokenize{reference:pypath.legacy.main.Direction.negative_straight}}\pysiglinewithargsret{\sphinxbfcode{\sphinxupquote{negative\_straight}}}{}{}
Checks if the \sphinxcode{\sphinxupquote{straight}} directionality is a negative
interaction.
\begin{quote}\begin{description}
\item[{Returns}] \leavevmode
(\sphinxstyleemphasis{bool}) \textendash{} \sphinxcode{\sphinxupquote{True}} if there is supporting information on
the \sphinxcode{\sphinxupquote{straight}} direction of the edge as inhibition.
\sphinxcode{\sphinxupquote{False}} otherwise.

\end{description}\end{quote}

\end{fulllineitems}

\index{positive\_reverse() (pypath.legacy.main.Direction method)@\spxentry{positive\_reverse()}\spxextra{pypath.legacy.main.Direction method}}

\begin{fulllineitems}
\phantomsection\label{\detokenize{reference:pypath.legacy.main.Direction.positive_reverse}}\pysiglinewithargsret{\sphinxbfcode{\sphinxupquote{positive\_reverse}}}{}{}
Checks if the \sphinxcode{\sphinxupquote{reverse}} directionality is a positive
interaction.
\begin{quote}\begin{description}
\item[{Returns}] \leavevmode
(\sphinxstyleemphasis{bool}) \textendash{} \sphinxcode{\sphinxupquote{True}} if there is supporting information on
the \sphinxcode{\sphinxupquote{reverse}} direction of the edge as activation.
\sphinxcode{\sphinxupquote{False}} otherwise.

\end{description}\end{quote}

\end{fulllineitems}

\index{positive\_sources\_reverse() (pypath.legacy.main.Direction method)@\spxentry{positive\_sources\_reverse()}\spxextra{pypath.legacy.main.Direction method}}

\begin{fulllineitems}
\phantomsection\label{\detokenize{reference:pypath.legacy.main.Direction.positive_sources_reverse}}\pysiglinewithargsret{\sphinxbfcode{\sphinxupquote{positive\_sources\_reverse}}}{}{}
Retrieves the list of sources for the \sphinxcode{\sphinxupquote{reverse}}
direction and positive sign.
\begin{quote}\begin{description}
\item[{Returns}] \leavevmode
(\sphinxstyleemphasis{set}) \textendash{} Contains the names of the sources supporting the
\sphinxcode{\sphinxupquote{reverse}} directionality of the edge with a
positive sign.

\end{description}\end{quote}

\end{fulllineitems}

\index{positive\_sources\_straight() (pypath.legacy.main.Direction method)@\spxentry{positive\_sources\_straight()}\spxextra{pypath.legacy.main.Direction method}}

\begin{fulllineitems}
\phantomsection\label{\detokenize{reference:pypath.legacy.main.Direction.positive_sources_straight}}\pysiglinewithargsret{\sphinxbfcode{\sphinxupquote{positive\_sources\_straight}}}{}{}
Retrieves the list of sources for the \sphinxcode{\sphinxupquote{straight}}
direction and positive sign.
\begin{quote}\begin{description}
\item[{Returns}] \leavevmode
(\sphinxstyleemphasis{set}) \textendash{} Contains the names of the sources supporting the
\sphinxcode{\sphinxupquote{straight}} directionality of the edge with a
positive sign.

\end{description}\end{quote}

\end{fulllineitems}

\index{positive\_straight() (pypath.legacy.main.Direction method)@\spxentry{positive\_straight()}\spxextra{pypath.legacy.main.Direction method}}

\begin{fulllineitems}
\phantomsection\label{\detokenize{reference:pypath.legacy.main.Direction.positive_straight}}\pysiglinewithargsret{\sphinxbfcode{\sphinxupquote{positive\_straight}}}{}{}
Checks if the \sphinxcode{\sphinxupquote{straight}} directionality is a positive
interaction.
\begin{quote}\begin{description}
\item[{Returns}] \leavevmode
(\sphinxstyleemphasis{bool}) \textendash{} \sphinxcode{\sphinxupquote{True}} if there is supporting information on
the \sphinxcode{\sphinxupquote{straight}} direction of the edge as activation.
\sphinxcode{\sphinxupquote{False}} otherwise.

\end{description}\end{quote}

\end{fulllineitems}

\index{reload() (pypath.legacy.main.Direction method)@\spxentry{reload()}\spxextra{pypath.legacy.main.Direction method}}

\begin{fulllineitems}
\phantomsection\label{\detokenize{reference:pypath.legacy.main.Direction.reload}}\pysiglinewithargsret{\sphinxbfcode{\sphinxupquote{reload}}}{}{}
Reloads the object from the module level.

\end{fulllineitems}

\index{set\_dir() (pypath.legacy.main.Direction method)@\spxentry{set\_dir()}\spxextra{pypath.legacy.main.Direction method}}

\begin{fulllineitems}
\phantomsection\label{\detokenize{reference:pypath.legacy.main.Direction.set_dir}}\pysiglinewithargsret{\sphinxbfcode{\sphinxupquote{set\_dir}}}{\emph{direction}, \emph{source}}{}
Adds directionality information with the corresponding data
source named. Modifies self attributes \sphinxcode{\sphinxupquote{dirs}} and
\sphinxcode{\sphinxupquote{sources}}.
\begin{quote}\begin{description}
\item[{Parameters}] \leavevmode\begin{itemize}
\item {} 
\sphinxstyleliteralstrong{\sphinxupquote{direction}} (\sphinxstyleliteralemphasis{\sphinxupquote{tuple}}) \textendash{} Or {[}str{]}, the directionality key for which the value on
\sphinxcode{\sphinxupquote{dirs}} has to be set \sphinxcode{\sphinxupquote{True}}.

\item {} 
\sphinxstyleliteralstrong{\sphinxupquote{source}} (\sphinxstyleliteralemphasis{\sphinxupquote{set}}) \textendash{} Contains the name(s) of the source(s) from which such
information was obtained.

\end{itemize}

\end{description}\end{quote}

\end{fulllineitems}

\index{set\_direction() (pypath.legacy.main.Direction method)@\spxentry{set\_direction()}\spxextra{pypath.legacy.main.Direction method}}

\begin{fulllineitems}
\phantomsection\label{\detokenize{reference:pypath.legacy.main.Direction.set_direction}}\pysiglinewithargsret{\sphinxbfcode{\sphinxupquote{set\_direction}}}{\emph{direction}, \emph{source}}{}
Adds directionality information with the corresponding data
source named. Modifies self attributes \sphinxcode{\sphinxupquote{dirs}} and
\sphinxcode{\sphinxupquote{sources}}.
\begin{quote}\begin{description}
\item[{Parameters}] \leavevmode\begin{itemize}
\item {} 
\sphinxstyleliteralstrong{\sphinxupquote{direction}} (\sphinxstyleliteralemphasis{\sphinxupquote{tuple}}) \textendash{} Or {[}str{]}, the directionality key for which the value on
\sphinxcode{\sphinxupquote{dirs}} has to be set \sphinxcode{\sphinxupquote{True}}.

\item {} 
\sphinxstyleliteralstrong{\sphinxupquote{source}} (\sphinxstyleliteralemphasis{\sphinxupquote{set}}) \textendash{} Contains the name(s) of the source(s) from which such
information was obtained.

\end{itemize}

\end{description}\end{quote}

\end{fulllineitems}

\index{set\_sign() (pypath.legacy.main.Direction method)@\spxentry{set\_sign()}\spxextra{pypath.legacy.main.Direction method}}

\begin{fulllineitems}
\phantomsection\label{\detokenize{reference:pypath.legacy.main.Direction.set_sign}}\pysiglinewithargsret{\sphinxbfcode{\sphinxupquote{set\_sign}}}{\emph{direction}, \emph{sign}, \emph{source}}{}
Sets sign and source information on a given direction of the
edge. Modifies the attributes \sphinxcode{\sphinxupquote{positive}} and
\sphinxcode{\sphinxupquote{positive\_sources}} or \sphinxcode{\sphinxupquote{negative}} and
\sphinxcode{\sphinxupquote{negative\_sources}} depending on the sign. Direction is
also updated accordingly, which also modifies the attributes
\sphinxcode{\sphinxupquote{dirs}} and \sphinxcode{\sphinxupquote{sources}}.
\begin{quote}\begin{description}
\item[{Parameters}] \leavevmode\begin{itemize}
\item {} 
\sphinxstyleliteralstrong{\sphinxupquote{direction}} (\sphinxstyleliteralemphasis{\sphinxupquote{tuple}}) \textendash{} Pair of edge nodes specifying the direction from which the
information is to be set/updated.

\item {} 
\sphinxstyleliteralstrong{\sphinxupquote{sign}} (\sphinxstyleliteralemphasis{\sphinxupquote{str}}) \textendash{} Specifies the type of interaction. If \sphinxcode{\sphinxupquote{'positive'}}, is
considered activation, otherwise, is assumed to be negative
(inhibition).

\item {} 
\sphinxstyleliteralstrong{\sphinxupquote{source}} (\sphinxstyleliteralemphasis{\sphinxupquote{set}}) \textendash{} Contains the name(s) of the source(s) from which the
information was obtained.

\end{itemize}

\end{description}\end{quote}

\end{fulllineitems}

\index{source() (pypath.legacy.main.Direction method)@\spxentry{source()}\spxextra{pypath.legacy.main.Direction method}}

\begin{fulllineitems}
\phantomsection\label{\detokenize{reference:pypath.legacy.main.Direction.source}}\pysiglinewithargsret{\sphinxbfcode{\sphinxupquote{source}}}{\emph{undirected=False}, \emph{resources=None}}{}
Returns the name(s) of the source node(s) for each existing
direction on the interaction.
\begin{quote}\begin{description}
\item[{Parameters}] \leavevmode
\sphinxstyleliteralstrong{\sphinxupquote{undirected}} (\sphinxstyleliteralemphasis{\sphinxupquote{bool}}) \textendash{} Optional, \sphinxcode{\sphinxupquote{False}} by default.

\item[{Returns}] \leavevmode
(\sphinxstyleemphasis{list}) \textendash{} Contains the name(s) for the source node(s).
This means if the interaction is bidirectional, the list
will contain both identifiers on the edge. If the
interaction is undirected, an empty list will be returned.

\end{description}\end{quote}

\end{fulllineitems}

\index{sources\_reverse() (pypath.legacy.main.Direction method)@\spxentry{sources\_reverse()}\spxextra{pypath.legacy.main.Direction method}}

\begin{fulllineitems}
\phantomsection\label{\detokenize{reference:pypath.legacy.main.Direction.sources_reverse}}\pysiglinewithargsret{\sphinxbfcode{\sphinxupquote{sources\_reverse}}}{}{}
Retrieves the list of sources for the \sphinxcode{\sphinxupquote{reverse}} direction.
\begin{quote}\begin{description}
\item[{Returns}] \leavevmode
(\sphinxstyleemphasis{set}) \textendash{} Contains the names of the sources supporting the
\sphinxcode{\sphinxupquote{reverse}} directionality of the edge.

\end{description}\end{quote}

\end{fulllineitems}

\index{sources\_straight() (pypath.legacy.main.Direction method)@\spxentry{sources\_straight()}\spxextra{pypath.legacy.main.Direction method}}

\begin{fulllineitems}
\phantomsection\label{\detokenize{reference:pypath.legacy.main.Direction.sources_straight}}\pysiglinewithargsret{\sphinxbfcode{\sphinxupquote{sources\_straight}}}{}{}
Retrieves the list of sources for the \sphinxcode{\sphinxupquote{straight}}
direction.
\begin{quote}\begin{description}
\item[{Returns}] \leavevmode
(\sphinxstyleemphasis{set}) \textendash{} Contains the names of the sources supporting the
\sphinxcode{\sphinxupquote{straight}} directionality of the edge.

\end{description}\end{quote}

\end{fulllineitems}

\index{sources\_undirected() (pypath.legacy.main.Direction method)@\spxentry{sources\_undirected()}\spxextra{pypath.legacy.main.Direction method}}

\begin{fulllineitems}
\phantomsection\label{\detokenize{reference:pypath.legacy.main.Direction.sources_undirected}}\pysiglinewithargsret{\sphinxbfcode{\sphinxupquote{sources\_undirected}}}{}{}
Retrieves the list of sources without directed information.
\begin{quote}\begin{description}
\item[{Returns}] \leavevmode
(\sphinxstyleemphasis{set}) \textendash{} Contains the names of the sources supporting the
edge presence but without specific directionality
information.

\end{description}\end{quote}

\end{fulllineitems}

\index{src() (pypath.legacy.main.Direction method)@\spxentry{src()}\spxextra{pypath.legacy.main.Direction method}}

\begin{fulllineitems}
\phantomsection\label{\detokenize{reference:pypath.legacy.main.Direction.src}}\pysiglinewithargsret{\sphinxbfcode{\sphinxupquote{src}}}{\emph{undirected=False}, \emph{resources=None}}{}
Returns the name(s) of the source node(s) for each existing
direction on the interaction.
\begin{quote}\begin{description}
\item[{Parameters}] \leavevmode
\sphinxstyleliteralstrong{\sphinxupquote{undirected}} (\sphinxstyleliteralemphasis{\sphinxupquote{bool}}) \textendash{} Optional, \sphinxcode{\sphinxupquote{False}} by default.

\item[{Returns}] \leavevmode
(\sphinxstyleemphasis{list}) \textendash{} Contains the name(s) for the source node(s).
This means if the interaction is bidirectional, the list
will contain both identifiers on the edge. If the
interaction is undirected, an empty list will be returned.

\end{description}\end{quote}

\end{fulllineitems}

\index{src\_by\_source() (pypath.legacy.main.Direction method)@\spxentry{src\_by\_source()}\spxextra{pypath.legacy.main.Direction method}}

\begin{fulllineitems}
\phantomsection\label{\detokenize{reference:pypath.legacy.main.Direction.src_by_source}}\pysiglinewithargsret{\sphinxbfcode{\sphinxupquote{src\_by\_source}}}{\emph{source}}{}
Returns the name(s) of the source node(s) for each existing
direction on the interaction for a specific \sphinxstyleemphasis{source}.
\begin{quote}\begin{description}
\item[{Parameters}] \leavevmode
\sphinxstyleliteralstrong{\sphinxupquote{source}} (\sphinxstyleliteralemphasis{\sphinxupquote{str}}) \textendash{} Name of the source according to which the information is to
be retrieved.

\item[{Returns}] \leavevmode
(\sphinxstyleemphasis{list}) \textendash{} Contains the name(s) for the source node(s)
according to the specified \sphinxstyleemphasis{source}. This means if the
interaction is bidirectional, the list will contain both
identifiers on the edge. If the specified \sphinxstyleemphasis{source} is not
found or invalid, an empty list will be returned.

\end{description}\end{quote}

\end{fulllineitems}

\index{target() (pypath.legacy.main.Direction method)@\spxentry{target()}\spxextra{pypath.legacy.main.Direction method}}

\begin{fulllineitems}
\phantomsection\label{\detokenize{reference:pypath.legacy.main.Direction.target}}\pysiglinewithargsret{\sphinxbfcode{\sphinxupquote{target}}}{\emph{undirected=False}, \emph{resources=None}}{}
Returns the name(s) of the target node(s) for each existing
direction on the interaction.
\begin{quote}\begin{description}
\item[{Parameters}] \leavevmode
\sphinxstyleliteralstrong{\sphinxupquote{undirected}} (\sphinxstyleliteralemphasis{\sphinxupquote{bool}}) \textendash{} Optional, \sphinxcode{\sphinxupquote{False}} by default.

\item[{Returns}] \leavevmode
(\sphinxstyleemphasis{list}) \textendash{} Contains the name(s) for the target node(s).
This means if the interaction is bidirectional, the list
will contain both identifiers on the edge. If the
interaction is undirected, an empty list will be returned.

\end{description}\end{quote}

\end{fulllineitems}

\index{tgt() (pypath.legacy.main.Direction method)@\spxentry{tgt()}\spxextra{pypath.legacy.main.Direction method}}

\begin{fulllineitems}
\phantomsection\label{\detokenize{reference:pypath.legacy.main.Direction.tgt}}\pysiglinewithargsret{\sphinxbfcode{\sphinxupquote{tgt}}}{\emph{undirected=False}, \emph{resources=None}}{}
Returns the name(s) of the target node(s) for each existing
direction on the interaction.
\begin{quote}\begin{description}
\item[{Parameters}] \leavevmode
\sphinxstyleliteralstrong{\sphinxupquote{undirected}} (\sphinxstyleliteralemphasis{\sphinxupquote{bool}}) \textendash{} Optional, \sphinxcode{\sphinxupquote{False}} by default.

\item[{Returns}] \leavevmode
(\sphinxstyleemphasis{list}) \textendash{} Contains the name(s) for the target node(s).
This means if the interaction is bidirectional, the list
will contain both identifiers on the edge. If the
interaction is undirected, an empty list will be returned.

\end{description}\end{quote}

\end{fulllineitems}

\index{tgt\_by\_source() (pypath.legacy.main.Direction method)@\spxentry{tgt\_by\_source()}\spxextra{pypath.legacy.main.Direction method}}

\begin{fulllineitems}
\phantomsection\label{\detokenize{reference:pypath.legacy.main.Direction.tgt_by_source}}\pysiglinewithargsret{\sphinxbfcode{\sphinxupquote{tgt\_by\_source}}}{\emph{source}}{}
Returns the name(s) of the target node(s) for each existing
direction on the interaction for a specific \sphinxstyleemphasis{source}.
\begin{quote}\begin{description}
\item[{Parameters}] \leavevmode
\sphinxstyleliteralstrong{\sphinxupquote{source}} (\sphinxstyleliteralemphasis{\sphinxupquote{str}}) \textendash{} Name of the source according to which the information is to
be retrieved.

\item[{Returns}] \leavevmode
(\sphinxstyleemphasis{list}) \textendash{} Contains the name(s) for the target node(s)
according to the specified \sphinxstyleemphasis{source}. This means if the
interaction is bidirectional, the list will contain both
identifiers on the edge. If the specified \sphinxstyleemphasis{source} is not
found or invalid, an empty list will be returned.

\end{description}\end{quote}

\end{fulllineitems}

\index{translate() (pypath.legacy.main.Direction method)@\spxentry{translate()}\spxextra{pypath.legacy.main.Direction method}}

\begin{fulllineitems}
\phantomsection\label{\detokenize{reference:pypath.legacy.main.Direction.translate}}\pysiglinewithargsret{\sphinxbfcode{\sphinxupquote{translate}}}{\emph{ids}}{}
Translates the node names/identifiers according to the
dictionary \sphinxstyleemphasis{ids}.
\begin{quote}\begin{description}
\item[{Parameters}] \leavevmode
\sphinxstyleliteralstrong{\sphinxupquote{ids}} (\sphinxstyleliteralemphasis{\sphinxupquote{dict}}) \textendash{} Dictionary containing (at least) the current names of the
nodes as keys and their translation as values.

\item[{Returns}] \leavevmode
(\sphinxstyleemphasis{pypath.main.Direction}) \textendash{} The copy of current edge object
with translated node names.

\end{description}\end{quote}

\end{fulllineitems}

\index{unset\_dir() (pypath.legacy.main.Direction method)@\spxentry{unset\_dir()}\spxextra{pypath.legacy.main.Direction method}}

\begin{fulllineitems}
\phantomsection\label{\detokenize{reference:pypath.legacy.main.Direction.unset_dir}}\pysiglinewithargsret{\sphinxbfcode{\sphinxupquote{unset\_dir}}}{\emph{direction}, \emph{source=None}}{}
Removes directionality and/or source information of the
specified \sphinxstyleemphasis{direction}. Modifies attribute \sphinxcode{\sphinxupquote{dirs}} and
\sphinxcode{\sphinxupquote{sources}}.
\begin{quote}\begin{description}
\item[{Parameters}] \leavevmode\begin{itemize}
\item {} 
\sphinxstyleliteralstrong{\sphinxupquote{direction}} (\sphinxstyleliteralemphasis{\sphinxupquote{tuple}}) \textendash{} Or {[}str{]} (if \sphinxcode{\sphinxupquote{'undirected'}}) the pair of nodes specifying
the directionality from which the information is to be
removed.

\item {} 
\sphinxstyleliteralstrong{\sphinxupquote{source}} (\sphinxstyleliteralemphasis{\sphinxupquote{set}}) \textendash{} Optional, \sphinxcode{\sphinxupquote{None}} by default. If specified, determines
which specific source(s) is(are) to be removed from
\sphinxcode{\sphinxupquote{sources}} attribute in the specified \sphinxstyleemphasis{direction}.

\end{itemize}

\end{description}\end{quote}

\end{fulllineitems}

\index{unset\_direction() (pypath.legacy.main.Direction method)@\spxentry{unset\_direction()}\spxextra{pypath.legacy.main.Direction method}}

\begin{fulllineitems}
\phantomsection\label{\detokenize{reference:pypath.legacy.main.Direction.unset_direction}}\pysiglinewithargsret{\sphinxbfcode{\sphinxupquote{unset\_direction}}}{\emph{direction}, \emph{source=None}}{}
Removes directionality and/or source information of the
specified \sphinxstyleemphasis{direction}. Modifies attribute \sphinxcode{\sphinxupquote{dirs}} and
\sphinxcode{\sphinxupquote{sources}}.
\begin{quote}\begin{description}
\item[{Parameters}] \leavevmode\begin{itemize}
\item {} 
\sphinxstyleliteralstrong{\sphinxupquote{direction}} (\sphinxstyleliteralemphasis{\sphinxupquote{tuple}}) \textendash{} Or {[}str{]} (if \sphinxcode{\sphinxupquote{'undirected'}}) the pair of nodes specifying
the directionality from which the information is to be
removed.

\item {} 
\sphinxstyleliteralstrong{\sphinxupquote{source}} (\sphinxstyleliteralemphasis{\sphinxupquote{set}}) \textendash{} Optional, \sphinxcode{\sphinxupquote{None}} by default. If specified, determines
which specific source(s) is(are) to be removed from
\sphinxcode{\sphinxupquote{sources}} attribute in the specified \sphinxstyleemphasis{direction}.

\end{itemize}

\end{description}\end{quote}

\end{fulllineitems}

\index{unset\_sign() (pypath.legacy.main.Direction method)@\spxentry{unset\_sign()}\spxextra{pypath.legacy.main.Direction method}}

\begin{fulllineitems}
\phantomsection\label{\detokenize{reference:pypath.legacy.main.Direction.unset_sign}}\pysiglinewithargsret{\sphinxbfcode{\sphinxupquote{unset\_sign}}}{\emph{direction}, \emph{sign}, \emph{source=None}}{}
Removes sign and/or source information of the specified
\sphinxstyleemphasis{direction} and \sphinxstyleemphasis{sign}. Modifies attribute \sphinxcode{\sphinxupquote{positive}}
and \sphinxcode{\sphinxupquote{positive\_sources}} or \sphinxcode{\sphinxupquote{negative}} and
\sphinxcode{\sphinxupquote{negative\_sources}} (or
\sphinxcode{\sphinxupquote{positive\_attributes}}/\sphinxcode{\sphinxupquote{negative\_sources}}
only if \sphinxcode{\sphinxupquote{source=True}}).
\begin{quote}\begin{description}
\item[{Parameters}] \leavevmode\begin{itemize}
\item {} 
\sphinxstyleliteralstrong{\sphinxupquote{direction}} (\sphinxstyleliteralemphasis{\sphinxupquote{tuple}}) \textendash{} The pair of nodes specifying the directionality from which
the information is to be removed.

\item {} 
\sphinxstyleliteralstrong{\sphinxupquote{sign}} (\sphinxstyleliteralemphasis{\sphinxupquote{str}}) \textendash{} Sign from which the information is to be removed. Must be
either \sphinxcode{\sphinxupquote{'positive'}} or \sphinxcode{\sphinxupquote{'negative'}}.

\item {} 
\sphinxstyleliteralstrong{\sphinxupquote{source}} (\sphinxstyleliteralemphasis{\sphinxupquote{set}}) \textendash{} Optional, \sphinxcode{\sphinxupquote{None}} by default. If specified, determines
which source(s) is(are) to be removed from the sources in
the specified \sphinxstyleemphasis{direction} and \sphinxstyleemphasis{sign}.

\end{itemize}

\end{description}\end{quote}

\end{fulllineitems}

\index{which\_directions() (pypath.legacy.main.Direction method)@\spxentry{which\_directions()}\spxextra{pypath.legacy.main.Direction method}}

\begin{fulllineitems}
\phantomsection\label{\detokenize{reference:pypath.legacy.main.Direction.which_directions}}\pysiglinewithargsret{\sphinxbfcode{\sphinxupquote{which\_directions}}}{\emph{resources=None}, \emph{effect=None}}{}
Returns the pair(s) of nodes for which there is information
about their directionality.
\begin{quote}\begin{description}
\item[{Parameters}] \leavevmode\begin{itemize}
\item {} 
\sphinxstyleliteralstrong{\sphinxupquote{effect}} (\sphinxstyleliteralemphasis{\sphinxupquote{str}}) \textendash{} Either \sphinxstyleemphasis{positive} or \sphinxstyleemphasis{negative}.

\item {} 
\sphinxstyleliteralstrong{\sphinxupquote{resources}} (\sphinxstyleliteralemphasis{\sphinxupquote{str}}\sphinxstyleliteralemphasis{\sphinxupquote{,}}\sphinxstyleliteralemphasis{\sphinxupquote{set}}) \textendash{} Limits the query to one or more resources. Optional.

\end{itemize}

\item[{Returns}] \leavevmode
(\sphinxstyleemphasis{tuple}) \textendash{} Tuple of tuples with pairs of nodes where the
first element is the source and the second is the target
entity, according to the given resources and limited to the
effect.

\end{description}\end{quote}

\end{fulllineitems}

\index{which\_dirs() (pypath.legacy.main.Direction method)@\spxentry{which\_dirs()}\spxextra{pypath.legacy.main.Direction method}}

\begin{fulllineitems}
\phantomsection\label{\detokenize{reference:pypath.legacy.main.Direction.which_dirs}}\pysiglinewithargsret{\sphinxbfcode{\sphinxupquote{which\_dirs}}}{\emph{resources=None}, \emph{effect=None}}{}
Returns the pair(s) of nodes for which there is information
about their directionality.
\begin{quote}\begin{description}
\item[{Parameters}] \leavevmode\begin{itemize}
\item {} 
\sphinxstyleliteralstrong{\sphinxupquote{effect}} (\sphinxstyleliteralemphasis{\sphinxupquote{str}}) \textendash{} Either \sphinxstyleemphasis{positive} or \sphinxstyleemphasis{negative}.

\item {} 
\sphinxstyleliteralstrong{\sphinxupquote{resources}} (\sphinxstyleliteralemphasis{\sphinxupquote{str}}\sphinxstyleliteralemphasis{\sphinxupquote{,}}\sphinxstyleliteralemphasis{\sphinxupquote{set}}) \textendash{} Limits the query to one or more resources. Optional.

\end{itemize}

\item[{Returns}] \leavevmode
(\sphinxstyleemphasis{tuple}) \textendash{} Tuple of tuples with pairs of nodes where the
first element is the source and the second is the target
entity, according to the given resources and limited to the
effect.

\end{description}\end{quote}

\end{fulllineitems}

\index{which\_signs() (pypath.legacy.main.Direction method)@\spxentry{which\_signs()}\spxextra{pypath.legacy.main.Direction method}}

\begin{fulllineitems}
\phantomsection\label{\detokenize{reference:pypath.legacy.main.Direction.which_signs}}\pysiglinewithargsret{\sphinxbfcode{\sphinxupquote{which\_signs}}}{\emph{resources=None}, \emph{effect=None}}{}
Returns the pair(s) of nodes for which there is information
about their effect signs.
\begin{quote}\begin{description}
\item[{Parameters}] \leavevmode\begin{itemize}
\item {} 
\sphinxstyleliteralstrong{\sphinxupquote{resources}} (\sphinxstyleliteralemphasis{\sphinxupquote{str}}\sphinxstyleliteralemphasis{\sphinxupquote{,}}\sphinxstyleliteralemphasis{\sphinxupquote{set}}) \textendash{} Limits the query to one or more resources. Optional.

\item {} 
\sphinxstyleliteralstrong{\sphinxupquote{effect}} (\sphinxstyleliteralemphasis{\sphinxupquote{str}}) \textendash{} Either \sphinxstyleemphasis{positive} or \sphinxstyleemphasis{negative}, limiting the query to positive
or negative effects; for any other values effects of both
signs will be returned.

\end{itemize}

\item[{Returns}] \leavevmode
(\sphinxstyleemphasis{tuple}) \textendash{} Tuple of tuples with pairs of nodes where the
first element is a tuple of the source and the target entity,
while the second element is the effect sign, according to
the given resources. E.g. (((‘A’, ‘B’), ‘positive’),)

\end{description}\end{quote}

\end{fulllineitems}


\end{fulllineitems}



\section{mirbase}
\label{\detokenize{reference:module-pypath.inputs.mirbase}}\label{\detokenize{reference:mirbase}}\index{pypath.inputs.mirbase (module)@\spxentry{pypath.inputs.mirbase}\spxextra{module}}\index{get\_mirbase\_aliases() (in module pypath.inputs.mirbase)@\spxentry{get\_mirbase\_aliases()}\spxextra{in module pypath.inputs.mirbase}}

\begin{fulllineitems}
\phantomsection\label{\detokenize{reference:pypath.inputs.mirbase.get_mirbase_aliases}}\pysiglinewithargsret{\sphinxcode{\sphinxupquote{pypath.inputs.mirbase.}}\sphinxbfcode{\sphinxupquote{get\_mirbase\_aliases}}}{\emph{organism=9606}}{}
Downloads and processes mapping tables from miRBase.

\end{fulllineitems}

\index{get\_uniprot\_sec() (in module pypath.inputs.mirbase)@\spxentry{get\_uniprot\_sec()}\spxextra{in module pypath.inputs.mirbase}}

\begin{fulllineitems}
\phantomsection\label{\detokenize{reference:pypath.inputs.mirbase.get_uniprot_sec}}\pysiglinewithargsret{\sphinxcode{\sphinxupquote{pypath.inputs.mirbase.}}\sphinxbfcode{\sphinxupquote{get\_uniprot\_sec}}}{\emph{organism=9606}}{}
Downloads and processes the mapping between secondary and
primary UniProt IDs.

Yields pairs of secondary and primary UniProt IDs.
\begin{quote}\begin{description}
\item[{Parameters}] \leavevmode
\sphinxstyleliteralstrong{\sphinxupquote{organism}} (\sphinxstyleliteralemphasis{\sphinxupquote{int}}) \textendash{} NCBI Taxonomy ID of the organism.

\end{description}\end{quote}

\end{fulllineitems}



\section{mapping}
\label{\detokenize{reference:module-pypath.utils.mapping}}\label{\detokenize{reference:mapping}}\index{pypath.utils.mapping (module)@\spxentry{pypath.utils.mapping}\spxextra{module}}\index{MapReader (class in pypath.utils.mapping)@\spxentry{MapReader}\spxextra{class in pypath.utils.mapping}}

\begin{fulllineitems}
\phantomsection\label{\detokenize{reference:pypath.utils.mapping.MapReader}}\pysiglinewithargsret{\sphinxbfcode{\sphinxupquote{class }}\sphinxcode{\sphinxupquote{pypath.utils.mapping.}}\sphinxbfcode{\sphinxupquote{MapReader}}}{\emph{param}, \emph{ncbi\_tax\_id=None}, \emph{entity\_type=None}, \emph{load\_a\_to\_b=True}, \emph{load\_b\_to\_a=False}, \emph{uniprots=None}, \emph{lifetime=300}}{}
Reads ID translation data and creates \sphinxcode{\sphinxupquote{MappingTable}} instances.
When initializing ID conversion tables for the first time
data is downloaded from UniProt and read into dictionaries.
It takes a couple of seconds. Data is saved to pickle
dumps, this way later the tables load much faster.
\begin{quote}\begin{description}
\item[{Parameters}] \leavevmode
\sphinxstyleliteralstrong{\sphinxupquote{str}} (\sphinxstyleliteralemphasis{\sphinxupquote{source\_type}}) \textendash{} Type of the resource, either \sphinxtitleref{file}, \sphinxtitleref{uniprot} or \sphinxtitleref{unprotlist}.

\end{description}\end{quote}
\index{load() (pypath.utils.mapping.MapReader method)@\spxentry{load()}\spxextra{pypath.utils.mapping.MapReader method}}

\begin{fulllineitems}
\phantomsection\label{\detokenize{reference:pypath.utils.mapping.MapReader.load}}\pysiglinewithargsret{\sphinxbfcode{\sphinxupquote{load}}}{}{}
The complete process of loading mapping tables. First sets up the
paths of the cache files, then loads the tables from the cache files
or the original sources if necessary. Upon successful loading from an
original source writes the results to cache files.

\end{fulllineitems}

\index{mapping\_table\_a\_to\_b (pypath.utils.mapping.MapReader attribute)@\spxentry{mapping\_table\_a\_to\_b}\spxextra{pypath.utils.mapping.MapReader attribute}}

\begin{fulllineitems}
\phantomsection\label{\detokenize{reference:pypath.utils.mapping.MapReader.mapping_table_a_to_b}}\pysigline{\sphinxbfcode{\sphinxupquote{mapping\_table\_a\_to\_b}}}
Returns a \sphinxcode{\sphinxupquote{MappingTable}} instance created from the already
loaded data.

\end{fulllineitems}

\index{mapping\_table\_b\_to\_a (pypath.utils.mapping.MapReader attribute)@\spxentry{mapping\_table\_b\_to\_a}\spxextra{pypath.utils.mapping.MapReader attribute}}

\begin{fulllineitems}
\phantomsection\label{\detokenize{reference:pypath.utils.mapping.MapReader.mapping_table_b_to_a}}\pysigline{\sphinxbfcode{\sphinxupquote{mapping\_table\_b\_to\_a}}}
Returns a \sphinxcode{\sphinxupquote{MappingTable}} instance created from the already
loaded data.

\end{fulllineitems}

\index{read() (pypath.utils.mapping.MapReader method)@\spxentry{read()}\spxextra{pypath.utils.mapping.MapReader method}}

\begin{fulllineitems}
\phantomsection\label{\detokenize{reference:pypath.utils.mapping.MapReader.read}}\pysiglinewithargsret{\sphinxbfcode{\sphinxupquote{read}}}{}{}
Reads the ID translation data from the original source.

\end{fulllineitems}

\index{read\_cache() (pypath.utils.mapping.MapReader method)@\spxentry{read\_cache()}\spxextra{pypath.utils.mapping.MapReader method}}

\begin{fulllineitems}
\phantomsection\label{\detokenize{reference:pypath.utils.mapping.MapReader.read_cache}}\pysiglinewithargsret{\sphinxbfcode{\sphinxupquote{read\_cache}}}{}{}
Reads the ID translation data from a previously saved pickle file.

\end{fulllineitems}

\index{read\_mapping\_uniprot() (pypath.utils.mapping.MapReader method)@\spxentry{read\_mapping\_uniprot()}\spxextra{pypath.utils.mapping.MapReader method}}

\begin{fulllineitems}
\phantomsection\label{\detokenize{reference:pypath.utils.mapping.MapReader.read_mapping_uniprot}}\pysiglinewithargsret{\sphinxbfcode{\sphinxupquote{read\_mapping\_uniprot}}}{}{}
Downloads ID mappings directly from UniProt.
See the names of possible identifiers here:
\sphinxurl{http://www.uniprot.org/help/programmatic\_access}

\end{fulllineitems}

\index{read\_mapping\_uniprot\_list() (pypath.utils.mapping.MapReader method)@\spxentry{read\_mapping\_uniprot\_list()}\spxextra{pypath.utils.mapping.MapReader method}}

\begin{fulllineitems}
\phantomsection\label{\detokenize{reference:pypath.utils.mapping.MapReader.read_mapping_uniprot_list}}\pysiglinewithargsret{\sphinxbfcode{\sphinxupquote{read\_mapping\_uniprot\_list}}}{}{}
Builds a mapping table by downloading data from UniProt’s
upload lists service.

\end{fulllineitems}

\index{set\_uniprot\_space() (pypath.utils.mapping.MapReader method)@\spxentry{set\_uniprot\_space()}\spxextra{pypath.utils.mapping.MapReader method}}

\begin{fulllineitems}
\phantomsection\label{\detokenize{reference:pypath.utils.mapping.MapReader.set_uniprot_space}}\pysiglinewithargsret{\sphinxbfcode{\sphinxupquote{set\_uniprot\_space}}}{\emph{swissprot=None}}{}
Sets up a search space of UniProt IDs.

\end{fulllineitems}

\index{setup\_cache() (pypath.utils.mapping.MapReader method)@\spxentry{setup\_cache()}\spxextra{pypath.utils.mapping.MapReader method}}

\begin{fulllineitems}
\phantomsection\label{\detokenize{reference:pypath.utils.mapping.MapReader.setup_cache}}\pysiglinewithargsret{\sphinxbfcode{\sphinxupquote{setup\_cache}}}{}{}
Constructs the cache file path as md5 hash of the parameters.

\end{fulllineitems}

\index{tables\_loaded() (pypath.utils.mapping.MapReader method)@\spxentry{tables\_loaded()}\spxextra{pypath.utils.mapping.MapReader method}}

\begin{fulllineitems}
\phantomsection\label{\detokenize{reference:pypath.utils.mapping.MapReader.tables_loaded}}\pysiglinewithargsret{\sphinxbfcode{\sphinxupquote{tables\_loaded}}}{}{}
Tells if the requested tables have been created.

\end{fulllineitems}

\index{write\_cache() (pypath.utils.mapping.MapReader method)@\spxentry{write\_cache()}\spxextra{pypath.utils.mapping.MapReader method}}

\begin{fulllineitems}
\phantomsection\label{\detokenize{reference:pypath.utils.mapping.MapReader.write_cache}}\pysiglinewithargsret{\sphinxbfcode{\sphinxupquote{write\_cache}}}{}{}
Exports the ID translation data into pickle files.

\end{fulllineitems}


\end{fulllineitems}

\index{MappingTable (class in pypath.utils.mapping)@\spxentry{MappingTable}\spxextra{class in pypath.utils.mapping}}

\begin{fulllineitems}
\phantomsection\label{\detokenize{reference:pypath.utils.mapping.MappingTable}}\pysiglinewithargsret{\sphinxbfcode{\sphinxupquote{class }}\sphinxcode{\sphinxupquote{pypath.utils.mapping.}}\sphinxbfcode{\sphinxupquote{MappingTable}}}{\emph{data}, \emph{id\_type}, \emph{target\_id\_type}, \emph{ncbi\_tax\_id}, \emph{lifetime=300}}{}
This is the class directly handling ID translation data.
It does not care about loading it or what kind of IDs these
only accepts the translation dictionary.
\begin{description}
\item[{lifetime}] \leavevmode{[}int{]}
If this table has not been used for longer than this preiod it is
to be removed at next cleanup. Time in seconds.

\end{description}

\end{fulllineitems}



\section{maps}
\label{\detokenize{reference:module-pypath.internals.maps}}\label{\detokenize{reference:maps}}\index{pypath.internals.maps (module)@\spxentry{pypath.internals.maps}\spxextra{module}}

\section{network}
\label{\detokenize{reference:module-pypath.core.network}}\label{\detokenize{reference:network}}\index{pypath.core.network (module)@\spxentry{pypath.core.network}\spxextra{module}}\index{Network (class in pypath.core.network)@\spxentry{Network}\spxextra{class in pypath.core.network}}

\begin{fulllineitems}
\phantomsection\label{\detokenize{reference:pypath.core.network.Network}}\pysiglinewithargsret{\sphinxbfcode{\sphinxupquote{class }}\sphinxcode{\sphinxupquote{pypath.core.network.}}\sphinxbfcode{\sphinxupquote{Network}}}{\emph{resources=None}, \emph{make\_df=False}, \emph{df\_by\_source=False}, \emph{df\_with\_references=False}, \emph{df\_columns=None}, \emph{df\_dtype=None}, \emph{pickle\_file=None}, \emph{ncbi\_tax\_id=9606}, \emph{allow\_loops=True}, \emph{**kwargs}}{}
Represents a molecular interaction network. Provides various methods to
query the network and its components. Optionally converts the network
to a \sphinxcode{\sphinxupquote{pandas.DataFrame}} of interactions.
\begin{quote}\begin{description}
\item[{Parameters}] \leavevmode\begin{itemize}
\item {} 
\sphinxstyleliteralstrong{\sphinxupquote{resources}} (\sphinxstyleliteralemphasis{\sphinxupquote{list}}\sphinxstyleliteralemphasis{\sphinxupquote{,}}\sphinxstyleliteralemphasis{\sphinxupquote{dict}}) \textendash{} One or more lists or dictionaries containing
\sphinxcode{\sphinxupquote{pypath.resource.NetworkResource}} objects.

\item {} 
\sphinxstyleliteralstrong{\sphinxupquote{make\_df}} (\sphinxstyleliteralemphasis{\sphinxupquote{bool}}) \textendash{} Create a \sphinxcode{\sphinxupquote{pandas.DataFrame}} already when creating the instance.
If no network data loaded no data frame will be created.

\item {} 
\sphinxstyleliteralstrong{\sphinxupquote{ncbi\_tax\_id}} (\sphinxstyleliteralemphasis{\sphinxupquote{int}}) \textendash{} Restrict the network only to this organism. If \sphinxcode{\sphinxupquote{None}} identifiers
from any organism will be allowed.

\item {} 
\sphinxstyleliteralstrong{\sphinxupquote{allow\_loops}} (\sphinxstyleliteralemphasis{\sphinxupquote{bool}}) \textendash{} Allow interactions with the their two endpoints being the same entity.

\end{itemize}

\end{description}\end{quote}
\index{activated\_by() (pypath.core.network.Network method)@\spxentry{activated\_by()}\spxextra{pypath.core.network.Network method}}

\begin{fulllineitems}
\phantomsection\label{\detokenize{reference:pypath.core.network.Network.activated_by}}\pysiglinewithargsret{\sphinxbfcode{\sphinxupquote{activated\_by}}}{}{}~\begin{quote}\begin{description}
\item[{Parameters}] \leavevmode\begin{itemize}
\item {} 
\sphinxstyleliteralstrong{\sphinxupquote{entity}} (\sphinxstyleliteralemphasis{\sphinxupquote{str}}\sphinxstyleliteralemphasis{\sphinxupquote{,}}{\hyperref[\detokenize{reference:pypath.core.entity.Entity}]{\sphinxcrossref{\sphinxstyleliteralemphasis{\sphinxupquote{Entity}}}}}\sphinxstyleliteralemphasis{\sphinxupquote{,}}\sphinxstyleliteralemphasis{\sphinxupquote{list}}\sphinxstyleliteralemphasis{\sphinxupquote{,}}\sphinxstyleliteralemphasis{\sphinxupquote{set}}\sphinxstyleliteralemphasis{\sphinxupquote{,}}\sphinxstyleliteralemphasis{\sphinxupquote{tuple}}\sphinxstyleliteralemphasis{\sphinxupquote{,}}\sphinxstyleliteralemphasis{\sphinxupquote{EntityList}}) \textendash{} An identifier or label of a molecular entity or an
\sphinxcode{\sphinxupquote{Entity}} object. Alternatively an iterator with the
elements of any of the types valid for a single entity argument,
e.g. a list of gene symbols.

\item {} 
\sphinxstyleliteralstrong{\sphinxupquote{mode}} (\sphinxstyleliteralemphasis{\sphinxupquote{str}}) \textendash{} Mode of counting the interactions: \sphinxtitleref{IN}, \sphinxtitleref{OUT} or \sphinxtitleref{ALL} , whether
to consider incoming, outgoing or all edges, respectively,
respective to the \sphinxtitleref{node defined in {}`entity{}`}.

\end{itemize}

\item[{Returns}] \leavevmode
\sphinxcode{\sphinxupquote{EntityList}} object containing the partners having
interactions to the queried node(s) matching all the criteria.
If \sphinxcode{\sphinxupquote{entity}} doesn’t present in the network the returned
\sphinxcode{\sphinxupquote{EntityList}} will be empty just like if no interaction matches
the criteria.

\end{description}\end{quote}

\end{fulllineitems}

\index{activates() (pypath.core.network.Network method)@\spxentry{activates()}\spxextra{pypath.core.network.Network method}}

\begin{fulllineitems}
\phantomsection\label{\detokenize{reference:pypath.core.network.Network.activates}}\pysiglinewithargsret{\sphinxbfcode{\sphinxupquote{activates}}}{}{}~\begin{quote}\begin{description}
\item[{Parameters}] \leavevmode\begin{itemize}
\item {} 
\sphinxstyleliteralstrong{\sphinxupquote{entity}} (\sphinxstyleliteralemphasis{\sphinxupquote{str}}\sphinxstyleliteralemphasis{\sphinxupquote{,}}{\hyperref[\detokenize{reference:pypath.core.entity.Entity}]{\sphinxcrossref{\sphinxstyleliteralemphasis{\sphinxupquote{Entity}}}}}\sphinxstyleliteralemphasis{\sphinxupquote{,}}\sphinxstyleliteralemphasis{\sphinxupquote{list}}\sphinxstyleliteralemphasis{\sphinxupquote{,}}\sphinxstyleliteralemphasis{\sphinxupquote{set}}\sphinxstyleliteralemphasis{\sphinxupquote{,}}\sphinxstyleliteralemphasis{\sphinxupquote{tuple}}\sphinxstyleliteralemphasis{\sphinxupquote{,}}\sphinxstyleliteralemphasis{\sphinxupquote{EntityList}}) \textendash{} An identifier or label of a molecular entity or an
\sphinxcode{\sphinxupquote{Entity}} object. Alternatively an iterator with the
elements of any of the types valid for a single entity argument,
e.g. a list of gene symbols.

\item {} 
\sphinxstyleliteralstrong{\sphinxupquote{mode}} (\sphinxstyleliteralemphasis{\sphinxupquote{str}}) \textendash{} Mode of counting the interactions: \sphinxtitleref{IN}, \sphinxtitleref{OUT} or \sphinxtitleref{ALL} , whether
to consider incoming, outgoing or all edges, respectively,
respective to the \sphinxtitleref{node defined in {}`entity{}`}.

\end{itemize}

\item[{Returns}] \leavevmode
\sphinxcode{\sphinxupquote{EntityList}} object containing the partners having
interactions to the queried node(s) matching all the criteria.
If \sphinxcode{\sphinxupquote{entity}} doesn’t present in the network the returned
\sphinxcode{\sphinxupquote{EntityList}} will be empty just like if no interaction matches
the criteria.

\end{description}\end{quote}

\end{fulllineitems}

\index{add\_interaction() (pypath.core.network.Network method)@\spxentry{add\_interaction()}\spxextra{pypath.core.network.Network method}}

\begin{fulllineitems}
\phantomsection\label{\detokenize{reference:pypath.core.network.Network.add_interaction}}\pysiglinewithargsret{\sphinxbfcode{\sphinxupquote{add\_interaction}}}{\emph{interaction}, \emph{attrs=None}, \emph{only\_directions=False}}{}
Adds a ready \sphinxcode{\sphinxupquote{pypath.interaction.Interaction}} object to the network.
If an interaction between the two endpoints already exists, the
interactions will be merged: this stands for the directions, signs,
evidences and other attributes.
\begin{quote}\begin{description}
\item[{Parameters}] \leavevmode\begin{itemize}
\item {} 
\sphinxstyleliteralstrong{\sphinxupquote{interaction}} ({\hyperref[\detokenize{reference:pypath.core.interaction.Interaction}]{\sphinxcrossref{\sphinxstyleliteralemphasis{\sphinxupquote{interaction.Interaction}}}}}) \textendash{} A \sphinxcode{\sphinxupquote{pypath.interaction.Interaction}} object.

\item {} 
\sphinxstyleliteralstrong{\sphinxupquote{attrs}} (\sphinxstyleliteralemphasis{\sphinxupquote{NoneType}}\sphinxstyleliteralemphasis{\sphinxupquote{,}}\sphinxstyleliteralemphasis{\sphinxupquote{dict}}) \textendash{} Optional, a dictionary of extra (usually resource specific)
attributes.

\item {} 
\sphinxstyleliteralstrong{\sphinxupquote{only\_directions}} (\sphinxstyleliteralemphasis{\sphinxupquote{bool}}) \textendash{} If the interaction between the two endpoints does not exist it
won’t be added to the network. Otherwise all attributes
(direction, effect sign, evidences, etc) will be merged to the
existing interaction. Apart from the endpoints also the
\sphinxcode{\sphinxupquote{interaction\_type}} of the existing interaction has to match the
interaction added here.

\end{itemize}

\end{description}\end{quote}

\end{fulllineitems}

\index{add\_node() (pypath.core.network.Network method)@\spxentry{add\_node()}\spxextra{pypath.core.network.Network method}}

\begin{fulllineitems}
\phantomsection\label{\detokenize{reference:pypath.core.network.Network.add_node}}\pysiglinewithargsret{\sphinxbfcode{\sphinxupquote{add\_node}}}{\emph{entity}, \emph{attrs=None}, \emph{add=True}}{}
Adds a molecular entity to the py:attr:\sphinxcode{\sphinxupquote{nodes}} and
py:attr:\sphinxcode{\sphinxupquote{nodes\_by\_label}} dictionaries.
\begin{quote}\begin{description}
\item[{Parameters}] \leavevmode\begin{itemize}
\item {} 
\sphinxstyleliteralstrong{\sphinxupquote{entity}} ({\hyperref[\detokenize{reference:pypath.core.entity.Entity}]{\sphinxcrossref{\sphinxstyleliteralemphasis{\sphinxupquote{entity.Entity}}}}}) \textendash{} An object representing a molecular entity.

\item {} 
\sphinxstyleliteralstrong{\sphinxupquote{attrs}} (\sphinxstyleliteralemphasis{\sphinxupquote{NoneType}}\sphinxstyleliteralemphasis{\sphinxupquote{,}}\sphinxstyleliteralemphasis{\sphinxupquote{dict}}) \textendash{} Optional extra attributes to be assigned to the entity.

\item {} 
\sphinxstyleliteralstrong{\sphinxupquote{add}} (\sphinxstyleliteralemphasis{\sphinxupquote{bool}}) \textendash{} Whether to add a new molecular entity to the network if it does
not exist yet. If \sphinxcode{\sphinxupquote{False}} will only update attributes for
existing entities otherwise will do nothing.

\end{itemize}

\end{description}\end{quote}

\end{fulllineitems}

\index{collect\_complex\_identifiers() (pypath.core.network.Network method)@\spxentry{collect\_complex\_identifiers()}\spxextra{pypath.core.network.Network method}}

\begin{fulllineitems}
\phantomsection\label{\detokenize{reference:pypath.core.network.Network.collect_complex_identifiers}}\pysiglinewithargsret{\sphinxbfcode{\sphinxupquote{collect\_complex\_identifiers}}}{\emph{effect=None}, \emph{resources=None}, \emph{data\_model=None}, \emph{interaction\_type=None}, \emph{via=None}, \emph{references=None}}{}
Builds a comprehensive collection of \sphinxtitleref{complex\_identifiers} entities across the network, counts unique and shared objects by resource, data model and interaction types.

\end{fulllineitems}

\index{collect\_complex\_labels() (pypath.core.network.Network method)@\spxentry{collect\_complex\_labels()}\spxextra{pypath.core.network.Network method}}

\begin{fulllineitems}
\phantomsection\label{\detokenize{reference:pypath.core.network.Network.collect_complex_labels}}\pysiglinewithargsret{\sphinxbfcode{\sphinxupquote{collect\_complex\_labels}}}{\emph{effect=None}, \emph{resources=None}, \emph{data\_model=None}, \emph{interaction\_type=None}, \emph{via=None}, \emph{references=None}}{}
Builds a comprehensive collection of \sphinxtitleref{complex\_labels} entities across the network, counts unique and shared objects by resource, data model and interaction types.

\end{fulllineitems}

\index{collect\_complexes() (pypath.core.network.Network method)@\spxentry{collect\_complexes()}\spxextra{pypath.core.network.Network method}}

\begin{fulllineitems}
\phantomsection\label{\detokenize{reference:pypath.core.network.Network.collect_complexes}}\pysiglinewithargsret{\sphinxbfcode{\sphinxupquote{collect\_complexes}}}{\emph{effect=None}, \emph{resources=None}, \emph{data\_model=None}, \emph{interaction\_type=None}, \emph{via=None}, \emph{references=None}}{}
Builds a comprehensive collection of \sphinxtitleref{complexes} entities across the network, counts unique and shared objects by resource, data model and interaction types.

\end{fulllineitems}

\index{collect\_curation\_effort() (pypath.core.network.Network method)@\spxentry{collect\_curation\_effort()}\spxextra{pypath.core.network.Network method}}

\begin{fulllineitems}
\phantomsection\label{\detokenize{reference:pypath.core.network.Network.collect_curation_effort}}\pysiglinewithargsret{\sphinxbfcode{\sphinxupquote{collect\_curation\_effort}}}{\emph{effect=None}, \emph{resources=None}, \emph{data\_model=None}, \emph{interaction\_type=None}, \emph{via=None}, \emph{references=None}}{}
Builds a comprehensive collection of \sphinxtitleref{curation\_effort} entities across the network, counts unique and shared objects by resource, data model and interaction types.

\end{fulllineitems}

\index{collect\_data\_models() (pypath.core.network.Network method)@\spxentry{collect\_data\_models()}\spxextra{pypath.core.network.Network method}}

\begin{fulllineitems}
\phantomsection\label{\detokenize{reference:pypath.core.network.Network.collect_data_models}}\pysiglinewithargsret{\sphinxbfcode{\sphinxupquote{collect\_data\_models}}}{\emph{effect=None}, \emph{resources=None}, \emph{data\_model=None}, \emph{interaction\_type=None}, \emph{via=None}, \emph{references=None}}{}
Builds a comprehensive collection of \sphinxtitleref{data\_models} entities across the network, counts unique and shared objects by resource, data model and interaction types.

\end{fulllineitems}

\index{collect\_degrees\_directed() (pypath.core.network.Network method)@\spxentry{collect\_degrees\_directed()}\spxextra{pypath.core.network.Network method}}

\begin{fulllineitems}
\phantomsection\label{\detokenize{reference:pypath.core.network.Network.collect_degrees_directed}}\pysiglinewithargsret{\sphinxbfcode{\sphinxupquote{collect\_degrees\_directed}}}{\emph{direction=None}, \emph{effect=None}, \emph{resources=None}, \emph{data\_model=None}, \emph{interaction\_type=None}, \emph{via=None}, \emph{references=None}}{}
Builds a comprehensive collection of \sphinxtitleref{degrees\_directed} entities across the network, counts unique and shared objects by resource, data model and interaction types.

\end{fulllineitems}

\index{collect\_degrees\_directed\_in() (pypath.core.network.Network method)@\spxentry{collect\_degrees\_directed\_in()}\spxextra{pypath.core.network.Network method}}

\begin{fulllineitems}
\phantomsection\label{\detokenize{reference:pypath.core.network.Network.collect_degrees_directed_in}}\pysiglinewithargsret{\sphinxbfcode{\sphinxupquote{collect\_degrees\_directed\_in}}}{\emph{direction=None}, \emph{effect=None}, \emph{resources=None}, \emph{data\_model=None}, \emph{interaction\_type=None}, \emph{via=None}, \emph{references=None}}{}
Builds a comprehensive collection of \sphinxtitleref{degrees\_directed\_in} entities across the network, counts unique and shared objects by resource, data model and interaction types.

\end{fulllineitems}

\index{collect\_degrees\_directed\_out() (pypath.core.network.Network method)@\spxentry{collect\_degrees\_directed\_out()}\spxextra{pypath.core.network.Network method}}

\begin{fulllineitems}
\phantomsection\label{\detokenize{reference:pypath.core.network.Network.collect_degrees_directed_out}}\pysiglinewithargsret{\sphinxbfcode{\sphinxupquote{collect\_degrees\_directed\_out}}}{\emph{direction=None}, \emph{effect=None}, \emph{resources=None}, \emph{data\_model=None}, \emph{interaction\_type=None}, \emph{via=None}, \emph{references=None}}{}
Builds a comprehensive collection of \sphinxtitleref{degrees\_directed\_out} entities across the network, counts unique and shared objects by resource, data model and interaction types.

\end{fulllineitems}

\index{collect\_degrees\_negative() (pypath.core.network.Network method)@\spxentry{collect\_degrees\_negative()}\spxextra{pypath.core.network.Network method}}

\begin{fulllineitems}
\phantomsection\label{\detokenize{reference:pypath.core.network.Network.collect_degrees_negative}}\pysiglinewithargsret{\sphinxbfcode{\sphinxupquote{collect\_degrees\_negative}}}{\emph{direction=None}, \emph{effect=None}, \emph{resources=None}, \emph{data\_model=None}, \emph{interaction\_type=None}, \emph{via=None}, \emph{references=None}}{}
Builds a comprehensive collection of \sphinxtitleref{degrees\_negative} entities across the network, counts unique and shared objects by resource, data model and interaction types.

\end{fulllineitems}

\index{collect\_degrees\_negative\_in() (pypath.core.network.Network method)@\spxentry{collect\_degrees\_negative\_in()}\spxextra{pypath.core.network.Network method}}

\begin{fulllineitems}
\phantomsection\label{\detokenize{reference:pypath.core.network.Network.collect_degrees_negative_in}}\pysiglinewithargsret{\sphinxbfcode{\sphinxupquote{collect\_degrees\_negative\_in}}}{\emph{direction=None}, \emph{effect=None}, \emph{resources=None}, \emph{data\_model=None}, \emph{interaction\_type=None}, \emph{via=None}, \emph{references=None}}{}
Builds a comprehensive collection of \sphinxtitleref{degrees\_negative\_in} entities across the network, counts unique and shared objects by resource, data model and interaction types.

\end{fulllineitems}

\index{collect\_degrees\_negative\_out() (pypath.core.network.Network method)@\spxentry{collect\_degrees\_negative\_out()}\spxextra{pypath.core.network.Network method}}

\begin{fulllineitems}
\phantomsection\label{\detokenize{reference:pypath.core.network.Network.collect_degrees_negative_out}}\pysiglinewithargsret{\sphinxbfcode{\sphinxupquote{collect\_degrees\_negative\_out}}}{\emph{direction=None}, \emph{effect=None}, \emph{resources=None}, \emph{data\_model=None}, \emph{interaction\_type=None}, \emph{via=None}, \emph{references=None}}{}
Builds a comprehensive collection of \sphinxtitleref{degrees\_negative\_out} entities across the network, counts unique and shared objects by resource, data model and interaction types.

\end{fulllineitems}

\index{collect\_degrees\_non\_directed() (pypath.core.network.Network method)@\spxentry{collect\_degrees\_non\_directed()}\spxextra{pypath.core.network.Network method}}

\begin{fulllineitems}
\phantomsection\label{\detokenize{reference:pypath.core.network.Network.collect_degrees_non_directed}}\pysiglinewithargsret{\sphinxbfcode{\sphinxupquote{collect\_degrees\_non\_directed}}}{\emph{direction=None}, \emph{effect=None}, \emph{resources=None}, \emph{data\_model=None}, \emph{interaction\_type=None}, \emph{via=None}, \emph{references=None}}{}
Builds a comprehensive collection of \sphinxtitleref{degrees\_non\_directed} entities across the network, counts unique and shared objects by resource, data model and interaction types.

\end{fulllineitems}

\index{collect\_degrees\_positive() (pypath.core.network.Network method)@\spxentry{collect\_degrees\_positive()}\spxextra{pypath.core.network.Network method}}

\begin{fulllineitems}
\phantomsection\label{\detokenize{reference:pypath.core.network.Network.collect_degrees_positive}}\pysiglinewithargsret{\sphinxbfcode{\sphinxupquote{collect\_degrees\_positive}}}{\emph{direction=None}, \emph{effect=None}, \emph{resources=None}, \emph{data\_model=None}, \emph{interaction\_type=None}, \emph{via=None}, \emph{references=None}}{}
Builds a comprehensive collection of \sphinxtitleref{degrees\_positive} entities across the network, counts unique and shared objects by resource, data model and interaction types.

\end{fulllineitems}

\index{collect\_degrees\_positive\_in() (pypath.core.network.Network method)@\spxentry{collect\_degrees\_positive\_in()}\spxextra{pypath.core.network.Network method}}

\begin{fulllineitems}
\phantomsection\label{\detokenize{reference:pypath.core.network.Network.collect_degrees_positive_in}}\pysiglinewithargsret{\sphinxbfcode{\sphinxupquote{collect\_degrees\_positive\_in}}}{\emph{direction=None}, \emph{effect=None}, \emph{resources=None}, \emph{data\_model=None}, \emph{interaction\_type=None}, \emph{via=None}, \emph{references=None}}{}
Builds a comprehensive collection of \sphinxtitleref{degrees\_positive\_in} entities across the network, counts unique and shared objects by resource, data model and interaction types.

\end{fulllineitems}

\index{collect\_degrees\_positive\_out() (pypath.core.network.Network method)@\spxentry{collect\_degrees\_positive\_out()}\spxextra{pypath.core.network.Network method}}

\begin{fulllineitems}
\phantomsection\label{\detokenize{reference:pypath.core.network.Network.collect_degrees_positive_out}}\pysiglinewithargsret{\sphinxbfcode{\sphinxupquote{collect\_degrees\_positive\_out}}}{\emph{direction=None}, \emph{effect=None}, \emph{resources=None}, \emph{data\_model=None}, \emph{interaction\_type=None}, \emph{via=None}, \emph{references=None}}{}
Builds a comprehensive collection of \sphinxtitleref{degrees\_positive\_out} entities across the network, counts unique and shared objects by resource, data model and interaction types.

\end{fulllineitems}

\index{collect\_degrees\_signed() (pypath.core.network.Network method)@\spxentry{collect\_degrees\_signed()}\spxextra{pypath.core.network.Network method}}

\begin{fulllineitems}
\phantomsection\label{\detokenize{reference:pypath.core.network.Network.collect_degrees_signed}}\pysiglinewithargsret{\sphinxbfcode{\sphinxupquote{collect\_degrees\_signed}}}{\emph{direction=None}, \emph{effect=None}, \emph{resources=None}, \emph{data\_model=None}, \emph{interaction\_type=None}, \emph{via=None}, \emph{references=None}}{}
Builds a comprehensive collection of \sphinxtitleref{degrees\_signed} entities across the network, counts unique and shared objects by resource, data model and interaction types.

\end{fulllineitems}

\index{collect\_degrees\_signed\_in() (pypath.core.network.Network method)@\spxentry{collect\_degrees\_signed\_in()}\spxextra{pypath.core.network.Network method}}

\begin{fulllineitems}
\phantomsection\label{\detokenize{reference:pypath.core.network.Network.collect_degrees_signed_in}}\pysiglinewithargsret{\sphinxbfcode{\sphinxupquote{collect\_degrees\_signed\_in}}}{\emph{direction=None}, \emph{effect=None}, \emph{resources=None}, \emph{data\_model=None}, \emph{interaction\_type=None}, \emph{via=None}, \emph{references=None}}{}
Builds a comprehensive collection of \sphinxtitleref{degrees\_signed\_in} entities across the network, counts unique and shared objects by resource, data model and interaction types.

\end{fulllineitems}

\index{collect\_degrees\_signed\_out() (pypath.core.network.Network method)@\spxentry{collect\_degrees\_signed\_out()}\spxextra{pypath.core.network.Network method}}

\begin{fulllineitems}
\phantomsection\label{\detokenize{reference:pypath.core.network.Network.collect_degrees_signed_out}}\pysiglinewithargsret{\sphinxbfcode{\sphinxupquote{collect\_degrees\_signed\_out}}}{\emph{direction=None}, \emph{effect=None}, \emph{resources=None}, \emph{data\_model=None}, \emph{interaction\_type=None}, \emph{via=None}, \emph{references=None}}{}
Builds a comprehensive collection of \sphinxtitleref{degrees\_signed\_out} entities across the network, counts unique and shared objects by resource, data model and interaction types.

\end{fulllineitems}

\index{collect\_degrees\_undirected() (pypath.core.network.Network method)@\spxentry{collect\_degrees\_undirected()}\spxextra{pypath.core.network.Network method}}

\begin{fulllineitems}
\phantomsection\label{\detokenize{reference:pypath.core.network.Network.collect_degrees_undirected}}\pysiglinewithargsret{\sphinxbfcode{\sphinxupquote{collect\_degrees\_undirected}}}{\emph{direction=None}, \emph{effect=None}, \emph{resources=None}, \emph{data\_model=None}, \emph{interaction\_type=None}, \emph{via=None}, \emph{references=None}}{}
Builds a comprehensive collection of \sphinxtitleref{degrees\_undirected} entities across the network, counts unique and shared objects by resource, data model and interaction types.

\end{fulllineitems}

\index{collect\_entities() (pypath.core.network.Network method)@\spxentry{collect\_entities()}\spxextra{pypath.core.network.Network method}}

\begin{fulllineitems}
\phantomsection\label{\detokenize{reference:pypath.core.network.Network.collect_entities}}\pysiglinewithargsret{\sphinxbfcode{\sphinxupquote{collect\_entities}}}{\emph{effect=None}, \emph{resources=None}, \emph{data\_model=None}, \emph{interaction\_type=None}, \emph{via=None}, \emph{references=None}}{}
Builds a comprehensive collection of \sphinxtitleref{entities} entities across the network, counts unique and shared objects by resource, data model and interaction types.

\end{fulllineitems}

\index{collect\_evidences() (pypath.core.network.Network method)@\spxentry{collect\_evidences()}\spxextra{pypath.core.network.Network method}}

\begin{fulllineitems}
\phantomsection\label{\detokenize{reference:pypath.core.network.Network.collect_evidences}}\pysiglinewithargsret{\sphinxbfcode{\sphinxupquote{collect\_evidences}}}{\emph{effect=None}, \emph{resources=None}, \emph{data\_model=None}, \emph{interaction\_type=None}, \emph{via=None}, \emph{references=None}}{}
Builds a comprehensive collection of \sphinxtitleref{evidences} entities across the network, counts unique and shared objects by resource, data model and interaction types.

\end{fulllineitems}

\index{collect\_identifiers() (pypath.core.network.Network method)@\spxentry{collect\_identifiers()}\spxextra{pypath.core.network.Network method}}

\begin{fulllineitems}
\phantomsection\label{\detokenize{reference:pypath.core.network.Network.collect_identifiers}}\pysiglinewithargsret{\sphinxbfcode{\sphinxupquote{collect\_identifiers}}}{\emph{effect=None}, \emph{resources=None}, \emph{data\_model=None}, \emph{interaction\_type=None}, \emph{via=None}, \emph{references=None}}{}
Builds a comprehensive collection of \sphinxtitleref{identifiers} entities across the network, counts unique and shared objects by resource, data model and interaction types.

\end{fulllineitems}

\index{collect\_interaction\_types() (pypath.core.network.Network method)@\spxentry{collect\_interaction\_types()}\spxextra{pypath.core.network.Network method}}

\begin{fulllineitems}
\phantomsection\label{\detokenize{reference:pypath.core.network.Network.collect_interaction_types}}\pysiglinewithargsret{\sphinxbfcode{\sphinxupquote{collect\_interaction\_types}}}{\emph{effect=None}, \emph{resources=None}, \emph{data\_model=None}, \emph{interaction\_type=None}, \emph{via=None}, \emph{references=None}}{}
Builds a comprehensive collection of \sphinxtitleref{interaction\_types} entities across the network, counts unique and shared objects by resource, data model and interaction types.

\end{fulllineitems}

\index{collect\_interactions() (pypath.core.network.Network method)@\spxentry{collect\_interactions()}\spxextra{pypath.core.network.Network method}}

\begin{fulllineitems}
\phantomsection\label{\detokenize{reference:pypath.core.network.Network.collect_interactions}}\pysiglinewithargsret{\sphinxbfcode{\sphinxupquote{collect\_interactions}}}{\emph{effect=None}, \emph{resources=None}, \emph{data\_model=None}, \emph{interaction\_type=None}, \emph{via=None}, \emph{references=None}}{}
Builds a comprehensive collection of \sphinxtitleref{interactions} entities across the network, counts unique and shared objects by resource, data model and interaction types.

\end{fulllineitems}

\index{collect\_interactions\_0() (pypath.core.network.Network method)@\spxentry{collect\_interactions\_0()}\spxextra{pypath.core.network.Network method}}

\begin{fulllineitems}
\phantomsection\label{\detokenize{reference:pypath.core.network.Network.collect_interactions_0}}\pysiglinewithargsret{\sphinxbfcode{\sphinxupquote{collect\_interactions\_0}}}{\emph{effect=None}, \emph{resources=None}, \emph{data\_model=None}, \emph{interaction\_type=None}, \emph{via=None}, \emph{references=None}}{}
Builds a comprehensive collection of \sphinxtitleref{interactions\_0} entities across the network, counts unique and shared objects by resource, data model and interaction types.

\end{fulllineitems}

\index{collect\_interactions\_directed() (pypath.core.network.Network method)@\spxentry{collect\_interactions\_directed()}\spxextra{pypath.core.network.Network method}}

\begin{fulllineitems}
\phantomsection\label{\detokenize{reference:pypath.core.network.Network.collect_interactions_directed}}\pysiglinewithargsret{\sphinxbfcode{\sphinxupquote{collect\_interactions\_directed}}}{\emph{effect=None}, \emph{resources=None}, \emph{data\_model=None}, \emph{interaction\_type=None}, \emph{via=None}, \emph{references=None}}{}
Builds a comprehensive collection of \sphinxtitleref{interactions\_directed} entities across the network, counts unique and shared objects by resource, data model and interaction types.

\end{fulllineitems}

\index{collect\_interactions\_mutual() (pypath.core.network.Network method)@\spxentry{collect\_interactions\_mutual()}\spxextra{pypath.core.network.Network method}}

\begin{fulllineitems}
\phantomsection\label{\detokenize{reference:pypath.core.network.Network.collect_interactions_mutual}}\pysiglinewithargsret{\sphinxbfcode{\sphinxupquote{collect\_interactions\_mutual}}}{\emph{effect=None}, \emph{resources=None}, \emph{data\_model=None}, \emph{interaction\_type=None}, \emph{via=None}, \emph{references=None}}{}
Builds a comprehensive collection of \sphinxtitleref{interactions\_mutual} entities across the network, counts unique and shared objects by resource, data model and interaction types.

\end{fulllineitems}

\index{collect\_interactions\_negative() (pypath.core.network.Network method)@\spxentry{collect\_interactions\_negative()}\spxextra{pypath.core.network.Network method}}

\begin{fulllineitems}
\phantomsection\label{\detokenize{reference:pypath.core.network.Network.collect_interactions_negative}}\pysiglinewithargsret{\sphinxbfcode{\sphinxupquote{collect\_interactions\_negative}}}{\emph{effect=None}, \emph{resources=None}, \emph{data\_model=None}, \emph{interaction\_type=None}, \emph{via=None}, \emph{references=None}}{}
Builds a comprehensive collection of \sphinxtitleref{interactions\_negative} entities across the network, counts unique and shared objects by resource, data model and interaction types.

\end{fulllineitems}

\index{collect\_interactions\_non\_directed() (pypath.core.network.Network method)@\spxentry{collect\_interactions\_non\_directed()}\spxextra{pypath.core.network.Network method}}

\begin{fulllineitems}
\phantomsection\label{\detokenize{reference:pypath.core.network.Network.collect_interactions_non_directed}}\pysiglinewithargsret{\sphinxbfcode{\sphinxupquote{collect\_interactions\_non\_directed}}}{\emph{effect=None}, \emph{resources=None}, \emph{data\_model=None}, \emph{interaction\_type=None}, \emph{via=None}, \emph{references=None}}{}
Builds a comprehensive collection of \sphinxtitleref{interactions\_non\_directed} entities across the network, counts unique and shared objects by resource, data model and interaction types.

\end{fulllineitems}

\index{collect\_interactions\_non\_directed\_0() (pypath.core.network.Network method)@\spxentry{collect\_interactions\_non\_directed\_0()}\spxextra{pypath.core.network.Network method}}

\begin{fulllineitems}
\phantomsection\label{\detokenize{reference:pypath.core.network.Network.collect_interactions_non_directed_0}}\pysiglinewithargsret{\sphinxbfcode{\sphinxupquote{collect\_interactions\_non\_directed\_0}}}{\emph{effect=None}, \emph{resources=None}, \emph{data\_model=None}, \emph{interaction\_type=None}, \emph{via=None}, \emph{references=None}}{}
Builds a comprehensive collection of \sphinxtitleref{interactions\_non\_directed\_0} entities across the network, counts unique and shared objects by resource, data model and interaction types.

\end{fulllineitems}

\index{collect\_interactions\_positive() (pypath.core.network.Network method)@\spxentry{collect\_interactions\_positive()}\spxextra{pypath.core.network.Network method}}

\begin{fulllineitems}
\phantomsection\label{\detokenize{reference:pypath.core.network.Network.collect_interactions_positive}}\pysiglinewithargsret{\sphinxbfcode{\sphinxupquote{collect\_interactions\_positive}}}{\emph{effect=None}, \emph{resources=None}, \emph{data\_model=None}, \emph{interaction\_type=None}, \emph{via=None}, \emph{references=None}}{}
Builds a comprehensive collection of \sphinxtitleref{interactions\_positive} entities across the network, counts unique and shared objects by resource, data model and interaction types.

\end{fulllineitems}

\index{collect\_interactions\_signed() (pypath.core.network.Network method)@\spxentry{collect\_interactions\_signed()}\spxextra{pypath.core.network.Network method}}

\begin{fulllineitems}
\phantomsection\label{\detokenize{reference:pypath.core.network.Network.collect_interactions_signed}}\pysiglinewithargsret{\sphinxbfcode{\sphinxupquote{collect\_interactions\_signed}}}{\emph{effect=None}, \emph{resources=None}, \emph{data\_model=None}, \emph{interaction\_type=None}, \emph{via=None}, \emph{references=None}}{}
Builds a comprehensive collection of \sphinxtitleref{interactions\_signed} entities across the network, counts unique and shared objects by resource, data model and interaction types.

\end{fulllineitems}

\index{collect\_interactions\_undirected() (pypath.core.network.Network method)@\spxentry{collect\_interactions\_undirected()}\spxextra{pypath.core.network.Network method}}

\begin{fulllineitems}
\phantomsection\label{\detokenize{reference:pypath.core.network.Network.collect_interactions_undirected}}\pysiglinewithargsret{\sphinxbfcode{\sphinxupquote{collect\_interactions\_undirected}}}{\emph{effect=None}, \emph{resources=None}, \emph{data\_model=None}, \emph{interaction\_type=None}, \emph{via=None}, \emph{references=None}}{}
Builds a comprehensive collection of \sphinxtitleref{interactions\_undirected} entities across the network, counts unique and shared objects by resource, data model and interaction types.

\end{fulllineitems}

\index{collect\_interactions\_undirected\_0() (pypath.core.network.Network method)@\spxentry{collect\_interactions\_undirected\_0()}\spxextra{pypath.core.network.Network method}}

\begin{fulllineitems}
\phantomsection\label{\detokenize{reference:pypath.core.network.Network.collect_interactions_undirected_0}}\pysiglinewithargsret{\sphinxbfcode{\sphinxupquote{collect\_interactions\_undirected\_0}}}{\emph{effect=None}, \emph{resources=None}, \emph{data\_model=None}, \emph{interaction\_type=None}, \emph{via=None}, \emph{references=None}}{}
Builds a comprehensive collection of \sphinxtitleref{interactions\_undirected\_0} entities across the network, counts unique and shared objects by resource, data model and interaction types.

\end{fulllineitems}

\index{collect\_labels() (pypath.core.network.Network method)@\spxentry{collect\_labels()}\spxextra{pypath.core.network.Network method}}

\begin{fulllineitems}
\phantomsection\label{\detokenize{reference:pypath.core.network.Network.collect_labels}}\pysiglinewithargsret{\sphinxbfcode{\sphinxupquote{collect\_labels}}}{\emph{effect=None}, \emph{resources=None}, \emph{data\_model=None}, \emph{interaction\_type=None}, \emph{via=None}, \emph{references=None}}{}
Builds a comprehensive collection of \sphinxtitleref{labels} entities across the network, counts unique and shared objects by resource, data model and interaction types.

\end{fulllineitems}

\index{collect\_lncrna\_identifiers() (pypath.core.network.Network method)@\spxentry{collect\_lncrna\_identifiers()}\spxextra{pypath.core.network.Network method}}

\begin{fulllineitems}
\phantomsection\label{\detokenize{reference:pypath.core.network.Network.collect_lncrna_identifiers}}\pysiglinewithargsret{\sphinxbfcode{\sphinxupquote{collect\_lncrna\_identifiers}}}{\emph{effect=None}, \emph{resources=None}, \emph{data\_model=None}, \emph{interaction\_type=None}, \emph{via=None}, \emph{references=None}}{}
Builds a comprehensive collection of \sphinxtitleref{lncrna\_identifiers} entities across the network, counts unique and shared objects by resource, data model and interaction types.

\end{fulllineitems}

\index{collect\_lncrna\_labels() (pypath.core.network.Network method)@\spxentry{collect\_lncrna\_labels()}\spxextra{pypath.core.network.Network method}}

\begin{fulllineitems}
\phantomsection\label{\detokenize{reference:pypath.core.network.Network.collect_lncrna_labels}}\pysiglinewithargsret{\sphinxbfcode{\sphinxupquote{collect\_lncrna\_labels}}}{\emph{effect=None}, \emph{resources=None}, \emph{data\_model=None}, \emph{interaction\_type=None}, \emph{via=None}, \emph{references=None}}{}
Builds a comprehensive collection of \sphinxtitleref{lncrna\_labels} entities across the network, counts unique and shared objects by resource, data model and interaction types.

\end{fulllineitems}

\index{collect\_lncrnas() (pypath.core.network.Network method)@\spxentry{collect\_lncrnas()}\spxextra{pypath.core.network.Network method}}

\begin{fulllineitems}
\phantomsection\label{\detokenize{reference:pypath.core.network.Network.collect_lncrnas}}\pysiglinewithargsret{\sphinxbfcode{\sphinxupquote{collect\_lncrnas}}}{\emph{effect=None}, \emph{resources=None}, \emph{data\_model=None}, \emph{interaction\_type=None}, \emph{via=None}, \emph{references=None}}{}
Builds a comprehensive collection of \sphinxtitleref{lncrnas} entities across the network, counts unique and shared objects by resource, data model and interaction types.

\end{fulllineitems}

\index{collect\_mirna\_identifiers() (pypath.core.network.Network method)@\spxentry{collect\_mirna\_identifiers()}\spxextra{pypath.core.network.Network method}}

\begin{fulllineitems}
\phantomsection\label{\detokenize{reference:pypath.core.network.Network.collect_mirna_identifiers}}\pysiglinewithargsret{\sphinxbfcode{\sphinxupquote{collect\_mirna\_identifiers}}}{\emph{effect=None}, \emph{resources=None}, \emph{data\_model=None}, \emph{interaction\_type=None}, \emph{via=None}, \emph{references=None}}{}
Builds a comprehensive collection of \sphinxtitleref{mirna\_identifiers} entities across the network, counts unique and shared objects by resource, data model and interaction types.

\end{fulllineitems}

\index{collect\_mirna\_labels() (pypath.core.network.Network method)@\spxentry{collect\_mirna\_labels()}\spxextra{pypath.core.network.Network method}}

\begin{fulllineitems}
\phantomsection\label{\detokenize{reference:pypath.core.network.Network.collect_mirna_labels}}\pysiglinewithargsret{\sphinxbfcode{\sphinxupquote{collect\_mirna\_labels}}}{\emph{effect=None}, \emph{resources=None}, \emph{data\_model=None}, \emph{interaction\_type=None}, \emph{via=None}, \emph{references=None}}{}
Builds a comprehensive collection of \sphinxtitleref{mirna\_labels} entities across the network, counts unique and shared objects by resource, data model and interaction types.

\end{fulllineitems}

\index{collect\_mirnas() (pypath.core.network.Network method)@\spxentry{collect\_mirnas()}\spxextra{pypath.core.network.Network method}}

\begin{fulllineitems}
\phantomsection\label{\detokenize{reference:pypath.core.network.Network.collect_mirnas}}\pysiglinewithargsret{\sphinxbfcode{\sphinxupquote{collect\_mirnas}}}{\emph{effect=None}, \emph{resources=None}, \emph{data\_model=None}, \emph{interaction\_type=None}, \emph{via=None}, \emph{references=None}}{}
Builds a comprehensive collection of \sphinxtitleref{mirnas} entities across the network, counts unique and shared objects by resource, data model and interaction types.

\end{fulllineitems}

\index{collect\_protein\_identifiers() (pypath.core.network.Network method)@\spxentry{collect\_protein\_identifiers()}\spxextra{pypath.core.network.Network method}}

\begin{fulllineitems}
\phantomsection\label{\detokenize{reference:pypath.core.network.Network.collect_protein_identifiers}}\pysiglinewithargsret{\sphinxbfcode{\sphinxupquote{collect\_protein\_identifiers}}}{\emph{effect=None}, \emph{resources=None}, \emph{data\_model=None}, \emph{interaction\_type=None}, \emph{via=None}, \emph{references=None}}{}
Builds a comprehensive collection of \sphinxtitleref{protein\_identifiers} entities across the network, counts unique and shared objects by resource, data model and interaction types.

\end{fulllineitems}

\index{collect\_protein\_labels() (pypath.core.network.Network method)@\spxentry{collect\_protein\_labels()}\spxextra{pypath.core.network.Network method}}

\begin{fulllineitems}
\phantomsection\label{\detokenize{reference:pypath.core.network.Network.collect_protein_labels}}\pysiglinewithargsret{\sphinxbfcode{\sphinxupquote{collect\_protein\_labels}}}{\emph{effect=None}, \emph{resources=None}, \emph{data\_model=None}, \emph{interaction\_type=None}, \emph{via=None}, \emph{references=None}}{}
Builds a comprehensive collection of \sphinxtitleref{protein\_labels} entities across the network, counts unique and shared objects by resource, data model and interaction types.

\end{fulllineitems}

\index{collect\_proteins() (pypath.core.network.Network method)@\spxentry{collect\_proteins()}\spxextra{pypath.core.network.Network method}}

\begin{fulllineitems}
\phantomsection\label{\detokenize{reference:pypath.core.network.Network.collect_proteins}}\pysiglinewithargsret{\sphinxbfcode{\sphinxupquote{collect\_proteins}}}{\emph{effect=None}, \emph{resources=None}, \emph{data\_model=None}, \emph{interaction\_type=None}, \emph{via=None}, \emph{references=None}}{}
Builds a comprehensive collection of \sphinxtitleref{proteins} entities across the network, counts unique and shared objects by resource, data model and interaction types.

\end{fulllineitems}

\index{collect\_references() (pypath.core.network.Network method)@\spxentry{collect\_references()}\spxextra{pypath.core.network.Network method}}

\begin{fulllineitems}
\phantomsection\label{\detokenize{reference:pypath.core.network.Network.collect_references}}\pysiglinewithargsret{\sphinxbfcode{\sphinxupquote{collect\_references}}}{\emph{effect=None}, \emph{resources=None}, \emph{data\_model=None}, \emph{interaction\_type=None}, \emph{via=None}, \emph{references=None}}{}
Builds a comprehensive collection of \sphinxtitleref{references} entities across the network, counts unique and shared objects by resource, data model and interaction types.

\end{fulllineitems}

\index{collect\_resource\_names() (pypath.core.network.Network method)@\spxentry{collect\_resource\_names()}\spxextra{pypath.core.network.Network method}}

\begin{fulllineitems}
\phantomsection\label{\detokenize{reference:pypath.core.network.Network.collect_resource_names}}\pysiglinewithargsret{\sphinxbfcode{\sphinxupquote{collect\_resource\_names}}}{\emph{effect=None}, \emph{resources=None}, \emph{data\_model=None}, \emph{interaction\_type=None}, \emph{via=None}, \emph{references=None}}{}
Builds a comprehensive collection of \sphinxtitleref{resource\_names} entities across the network, counts unique and shared objects by resource, data model and interaction types.

\end{fulllineitems}

\index{collect\_resource\_names\_via() (pypath.core.network.Network method)@\spxentry{collect\_resource\_names\_via()}\spxextra{pypath.core.network.Network method}}

\begin{fulllineitems}
\phantomsection\label{\detokenize{reference:pypath.core.network.Network.collect_resource_names_via}}\pysiglinewithargsret{\sphinxbfcode{\sphinxupquote{collect\_resource\_names\_via}}}{\emph{effect=None}, \emph{resources=None}, \emph{data\_model=None}, \emph{interaction\_type=None}, \emph{via=None}, \emph{references=None}}{}
Builds a comprehensive collection of \sphinxtitleref{resource\_names\_via} entities across the network, counts unique and shared objects by resource, data model and interaction types.

\end{fulllineitems}

\index{collect\_resources() (pypath.core.network.Network method)@\spxentry{collect\_resources()}\spxextra{pypath.core.network.Network method}}

\begin{fulllineitems}
\phantomsection\label{\detokenize{reference:pypath.core.network.Network.collect_resources}}\pysiglinewithargsret{\sphinxbfcode{\sphinxupquote{collect\_resources}}}{\emph{effect=None}, \emph{resources=None}, \emph{data\_model=None}, \emph{interaction\_type=None}, \emph{via=None}, \emph{references=None}}{}
Builds a comprehensive collection of \sphinxtitleref{resources} entities across the network, counts unique and shared objects by resource, data model and interaction types.

\end{fulllineitems}

\index{collect\_resources\_via() (pypath.core.network.Network method)@\spxentry{collect\_resources\_via()}\spxextra{pypath.core.network.Network method}}

\begin{fulllineitems}
\phantomsection\label{\detokenize{reference:pypath.core.network.Network.collect_resources_via}}\pysiglinewithargsret{\sphinxbfcode{\sphinxupquote{collect\_resources\_via}}}{\emph{effect=None}, \emph{resources=None}, \emph{data\_model=None}, \emph{interaction\_type=None}, \emph{via=None}, \emph{references=None}}{}
Builds a comprehensive collection of \sphinxtitleref{resources\_via} entities across the network, counts unique and shared objects by resource, data model and interaction types.

\end{fulllineitems}

\index{collect\_small\_molecule\_identifiers() (pypath.core.network.Network method)@\spxentry{collect\_small\_molecule\_identifiers()}\spxextra{pypath.core.network.Network method}}

\begin{fulllineitems}
\phantomsection\label{\detokenize{reference:pypath.core.network.Network.collect_small_molecule_identifiers}}\pysiglinewithargsret{\sphinxbfcode{\sphinxupquote{collect\_small\_molecule\_identifiers}}}{\emph{effect=None}, \emph{resources=None}, \emph{data\_model=None}, \emph{interaction\_type=None}, \emph{via=None}, \emph{references=None}}{}
Builds a comprehensive collection of \sphinxtitleref{small\_molecule\_identifiers} entities across the network, counts unique and shared objects by resource, data model and interaction types.

\end{fulllineitems}

\index{collect\_small\_molecule\_labels() (pypath.core.network.Network method)@\spxentry{collect\_small\_molecule\_labels()}\spxextra{pypath.core.network.Network method}}

\begin{fulllineitems}
\phantomsection\label{\detokenize{reference:pypath.core.network.Network.collect_small_molecule_labels}}\pysiglinewithargsret{\sphinxbfcode{\sphinxupquote{collect\_small\_molecule\_labels}}}{\emph{effect=None}, \emph{resources=None}, \emph{data\_model=None}, \emph{interaction\_type=None}, \emph{via=None}, \emph{references=None}}{}
Builds a comprehensive collection of \sphinxtitleref{small\_molecule\_labels} entities across the network, counts unique and shared objects by resource, data model and interaction types.

\end{fulllineitems}

\index{collect\_small\_molecules() (pypath.core.network.Network method)@\spxentry{collect\_small\_molecules()}\spxextra{pypath.core.network.Network method}}

\begin{fulllineitems}
\phantomsection\label{\detokenize{reference:pypath.core.network.Network.collect_small_molecules}}\pysiglinewithargsret{\sphinxbfcode{\sphinxupquote{collect\_small\_molecules}}}{\emph{effect=None}, \emph{resources=None}, \emph{data\_model=None}, \emph{interaction\_type=None}, \emph{via=None}, \emph{references=None}}{}
Builds a comprehensive collection of \sphinxtitleref{small\_molecules} entities across the network, counts unique and shared objects by resource, data model and interaction types.

\end{fulllineitems}

\index{complex\_identifiers\_by\_data\_model() (pypath.core.network.Network method)@\spxentry{complex\_identifiers\_by\_data\_model()}\spxextra{pypath.core.network.Network method}}

\begin{fulllineitems}
\phantomsection\label{\detokenize{reference:pypath.core.network.Network.complex_identifiers_by_data_model}}\pysiglinewithargsret{\sphinxbfcode{\sphinxupquote{complex\_identifiers\_by\_data\_model}}}{}{}
Built-in immutable sequence.

If no argument is given, the constructor returns an empty tuple.
If iterable is specified the tuple is initialized from iterable’s items.

If the argument is a tuple, the return value is the same object.

\end{fulllineitems}

\index{complex\_identifiers\_by\_interaction\_type() (pypath.core.network.Network method)@\spxentry{complex\_identifiers\_by\_interaction\_type()}\spxextra{pypath.core.network.Network method}}

\begin{fulllineitems}
\phantomsection\label{\detokenize{reference:pypath.core.network.Network.complex_identifiers_by_interaction_type}}\pysiglinewithargsret{\sphinxbfcode{\sphinxupquote{complex\_identifiers\_by\_interaction\_type}}}{}{}
Built-in immutable sequence.

If no argument is given, the constructor returns an empty tuple.
If iterable is specified the tuple is initialized from iterable’s items.

If the argument is a tuple, the return value is the same object.

\end{fulllineitems}

\index{complex\_identifiers\_by\_interaction\_type\_and\_data\_model() (pypath.core.network.Network method)@\spxentry{complex\_identifiers\_by\_interaction\_type\_and\_data\_model()}\spxextra{pypath.core.network.Network method}}

\begin{fulllineitems}
\phantomsection\label{\detokenize{reference:pypath.core.network.Network.complex_identifiers_by_interaction_type_and_data_model}}\pysiglinewithargsret{\sphinxbfcode{\sphinxupquote{complex\_identifiers\_by\_interaction\_type\_and\_data\_model}}}{}{}
Built-in immutable sequence.

If no argument is given, the constructor returns an empty tuple.
If iterable is specified the tuple is initialized from iterable’s items.

If the argument is a tuple, the return value is the same object.

\end{fulllineitems}

\index{complex\_identifiers\_by\_interaction\_type\_and\_data\_model\_and\_resource() (pypath.core.network.Network method)@\spxentry{complex\_identifiers\_by\_interaction\_type\_and\_data\_model\_and\_resource()}\spxextra{pypath.core.network.Network method}}

\begin{fulllineitems}
\phantomsection\label{\detokenize{reference:pypath.core.network.Network.complex_identifiers_by_interaction_type_and_data_model_and_resource}}\pysiglinewithargsret{\sphinxbfcode{\sphinxupquote{complex\_identifiers\_by\_interaction\_type\_and\_data\_model\_and\_resource}}}{}{}
Built-in immutable sequence.

If no argument is given, the constructor returns an empty tuple.
If iterable is specified the tuple is initialized from iterable’s items.

If the argument is a tuple, the return value is the same object.

\end{fulllineitems}

\index{complex\_identifiers\_by\_reference() (pypath.core.network.Network method)@\spxentry{complex\_identifiers\_by\_reference()}\spxextra{pypath.core.network.Network method}}

\begin{fulllineitems}
\phantomsection\label{\detokenize{reference:pypath.core.network.Network.complex_identifiers_by_reference}}\pysiglinewithargsret{\sphinxbfcode{\sphinxupquote{complex\_identifiers\_by\_reference}}}{}{}
Built-in immutable sequence.

If no argument is given, the constructor returns an empty tuple.
If iterable is specified the tuple is initialized from iterable’s items.

If the argument is a tuple, the return value is the same object.

\end{fulllineitems}

\index{complex\_identifiers\_by\_resource() (pypath.core.network.Network method)@\spxentry{complex\_identifiers\_by\_resource()}\spxextra{pypath.core.network.Network method}}

\begin{fulllineitems}
\phantomsection\label{\detokenize{reference:pypath.core.network.Network.complex_identifiers_by_resource}}\pysiglinewithargsret{\sphinxbfcode{\sphinxupquote{complex\_identifiers\_by\_resource}}}{}{}
Built-in immutable sequence.

If no argument is given, the constructor returns an empty tuple.
If iterable is specified the tuple is initialized from iterable’s items.

If the argument is a tuple, the return value is the same object.

\end{fulllineitems}

\index{complex\_labels\_by\_data\_model() (pypath.core.network.Network method)@\spxentry{complex\_labels\_by\_data\_model()}\spxextra{pypath.core.network.Network method}}

\begin{fulllineitems}
\phantomsection\label{\detokenize{reference:pypath.core.network.Network.complex_labels_by_data_model}}\pysiglinewithargsret{\sphinxbfcode{\sphinxupquote{complex\_labels\_by\_data\_model}}}{}{}
Built-in immutable sequence.

If no argument is given, the constructor returns an empty tuple.
If iterable is specified the tuple is initialized from iterable’s items.

If the argument is a tuple, the return value is the same object.

\end{fulllineitems}

\index{complex\_labels\_by\_interaction\_type() (pypath.core.network.Network method)@\spxentry{complex\_labels\_by\_interaction\_type()}\spxextra{pypath.core.network.Network method}}

\begin{fulllineitems}
\phantomsection\label{\detokenize{reference:pypath.core.network.Network.complex_labels_by_interaction_type}}\pysiglinewithargsret{\sphinxbfcode{\sphinxupquote{complex\_labels\_by\_interaction\_type}}}{}{}
Built-in immutable sequence.

If no argument is given, the constructor returns an empty tuple.
If iterable is specified the tuple is initialized from iterable’s items.

If the argument is a tuple, the return value is the same object.

\end{fulllineitems}

\index{complex\_labels\_by\_interaction\_type\_and\_data\_model() (pypath.core.network.Network method)@\spxentry{complex\_labels\_by\_interaction\_type\_and\_data\_model()}\spxextra{pypath.core.network.Network method}}

\begin{fulllineitems}
\phantomsection\label{\detokenize{reference:pypath.core.network.Network.complex_labels_by_interaction_type_and_data_model}}\pysiglinewithargsret{\sphinxbfcode{\sphinxupquote{complex\_labels\_by\_interaction\_type\_and\_data\_model}}}{}{}
Built-in immutable sequence.

If no argument is given, the constructor returns an empty tuple.
If iterable is specified the tuple is initialized from iterable’s items.

If the argument is a tuple, the return value is the same object.

\end{fulllineitems}

\index{complex\_labels\_by\_interaction\_type\_and\_data\_model\_and\_resource() (pypath.core.network.Network method)@\spxentry{complex\_labels\_by\_interaction\_type\_and\_data\_model\_and\_resource()}\spxextra{pypath.core.network.Network method}}

\begin{fulllineitems}
\phantomsection\label{\detokenize{reference:pypath.core.network.Network.complex_labels_by_interaction_type_and_data_model_and_resource}}\pysiglinewithargsret{\sphinxbfcode{\sphinxupquote{complex\_labels\_by\_interaction\_type\_and\_data\_model\_and\_resource}}}{}{}
Built-in immutable sequence.

If no argument is given, the constructor returns an empty tuple.
If iterable is specified the tuple is initialized from iterable’s items.

If the argument is a tuple, the return value is the same object.

\end{fulllineitems}

\index{complex\_labels\_by\_reference() (pypath.core.network.Network method)@\spxentry{complex\_labels\_by\_reference()}\spxextra{pypath.core.network.Network method}}

\begin{fulllineitems}
\phantomsection\label{\detokenize{reference:pypath.core.network.Network.complex_labels_by_reference}}\pysiglinewithargsret{\sphinxbfcode{\sphinxupquote{complex\_labels\_by\_reference}}}{}{}
Built-in immutable sequence.

If no argument is given, the constructor returns an empty tuple.
If iterable is specified the tuple is initialized from iterable’s items.

If the argument is a tuple, the return value is the same object.

\end{fulllineitems}

\index{complex\_labels\_by\_resource() (pypath.core.network.Network method)@\spxentry{complex\_labels\_by\_resource()}\spxextra{pypath.core.network.Network method}}

\begin{fulllineitems}
\phantomsection\label{\detokenize{reference:pypath.core.network.Network.complex_labels_by_resource}}\pysiglinewithargsret{\sphinxbfcode{\sphinxupquote{complex\_labels\_by\_resource}}}{}{}
Built-in immutable sequence.

If no argument is given, the constructor returns an empty tuple.
If iterable is specified the tuple is initialized from iterable’s items.

If the argument is a tuple, the return value is the same object.

\end{fulllineitems}

\index{complexes\_by\_data\_model() (pypath.core.network.Network method)@\spxentry{complexes\_by\_data\_model()}\spxextra{pypath.core.network.Network method}}

\begin{fulllineitems}
\phantomsection\label{\detokenize{reference:pypath.core.network.Network.complexes_by_data_model}}\pysiglinewithargsret{\sphinxbfcode{\sphinxupquote{complexes\_by\_data\_model}}}{}{}
Built-in immutable sequence.

If no argument is given, the constructor returns an empty tuple.
If iterable is specified the tuple is initialized from iterable’s items.

If the argument is a tuple, the return value is the same object.

\end{fulllineitems}

\index{complexes\_by\_interaction\_type() (pypath.core.network.Network method)@\spxentry{complexes\_by\_interaction\_type()}\spxextra{pypath.core.network.Network method}}

\begin{fulllineitems}
\phantomsection\label{\detokenize{reference:pypath.core.network.Network.complexes_by_interaction_type}}\pysiglinewithargsret{\sphinxbfcode{\sphinxupquote{complexes\_by\_interaction\_type}}}{}{}
Built-in immutable sequence.

If no argument is given, the constructor returns an empty tuple.
If iterable is specified the tuple is initialized from iterable’s items.

If the argument is a tuple, the return value is the same object.

\end{fulllineitems}

\index{complexes\_by\_interaction\_type\_and\_data\_model() (pypath.core.network.Network method)@\spxentry{complexes\_by\_interaction\_type\_and\_data\_model()}\spxextra{pypath.core.network.Network method}}

\begin{fulllineitems}
\phantomsection\label{\detokenize{reference:pypath.core.network.Network.complexes_by_interaction_type_and_data_model}}\pysiglinewithargsret{\sphinxbfcode{\sphinxupquote{complexes\_by\_interaction\_type\_and\_data\_model}}}{}{}
Built-in immutable sequence.

If no argument is given, the constructor returns an empty tuple.
If iterable is specified the tuple is initialized from iterable’s items.

If the argument is a tuple, the return value is the same object.

\end{fulllineitems}

\index{complexes\_by\_interaction\_type\_and\_data\_model\_and\_resource() (pypath.core.network.Network method)@\spxentry{complexes\_by\_interaction\_type\_and\_data\_model\_and\_resource()}\spxextra{pypath.core.network.Network method}}

\begin{fulllineitems}
\phantomsection\label{\detokenize{reference:pypath.core.network.Network.complexes_by_interaction_type_and_data_model_and_resource}}\pysiglinewithargsret{\sphinxbfcode{\sphinxupquote{complexes\_by\_interaction\_type\_and\_data\_model\_and\_resource}}}{}{}
Built-in immutable sequence.

If no argument is given, the constructor returns an empty tuple.
If iterable is specified the tuple is initialized from iterable’s items.

If the argument is a tuple, the return value is the same object.

\end{fulllineitems}

\index{complexes\_by\_reference() (pypath.core.network.Network method)@\spxentry{complexes\_by\_reference()}\spxextra{pypath.core.network.Network method}}

\begin{fulllineitems}
\phantomsection\label{\detokenize{reference:pypath.core.network.Network.complexes_by_reference}}\pysiglinewithargsret{\sphinxbfcode{\sphinxupquote{complexes\_by\_reference}}}{}{}
Built-in immutable sequence.

If no argument is given, the constructor returns an empty tuple.
If iterable is specified the tuple is initialized from iterable’s items.

If the argument is a tuple, the return value is the same object.

\end{fulllineitems}

\index{complexes\_by\_resource() (pypath.core.network.Network method)@\spxentry{complexes\_by\_resource()}\spxextra{pypath.core.network.Network method}}

\begin{fulllineitems}
\phantomsection\label{\detokenize{reference:pypath.core.network.Network.complexes_by_resource}}\pysiglinewithargsret{\sphinxbfcode{\sphinxupquote{complexes\_by\_resource}}}{}{}
Built-in immutable sequence.

If no argument is given, the constructor returns an empty tuple.
If iterable is specified the tuple is initialized from iterable’s items.

If the argument is a tuple, the return value is the same object.

\end{fulllineitems}

\index{count\_activated\_by() (pypath.core.network.Network method)@\spxentry{count\_activated\_by()}\spxextra{pypath.core.network.Network method}}

\begin{fulllineitems}
\phantomsection\label{\detokenize{reference:pypath.core.network.Network.count_activated_by}}\pysiglinewithargsret{\sphinxbfcode{\sphinxupquote{count\_activated\_by}}}{}{}
Returns the count of the interacting partners for one or more
entities according to the specified criteria.
Please refer to the docs of the \sphinxcode{\sphinxupquote{partners}} method.

\end{fulllineitems}

\index{count\_activates() (pypath.core.network.Network method)@\spxentry{count\_activates()}\spxextra{pypath.core.network.Network method}}

\begin{fulllineitems}
\phantomsection\label{\detokenize{reference:pypath.core.network.Network.count_activates}}\pysiglinewithargsret{\sphinxbfcode{\sphinxupquote{count\_activates}}}{}{}
Returns the count of the interacting partners for one or more
entities according to the specified criteria.
Please refer to the docs of the \sphinxcode{\sphinxupquote{partners}} method.

\end{fulllineitems}

\index{count\_complex\_identifiers() (pypath.core.network.Network method)@\spxentry{count\_complex\_identifiers()}\spxextra{pypath.core.network.Network method}}

\begin{fulllineitems}
\phantomsection\label{\detokenize{reference:pypath.core.network.Network.count_complex_identifiers}}\pysiglinewithargsret{\sphinxbfcode{\sphinxupquote{count\_complex\_identifiers}}}{}{}
str(object=’‘) -\textgreater{} str
str(bytes\_or\_buffer{[}, encoding{[}, errors{]}{]}) -\textgreater{} str

Create a new string object from the given object. If encoding or
errors is specified, then the object must expose a data buffer
that will be decoded using the given encoding and error handler.
Otherwise, returns the result of object.\_\_str\_\_() (if defined)
or repr(object).
encoding defaults to sys.getdefaultencoding().
errors defaults to ‘strict’.

\end{fulllineitems}

\index{count\_complex\_identifiers\_by\_data\_model() (pypath.core.network.Network method)@\spxentry{count\_complex\_identifiers\_by\_data\_model()}\spxextra{pypath.core.network.Network method}}

\begin{fulllineitems}
\phantomsection\label{\detokenize{reference:pypath.core.network.Network.count_complex_identifiers_by_data_model}}\pysiglinewithargsret{\sphinxbfcode{\sphinxupquote{count\_complex\_identifiers\_by\_data\_model}}}{}{}
str(object=’‘) -\textgreater{} str
str(bytes\_or\_buffer{[}, encoding{[}, errors{]}{]}) -\textgreater{} str

Create a new string object from the given object. If encoding or
errors is specified, then the object must expose a data buffer
that will be decoded using the given encoding and error handler.
Otherwise, returns the result of object.\_\_str\_\_() (if defined)
or repr(object).
encoding defaults to sys.getdefaultencoding().
errors defaults to ‘strict’.

\end{fulllineitems}

\index{count\_complex\_identifiers\_by\_interaction\_type() (pypath.core.network.Network method)@\spxentry{count\_complex\_identifiers\_by\_interaction\_type()}\spxextra{pypath.core.network.Network method}}

\begin{fulllineitems}
\phantomsection\label{\detokenize{reference:pypath.core.network.Network.count_complex_identifiers_by_interaction_type}}\pysiglinewithargsret{\sphinxbfcode{\sphinxupquote{count\_complex\_identifiers\_by\_interaction\_type}}}{}{}
str(object=’‘) -\textgreater{} str
str(bytes\_or\_buffer{[}, encoding{[}, errors{]}{]}) -\textgreater{} str

Create a new string object from the given object. If encoding or
errors is specified, then the object must expose a data buffer
that will be decoded using the given encoding and error handler.
Otherwise, returns the result of object.\_\_str\_\_() (if defined)
or repr(object).
encoding defaults to sys.getdefaultencoding().
errors defaults to ‘strict’.

\end{fulllineitems}

\index{count\_complex\_identifiers\_by\_interaction\_type\_and\_data\_model() (pypath.core.network.Network method)@\spxentry{count\_complex\_identifiers\_by\_interaction\_type\_and\_data\_model()}\spxextra{pypath.core.network.Network method}}

\begin{fulllineitems}
\phantomsection\label{\detokenize{reference:pypath.core.network.Network.count_complex_identifiers_by_interaction_type_and_data_model}}\pysiglinewithargsret{\sphinxbfcode{\sphinxupquote{count\_complex\_identifiers\_by\_interaction\_type\_and\_data\_model}}}{}{}
str(object=’‘) -\textgreater{} str
str(bytes\_or\_buffer{[}, encoding{[}, errors{]}{]}) -\textgreater{} str

Create a new string object from the given object. If encoding or
errors is specified, then the object must expose a data buffer
that will be decoded using the given encoding and error handler.
Otherwise, returns the result of object.\_\_str\_\_() (if defined)
or repr(object).
encoding defaults to sys.getdefaultencoding().
errors defaults to ‘strict’.

\end{fulllineitems}

\index{count\_complex\_identifiers\_by\_interaction\_type\_and\_data\_model\_and\_resource() (pypath.core.network.Network method)@\spxentry{count\_complex\_identifiers\_by\_interaction\_type\_and\_data\_model\_and\_resource()}\spxextra{pypath.core.network.Network method}}

\begin{fulllineitems}
\phantomsection\label{\detokenize{reference:pypath.core.network.Network.count_complex_identifiers_by_interaction_type_and_data_model_and_resource}}\pysiglinewithargsret{\sphinxbfcode{\sphinxupquote{count\_complex\_identifiers\_by\_interaction\_type\_and\_data\_model\_and\_resource}}}{}{}
str(object=’‘) -\textgreater{} str
str(bytes\_or\_buffer{[}, encoding{[}, errors{]}{]}) -\textgreater{} str

Create a new string object from the given object. If encoding or
errors is specified, then the object must expose a data buffer
that will be decoded using the given encoding and error handler.
Otherwise, returns the result of object.\_\_str\_\_() (if defined)
or repr(object).
encoding defaults to sys.getdefaultencoding().
errors defaults to ‘strict’.

\end{fulllineitems}

\index{count\_complex\_identifiers\_by\_reference() (pypath.core.network.Network method)@\spxentry{count\_complex\_identifiers\_by\_reference()}\spxextra{pypath.core.network.Network method}}

\begin{fulllineitems}
\phantomsection\label{\detokenize{reference:pypath.core.network.Network.count_complex_identifiers_by_reference}}\pysiglinewithargsret{\sphinxbfcode{\sphinxupquote{count\_complex\_identifiers\_by\_reference}}}{}{}
str(object=’‘) -\textgreater{} str
str(bytes\_or\_buffer{[}, encoding{[}, errors{]}{]}) -\textgreater{} str

Create a new string object from the given object. If encoding or
errors is specified, then the object must expose a data buffer
that will be decoded using the given encoding and error handler.
Otherwise, returns the result of object.\_\_str\_\_() (if defined)
or repr(object).
encoding defaults to sys.getdefaultencoding().
errors defaults to ‘strict’.

\end{fulllineitems}

\index{count\_complex\_identifiers\_by\_resource() (pypath.core.network.Network method)@\spxentry{count\_complex\_identifiers\_by\_resource()}\spxextra{pypath.core.network.Network method}}

\begin{fulllineitems}
\phantomsection\label{\detokenize{reference:pypath.core.network.Network.count_complex_identifiers_by_resource}}\pysiglinewithargsret{\sphinxbfcode{\sphinxupquote{count\_complex\_identifiers\_by\_resource}}}{}{}
str(object=’‘) -\textgreater{} str
str(bytes\_or\_buffer{[}, encoding{[}, errors{]}{]}) -\textgreater{} str

Create a new string object from the given object. If encoding or
errors is specified, then the object must expose a data buffer
that will be decoded using the given encoding and error handler.
Otherwise, returns the result of object.\_\_str\_\_() (if defined)
or repr(object).
encoding defaults to sys.getdefaultencoding().
errors defaults to ‘strict’.

\end{fulllineitems}

\index{count\_complex\_labels() (pypath.core.network.Network method)@\spxentry{count\_complex\_labels()}\spxextra{pypath.core.network.Network method}}

\begin{fulllineitems}
\phantomsection\label{\detokenize{reference:pypath.core.network.Network.count_complex_labels}}\pysiglinewithargsret{\sphinxbfcode{\sphinxupquote{count\_complex\_labels}}}{}{}
str(object=’‘) -\textgreater{} str
str(bytes\_or\_buffer{[}, encoding{[}, errors{]}{]}) -\textgreater{} str

Create a new string object from the given object. If encoding or
errors is specified, then the object must expose a data buffer
that will be decoded using the given encoding and error handler.
Otherwise, returns the result of object.\_\_str\_\_() (if defined)
or repr(object).
encoding defaults to sys.getdefaultencoding().
errors defaults to ‘strict’.

\end{fulllineitems}

\index{count\_complex\_labels\_by\_data\_model() (pypath.core.network.Network method)@\spxentry{count\_complex\_labels\_by\_data\_model()}\spxextra{pypath.core.network.Network method}}

\begin{fulllineitems}
\phantomsection\label{\detokenize{reference:pypath.core.network.Network.count_complex_labels_by_data_model}}\pysiglinewithargsret{\sphinxbfcode{\sphinxupquote{count\_complex\_labels\_by\_data\_model}}}{}{}
str(object=’‘) -\textgreater{} str
str(bytes\_or\_buffer{[}, encoding{[}, errors{]}{]}) -\textgreater{} str

Create a new string object from the given object. If encoding or
errors is specified, then the object must expose a data buffer
that will be decoded using the given encoding and error handler.
Otherwise, returns the result of object.\_\_str\_\_() (if defined)
or repr(object).
encoding defaults to sys.getdefaultencoding().
errors defaults to ‘strict’.

\end{fulllineitems}

\index{count\_complex\_labels\_by\_interaction\_type() (pypath.core.network.Network method)@\spxentry{count\_complex\_labels\_by\_interaction\_type()}\spxextra{pypath.core.network.Network method}}

\begin{fulllineitems}
\phantomsection\label{\detokenize{reference:pypath.core.network.Network.count_complex_labels_by_interaction_type}}\pysiglinewithargsret{\sphinxbfcode{\sphinxupquote{count\_complex\_labels\_by\_interaction\_type}}}{}{}
str(object=’‘) -\textgreater{} str
str(bytes\_or\_buffer{[}, encoding{[}, errors{]}{]}) -\textgreater{} str

Create a new string object from the given object. If encoding or
errors is specified, then the object must expose a data buffer
that will be decoded using the given encoding and error handler.
Otherwise, returns the result of object.\_\_str\_\_() (if defined)
or repr(object).
encoding defaults to sys.getdefaultencoding().
errors defaults to ‘strict’.

\end{fulllineitems}

\index{count\_complex\_labels\_by\_interaction\_type\_and\_data\_model() (pypath.core.network.Network method)@\spxentry{count\_complex\_labels\_by\_interaction\_type\_and\_data\_model()}\spxextra{pypath.core.network.Network method}}

\begin{fulllineitems}
\phantomsection\label{\detokenize{reference:pypath.core.network.Network.count_complex_labels_by_interaction_type_and_data_model}}\pysiglinewithargsret{\sphinxbfcode{\sphinxupquote{count\_complex\_labels\_by\_interaction\_type\_and\_data\_model}}}{}{}
str(object=’‘) -\textgreater{} str
str(bytes\_or\_buffer{[}, encoding{[}, errors{]}{]}) -\textgreater{} str

Create a new string object from the given object. If encoding or
errors is specified, then the object must expose a data buffer
that will be decoded using the given encoding and error handler.
Otherwise, returns the result of object.\_\_str\_\_() (if defined)
or repr(object).
encoding defaults to sys.getdefaultencoding().
errors defaults to ‘strict’.

\end{fulllineitems}

\index{count\_complex\_labels\_by\_interaction\_type\_and\_data\_model\_and\_resource() (pypath.core.network.Network method)@\spxentry{count\_complex\_labels\_by\_interaction\_type\_and\_data\_model\_and\_resource()}\spxextra{pypath.core.network.Network method}}

\begin{fulllineitems}
\phantomsection\label{\detokenize{reference:pypath.core.network.Network.count_complex_labels_by_interaction_type_and_data_model_and_resource}}\pysiglinewithargsret{\sphinxbfcode{\sphinxupquote{count\_complex\_labels\_by\_interaction\_type\_and\_data\_model\_and\_resource}}}{}{}
str(object=’‘) -\textgreater{} str
str(bytes\_or\_buffer{[}, encoding{[}, errors{]}{]}) -\textgreater{} str

Create a new string object from the given object. If encoding or
errors is specified, then the object must expose a data buffer
that will be decoded using the given encoding and error handler.
Otherwise, returns the result of object.\_\_str\_\_() (if defined)
or repr(object).
encoding defaults to sys.getdefaultencoding().
errors defaults to ‘strict’.

\end{fulllineitems}

\index{count\_complex\_labels\_by\_reference() (pypath.core.network.Network method)@\spxentry{count\_complex\_labels\_by\_reference()}\spxextra{pypath.core.network.Network method}}

\begin{fulllineitems}
\phantomsection\label{\detokenize{reference:pypath.core.network.Network.count_complex_labels_by_reference}}\pysiglinewithargsret{\sphinxbfcode{\sphinxupquote{count\_complex\_labels\_by\_reference}}}{}{}
str(object=’‘) -\textgreater{} str
str(bytes\_or\_buffer{[}, encoding{[}, errors{]}{]}) -\textgreater{} str

Create a new string object from the given object. If encoding or
errors is specified, then the object must expose a data buffer
that will be decoded using the given encoding and error handler.
Otherwise, returns the result of object.\_\_str\_\_() (if defined)
or repr(object).
encoding defaults to sys.getdefaultencoding().
errors defaults to ‘strict’.

\end{fulllineitems}

\index{count\_complex\_labels\_by\_resource() (pypath.core.network.Network method)@\spxentry{count\_complex\_labels\_by\_resource()}\spxextra{pypath.core.network.Network method}}

\begin{fulllineitems}
\phantomsection\label{\detokenize{reference:pypath.core.network.Network.count_complex_labels_by_resource}}\pysiglinewithargsret{\sphinxbfcode{\sphinxupquote{count\_complex\_labels\_by\_resource}}}{}{}
str(object=’‘) -\textgreater{} str
str(bytes\_or\_buffer{[}, encoding{[}, errors{]}{]}) -\textgreater{} str

Create a new string object from the given object. If encoding or
errors is specified, then the object must expose a data buffer
that will be decoded using the given encoding and error handler.
Otherwise, returns the result of object.\_\_str\_\_() (if defined)
or repr(object).
encoding defaults to sys.getdefaultencoding().
errors defaults to ‘strict’.

\end{fulllineitems}

\index{count\_complexes() (pypath.core.network.Network method)@\spxentry{count\_complexes()}\spxextra{pypath.core.network.Network method}}

\begin{fulllineitems}
\phantomsection\label{\detokenize{reference:pypath.core.network.Network.count_complexes}}\pysiglinewithargsret{\sphinxbfcode{\sphinxupquote{count\_complexes}}}{}{}
str(object=’‘) -\textgreater{} str
str(bytes\_or\_buffer{[}, encoding{[}, errors{]}{]}) -\textgreater{} str

Create a new string object from the given object. If encoding or
errors is specified, then the object must expose a data buffer
that will be decoded using the given encoding and error handler.
Otherwise, returns the result of object.\_\_str\_\_() (if defined)
or repr(object).
encoding defaults to sys.getdefaultencoding().
errors defaults to ‘strict’.

\end{fulllineitems}

\index{count\_complexes\_by\_data\_model() (pypath.core.network.Network method)@\spxentry{count\_complexes\_by\_data\_model()}\spxextra{pypath.core.network.Network method}}

\begin{fulllineitems}
\phantomsection\label{\detokenize{reference:pypath.core.network.Network.count_complexes_by_data_model}}\pysiglinewithargsret{\sphinxbfcode{\sphinxupquote{count\_complexes\_by\_data\_model}}}{}{}
str(object=’‘) -\textgreater{} str
str(bytes\_or\_buffer{[}, encoding{[}, errors{]}{]}) -\textgreater{} str

Create a new string object from the given object. If encoding or
errors is specified, then the object must expose a data buffer
that will be decoded using the given encoding and error handler.
Otherwise, returns the result of object.\_\_str\_\_() (if defined)
or repr(object).
encoding defaults to sys.getdefaultencoding().
errors defaults to ‘strict’.

\end{fulllineitems}

\index{count\_complexes\_by\_interaction\_type() (pypath.core.network.Network method)@\spxentry{count\_complexes\_by\_interaction\_type()}\spxextra{pypath.core.network.Network method}}

\begin{fulllineitems}
\phantomsection\label{\detokenize{reference:pypath.core.network.Network.count_complexes_by_interaction_type}}\pysiglinewithargsret{\sphinxbfcode{\sphinxupquote{count\_complexes\_by\_interaction\_type}}}{}{}
str(object=’‘) -\textgreater{} str
str(bytes\_or\_buffer{[}, encoding{[}, errors{]}{]}) -\textgreater{} str

Create a new string object from the given object. If encoding or
errors is specified, then the object must expose a data buffer
that will be decoded using the given encoding and error handler.
Otherwise, returns the result of object.\_\_str\_\_() (if defined)
or repr(object).
encoding defaults to sys.getdefaultencoding().
errors defaults to ‘strict’.

\end{fulllineitems}

\index{count\_complexes\_by\_interaction\_type\_and\_data\_model() (pypath.core.network.Network method)@\spxentry{count\_complexes\_by\_interaction\_type\_and\_data\_model()}\spxextra{pypath.core.network.Network method}}

\begin{fulllineitems}
\phantomsection\label{\detokenize{reference:pypath.core.network.Network.count_complexes_by_interaction_type_and_data_model}}\pysiglinewithargsret{\sphinxbfcode{\sphinxupquote{count\_complexes\_by\_interaction\_type\_and\_data\_model}}}{}{}
str(object=’‘) -\textgreater{} str
str(bytes\_or\_buffer{[}, encoding{[}, errors{]}{]}) -\textgreater{} str

Create a new string object from the given object. If encoding or
errors is specified, then the object must expose a data buffer
that will be decoded using the given encoding and error handler.
Otherwise, returns the result of object.\_\_str\_\_() (if defined)
or repr(object).
encoding defaults to sys.getdefaultencoding().
errors defaults to ‘strict’.

\end{fulllineitems}

\index{count\_complexes\_by\_interaction\_type\_and\_data\_model\_and\_resource() (pypath.core.network.Network method)@\spxentry{count\_complexes\_by\_interaction\_type\_and\_data\_model\_and\_resource()}\spxextra{pypath.core.network.Network method}}

\begin{fulllineitems}
\phantomsection\label{\detokenize{reference:pypath.core.network.Network.count_complexes_by_interaction_type_and_data_model_and_resource}}\pysiglinewithargsret{\sphinxbfcode{\sphinxupquote{count\_complexes\_by\_interaction\_type\_and\_data\_model\_and\_resource}}}{}{}
str(object=’‘) -\textgreater{} str
str(bytes\_or\_buffer{[}, encoding{[}, errors{]}{]}) -\textgreater{} str

Create a new string object from the given object. If encoding or
errors is specified, then the object must expose a data buffer
that will be decoded using the given encoding and error handler.
Otherwise, returns the result of object.\_\_str\_\_() (if defined)
or repr(object).
encoding defaults to sys.getdefaultencoding().
errors defaults to ‘strict’.

\end{fulllineitems}

\index{count\_complexes\_by\_reference() (pypath.core.network.Network method)@\spxentry{count\_complexes\_by\_reference()}\spxextra{pypath.core.network.Network method}}

\begin{fulllineitems}
\phantomsection\label{\detokenize{reference:pypath.core.network.Network.count_complexes_by_reference}}\pysiglinewithargsret{\sphinxbfcode{\sphinxupquote{count\_complexes\_by\_reference}}}{}{}
str(object=’‘) -\textgreater{} str
str(bytes\_or\_buffer{[}, encoding{[}, errors{]}{]}) -\textgreater{} str

Create a new string object from the given object. If encoding or
errors is specified, then the object must expose a data buffer
that will be decoded using the given encoding and error handler.
Otherwise, returns the result of object.\_\_str\_\_() (if defined)
or repr(object).
encoding defaults to sys.getdefaultencoding().
errors defaults to ‘strict’.

\end{fulllineitems}

\index{count\_complexes\_by\_resource() (pypath.core.network.Network method)@\spxentry{count\_complexes\_by\_resource()}\spxextra{pypath.core.network.Network method}}

\begin{fulllineitems}
\phantomsection\label{\detokenize{reference:pypath.core.network.Network.count_complexes_by_resource}}\pysiglinewithargsret{\sphinxbfcode{\sphinxupquote{count\_complexes\_by\_resource}}}{}{}
str(object=’‘) -\textgreater{} str
str(bytes\_or\_buffer{[}, encoding{[}, errors{]}{]}) -\textgreater{} str

Create a new string object from the given object. If encoding or
errors is specified, then the object must expose a data buffer
that will be decoded using the given encoding and error handler.
Otherwise, returns the result of object.\_\_str\_\_() (if defined)
or repr(object).
encoding defaults to sys.getdefaultencoding().
errors defaults to ‘strict’.

\end{fulllineitems}

\index{count\_curation\_effort() (pypath.core.network.Network method)@\spxentry{count\_curation\_effort()}\spxextra{pypath.core.network.Network method}}

\begin{fulllineitems}
\phantomsection\label{\detokenize{reference:pypath.core.network.Network.count_curation_effort}}\pysiglinewithargsret{\sphinxbfcode{\sphinxupquote{count\_curation\_effort}}}{}{}
str(object=’‘) -\textgreater{} str
str(bytes\_or\_buffer{[}, encoding{[}, errors{]}{]}) -\textgreater{} str

Create a new string object from the given object. If encoding or
errors is specified, then the object must expose a data buffer
that will be decoded using the given encoding and error handler.
Otherwise, returns the result of object.\_\_str\_\_() (if defined)
or repr(object).
encoding defaults to sys.getdefaultencoding().
errors defaults to ‘strict’.

\end{fulllineitems}

\index{count\_curation\_effort\_by\_data\_model() (pypath.core.network.Network method)@\spxentry{count\_curation\_effort\_by\_data\_model()}\spxextra{pypath.core.network.Network method}}

\begin{fulllineitems}
\phantomsection\label{\detokenize{reference:pypath.core.network.Network.count_curation_effort_by_data_model}}\pysiglinewithargsret{\sphinxbfcode{\sphinxupquote{count\_curation\_effort\_by\_data\_model}}}{}{}
str(object=’‘) -\textgreater{} str
str(bytes\_or\_buffer{[}, encoding{[}, errors{]}{]}) -\textgreater{} str

Create a new string object from the given object. If encoding or
errors is specified, then the object must expose a data buffer
that will be decoded using the given encoding and error handler.
Otherwise, returns the result of object.\_\_str\_\_() (if defined)
or repr(object).
encoding defaults to sys.getdefaultencoding().
errors defaults to ‘strict’.

\end{fulllineitems}

\index{count\_curation\_effort\_by\_interaction\_type() (pypath.core.network.Network method)@\spxentry{count\_curation\_effort\_by\_interaction\_type()}\spxextra{pypath.core.network.Network method}}

\begin{fulllineitems}
\phantomsection\label{\detokenize{reference:pypath.core.network.Network.count_curation_effort_by_interaction_type}}\pysiglinewithargsret{\sphinxbfcode{\sphinxupquote{count\_curation\_effort\_by\_interaction\_type}}}{}{}
str(object=’‘) -\textgreater{} str
str(bytes\_or\_buffer{[}, encoding{[}, errors{]}{]}) -\textgreater{} str

Create a new string object from the given object. If encoding or
errors is specified, then the object must expose a data buffer
that will be decoded using the given encoding and error handler.
Otherwise, returns the result of object.\_\_str\_\_() (if defined)
or repr(object).
encoding defaults to sys.getdefaultencoding().
errors defaults to ‘strict’.

\end{fulllineitems}

\index{count\_curation\_effort\_by\_interaction\_type\_and\_data\_model() (pypath.core.network.Network method)@\spxentry{count\_curation\_effort\_by\_interaction\_type\_and\_data\_model()}\spxextra{pypath.core.network.Network method}}

\begin{fulllineitems}
\phantomsection\label{\detokenize{reference:pypath.core.network.Network.count_curation_effort_by_interaction_type_and_data_model}}\pysiglinewithargsret{\sphinxbfcode{\sphinxupquote{count\_curation\_effort\_by\_interaction\_type\_and\_data\_model}}}{}{}
str(object=’‘) -\textgreater{} str
str(bytes\_or\_buffer{[}, encoding{[}, errors{]}{]}) -\textgreater{} str

Create a new string object from the given object. If encoding or
errors is specified, then the object must expose a data buffer
that will be decoded using the given encoding and error handler.
Otherwise, returns the result of object.\_\_str\_\_() (if defined)
or repr(object).
encoding defaults to sys.getdefaultencoding().
errors defaults to ‘strict’.

\end{fulllineitems}

\index{count\_curation\_effort\_by\_interaction\_type\_and\_data\_model\_and\_resource() (pypath.core.network.Network method)@\spxentry{count\_curation\_effort\_by\_interaction\_type\_and\_data\_model\_and\_resource()}\spxextra{pypath.core.network.Network method}}

\begin{fulllineitems}
\phantomsection\label{\detokenize{reference:pypath.core.network.Network.count_curation_effort_by_interaction_type_and_data_model_and_resource}}\pysiglinewithargsret{\sphinxbfcode{\sphinxupquote{count\_curation\_effort\_by\_interaction\_type\_and\_data\_model\_and\_resource}}}{}{}
str(object=’‘) -\textgreater{} str
str(bytes\_or\_buffer{[}, encoding{[}, errors{]}{]}) -\textgreater{} str

Create a new string object from the given object. If encoding or
errors is specified, then the object must expose a data buffer
that will be decoded using the given encoding and error handler.
Otherwise, returns the result of object.\_\_str\_\_() (if defined)
or repr(object).
encoding defaults to sys.getdefaultencoding().
errors defaults to ‘strict’.

\end{fulllineitems}

\index{count\_curation\_effort\_by\_reference() (pypath.core.network.Network method)@\spxentry{count\_curation\_effort\_by\_reference()}\spxextra{pypath.core.network.Network method}}

\begin{fulllineitems}
\phantomsection\label{\detokenize{reference:pypath.core.network.Network.count_curation_effort_by_reference}}\pysiglinewithargsret{\sphinxbfcode{\sphinxupquote{count\_curation\_effort\_by\_reference}}}{}{}
str(object=’‘) -\textgreater{} str
str(bytes\_or\_buffer{[}, encoding{[}, errors{]}{]}) -\textgreater{} str

Create a new string object from the given object. If encoding or
errors is specified, then the object must expose a data buffer
that will be decoded using the given encoding and error handler.
Otherwise, returns the result of object.\_\_str\_\_() (if defined)
or repr(object).
encoding defaults to sys.getdefaultencoding().
errors defaults to ‘strict’.

\end{fulllineitems}

\index{count\_curation\_effort\_by\_resource() (pypath.core.network.Network method)@\spxentry{count\_curation\_effort\_by\_resource()}\spxextra{pypath.core.network.Network method}}

\begin{fulllineitems}
\phantomsection\label{\detokenize{reference:pypath.core.network.Network.count_curation_effort_by_resource}}\pysiglinewithargsret{\sphinxbfcode{\sphinxupquote{count\_curation\_effort\_by\_resource}}}{}{}
str(object=’‘) -\textgreater{} str
str(bytes\_or\_buffer{[}, encoding{[}, errors{]}{]}) -\textgreater{} str

Create a new string object from the given object. If encoding or
errors is specified, then the object must expose a data buffer
that will be decoded using the given encoding and error handler.
Otherwise, returns the result of object.\_\_str\_\_() (if defined)
or repr(object).
encoding defaults to sys.getdefaultencoding().
errors defaults to ‘strict’.

\end{fulllineitems}

\index{count\_data\_models() (pypath.core.network.Network method)@\spxentry{count\_data\_models()}\spxextra{pypath.core.network.Network method}}

\begin{fulllineitems}
\phantomsection\label{\detokenize{reference:pypath.core.network.Network.count_data_models}}\pysiglinewithargsret{\sphinxbfcode{\sphinxupquote{count\_data\_models}}}{}{}
str(object=’‘) -\textgreater{} str
str(bytes\_or\_buffer{[}, encoding{[}, errors{]}{]}) -\textgreater{} str

Create a new string object from the given object. If encoding or
errors is specified, then the object must expose a data buffer
that will be decoded using the given encoding and error handler.
Otherwise, returns the result of object.\_\_str\_\_() (if defined)
or repr(object).
encoding defaults to sys.getdefaultencoding().
errors defaults to ‘strict’.

\end{fulllineitems}

\index{count\_data\_models\_by\_data\_model() (pypath.core.network.Network method)@\spxentry{count\_data\_models\_by\_data\_model()}\spxextra{pypath.core.network.Network method}}

\begin{fulllineitems}
\phantomsection\label{\detokenize{reference:pypath.core.network.Network.count_data_models_by_data_model}}\pysiglinewithargsret{\sphinxbfcode{\sphinxupquote{count\_data\_models\_by\_data\_model}}}{}{}
str(object=’‘) -\textgreater{} str
str(bytes\_or\_buffer{[}, encoding{[}, errors{]}{]}) -\textgreater{} str

Create a new string object from the given object. If encoding or
errors is specified, then the object must expose a data buffer
that will be decoded using the given encoding and error handler.
Otherwise, returns the result of object.\_\_str\_\_() (if defined)
or repr(object).
encoding defaults to sys.getdefaultencoding().
errors defaults to ‘strict’.

\end{fulllineitems}

\index{count\_data\_models\_by\_interaction\_type() (pypath.core.network.Network method)@\spxentry{count\_data\_models\_by\_interaction\_type()}\spxextra{pypath.core.network.Network method}}

\begin{fulllineitems}
\phantomsection\label{\detokenize{reference:pypath.core.network.Network.count_data_models_by_interaction_type}}\pysiglinewithargsret{\sphinxbfcode{\sphinxupquote{count\_data\_models\_by\_interaction\_type}}}{}{}
str(object=’‘) -\textgreater{} str
str(bytes\_or\_buffer{[}, encoding{[}, errors{]}{]}) -\textgreater{} str

Create a new string object from the given object. If encoding or
errors is specified, then the object must expose a data buffer
that will be decoded using the given encoding and error handler.
Otherwise, returns the result of object.\_\_str\_\_() (if defined)
or repr(object).
encoding defaults to sys.getdefaultencoding().
errors defaults to ‘strict’.

\end{fulllineitems}

\index{count\_data\_models\_by\_interaction\_type\_and\_data\_model() (pypath.core.network.Network method)@\spxentry{count\_data\_models\_by\_interaction\_type\_and\_data\_model()}\spxextra{pypath.core.network.Network method}}

\begin{fulllineitems}
\phantomsection\label{\detokenize{reference:pypath.core.network.Network.count_data_models_by_interaction_type_and_data_model}}\pysiglinewithargsret{\sphinxbfcode{\sphinxupquote{count\_data\_models\_by\_interaction\_type\_and\_data\_model}}}{}{}
str(object=’‘) -\textgreater{} str
str(bytes\_or\_buffer{[}, encoding{[}, errors{]}{]}) -\textgreater{} str

Create a new string object from the given object. If encoding or
errors is specified, then the object must expose a data buffer
that will be decoded using the given encoding and error handler.
Otherwise, returns the result of object.\_\_str\_\_() (if defined)
or repr(object).
encoding defaults to sys.getdefaultencoding().
errors defaults to ‘strict’.

\end{fulllineitems}

\index{count\_data\_models\_by\_interaction\_type\_and\_data\_model\_and\_resource() (pypath.core.network.Network method)@\spxentry{count\_data\_models\_by\_interaction\_type\_and\_data\_model\_and\_resource()}\spxextra{pypath.core.network.Network method}}

\begin{fulllineitems}
\phantomsection\label{\detokenize{reference:pypath.core.network.Network.count_data_models_by_interaction_type_and_data_model_and_resource}}\pysiglinewithargsret{\sphinxbfcode{\sphinxupquote{count\_data\_models\_by\_interaction\_type\_and\_data\_model\_and\_resource}}}{}{}
str(object=’‘) -\textgreater{} str
str(bytes\_or\_buffer{[}, encoding{[}, errors{]}{]}) -\textgreater{} str

Create a new string object from the given object. If encoding or
errors is specified, then the object must expose a data buffer
that will be decoded using the given encoding and error handler.
Otherwise, returns the result of object.\_\_str\_\_() (if defined)
or repr(object).
encoding defaults to sys.getdefaultencoding().
errors defaults to ‘strict’.

\end{fulllineitems}

\index{count\_data\_models\_by\_reference() (pypath.core.network.Network method)@\spxentry{count\_data\_models\_by\_reference()}\spxextra{pypath.core.network.Network method}}

\begin{fulllineitems}
\phantomsection\label{\detokenize{reference:pypath.core.network.Network.count_data_models_by_reference}}\pysiglinewithargsret{\sphinxbfcode{\sphinxupquote{count\_data\_models\_by\_reference}}}{}{}
str(object=’‘) -\textgreater{} str
str(bytes\_or\_buffer{[}, encoding{[}, errors{]}{]}) -\textgreater{} str

Create a new string object from the given object. If encoding or
errors is specified, then the object must expose a data buffer
that will be decoded using the given encoding and error handler.
Otherwise, returns the result of object.\_\_str\_\_() (if defined)
or repr(object).
encoding defaults to sys.getdefaultencoding().
errors defaults to ‘strict’.

\end{fulllineitems}

\index{count\_data\_models\_by\_resource() (pypath.core.network.Network method)@\spxentry{count\_data\_models\_by\_resource()}\spxextra{pypath.core.network.Network method}}

\begin{fulllineitems}
\phantomsection\label{\detokenize{reference:pypath.core.network.Network.count_data_models_by_resource}}\pysiglinewithargsret{\sphinxbfcode{\sphinxupquote{count\_data\_models\_by\_resource}}}{}{}
str(object=’‘) -\textgreater{} str
str(bytes\_or\_buffer{[}, encoding{[}, errors{]}{]}) -\textgreater{} str

Create a new string object from the given object. If encoding or
errors is specified, then the object must expose a data buffer
that will be decoded using the given encoding and error handler.
Otherwise, returns the result of object.\_\_str\_\_() (if defined)
or repr(object).
encoding defaults to sys.getdefaultencoding().
errors defaults to ‘strict’.

\end{fulllineitems}

\index{count\_degrees\_directed() (pypath.core.network.Network method)@\spxentry{count\_degrees\_directed()}\spxextra{pypath.core.network.Network method}}

\begin{fulllineitems}
\phantomsection\label{\detokenize{reference:pypath.core.network.Network.count_degrees_directed}}\pysiglinewithargsret{\sphinxbfcode{\sphinxupquote{count\_degrees\_directed}}}{}{}
str(object=’‘) -\textgreater{} str
str(bytes\_or\_buffer{[}, encoding{[}, errors{]}{]}) -\textgreater{} str

Create a new string object from the given object. If encoding or
errors is specified, then the object must expose a data buffer
that will be decoded using the given encoding and error handler.
Otherwise, returns the result of object.\_\_str\_\_() (if defined)
or repr(object).
encoding defaults to sys.getdefaultencoding().
errors defaults to ‘strict’.

\end{fulllineitems}

\index{count\_degrees\_directed\_by\_data\_model() (pypath.core.network.Network method)@\spxentry{count\_degrees\_directed\_by\_data\_model()}\spxextra{pypath.core.network.Network method}}

\begin{fulllineitems}
\phantomsection\label{\detokenize{reference:pypath.core.network.Network.count_degrees_directed_by_data_model}}\pysiglinewithargsret{\sphinxbfcode{\sphinxupquote{count\_degrees\_directed\_by\_data\_model}}}{}{}
str(object=’‘) -\textgreater{} str
str(bytes\_or\_buffer{[}, encoding{[}, errors{]}{]}) -\textgreater{} str

Create a new string object from the given object. If encoding or
errors is specified, then the object must expose a data buffer
that will be decoded using the given encoding and error handler.
Otherwise, returns the result of object.\_\_str\_\_() (if defined)
or repr(object).
encoding defaults to sys.getdefaultencoding().
errors defaults to ‘strict’.

\end{fulllineitems}

\index{count\_degrees\_directed\_by\_interaction\_type() (pypath.core.network.Network method)@\spxentry{count\_degrees\_directed\_by\_interaction\_type()}\spxextra{pypath.core.network.Network method}}

\begin{fulllineitems}
\phantomsection\label{\detokenize{reference:pypath.core.network.Network.count_degrees_directed_by_interaction_type}}\pysiglinewithargsret{\sphinxbfcode{\sphinxupquote{count\_degrees\_directed\_by\_interaction\_type}}}{}{}
str(object=’‘) -\textgreater{} str
str(bytes\_or\_buffer{[}, encoding{[}, errors{]}{]}) -\textgreater{} str

Create a new string object from the given object. If encoding or
errors is specified, then the object must expose a data buffer
that will be decoded using the given encoding and error handler.
Otherwise, returns the result of object.\_\_str\_\_() (if defined)
or repr(object).
encoding defaults to sys.getdefaultencoding().
errors defaults to ‘strict’.

\end{fulllineitems}

\index{count\_degrees\_directed\_by\_interaction\_type\_and\_data\_model() (pypath.core.network.Network method)@\spxentry{count\_degrees\_directed\_by\_interaction\_type\_and\_data\_model()}\spxextra{pypath.core.network.Network method}}

\begin{fulllineitems}
\phantomsection\label{\detokenize{reference:pypath.core.network.Network.count_degrees_directed_by_interaction_type_and_data_model}}\pysiglinewithargsret{\sphinxbfcode{\sphinxupquote{count\_degrees\_directed\_by\_interaction\_type\_and\_data\_model}}}{}{}
str(object=’‘) -\textgreater{} str
str(bytes\_or\_buffer{[}, encoding{[}, errors{]}{]}) -\textgreater{} str

Create a new string object from the given object. If encoding or
errors is specified, then the object must expose a data buffer
that will be decoded using the given encoding and error handler.
Otherwise, returns the result of object.\_\_str\_\_() (if defined)
or repr(object).
encoding defaults to sys.getdefaultencoding().
errors defaults to ‘strict’.

\end{fulllineitems}

\index{count\_degrees\_directed\_by\_interaction\_type\_and\_data\_model\_and\_resource() (pypath.core.network.Network method)@\spxentry{count\_degrees\_directed\_by\_interaction\_type\_and\_data\_model\_and\_resource()}\spxextra{pypath.core.network.Network method}}

\begin{fulllineitems}
\phantomsection\label{\detokenize{reference:pypath.core.network.Network.count_degrees_directed_by_interaction_type_and_data_model_and_resource}}\pysiglinewithargsret{\sphinxbfcode{\sphinxupquote{count\_degrees\_directed\_by\_interaction\_type\_and\_data\_model\_and\_resource}}}{}{}
str(object=’‘) -\textgreater{} str
str(bytes\_or\_buffer{[}, encoding{[}, errors{]}{]}) -\textgreater{} str

Create a new string object from the given object. If encoding or
errors is specified, then the object must expose a data buffer
that will be decoded using the given encoding and error handler.
Otherwise, returns the result of object.\_\_str\_\_() (if defined)
or repr(object).
encoding defaults to sys.getdefaultencoding().
errors defaults to ‘strict’.

\end{fulllineitems}

\index{count\_degrees\_directed\_by\_reference() (pypath.core.network.Network method)@\spxentry{count\_degrees\_directed\_by\_reference()}\spxextra{pypath.core.network.Network method}}

\begin{fulllineitems}
\phantomsection\label{\detokenize{reference:pypath.core.network.Network.count_degrees_directed_by_reference}}\pysiglinewithargsret{\sphinxbfcode{\sphinxupquote{count\_degrees\_directed\_by\_reference}}}{}{}
str(object=’‘) -\textgreater{} str
str(bytes\_or\_buffer{[}, encoding{[}, errors{]}{]}) -\textgreater{} str

Create a new string object from the given object. If encoding or
errors is specified, then the object must expose a data buffer
that will be decoded using the given encoding and error handler.
Otherwise, returns the result of object.\_\_str\_\_() (if defined)
or repr(object).
encoding defaults to sys.getdefaultencoding().
errors defaults to ‘strict’.

\end{fulllineitems}

\index{count\_degrees\_directed\_by\_resource() (pypath.core.network.Network method)@\spxentry{count\_degrees\_directed\_by\_resource()}\spxextra{pypath.core.network.Network method}}

\begin{fulllineitems}
\phantomsection\label{\detokenize{reference:pypath.core.network.Network.count_degrees_directed_by_resource}}\pysiglinewithargsret{\sphinxbfcode{\sphinxupquote{count\_degrees\_directed\_by\_resource}}}{}{}
str(object=’‘) -\textgreater{} str
str(bytes\_or\_buffer{[}, encoding{[}, errors{]}{]}) -\textgreater{} str

Create a new string object from the given object. If encoding or
errors is specified, then the object must expose a data buffer
that will be decoded using the given encoding and error handler.
Otherwise, returns the result of object.\_\_str\_\_() (if defined)
or repr(object).
encoding defaults to sys.getdefaultencoding().
errors defaults to ‘strict’.

\end{fulllineitems}

\index{count\_degrees\_directed\_in() (pypath.core.network.Network method)@\spxentry{count\_degrees\_directed\_in()}\spxextra{pypath.core.network.Network method}}

\begin{fulllineitems}
\phantomsection\label{\detokenize{reference:pypath.core.network.Network.count_degrees_directed_in}}\pysiglinewithargsret{\sphinxbfcode{\sphinxupquote{count\_degrees\_directed\_in}}}{}{}
str(object=’‘) -\textgreater{} str
str(bytes\_or\_buffer{[}, encoding{[}, errors{]}{]}) -\textgreater{} str

Create a new string object from the given object. If encoding or
errors is specified, then the object must expose a data buffer
that will be decoded using the given encoding and error handler.
Otherwise, returns the result of object.\_\_str\_\_() (if defined)
or repr(object).
encoding defaults to sys.getdefaultencoding().
errors defaults to ‘strict’.

\end{fulllineitems}

\index{count\_degrees\_directed\_in\_by\_data\_model() (pypath.core.network.Network method)@\spxentry{count\_degrees\_directed\_in\_by\_data\_model()}\spxextra{pypath.core.network.Network method}}

\begin{fulllineitems}
\phantomsection\label{\detokenize{reference:pypath.core.network.Network.count_degrees_directed_in_by_data_model}}\pysiglinewithargsret{\sphinxbfcode{\sphinxupquote{count\_degrees\_directed\_in\_by\_data\_model}}}{}{}
str(object=’‘) -\textgreater{} str
str(bytes\_or\_buffer{[}, encoding{[}, errors{]}{]}) -\textgreater{} str

Create a new string object from the given object. If encoding or
errors is specified, then the object must expose a data buffer
that will be decoded using the given encoding and error handler.
Otherwise, returns the result of object.\_\_str\_\_() (if defined)
or repr(object).
encoding defaults to sys.getdefaultencoding().
errors defaults to ‘strict’.

\end{fulllineitems}

\index{count\_degrees\_directed\_in\_by\_interaction\_type() (pypath.core.network.Network method)@\spxentry{count\_degrees\_directed\_in\_by\_interaction\_type()}\spxextra{pypath.core.network.Network method}}

\begin{fulllineitems}
\phantomsection\label{\detokenize{reference:pypath.core.network.Network.count_degrees_directed_in_by_interaction_type}}\pysiglinewithargsret{\sphinxbfcode{\sphinxupquote{count\_degrees\_directed\_in\_by\_interaction\_type}}}{}{}
str(object=’‘) -\textgreater{} str
str(bytes\_or\_buffer{[}, encoding{[}, errors{]}{]}) -\textgreater{} str

Create a new string object from the given object. If encoding or
errors is specified, then the object must expose a data buffer
that will be decoded using the given encoding and error handler.
Otherwise, returns the result of object.\_\_str\_\_() (if defined)
or repr(object).
encoding defaults to sys.getdefaultencoding().
errors defaults to ‘strict’.

\end{fulllineitems}

\index{count\_degrees\_directed\_in\_by\_interaction\_type\_and\_data\_model() (pypath.core.network.Network method)@\spxentry{count\_degrees\_directed\_in\_by\_interaction\_type\_and\_data\_model()}\spxextra{pypath.core.network.Network method}}

\begin{fulllineitems}
\phantomsection\label{\detokenize{reference:pypath.core.network.Network.count_degrees_directed_in_by_interaction_type_and_data_model}}\pysiglinewithargsret{\sphinxbfcode{\sphinxupquote{count\_degrees\_directed\_in\_by\_interaction\_type\_and\_data\_model}}}{}{}
str(object=’‘) -\textgreater{} str
str(bytes\_or\_buffer{[}, encoding{[}, errors{]}{]}) -\textgreater{} str

Create a new string object from the given object. If encoding or
errors is specified, then the object must expose a data buffer
that will be decoded using the given encoding and error handler.
Otherwise, returns the result of object.\_\_str\_\_() (if defined)
or repr(object).
encoding defaults to sys.getdefaultencoding().
errors defaults to ‘strict’.

\end{fulllineitems}

\index{count\_degrees\_directed\_in\_by\_interaction\_type\_and\_data\_model\_and\_resource() (pypath.core.network.Network method)@\spxentry{count\_degrees\_directed\_in\_by\_interaction\_type\_and\_data\_model\_and\_resource()}\spxextra{pypath.core.network.Network method}}

\begin{fulllineitems}
\phantomsection\label{\detokenize{reference:pypath.core.network.Network.count_degrees_directed_in_by_interaction_type_and_data_model_and_resource}}\pysiglinewithargsret{\sphinxbfcode{\sphinxupquote{count\_degrees\_directed\_in\_by\_interaction\_type\_and\_data\_model\_and\_resource}}}{}{}
str(object=’‘) -\textgreater{} str
str(bytes\_or\_buffer{[}, encoding{[}, errors{]}{]}) -\textgreater{} str

Create a new string object from the given object. If encoding or
errors is specified, then the object must expose a data buffer
that will be decoded using the given encoding and error handler.
Otherwise, returns the result of object.\_\_str\_\_() (if defined)
or repr(object).
encoding defaults to sys.getdefaultencoding().
errors defaults to ‘strict’.

\end{fulllineitems}

\index{count\_degrees\_directed\_in\_by\_reference() (pypath.core.network.Network method)@\spxentry{count\_degrees\_directed\_in\_by\_reference()}\spxextra{pypath.core.network.Network method}}

\begin{fulllineitems}
\phantomsection\label{\detokenize{reference:pypath.core.network.Network.count_degrees_directed_in_by_reference}}\pysiglinewithargsret{\sphinxbfcode{\sphinxupquote{count\_degrees\_directed\_in\_by\_reference}}}{}{}
str(object=’‘) -\textgreater{} str
str(bytes\_or\_buffer{[}, encoding{[}, errors{]}{]}) -\textgreater{} str

Create a new string object from the given object. If encoding or
errors is specified, then the object must expose a data buffer
that will be decoded using the given encoding and error handler.
Otherwise, returns the result of object.\_\_str\_\_() (if defined)
or repr(object).
encoding defaults to sys.getdefaultencoding().
errors defaults to ‘strict’.

\end{fulllineitems}

\index{count\_degrees\_directed\_in\_by\_resource() (pypath.core.network.Network method)@\spxentry{count\_degrees\_directed\_in\_by\_resource()}\spxextra{pypath.core.network.Network method}}

\begin{fulllineitems}
\phantomsection\label{\detokenize{reference:pypath.core.network.Network.count_degrees_directed_in_by_resource}}\pysiglinewithargsret{\sphinxbfcode{\sphinxupquote{count\_degrees\_directed\_in\_by\_resource}}}{}{}
str(object=’‘) -\textgreater{} str
str(bytes\_or\_buffer{[}, encoding{[}, errors{]}{]}) -\textgreater{} str

Create a new string object from the given object. If encoding or
errors is specified, then the object must expose a data buffer
that will be decoded using the given encoding and error handler.
Otherwise, returns the result of object.\_\_str\_\_() (if defined)
or repr(object).
encoding defaults to sys.getdefaultencoding().
errors defaults to ‘strict’.

\end{fulllineitems}

\index{count\_degrees\_directed\_out() (pypath.core.network.Network method)@\spxentry{count\_degrees\_directed\_out()}\spxextra{pypath.core.network.Network method}}

\begin{fulllineitems}
\phantomsection\label{\detokenize{reference:pypath.core.network.Network.count_degrees_directed_out}}\pysiglinewithargsret{\sphinxbfcode{\sphinxupquote{count\_degrees\_directed\_out}}}{}{}
str(object=’‘) -\textgreater{} str
str(bytes\_or\_buffer{[}, encoding{[}, errors{]}{]}) -\textgreater{} str

Create a new string object from the given object. If encoding or
errors is specified, then the object must expose a data buffer
that will be decoded using the given encoding and error handler.
Otherwise, returns the result of object.\_\_str\_\_() (if defined)
or repr(object).
encoding defaults to sys.getdefaultencoding().
errors defaults to ‘strict’.

\end{fulllineitems}

\index{count\_degrees\_directed\_out\_by\_data\_model() (pypath.core.network.Network method)@\spxentry{count\_degrees\_directed\_out\_by\_data\_model()}\spxextra{pypath.core.network.Network method}}

\begin{fulllineitems}
\phantomsection\label{\detokenize{reference:pypath.core.network.Network.count_degrees_directed_out_by_data_model}}\pysiglinewithargsret{\sphinxbfcode{\sphinxupquote{count\_degrees\_directed\_out\_by\_data\_model}}}{}{}
str(object=’‘) -\textgreater{} str
str(bytes\_or\_buffer{[}, encoding{[}, errors{]}{]}) -\textgreater{} str

Create a new string object from the given object. If encoding or
errors is specified, then the object must expose a data buffer
that will be decoded using the given encoding and error handler.
Otherwise, returns the result of object.\_\_str\_\_() (if defined)
or repr(object).
encoding defaults to sys.getdefaultencoding().
errors defaults to ‘strict’.

\end{fulllineitems}

\index{count\_degrees\_directed\_out\_by\_interaction\_type() (pypath.core.network.Network method)@\spxentry{count\_degrees\_directed\_out\_by\_interaction\_type()}\spxextra{pypath.core.network.Network method}}

\begin{fulllineitems}
\phantomsection\label{\detokenize{reference:pypath.core.network.Network.count_degrees_directed_out_by_interaction_type}}\pysiglinewithargsret{\sphinxbfcode{\sphinxupquote{count\_degrees\_directed\_out\_by\_interaction\_type}}}{}{}
str(object=’‘) -\textgreater{} str
str(bytes\_or\_buffer{[}, encoding{[}, errors{]}{]}) -\textgreater{} str

Create a new string object from the given object. If encoding or
errors is specified, then the object must expose a data buffer
that will be decoded using the given encoding and error handler.
Otherwise, returns the result of object.\_\_str\_\_() (if defined)
or repr(object).
encoding defaults to sys.getdefaultencoding().
errors defaults to ‘strict’.

\end{fulllineitems}

\index{count\_degrees\_directed\_out\_by\_interaction\_type\_and\_data\_model() (pypath.core.network.Network method)@\spxentry{count\_degrees\_directed\_out\_by\_interaction\_type\_and\_data\_model()}\spxextra{pypath.core.network.Network method}}

\begin{fulllineitems}
\phantomsection\label{\detokenize{reference:pypath.core.network.Network.count_degrees_directed_out_by_interaction_type_and_data_model}}\pysiglinewithargsret{\sphinxbfcode{\sphinxupquote{count\_degrees\_directed\_out\_by\_interaction\_type\_and\_data\_model}}}{}{}
str(object=’‘) -\textgreater{} str
str(bytes\_or\_buffer{[}, encoding{[}, errors{]}{]}) -\textgreater{} str

Create a new string object from the given object. If encoding or
errors is specified, then the object must expose a data buffer
that will be decoded using the given encoding and error handler.
Otherwise, returns the result of object.\_\_str\_\_() (if defined)
or repr(object).
encoding defaults to sys.getdefaultencoding().
errors defaults to ‘strict’.

\end{fulllineitems}

\index{count\_degrees\_directed\_out\_by\_interaction\_type\_and\_data\_model\_and\_resource() (pypath.core.network.Network method)@\spxentry{count\_degrees\_directed\_out\_by\_interaction\_type\_and\_data\_model\_and\_resource()}\spxextra{pypath.core.network.Network method}}

\begin{fulllineitems}
\phantomsection\label{\detokenize{reference:pypath.core.network.Network.count_degrees_directed_out_by_interaction_type_and_data_model_and_resource}}\pysiglinewithargsret{\sphinxbfcode{\sphinxupquote{count\_degrees\_directed\_out\_by\_interaction\_type\_and\_data\_model\_and\_resource}}}{}{}
str(object=’‘) -\textgreater{} str
str(bytes\_or\_buffer{[}, encoding{[}, errors{]}{]}) -\textgreater{} str

Create a new string object from the given object. If encoding or
errors is specified, then the object must expose a data buffer
that will be decoded using the given encoding and error handler.
Otherwise, returns the result of object.\_\_str\_\_() (if defined)
or repr(object).
encoding defaults to sys.getdefaultencoding().
errors defaults to ‘strict’.

\end{fulllineitems}

\index{count\_degrees\_directed\_out\_by\_reference() (pypath.core.network.Network method)@\spxentry{count\_degrees\_directed\_out\_by\_reference()}\spxextra{pypath.core.network.Network method}}

\begin{fulllineitems}
\phantomsection\label{\detokenize{reference:pypath.core.network.Network.count_degrees_directed_out_by_reference}}\pysiglinewithargsret{\sphinxbfcode{\sphinxupquote{count\_degrees\_directed\_out\_by\_reference}}}{}{}
str(object=’‘) -\textgreater{} str
str(bytes\_or\_buffer{[}, encoding{[}, errors{]}{]}) -\textgreater{} str

Create a new string object from the given object. If encoding or
errors is specified, then the object must expose a data buffer
that will be decoded using the given encoding and error handler.
Otherwise, returns the result of object.\_\_str\_\_() (if defined)
or repr(object).
encoding defaults to sys.getdefaultencoding().
errors defaults to ‘strict’.

\end{fulllineitems}

\index{count\_degrees\_directed\_out\_by\_resource() (pypath.core.network.Network method)@\spxentry{count\_degrees\_directed\_out\_by\_resource()}\spxextra{pypath.core.network.Network method}}

\begin{fulllineitems}
\phantomsection\label{\detokenize{reference:pypath.core.network.Network.count_degrees_directed_out_by_resource}}\pysiglinewithargsret{\sphinxbfcode{\sphinxupquote{count\_degrees\_directed\_out\_by\_resource}}}{}{}
str(object=’‘) -\textgreater{} str
str(bytes\_or\_buffer{[}, encoding{[}, errors{]}{]}) -\textgreater{} str

Create a new string object from the given object. If encoding or
errors is specified, then the object must expose a data buffer
that will be decoded using the given encoding and error handler.
Otherwise, returns the result of object.\_\_str\_\_() (if defined)
or repr(object).
encoding defaults to sys.getdefaultencoding().
errors defaults to ‘strict’.

\end{fulllineitems}

\index{count\_degrees\_negative() (pypath.core.network.Network method)@\spxentry{count\_degrees\_negative()}\spxextra{pypath.core.network.Network method}}

\begin{fulllineitems}
\phantomsection\label{\detokenize{reference:pypath.core.network.Network.count_degrees_negative}}\pysiglinewithargsret{\sphinxbfcode{\sphinxupquote{count\_degrees\_negative}}}{}{}
str(object=’‘) -\textgreater{} str
str(bytes\_or\_buffer{[}, encoding{[}, errors{]}{]}) -\textgreater{} str

Create a new string object from the given object. If encoding or
errors is specified, then the object must expose a data buffer
that will be decoded using the given encoding and error handler.
Otherwise, returns the result of object.\_\_str\_\_() (if defined)
or repr(object).
encoding defaults to sys.getdefaultencoding().
errors defaults to ‘strict’.

\end{fulllineitems}

\index{count\_degrees\_negative\_by\_data\_model() (pypath.core.network.Network method)@\spxentry{count\_degrees\_negative\_by\_data\_model()}\spxextra{pypath.core.network.Network method}}

\begin{fulllineitems}
\phantomsection\label{\detokenize{reference:pypath.core.network.Network.count_degrees_negative_by_data_model}}\pysiglinewithargsret{\sphinxbfcode{\sphinxupquote{count\_degrees\_negative\_by\_data\_model}}}{}{}
str(object=’‘) -\textgreater{} str
str(bytes\_or\_buffer{[}, encoding{[}, errors{]}{]}) -\textgreater{} str

Create a new string object from the given object. If encoding or
errors is specified, then the object must expose a data buffer
that will be decoded using the given encoding and error handler.
Otherwise, returns the result of object.\_\_str\_\_() (if defined)
or repr(object).
encoding defaults to sys.getdefaultencoding().
errors defaults to ‘strict’.

\end{fulllineitems}

\index{count\_degrees\_negative\_by\_interaction\_type() (pypath.core.network.Network method)@\spxentry{count\_degrees\_negative\_by\_interaction\_type()}\spxextra{pypath.core.network.Network method}}

\begin{fulllineitems}
\phantomsection\label{\detokenize{reference:pypath.core.network.Network.count_degrees_negative_by_interaction_type}}\pysiglinewithargsret{\sphinxbfcode{\sphinxupquote{count\_degrees\_negative\_by\_interaction\_type}}}{}{}
str(object=’‘) -\textgreater{} str
str(bytes\_or\_buffer{[}, encoding{[}, errors{]}{]}) -\textgreater{} str

Create a new string object from the given object. If encoding or
errors is specified, then the object must expose a data buffer
that will be decoded using the given encoding and error handler.
Otherwise, returns the result of object.\_\_str\_\_() (if defined)
or repr(object).
encoding defaults to sys.getdefaultencoding().
errors defaults to ‘strict’.

\end{fulllineitems}

\index{count\_degrees\_negative\_by\_interaction\_type\_and\_data\_model() (pypath.core.network.Network method)@\spxentry{count\_degrees\_negative\_by\_interaction\_type\_and\_data\_model()}\spxextra{pypath.core.network.Network method}}

\begin{fulllineitems}
\phantomsection\label{\detokenize{reference:pypath.core.network.Network.count_degrees_negative_by_interaction_type_and_data_model}}\pysiglinewithargsret{\sphinxbfcode{\sphinxupquote{count\_degrees\_negative\_by\_interaction\_type\_and\_data\_model}}}{}{}
str(object=’‘) -\textgreater{} str
str(bytes\_or\_buffer{[}, encoding{[}, errors{]}{]}) -\textgreater{} str

Create a new string object from the given object. If encoding or
errors is specified, then the object must expose a data buffer
that will be decoded using the given encoding and error handler.
Otherwise, returns the result of object.\_\_str\_\_() (if defined)
or repr(object).
encoding defaults to sys.getdefaultencoding().
errors defaults to ‘strict’.

\end{fulllineitems}

\index{count\_degrees\_negative\_by\_interaction\_type\_and\_data\_model\_and\_resource() (pypath.core.network.Network method)@\spxentry{count\_degrees\_negative\_by\_interaction\_type\_and\_data\_model\_and\_resource()}\spxextra{pypath.core.network.Network method}}

\begin{fulllineitems}
\phantomsection\label{\detokenize{reference:pypath.core.network.Network.count_degrees_negative_by_interaction_type_and_data_model_and_resource}}\pysiglinewithargsret{\sphinxbfcode{\sphinxupquote{count\_degrees\_negative\_by\_interaction\_type\_and\_data\_model\_and\_resource}}}{}{}
str(object=’‘) -\textgreater{} str
str(bytes\_or\_buffer{[}, encoding{[}, errors{]}{]}) -\textgreater{} str

Create a new string object from the given object. If encoding or
errors is specified, then the object must expose a data buffer
that will be decoded using the given encoding and error handler.
Otherwise, returns the result of object.\_\_str\_\_() (if defined)
or repr(object).
encoding defaults to sys.getdefaultencoding().
errors defaults to ‘strict’.

\end{fulllineitems}

\index{count\_degrees\_negative\_by\_reference() (pypath.core.network.Network method)@\spxentry{count\_degrees\_negative\_by\_reference()}\spxextra{pypath.core.network.Network method}}

\begin{fulllineitems}
\phantomsection\label{\detokenize{reference:pypath.core.network.Network.count_degrees_negative_by_reference}}\pysiglinewithargsret{\sphinxbfcode{\sphinxupquote{count\_degrees\_negative\_by\_reference}}}{}{}
str(object=’‘) -\textgreater{} str
str(bytes\_or\_buffer{[}, encoding{[}, errors{]}{]}) -\textgreater{} str

Create a new string object from the given object. If encoding or
errors is specified, then the object must expose a data buffer
that will be decoded using the given encoding and error handler.
Otherwise, returns the result of object.\_\_str\_\_() (if defined)
or repr(object).
encoding defaults to sys.getdefaultencoding().
errors defaults to ‘strict’.

\end{fulllineitems}

\index{count\_degrees\_negative\_by\_resource() (pypath.core.network.Network method)@\spxentry{count\_degrees\_negative\_by\_resource()}\spxextra{pypath.core.network.Network method}}

\begin{fulllineitems}
\phantomsection\label{\detokenize{reference:pypath.core.network.Network.count_degrees_negative_by_resource}}\pysiglinewithargsret{\sphinxbfcode{\sphinxupquote{count\_degrees\_negative\_by\_resource}}}{}{}
str(object=’‘) -\textgreater{} str
str(bytes\_or\_buffer{[}, encoding{[}, errors{]}{]}) -\textgreater{} str

Create a new string object from the given object. If encoding or
errors is specified, then the object must expose a data buffer
that will be decoded using the given encoding and error handler.
Otherwise, returns the result of object.\_\_str\_\_() (if defined)
or repr(object).
encoding defaults to sys.getdefaultencoding().
errors defaults to ‘strict’.

\end{fulllineitems}

\index{count\_degrees\_negative\_in() (pypath.core.network.Network method)@\spxentry{count\_degrees\_negative\_in()}\spxextra{pypath.core.network.Network method}}

\begin{fulllineitems}
\phantomsection\label{\detokenize{reference:pypath.core.network.Network.count_degrees_negative_in}}\pysiglinewithargsret{\sphinxbfcode{\sphinxupquote{count\_degrees\_negative\_in}}}{}{}
str(object=’‘) -\textgreater{} str
str(bytes\_or\_buffer{[}, encoding{[}, errors{]}{]}) -\textgreater{} str

Create a new string object from the given object. If encoding or
errors is specified, then the object must expose a data buffer
that will be decoded using the given encoding and error handler.
Otherwise, returns the result of object.\_\_str\_\_() (if defined)
or repr(object).
encoding defaults to sys.getdefaultencoding().
errors defaults to ‘strict’.

\end{fulllineitems}

\index{count\_degrees\_negative\_in\_by\_data\_model() (pypath.core.network.Network method)@\spxentry{count\_degrees\_negative\_in\_by\_data\_model()}\spxextra{pypath.core.network.Network method}}

\begin{fulllineitems}
\phantomsection\label{\detokenize{reference:pypath.core.network.Network.count_degrees_negative_in_by_data_model}}\pysiglinewithargsret{\sphinxbfcode{\sphinxupquote{count\_degrees\_negative\_in\_by\_data\_model}}}{}{}
str(object=’‘) -\textgreater{} str
str(bytes\_or\_buffer{[}, encoding{[}, errors{]}{]}) -\textgreater{} str

Create a new string object from the given object. If encoding or
errors is specified, then the object must expose a data buffer
that will be decoded using the given encoding and error handler.
Otherwise, returns the result of object.\_\_str\_\_() (if defined)
or repr(object).
encoding defaults to sys.getdefaultencoding().
errors defaults to ‘strict’.

\end{fulllineitems}

\index{count\_degrees\_negative\_in\_by\_interaction\_type() (pypath.core.network.Network method)@\spxentry{count\_degrees\_negative\_in\_by\_interaction\_type()}\spxextra{pypath.core.network.Network method}}

\begin{fulllineitems}
\phantomsection\label{\detokenize{reference:pypath.core.network.Network.count_degrees_negative_in_by_interaction_type}}\pysiglinewithargsret{\sphinxbfcode{\sphinxupquote{count\_degrees\_negative\_in\_by\_interaction\_type}}}{}{}
str(object=’‘) -\textgreater{} str
str(bytes\_or\_buffer{[}, encoding{[}, errors{]}{]}) -\textgreater{} str

Create a new string object from the given object. If encoding or
errors is specified, then the object must expose a data buffer
that will be decoded using the given encoding and error handler.
Otherwise, returns the result of object.\_\_str\_\_() (if defined)
or repr(object).
encoding defaults to sys.getdefaultencoding().
errors defaults to ‘strict’.

\end{fulllineitems}

\index{count\_degrees\_negative\_in\_by\_interaction\_type\_and\_data\_model() (pypath.core.network.Network method)@\spxentry{count\_degrees\_negative\_in\_by\_interaction\_type\_and\_data\_model()}\spxextra{pypath.core.network.Network method}}

\begin{fulllineitems}
\phantomsection\label{\detokenize{reference:pypath.core.network.Network.count_degrees_negative_in_by_interaction_type_and_data_model}}\pysiglinewithargsret{\sphinxbfcode{\sphinxupquote{count\_degrees\_negative\_in\_by\_interaction\_type\_and\_data\_model}}}{}{}
str(object=’‘) -\textgreater{} str
str(bytes\_or\_buffer{[}, encoding{[}, errors{]}{]}) -\textgreater{} str

Create a new string object from the given object. If encoding or
errors is specified, then the object must expose a data buffer
that will be decoded using the given encoding and error handler.
Otherwise, returns the result of object.\_\_str\_\_() (if defined)
or repr(object).
encoding defaults to sys.getdefaultencoding().
errors defaults to ‘strict’.

\end{fulllineitems}

\index{count\_degrees\_negative\_in\_by\_interaction\_type\_and\_data\_model\_and\_resource() (pypath.core.network.Network method)@\spxentry{count\_degrees\_negative\_in\_by\_interaction\_type\_and\_data\_model\_and\_resource()}\spxextra{pypath.core.network.Network method}}

\begin{fulllineitems}
\phantomsection\label{\detokenize{reference:pypath.core.network.Network.count_degrees_negative_in_by_interaction_type_and_data_model_and_resource}}\pysiglinewithargsret{\sphinxbfcode{\sphinxupquote{count\_degrees\_negative\_in\_by\_interaction\_type\_and\_data\_model\_and\_resource}}}{}{}
str(object=’‘) -\textgreater{} str
str(bytes\_or\_buffer{[}, encoding{[}, errors{]}{]}) -\textgreater{} str

Create a new string object from the given object. If encoding or
errors is specified, then the object must expose a data buffer
that will be decoded using the given encoding and error handler.
Otherwise, returns the result of object.\_\_str\_\_() (if defined)
or repr(object).
encoding defaults to sys.getdefaultencoding().
errors defaults to ‘strict’.

\end{fulllineitems}

\index{count\_degrees\_negative\_in\_by\_reference() (pypath.core.network.Network method)@\spxentry{count\_degrees\_negative\_in\_by\_reference()}\spxextra{pypath.core.network.Network method}}

\begin{fulllineitems}
\phantomsection\label{\detokenize{reference:pypath.core.network.Network.count_degrees_negative_in_by_reference}}\pysiglinewithargsret{\sphinxbfcode{\sphinxupquote{count\_degrees\_negative\_in\_by\_reference}}}{}{}
str(object=’‘) -\textgreater{} str
str(bytes\_or\_buffer{[}, encoding{[}, errors{]}{]}) -\textgreater{} str

Create a new string object from the given object. If encoding or
errors is specified, then the object must expose a data buffer
that will be decoded using the given encoding and error handler.
Otherwise, returns the result of object.\_\_str\_\_() (if defined)
or repr(object).
encoding defaults to sys.getdefaultencoding().
errors defaults to ‘strict’.

\end{fulllineitems}

\index{count\_degrees\_negative\_in\_by\_resource() (pypath.core.network.Network method)@\spxentry{count\_degrees\_negative\_in\_by\_resource()}\spxextra{pypath.core.network.Network method}}

\begin{fulllineitems}
\phantomsection\label{\detokenize{reference:pypath.core.network.Network.count_degrees_negative_in_by_resource}}\pysiglinewithargsret{\sphinxbfcode{\sphinxupquote{count\_degrees\_negative\_in\_by\_resource}}}{}{}
str(object=’‘) -\textgreater{} str
str(bytes\_or\_buffer{[}, encoding{[}, errors{]}{]}) -\textgreater{} str

Create a new string object from the given object. If encoding or
errors is specified, then the object must expose a data buffer
that will be decoded using the given encoding and error handler.
Otherwise, returns the result of object.\_\_str\_\_() (if defined)
or repr(object).
encoding defaults to sys.getdefaultencoding().
errors defaults to ‘strict’.

\end{fulllineitems}

\index{count\_degrees\_negative\_out() (pypath.core.network.Network method)@\spxentry{count\_degrees\_negative\_out()}\spxextra{pypath.core.network.Network method}}

\begin{fulllineitems}
\phantomsection\label{\detokenize{reference:pypath.core.network.Network.count_degrees_negative_out}}\pysiglinewithargsret{\sphinxbfcode{\sphinxupquote{count\_degrees\_negative\_out}}}{}{}
str(object=’‘) -\textgreater{} str
str(bytes\_or\_buffer{[}, encoding{[}, errors{]}{]}) -\textgreater{} str

Create a new string object from the given object. If encoding or
errors is specified, then the object must expose a data buffer
that will be decoded using the given encoding and error handler.
Otherwise, returns the result of object.\_\_str\_\_() (if defined)
or repr(object).
encoding defaults to sys.getdefaultencoding().
errors defaults to ‘strict’.

\end{fulllineitems}

\index{count\_degrees\_negative\_out\_by\_data\_model() (pypath.core.network.Network method)@\spxentry{count\_degrees\_negative\_out\_by\_data\_model()}\spxextra{pypath.core.network.Network method}}

\begin{fulllineitems}
\phantomsection\label{\detokenize{reference:pypath.core.network.Network.count_degrees_negative_out_by_data_model}}\pysiglinewithargsret{\sphinxbfcode{\sphinxupquote{count\_degrees\_negative\_out\_by\_data\_model}}}{}{}
str(object=’‘) -\textgreater{} str
str(bytes\_or\_buffer{[}, encoding{[}, errors{]}{]}) -\textgreater{} str

Create a new string object from the given object. If encoding or
errors is specified, then the object must expose a data buffer
that will be decoded using the given encoding and error handler.
Otherwise, returns the result of object.\_\_str\_\_() (if defined)
or repr(object).
encoding defaults to sys.getdefaultencoding().
errors defaults to ‘strict’.

\end{fulllineitems}

\index{count\_degrees\_negative\_out\_by\_interaction\_type() (pypath.core.network.Network method)@\spxentry{count\_degrees\_negative\_out\_by\_interaction\_type()}\spxextra{pypath.core.network.Network method}}

\begin{fulllineitems}
\phantomsection\label{\detokenize{reference:pypath.core.network.Network.count_degrees_negative_out_by_interaction_type}}\pysiglinewithargsret{\sphinxbfcode{\sphinxupquote{count\_degrees\_negative\_out\_by\_interaction\_type}}}{}{}
str(object=’‘) -\textgreater{} str
str(bytes\_or\_buffer{[}, encoding{[}, errors{]}{]}) -\textgreater{} str

Create a new string object from the given object. If encoding or
errors is specified, then the object must expose a data buffer
that will be decoded using the given encoding and error handler.
Otherwise, returns the result of object.\_\_str\_\_() (if defined)
or repr(object).
encoding defaults to sys.getdefaultencoding().
errors defaults to ‘strict’.

\end{fulllineitems}

\index{count\_degrees\_negative\_out\_by\_interaction\_type\_and\_data\_model() (pypath.core.network.Network method)@\spxentry{count\_degrees\_negative\_out\_by\_interaction\_type\_and\_data\_model()}\spxextra{pypath.core.network.Network method}}

\begin{fulllineitems}
\phantomsection\label{\detokenize{reference:pypath.core.network.Network.count_degrees_negative_out_by_interaction_type_and_data_model}}\pysiglinewithargsret{\sphinxbfcode{\sphinxupquote{count\_degrees\_negative\_out\_by\_interaction\_type\_and\_data\_model}}}{}{}
str(object=’‘) -\textgreater{} str
str(bytes\_or\_buffer{[}, encoding{[}, errors{]}{]}) -\textgreater{} str

Create a new string object from the given object. If encoding or
errors is specified, then the object must expose a data buffer
that will be decoded using the given encoding and error handler.
Otherwise, returns the result of object.\_\_str\_\_() (if defined)
or repr(object).
encoding defaults to sys.getdefaultencoding().
errors defaults to ‘strict’.

\end{fulllineitems}

\index{count\_degrees\_negative\_out\_by\_interaction\_type\_and\_data\_model\_and\_resource() (pypath.core.network.Network method)@\spxentry{count\_degrees\_negative\_out\_by\_interaction\_type\_and\_data\_model\_and\_resource()}\spxextra{pypath.core.network.Network method}}

\begin{fulllineitems}
\phantomsection\label{\detokenize{reference:pypath.core.network.Network.count_degrees_negative_out_by_interaction_type_and_data_model_and_resource}}\pysiglinewithargsret{\sphinxbfcode{\sphinxupquote{count\_degrees\_negative\_out\_by\_interaction\_type\_and\_data\_model\_and\_resource}}}{}{}
str(object=’‘) -\textgreater{} str
str(bytes\_or\_buffer{[}, encoding{[}, errors{]}{]}) -\textgreater{} str

Create a new string object from the given object. If encoding or
errors is specified, then the object must expose a data buffer
that will be decoded using the given encoding and error handler.
Otherwise, returns the result of object.\_\_str\_\_() (if defined)
or repr(object).
encoding defaults to sys.getdefaultencoding().
errors defaults to ‘strict’.

\end{fulllineitems}

\index{count\_degrees\_negative\_out\_by\_reference() (pypath.core.network.Network method)@\spxentry{count\_degrees\_negative\_out\_by\_reference()}\spxextra{pypath.core.network.Network method}}

\begin{fulllineitems}
\phantomsection\label{\detokenize{reference:pypath.core.network.Network.count_degrees_negative_out_by_reference}}\pysiglinewithargsret{\sphinxbfcode{\sphinxupquote{count\_degrees\_negative\_out\_by\_reference}}}{}{}
str(object=’‘) -\textgreater{} str
str(bytes\_or\_buffer{[}, encoding{[}, errors{]}{]}) -\textgreater{} str

Create a new string object from the given object. If encoding or
errors is specified, then the object must expose a data buffer
that will be decoded using the given encoding and error handler.
Otherwise, returns the result of object.\_\_str\_\_() (if defined)
or repr(object).
encoding defaults to sys.getdefaultencoding().
errors defaults to ‘strict’.

\end{fulllineitems}

\index{count\_degrees\_negative\_out\_by\_resource() (pypath.core.network.Network method)@\spxentry{count\_degrees\_negative\_out\_by\_resource()}\spxextra{pypath.core.network.Network method}}

\begin{fulllineitems}
\phantomsection\label{\detokenize{reference:pypath.core.network.Network.count_degrees_negative_out_by_resource}}\pysiglinewithargsret{\sphinxbfcode{\sphinxupquote{count\_degrees\_negative\_out\_by\_resource}}}{}{}
str(object=’‘) -\textgreater{} str
str(bytes\_or\_buffer{[}, encoding{[}, errors{]}{]}) -\textgreater{} str

Create a new string object from the given object. If encoding or
errors is specified, then the object must expose a data buffer
that will be decoded using the given encoding and error handler.
Otherwise, returns the result of object.\_\_str\_\_() (if defined)
or repr(object).
encoding defaults to sys.getdefaultencoding().
errors defaults to ‘strict’.

\end{fulllineitems}

\index{count\_degrees\_non\_directed() (pypath.core.network.Network method)@\spxentry{count\_degrees\_non\_directed()}\spxextra{pypath.core.network.Network method}}

\begin{fulllineitems}
\phantomsection\label{\detokenize{reference:pypath.core.network.Network.count_degrees_non_directed}}\pysiglinewithargsret{\sphinxbfcode{\sphinxupquote{count\_degrees\_non\_directed}}}{}{}
str(object=’‘) -\textgreater{} str
str(bytes\_or\_buffer{[}, encoding{[}, errors{]}{]}) -\textgreater{} str

Create a new string object from the given object. If encoding or
errors is specified, then the object must expose a data buffer
that will be decoded using the given encoding and error handler.
Otherwise, returns the result of object.\_\_str\_\_() (if defined)
or repr(object).
encoding defaults to sys.getdefaultencoding().
errors defaults to ‘strict’.

\end{fulllineitems}

\index{count\_degrees\_non\_directed\_by\_data\_model() (pypath.core.network.Network method)@\spxentry{count\_degrees\_non\_directed\_by\_data\_model()}\spxextra{pypath.core.network.Network method}}

\begin{fulllineitems}
\phantomsection\label{\detokenize{reference:pypath.core.network.Network.count_degrees_non_directed_by_data_model}}\pysiglinewithargsret{\sphinxbfcode{\sphinxupquote{count\_degrees\_non\_directed\_by\_data\_model}}}{}{}
str(object=’‘) -\textgreater{} str
str(bytes\_or\_buffer{[}, encoding{[}, errors{]}{]}) -\textgreater{} str

Create a new string object from the given object. If encoding or
errors is specified, then the object must expose a data buffer
that will be decoded using the given encoding and error handler.
Otherwise, returns the result of object.\_\_str\_\_() (if defined)
or repr(object).
encoding defaults to sys.getdefaultencoding().
errors defaults to ‘strict’.

\end{fulllineitems}

\index{count\_degrees\_non\_directed\_by\_interaction\_type() (pypath.core.network.Network method)@\spxentry{count\_degrees\_non\_directed\_by\_interaction\_type()}\spxextra{pypath.core.network.Network method}}

\begin{fulllineitems}
\phantomsection\label{\detokenize{reference:pypath.core.network.Network.count_degrees_non_directed_by_interaction_type}}\pysiglinewithargsret{\sphinxbfcode{\sphinxupquote{count\_degrees\_non\_directed\_by\_interaction\_type}}}{}{}
str(object=’‘) -\textgreater{} str
str(bytes\_or\_buffer{[}, encoding{[}, errors{]}{]}) -\textgreater{} str

Create a new string object from the given object. If encoding or
errors is specified, then the object must expose a data buffer
that will be decoded using the given encoding and error handler.
Otherwise, returns the result of object.\_\_str\_\_() (if defined)
or repr(object).
encoding defaults to sys.getdefaultencoding().
errors defaults to ‘strict’.

\end{fulllineitems}

\index{count\_degrees\_non\_directed\_by\_interaction\_type\_and\_data\_model() (pypath.core.network.Network method)@\spxentry{count\_degrees\_non\_directed\_by\_interaction\_type\_and\_data\_model()}\spxextra{pypath.core.network.Network method}}

\begin{fulllineitems}
\phantomsection\label{\detokenize{reference:pypath.core.network.Network.count_degrees_non_directed_by_interaction_type_and_data_model}}\pysiglinewithargsret{\sphinxbfcode{\sphinxupquote{count\_degrees\_non\_directed\_by\_interaction\_type\_and\_data\_model}}}{}{}
str(object=’‘) -\textgreater{} str
str(bytes\_or\_buffer{[}, encoding{[}, errors{]}{]}) -\textgreater{} str

Create a new string object from the given object. If encoding or
errors is specified, then the object must expose a data buffer
that will be decoded using the given encoding and error handler.
Otherwise, returns the result of object.\_\_str\_\_() (if defined)
or repr(object).
encoding defaults to sys.getdefaultencoding().
errors defaults to ‘strict’.

\end{fulllineitems}

\index{count\_degrees\_non\_directed\_by\_interaction\_type\_and\_data\_model\_and\_resource() (pypath.core.network.Network method)@\spxentry{count\_degrees\_non\_directed\_by\_interaction\_type\_and\_data\_model\_and\_resource()}\spxextra{pypath.core.network.Network method}}

\begin{fulllineitems}
\phantomsection\label{\detokenize{reference:pypath.core.network.Network.count_degrees_non_directed_by_interaction_type_and_data_model_and_resource}}\pysiglinewithargsret{\sphinxbfcode{\sphinxupquote{count\_degrees\_non\_directed\_by\_interaction\_type\_and\_data\_model\_and\_resource}}}{}{}
str(object=’‘) -\textgreater{} str
str(bytes\_or\_buffer{[}, encoding{[}, errors{]}{]}) -\textgreater{} str

Create a new string object from the given object. If encoding or
errors is specified, then the object must expose a data buffer
that will be decoded using the given encoding and error handler.
Otherwise, returns the result of object.\_\_str\_\_() (if defined)
or repr(object).
encoding defaults to sys.getdefaultencoding().
errors defaults to ‘strict’.

\end{fulllineitems}

\index{count\_degrees\_non\_directed\_by\_reference() (pypath.core.network.Network method)@\spxentry{count\_degrees\_non\_directed\_by\_reference()}\spxextra{pypath.core.network.Network method}}

\begin{fulllineitems}
\phantomsection\label{\detokenize{reference:pypath.core.network.Network.count_degrees_non_directed_by_reference}}\pysiglinewithargsret{\sphinxbfcode{\sphinxupquote{count\_degrees\_non\_directed\_by\_reference}}}{}{}
str(object=’‘) -\textgreater{} str
str(bytes\_or\_buffer{[}, encoding{[}, errors{]}{]}) -\textgreater{} str

Create a new string object from the given object. If encoding or
errors is specified, then the object must expose a data buffer
that will be decoded using the given encoding and error handler.
Otherwise, returns the result of object.\_\_str\_\_() (if defined)
or repr(object).
encoding defaults to sys.getdefaultencoding().
errors defaults to ‘strict’.

\end{fulllineitems}

\index{count\_degrees\_non\_directed\_by\_resource() (pypath.core.network.Network method)@\spxentry{count\_degrees\_non\_directed\_by\_resource()}\spxextra{pypath.core.network.Network method}}

\begin{fulllineitems}
\phantomsection\label{\detokenize{reference:pypath.core.network.Network.count_degrees_non_directed_by_resource}}\pysiglinewithargsret{\sphinxbfcode{\sphinxupquote{count\_degrees\_non\_directed\_by\_resource}}}{}{}
str(object=’‘) -\textgreater{} str
str(bytes\_or\_buffer{[}, encoding{[}, errors{]}{]}) -\textgreater{} str

Create a new string object from the given object. If encoding or
errors is specified, then the object must expose a data buffer
that will be decoded using the given encoding and error handler.
Otherwise, returns the result of object.\_\_str\_\_() (if defined)
or repr(object).
encoding defaults to sys.getdefaultencoding().
errors defaults to ‘strict’.

\end{fulllineitems}

\index{count\_degrees\_positive() (pypath.core.network.Network method)@\spxentry{count\_degrees\_positive()}\spxextra{pypath.core.network.Network method}}

\begin{fulllineitems}
\phantomsection\label{\detokenize{reference:pypath.core.network.Network.count_degrees_positive}}\pysiglinewithargsret{\sphinxbfcode{\sphinxupquote{count\_degrees\_positive}}}{}{}
str(object=’‘) -\textgreater{} str
str(bytes\_or\_buffer{[}, encoding{[}, errors{]}{]}) -\textgreater{} str

Create a new string object from the given object. If encoding or
errors is specified, then the object must expose a data buffer
that will be decoded using the given encoding and error handler.
Otherwise, returns the result of object.\_\_str\_\_() (if defined)
or repr(object).
encoding defaults to sys.getdefaultencoding().
errors defaults to ‘strict’.

\end{fulllineitems}

\index{count\_degrees\_positive\_by\_data\_model() (pypath.core.network.Network method)@\spxentry{count\_degrees\_positive\_by\_data\_model()}\spxextra{pypath.core.network.Network method}}

\begin{fulllineitems}
\phantomsection\label{\detokenize{reference:pypath.core.network.Network.count_degrees_positive_by_data_model}}\pysiglinewithargsret{\sphinxbfcode{\sphinxupquote{count\_degrees\_positive\_by\_data\_model}}}{}{}
str(object=’‘) -\textgreater{} str
str(bytes\_or\_buffer{[}, encoding{[}, errors{]}{]}) -\textgreater{} str

Create a new string object from the given object. If encoding or
errors is specified, then the object must expose a data buffer
that will be decoded using the given encoding and error handler.
Otherwise, returns the result of object.\_\_str\_\_() (if defined)
or repr(object).
encoding defaults to sys.getdefaultencoding().
errors defaults to ‘strict’.

\end{fulllineitems}

\index{count\_degrees\_positive\_by\_interaction\_type() (pypath.core.network.Network method)@\spxentry{count\_degrees\_positive\_by\_interaction\_type()}\spxextra{pypath.core.network.Network method}}

\begin{fulllineitems}
\phantomsection\label{\detokenize{reference:pypath.core.network.Network.count_degrees_positive_by_interaction_type}}\pysiglinewithargsret{\sphinxbfcode{\sphinxupquote{count\_degrees\_positive\_by\_interaction\_type}}}{}{}
str(object=’‘) -\textgreater{} str
str(bytes\_or\_buffer{[}, encoding{[}, errors{]}{]}) -\textgreater{} str

Create a new string object from the given object. If encoding or
errors is specified, then the object must expose a data buffer
that will be decoded using the given encoding and error handler.
Otherwise, returns the result of object.\_\_str\_\_() (if defined)
or repr(object).
encoding defaults to sys.getdefaultencoding().
errors defaults to ‘strict’.

\end{fulllineitems}

\index{count\_degrees\_positive\_by\_interaction\_type\_and\_data\_model() (pypath.core.network.Network method)@\spxentry{count\_degrees\_positive\_by\_interaction\_type\_and\_data\_model()}\spxextra{pypath.core.network.Network method}}

\begin{fulllineitems}
\phantomsection\label{\detokenize{reference:pypath.core.network.Network.count_degrees_positive_by_interaction_type_and_data_model}}\pysiglinewithargsret{\sphinxbfcode{\sphinxupquote{count\_degrees\_positive\_by\_interaction\_type\_and\_data\_model}}}{}{}
str(object=’‘) -\textgreater{} str
str(bytes\_or\_buffer{[}, encoding{[}, errors{]}{]}) -\textgreater{} str

Create a new string object from the given object. If encoding or
errors is specified, then the object must expose a data buffer
that will be decoded using the given encoding and error handler.
Otherwise, returns the result of object.\_\_str\_\_() (if defined)
or repr(object).
encoding defaults to sys.getdefaultencoding().
errors defaults to ‘strict’.

\end{fulllineitems}

\index{count\_degrees\_positive\_by\_interaction\_type\_and\_data\_model\_and\_resource() (pypath.core.network.Network method)@\spxentry{count\_degrees\_positive\_by\_interaction\_type\_and\_data\_model\_and\_resource()}\spxextra{pypath.core.network.Network method}}

\begin{fulllineitems}
\phantomsection\label{\detokenize{reference:pypath.core.network.Network.count_degrees_positive_by_interaction_type_and_data_model_and_resource}}\pysiglinewithargsret{\sphinxbfcode{\sphinxupquote{count\_degrees\_positive\_by\_interaction\_type\_and\_data\_model\_and\_resource}}}{}{}
str(object=’‘) -\textgreater{} str
str(bytes\_or\_buffer{[}, encoding{[}, errors{]}{]}) -\textgreater{} str

Create a new string object from the given object. If encoding or
errors is specified, then the object must expose a data buffer
that will be decoded using the given encoding and error handler.
Otherwise, returns the result of object.\_\_str\_\_() (if defined)
or repr(object).
encoding defaults to sys.getdefaultencoding().
errors defaults to ‘strict’.

\end{fulllineitems}

\index{count\_degrees\_positive\_by\_reference() (pypath.core.network.Network method)@\spxentry{count\_degrees\_positive\_by\_reference()}\spxextra{pypath.core.network.Network method}}

\begin{fulllineitems}
\phantomsection\label{\detokenize{reference:pypath.core.network.Network.count_degrees_positive_by_reference}}\pysiglinewithargsret{\sphinxbfcode{\sphinxupquote{count\_degrees\_positive\_by\_reference}}}{}{}
str(object=’‘) -\textgreater{} str
str(bytes\_or\_buffer{[}, encoding{[}, errors{]}{]}) -\textgreater{} str

Create a new string object from the given object. If encoding or
errors is specified, then the object must expose a data buffer
that will be decoded using the given encoding and error handler.
Otherwise, returns the result of object.\_\_str\_\_() (if defined)
or repr(object).
encoding defaults to sys.getdefaultencoding().
errors defaults to ‘strict’.

\end{fulllineitems}

\index{count\_degrees\_positive\_by\_resource() (pypath.core.network.Network method)@\spxentry{count\_degrees\_positive\_by\_resource()}\spxextra{pypath.core.network.Network method}}

\begin{fulllineitems}
\phantomsection\label{\detokenize{reference:pypath.core.network.Network.count_degrees_positive_by_resource}}\pysiglinewithargsret{\sphinxbfcode{\sphinxupquote{count\_degrees\_positive\_by\_resource}}}{}{}
str(object=’‘) -\textgreater{} str
str(bytes\_or\_buffer{[}, encoding{[}, errors{]}{]}) -\textgreater{} str

Create a new string object from the given object. If encoding or
errors is specified, then the object must expose a data buffer
that will be decoded using the given encoding and error handler.
Otherwise, returns the result of object.\_\_str\_\_() (if defined)
or repr(object).
encoding defaults to sys.getdefaultencoding().
errors defaults to ‘strict’.

\end{fulllineitems}

\index{count\_degrees\_positive\_in() (pypath.core.network.Network method)@\spxentry{count\_degrees\_positive\_in()}\spxextra{pypath.core.network.Network method}}

\begin{fulllineitems}
\phantomsection\label{\detokenize{reference:pypath.core.network.Network.count_degrees_positive_in}}\pysiglinewithargsret{\sphinxbfcode{\sphinxupquote{count\_degrees\_positive\_in}}}{}{}
str(object=’‘) -\textgreater{} str
str(bytes\_or\_buffer{[}, encoding{[}, errors{]}{]}) -\textgreater{} str

Create a new string object from the given object. If encoding or
errors is specified, then the object must expose a data buffer
that will be decoded using the given encoding and error handler.
Otherwise, returns the result of object.\_\_str\_\_() (if defined)
or repr(object).
encoding defaults to sys.getdefaultencoding().
errors defaults to ‘strict’.

\end{fulllineitems}

\index{count\_degrees\_positive\_in\_by\_data\_model() (pypath.core.network.Network method)@\spxentry{count\_degrees\_positive\_in\_by\_data\_model()}\spxextra{pypath.core.network.Network method}}

\begin{fulllineitems}
\phantomsection\label{\detokenize{reference:pypath.core.network.Network.count_degrees_positive_in_by_data_model}}\pysiglinewithargsret{\sphinxbfcode{\sphinxupquote{count\_degrees\_positive\_in\_by\_data\_model}}}{}{}
str(object=’‘) -\textgreater{} str
str(bytes\_or\_buffer{[}, encoding{[}, errors{]}{]}) -\textgreater{} str

Create a new string object from the given object. If encoding or
errors is specified, then the object must expose a data buffer
that will be decoded using the given encoding and error handler.
Otherwise, returns the result of object.\_\_str\_\_() (if defined)
or repr(object).
encoding defaults to sys.getdefaultencoding().
errors defaults to ‘strict’.

\end{fulllineitems}

\index{count\_degrees\_positive\_in\_by\_interaction\_type() (pypath.core.network.Network method)@\spxentry{count\_degrees\_positive\_in\_by\_interaction\_type()}\spxextra{pypath.core.network.Network method}}

\begin{fulllineitems}
\phantomsection\label{\detokenize{reference:pypath.core.network.Network.count_degrees_positive_in_by_interaction_type}}\pysiglinewithargsret{\sphinxbfcode{\sphinxupquote{count\_degrees\_positive\_in\_by\_interaction\_type}}}{}{}
str(object=’‘) -\textgreater{} str
str(bytes\_or\_buffer{[}, encoding{[}, errors{]}{]}) -\textgreater{} str

Create a new string object from the given object. If encoding or
errors is specified, then the object must expose a data buffer
that will be decoded using the given encoding and error handler.
Otherwise, returns the result of object.\_\_str\_\_() (if defined)
or repr(object).
encoding defaults to sys.getdefaultencoding().
errors defaults to ‘strict’.

\end{fulllineitems}

\index{count\_degrees\_positive\_in\_by\_interaction\_type\_and\_data\_model() (pypath.core.network.Network method)@\spxentry{count\_degrees\_positive\_in\_by\_interaction\_type\_and\_data\_model()}\spxextra{pypath.core.network.Network method}}

\begin{fulllineitems}
\phantomsection\label{\detokenize{reference:pypath.core.network.Network.count_degrees_positive_in_by_interaction_type_and_data_model}}\pysiglinewithargsret{\sphinxbfcode{\sphinxupquote{count\_degrees\_positive\_in\_by\_interaction\_type\_and\_data\_model}}}{}{}
str(object=’‘) -\textgreater{} str
str(bytes\_or\_buffer{[}, encoding{[}, errors{]}{]}) -\textgreater{} str

Create a new string object from the given object. If encoding or
errors is specified, then the object must expose a data buffer
that will be decoded using the given encoding and error handler.
Otherwise, returns the result of object.\_\_str\_\_() (if defined)
or repr(object).
encoding defaults to sys.getdefaultencoding().
errors defaults to ‘strict’.

\end{fulllineitems}

\index{count\_degrees\_positive\_in\_by\_interaction\_type\_and\_data\_model\_and\_resource() (pypath.core.network.Network method)@\spxentry{count\_degrees\_positive\_in\_by\_interaction\_type\_and\_data\_model\_and\_resource()}\spxextra{pypath.core.network.Network method}}

\begin{fulllineitems}
\phantomsection\label{\detokenize{reference:pypath.core.network.Network.count_degrees_positive_in_by_interaction_type_and_data_model_and_resource}}\pysiglinewithargsret{\sphinxbfcode{\sphinxupquote{count\_degrees\_positive\_in\_by\_interaction\_type\_and\_data\_model\_and\_resource}}}{}{}
str(object=’‘) -\textgreater{} str
str(bytes\_or\_buffer{[}, encoding{[}, errors{]}{]}) -\textgreater{} str

Create a new string object from the given object. If encoding or
errors is specified, then the object must expose a data buffer
that will be decoded using the given encoding and error handler.
Otherwise, returns the result of object.\_\_str\_\_() (if defined)
or repr(object).
encoding defaults to sys.getdefaultencoding().
errors defaults to ‘strict’.

\end{fulllineitems}

\index{count\_degrees\_positive\_in\_by\_reference() (pypath.core.network.Network method)@\spxentry{count\_degrees\_positive\_in\_by\_reference()}\spxextra{pypath.core.network.Network method}}

\begin{fulllineitems}
\phantomsection\label{\detokenize{reference:pypath.core.network.Network.count_degrees_positive_in_by_reference}}\pysiglinewithargsret{\sphinxbfcode{\sphinxupquote{count\_degrees\_positive\_in\_by\_reference}}}{}{}
str(object=’‘) -\textgreater{} str
str(bytes\_or\_buffer{[}, encoding{[}, errors{]}{]}) -\textgreater{} str

Create a new string object from the given object. If encoding or
errors is specified, then the object must expose a data buffer
that will be decoded using the given encoding and error handler.
Otherwise, returns the result of object.\_\_str\_\_() (if defined)
or repr(object).
encoding defaults to sys.getdefaultencoding().
errors defaults to ‘strict’.

\end{fulllineitems}

\index{count\_degrees\_positive\_in\_by\_resource() (pypath.core.network.Network method)@\spxentry{count\_degrees\_positive\_in\_by\_resource()}\spxextra{pypath.core.network.Network method}}

\begin{fulllineitems}
\phantomsection\label{\detokenize{reference:pypath.core.network.Network.count_degrees_positive_in_by_resource}}\pysiglinewithargsret{\sphinxbfcode{\sphinxupquote{count\_degrees\_positive\_in\_by\_resource}}}{}{}
str(object=’‘) -\textgreater{} str
str(bytes\_or\_buffer{[}, encoding{[}, errors{]}{]}) -\textgreater{} str

Create a new string object from the given object. If encoding or
errors is specified, then the object must expose a data buffer
that will be decoded using the given encoding and error handler.
Otherwise, returns the result of object.\_\_str\_\_() (if defined)
or repr(object).
encoding defaults to sys.getdefaultencoding().
errors defaults to ‘strict’.

\end{fulllineitems}

\index{count\_degrees\_positive\_out() (pypath.core.network.Network method)@\spxentry{count\_degrees\_positive\_out()}\spxextra{pypath.core.network.Network method}}

\begin{fulllineitems}
\phantomsection\label{\detokenize{reference:pypath.core.network.Network.count_degrees_positive_out}}\pysiglinewithargsret{\sphinxbfcode{\sphinxupquote{count\_degrees\_positive\_out}}}{}{}
str(object=’‘) -\textgreater{} str
str(bytes\_or\_buffer{[}, encoding{[}, errors{]}{]}) -\textgreater{} str

Create a new string object from the given object. If encoding or
errors is specified, then the object must expose a data buffer
that will be decoded using the given encoding and error handler.
Otherwise, returns the result of object.\_\_str\_\_() (if defined)
or repr(object).
encoding defaults to sys.getdefaultencoding().
errors defaults to ‘strict’.

\end{fulllineitems}

\index{count\_degrees\_positive\_out\_by\_data\_model() (pypath.core.network.Network method)@\spxentry{count\_degrees\_positive\_out\_by\_data\_model()}\spxextra{pypath.core.network.Network method}}

\begin{fulllineitems}
\phantomsection\label{\detokenize{reference:pypath.core.network.Network.count_degrees_positive_out_by_data_model}}\pysiglinewithargsret{\sphinxbfcode{\sphinxupquote{count\_degrees\_positive\_out\_by\_data\_model}}}{}{}
str(object=’‘) -\textgreater{} str
str(bytes\_or\_buffer{[}, encoding{[}, errors{]}{]}) -\textgreater{} str

Create a new string object from the given object. If encoding or
errors is specified, then the object must expose a data buffer
that will be decoded using the given encoding and error handler.
Otherwise, returns the result of object.\_\_str\_\_() (if defined)
or repr(object).
encoding defaults to sys.getdefaultencoding().
errors defaults to ‘strict’.

\end{fulllineitems}

\index{count\_degrees\_positive\_out\_by\_interaction\_type() (pypath.core.network.Network method)@\spxentry{count\_degrees\_positive\_out\_by\_interaction\_type()}\spxextra{pypath.core.network.Network method}}

\begin{fulllineitems}
\phantomsection\label{\detokenize{reference:pypath.core.network.Network.count_degrees_positive_out_by_interaction_type}}\pysiglinewithargsret{\sphinxbfcode{\sphinxupquote{count\_degrees\_positive\_out\_by\_interaction\_type}}}{}{}
str(object=’‘) -\textgreater{} str
str(bytes\_or\_buffer{[}, encoding{[}, errors{]}{]}) -\textgreater{} str

Create a new string object from the given object. If encoding or
errors is specified, then the object must expose a data buffer
that will be decoded using the given encoding and error handler.
Otherwise, returns the result of object.\_\_str\_\_() (if defined)
or repr(object).
encoding defaults to sys.getdefaultencoding().
errors defaults to ‘strict’.

\end{fulllineitems}

\index{count\_degrees\_positive\_out\_by\_interaction\_type\_and\_data\_model() (pypath.core.network.Network method)@\spxentry{count\_degrees\_positive\_out\_by\_interaction\_type\_and\_data\_model()}\spxextra{pypath.core.network.Network method}}

\begin{fulllineitems}
\phantomsection\label{\detokenize{reference:pypath.core.network.Network.count_degrees_positive_out_by_interaction_type_and_data_model}}\pysiglinewithargsret{\sphinxbfcode{\sphinxupquote{count\_degrees\_positive\_out\_by\_interaction\_type\_and\_data\_model}}}{}{}
str(object=’‘) -\textgreater{} str
str(bytes\_or\_buffer{[}, encoding{[}, errors{]}{]}) -\textgreater{} str

Create a new string object from the given object. If encoding or
errors is specified, then the object must expose a data buffer
that will be decoded using the given encoding and error handler.
Otherwise, returns the result of object.\_\_str\_\_() (if defined)
or repr(object).
encoding defaults to sys.getdefaultencoding().
errors defaults to ‘strict’.

\end{fulllineitems}

\index{count\_degrees\_positive\_out\_by\_interaction\_type\_and\_data\_model\_and\_resource() (pypath.core.network.Network method)@\spxentry{count\_degrees\_positive\_out\_by\_interaction\_type\_and\_data\_model\_and\_resource()}\spxextra{pypath.core.network.Network method}}

\begin{fulllineitems}
\phantomsection\label{\detokenize{reference:pypath.core.network.Network.count_degrees_positive_out_by_interaction_type_and_data_model_and_resource}}\pysiglinewithargsret{\sphinxbfcode{\sphinxupquote{count\_degrees\_positive\_out\_by\_interaction\_type\_and\_data\_model\_and\_resource}}}{}{}
str(object=’‘) -\textgreater{} str
str(bytes\_or\_buffer{[}, encoding{[}, errors{]}{]}) -\textgreater{} str

Create a new string object from the given object. If encoding or
errors is specified, then the object must expose a data buffer
that will be decoded using the given encoding and error handler.
Otherwise, returns the result of object.\_\_str\_\_() (if defined)
or repr(object).
encoding defaults to sys.getdefaultencoding().
errors defaults to ‘strict’.

\end{fulllineitems}

\index{count\_degrees\_positive\_out\_by\_reference() (pypath.core.network.Network method)@\spxentry{count\_degrees\_positive\_out\_by\_reference()}\spxextra{pypath.core.network.Network method}}

\begin{fulllineitems}
\phantomsection\label{\detokenize{reference:pypath.core.network.Network.count_degrees_positive_out_by_reference}}\pysiglinewithargsret{\sphinxbfcode{\sphinxupquote{count\_degrees\_positive\_out\_by\_reference}}}{}{}
str(object=’‘) -\textgreater{} str
str(bytes\_or\_buffer{[}, encoding{[}, errors{]}{]}) -\textgreater{} str

Create a new string object from the given object. If encoding or
errors is specified, then the object must expose a data buffer
that will be decoded using the given encoding and error handler.
Otherwise, returns the result of object.\_\_str\_\_() (if defined)
or repr(object).
encoding defaults to sys.getdefaultencoding().
errors defaults to ‘strict’.

\end{fulllineitems}

\index{count\_degrees\_positive\_out\_by\_resource() (pypath.core.network.Network method)@\spxentry{count\_degrees\_positive\_out\_by\_resource()}\spxextra{pypath.core.network.Network method}}

\begin{fulllineitems}
\phantomsection\label{\detokenize{reference:pypath.core.network.Network.count_degrees_positive_out_by_resource}}\pysiglinewithargsret{\sphinxbfcode{\sphinxupquote{count\_degrees\_positive\_out\_by\_resource}}}{}{}
str(object=’‘) -\textgreater{} str
str(bytes\_or\_buffer{[}, encoding{[}, errors{]}{]}) -\textgreater{} str

Create a new string object from the given object. If encoding or
errors is specified, then the object must expose a data buffer
that will be decoded using the given encoding and error handler.
Otherwise, returns the result of object.\_\_str\_\_() (if defined)
or repr(object).
encoding defaults to sys.getdefaultencoding().
errors defaults to ‘strict’.

\end{fulllineitems}

\index{count\_degrees\_signed() (pypath.core.network.Network method)@\spxentry{count\_degrees\_signed()}\spxextra{pypath.core.network.Network method}}

\begin{fulllineitems}
\phantomsection\label{\detokenize{reference:pypath.core.network.Network.count_degrees_signed}}\pysiglinewithargsret{\sphinxbfcode{\sphinxupquote{count\_degrees\_signed}}}{}{}
str(object=’‘) -\textgreater{} str
str(bytes\_or\_buffer{[}, encoding{[}, errors{]}{]}) -\textgreater{} str

Create a new string object from the given object. If encoding or
errors is specified, then the object must expose a data buffer
that will be decoded using the given encoding and error handler.
Otherwise, returns the result of object.\_\_str\_\_() (if defined)
or repr(object).
encoding defaults to sys.getdefaultencoding().
errors defaults to ‘strict’.

\end{fulllineitems}

\index{count\_degrees\_signed\_by\_data\_model() (pypath.core.network.Network method)@\spxentry{count\_degrees\_signed\_by\_data\_model()}\spxextra{pypath.core.network.Network method}}

\begin{fulllineitems}
\phantomsection\label{\detokenize{reference:pypath.core.network.Network.count_degrees_signed_by_data_model}}\pysiglinewithargsret{\sphinxbfcode{\sphinxupquote{count\_degrees\_signed\_by\_data\_model}}}{}{}
str(object=’‘) -\textgreater{} str
str(bytes\_or\_buffer{[}, encoding{[}, errors{]}{]}) -\textgreater{} str

Create a new string object from the given object. If encoding or
errors is specified, then the object must expose a data buffer
that will be decoded using the given encoding and error handler.
Otherwise, returns the result of object.\_\_str\_\_() (if defined)
or repr(object).
encoding defaults to sys.getdefaultencoding().
errors defaults to ‘strict’.

\end{fulllineitems}

\index{count\_degrees\_signed\_by\_interaction\_type() (pypath.core.network.Network method)@\spxentry{count\_degrees\_signed\_by\_interaction\_type()}\spxextra{pypath.core.network.Network method}}

\begin{fulllineitems}
\phantomsection\label{\detokenize{reference:pypath.core.network.Network.count_degrees_signed_by_interaction_type}}\pysiglinewithargsret{\sphinxbfcode{\sphinxupquote{count\_degrees\_signed\_by\_interaction\_type}}}{}{}
str(object=’‘) -\textgreater{} str
str(bytes\_or\_buffer{[}, encoding{[}, errors{]}{]}) -\textgreater{} str

Create a new string object from the given object. If encoding or
errors is specified, then the object must expose a data buffer
that will be decoded using the given encoding and error handler.
Otherwise, returns the result of object.\_\_str\_\_() (if defined)
or repr(object).
encoding defaults to sys.getdefaultencoding().
errors defaults to ‘strict’.

\end{fulllineitems}

\index{count\_degrees\_signed\_by\_interaction\_type\_and\_data\_model() (pypath.core.network.Network method)@\spxentry{count\_degrees\_signed\_by\_interaction\_type\_and\_data\_model()}\spxextra{pypath.core.network.Network method}}

\begin{fulllineitems}
\phantomsection\label{\detokenize{reference:pypath.core.network.Network.count_degrees_signed_by_interaction_type_and_data_model}}\pysiglinewithargsret{\sphinxbfcode{\sphinxupquote{count\_degrees\_signed\_by\_interaction\_type\_and\_data\_model}}}{}{}
str(object=’‘) -\textgreater{} str
str(bytes\_or\_buffer{[}, encoding{[}, errors{]}{]}) -\textgreater{} str

Create a new string object from the given object. If encoding or
errors is specified, then the object must expose a data buffer
that will be decoded using the given encoding and error handler.
Otherwise, returns the result of object.\_\_str\_\_() (if defined)
or repr(object).
encoding defaults to sys.getdefaultencoding().
errors defaults to ‘strict’.

\end{fulllineitems}

\index{count\_degrees\_signed\_by\_interaction\_type\_and\_data\_model\_and\_resource() (pypath.core.network.Network method)@\spxentry{count\_degrees\_signed\_by\_interaction\_type\_and\_data\_model\_and\_resource()}\spxextra{pypath.core.network.Network method}}

\begin{fulllineitems}
\phantomsection\label{\detokenize{reference:pypath.core.network.Network.count_degrees_signed_by_interaction_type_and_data_model_and_resource}}\pysiglinewithargsret{\sphinxbfcode{\sphinxupquote{count\_degrees\_signed\_by\_interaction\_type\_and\_data\_model\_and\_resource}}}{}{}
str(object=’‘) -\textgreater{} str
str(bytes\_or\_buffer{[}, encoding{[}, errors{]}{]}) -\textgreater{} str

Create a new string object from the given object. If encoding or
errors is specified, then the object must expose a data buffer
that will be decoded using the given encoding and error handler.
Otherwise, returns the result of object.\_\_str\_\_() (if defined)
or repr(object).
encoding defaults to sys.getdefaultencoding().
errors defaults to ‘strict’.

\end{fulllineitems}

\index{count\_degrees\_signed\_by\_reference() (pypath.core.network.Network method)@\spxentry{count\_degrees\_signed\_by\_reference()}\spxextra{pypath.core.network.Network method}}

\begin{fulllineitems}
\phantomsection\label{\detokenize{reference:pypath.core.network.Network.count_degrees_signed_by_reference}}\pysiglinewithargsret{\sphinxbfcode{\sphinxupquote{count\_degrees\_signed\_by\_reference}}}{}{}
str(object=’‘) -\textgreater{} str
str(bytes\_or\_buffer{[}, encoding{[}, errors{]}{]}) -\textgreater{} str

Create a new string object from the given object. If encoding or
errors is specified, then the object must expose a data buffer
that will be decoded using the given encoding and error handler.
Otherwise, returns the result of object.\_\_str\_\_() (if defined)
or repr(object).
encoding defaults to sys.getdefaultencoding().
errors defaults to ‘strict’.

\end{fulllineitems}

\index{count\_degrees\_signed\_by\_resource() (pypath.core.network.Network method)@\spxentry{count\_degrees\_signed\_by\_resource()}\spxextra{pypath.core.network.Network method}}

\begin{fulllineitems}
\phantomsection\label{\detokenize{reference:pypath.core.network.Network.count_degrees_signed_by_resource}}\pysiglinewithargsret{\sphinxbfcode{\sphinxupquote{count\_degrees\_signed\_by\_resource}}}{}{}
str(object=’‘) -\textgreater{} str
str(bytes\_or\_buffer{[}, encoding{[}, errors{]}{]}) -\textgreater{} str

Create a new string object from the given object. If encoding or
errors is specified, then the object must expose a data buffer
that will be decoded using the given encoding and error handler.
Otherwise, returns the result of object.\_\_str\_\_() (if defined)
or repr(object).
encoding defaults to sys.getdefaultencoding().
errors defaults to ‘strict’.

\end{fulllineitems}

\index{count\_degrees\_signed\_in() (pypath.core.network.Network method)@\spxentry{count\_degrees\_signed\_in()}\spxextra{pypath.core.network.Network method}}

\begin{fulllineitems}
\phantomsection\label{\detokenize{reference:pypath.core.network.Network.count_degrees_signed_in}}\pysiglinewithargsret{\sphinxbfcode{\sphinxupquote{count\_degrees\_signed\_in}}}{}{}
str(object=’‘) -\textgreater{} str
str(bytes\_or\_buffer{[}, encoding{[}, errors{]}{]}) -\textgreater{} str

Create a new string object from the given object. If encoding or
errors is specified, then the object must expose a data buffer
that will be decoded using the given encoding and error handler.
Otherwise, returns the result of object.\_\_str\_\_() (if defined)
or repr(object).
encoding defaults to sys.getdefaultencoding().
errors defaults to ‘strict’.

\end{fulllineitems}

\index{count\_degrees\_signed\_in\_by\_data\_model() (pypath.core.network.Network method)@\spxentry{count\_degrees\_signed\_in\_by\_data\_model()}\spxextra{pypath.core.network.Network method}}

\begin{fulllineitems}
\phantomsection\label{\detokenize{reference:pypath.core.network.Network.count_degrees_signed_in_by_data_model}}\pysiglinewithargsret{\sphinxbfcode{\sphinxupquote{count\_degrees\_signed\_in\_by\_data\_model}}}{}{}
str(object=’‘) -\textgreater{} str
str(bytes\_or\_buffer{[}, encoding{[}, errors{]}{]}) -\textgreater{} str

Create a new string object from the given object. If encoding or
errors is specified, then the object must expose a data buffer
that will be decoded using the given encoding and error handler.
Otherwise, returns the result of object.\_\_str\_\_() (if defined)
or repr(object).
encoding defaults to sys.getdefaultencoding().
errors defaults to ‘strict’.

\end{fulllineitems}

\index{count\_degrees\_signed\_in\_by\_interaction\_type() (pypath.core.network.Network method)@\spxentry{count\_degrees\_signed\_in\_by\_interaction\_type()}\spxextra{pypath.core.network.Network method}}

\begin{fulllineitems}
\phantomsection\label{\detokenize{reference:pypath.core.network.Network.count_degrees_signed_in_by_interaction_type}}\pysiglinewithargsret{\sphinxbfcode{\sphinxupquote{count\_degrees\_signed\_in\_by\_interaction\_type}}}{}{}
str(object=’‘) -\textgreater{} str
str(bytes\_or\_buffer{[}, encoding{[}, errors{]}{]}) -\textgreater{} str

Create a new string object from the given object. If encoding or
errors is specified, then the object must expose a data buffer
that will be decoded using the given encoding and error handler.
Otherwise, returns the result of object.\_\_str\_\_() (if defined)
or repr(object).
encoding defaults to sys.getdefaultencoding().
errors defaults to ‘strict’.

\end{fulllineitems}

\index{count\_degrees\_signed\_in\_by\_interaction\_type\_and\_data\_model() (pypath.core.network.Network method)@\spxentry{count\_degrees\_signed\_in\_by\_interaction\_type\_and\_data\_model()}\spxextra{pypath.core.network.Network method}}

\begin{fulllineitems}
\phantomsection\label{\detokenize{reference:pypath.core.network.Network.count_degrees_signed_in_by_interaction_type_and_data_model}}\pysiglinewithargsret{\sphinxbfcode{\sphinxupquote{count\_degrees\_signed\_in\_by\_interaction\_type\_and\_data\_model}}}{}{}
str(object=’‘) -\textgreater{} str
str(bytes\_or\_buffer{[}, encoding{[}, errors{]}{]}) -\textgreater{} str

Create a new string object from the given object. If encoding or
errors is specified, then the object must expose a data buffer
that will be decoded using the given encoding and error handler.
Otherwise, returns the result of object.\_\_str\_\_() (if defined)
or repr(object).
encoding defaults to sys.getdefaultencoding().
errors defaults to ‘strict’.

\end{fulllineitems}

\index{count\_degrees\_signed\_in\_by\_interaction\_type\_and\_data\_model\_and\_resource() (pypath.core.network.Network method)@\spxentry{count\_degrees\_signed\_in\_by\_interaction\_type\_and\_data\_model\_and\_resource()}\spxextra{pypath.core.network.Network method}}

\begin{fulllineitems}
\phantomsection\label{\detokenize{reference:pypath.core.network.Network.count_degrees_signed_in_by_interaction_type_and_data_model_and_resource}}\pysiglinewithargsret{\sphinxbfcode{\sphinxupquote{count\_degrees\_signed\_in\_by\_interaction\_type\_and\_data\_model\_and\_resource}}}{}{}
str(object=’‘) -\textgreater{} str
str(bytes\_or\_buffer{[}, encoding{[}, errors{]}{]}) -\textgreater{} str

Create a new string object from the given object. If encoding or
errors is specified, then the object must expose a data buffer
that will be decoded using the given encoding and error handler.
Otherwise, returns the result of object.\_\_str\_\_() (if defined)
or repr(object).
encoding defaults to sys.getdefaultencoding().
errors defaults to ‘strict’.

\end{fulllineitems}

\index{count\_degrees\_signed\_in\_by\_reference() (pypath.core.network.Network method)@\spxentry{count\_degrees\_signed\_in\_by\_reference()}\spxextra{pypath.core.network.Network method}}

\begin{fulllineitems}
\phantomsection\label{\detokenize{reference:pypath.core.network.Network.count_degrees_signed_in_by_reference}}\pysiglinewithargsret{\sphinxbfcode{\sphinxupquote{count\_degrees\_signed\_in\_by\_reference}}}{}{}
str(object=’‘) -\textgreater{} str
str(bytes\_or\_buffer{[}, encoding{[}, errors{]}{]}) -\textgreater{} str

Create a new string object from the given object. If encoding or
errors is specified, then the object must expose a data buffer
that will be decoded using the given encoding and error handler.
Otherwise, returns the result of object.\_\_str\_\_() (if defined)
or repr(object).
encoding defaults to sys.getdefaultencoding().
errors defaults to ‘strict’.

\end{fulllineitems}

\index{count\_degrees\_signed\_in\_by\_resource() (pypath.core.network.Network method)@\spxentry{count\_degrees\_signed\_in\_by\_resource()}\spxextra{pypath.core.network.Network method}}

\begin{fulllineitems}
\phantomsection\label{\detokenize{reference:pypath.core.network.Network.count_degrees_signed_in_by_resource}}\pysiglinewithargsret{\sphinxbfcode{\sphinxupquote{count\_degrees\_signed\_in\_by\_resource}}}{}{}
str(object=’‘) -\textgreater{} str
str(bytes\_or\_buffer{[}, encoding{[}, errors{]}{]}) -\textgreater{} str

Create a new string object from the given object. If encoding or
errors is specified, then the object must expose a data buffer
that will be decoded using the given encoding and error handler.
Otherwise, returns the result of object.\_\_str\_\_() (if defined)
or repr(object).
encoding defaults to sys.getdefaultencoding().
errors defaults to ‘strict’.

\end{fulllineitems}

\index{count\_degrees\_signed\_out() (pypath.core.network.Network method)@\spxentry{count\_degrees\_signed\_out()}\spxextra{pypath.core.network.Network method}}

\begin{fulllineitems}
\phantomsection\label{\detokenize{reference:pypath.core.network.Network.count_degrees_signed_out}}\pysiglinewithargsret{\sphinxbfcode{\sphinxupquote{count\_degrees\_signed\_out}}}{}{}
str(object=’‘) -\textgreater{} str
str(bytes\_or\_buffer{[}, encoding{[}, errors{]}{]}) -\textgreater{} str

Create a new string object from the given object. If encoding or
errors is specified, then the object must expose a data buffer
that will be decoded using the given encoding and error handler.
Otherwise, returns the result of object.\_\_str\_\_() (if defined)
or repr(object).
encoding defaults to sys.getdefaultencoding().
errors defaults to ‘strict’.

\end{fulllineitems}

\index{count\_degrees\_signed\_out\_by\_data\_model() (pypath.core.network.Network method)@\spxentry{count\_degrees\_signed\_out\_by\_data\_model()}\spxextra{pypath.core.network.Network method}}

\begin{fulllineitems}
\phantomsection\label{\detokenize{reference:pypath.core.network.Network.count_degrees_signed_out_by_data_model}}\pysiglinewithargsret{\sphinxbfcode{\sphinxupquote{count\_degrees\_signed\_out\_by\_data\_model}}}{}{}
str(object=’‘) -\textgreater{} str
str(bytes\_or\_buffer{[}, encoding{[}, errors{]}{]}) -\textgreater{} str

Create a new string object from the given object. If encoding or
errors is specified, then the object must expose a data buffer
that will be decoded using the given encoding and error handler.
Otherwise, returns the result of object.\_\_str\_\_() (if defined)
or repr(object).
encoding defaults to sys.getdefaultencoding().
errors defaults to ‘strict’.

\end{fulllineitems}

\index{count\_degrees\_signed\_out\_by\_interaction\_type() (pypath.core.network.Network method)@\spxentry{count\_degrees\_signed\_out\_by\_interaction\_type()}\spxextra{pypath.core.network.Network method}}

\begin{fulllineitems}
\phantomsection\label{\detokenize{reference:pypath.core.network.Network.count_degrees_signed_out_by_interaction_type}}\pysiglinewithargsret{\sphinxbfcode{\sphinxupquote{count\_degrees\_signed\_out\_by\_interaction\_type}}}{}{}
str(object=’‘) -\textgreater{} str
str(bytes\_or\_buffer{[}, encoding{[}, errors{]}{]}) -\textgreater{} str

Create a new string object from the given object. If encoding or
errors is specified, then the object must expose a data buffer
that will be decoded using the given encoding and error handler.
Otherwise, returns the result of object.\_\_str\_\_() (if defined)
or repr(object).
encoding defaults to sys.getdefaultencoding().
errors defaults to ‘strict’.

\end{fulllineitems}

\index{count\_degrees\_signed\_out\_by\_interaction\_type\_and\_data\_model() (pypath.core.network.Network method)@\spxentry{count\_degrees\_signed\_out\_by\_interaction\_type\_and\_data\_model()}\spxextra{pypath.core.network.Network method}}

\begin{fulllineitems}
\phantomsection\label{\detokenize{reference:pypath.core.network.Network.count_degrees_signed_out_by_interaction_type_and_data_model}}\pysiglinewithargsret{\sphinxbfcode{\sphinxupquote{count\_degrees\_signed\_out\_by\_interaction\_type\_and\_data\_model}}}{}{}
str(object=’‘) -\textgreater{} str
str(bytes\_or\_buffer{[}, encoding{[}, errors{]}{]}) -\textgreater{} str

Create a new string object from the given object. If encoding or
errors is specified, then the object must expose a data buffer
that will be decoded using the given encoding and error handler.
Otherwise, returns the result of object.\_\_str\_\_() (if defined)
or repr(object).
encoding defaults to sys.getdefaultencoding().
errors defaults to ‘strict’.

\end{fulllineitems}

\index{count\_degrees\_signed\_out\_by\_interaction\_type\_and\_data\_model\_and\_resource() (pypath.core.network.Network method)@\spxentry{count\_degrees\_signed\_out\_by\_interaction\_type\_and\_data\_model\_and\_resource()}\spxextra{pypath.core.network.Network method}}

\begin{fulllineitems}
\phantomsection\label{\detokenize{reference:pypath.core.network.Network.count_degrees_signed_out_by_interaction_type_and_data_model_and_resource}}\pysiglinewithargsret{\sphinxbfcode{\sphinxupquote{count\_degrees\_signed\_out\_by\_interaction\_type\_and\_data\_model\_and\_resource}}}{}{}
str(object=’‘) -\textgreater{} str
str(bytes\_or\_buffer{[}, encoding{[}, errors{]}{]}) -\textgreater{} str

Create a new string object from the given object. If encoding or
errors is specified, then the object must expose a data buffer
that will be decoded using the given encoding and error handler.
Otherwise, returns the result of object.\_\_str\_\_() (if defined)
or repr(object).
encoding defaults to sys.getdefaultencoding().
errors defaults to ‘strict’.

\end{fulllineitems}

\index{count\_degrees\_signed\_out\_by\_reference() (pypath.core.network.Network method)@\spxentry{count\_degrees\_signed\_out\_by\_reference()}\spxextra{pypath.core.network.Network method}}

\begin{fulllineitems}
\phantomsection\label{\detokenize{reference:pypath.core.network.Network.count_degrees_signed_out_by_reference}}\pysiglinewithargsret{\sphinxbfcode{\sphinxupquote{count\_degrees\_signed\_out\_by\_reference}}}{}{}
str(object=’‘) -\textgreater{} str
str(bytes\_or\_buffer{[}, encoding{[}, errors{]}{]}) -\textgreater{} str

Create a new string object from the given object. If encoding or
errors is specified, then the object must expose a data buffer
that will be decoded using the given encoding and error handler.
Otherwise, returns the result of object.\_\_str\_\_() (if defined)
or repr(object).
encoding defaults to sys.getdefaultencoding().
errors defaults to ‘strict’.

\end{fulllineitems}

\index{count\_degrees\_signed\_out\_by\_resource() (pypath.core.network.Network method)@\spxentry{count\_degrees\_signed\_out\_by\_resource()}\spxextra{pypath.core.network.Network method}}

\begin{fulllineitems}
\phantomsection\label{\detokenize{reference:pypath.core.network.Network.count_degrees_signed_out_by_resource}}\pysiglinewithargsret{\sphinxbfcode{\sphinxupquote{count\_degrees\_signed\_out\_by\_resource}}}{}{}
str(object=’‘) -\textgreater{} str
str(bytes\_or\_buffer{[}, encoding{[}, errors{]}{]}) -\textgreater{} str

Create a new string object from the given object. If encoding or
errors is specified, then the object must expose a data buffer
that will be decoded using the given encoding and error handler.
Otherwise, returns the result of object.\_\_str\_\_() (if defined)
or repr(object).
encoding defaults to sys.getdefaultencoding().
errors defaults to ‘strict’.

\end{fulllineitems}

\index{count\_degrees\_undirected() (pypath.core.network.Network method)@\spxentry{count\_degrees\_undirected()}\spxextra{pypath.core.network.Network method}}

\begin{fulllineitems}
\phantomsection\label{\detokenize{reference:pypath.core.network.Network.count_degrees_undirected}}\pysiglinewithargsret{\sphinxbfcode{\sphinxupquote{count\_degrees\_undirected}}}{}{}
str(object=’‘) -\textgreater{} str
str(bytes\_or\_buffer{[}, encoding{[}, errors{]}{]}) -\textgreater{} str

Create a new string object from the given object. If encoding or
errors is specified, then the object must expose a data buffer
that will be decoded using the given encoding and error handler.
Otherwise, returns the result of object.\_\_str\_\_() (if defined)
or repr(object).
encoding defaults to sys.getdefaultencoding().
errors defaults to ‘strict’.

\end{fulllineitems}

\index{count\_degrees\_undirected\_by\_data\_model() (pypath.core.network.Network method)@\spxentry{count\_degrees\_undirected\_by\_data\_model()}\spxextra{pypath.core.network.Network method}}

\begin{fulllineitems}
\phantomsection\label{\detokenize{reference:pypath.core.network.Network.count_degrees_undirected_by_data_model}}\pysiglinewithargsret{\sphinxbfcode{\sphinxupquote{count\_degrees\_undirected\_by\_data\_model}}}{}{}
str(object=’‘) -\textgreater{} str
str(bytes\_or\_buffer{[}, encoding{[}, errors{]}{]}) -\textgreater{} str

Create a new string object from the given object. If encoding or
errors is specified, then the object must expose a data buffer
that will be decoded using the given encoding and error handler.
Otherwise, returns the result of object.\_\_str\_\_() (if defined)
or repr(object).
encoding defaults to sys.getdefaultencoding().
errors defaults to ‘strict’.

\end{fulllineitems}

\index{count\_degrees\_undirected\_by\_interaction\_type() (pypath.core.network.Network method)@\spxentry{count\_degrees\_undirected\_by\_interaction\_type()}\spxextra{pypath.core.network.Network method}}

\begin{fulllineitems}
\phantomsection\label{\detokenize{reference:pypath.core.network.Network.count_degrees_undirected_by_interaction_type}}\pysiglinewithargsret{\sphinxbfcode{\sphinxupquote{count\_degrees\_undirected\_by\_interaction\_type}}}{}{}
str(object=’‘) -\textgreater{} str
str(bytes\_or\_buffer{[}, encoding{[}, errors{]}{]}) -\textgreater{} str

Create a new string object from the given object. If encoding or
errors is specified, then the object must expose a data buffer
that will be decoded using the given encoding and error handler.
Otherwise, returns the result of object.\_\_str\_\_() (if defined)
or repr(object).
encoding defaults to sys.getdefaultencoding().
errors defaults to ‘strict’.

\end{fulllineitems}

\index{count\_degrees\_undirected\_by\_interaction\_type\_and\_data\_model() (pypath.core.network.Network method)@\spxentry{count\_degrees\_undirected\_by\_interaction\_type\_and\_data\_model()}\spxextra{pypath.core.network.Network method}}

\begin{fulllineitems}
\phantomsection\label{\detokenize{reference:pypath.core.network.Network.count_degrees_undirected_by_interaction_type_and_data_model}}\pysiglinewithargsret{\sphinxbfcode{\sphinxupquote{count\_degrees\_undirected\_by\_interaction\_type\_and\_data\_model}}}{}{}
str(object=’‘) -\textgreater{} str
str(bytes\_or\_buffer{[}, encoding{[}, errors{]}{]}) -\textgreater{} str

Create a new string object from the given object. If encoding or
errors is specified, then the object must expose a data buffer
that will be decoded using the given encoding and error handler.
Otherwise, returns the result of object.\_\_str\_\_() (if defined)
or repr(object).
encoding defaults to sys.getdefaultencoding().
errors defaults to ‘strict’.

\end{fulllineitems}

\index{count\_degrees\_undirected\_by\_interaction\_type\_and\_data\_model\_and\_resource() (pypath.core.network.Network method)@\spxentry{count\_degrees\_undirected\_by\_interaction\_type\_and\_data\_model\_and\_resource()}\spxextra{pypath.core.network.Network method}}

\begin{fulllineitems}
\phantomsection\label{\detokenize{reference:pypath.core.network.Network.count_degrees_undirected_by_interaction_type_and_data_model_and_resource}}\pysiglinewithargsret{\sphinxbfcode{\sphinxupquote{count\_degrees\_undirected\_by\_interaction\_type\_and\_data\_model\_and\_resource}}}{}{}
str(object=’‘) -\textgreater{} str
str(bytes\_or\_buffer{[}, encoding{[}, errors{]}{]}) -\textgreater{} str

Create a new string object from the given object. If encoding or
errors is specified, then the object must expose a data buffer
that will be decoded using the given encoding and error handler.
Otherwise, returns the result of object.\_\_str\_\_() (if defined)
or repr(object).
encoding defaults to sys.getdefaultencoding().
errors defaults to ‘strict’.

\end{fulllineitems}

\index{count\_degrees\_undirected\_by\_reference() (pypath.core.network.Network method)@\spxentry{count\_degrees\_undirected\_by\_reference()}\spxextra{pypath.core.network.Network method}}

\begin{fulllineitems}
\phantomsection\label{\detokenize{reference:pypath.core.network.Network.count_degrees_undirected_by_reference}}\pysiglinewithargsret{\sphinxbfcode{\sphinxupquote{count\_degrees\_undirected\_by\_reference}}}{}{}
str(object=’‘) -\textgreater{} str
str(bytes\_or\_buffer{[}, encoding{[}, errors{]}{]}) -\textgreater{} str

Create a new string object from the given object. If encoding or
errors is specified, then the object must expose a data buffer
that will be decoded using the given encoding and error handler.
Otherwise, returns the result of object.\_\_str\_\_() (if defined)
or repr(object).
encoding defaults to sys.getdefaultencoding().
errors defaults to ‘strict’.

\end{fulllineitems}

\index{count\_degrees\_undirected\_by\_resource() (pypath.core.network.Network method)@\spxentry{count\_degrees\_undirected\_by\_resource()}\spxextra{pypath.core.network.Network method}}

\begin{fulllineitems}
\phantomsection\label{\detokenize{reference:pypath.core.network.Network.count_degrees_undirected_by_resource}}\pysiglinewithargsret{\sphinxbfcode{\sphinxupquote{count\_degrees\_undirected\_by\_resource}}}{}{}
str(object=’‘) -\textgreater{} str
str(bytes\_or\_buffer{[}, encoding{[}, errors{]}{]}) -\textgreater{} str

Create a new string object from the given object. If encoding or
errors is specified, then the object must expose a data buffer
that will be decoded using the given encoding and error handler.
Otherwise, returns the result of object.\_\_str\_\_() (if defined)
or repr(object).
encoding defaults to sys.getdefaultencoding().
errors defaults to ‘strict’.

\end{fulllineitems}

\index{count\_entities() (pypath.core.network.Network method)@\spxentry{count\_entities()}\spxextra{pypath.core.network.Network method}}

\begin{fulllineitems}
\phantomsection\label{\detokenize{reference:pypath.core.network.Network.count_entities}}\pysiglinewithargsret{\sphinxbfcode{\sphinxupquote{count\_entities}}}{}{}
str(object=’‘) -\textgreater{} str
str(bytes\_or\_buffer{[}, encoding{[}, errors{]}{]}) -\textgreater{} str

Create a new string object from the given object. If encoding or
errors is specified, then the object must expose a data buffer
that will be decoded using the given encoding and error handler.
Otherwise, returns the result of object.\_\_str\_\_() (if defined)
or repr(object).
encoding defaults to sys.getdefaultencoding().
errors defaults to ‘strict’.

\end{fulllineitems}

\index{count\_entities\_by\_data\_model() (pypath.core.network.Network method)@\spxentry{count\_entities\_by\_data\_model()}\spxextra{pypath.core.network.Network method}}

\begin{fulllineitems}
\phantomsection\label{\detokenize{reference:pypath.core.network.Network.count_entities_by_data_model}}\pysiglinewithargsret{\sphinxbfcode{\sphinxupquote{count\_entities\_by\_data\_model}}}{}{}
str(object=’‘) -\textgreater{} str
str(bytes\_or\_buffer{[}, encoding{[}, errors{]}{]}) -\textgreater{} str

Create a new string object from the given object. If encoding or
errors is specified, then the object must expose a data buffer
that will be decoded using the given encoding and error handler.
Otherwise, returns the result of object.\_\_str\_\_() (if defined)
or repr(object).
encoding defaults to sys.getdefaultencoding().
errors defaults to ‘strict’.

\end{fulllineitems}

\index{count\_entities\_by\_interaction\_type() (pypath.core.network.Network method)@\spxentry{count\_entities\_by\_interaction\_type()}\spxextra{pypath.core.network.Network method}}

\begin{fulllineitems}
\phantomsection\label{\detokenize{reference:pypath.core.network.Network.count_entities_by_interaction_type}}\pysiglinewithargsret{\sphinxbfcode{\sphinxupquote{count\_entities\_by\_interaction\_type}}}{}{}
str(object=’‘) -\textgreater{} str
str(bytes\_or\_buffer{[}, encoding{[}, errors{]}{]}) -\textgreater{} str

Create a new string object from the given object. If encoding or
errors is specified, then the object must expose a data buffer
that will be decoded using the given encoding and error handler.
Otherwise, returns the result of object.\_\_str\_\_() (if defined)
or repr(object).
encoding defaults to sys.getdefaultencoding().
errors defaults to ‘strict’.

\end{fulllineitems}

\index{count\_entities\_by\_interaction\_type\_and\_data\_model() (pypath.core.network.Network method)@\spxentry{count\_entities\_by\_interaction\_type\_and\_data\_model()}\spxextra{pypath.core.network.Network method}}

\begin{fulllineitems}
\phantomsection\label{\detokenize{reference:pypath.core.network.Network.count_entities_by_interaction_type_and_data_model}}\pysiglinewithargsret{\sphinxbfcode{\sphinxupquote{count\_entities\_by\_interaction\_type\_and\_data\_model}}}{}{}
str(object=’‘) -\textgreater{} str
str(bytes\_or\_buffer{[}, encoding{[}, errors{]}{]}) -\textgreater{} str

Create a new string object from the given object. If encoding or
errors is specified, then the object must expose a data buffer
that will be decoded using the given encoding and error handler.
Otherwise, returns the result of object.\_\_str\_\_() (if defined)
or repr(object).
encoding defaults to sys.getdefaultencoding().
errors defaults to ‘strict’.

\end{fulllineitems}

\index{count\_entities\_by\_interaction\_type\_and\_data\_model\_and\_resource() (pypath.core.network.Network method)@\spxentry{count\_entities\_by\_interaction\_type\_and\_data\_model\_and\_resource()}\spxextra{pypath.core.network.Network method}}

\begin{fulllineitems}
\phantomsection\label{\detokenize{reference:pypath.core.network.Network.count_entities_by_interaction_type_and_data_model_and_resource}}\pysiglinewithargsret{\sphinxbfcode{\sphinxupquote{count\_entities\_by\_interaction\_type\_and\_data\_model\_and\_resource}}}{}{}
str(object=’‘) -\textgreater{} str
str(bytes\_or\_buffer{[}, encoding{[}, errors{]}{]}) -\textgreater{} str

Create a new string object from the given object. If encoding or
errors is specified, then the object must expose a data buffer
that will be decoded using the given encoding and error handler.
Otherwise, returns the result of object.\_\_str\_\_() (if defined)
or repr(object).
encoding defaults to sys.getdefaultencoding().
errors defaults to ‘strict’.

\end{fulllineitems}

\index{count\_entities\_by\_reference() (pypath.core.network.Network method)@\spxentry{count\_entities\_by\_reference()}\spxextra{pypath.core.network.Network method}}

\begin{fulllineitems}
\phantomsection\label{\detokenize{reference:pypath.core.network.Network.count_entities_by_reference}}\pysiglinewithargsret{\sphinxbfcode{\sphinxupquote{count\_entities\_by\_reference}}}{}{}
str(object=’‘) -\textgreater{} str
str(bytes\_or\_buffer{[}, encoding{[}, errors{]}{]}) -\textgreater{} str

Create a new string object from the given object. If encoding or
errors is specified, then the object must expose a data buffer
that will be decoded using the given encoding and error handler.
Otherwise, returns the result of object.\_\_str\_\_() (if defined)
or repr(object).
encoding defaults to sys.getdefaultencoding().
errors defaults to ‘strict’.

\end{fulllineitems}

\index{count\_entities\_by\_resource() (pypath.core.network.Network method)@\spxentry{count\_entities\_by\_resource()}\spxextra{pypath.core.network.Network method}}

\begin{fulllineitems}
\phantomsection\label{\detokenize{reference:pypath.core.network.Network.count_entities_by_resource}}\pysiglinewithargsret{\sphinxbfcode{\sphinxupquote{count\_entities\_by\_resource}}}{}{}
str(object=’‘) -\textgreater{} str
str(bytes\_or\_buffer{[}, encoding{[}, errors{]}{]}) -\textgreater{} str

Create a new string object from the given object. If encoding or
errors is specified, then the object must expose a data buffer
that will be decoded using the given encoding and error handler.
Otherwise, returns the result of object.\_\_str\_\_() (if defined)
or repr(object).
encoding defaults to sys.getdefaultencoding().
errors defaults to ‘strict’.

\end{fulllineitems}

\index{count\_evidences() (pypath.core.network.Network method)@\spxentry{count\_evidences()}\spxextra{pypath.core.network.Network method}}

\begin{fulllineitems}
\phantomsection\label{\detokenize{reference:pypath.core.network.Network.count_evidences}}\pysiglinewithargsret{\sphinxbfcode{\sphinxupquote{count\_evidences}}}{}{}
str(object=’‘) -\textgreater{} str
str(bytes\_or\_buffer{[}, encoding{[}, errors{]}{]}) -\textgreater{} str

Create a new string object from the given object. If encoding or
errors is specified, then the object must expose a data buffer
that will be decoded using the given encoding and error handler.
Otherwise, returns the result of object.\_\_str\_\_() (if defined)
or repr(object).
encoding defaults to sys.getdefaultencoding().
errors defaults to ‘strict’.

\end{fulllineitems}

\index{count\_evidences\_by\_data\_model() (pypath.core.network.Network method)@\spxentry{count\_evidences\_by\_data\_model()}\spxextra{pypath.core.network.Network method}}

\begin{fulllineitems}
\phantomsection\label{\detokenize{reference:pypath.core.network.Network.count_evidences_by_data_model}}\pysiglinewithargsret{\sphinxbfcode{\sphinxupquote{count\_evidences\_by\_data\_model}}}{}{}
str(object=’‘) -\textgreater{} str
str(bytes\_or\_buffer{[}, encoding{[}, errors{]}{]}) -\textgreater{} str

Create a new string object from the given object. If encoding or
errors is specified, then the object must expose a data buffer
that will be decoded using the given encoding and error handler.
Otherwise, returns the result of object.\_\_str\_\_() (if defined)
or repr(object).
encoding defaults to sys.getdefaultencoding().
errors defaults to ‘strict’.

\end{fulllineitems}

\index{count\_evidences\_by\_interaction\_type() (pypath.core.network.Network method)@\spxentry{count\_evidences\_by\_interaction\_type()}\spxextra{pypath.core.network.Network method}}

\begin{fulllineitems}
\phantomsection\label{\detokenize{reference:pypath.core.network.Network.count_evidences_by_interaction_type}}\pysiglinewithargsret{\sphinxbfcode{\sphinxupquote{count\_evidences\_by\_interaction\_type}}}{}{}
str(object=’‘) -\textgreater{} str
str(bytes\_or\_buffer{[}, encoding{[}, errors{]}{]}) -\textgreater{} str

Create a new string object from the given object. If encoding or
errors is specified, then the object must expose a data buffer
that will be decoded using the given encoding and error handler.
Otherwise, returns the result of object.\_\_str\_\_() (if defined)
or repr(object).
encoding defaults to sys.getdefaultencoding().
errors defaults to ‘strict’.

\end{fulllineitems}

\index{count\_evidences\_by\_interaction\_type\_and\_data\_model() (pypath.core.network.Network method)@\spxentry{count\_evidences\_by\_interaction\_type\_and\_data\_model()}\spxextra{pypath.core.network.Network method}}

\begin{fulllineitems}
\phantomsection\label{\detokenize{reference:pypath.core.network.Network.count_evidences_by_interaction_type_and_data_model}}\pysiglinewithargsret{\sphinxbfcode{\sphinxupquote{count\_evidences\_by\_interaction\_type\_and\_data\_model}}}{}{}
str(object=’‘) -\textgreater{} str
str(bytes\_or\_buffer{[}, encoding{[}, errors{]}{]}) -\textgreater{} str

Create a new string object from the given object. If encoding or
errors is specified, then the object must expose a data buffer
that will be decoded using the given encoding and error handler.
Otherwise, returns the result of object.\_\_str\_\_() (if defined)
or repr(object).
encoding defaults to sys.getdefaultencoding().
errors defaults to ‘strict’.

\end{fulllineitems}

\index{count\_evidences\_by\_interaction\_type\_and\_data\_model\_and\_resource() (pypath.core.network.Network method)@\spxentry{count\_evidences\_by\_interaction\_type\_and\_data\_model\_and\_resource()}\spxextra{pypath.core.network.Network method}}

\begin{fulllineitems}
\phantomsection\label{\detokenize{reference:pypath.core.network.Network.count_evidences_by_interaction_type_and_data_model_and_resource}}\pysiglinewithargsret{\sphinxbfcode{\sphinxupquote{count\_evidences\_by\_interaction\_type\_and\_data\_model\_and\_resource}}}{}{}
str(object=’‘) -\textgreater{} str
str(bytes\_or\_buffer{[}, encoding{[}, errors{]}{]}) -\textgreater{} str

Create a new string object from the given object. If encoding or
errors is specified, then the object must expose a data buffer
that will be decoded using the given encoding and error handler.
Otherwise, returns the result of object.\_\_str\_\_() (if defined)
or repr(object).
encoding defaults to sys.getdefaultencoding().
errors defaults to ‘strict’.

\end{fulllineitems}

\index{count\_evidences\_by\_reference() (pypath.core.network.Network method)@\spxentry{count\_evidences\_by\_reference()}\spxextra{pypath.core.network.Network method}}

\begin{fulllineitems}
\phantomsection\label{\detokenize{reference:pypath.core.network.Network.count_evidences_by_reference}}\pysiglinewithargsret{\sphinxbfcode{\sphinxupquote{count\_evidences\_by\_reference}}}{}{}
str(object=’‘) -\textgreater{} str
str(bytes\_or\_buffer{[}, encoding{[}, errors{]}{]}) -\textgreater{} str

Create a new string object from the given object. If encoding or
errors is specified, then the object must expose a data buffer
that will be decoded using the given encoding and error handler.
Otherwise, returns the result of object.\_\_str\_\_() (if defined)
or repr(object).
encoding defaults to sys.getdefaultencoding().
errors defaults to ‘strict’.

\end{fulllineitems}

\index{count\_evidences\_by\_resource() (pypath.core.network.Network method)@\spxentry{count\_evidences\_by\_resource()}\spxextra{pypath.core.network.Network method}}

\begin{fulllineitems}
\phantomsection\label{\detokenize{reference:pypath.core.network.Network.count_evidences_by_resource}}\pysiglinewithargsret{\sphinxbfcode{\sphinxupquote{count\_evidences\_by\_resource}}}{}{}
str(object=’‘) -\textgreater{} str
str(bytes\_or\_buffer{[}, encoding{[}, errors{]}{]}) -\textgreater{} str

Create a new string object from the given object. If encoding or
errors is specified, then the object must expose a data buffer
that will be decoded using the given encoding and error handler.
Otherwise, returns the result of object.\_\_str\_\_() (if defined)
or repr(object).
encoding defaults to sys.getdefaultencoding().
errors defaults to ‘strict’.

\end{fulllineitems}

\index{count\_identifiers() (pypath.core.network.Network method)@\spxentry{count\_identifiers()}\spxextra{pypath.core.network.Network method}}

\begin{fulllineitems}
\phantomsection\label{\detokenize{reference:pypath.core.network.Network.count_identifiers}}\pysiglinewithargsret{\sphinxbfcode{\sphinxupquote{count\_identifiers}}}{}{}
str(object=’‘) -\textgreater{} str
str(bytes\_or\_buffer{[}, encoding{[}, errors{]}{]}) -\textgreater{} str

Create a new string object from the given object. If encoding or
errors is specified, then the object must expose a data buffer
that will be decoded using the given encoding and error handler.
Otherwise, returns the result of object.\_\_str\_\_() (if defined)
or repr(object).
encoding defaults to sys.getdefaultencoding().
errors defaults to ‘strict’.

\end{fulllineitems}

\index{count\_identifiers\_by\_data\_model() (pypath.core.network.Network method)@\spxentry{count\_identifiers\_by\_data\_model()}\spxextra{pypath.core.network.Network method}}

\begin{fulllineitems}
\phantomsection\label{\detokenize{reference:pypath.core.network.Network.count_identifiers_by_data_model}}\pysiglinewithargsret{\sphinxbfcode{\sphinxupquote{count\_identifiers\_by\_data\_model}}}{}{}
str(object=’‘) -\textgreater{} str
str(bytes\_or\_buffer{[}, encoding{[}, errors{]}{]}) -\textgreater{} str

Create a new string object from the given object. If encoding or
errors is specified, then the object must expose a data buffer
that will be decoded using the given encoding and error handler.
Otherwise, returns the result of object.\_\_str\_\_() (if defined)
or repr(object).
encoding defaults to sys.getdefaultencoding().
errors defaults to ‘strict’.

\end{fulllineitems}

\index{count\_identifiers\_by\_interaction\_type() (pypath.core.network.Network method)@\spxentry{count\_identifiers\_by\_interaction\_type()}\spxextra{pypath.core.network.Network method}}

\begin{fulllineitems}
\phantomsection\label{\detokenize{reference:pypath.core.network.Network.count_identifiers_by_interaction_type}}\pysiglinewithargsret{\sphinxbfcode{\sphinxupquote{count\_identifiers\_by\_interaction\_type}}}{}{}
str(object=’‘) -\textgreater{} str
str(bytes\_or\_buffer{[}, encoding{[}, errors{]}{]}) -\textgreater{} str

Create a new string object from the given object. If encoding or
errors is specified, then the object must expose a data buffer
that will be decoded using the given encoding and error handler.
Otherwise, returns the result of object.\_\_str\_\_() (if defined)
or repr(object).
encoding defaults to sys.getdefaultencoding().
errors defaults to ‘strict’.

\end{fulllineitems}

\index{count\_identifiers\_by\_interaction\_type\_and\_data\_model() (pypath.core.network.Network method)@\spxentry{count\_identifiers\_by\_interaction\_type\_and\_data\_model()}\spxextra{pypath.core.network.Network method}}

\begin{fulllineitems}
\phantomsection\label{\detokenize{reference:pypath.core.network.Network.count_identifiers_by_interaction_type_and_data_model}}\pysiglinewithargsret{\sphinxbfcode{\sphinxupquote{count\_identifiers\_by\_interaction\_type\_and\_data\_model}}}{}{}
str(object=’‘) -\textgreater{} str
str(bytes\_or\_buffer{[}, encoding{[}, errors{]}{]}) -\textgreater{} str

Create a new string object from the given object. If encoding or
errors is specified, then the object must expose a data buffer
that will be decoded using the given encoding and error handler.
Otherwise, returns the result of object.\_\_str\_\_() (if defined)
or repr(object).
encoding defaults to sys.getdefaultencoding().
errors defaults to ‘strict’.

\end{fulllineitems}

\index{count\_identifiers\_by\_interaction\_type\_and\_data\_model\_and\_resource() (pypath.core.network.Network method)@\spxentry{count\_identifiers\_by\_interaction\_type\_and\_data\_model\_and\_resource()}\spxextra{pypath.core.network.Network method}}

\begin{fulllineitems}
\phantomsection\label{\detokenize{reference:pypath.core.network.Network.count_identifiers_by_interaction_type_and_data_model_and_resource}}\pysiglinewithargsret{\sphinxbfcode{\sphinxupquote{count\_identifiers\_by\_interaction\_type\_and\_data\_model\_and\_resource}}}{}{}
str(object=’‘) -\textgreater{} str
str(bytes\_or\_buffer{[}, encoding{[}, errors{]}{]}) -\textgreater{} str

Create a new string object from the given object. If encoding or
errors is specified, then the object must expose a data buffer
that will be decoded using the given encoding and error handler.
Otherwise, returns the result of object.\_\_str\_\_() (if defined)
or repr(object).
encoding defaults to sys.getdefaultencoding().
errors defaults to ‘strict’.

\end{fulllineitems}

\index{count\_identifiers\_by\_reference() (pypath.core.network.Network method)@\spxentry{count\_identifiers\_by\_reference()}\spxextra{pypath.core.network.Network method}}

\begin{fulllineitems}
\phantomsection\label{\detokenize{reference:pypath.core.network.Network.count_identifiers_by_reference}}\pysiglinewithargsret{\sphinxbfcode{\sphinxupquote{count\_identifiers\_by\_reference}}}{}{}
str(object=’‘) -\textgreater{} str
str(bytes\_or\_buffer{[}, encoding{[}, errors{]}{]}) -\textgreater{} str

Create a new string object from the given object. If encoding or
errors is specified, then the object must expose a data buffer
that will be decoded using the given encoding and error handler.
Otherwise, returns the result of object.\_\_str\_\_() (if defined)
or repr(object).
encoding defaults to sys.getdefaultencoding().
errors defaults to ‘strict’.

\end{fulllineitems}

\index{count\_identifiers\_by\_resource() (pypath.core.network.Network method)@\spxentry{count\_identifiers\_by\_resource()}\spxextra{pypath.core.network.Network method}}

\begin{fulllineitems}
\phantomsection\label{\detokenize{reference:pypath.core.network.Network.count_identifiers_by_resource}}\pysiglinewithargsret{\sphinxbfcode{\sphinxupquote{count\_identifiers\_by\_resource}}}{}{}
str(object=’‘) -\textgreater{} str
str(bytes\_or\_buffer{[}, encoding{[}, errors{]}{]}) -\textgreater{} str

Create a new string object from the given object. If encoding or
errors is specified, then the object must expose a data buffer
that will be decoded using the given encoding and error handler.
Otherwise, returns the result of object.\_\_str\_\_() (if defined)
or repr(object).
encoding defaults to sys.getdefaultencoding().
errors defaults to ‘strict’.

\end{fulllineitems}

\index{count\_interaction\_types() (pypath.core.network.Network method)@\spxentry{count\_interaction\_types()}\spxextra{pypath.core.network.Network method}}

\begin{fulllineitems}
\phantomsection\label{\detokenize{reference:pypath.core.network.Network.count_interaction_types}}\pysiglinewithargsret{\sphinxbfcode{\sphinxupquote{count\_interaction\_types}}}{}{}
str(object=’‘) -\textgreater{} str
str(bytes\_or\_buffer{[}, encoding{[}, errors{]}{]}) -\textgreater{} str

Create a new string object from the given object. If encoding or
errors is specified, then the object must expose a data buffer
that will be decoded using the given encoding and error handler.
Otherwise, returns the result of object.\_\_str\_\_() (if defined)
or repr(object).
encoding defaults to sys.getdefaultencoding().
errors defaults to ‘strict’.

\end{fulllineitems}

\index{count\_interaction\_types\_by\_data\_model() (pypath.core.network.Network method)@\spxentry{count\_interaction\_types\_by\_data\_model()}\spxextra{pypath.core.network.Network method}}

\begin{fulllineitems}
\phantomsection\label{\detokenize{reference:pypath.core.network.Network.count_interaction_types_by_data_model}}\pysiglinewithargsret{\sphinxbfcode{\sphinxupquote{count\_interaction\_types\_by\_data\_model}}}{}{}
str(object=’‘) -\textgreater{} str
str(bytes\_or\_buffer{[}, encoding{[}, errors{]}{]}) -\textgreater{} str

Create a new string object from the given object. If encoding or
errors is specified, then the object must expose a data buffer
that will be decoded using the given encoding and error handler.
Otherwise, returns the result of object.\_\_str\_\_() (if defined)
or repr(object).
encoding defaults to sys.getdefaultencoding().
errors defaults to ‘strict’.

\end{fulllineitems}

\index{count\_interaction\_types\_by\_interaction\_type() (pypath.core.network.Network method)@\spxentry{count\_interaction\_types\_by\_interaction\_type()}\spxextra{pypath.core.network.Network method}}

\begin{fulllineitems}
\phantomsection\label{\detokenize{reference:pypath.core.network.Network.count_interaction_types_by_interaction_type}}\pysiglinewithargsret{\sphinxbfcode{\sphinxupquote{count\_interaction\_types\_by\_interaction\_type}}}{}{}
str(object=’‘) -\textgreater{} str
str(bytes\_or\_buffer{[}, encoding{[}, errors{]}{]}) -\textgreater{} str

Create a new string object from the given object. If encoding or
errors is specified, then the object must expose a data buffer
that will be decoded using the given encoding and error handler.
Otherwise, returns the result of object.\_\_str\_\_() (if defined)
or repr(object).
encoding defaults to sys.getdefaultencoding().
errors defaults to ‘strict’.

\end{fulllineitems}

\index{count\_interaction\_types\_by\_interaction\_type\_and\_data\_model() (pypath.core.network.Network method)@\spxentry{count\_interaction\_types\_by\_interaction\_type\_and\_data\_model()}\spxextra{pypath.core.network.Network method}}

\begin{fulllineitems}
\phantomsection\label{\detokenize{reference:pypath.core.network.Network.count_interaction_types_by_interaction_type_and_data_model}}\pysiglinewithargsret{\sphinxbfcode{\sphinxupquote{count\_interaction\_types\_by\_interaction\_type\_and\_data\_model}}}{}{}
str(object=’‘) -\textgreater{} str
str(bytes\_or\_buffer{[}, encoding{[}, errors{]}{]}) -\textgreater{} str

Create a new string object from the given object. If encoding or
errors is specified, then the object must expose a data buffer
that will be decoded using the given encoding and error handler.
Otherwise, returns the result of object.\_\_str\_\_() (if defined)
or repr(object).
encoding defaults to sys.getdefaultencoding().
errors defaults to ‘strict’.

\end{fulllineitems}

\index{count\_interaction\_types\_by\_interaction\_type\_and\_data\_model\_and\_resource() (pypath.core.network.Network method)@\spxentry{count\_interaction\_types\_by\_interaction\_type\_and\_data\_model\_and\_resource()}\spxextra{pypath.core.network.Network method}}

\begin{fulllineitems}
\phantomsection\label{\detokenize{reference:pypath.core.network.Network.count_interaction_types_by_interaction_type_and_data_model_and_resource}}\pysiglinewithargsret{\sphinxbfcode{\sphinxupquote{count\_interaction\_types\_by\_interaction\_type\_and\_data\_model\_and\_resource}}}{}{}
str(object=’‘) -\textgreater{} str
str(bytes\_or\_buffer{[}, encoding{[}, errors{]}{]}) -\textgreater{} str

Create a new string object from the given object. If encoding or
errors is specified, then the object must expose a data buffer
that will be decoded using the given encoding and error handler.
Otherwise, returns the result of object.\_\_str\_\_() (if defined)
or repr(object).
encoding defaults to sys.getdefaultencoding().
errors defaults to ‘strict’.

\end{fulllineitems}

\index{count\_interaction\_types\_by\_reference() (pypath.core.network.Network method)@\spxentry{count\_interaction\_types\_by\_reference()}\spxextra{pypath.core.network.Network method}}

\begin{fulllineitems}
\phantomsection\label{\detokenize{reference:pypath.core.network.Network.count_interaction_types_by_reference}}\pysiglinewithargsret{\sphinxbfcode{\sphinxupquote{count\_interaction\_types\_by\_reference}}}{}{}
str(object=’‘) -\textgreater{} str
str(bytes\_or\_buffer{[}, encoding{[}, errors{]}{]}) -\textgreater{} str

Create a new string object from the given object. If encoding or
errors is specified, then the object must expose a data buffer
that will be decoded using the given encoding and error handler.
Otherwise, returns the result of object.\_\_str\_\_() (if defined)
or repr(object).
encoding defaults to sys.getdefaultencoding().
errors defaults to ‘strict’.

\end{fulllineitems}

\index{count\_interaction\_types\_by\_resource() (pypath.core.network.Network method)@\spxentry{count\_interaction\_types\_by\_resource()}\spxextra{pypath.core.network.Network method}}

\begin{fulllineitems}
\phantomsection\label{\detokenize{reference:pypath.core.network.Network.count_interaction_types_by_resource}}\pysiglinewithargsret{\sphinxbfcode{\sphinxupquote{count\_interaction\_types\_by\_resource}}}{}{}
str(object=’‘) -\textgreater{} str
str(bytes\_or\_buffer{[}, encoding{[}, errors{]}{]}) -\textgreater{} str

Create a new string object from the given object. If encoding or
errors is specified, then the object must expose a data buffer
that will be decoded using the given encoding and error handler.
Otherwise, returns the result of object.\_\_str\_\_() (if defined)
or repr(object).
encoding defaults to sys.getdefaultencoding().
errors defaults to ‘strict’.

\end{fulllineitems}

\index{count\_interactions() (pypath.core.network.Network method)@\spxentry{count\_interactions()}\spxextra{pypath.core.network.Network method}}

\begin{fulllineitems}
\phantomsection\label{\detokenize{reference:pypath.core.network.Network.count_interactions}}\pysiglinewithargsret{\sphinxbfcode{\sphinxupquote{count\_interactions}}}{}{}
str(object=’‘) -\textgreater{} str
str(bytes\_or\_buffer{[}, encoding{[}, errors{]}{]}) -\textgreater{} str

Create a new string object from the given object. If encoding or
errors is specified, then the object must expose a data buffer
that will be decoded using the given encoding and error handler.
Otherwise, returns the result of object.\_\_str\_\_() (if defined)
or repr(object).
encoding defaults to sys.getdefaultencoding().
errors defaults to ‘strict’.

\end{fulllineitems}

\index{count\_interactions\_0() (pypath.core.network.Network method)@\spxentry{count\_interactions\_0()}\spxextra{pypath.core.network.Network method}}

\begin{fulllineitems}
\phantomsection\label{\detokenize{reference:pypath.core.network.Network.count_interactions_0}}\pysiglinewithargsret{\sphinxbfcode{\sphinxupquote{count\_interactions\_0}}}{}{}
str(object=’‘) -\textgreater{} str
str(bytes\_or\_buffer{[}, encoding{[}, errors{]}{]}) -\textgreater{} str

Create a new string object from the given object. If encoding or
errors is specified, then the object must expose a data buffer
that will be decoded using the given encoding and error handler.
Otherwise, returns the result of object.\_\_str\_\_() (if defined)
or repr(object).
encoding defaults to sys.getdefaultencoding().
errors defaults to ‘strict’.

\end{fulllineitems}

\index{count\_interactions\_0\_by\_data\_model() (pypath.core.network.Network method)@\spxentry{count\_interactions\_0\_by\_data\_model()}\spxextra{pypath.core.network.Network method}}

\begin{fulllineitems}
\phantomsection\label{\detokenize{reference:pypath.core.network.Network.count_interactions_0_by_data_model}}\pysiglinewithargsret{\sphinxbfcode{\sphinxupquote{count\_interactions\_0\_by\_data\_model}}}{}{}
str(object=’‘) -\textgreater{} str
str(bytes\_or\_buffer{[}, encoding{[}, errors{]}{]}) -\textgreater{} str

Create a new string object from the given object. If encoding or
errors is specified, then the object must expose a data buffer
that will be decoded using the given encoding and error handler.
Otherwise, returns the result of object.\_\_str\_\_() (if defined)
or repr(object).
encoding defaults to sys.getdefaultencoding().
errors defaults to ‘strict’.

\end{fulllineitems}

\index{count\_interactions\_0\_by\_interaction\_type() (pypath.core.network.Network method)@\spxentry{count\_interactions\_0\_by\_interaction\_type()}\spxextra{pypath.core.network.Network method}}

\begin{fulllineitems}
\phantomsection\label{\detokenize{reference:pypath.core.network.Network.count_interactions_0_by_interaction_type}}\pysiglinewithargsret{\sphinxbfcode{\sphinxupquote{count\_interactions\_0\_by\_interaction\_type}}}{}{}
str(object=’‘) -\textgreater{} str
str(bytes\_or\_buffer{[}, encoding{[}, errors{]}{]}) -\textgreater{} str

Create a new string object from the given object. If encoding or
errors is specified, then the object must expose a data buffer
that will be decoded using the given encoding and error handler.
Otherwise, returns the result of object.\_\_str\_\_() (if defined)
or repr(object).
encoding defaults to sys.getdefaultencoding().
errors defaults to ‘strict’.

\end{fulllineitems}

\index{count\_interactions\_0\_by\_interaction\_type\_and\_data\_model() (pypath.core.network.Network method)@\spxentry{count\_interactions\_0\_by\_interaction\_type\_and\_data\_model()}\spxextra{pypath.core.network.Network method}}

\begin{fulllineitems}
\phantomsection\label{\detokenize{reference:pypath.core.network.Network.count_interactions_0_by_interaction_type_and_data_model}}\pysiglinewithargsret{\sphinxbfcode{\sphinxupquote{count\_interactions\_0\_by\_interaction\_type\_and\_data\_model}}}{}{}
str(object=’‘) -\textgreater{} str
str(bytes\_or\_buffer{[}, encoding{[}, errors{]}{]}) -\textgreater{} str

Create a new string object from the given object. If encoding or
errors is specified, then the object must expose a data buffer
that will be decoded using the given encoding and error handler.
Otherwise, returns the result of object.\_\_str\_\_() (if defined)
or repr(object).
encoding defaults to sys.getdefaultencoding().
errors defaults to ‘strict’.

\end{fulllineitems}

\index{count\_interactions\_0\_by\_interaction\_type\_and\_data\_model\_and\_resource() (pypath.core.network.Network method)@\spxentry{count\_interactions\_0\_by\_interaction\_type\_and\_data\_model\_and\_resource()}\spxextra{pypath.core.network.Network method}}

\begin{fulllineitems}
\phantomsection\label{\detokenize{reference:pypath.core.network.Network.count_interactions_0_by_interaction_type_and_data_model_and_resource}}\pysiglinewithargsret{\sphinxbfcode{\sphinxupquote{count\_interactions\_0\_by\_interaction\_type\_and\_data\_model\_and\_resource}}}{}{}
str(object=’‘) -\textgreater{} str
str(bytes\_or\_buffer{[}, encoding{[}, errors{]}{]}) -\textgreater{} str

Create a new string object from the given object. If encoding or
errors is specified, then the object must expose a data buffer
that will be decoded using the given encoding and error handler.
Otherwise, returns the result of object.\_\_str\_\_() (if defined)
or repr(object).
encoding defaults to sys.getdefaultencoding().
errors defaults to ‘strict’.

\end{fulllineitems}

\index{count\_interactions\_0\_by\_reference() (pypath.core.network.Network method)@\spxentry{count\_interactions\_0\_by\_reference()}\spxextra{pypath.core.network.Network method}}

\begin{fulllineitems}
\phantomsection\label{\detokenize{reference:pypath.core.network.Network.count_interactions_0_by_reference}}\pysiglinewithargsret{\sphinxbfcode{\sphinxupquote{count\_interactions\_0\_by\_reference}}}{}{}
str(object=’‘) -\textgreater{} str
str(bytes\_or\_buffer{[}, encoding{[}, errors{]}{]}) -\textgreater{} str

Create a new string object from the given object. If encoding or
errors is specified, then the object must expose a data buffer
that will be decoded using the given encoding and error handler.
Otherwise, returns the result of object.\_\_str\_\_() (if defined)
or repr(object).
encoding defaults to sys.getdefaultencoding().
errors defaults to ‘strict’.

\end{fulllineitems}

\index{count\_interactions\_0\_by\_resource() (pypath.core.network.Network method)@\spxentry{count\_interactions\_0\_by\_resource()}\spxextra{pypath.core.network.Network method}}

\begin{fulllineitems}
\phantomsection\label{\detokenize{reference:pypath.core.network.Network.count_interactions_0_by_resource}}\pysiglinewithargsret{\sphinxbfcode{\sphinxupquote{count\_interactions\_0\_by\_resource}}}{}{}
str(object=’‘) -\textgreater{} str
str(bytes\_or\_buffer{[}, encoding{[}, errors{]}{]}) -\textgreater{} str

Create a new string object from the given object. If encoding or
errors is specified, then the object must expose a data buffer
that will be decoded using the given encoding and error handler.
Otherwise, returns the result of object.\_\_str\_\_() (if defined)
or repr(object).
encoding defaults to sys.getdefaultencoding().
errors defaults to ‘strict’.

\end{fulllineitems}

\index{count\_interactions\_by\_data\_model() (pypath.core.network.Network method)@\spxentry{count\_interactions\_by\_data\_model()}\spxextra{pypath.core.network.Network method}}

\begin{fulllineitems}
\phantomsection\label{\detokenize{reference:pypath.core.network.Network.count_interactions_by_data_model}}\pysiglinewithargsret{\sphinxbfcode{\sphinxupquote{count\_interactions\_by\_data\_model}}}{}{}
str(object=’‘) -\textgreater{} str
str(bytes\_or\_buffer{[}, encoding{[}, errors{]}{]}) -\textgreater{} str

Create a new string object from the given object. If encoding or
errors is specified, then the object must expose a data buffer
that will be decoded using the given encoding and error handler.
Otherwise, returns the result of object.\_\_str\_\_() (if defined)
or repr(object).
encoding defaults to sys.getdefaultencoding().
errors defaults to ‘strict’.

\end{fulllineitems}

\index{count\_interactions\_by\_interaction\_type() (pypath.core.network.Network method)@\spxentry{count\_interactions\_by\_interaction\_type()}\spxextra{pypath.core.network.Network method}}

\begin{fulllineitems}
\phantomsection\label{\detokenize{reference:pypath.core.network.Network.count_interactions_by_interaction_type}}\pysiglinewithargsret{\sphinxbfcode{\sphinxupquote{count\_interactions\_by\_interaction\_type}}}{}{}
str(object=’‘) -\textgreater{} str
str(bytes\_or\_buffer{[}, encoding{[}, errors{]}{]}) -\textgreater{} str

Create a new string object from the given object. If encoding or
errors is specified, then the object must expose a data buffer
that will be decoded using the given encoding and error handler.
Otherwise, returns the result of object.\_\_str\_\_() (if defined)
or repr(object).
encoding defaults to sys.getdefaultencoding().
errors defaults to ‘strict’.

\end{fulllineitems}

\index{count\_interactions\_by\_interaction\_type\_and\_data\_model() (pypath.core.network.Network method)@\spxentry{count\_interactions\_by\_interaction\_type\_and\_data\_model()}\spxextra{pypath.core.network.Network method}}

\begin{fulllineitems}
\phantomsection\label{\detokenize{reference:pypath.core.network.Network.count_interactions_by_interaction_type_and_data_model}}\pysiglinewithargsret{\sphinxbfcode{\sphinxupquote{count\_interactions\_by\_interaction\_type\_and\_data\_model}}}{}{}
str(object=’‘) -\textgreater{} str
str(bytes\_or\_buffer{[}, encoding{[}, errors{]}{]}) -\textgreater{} str

Create a new string object from the given object. If encoding or
errors is specified, then the object must expose a data buffer
that will be decoded using the given encoding and error handler.
Otherwise, returns the result of object.\_\_str\_\_() (if defined)
or repr(object).
encoding defaults to sys.getdefaultencoding().
errors defaults to ‘strict’.

\end{fulllineitems}

\index{count\_interactions\_by\_interaction\_type\_and\_data\_model\_and\_resource() (pypath.core.network.Network method)@\spxentry{count\_interactions\_by\_interaction\_type\_and\_data\_model\_and\_resource()}\spxextra{pypath.core.network.Network method}}

\begin{fulllineitems}
\phantomsection\label{\detokenize{reference:pypath.core.network.Network.count_interactions_by_interaction_type_and_data_model_and_resource}}\pysiglinewithargsret{\sphinxbfcode{\sphinxupquote{count\_interactions\_by\_interaction\_type\_and\_data\_model\_and\_resource}}}{}{}
str(object=’‘) -\textgreater{} str
str(bytes\_or\_buffer{[}, encoding{[}, errors{]}{]}) -\textgreater{} str

Create a new string object from the given object. If encoding or
errors is specified, then the object must expose a data buffer
that will be decoded using the given encoding and error handler.
Otherwise, returns the result of object.\_\_str\_\_() (if defined)
or repr(object).
encoding defaults to sys.getdefaultencoding().
errors defaults to ‘strict’.

\end{fulllineitems}

\index{count\_interactions\_by\_reference() (pypath.core.network.Network method)@\spxentry{count\_interactions\_by\_reference()}\spxextra{pypath.core.network.Network method}}

\begin{fulllineitems}
\phantomsection\label{\detokenize{reference:pypath.core.network.Network.count_interactions_by_reference}}\pysiglinewithargsret{\sphinxbfcode{\sphinxupquote{count\_interactions\_by\_reference}}}{}{}
str(object=’‘) -\textgreater{} str
str(bytes\_or\_buffer{[}, encoding{[}, errors{]}{]}) -\textgreater{} str

Create a new string object from the given object. If encoding or
errors is specified, then the object must expose a data buffer
that will be decoded using the given encoding and error handler.
Otherwise, returns the result of object.\_\_str\_\_() (if defined)
or repr(object).
encoding defaults to sys.getdefaultencoding().
errors defaults to ‘strict’.

\end{fulllineitems}

\index{count\_interactions\_by\_resource() (pypath.core.network.Network method)@\spxentry{count\_interactions\_by\_resource()}\spxextra{pypath.core.network.Network method}}

\begin{fulllineitems}
\phantomsection\label{\detokenize{reference:pypath.core.network.Network.count_interactions_by_resource}}\pysiglinewithargsret{\sphinxbfcode{\sphinxupquote{count\_interactions\_by\_resource}}}{}{}
str(object=’‘) -\textgreater{} str
str(bytes\_or\_buffer{[}, encoding{[}, errors{]}{]}) -\textgreater{} str

Create a new string object from the given object. If encoding or
errors is specified, then the object must expose a data buffer
that will be decoded using the given encoding and error handler.
Otherwise, returns the result of object.\_\_str\_\_() (if defined)
or repr(object).
encoding defaults to sys.getdefaultencoding().
errors defaults to ‘strict’.

\end{fulllineitems}

\index{count\_interactions\_directed() (pypath.core.network.Network method)@\spxentry{count\_interactions\_directed()}\spxextra{pypath.core.network.Network method}}

\begin{fulllineitems}
\phantomsection\label{\detokenize{reference:pypath.core.network.Network.count_interactions_directed}}\pysiglinewithargsret{\sphinxbfcode{\sphinxupquote{count\_interactions\_directed}}}{}{}
str(object=’‘) -\textgreater{} str
str(bytes\_or\_buffer{[}, encoding{[}, errors{]}{]}) -\textgreater{} str

Create a new string object from the given object. If encoding or
errors is specified, then the object must expose a data buffer
that will be decoded using the given encoding and error handler.
Otherwise, returns the result of object.\_\_str\_\_() (if defined)
or repr(object).
encoding defaults to sys.getdefaultencoding().
errors defaults to ‘strict’.

\end{fulllineitems}

\index{count\_interactions\_directed\_by\_data\_model() (pypath.core.network.Network method)@\spxentry{count\_interactions\_directed\_by\_data\_model()}\spxextra{pypath.core.network.Network method}}

\begin{fulllineitems}
\phantomsection\label{\detokenize{reference:pypath.core.network.Network.count_interactions_directed_by_data_model}}\pysiglinewithargsret{\sphinxbfcode{\sphinxupquote{count\_interactions\_directed\_by\_data\_model}}}{}{}
str(object=’‘) -\textgreater{} str
str(bytes\_or\_buffer{[}, encoding{[}, errors{]}{]}) -\textgreater{} str

Create a new string object from the given object. If encoding or
errors is specified, then the object must expose a data buffer
that will be decoded using the given encoding and error handler.
Otherwise, returns the result of object.\_\_str\_\_() (if defined)
or repr(object).
encoding defaults to sys.getdefaultencoding().
errors defaults to ‘strict’.

\end{fulllineitems}

\index{count\_interactions\_directed\_by\_interaction\_type() (pypath.core.network.Network method)@\spxentry{count\_interactions\_directed\_by\_interaction\_type()}\spxextra{pypath.core.network.Network method}}

\begin{fulllineitems}
\phantomsection\label{\detokenize{reference:pypath.core.network.Network.count_interactions_directed_by_interaction_type}}\pysiglinewithargsret{\sphinxbfcode{\sphinxupquote{count\_interactions\_directed\_by\_interaction\_type}}}{}{}
str(object=’‘) -\textgreater{} str
str(bytes\_or\_buffer{[}, encoding{[}, errors{]}{]}) -\textgreater{} str

Create a new string object from the given object. If encoding or
errors is specified, then the object must expose a data buffer
that will be decoded using the given encoding and error handler.
Otherwise, returns the result of object.\_\_str\_\_() (if defined)
or repr(object).
encoding defaults to sys.getdefaultencoding().
errors defaults to ‘strict’.

\end{fulllineitems}

\index{count\_interactions\_directed\_by\_interaction\_type\_and\_data\_model() (pypath.core.network.Network method)@\spxentry{count\_interactions\_directed\_by\_interaction\_type\_and\_data\_model()}\spxextra{pypath.core.network.Network method}}

\begin{fulllineitems}
\phantomsection\label{\detokenize{reference:pypath.core.network.Network.count_interactions_directed_by_interaction_type_and_data_model}}\pysiglinewithargsret{\sphinxbfcode{\sphinxupquote{count\_interactions\_directed\_by\_interaction\_type\_and\_data\_model}}}{}{}
str(object=’‘) -\textgreater{} str
str(bytes\_or\_buffer{[}, encoding{[}, errors{]}{]}) -\textgreater{} str

Create a new string object from the given object. If encoding or
errors is specified, then the object must expose a data buffer
that will be decoded using the given encoding and error handler.
Otherwise, returns the result of object.\_\_str\_\_() (if defined)
or repr(object).
encoding defaults to sys.getdefaultencoding().
errors defaults to ‘strict’.

\end{fulllineitems}

\index{count\_interactions\_directed\_by\_interaction\_type\_and\_data\_model\_and\_resource() (pypath.core.network.Network method)@\spxentry{count\_interactions\_directed\_by\_interaction\_type\_and\_data\_model\_and\_resource()}\spxextra{pypath.core.network.Network method}}

\begin{fulllineitems}
\phantomsection\label{\detokenize{reference:pypath.core.network.Network.count_interactions_directed_by_interaction_type_and_data_model_and_resource}}\pysiglinewithargsret{\sphinxbfcode{\sphinxupquote{count\_interactions\_directed\_by\_interaction\_type\_and\_data\_model\_and\_resource}}}{}{}
str(object=’‘) -\textgreater{} str
str(bytes\_or\_buffer{[}, encoding{[}, errors{]}{]}) -\textgreater{} str

Create a new string object from the given object. If encoding or
errors is specified, then the object must expose a data buffer
that will be decoded using the given encoding and error handler.
Otherwise, returns the result of object.\_\_str\_\_() (if defined)
or repr(object).
encoding defaults to sys.getdefaultencoding().
errors defaults to ‘strict’.

\end{fulllineitems}

\index{count\_interactions\_directed\_by\_reference() (pypath.core.network.Network method)@\spxentry{count\_interactions\_directed\_by\_reference()}\spxextra{pypath.core.network.Network method}}

\begin{fulllineitems}
\phantomsection\label{\detokenize{reference:pypath.core.network.Network.count_interactions_directed_by_reference}}\pysiglinewithargsret{\sphinxbfcode{\sphinxupquote{count\_interactions\_directed\_by\_reference}}}{}{}
str(object=’‘) -\textgreater{} str
str(bytes\_or\_buffer{[}, encoding{[}, errors{]}{]}) -\textgreater{} str

Create a new string object from the given object. If encoding or
errors is specified, then the object must expose a data buffer
that will be decoded using the given encoding and error handler.
Otherwise, returns the result of object.\_\_str\_\_() (if defined)
or repr(object).
encoding defaults to sys.getdefaultencoding().
errors defaults to ‘strict’.

\end{fulllineitems}

\index{count\_interactions\_directed\_by\_resource() (pypath.core.network.Network method)@\spxentry{count\_interactions\_directed\_by\_resource()}\spxextra{pypath.core.network.Network method}}

\begin{fulllineitems}
\phantomsection\label{\detokenize{reference:pypath.core.network.Network.count_interactions_directed_by_resource}}\pysiglinewithargsret{\sphinxbfcode{\sphinxupquote{count\_interactions\_directed\_by\_resource}}}{}{}
str(object=’‘) -\textgreater{} str
str(bytes\_or\_buffer{[}, encoding{[}, errors{]}{]}) -\textgreater{} str

Create a new string object from the given object. If encoding or
errors is specified, then the object must expose a data buffer
that will be decoded using the given encoding and error handler.
Otherwise, returns the result of object.\_\_str\_\_() (if defined)
or repr(object).
encoding defaults to sys.getdefaultencoding().
errors defaults to ‘strict’.

\end{fulllineitems}

\index{count\_interactions\_mutual() (pypath.core.network.Network method)@\spxentry{count\_interactions\_mutual()}\spxextra{pypath.core.network.Network method}}

\begin{fulllineitems}
\phantomsection\label{\detokenize{reference:pypath.core.network.Network.count_interactions_mutual}}\pysiglinewithargsret{\sphinxbfcode{\sphinxupquote{count\_interactions\_mutual}}}{}{}
str(object=’‘) -\textgreater{} str
str(bytes\_or\_buffer{[}, encoding{[}, errors{]}{]}) -\textgreater{} str

Create a new string object from the given object. If encoding or
errors is specified, then the object must expose a data buffer
that will be decoded using the given encoding and error handler.
Otherwise, returns the result of object.\_\_str\_\_() (if defined)
or repr(object).
encoding defaults to sys.getdefaultencoding().
errors defaults to ‘strict’.

\end{fulllineitems}

\index{count\_interactions\_mutual\_by\_data\_model() (pypath.core.network.Network method)@\spxentry{count\_interactions\_mutual\_by\_data\_model()}\spxextra{pypath.core.network.Network method}}

\begin{fulllineitems}
\phantomsection\label{\detokenize{reference:pypath.core.network.Network.count_interactions_mutual_by_data_model}}\pysiglinewithargsret{\sphinxbfcode{\sphinxupquote{count\_interactions\_mutual\_by\_data\_model}}}{}{}
str(object=’‘) -\textgreater{} str
str(bytes\_or\_buffer{[}, encoding{[}, errors{]}{]}) -\textgreater{} str

Create a new string object from the given object. If encoding or
errors is specified, then the object must expose a data buffer
that will be decoded using the given encoding and error handler.
Otherwise, returns the result of object.\_\_str\_\_() (if defined)
or repr(object).
encoding defaults to sys.getdefaultencoding().
errors defaults to ‘strict’.

\end{fulllineitems}

\index{count\_interactions\_mutual\_by\_interaction\_type() (pypath.core.network.Network method)@\spxentry{count\_interactions\_mutual\_by\_interaction\_type()}\spxextra{pypath.core.network.Network method}}

\begin{fulllineitems}
\phantomsection\label{\detokenize{reference:pypath.core.network.Network.count_interactions_mutual_by_interaction_type}}\pysiglinewithargsret{\sphinxbfcode{\sphinxupquote{count\_interactions\_mutual\_by\_interaction\_type}}}{}{}
str(object=’‘) -\textgreater{} str
str(bytes\_or\_buffer{[}, encoding{[}, errors{]}{]}) -\textgreater{} str

Create a new string object from the given object. If encoding or
errors is specified, then the object must expose a data buffer
that will be decoded using the given encoding and error handler.
Otherwise, returns the result of object.\_\_str\_\_() (if defined)
or repr(object).
encoding defaults to sys.getdefaultencoding().
errors defaults to ‘strict’.

\end{fulllineitems}

\index{count\_interactions\_mutual\_by\_interaction\_type\_and\_data\_model() (pypath.core.network.Network method)@\spxentry{count\_interactions\_mutual\_by\_interaction\_type\_and\_data\_model()}\spxextra{pypath.core.network.Network method}}

\begin{fulllineitems}
\phantomsection\label{\detokenize{reference:pypath.core.network.Network.count_interactions_mutual_by_interaction_type_and_data_model}}\pysiglinewithargsret{\sphinxbfcode{\sphinxupquote{count\_interactions\_mutual\_by\_interaction\_type\_and\_data\_model}}}{}{}
str(object=’‘) -\textgreater{} str
str(bytes\_or\_buffer{[}, encoding{[}, errors{]}{]}) -\textgreater{} str

Create a new string object from the given object. If encoding or
errors is specified, then the object must expose a data buffer
that will be decoded using the given encoding and error handler.
Otherwise, returns the result of object.\_\_str\_\_() (if defined)
or repr(object).
encoding defaults to sys.getdefaultencoding().
errors defaults to ‘strict’.

\end{fulllineitems}

\index{count\_interactions\_mutual\_by\_interaction\_type\_and\_data\_model\_and\_resource() (pypath.core.network.Network method)@\spxentry{count\_interactions\_mutual\_by\_interaction\_type\_and\_data\_model\_and\_resource()}\spxextra{pypath.core.network.Network method}}

\begin{fulllineitems}
\phantomsection\label{\detokenize{reference:pypath.core.network.Network.count_interactions_mutual_by_interaction_type_and_data_model_and_resource}}\pysiglinewithargsret{\sphinxbfcode{\sphinxupquote{count\_interactions\_mutual\_by\_interaction\_type\_and\_data\_model\_and\_resource}}}{}{}
str(object=’‘) -\textgreater{} str
str(bytes\_or\_buffer{[}, encoding{[}, errors{]}{]}) -\textgreater{} str

Create a new string object from the given object. If encoding or
errors is specified, then the object must expose a data buffer
that will be decoded using the given encoding and error handler.
Otherwise, returns the result of object.\_\_str\_\_() (if defined)
or repr(object).
encoding defaults to sys.getdefaultencoding().
errors defaults to ‘strict’.

\end{fulllineitems}

\index{count\_interactions\_mutual\_by\_reference() (pypath.core.network.Network method)@\spxentry{count\_interactions\_mutual\_by\_reference()}\spxextra{pypath.core.network.Network method}}

\begin{fulllineitems}
\phantomsection\label{\detokenize{reference:pypath.core.network.Network.count_interactions_mutual_by_reference}}\pysiglinewithargsret{\sphinxbfcode{\sphinxupquote{count\_interactions\_mutual\_by\_reference}}}{}{}
str(object=’‘) -\textgreater{} str
str(bytes\_or\_buffer{[}, encoding{[}, errors{]}{]}) -\textgreater{} str

Create a new string object from the given object. If encoding or
errors is specified, then the object must expose a data buffer
that will be decoded using the given encoding and error handler.
Otherwise, returns the result of object.\_\_str\_\_() (if defined)
or repr(object).
encoding defaults to sys.getdefaultencoding().
errors defaults to ‘strict’.

\end{fulllineitems}

\index{count\_interactions\_mutual\_by\_resource() (pypath.core.network.Network method)@\spxentry{count\_interactions\_mutual\_by\_resource()}\spxextra{pypath.core.network.Network method}}

\begin{fulllineitems}
\phantomsection\label{\detokenize{reference:pypath.core.network.Network.count_interactions_mutual_by_resource}}\pysiglinewithargsret{\sphinxbfcode{\sphinxupquote{count\_interactions\_mutual\_by\_resource}}}{}{}
str(object=’‘) -\textgreater{} str
str(bytes\_or\_buffer{[}, encoding{[}, errors{]}{]}) -\textgreater{} str

Create a new string object from the given object. If encoding or
errors is specified, then the object must expose a data buffer
that will be decoded using the given encoding and error handler.
Otherwise, returns the result of object.\_\_str\_\_() (if defined)
or repr(object).
encoding defaults to sys.getdefaultencoding().
errors defaults to ‘strict’.

\end{fulllineitems}

\index{count\_interactions\_negative() (pypath.core.network.Network method)@\spxentry{count\_interactions\_negative()}\spxextra{pypath.core.network.Network method}}

\begin{fulllineitems}
\phantomsection\label{\detokenize{reference:pypath.core.network.Network.count_interactions_negative}}\pysiglinewithargsret{\sphinxbfcode{\sphinxupquote{count\_interactions\_negative}}}{}{}
str(object=’‘) -\textgreater{} str
str(bytes\_or\_buffer{[}, encoding{[}, errors{]}{]}) -\textgreater{} str

Create a new string object from the given object. If encoding or
errors is specified, then the object must expose a data buffer
that will be decoded using the given encoding and error handler.
Otherwise, returns the result of object.\_\_str\_\_() (if defined)
or repr(object).
encoding defaults to sys.getdefaultencoding().
errors defaults to ‘strict’.

\end{fulllineitems}

\index{count\_interactions\_negative\_by\_data\_model() (pypath.core.network.Network method)@\spxentry{count\_interactions\_negative\_by\_data\_model()}\spxextra{pypath.core.network.Network method}}

\begin{fulllineitems}
\phantomsection\label{\detokenize{reference:pypath.core.network.Network.count_interactions_negative_by_data_model}}\pysiglinewithargsret{\sphinxbfcode{\sphinxupquote{count\_interactions\_negative\_by\_data\_model}}}{}{}
str(object=’‘) -\textgreater{} str
str(bytes\_or\_buffer{[}, encoding{[}, errors{]}{]}) -\textgreater{} str

Create a new string object from the given object. If encoding or
errors is specified, then the object must expose a data buffer
that will be decoded using the given encoding and error handler.
Otherwise, returns the result of object.\_\_str\_\_() (if defined)
or repr(object).
encoding defaults to sys.getdefaultencoding().
errors defaults to ‘strict’.

\end{fulllineitems}

\index{count\_interactions\_negative\_by\_interaction\_type() (pypath.core.network.Network method)@\spxentry{count\_interactions\_negative\_by\_interaction\_type()}\spxextra{pypath.core.network.Network method}}

\begin{fulllineitems}
\phantomsection\label{\detokenize{reference:pypath.core.network.Network.count_interactions_negative_by_interaction_type}}\pysiglinewithargsret{\sphinxbfcode{\sphinxupquote{count\_interactions\_negative\_by\_interaction\_type}}}{}{}
str(object=’‘) -\textgreater{} str
str(bytes\_or\_buffer{[}, encoding{[}, errors{]}{]}) -\textgreater{} str

Create a new string object from the given object. If encoding or
errors is specified, then the object must expose a data buffer
that will be decoded using the given encoding and error handler.
Otherwise, returns the result of object.\_\_str\_\_() (if defined)
or repr(object).
encoding defaults to sys.getdefaultencoding().
errors defaults to ‘strict’.

\end{fulllineitems}

\index{count\_interactions\_negative\_by\_interaction\_type\_and\_data\_model() (pypath.core.network.Network method)@\spxentry{count\_interactions\_negative\_by\_interaction\_type\_and\_data\_model()}\spxextra{pypath.core.network.Network method}}

\begin{fulllineitems}
\phantomsection\label{\detokenize{reference:pypath.core.network.Network.count_interactions_negative_by_interaction_type_and_data_model}}\pysiglinewithargsret{\sphinxbfcode{\sphinxupquote{count\_interactions\_negative\_by\_interaction\_type\_and\_data\_model}}}{}{}
str(object=’‘) -\textgreater{} str
str(bytes\_or\_buffer{[}, encoding{[}, errors{]}{]}) -\textgreater{} str

Create a new string object from the given object. If encoding or
errors is specified, then the object must expose a data buffer
that will be decoded using the given encoding and error handler.
Otherwise, returns the result of object.\_\_str\_\_() (if defined)
or repr(object).
encoding defaults to sys.getdefaultencoding().
errors defaults to ‘strict’.

\end{fulllineitems}

\index{count\_interactions\_negative\_by\_interaction\_type\_and\_data\_model\_and\_resource() (pypath.core.network.Network method)@\spxentry{count\_interactions\_negative\_by\_interaction\_type\_and\_data\_model\_and\_resource()}\spxextra{pypath.core.network.Network method}}

\begin{fulllineitems}
\phantomsection\label{\detokenize{reference:pypath.core.network.Network.count_interactions_negative_by_interaction_type_and_data_model_and_resource}}\pysiglinewithargsret{\sphinxbfcode{\sphinxupquote{count\_interactions\_negative\_by\_interaction\_type\_and\_data\_model\_and\_resource}}}{}{}
str(object=’‘) -\textgreater{} str
str(bytes\_or\_buffer{[}, encoding{[}, errors{]}{]}) -\textgreater{} str

Create a new string object from the given object. If encoding or
errors is specified, then the object must expose a data buffer
that will be decoded using the given encoding and error handler.
Otherwise, returns the result of object.\_\_str\_\_() (if defined)
or repr(object).
encoding defaults to sys.getdefaultencoding().
errors defaults to ‘strict’.

\end{fulllineitems}

\index{count\_interactions\_negative\_by\_reference() (pypath.core.network.Network method)@\spxentry{count\_interactions\_negative\_by\_reference()}\spxextra{pypath.core.network.Network method}}

\begin{fulllineitems}
\phantomsection\label{\detokenize{reference:pypath.core.network.Network.count_interactions_negative_by_reference}}\pysiglinewithargsret{\sphinxbfcode{\sphinxupquote{count\_interactions\_negative\_by\_reference}}}{}{}
str(object=’‘) -\textgreater{} str
str(bytes\_or\_buffer{[}, encoding{[}, errors{]}{]}) -\textgreater{} str

Create a new string object from the given object. If encoding or
errors is specified, then the object must expose a data buffer
that will be decoded using the given encoding and error handler.
Otherwise, returns the result of object.\_\_str\_\_() (if defined)
or repr(object).
encoding defaults to sys.getdefaultencoding().
errors defaults to ‘strict’.

\end{fulllineitems}

\index{count\_interactions\_negative\_by\_resource() (pypath.core.network.Network method)@\spxentry{count\_interactions\_negative\_by\_resource()}\spxextra{pypath.core.network.Network method}}

\begin{fulllineitems}
\phantomsection\label{\detokenize{reference:pypath.core.network.Network.count_interactions_negative_by_resource}}\pysiglinewithargsret{\sphinxbfcode{\sphinxupquote{count\_interactions\_negative\_by\_resource}}}{}{}
str(object=’‘) -\textgreater{} str
str(bytes\_or\_buffer{[}, encoding{[}, errors{]}{]}) -\textgreater{} str

Create a new string object from the given object. If encoding or
errors is specified, then the object must expose a data buffer
that will be decoded using the given encoding and error handler.
Otherwise, returns the result of object.\_\_str\_\_() (if defined)
or repr(object).
encoding defaults to sys.getdefaultencoding().
errors defaults to ‘strict’.

\end{fulllineitems}

\index{count\_interactions\_non\_directed() (pypath.core.network.Network method)@\spxentry{count\_interactions\_non\_directed()}\spxextra{pypath.core.network.Network method}}

\begin{fulllineitems}
\phantomsection\label{\detokenize{reference:pypath.core.network.Network.count_interactions_non_directed}}\pysiglinewithargsret{\sphinxbfcode{\sphinxupquote{count\_interactions\_non\_directed}}}{}{}
str(object=’‘) -\textgreater{} str
str(bytes\_or\_buffer{[}, encoding{[}, errors{]}{]}) -\textgreater{} str

Create a new string object from the given object. If encoding or
errors is specified, then the object must expose a data buffer
that will be decoded using the given encoding and error handler.
Otherwise, returns the result of object.\_\_str\_\_() (if defined)
or repr(object).
encoding defaults to sys.getdefaultencoding().
errors defaults to ‘strict’.

\end{fulllineitems}

\index{count\_interactions\_non\_directed\_0() (pypath.core.network.Network method)@\spxentry{count\_interactions\_non\_directed\_0()}\spxextra{pypath.core.network.Network method}}

\begin{fulllineitems}
\phantomsection\label{\detokenize{reference:pypath.core.network.Network.count_interactions_non_directed_0}}\pysiglinewithargsret{\sphinxbfcode{\sphinxupquote{count\_interactions\_non\_directed\_0}}}{}{}
str(object=’‘) -\textgreater{} str
str(bytes\_or\_buffer{[}, encoding{[}, errors{]}{]}) -\textgreater{} str

Create a new string object from the given object. If encoding or
errors is specified, then the object must expose a data buffer
that will be decoded using the given encoding and error handler.
Otherwise, returns the result of object.\_\_str\_\_() (if defined)
or repr(object).
encoding defaults to sys.getdefaultencoding().
errors defaults to ‘strict’.

\end{fulllineitems}

\index{count\_interactions\_non\_directed\_0\_by\_data\_model() (pypath.core.network.Network method)@\spxentry{count\_interactions\_non\_directed\_0\_by\_data\_model()}\spxextra{pypath.core.network.Network method}}

\begin{fulllineitems}
\phantomsection\label{\detokenize{reference:pypath.core.network.Network.count_interactions_non_directed_0_by_data_model}}\pysiglinewithargsret{\sphinxbfcode{\sphinxupquote{count\_interactions\_non\_directed\_0\_by\_data\_model}}}{}{}
str(object=’‘) -\textgreater{} str
str(bytes\_or\_buffer{[}, encoding{[}, errors{]}{]}) -\textgreater{} str

Create a new string object from the given object. If encoding or
errors is specified, then the object must expose a data buffer
that will be decoded using the given encoding and error handler.
Otherwise, returns the result of object.\_\_str\_\_() (if defined)
or repr(object).
encoding defaults to sys.getdefaultencoding().
errors defaults to ‘strict’.

\end{fulllineitems}

\index{count\_interactions\_non\_directed\_0\_by\_interaction\_type() (pypath.core.network.Network method)@\spxentry{count\_interactions\_non\_directed\_0\_by\_interaction\_type()}\spxextra{pypath.core.network.Network method}}

\begin{fulllineitems}
\phantomsection\label{\detokenize{reference:pypath.core.network.Network.count_interactions_non_directed_0_by_interaction_type}}\pysiglinewithargsret{\sphinxbfcode{\sphinxupquote{count\_interactions\_non\_directed\_0\_by\_interaction\_type}}}{}{}
str(object=’‘) -\textgreater{} str
str(bytes\_or\_buffer{[}, encoding{[}, errors{]}{]}) -\textgreater{} str

Create a new string object from the given object. If encoding or
errors is specified, then the object must expose a data buffer
that will be decoded using the given encoding and error handler.
Otherwise, returns the result of object.\_\_str\_\_() (if defined)
or repr(object).
encoding defaults to sys.getdefaultencoding().
errors defaults to ‘strict’.

\end{fulllineitems}

\index{count\_interactions\_non\_directed\_0\_by\_interaction\_type\_and\_data\_model() (pypath.core.network.Network method)@\spxentry{count\_interactions\_non\_directed\_0\_by\_interaction\_type\_and\_data\_model()}\spxextra{pypath.core.network.Network method}}

\begin{fulllineitems}
\phantomsection\label{\detokenize{reference:pypath.core.network.Network.count_interactions_non_directed_0_by_interaction_type_and_data_model}}\pysiglinewithargsret{\sphinxbfcode{\sphinxupquote{count\_interactions\_non\_directed\_0\_by\_interaction\_type\_and\_data\_model}}}{}{}
str(object=’‘) -\textgreater{} str
str(bytes\_or\_buffer{[}, encoding{[}, errors{]}{]}) -\textgreater{} str

Create a new string object from the given object. If encoding or
errors is specified, then the object must expose a data buffer
that will be decoded using the given encoding and error handler.
Otherwise, returns the result of object.\_\_str\_\_() (if defined)
or repr(object).
encoding defaults to sys.getdefaultencoding().
errors defaults to ‘strict’.

\end{fulllineitems}

\index{count\_interactions\_non\_directed\_0\_by\_interaction\_type\_and\_data\_model\_and\_resource() (pypath.core.network.Network method)@\spxentry{count\_interactions\_non\_directed\_0\_by\_interaction\_type\_and\_data\_model\_and\_resource()}\spxextra{pypath.core.network.Network method}}

\begin{fulllineitems}
\phantomsection\label{\detokenize{reference:pypath.core.network.Network.count_interactions_non_directed_0_by_interaction_type_and_data_model_and_resource}}\pysiglinewithargsret{\sphinxbfcode{\sphinxupquote{count\_interactions\_non\_directed\_0\_by\_interaction\_type\_and\_data\_model\_and\_resource}}}{}{}
str(object=’‘) -\textgreater{} str
str(bytes\_or\_buffer{[}, encoding{[}, errors{]}{]}) -\textgreater{} str

Create a new string object from the given object. If encoding or
errors is specified, then the object must expose a data buffer
that will be decoded using the given encoding and error handler.
Otherwise, returns the result of object.\_\_str\_\_() (if defined)
or repr(object).
encoding defaults to sys.getdefaultencoding().
errors defaults to ‘strict’.

\end{fulllineitems}

\index{count\_interactions\_non\_directed\_0\_by\_reference() (pypath.core.network.Network method)@\spxentry{count\_interactions\_non\_directed\_0\_by\_reference()}\spxextra{pypath.core.network.Network method}}

\begin{fulllineitems}
\phantomsection\label{\detokenize{reference:pypath.core.network.Network.count_interactions_non_directed_0_by_reference}}\pysiglinewithargsret{\sphinxbfcode{\sphinxupquote{count\_interactions\_non\_directed\_0\_by\_reference}}}{}{}
str(object=’‘) -\textgreater{} str
str(bytes\_or\_buffer{[}, encoding{[}, errors{]}{]}) -\textgreater{} str

Create a new string object from the given object. If encoding or
errors is specified, then the object must expose a data buffer
that will be decoded using the given encoding and error handler.
Otherwise, returns the result of object.\_\_str\_\_() (if defined)
or repr(object).
encoding defaults to sys.getdefaultencoding().
errors defaults to ‘strict’.

\end{fulllineitems}

\index{count\_interactions\_non\_directed\_0\_by\_resource() (pypath.core.network.Network method)@\spxentry{count\_interactions\_non\_directed\_0\_by\_resource()}\spxextra{pypath.core.network.Network method}}

\begin{fulllineitems}
\phantomsection\label{\detokenize{reference:pypath.core.network.Network.count_interactions_non_directed_0_by_resource}}\pysiglinewithargsret{\sphinxbfcode{\sphinxupquote{count\_interactions\_non\_directed\_0\_by\_resource}}}{}{}
str(object=’‘) -\textgreater{} str
str(bytes\_or\_buffer{[}, encoding{[}, errors{]}{]}) -\textgreater{} str

Create a new string object from the given object. If encoding or
errors is specified, then the object must expose a data buffer
that will be decoded using the given encoding and error handler.
Otherwise, returns the result of object.\_\_str\_\_() (if defined)
or repr(object).
encoding defaults to sys.getdefaultencoding().
errors defaults to ‘strict’.

\end{fulllineitems}

\index{count\_interactions\_non\_directed\_by\_data\_model() (pypath.core.network.Network method)@\spxentry{count\_interactions\_non\_directed\_by\_data\_model()}\spxextra{pypath.core.network.Network method}}

\begin{fulllineitems}
\phantomsection\label{\detokenize{reference:pypath.core.network.Network.count_interactions_non_directed_by_data_model}}\pysiglinewithargsret{\sphinxbfcode{\sphinxupquote{count\_interactions\_non\_directed\_by\_data\_model}}}{}{}
str(object=’‘) -\textgreater{} str
str(bytes\_or\_buffer{[}, encoding{[}, errors{]}{]}) -\textgreater{} str

Create a new string object from the given object. If encoding or
errors is specified, then the object must expose a data buffer
that will be decoded using the given encoding and error handler.
Otherwise, returns the result of object.\_\_str\_\_() (if defined)
or repr(object).
encoding defaults to sys.getdefaultencoding().
errors defaults to ‘strict’.

\end{fulllineitems}

\index{count\_interactions\_non\_directed\_by\_interaction\_type() (pypath.core.network.Network method)@\spxentry{count\_interactions\_non\_directed\_by\_interaction\_type()}\spxextra{pypath.core.network.Network method}}

\begin{fulllineitems}
\phantomsection\label{\detokenize{reference:pypath.core.network.Network.count_interactions_non_directed_by_interaction_type}}\pysiglinewithargsret{\sphinxbfcode{\sphinxupquote{count\_interactions\_non\_directed\_by\_interaction\_type}}}{}{}
str(object=’‘) -\textgreater{} str
str(bytes\_or\_buffer{[}, encoding{[}, errors{]}{]}) -\textgreater{} str

Create a new string object from the given object. If encoding or
errors is specified, then the object must expose a data buffer
that will be decoded using the given encoding and error handler.
Otherwise, returns the result of object.\_\_str\_\_() (if defined)
or repr(object).
encoding defaults to sys.getdefaultencoding().
errors defaults to ‘strict’.

\end{fulllineitems}

\index{count\_interactions\_non\_directed\_by\_interaction\_type\_and\_data\_model() (pypath.core.network.Network method)@\spxentry{count\_interactions\_non\_directed\_by\_interaction\_type\_and\_data\_model()}\spxextra{pypath.core.network.Network method}}

\begin{fulllineitems}
\phantomsection\label{\detokenize{reference:pypath.core.network.Network.count_interactions_non_directed_by_interaction_type_and_data_model}}\pysiglinewithargsret{\sphinxbfcode{\sphinxupquote{count\_interactions\_non\_directed\_by\_interaction\_type\_and\_data\_model}}}{}{}
str(object=’‘) -\textgreater{} str
str(bytes\_or\_buffer{[}, encoding{[}, errors{]}{]}) -\textgreater{} str

Create a new string object from the given object. If encoding or
errors is specified, then the object must expose a data buffer
that will be decoded using the given encoding and error handler.
Otherwise, returns the result of object.\_\_str\_\_() (if defined)
or repr(object).
encoding defaults to sys.getdefaultencoding().
errors defaults to ‘strict’.

\end{fulllineitems}

\index{count\_interactions\_non\_directed\_by\_interaction\_type\_and\_data\_model\_and\_resource() (pypath.core.network.Network method)@\spxentry{count\_interactions\_non\_directed\_by\_interaction\_type\_and\_data\_model\_and\_resource()}\spxextra{pypath.core.network.Network method}}

\begin{fulllineitems}
\phantomsection\label{\detokenize{reference:pypath.core.network.Network.count_interactions_non_directed_by_interaction_type_and_data_model_and_resource}}\pysiglinewithargsret{\sphinxbfcode{\sphinxupquote{count\_interactions\_non\_directed\_by\_interaction\_type\_and\_data\_model\_and\_resource}}}{}{}
str(object=’‘) -\textgreater{} str
str(bytes\_or\_buffer{[}, encoding{[}, errors{]}{]}) -\textgreater{} str

Create a new string object from the given object. If encoding or
errors is specified, then the object must expose a data buffer
that will be decoded using the given encoding and error handler.
Otherwise, returns the result of object.\_\_str\_\_() (if defined)
or repr(object).
encoding defaults to sys.getdefaultencoding().
errors defaults to ‘strict’.

\end{fulllineitems}

\index{count\_interactions\_non\_directed\_by\_reference() (pypath.core.network.Network method)@\spxentry{count\_interactions\_non\_directed\_by\_reference()}\spxextra{pypath.core.network.Network method}}

\begin{fulllineitems}
\phantomsection\label{\detokenize{reference:pypath.core.network.Network.count_interactions_non_directed_by_reference}}\pysiglinewithargsret{\sphinxbfcode{\sphinxupquote{count\_interactions\_non\_directed\_by\_reference}}}{}{}
str(object=’‘) -\textgreater{} str
str(bytes\_or\_buffer{[}, encoding{[}, errors{]}{]}) -\textgreater{} str

Create a new string object from the given object. If encoding or
errors is specified, then the object must expose a data buffer
that will be decoded using the given encoding and error handler.
Otherwise, returns the result of object.\_\_str\_\_() (if defined)
or repr(object).
encoding defaults to sys.getdefaultencoding().
errors defaults to ‘strict’.

\end{fulllineitems}

\index{count\_interactions\_non\_directed\_by\_resource() (pypath.core.network.Network method)@\spxentry{count\_interactions\_non\_directed\_by\_resource()}\spxextra{pypath.core.network.Network method}}

\begin{fulllineitems}
\phantomsection\label{\detokenize{reference:pypath.core.network.Network.count_interactions_non_directed_by_resource}}\pysiglinewithargsret{\sphinxbfcode{\sphinxupquote{count\_interactions\_non\_directed\_by\_resource}}}{}{}
str(object=’‘) -\textgreater{} str
str(bytes\_or\_buffer{[}, encoding{[}, errors{]}{]}) -\textgreater{} str

Create a new string object from the given object. If encoding or
errors is specified, then the object must expose a data buffer
that will be decoded using the given encoding and error handler.
Otherwise, returns the result of object.\_\_str\_\_() (if defined)
or repr(object).
encoding defaults to sys.getdefaultencoding().
errors defaults to ‘strict’.

\end{fulllineitems}

\index{count\_interactions\_positive() (pypath.core.network.Network method)@\spxentry{count\_interactions\_positive()}\spxextra{pypath.core.network.Network method}}

\begin{fulllineitems}
\phantomsection\label{\detokenize{reference:pypath.core.network.Network.count_interactions_positive}}\pysiglinewithargsret{\sphinxbfcode{\sphinxupquote{count\_interactions\_positive}}}{}{}
str(object=’‘) -\textgreater{} str
str(bytes\_or\_buffer{[}, encoding{[}, errors{]}{]}) -\textgreater{} str

Create a new string object from the given object. If encoding or
errors is specified, then the object must expose a data buffer
that will be decoded using the given encoding and error handler.
Otherwise, returns the result of object.\_\_str\_\_() (if defined)
or repr(object).
encoding defaults to sys.getdefaultencoding().
errors defaults to ‘strict’.

\end{fulllineitems}

\index{count\_interactions\_positive\_by\_data\_model() (pypath.core.network.Network method)@\spxentry{count\_interactions\_positive\_by\_data\_model()}\spxextra{pypath.core.network.Network method}}

\begin{fulllineitems}
\phantomsection\label{\detokenize{reference:pypath.core.network.Network.count_interactions_positive_by_data_model}}\pysiglinewithargsret{\sphinxbfcode{\sphinxupquote{count\_interactions\_positive\_by\_data\_model}}}{}{}
str(object=’‘) -\textgreater{} str
str(bytes\_or\_buffer{[}, encoding{[}, errors{]}{]}) -\textgreater{} str

Create a new string object from the given object. If encoding or
errors is specified, then the object must expose a data buffer
that will be decoded using the given encoding and error handler.
Otherwise, returns the result of object.\_\_str\_\_() (if defined)
or repr(object).
encoding defaults to sys.getdefaultencoding().
errors defaults to ‘strict’.

\end{fulllineitems}

\index{count\_interactions\_positive\_by\_interaction\_type() (pypath.core.network.Network method)@\spxentry{count\_interactions\_positive\_by\_interaction\_type()}\spxextra{pypath.core.network.Network method}}

\begin{fulllineitems}
\phantomsection\label{\detokenize{reference:pypath.core.network.Network.count_interactions_positive_by_interaction_type}}\pysiglinewithargsret{\sphinxbfcode{\sphinxupquote{count\_interactions\_positive\_by\_interaction\_type}}}{}{}
str(object=’‘) -\textgreater{} str
str(bytes\_or\_buffer{[}, encoding{[}, errors{]}{]}) -\textgreater{} str

Create a new string object from the given object. If encoding or
errors is specified, then the object must expose a data buffer
that will be decoded using the given encoding and error handler.
Otherwise, returns the result of object.\_\_str\_\_() (if defined)
or repr(object).
encoding defaults to sys.getdefaultencoding().
errors defaults to ‘strict’.

\end{fulllineitems}

\index{count\_interactions\_positive\_by\_interaction\_type\_and\_data\_model() (pypath.core.network.Network method)@\spxentry{count\_interactions\_positive\_by\_interaction\_type\_and\_data\_model()}\spxextra{pypath.core.network.Network method}}

\begin{fulllineitems}
\phantomsection\label{\detokenize{reference:pypath.core.network.Network.count_interactions_positive_by_interaction_type_and_data_model}}\pysiglinewithargsret{\sphinxbfcode{\sphinxupquote{count\_interactions\_positive\_by\_interaction\_type\_and\_data\_model}}}{}{}
str(object=’‘) -\textgreater{} str
str(bytes\_or\_buffer{[}, encoding{[}, errors{]}{]}) -\textgreater{} str

Create a new string object from the given object. If encoding or
errors is specified, then the object must expose a data buffer
that will be decoded using the given encoding and error handler.
Otherwise, returns the result of object.\_\_str\_\_() (if defined)
or repr(object).
encoding defaults to sys.getdefaultencoding().
errors defaults to ‘strict’.

\end{fulllineitems}

\index{count\_interactions\_positive\_by\_interaction\_type\_and\_data\_model\_and\_resource() (pypath.core.network.Network method)@\spxentry{count\_interactions\_positive\_by\_interaction\_type\_and\_data\_model\_and\_resource()}\spxextra{pypath.core.network.Network method}}

\begin{fulllineitems}
\phantomsection\label{\detokenize{reference:pypath.core.network.Network.count_interactions_positive_by_interaction_type_and_data_model_and_resource}}\pysiglinewithargsret{\sphinxbfcode{\sphinxupquote{count\_interactions\_positive\_by\_interaction\_type\_and\_data\_model\_and\_resource}}}{}{}
str(object=’‘) -\textgreater{} str
str(bytes\_or\_buffer{[}, encoding{[}, errors{]}{]}) -\textgreater{} str

Create a new string object from the given object. If encoding or
errors is specified, then the object must expose a data buffer
that will be decoded using the given encoding and error handler.
Otherwise, returns the result of object.\_\_str\_\_() (if defined)
or repr(object).
encoding defaults to sys.getdefaultencoding().
errors defaults to ‘strict’.

\end{fulllineitems}

\index{count\_interactions\_positive\_by\_reference() (pypath.core.network.Network method)@\spxentry{count\_interactions\_positive\_by\_reference()}\spxextra{pypath.core.network.Network method}}

\begin{fulllineitems}
\phantomsection\label{\detokenize{reference:pypath.core.network.Network.count_interactions_positive_by_reference}}\pysiglinewithargsret{\sphinxbfcode{\sphinxupquote{count\_interactions\_positive\_by\_reference}}}{}{}
str(object=’‘) -\textgreater{} str
str(bytes\_or\_buffer{[}, encoding{[}, errors{]}{]}) -\textgreater{} str

Create a new string object from the given object. If encoding or
errors is specified, then the object must expose a data buffer
that will be decoded using the given encoding and error handler.
Otherwise, returns the result of object.\_\_str\_\_() (if defined)
or repr(object).
encoding defaults to sys.getdefaultencoding().
errors defaults to ‘strict’.

\end{fulllineitems}

\index{count\_interactions\_positive\_by\_resource() (pypath.core.network.Network method)@\spxentry{count\_interactions\_positive\_by\_resource()}\spxextra{pypath.core.network.Network method}}

\begin{fulllineitems}
\phantomsection\label{\detokenize{reference:pypath.core.network.Network.count_interactions_positive_by_resource}}\pysiglinewithargsret{\sphinxbfcode{\sphinxupquote{count\_interactions\_positive\_by\_resource}}}{}{}
str(object=’‘) -\textgreater{} str
str(bytes\_or\_buffer{[}, encoding{[}, errors{]}{]}) -\textgreater{} str

Create a new string object from the given object. If encoding or
errors is specified, then the object must expose a data buffer
that will be decoded using the given encoding and error handler.
Otherwise, returns the result of object.\_\_str\_\_() (if defined)
or repr(object).
encoding defaults to sys.getdefaultencoding().
errors defaults to ‘strict’.

\end{fulllineitems}

\index{count\_interactions\_signed() (pypath.core.network.Network method)@\spxentry{count\_interactions\_signed()}\spxextra{pypath.core.network.Network method}}

\begin{fulllineitems}
\phantomsection\label{\detokenize{reference:pypath.core.network.Network.count_interactions_signed}}\pysiglinewithargsret{\sphinxbfcode{\sphinxupquote{count\_interactions\_signed}}}{}{}
str(object=’‘) -\textgreater{} str
str(bytes\_or\_buffer{[}, encoding{[}, errors{]}{]}) -\textgreater{} str

Create a new string object from the given object. If encoding or
errors is specified, then the object must expose a data buffer
that will be decoded using the given encoding and error handler.
Otherwise, returns the result of object.\_\_str\_\_() (if defined)
or repr(object).
encoding defaults to sys.getdefaultencoding().
errors defaults to ‘strict’.

\end{fulllineitems}

\index{count\_interactions\_signed\_by\_data\_model() (pypath.core.network.Network method)@\spxentry{count\_interactions\_signed\_by\_data\_model()}\spxextra{pypath.core.network.Network method}}

\begin{fulllineitems}
\phantomsection\label{\detokenize{reference:pypath.core.network.Network.count_interactions_signed_by_data_model}}\pysiglinewithargsret{\sphinxbfcode{\sphinxupquote{count\_interactions\_signed\_by\_data\_model}}}{}{}
str(object=’‘) -\textgreater{} str
str(bytes\_or\_buffer{[}, encoding{[}, errors{]}{]}) -\textgreater{} str

Create a new string object from the given object. If encoding or
errors is specified, then the object must expose a data buffer
that will be decoded using the given encoding and error handler.
Otherwise, returns the result of object.\_\_str\_\_() (if defined)
or repr(object).
encoding defaults to sys.getdefaultencoding().
errors defaults to ‘strict’.

\end{fulllineitems}

\index{count\_interactions\_signed\_by\_interaction\_type() (pypath.core.network.Network method)@\spxentry{count\_interactions\_signed\_by\_interaction\_type()}\spxextra{pypath.core.network.Network method}}

\begin{fulllineitems}
\phantomsection\label{\detokenize{reference:pypath.core.network.Network.count_interactions_signed_by_interaction_type}}\pysiglinewithargsret{\sphinxbfcode{\sphinxupquote{count\_interactions\_signed\_by\_interaction\_type}}}{}{}
str(object=’‘) -\textgreater{} str
str(bytes\_or\_buffer{[}, encoding{[}, errors{]}{]}) -\textgreater{} str

Create a new string object from the given object. If encoding or
errors is specified, then the object must expose a data buffer
that will be decoded using the given encoding and error handler.
Otherwise, returns the result of object.\_\_str\_\_() (if defined)
or repr(object).
encoding defaults to sys.getdefaultencoding().
errors defaults to ‘strict’.

\end{fulllineitems}

\index{count\_interactions\_signed\_by\_interaction\_type\_and\_data\_model() (pypath.core.network.Network method)@\spxentry{count\_interactions\_signed\_by\_interaction\_type\_and\_data\_model()}\spxextra{pypath.core.network.Network method}}

\begin{fulllineitems}
\phantomsection\label{\detokenize{reference:pypath.core.network.Network.count_interactions_signed_by_interaction_type_and_data_model}}\pysiglinewithargsret{\sphinxbfcode{\sphinxupquote{count\_interactions\_signed\_by\_interaction\_type\_and\_data\_model}}}{}{}
str(object=’‘) -\textgreater{} str
str(bytes\_or\_buffer{[}, encoding{[}, errors{]}{]}) -\textgreater{} str

Create a new string object from the given object. If encoding or
errors is specified, then the object must expose a data buffer
that will be decoded using the given encoding and error handler.
Otherwise, returns the result of object.\_\_str\_\_() (if defined)
or repr(object).
encoding defaults to sys.getdefaultencoding().
errors defaults to ‘strict’.

\end{fulllineitems}

\index{count\_interactions\_signed\_by\_interaction\_type\_and\_data\_model\_and\_resource() (pypath.core.network.Network method)@\spxentry{count\_interactions\_signed\_by\_interaction\_type\_and\_data\_model\_and\_resource()}\spxextra{pypath.core.network.Network method}}

\begin{fulllineitems}
\phantomsection\label{\detokenize{reference:pypath.core.network.Network.count_interactions_signed_by_interaction_type_and_data_model_and_resource}}\pysiglinewithargsret{\sphinxbfcode{\sphinxupquote{count\_interactions\_signed\_by\_interaction\_type\_and\_data\_model\_and\_resource}}}{}{}
str(object=’‘) -\textgreater{} str
str(bytes\_or\_buffer{[}, encoding{[}, errors{]}{]}) -\textgreater{} str

Create a new string object from the given object. If encoding or
errors is specified, then the object must expose a data buffer
that will be decoded using the given encoding and error handler.
Otherwise, returns the result of object.\_\_str\_\_() (if defined)
or repr(object).
encoding defaults to sys.getdefaultencoding().
errors defaults to ‘strict’.

\end{fulllineitems}

\index{count\_interactions\_signed\_by\_reference() (pypath.core.network.Network method)@\spxentry{count\_interactions\_signed\_by\_reference()}\spxextra{pypath.core.network.Network method}}

\begin{fulllineitems}
\phantomsection\label{\detokenize{reference:pypath.core.network.Network.count_interactions_signed_by_reference}}\pysiglinewithargsret{\sphinxbfcode{\sphinxupquote{count\_interactions\_signed\_by\_reference}}}{}{}
str(object=’‘) -\textgreater{} str
str(bytes\_or\_buffer{[}, encoding{[}, errors{]}{]}) -\textgreater{} str

Create a new string object from the given object. If encoding or
errors is specified, then the object must expose a data buffer
that will be decoded using the given encoding and error handler.
Otherwise, returns the result of object.\_\_str\_\_() (if defined)
or repr(object).
encoding defaults to sys.getdefaultencoding().
errors defaults to ‘strict’.

\end{fulllineitems}

\index{count\_interactions\_signed\_by\_resource() (pypath.core.network.Network method)@\spxentry{count\_interactions\_signed\_by\_resource()}\spxextra{pypath.core.network.Network method}}

\begin{fulllineitems}
\phantomsection\label{\detokenize{reference:pypath.core.network.Network.count_interactions_signed_by_resource}}\pysiglinewithargsret{\sphinxbfcode{\sphinxupquote{count\_interactions\_signed\_by\_resource}}}{}{}
str(object=’‘) -\textgreater{} str
str(bytes\_or\_buffer{[}, encoding{[}, errors{]}{]}) -\textgreater{} str

Create a new string object from the given object. If encoding or
errors is specified, then the object must expose a data buffer
that will be decoded using the given encoding and error handler.
Otherwise, returns the result of object.\_\_str\_\_() (if defined)
or repr(object).
encoding defaults to sys.getdefaultencoding().
errors defaults to ‘strict’.

\end{fulllineitems}

\index{count\_interactions\_undirected() (pypath.core.network.Network method)@\spxentry{count\_interactions\_undirected()}\spxextra{pypath.core.network.Network method}}

\begin{fulllineitems}
\phantomsection\label{\detokenize{reference:pypath.core.network.Network.count_interactions_undirected}}\pysiglinewithargsret{\sphinxbfcode{\sphinxupquote{count\_interactions\_undirected}}}{}{}
str(object=’‘) -\textgreater{} str
str(bytes\_or\_buffer{[}, encoding{[}, errors{]}{]}) -\textgreater{} str

Create a new string object from the given object. If encoding or
errors is specified, then the object must expose a data buffer
that will be decoded using the given encoding and error handler.
Otherwise, returns the result of object.\_\_str\_\_() (if defined)
or repr(object).
encoding defaults to sys.getdefaultencoding().
errors defaults to ‘strict’.

\end{fulllineitems}

\index{count\_interactions\_undirected\_0() (pypath.core.network.Network method)@\spxentry{count\_interactions\_undirected\_0()}\spxextra{pypath.core.network.Network method}}

\begin{fulllineitems}
\phantomsection\label{\detokenize{reference:pypath.core.network.Network.count_interactions_undirected_0}}\pysiglinewithargsret{\sphinxbfcode{\sphinxupquote{count\_interactions\_undirected\_0}}}{}{}
str(object=’‘) -\textgreater{} str
str(bytes\_or\_buffer{[}, encoding{[}, errors{]}{]}) -\textgreater{} str

Create a new string object from the given object. If encoding or
errors is specified, then the object must expose a data buffer
that will be decoded using the given encoding and error handler.
Otherwise, returns the result of object.\_\_str\_\_() (if defined)
or repr(object).
encoding defaults to sys.getdefaultencoding().
errors defaults to ‘strict’.

\end{fulllineitems}

\index{count\_interactions\_undirected\_0\_by\_data\_model() (pypath.core.network.Network method)@\spxentry{count\_interactions\_undirected\_0\_by\_data\_model()}\spxextra{pypath.core.network.Network method}}

\begin{fulllineitems}
\phantomsection\label{\detokenize{reference:pypath.core.network.Network.count_interactions_undirected_0_by_data_model}}\pysiglinewithargsret{\sphinxbfcode{\sphinxupquote{count\_interactions\_undirected\_0\_by\_data\_model}}}{}{}
str(object=’‘) -\textgreater{} str
str(bytes\_or\_buffer{[}, encoding{[}, errors{]}{]}) -\textgreater{} str

Create a new string object from the given object. If encoding or
errors is specified, then the object must expose a data buffer
that will be decoded using the given encoding and error handler.
Otherwise, returns the result of object.\_\_str\_\_() (if defined)
or repr(object).
encoding defaults to sys.getdefaultencoding().
errors defaults to ‘strict’.

\end{fulllineitems}

\index{count\_interactions\_undirected\_0\_by\_interaction\_type() (pypath.core.network.Network method)@\spxentry{count\_interactions\_undirected\_0\_by\_interaction\_type()}\spxextra{pypath.core.network.Network method}}

\begin{fulllineitems}
\phantomsection\label{\detokenize{reference:pypath.core.network.Network.count_interactions_undirected_0_by_interaction_type}}\pysiglinewithargsret{\sphinxbfcode{\sphinxupquote{count\_interactions\_undirected\_0\_by\_interaction\_type}}}{}{}
str(object=’‘) -\textgreater{} str
str(bytes\_or\_buffer{[}, encoding{[}, errors{]}{]}) -\textgreater{} str

Create a new string object from the given object. If encoding or
errors is specified, then the object must expose a data buffer
that will be decoded using the given encoding and error handler.
Otherwise, returns the result of object.\_\_str\_\_() (if defined)
or repr(object).
encoding defaults to sys.getdefaultencoding().
errors defaults to ‘strict’.

\end{fulllineitems}

\index{count\_interactions\_undirected\_0\_by\_interaction\_type\_and\_data\_model() (pypath.core.network.Network method)@\spxentry{count\_interactions\_undirected\_0\_by\_interaction\_type\_and\_data\_model()}\spxextra{pypath.core.network.Network method}}

\begin{fulllineitems}
\phantomsection\label{\detokenize{reference:pypath.core.network.Network.count_interactions_undirected_0_by_interaction_type_and_data_model}}\pysiglinewithargsret{\sphinxbfcode{\sphinxupquote{count\_interactions\_undirected\_0\_by\_interaction\_type\_and\_data\_model}}}{}{}
str(object=’‘) -\textgreater{} str
str(bytes\_or\_buffer{[}, encoding{[}, errors{]}{]}) -\textgreater{} str

Create a new string object from the given object. If encoding or
errors is specified, then the object must expose a data buffer
that will be decoded using the given encoding and error handler.
Otherwise, returns the result of object.\_\_str\_\_() (if defined)
or repr(object).
encoding defaults to sys.getdefaultencoding().
errors defaults to ‘strict’.

\end{fulllineitems}

\index{count\_interactions\_undirected\_0\_by\_interaction\_type\_and\_data\_model\_and\_resource() (pypath.core.network.Network method)@\spxentry{count\_interactions\_undirected\_0\_by\_interaction\_type\_and\_data\_model\_and\_resource()}\spxextra{pypath.core.network.Network method}}

\begin{fulllineitems}
\phantomsection\label{\detokenize{reference:pypath.core.network.Network.count_interactions_undirected_0_by_interaction_type_and_data_model_and_resource}}\pysiglinewithargsret{\sphinxbfcode{\sphinxupquote{count\_interactions\_undirected\_0\_by\_interaction\_type\_and\_data\_model\_and\_resource}}}{}{}
str(object=’‘) -\textgreater{} str
str(bytes\_or\_buffer{[}, encoding{[}, errors{]}{]}) -\textgreater{} str

Create a new string object from the given object. If encoding or
errors is specified, then the object must expose a data buffer
that will be decoded using the given encoding and error handler.
Otherwise, returns the result of object.\_\_str\_\_() (if defined)
or repr(object).
encoding defaults to sys.getdefaultencoding().
errors defaults to ‘strict’.

\end{fulllineitems}

\index{count\_interactions\_undirected\_0\_by\_reference() (pypath.core.network.Network method)@\spxentry{count\_interactions\_undirected\_0\_by\_reference()}\spxextra{pypath.core.network.Network method}}

\begin{fulllineitems}
\phantomsection\label{\detokenize{reference:pypath.core.network.Network.count_interactions_undirected_0_by_reference}}\pysiglinewithargsret{\sphinxbfcode{\sphinxupquote{count\_interactions\_undirected\_0\_by\_reference}}}{}{}
str(object=’‘) -\textgreater{} str
str(bytes\_or\_buffer{[}, encoding{[}, errors{]}{]}) -\textgreater{} str

Create a new string object from the given object. If encoding or
errors is specified, then the object must expose a data buffer
that will be decoded using the given encoding and error handler.
Otherwise, returns the result of object.\_\_str\_\_() (if defined)
or repr(object).
encoding defaults to sys.getdefaultencoding().
errors defaults to ‘strict’.

\end{fulllineitems}

\index{count\_interactions\_undirected\_0\_by\_resource() (pypath.core.network.Network method)@\spxentry{count\_interactions\_undirected\_0\_by\_resource()}\spxextra{pypath.core.network.Network method}}

\begin{fulllineitems}
\phantomsection\label{\detokenize{reference:pypath.core.network.Network.count_interactions_undirected_0_by_resource}}\pysiglinewithargsret{\sphinxbfcode{\sphinxupquote{count\_interactions\_undirected\_0\_by\_resource}}}{}{}
str(object=’‘) -\textgreater{} str
str(bytes\_or\_buffer{[}, encoding{[}, errors{]}{]}) -\textgreater{} str

Create a new string object from the given object. If encoding or
errors is specified, then the object must expose a data buffer
that will be decoded using the given encoding and error handler.
Otherwise, returns the result of object.\_\_str\_\_() (if defined)
or repr(object).
encoding defaults to sys.getdefaultencoding().
errors defaults to ‘strict’.

\end{fulllineitems}

\index{count\_interactions\_undirected\_by\_data\_model() (pypath.core.network.Network method)@\spxentry{count\_interactions\_undirected\_by\_data\_model()}\spxextra{pypath.core.network.Network method}}

\begin{fulllineitems}
\phantomsection\label{\detokenize{reference:pypath.core.network.Network.count_interactions_undirected_by_data_model}}\pysiglinewithargsret{\sphinxbfcode{\sphinxupquote{count\_interactions\_undirected\_by\_data\_model}}}{}{}
str(object=’‘) -\textgreater{} str
str(bytes\_or\_buffer{[}, encoding{[}, errors{]}{]}) -\textgreater{} str

Create a new string object from the given object. If encoding or
errors is specified, then the object must expose a data buffer
that will be decoded using the given encoding and error handler.
Otherwise, returns the result of object.\_\_str\_\_() (if defined)
or repr(object).
encoding defaults to sys.getdefaultencoding().
errors defaults to ‘strict’.

\end{fulllineitems}

\index{count\_interactions\_undirected\_by\_interaction\_type() (pypath.core.network.Network method)@\spxentry{count\_interactions\_undirected\_by\_interaction\_type()}\spxextra{pypath.core.network.Network method}}

\begin{fulllineitems}
\phantomsection\label{\detokenize{reference:pypath.core.network.Network.count_interactions_undirected_by_interaction_type}}\pysiglinewithargsret{\sphinxbfcode{\sphinxupquote{count\_interactions\_undirected\_by\_interaction\_type}}}{}{}
str(object=’‘) -\textgreater{} str
str(bytes\_or\_buffer{[}, encoding{[}, errors{]}{]}) -\textgreater{} str

Create a new string object from the given object. If encoding or
errors is specified, then the object must expose a data buffer
that will be decoded using the given encoding and error handler.
Otherwise, returns the result of object.\_\_str\_\_() (if defined)
or repr(object).
encoding defaults to sys.getdefaultencoding().
errors defaults to ‘strict’.

\end{fulllineitems}

\index{count\_interactions\_undirected\_by\_interaction\_type\_and\_data\_model() (pypath.core.network.Network method)@\spxentry{count\_interactions\_undirected\_by\_interaction\_type\_and\_data\_model()}\spxextra{pypath.core.network.Network method}}

\begin{fulllineitems}
\phantomsection\label{\detokenize{reference:pypath.core.network.Network.count_interactions_undirected_by_interaction_type_and_data_model}}\pysiglinewithargsret{\sphinxbfcode{\sphinxupquote{count\_interactions\_undirected\_by\_interaction\_type\_and\_data\_model}}}{}{}
str(object=’‘) -\textgreater{} str
str(bytes\_or\_buffer{[}, encoding{[}, errors{]}{]}) -\textgreater{} str

Create a new string object from the given object. If encoding or
errors is specified, then the object must expose a data buffer
that will be decoded using the given encoding and error handler.
Otherwise, returns the result of object.\_\_str\_\_() (if defined)
or repr(object).
encoding defaults to sys.getdefaultencoding().
errors defaults to ‘strict’.

\end{fulllineitems}

\index{count\_interactions\_undirected\_by\_interaction\_type\_and\_data\_model\_and\_resource() (pypath.core.network.Network method)@\spxentry{count\_interactions\_undirected\_by\_interaction\_type\_and\_data\_model\_and\_resource()}\spxextra{pypath.core.network.Network method}}

\begin{fulllineitems}
\phantomsection\label{\detokenize{reference:pypath.core.network.Network.count_interactions_undirected_by_interaction_type_and_data_model_and_resource}}\pysiglinewithargsret{\sphinxbfcode{\sphinxupquote{count\_interactions\_undirected\_by\_interaction\_type\_and\_data\_model\_and\_resource}}}{}{}
str(object=’‘) -\textgreater{} str
str(bytes\_or\_buffer{[}, encoding{[}, errors{]}{]}) -\textgreater{} str

Create a new string object from the given object. If encoding or
errors is specified, then the object must expose a data buffer
that will be decoded using the given encoding and error handler.
Otherwise, returns the result of object.\_\_str\_\_() (if defined)
or repr(object).
encoding defaults to sys.getdefaultencoding().
errors defaults to ‘strict’.

\end{fulllineitems}

\index{count\_interactions\_undirected\_by\_reference() (pypath.core.network.Network method)@\spxentry{count\_interactions\_undirected\_by\_reference()}\spxextra{pypath.core.network.Network method}}

\begin{fulllineitems}
\phantomsection\label{\detokenize{reference:pypath.core.network.Network.count_interactions_undirected_by_reference}}\pysiglinewithargsret{\sphinxbfcode{\sphinxupquote{count\_interactions\_undirected\_by\_reference}}}{}{}
str(object=’‘) -\textgreater{} str
str(bytes\_or\_buffer{[}, encoding{[}, errors{]}{]}) -\textgreater{} str

Create a new string object from the given object. If encoding or
errors is specified, then the object must expose a data buffer
that will be decoded using the given encoding and error handler.
Otherwise, returns the result of object.\_\_str\_\_() (if defined)
or repr(object).
encoding defaults to sys.getdefaultencoding().
errors defaults to ‘strict’.

\end{fulllineitems}

\index{count\_interactions\_undirected\_by\_resource() (pypath.core.network.Network method)@\spxentry{count\_interactions\_undirected\_by\_resource()}\spxextra{pypath.core.network.Network method}}

\begin{fulllineitems}
\phantomsection\label{\detokenize{reference:pypath.core.network.Network.count_interactions_undirected_by_resource}}\pysiglinewithargsret{\sphinxbfcode{\sphinxupquote{count\_interactions\_undirected\_by\_resource}}}{}{}
str(object=’‘) -\textgreater{} str
str(bytes\_or\_buffer{[}, encoding{[}, errors{]}{]}) -\textgreater{} str

Create a new string object from the given object. If encoding or
errors is specified, then the object must expose a data buffer
that will be decoded using the given encoding and error handler.
Otherwise, returns the result of object.\_\_str\_\_() (if defined)
or repr(object).
encoding defaults to sys.getdefaultencoding().
errors defaults to ‘strict’.

\end{fulllineitems}

\index{count\_labels() (pypath.core.network.Network method)@\spxentry{count\_labels()}\spxextra{pypath.core.network.Network method}}

\begin{fulllineitems}
\phantomsection\label{\detokenize{reference:pypath.core.network.Network.count_labels}}\pysiglinewithargsret{\sphinxbfcode{\sphinxupquote{count\_labels}}}{}{}
str(object=’‘) -\textgreater{} str
str(bytes\_or\_buffer{[}, encoding{[}, errors{]}{]}) -\textgreater{} str

Create a new string object from the given object. If encoding or
errors is specified, then the object must expose a data buffer
that will be decoded using the given encoding and error handler.
Otherwise, returns the result of object.\_\_str\_\_() (if defined)
or repr(object).
encoding defaults to sys.getdefaultencoding().
errors defaults to ‘strict’.

\end{fulllineitems}

\index{count\_labels\_by\_data\_model() (pypath.core.network.Network method)@\spxentry{count\_labels\_by\_data\_model()}\spxextra{pypath.core.network.Network method}}

\begin{fulllineitems}
\phantomsection\label{\detokenize{reference:pypath.core.network.Network.count_labels_by_data_model}}\pysiglinewithargsret{\sphinxbfcode{\sphinxupquote{count\_labels\_by\_data\_model}}}{}{}
str(object=’‘) -\textgreater{} str
str(bytes\_or\_buffer{[}, encoding{[}, errors{]}{]}) -\textgreater{} str

Create a new string object from the given object. If encoding or
errors is specified, then the object must expose a data buffer
that will be decoded using the given encoding and error handler.
Otherwise, returns the result of object.\_\_str\_\_() (if defined)
or repr(object).
encoding defaults to sys.getdefaultencoding().
errors defaults to ‘strict’.

\end{fulllineitems}

\index{count\_labels\_by\_interaction\_type() (pypath.core.network.Network method)@\spxentry{count\_labels\_by\_interaction\_type()}\spxextra{pypath.core.network.Network method}}

\begin{fulllineitems}
\phantomsection\label{\detokenize{reference:pypath.core.network.Network.count_labels_by_interaction_type}}\pysiglinewithargsret{\sphinxbfcode{\sphinxupquote{count\_labels\_by\_interaction\_type}}}{}{}
str(object=’‘) -\textgreater{} str
str(bytes\_or\_buffer{[}, encoding{[}, errors{]}{]}) -\textgreater{} str

Create a new string object from the given object. If encoding or
errors is specified, then the object must expose a data buffer
that will be decoded using the given encoding and error handler.
Otherwise, returns the result of object.\_\_str\_\_() (if defined)
or repr(object).
encoding defaults to sys.getdefaultencoding().
errors defaults to ‘strict’.

\end{fulllineitems}

\index{count\_labels\_by\_interaction\_type\_and\_data\_model() (pypath.core.network.Network method)@\spxentry{count\_labels\_by\_interaction\_type\_and\_data\_model()}\spxextra{pypath.core.network.Network method}}

\begin{fulllineitems}
\phantomsection\label{\detokenize{reference:pypath.core.network.Network.count_labels_by_interaction_type_and_data_model}}\pysiglinewithargsret{\sphinxbfcode{\sphinxupquote{count\_labels\_by\_interaction\_type\_and\_data\_model}}}{}{}
str(object=’‘) -\textgreater{} str
str(bytes\_or\_buffer{[}, encoding{[}, errors{]}{]}) -\textgreater{} str

Create a new string object from the given object. If encoding or
errors is specified, then the object must expose a data buffer
that will be decoded using the given encoding and error handler.
Otherwise, returns the result of object.\_\_str\_\_() (if defined)
or repr(object).
encoding defaults to sys.getdefaultencoding().
errors defaults to ‘strict’.

\end{fulllineitems}

\index{count\_labels\_by\_interaction\_type\_and\_data\_model\_and\_resource() (pypath.core.network.Network method)@\spxentry{count\_labels\_by\_interaction\_type\_and\_data\_model\_and\_resource()}\spxextra{pypath.core.network.Network method}}

\begin{fulllineitems}
\phantomsection\label{\detokenize{reference:pypath.core.network.Network.count_labels_by_interaction_type_and_data_model_and_resource}}\pysiglinewithargsret{\sphinxbfcode{\sphinxupquote{count\_labels\_by\_interaction\_type\_and\_data\_model\_and\_resource}}}{}{}
str(object=’‘) -\textgreater{} str
str(bytes\_or\_buffer{[}, encoding{[}, errors{]}{]}) -\textgreater{} str

Create a new string object from the given object. If encoding or
errors is specified, then the object must expose a data buffer
that will be decoded using the given encoding and error handler.
Otherwise, returns the result of object.\_\_str\_\_() (if defined)
or repr(object).
encoding defaults to sys.getdefaultencoding().
errors defaults to ‘strict’.

\end{fulllineitems}

\index{count\_labels\_by\_reference() (pypath.core.network.Network method)@\spxentry{count\_labels\_by\_reference()}\spxextra{pypath.core.network.Network method}}

\begin{fulllineitems}
\phantomsection\label{\detokenize{reference:pypath.core.network.Network.count_labels_by_reference}}\pysiglinewithargsret{\sphinxbfcode{\sphinxupquote{count\_labels\_by\_reference}}}{}{}
str(object=’‘) -\textgreater{} str
str(bytes\_or\_buffer{[}, encoding{[}, errors{]}{]}) -\textgreater{} str

Create a new string object from the given object. If encoding or
errors is specified, then the object must expose a data buffer
that will be decoded using the given encoding and error handler.
Otherwise, returns the result of object.\_\_str\_\_() (if defined)
or repr(object).
encoding defaults to sys.getdefaultencoding().
errors defaults to ‘strict’.

\end{fulllineitems}

\index{count\_labels\_by\_resource() (pypath.core.network.Network method)@\spxentry{count\_labels\_by\_resource()}\spxextra{pypath.core.network.Network method}}

\begin{fulllineitems}
\phantomsection\label{\detokenize{reference:pypath.core.network.Network.count_labels_by_resource}}\pysiglinewithargsret{\sphinxbfcode{\sphinxupquote{count\_labels\_by\_resource}}}{}{}
str(object=’‘) -\textgreater{} str
str(bytes\_or\_buffer{[}, encoding{[}, errors{]}{]}) -\textgreater{} str

Create a new string object from the given object. If encoding or
errors is specified, then the object must expose a data buffer
that will be decoded using the given encoding and error handler.
Otherwise, returns the result of object.\_\_str\_\_() (if defined)
or repr(object).
encoding defaults to sys.getdefaultencoding().
errors defaults to ‘strict’.

\end{fulllineitems}

\index{count\_lncrna\_identifiers() (pypath.core.network.Network method)@\spxentry{count\_lncrna\_identifiers()}\spxextra{pypath.core.network.Network method}}

\begin{fulllineitems}
\phantomsection\label{\detokenize{reference:pypath.core.network.Network.count_lncrna_identifiers}}\pysiglinewithargsret{\sphinxbfcode{\sphinxupquote{count\_lncrna\_identifiers}}}{}{}
str(object=’‘) -\textgreater{} str
str(bytes\_or\_buffer{[}, encoding{[}, errors{]}{]}) -\textgreater{} str

Create a new string object from the given object. If encoding or
errors is specified, then the object must expose a data buffer
that will be decoded using the given encoding and error handler.
Otherwise, returns the result of object.\_\_str\_\_() (if defined)
or repr(object).
encoding defaults to sys.getdefaultencoding().
errors defaults to ‘strict’.

\end{fulllineitems}

\index{count\_lncrna\_identifiers\_by\_data\_model() (pypath.core.network.Network method)@\spxentry{count\_lncrna\_identifiers\_by\_data\_model()}\spxextra{pypath.core.network.Network method}}

\begin{fulllineitems}
\phantomsection\label{\detokenize{reference:pypath.core.network.Network.count_lncrna_identifiers_by_data_model}}\pysiglinewithargsret{\sphinxbfcode{\sphinxupquote{count\_lncrna\_identifiers\_by\_data\_model}}}{}{}
str(object=’‘) -\textgreater{} str
str(bytes\_or\_buffer{[}, encoding{[}, errors{]}{]}) -\textgreater{} str

Create a new string object from the given object. If encoding or
errors is specified, then the object must expose a data buffer
that will be decoded using the given encoding and error handler.
Otherwise, returns the result of object.\_\_str\_\_() (if defined)
or repr(object).
encoding defaults to sys.getdefaultencoding().
errors defaults to ‘strict’.

\end{fulllineitems}

\index{count\_lncrna\_identifiers\_by\_interaction\_type() (pypath.core.network.Network method)@\spxentry{count\_lncrna\_identifiers\_by\_interaction\_type()}\spxextra{pypath.core.network.Network method}}

\begin{fulllineitems}
\phantomsection\label{\detokenize{reference:pypath.core.network.Network.count_lncrna_identifiers_by_interaction_type}}\pysiglinewithargsret{\sphinxbfcode{\sphinxupquote{count\_lncrna\_identifiers\_by\_interaction\_type}}}{}{}
str(object=’‘) -\textgreater{} str
str(bytes\_or\_buffer{[}, encoding{[}, errors{]}{]}) -\textgreater{} str

Create a new string object from the given object. If encoding or
errors is specified, then the object must expose a data buffer
that will be decoded using the given encoding and error handler.
Otherwise, returns the result of object.\_\_str\_\_() (if defined)
or repr(object).
encoding defaults to sys.getdefaultencoding().
errors defaults to ‘strict’.

\end{fulllineitems}

\index{count\_lncrna\_identifiers\_by\_interaction\_type\_and\_data\_model() (pypath.core.network.Network method)@\spxentry{count\_lncrna\_identifiers\_by\_interaction\_type\_and\_data\_model()}\spxextra{pypath.core.network.Network method}}

\begin{fulllineitems}
\phantomsection\label{\detokenize{reference:pypath.core.network.Network.count_lncrna_identifiers_by_interaction_type_and_data_model}}\pysiglinewithargsret{\sphinxbfcode{\sphinxupquote{count\_lncrna\_identifiers\_by\_interaction\_type\_and\_data\_model}}}{}{}
str(object=’‘) -\textgreater{} str
str(bytes\_or\_buffer{[}, encoding{[}, errors{]}{]}) -\textgreater{} str

Create a new string object from the given object. If encoding or
errors is specified, then the object must expose a data buffer
that will be decoded using the given encoding and error handler.
Otherwise, returns the result of object.\_\_str\_\_() (if defined)
or repr(object).
encoding defaults to sys.getdefaultencoding().
errors defaults to ‘strict’.

\end{fulllineitems}

\index{count\_lncrna\_identifiers\_by\_interaction\_type\_and\_data\_model\_and\_resource() (pypath.core.network.Network method)@\spxentry{count\_lncrna\_identifiers\_by\_interaction\_type\_and\_data\_model\_and\_resource()}\spxextra{pypath.core.network.Network method}}

\begin{fulllineitems}
\phantomsection\label{\detokenize{reference:pypath.core.network.Network.count_lncrna_identifiers_by_interaction_type_and_data_model_and_resource}}\pysiglinewithargsret{\sphinxbfcode{\sphinxupquote{count\_lncrna\_identifiers\_by\_interaction\_type\_and\_data\_model\_and\_resource}}}{}{}
str(object=’‘) -\textgreater{} str
str(bytes\_or\_buffer{[}, encoding{[}, errors{]}{]}) -\textgreater{} str

Create a new string object from the given object. If encoding or
errors is specified, then the object must expose a data buffer
that will be decoded using the given encoding and error handler.
Otherwise, returns the result of object.\_\_str\_\_() (if defined)
or repr(object).
encoding defaults to sys.getdefaultencoding().
errors defaults to ‘strict’.

\end{fulllineitems}

\index{count\_lncrna\_identifiers\_by\_reference() (pypath.core.network.Network method)@\spxentry{count\_lncrna\_identifiers\_by\_reference()}\spxextra{pypath.core.network.Network method}}

\begin{fulllineitems}
\phantomsection\label{\detokenize{reference:pypath.core.network.Network.count_lncrna_identifiers_by_reference}}\pysiglinewithargsret{\sphinxbfcode{\sphinxupquote{count\_lncrna\_identifiers\_by\_reference}}}{}{}
str(object=’‘) -\textgreater{} str
str(bytes\_or\_buffer{[}, encoding{[}, errors{]}{]}) -\textgreater{} str

Create a new string object from the given object. If encoding or
errors is specified, then the object must expose a data buffer
that will be decoded using the given encoding and error handler.
Otherwise, returns the result of object.\_\_str\_\_() (if defined)
or repr(object).
encoding defaults to sys.getdefaultencoding().
errors defaults to ‘strict’.

\end{fulllineitems}

\index{count\_lncrna\_identifiers\_by\_resource() (pypath.core.network.Network method)@\spxentry{count\_lncrna\_identifiers\_by\_resource()}\spxextra{pypath.core.network.Network method}}

\begin{fulllineitems}
\phantomsection\label{\detokenize{reference:pypath.core.network.Network.count_lncrna_identifiers_by_resource}}\pysiglinewithargsret{\sphinxbfcode{\sphinxupquote{count\_lncrna\_identifiers\_by\_resource}}}{}{}
str(object=’‘) -\textgreater{} str
str(bytes\_or\_buffer{[}, encoding{[}, errors{]}{]}) -\textgreater{} str

Create a new string object from the given object. If encoding or
errors is specified, then the object must expose a data buffer
that will be decoded using the given encoding and error handler.
Otherwise, returns the result of object.\_\_str\_\_() (if defined)
or repr(object).
encoding defaults to sys.getdefaultencoding().
errors defaults to ‘strict’.

\end{fulllineitems}

\index{count\_lncrna\_labels() (pypath.core.network.Network method)@\spxentry{count\_lncrna\_labels()}\spxextra{pypath.core.network.Network method}}

\begin{fulllineitems}
\phantomsection\label{\detokenize{reference:pypath.core.network.Network.count_lncrna_labels}}\pysiglinewithargsret{\sphinxbfcode{\sphinxupquote{count\_lncrna\_labels}}}{}{}
str(object=’‘) -\textgreater{} str
str(bytes\_or\_buffer{[}, encoding{[}, errors{]}{]}) -\textgreater{} str

Create a new string object from the given object. If encoding or
errors is specified, then the object must expose a data buffer
that will be decoded using the given encoding and error handler.
Otherwise, returns the result of object.\_\_str\_\_() (if defined)
or repr(object).
encoding defaults to sys.getdefaultencoding().
errors defaults to ‘strict’.

\end{fulllineitems}

\index{count\_lncrna\_labels\_by\_data\_model() (pypath.core.network.Network method)@\spxentry{count\_lncrna\_labels\_by\_data\_model()}\spxextra{pypath.core.network.Network method}}

\begin{fulllineitems}
\phantomsection\label{\detokenize{reference:pypath.core.network.Network.count_lncrna_labels_by_data_model}}\pysiglinewithargsret{\sphinxbfcode{\sphinxupquote{count\_lncrna\_labels\_by\_data\_model}}}{}{}
str(object=’‘) -\textgreater{} str
str(bytes\_or\_buffer{[}, encoding{[}, errors{]}{]}) -\textgreater{} str

Create a new string object from the given object. If encoding or
errors is specified, then the object must expose a data buffer
that will be decoded using the given encoding and error handler.
Otherwise, returns the result of object.\_\_str\_\_() (if defined)
or repr(object).
encoding defaults to sys.getdefaultencoding().
errors defaults to ‘strict’.

\end{fulllineitems}

\index{count\_lncrna\_labels\_by\_interaction\_type() (pypath.core.network.Network method)@\spxentry{count\_lncrna\_labels\_by\_interaction\_type()}\spxextra{pypath.core.network.Network method}}

\begin{fulllineitems}
\phantomsection\label{\detokenize{reference:pypath.core.network.Network.count_lncrna_labels_by_interaction_type}}\pysiglinewithargsret{\sphinxbfcode{\sphinxupquote{count\_lncrna\_labels\_by\_interaction\_type}}}{}{}
str(object=’‘) -\textgreater{} str
str(bytes\_or\_buffer{[}, encoding{[}, errors{]}{]}) -\textgreater{} str

Create a new string object from the given object. If encoding or
errors is specified, then the object must expose a data buffer
that will be decoded using the given encoding and error handler.
Otherwise, returns the result of object.\_\_str\_\_() (if defined)
or repr(object).
encoding defaults to sys.getdefaultencoding().
errors defaults to ‘strict’.

\end{fulllineitems}

\index{count\_lncrna\_labels\_by\_interaction\_type\_and\_data\_model() (pypath.core.network.Network method)@\spxentry{count\_lncrna\_labels\_by\_interaction\_type\_and\_data\_model()}\spxextra{pypath.core.network.Network method}}

\begin{fulllineitems}
\phantomsection\label{\detokenize{reference:pypath.core.network.Network.count_lncrna_labels_by_interaction_type_and_data_model}}\pysiglinewithargsret{\sphinxbfcode{\sphinxupquote{count\_lncrna\_labels\_by\_interaction\_type\_and\_data\_model}}}{}{}
str(object=’‘) -\textgreater{} str
str(bytes\_or\_buffer{[}, encoding{[}, errors{]}{]}) -\textgreater{} str

Create a new string object from the given object. If encoding or
errors is specified, then the object must expose a data buffer
that will be decoded using the given encoding and error handler.
Otherwise, returns the result of object.\_\_str\_\_() (if defined)
or repr(object).
encoding defaults to sys.getdefaultencoding().
errors defaults to ‘strict’.

\end{fulllineitems}

\index{count\_lncrna\_labels\_by\_interaction\_type\_and\_data\_model\_and\_resource() (pypath.core.network.Network method)@\spxentry{count\_lncrna\_labels\_by\_interaction\_type\_and\_data\_model\_and\_resource()}\spxextra{pypath.core.network.Network method}}

\begin{fulllineitems}
\phantomsection\label{\detokenize{reference:pypath.core.network.Network.count_lncrna_labels_by_interaction_type_and_data_model_and_resource}}\pysiglinewithargsret{\sphinxbfcode{\sphinxupquote{count\_lncrna\_labels\_by\_interaction\_type\_and\_data\_model\_and\_resource}}}{}{}
str(object=’‘) -\textgreater{} str
str(bytes\_or\_buffer{[}, encoding{[}, errors{]}{]}) -\textgreater{} str

Create a new string object from the given object. If encoding or
errors is specified, then the object must expose a data buffer
that will be decoded using the given encoding and error handler.
Otherwise, returns the result of object.\_\_str\_\_() (if defined)
or repr(object).
encoding defaults to sys.getdefaultencoding().
errors defaults to ‘strict’.

\end{fulllineitems}

\index{count\_lncrna\_labels\_by\_reference() (pypath.core.network.Network method)@\spxentry{count\_lncrna\_labels\_by\_reference()}\spxextra{pypath.core.network.Network method}}

\begin{fulllineitems}
\phantomsection\label{\detokenize{reference:pypath.core.network.Network.count_lncrna_labels_by_reference}}\pysiglinewithargsret{\sphinxbfcode{\sphinxupquote{count\_lncrna\_labels\_by\_reference}}}{}{}
str(object=’‘) -\textgreater{} str
str(bytes\_or\_buffer{[}, encoding{[}, errors{]}{]}) -\textgreater{} str

Create a new string object from the given object. If encoding or
errors is specified, then the object must expose a data buffer
that will be decoded using the given encoding and error handler.
Otherwise, returns the result of object.\_\_str\_\_() (if defined)
or repr(object).
encoding defaults to sys.getdefaultencoding().
errors defaults to ‘strict’.

\end{fulllineitems}

\index{count\_lncrna\_labels\_by\_resource() (pypath.core.network.Network method)@\spxentry{count\_lncrna\_labels\_by\_resource()}\spxextra{pypath.core.network.Network method}}

\begin{fulllineitems}
\phantomsection\label{\detokenize{reference:pypath.core.network.Network.count_lncrna_labels_by_resource}}\pysiglinewithargsret{\sphinxbfcode{\sphinxupquote{count\_lncrna\_labels\_by\_resource}}}{}{}
str(object=’‘) -\textgreater{} str
str(bytes\_or\_buffer{[}, encoding{[}, errors{]}{]}) -\textgreater{} str

Create a new string object from the given object. If encoding or
errors is specified, then the object must expose a data buffer
that will be decoded using the given encoding and error handler.
Otherwise, returns the result of object.\_\_str\_\_() (if defined)
or repr(object).
encoding defaults to sys.getdefaultencoding().
errors defaults to ‘strict’.

\end{fulllineitems}

\index{count\_lncrnas() (pypath.core.network.Network method)@\spxentry{count\_lncrnas()}\spxextra{pypath.core.network.Network method}}

\begin{fulllineitems}
\phantomsection\label{\detokenize{reference:pypath.core.network.Network.count_lncrnas}}\pysiglinewithargsret{\sphinxbfcode{\sphinxupquote{count\_lncrnas}}}{}{}
str(object=’‘) -\textgreater{} str
str(bytes\_or\_buffer{[}, encoding{[}, errors{]}{]}) -\textgreater{} str

Create a new string object from the given object. If encoding or
errors is specified, then the object must expose a data buffer
that will be decoded using the given encoding and error handler.
Otherwise, returns the result of object.\_\_str\_\_() (if defined)
or repr(object).
encoding defaults to sys.getdefaultencoding().
errors defaults to ‘strict’.

\end{fulllineitems}

\index{count\_lncrnas\_by\_data\_model() (pypath.core.network.Network method)@\spxentry{count\_lncrnas\_by\_data\_model()}\spxextra{pypath.core.network.Network method}}

\begin{fulllineitems}
\phantomsection\label{\detokenize{reference:pypath.core.network.Network.count_lncrnas_by_data_model}}\pysiglinewithargsret{\sphinxbfcode{\sphinxupquote{count\_lncrnas\_by\_data\_model}}}{}{}
str(object=’‘) -\textgreater{} str
str(bytes\_or\_buffer{[}, encoding{[}, errors{]}{]}) -\textgreater{} str

Create a new string object from the given object. If encoding or
errors is specified, then the object must expose a data buffer
that will be decoded using the given encoding and error handler.
Otherwise, returns the result of object.\_\_str\_\_() (if defined)
or repr(object).
encoding defaults to sys.getdefaultencoding().
errors defaults to ‘strict’.

\end{fulllineitems}

\index{count\_lncrnas\_by\_interaction\_type() (pypath.core.network.Network method)@\spxentry{count\_lncrnas\_by\_interaction\_type()}\spxextra{pypath.core.network.Network method}}

\begin{fulllineitems}
\phantomsection\label{\detokenize{reference:pypath.core.network.Network.count_lncrnas_by_interaction_type}}\pysiglinewithargsret{\sphinxbfcode{\sphinxupquote{count\_lncrnas\_by\_interaction\_type}}}{}{}
str(object=’‘) -\textgreater{} str
str(bytes\_or\_buffer{[}, encoding{[}, errors{]}{]}) -\textgreater{} str

Create a new string object from the given object. If encoding or
errors is specified, then the object must expose a data buffer
that will be decoded using the given encoding and error handler.
Otherwise, returns the result of object.\_\_str\_\_() (if defined)
or repr(object).
encoding defaults to sys.getdefaultencoding().
errors defaults to ‘strict’.

\end{fulllineitems}

\index{count\_lncrnas\_by\_interaction\_type\_and\_data\_model() (pypath.core.network.Network method)@\spxentry{count\_lncrnas\_by\_interaction\_type\_and\_data\_model()}\spxextra{pypath.core.network.Network method}}

\begin{fulllineitems}
\phantomsection\label{\detokenize{reference:pypath.core.network.Network.count_lncrnas_by_interaction_type_and_data_model}}\pysiglinewithargsret{\sphinxbfcode{\sphinxupquote{count\_lncrnas\_by\_interaction\_type\_and\_data\_model}}}{}{}
str(object=’‘) -\textgreater{} str
str(bytes\_or\_buffer{[}, encoding{[}, errors{]}{]}) -\textgreater{} str

Create a new string object from the given object. If encoding or
errors is specified, then the object must expose a data buffer
that will be decoded using the given encoding and error handler.
Otherwise, returns the result of object.\_\_str\_\_() (if defined)
or repr(object).
encoding defaults to sys.getdefaultencoding().
errors defaults to ‘strict’.

\end{fulllineitems}

\index{count\_lncrnas\_by\_interaction\_type\_and\_data\_model\_and\_resource() (pypath.core.network.Network method)@\spxentry{count\_lncrnas\_by\_interaction\_type\_and\_data\_model\_and\_resource()}\spxextra{pypath.core.network.Network method}}

\begin{fulllineitems}
\phantomsection\label{\detokenize{reference:pypath.core.network.Network.count_lncrnas_by_interaction_type_and_data_model_and_resource}}\pysiglinewithargsret{\sphinxbfcode{\sphinxupquote{count\_lncrnas\_by\_interaction\_type\_and\_data\_model\_and\_resource}}}{}{}
str(object=’‘) -\textgreater{} str
str(bytes\_or\_buffer{[}, encoding{[}, errors{]}{]}) -\textgreater{} str

Create a new string object from the given object. If encoding or
errors is specified, then the object must expose a data buffer
that will be decoded using the given encoding and error handler.
Otherwise, returns the result of object.\_\_str\_\_() (if defined)
or repr(object).
encoding defaults to sys.getdefaultencoding().
errors defaults to ‘strict’.

\end{fulllineitems}

\index{count\_lncrnas\_by\_reference() (pypath.core.network.Network method)@\spxentry{count\_lncrnas\_by\_reference()}\spxextra{pypath.core.network.Network method}}

\begin{fulllineitems}
\phantomsection\label{\detokenize{reference:pypath.core.network.Network.count_lncrnas_by_reference}}\pysiglinewithargsret{\sphinxbfcode{\sphinxupquote{count\_lncrnas\_by\_reference}}}{}{}
str(object=’‘) -\textgreater{} str
str(bytes\_or\_buffer{[}, encoding{[}, errors{]}{]}) -\textgreater{} str

Create a new string object from the given object. If encoding or
errors is specified, then the object must expose a data buffer
that will be decoded using the given encoding and error handler.
Otherwise, returns the result of object.\_\_str\_\_() (if defined)
or repr(object).
encoding defaults to sys.getdefaultencoding().
errors defaults to ‘strict’.

\end{fulllineitems}

\index{count\_lncrnas\_by\_resource() (pypath.core.network.Network method)@\spxentry{count\_lncrnas\_by\_resource()}\spxextra{pypath.core.network.Network method}}

\begin{fulllineitems}
\phantomsection\label{\detokenize{reference:pypath.core.network.Network.count_lncrnas_by_resource}}\pysiglinewithargsret{\sphinxbfcode{\sphinxupquote{count\_lncrnas\_by\_resource}}}{}{}
str(object=’‘) -\textgreater{} str
str(bytes\_or\_buffer{[}, encoding{[}, errors{]}{]}) -\textgreater{} str

Create a new string object from the given object. If encoding or
errors is specified, then the object must expose a data buffer
that will be decoded using the given encoding and error handler.
Otherwise, returns the result of object.\_\_str\_\_() (if defined)
or repr(object).
encoding defaults to sys.getdefaultencoding().
errors defaults to ‘strict’.

\end{fulllineitems}

\index{count\_mirna\_identifiers() (pypath.core.network.Network method)@\spxentry{count\_mirna\_identifiers()}\spxextra{pypath.core.network.Network method}}

\begin{fulllineitems}
\phantomsection\label{\detokenize{reference:pypath.core.network.Network.count_mirna_identifiers}}\pysiglinewithargsret{\sphinxbfcode{\sphinxupquote{count\_mirna\_identifiers}}}{}{}
str(object=’‘) -\textgreater{} str
str(bytes\_or\_buffer{[}, encoding{[}, errors{]}{]}) -\textgreater{} str

Create a new string object from the given object. If encoding or
errors is specified, then the object must expose a data buffer
that will be decoded using the given encoding and error handler.
Otherwise, returns the result of object.\_\_str\_\_() (if defined)
or repr(object).
encoding defaults to sys.getdefaultencoding().
errors defaults to ‘strict’.

\end{fulllineitems}

\index{count\_mirna\_identifiers\_by\_data\_model() (pypath.core.network.Network method)@\spxentry{count\_mirna\_identifiers\_by\_data\_model()}\spxextra{pypath.core.network.Network method}}

\begin{fulllineitems}
\phantomsection\label{\detokenize{reference:pypath.core.network.Network.count_mirna_identifiers_by_data_model}}\pysiglinewithargsret{\sphinxbfcode{\sphinxupquote{count\_mirna\_identifiers\_by\_data\_model}}}{}{}
str(object=’‘) -\textgreater{} str
str(bytes\_or\_buffer{[}, encoding{[}, errors{]}{]}) -\textgreater{} str

Create a new string object from the given object. If encoding or
errors is specified, then the object must expose a data buffer
that will be decoded using the given encoding and error handler.
Otherwise, returns the result of object.\_\_str\_\_() (if defined)
or repr(object).
encoding defaults to sys.getdefaultencoding().
errors defaults to ‘strict’.

\end{fulllineitems}

\index{count\_mirna\_identifiers\_by\_interaction\_type() (pypath.core.network.Network method)@\spxentry{count\_mirna\_identifiers\_by\_interaction\_type()}\spxextra{pypath.core.network.Network method}}

\begin{fulllineitems}
\phantomsection\label{\detokenize{reference:pypath.core.network.Network.count_mirna_identifiers_by_interaction_type}}\pysiglinewithargsret{\sphinxbfcode{\sphinxupquote{count\_mirna\_identifiers\_by\_interaction\_type}}}{}{}
str(object=’‘) -\textgreater{} str
str(bytes\_or\_buffer{[}, encoding{[}, errors{]}{]}) -\textgreater{} str

Create a new string object from the given object. If encoding or
errors is specified, then the object must expose a data buffer
that will be decoded using the given encoding and error handler.
Otherwise, returns the result of object.\_\_str\_\_() (if defined)
or repr(object).
encoding defaults to sys.getdefaultencoding().
errors defaults to ‘strict’.

\end{fulllineitems}

\index{count\_mirna\_identifiers\_by\_interaction\_type\_and\_data\_model() (pypath.core.network.Network method)@\spxentry{count\_mirna\_identifiers\_by\_interaction\_type\_and\_data\_model()}\spxextra{pypath.core.network.Network method}}

\begin{fulllineitems}
\phantomsection\label{\detokenize{reference:pypath.core.network.Network.count_mirna_identifiers_by_interaction_type_and_data_model}}\pysiglinewithargsret{\sphinxbfcode{\sphinxupquote{count\_mirna\_identifiers\_by\_interaction\_type\_and\_data\_model}}}{}{}
str(object=’‘) -\textgreater{} str
str(bytes\_or\_buffer{[}, encoding{[}, errors{]}{]}) -\textgreater{} str

Create a new string object from the given object. If encoding or
errors is specified, then the object must expose a data buffer
that will be decoded using the given encoding and error handler.
Otherwise, returns the result of object.\_\_str\_\_() (if defined)
or repr(object).
encoding defaults to sys.getdefaultencoding().
errors defaults to ‘strict’.

\end{fulllineitems}

\index{count\_mirna\_identifiers\_by\_interaction\_type\_and\_data\_model\_and\_resource() (pypath.core.network.Network method)@\spxentry{count\_mirna\_identifiers\_by\_interaction\_type\_and\_data\_model\_and\_resource()}\spxextra{pypath.core.network.Network method}}

\begin{fulllineitems}
\phantomsection\label{\detokenize{reference:pypath.core.network.Network.count_mirna_identifiers_by_interaction_type_and_data_model_and_resource}}\pysiglinewithargsret{\sphinxbfcode{\sphinxupquote{count\_mirna\_identifiers\_by\_interaction\_type\_and\_data\_model\_and\_resource}}}{}{}
str(object=’‘) -\textgreater{} str
str(bytes\_or\_buffer{[}, encoding{[}, errors{]}{]}) -\textgreater{} str

Create a new string object from the given object. If encoding or
errors is specified, then the object must expose a data buffer
that will be decoded using the given encoding and error handler.
Otherwise, returns the result of object.\_\_str\_\_() (if defined)
or repr(object).
encoding defaults to sys.getdefaultencoding().
errors defaults to ‘strict’.

\end{fulllineitems}

\index{count\_mirna\_identifiers\_by\_reference() (pypath.core.network.Network method)@\spxentry{count\_mirna\_identifiers\_by\_reference()}\spxextra{pypath.core.network.Network method}}

\begin{fulllineitems}
\phantomsection\label{\detokenize{reference:pypath.core.network.Network.count_mirna_identifiers_by_reference}}\pysiglinewithargsret{\sphinxbfcode{\sphinxupquote{count\_mirna\_identifiers\_by\_reference}}}{}{}
str(object=’‘) -\textgreater{} str
str(bytes\_or\_buffer{[}, encoding{[}, errors{]}{]}) -\textgreater{} str

Create a new string object from the given object. If encoding or
errors is specified, then the object must expose a data buffer
that will be decoded using the given encoding and error handler.
Otherwise, returns the result of object.\_\_str\_\_() (if defined)
or repr(object).
encoding defaults to sys.getdefaultencoding().
errors defaults to ‘strict’.

\end{fulllineitems}

\index{count\_mirna\_identifiers\_by\_resource() (pypath.core.network.Network method)@\spxentry{count\_mirna\_identifiers\_by\_resource()}\spxextra{pypath.core.network.Network method}}

\begin{fulllineitems}
\phantomsection\label{\detokenize{reference:pypath.core.network.Network.count_mirna_identifiers_by_resource}}\pysiglinewithargsret{\sphinxbfcode{\sphinxupquote{count\_mirna\_identifiers\_by\_resource}}}{}{}
str(object=’‘) -\textgreater{} str
str(bytes\_or\_buffer{[}, encoding{[}, errors{]}{]}) -\textgreater{} str

Create a new string object from the given object. If encoding or
errors is specified, then the object must expose a data buffer
that will be decoded using the given encoding and error handler.
Otherwise, returns the result of object.\_\_str\_\_() (if defined)
or repr(object).
encoding defaults to sys.getdefaultencoding().
errors defaults to ‘strict’.

\end{fulllineitems}

\index{count\_mirna\_labels() (pypath.core.network.Network method)@\spxentry{count\_mirna\_labels()}\spxextra{pypath.core.network.Network method}}

\begin{fulllineitems}
\phantomsection\label{\detokenize{reference:pypath.core.network.Network.count_mirna_labels}}\pysiglinewithargsret{\sphinxbfcode{\sphinxupquote{count\_mirna\_labels}}}{}{}
str(object=’‘) -\textgreater{} str
str(bytes\_or\_buffer{[}, encoding{[}, errors{]}{]}) -\textgreater{} str

Create a new string object from the given object. If encoding or
errors is specified, then the object must expose a data buffer
that will be decoded using the given encoding and error handler.
Otherwise, returns the result of object.\_\_str\_\_() (if defined)
or repr(object).
encoding defaults to sys.getdefaultencoding().
errors defaults to ‘strict’.

\end{fulllineitems}

\index{count\_mirna\_labels\_by\_data\_model() (pypath.core.network.Network method)@\spxentry{count\_mirna\_labels\_by\_data\_model()}\spxextra{pypath.core.network.Network method}}

\begin{fulllineitems}
\phantomsection\label{\detokenize{reference:pypath.core.network.Network.count_mirna_labels_by_data_model}}\pysiglinewithargsret{\sphinxbfcode{\sphinxupquote{count\_mirna\_labels\_by\_data\_model}}}{}{}
str(object=’‘) -\textgreater{} str
str(bytes\_or\_buffer{[}, encoding{[}, errors{]}{]}) -\textgreater{} str

Create a new string object from the given object. If encoding or
errors is specified, then the object must expose a data buffer
that will be decoded using the given encoding and error handler.
Otherwise, returns the result of object.\_\_str\_\_() (if defined)
or repr(object).
encoding defaults to sys.getdefaultencoding().
errors defaults to ‘strict’.

\end{fulllineitems}

\index{count\_mirna\_labels\_by\_interaction\_type() (pypath.core.network.Network method)@\spxentry{count\_mirna\_labels\_by\_interaction\_type()}\spxextra{pypath.core.network.Network method}}

\begin{fulllineitems}
\phantomsection\label{\detokenize{reference:pypath.core.network.Network.count_mirna_labels_by_interaction_type}}\pysiglinewithargsret{\sphinxbfcode{\sphinxupquote{count\_mirna\_labels\_by\_interaction\_type}}}{}{}
str(object=’‘) -\textgreater{} str
str(bytes\_or\_buffer{[}, encoding{[}, errors{]}{]}) -\textgreater{} str

Create a new string object from the given object. If encoding or
errors is specified, then the object must expose a data buffer
that will be decoded using the given encoding and error handler.
Otherwise, returns the result of object.\_\_str\_\_() (if defined)
or repr(object).
encoding defaults to sys.getdefaultencoding().
errors defaults to ‘strict’.

\end{fulllineitems}

\index{count\_mirna\_labels\_by\_interaction\_type\_and\_data\_model() (pypath.core.network.Network method)@\spxentry{count\_mirna\_labels\_by\_interaction\_type\_and\_data\_model()}\spxextra{pypath.core.network.Network method}}

\begin{fulllineitems}
\phantomsection\label{\detokenize{reference:pypath.core.network.Network.count_mirna_labels_by_interaction_type_and_data_model}}\pysiglinewithargsret{\sphinxbfcode{\sphinxupquote{count\_mirna\_labels\_by\_interaction\_type\_and\_data\_model}}}{}{}
str(object=’‘) -\textgreater{} str
str(bytes\_or\_buffer{[}, encoding{[}, errors{]}{]}) -\textgreater{} str

Create a new string object from the given object. If encoding or
errors is specified, then the object must expose a data buffer
that will be decoded using the given encoding and error handler.
Otherwise, returns the result of object.\_\_str\_\_() (if defined)
or repr(object).
encoding defaults to sys.getdefaultencoding().
errors defaults to ‘strict’.

\end{fulllineitems}

\index{count\_mirna\_labels\_by\_interaction\_type\_and\_data\_model\_and\_resource() (pypath.core.network.Network method)@\spxentry{count\_mirna\_labels\_by\_interaction\_type\_and\_data\_model\_and\_resource()}\spxextra{pypath.core.network.Network method}}

\begin{fulllineitems}
\phantomsection\label{\detokenize{reference:pypath.core.network.Network.count_mirna_labels_by_interaction_type_and_data_model_and_resource}}\pysiglinewithargsret{\sphinxbfcode{\sphinxupquote{count\_mirna\_labels\_by\_interaction\_type\_and\_data\_model\_and\_resource}}}{}{}
str(object=’‘) -\textgreater{} str
str(bytes\_or\_buffer{[}, encoding{[}, errors{]}{]}) -\textgreater{} str

Create a new string object from the given object. If encoding or
errors is specified, then the object must expose a data buffer
that will be decoded using the given encoding and error handler.
Otherwise, returns the result of object.\_\_str\_\_() (if defined)
or repr(object).
encoding defaults to sys.getdefaultencoding().
errors defaults to ‘strict’.

\end{fulllineitems}

\index{count\_mirna\_labels\_by\_reference() (pypath.core.network.Network method)@\spxentry{count\_mirna\_labels\_by\_reference()}\spxextra{pypath.core.network.Network method}}

\begin{fulllineitems}
\phantomsection\label{\detokenize{reference:pypath.core.network.Network.count_mirna_labels_by_reference}}\pysiglinewithargsret{\sphinxbfcode{\sphinxupquote{count\_mirna\_labels\_by\_reference}}}{}{}
str(object=’‘) -\textgreater{} str
str(bytes\_or\_buffer{[}, encoding{[}, errors{]}{]}) -\textgreater{} str

Create a new string object from the given object. If encoding or
errors is specified, then the object must expose a data buffer
that will be decoded using the given encoding and error handler.
Otherwise, returns the result of object.\_\_str\_\_() (if defined)
or repr(object).
encoding defaults to sys.getdefaultencoding().
errors defaults to ‘strict’.

\end{fulllineitems}

\index{count\_mirna\_labels\_by\_resource() (pypath.core.network.Network method)@\spxentry{count\_mirna\_labels\_by\_resource()}\spxextra{pypath.core.network.Network method}}

\begin{fulllineitems}
\phantomsection\label{\detokenize{reference:pypath.core.network.Network.count_mirna_labels_by_resource}}\pysiglinewithargsret{\sphinxbfcode{\sphinxupquote{count\_mirna\_labels\_by\_resource}}}{}{}
str(object=’‘) -\textgreater{} str
str(bytes\_or\_buffer{[}, encoding{[}, errors{]}{]}) -\textgreater{} str

Create a new string object from the given object. If encoding or
errors is specified, then the object must expose a data buffer
that will be decoded using the given encoding and error handler.
Otherwise, returns the result of object.\_\_str\_\_() (if defined)
or repr(object).
encoding defaults to sys.getdefaultencoding().
errors defaults to ‘strict’.

\end{fulllineitems}

\index{count\_mirnas() (pypath.core.network.Network method)@\spxentry{count\_mirnas()}\spxextra{pypath.core.network.Network method}}

\begin{fulllineitems}
\phantomsection\label{\detokenize{reference:pypath.core.network.Network.count_mirnas}}\pysiglinewithargsret{\sphinxbfcode{\sphinxupquote{count\_mirnas}}}{}{}
str(object=’‘) -\textgreater{} str
str(bytes\_or\_buffer{[}, encoding{[}, errors{]}{]}) -\textgreater{} str

Create a new string object from the given object. If encoding or
errors is specified, then the object must expose a data buffer
that will be decoded using the given encoding and error handler.
Otherwise, returns the result of object.\_\_str\_\_() (if defined)
or repr(object).
encoding defaults to sys.getdefaultencoding().
errors defaults to ‘strict’.

\end{fulllineitems}

\index{count\_mirnas\_by\_data\_model() (pypath.core.network.Network method)@\spxentry{count\_mirnas\_by\_data\_model()}\spxextra{pypath.core.network.Network method}}

\begin{fulllineitems}
\phantomsection\label{\detokenize{reference:pypath.core.network.Network.count_mirnas_by_data_model}}\pysiglinewithargsret{\sphinxbfcode{\sphinxupquote{count\_mirnas\_by\_data\_model}}}{}{}
str(object=’‘) -\textgreater{} str
str(bytes\_or\_buffer{[}, encoding{[}, errors{]}{]}) -\textgreater{} str

Create a new string object from the given object. If encoding or
errors is specified, then the object must expose a data buffer
that will be decoded using the given encoding and error handler.
Otherwise, returns the result of object.\_\_str\_\_() (if defined)
or repr(object).
encoding defaults to sys.getdefaultencoding().
errors defaults to ‘strict’.

\end{fulllineitems}

\index{count\_mirnas\_by\_interaction\_type() (pypath.core.network.Network method)@\spxentry{count\_mirnas\_by\_interaction\_type()}\spxextra{pypath.core.network.Network method}}

\begin{fulllineitems}
\phantomsection\label{\detokenize{reference:pypath.core.network.Network.count_mirnas_by_interaction_type}}\pysiglinewithargsret{\sphinxbfcode{\sphinxupquote{count\_mirnas\_by\_interaction\_type}}}{}{}
str(object=’‘) -\textgreater{} str
str(bytes\_or\_buffer{[}, encoding{[}, errors{]}{]}) -\textgreater{} str

Create a new string object from the given object. If encoding or
errors is specified, then the object must expose a data buffer
that will be decoded using the given encoding and error handler.
Otherwise, returns the result of object.\_\_str\_\_() (if defined)
or repr(object).
encoding defaults to sys.getdefaultencoding().
errors defaults to ‘strict’.

\end{fulllineitems}

\index{count\_mirnas\_by\_interaction\_type\_and\_data\_model() (pypath.core.network.Network method)@\spxentry{count\_mirnas\_by\_interaction\_type\_and\_data\_model()}\spxextra{pypath.core.network.Network method}}

\begin{fulllineitems}
\phantomsection\label{\detokenize{reference:pypath.core.network.Network.count_mirnas_by_interaction_type_and_data_model}}\pysiglinewithargsret{\sphinxbfcode{\sphinxupquote{count\_mirnas\_by\_interaction\_type\_and\_data\_model}}}{}{}
str(object=’‘) -\textgreater{} str
str(bytes\_or\_buffer{[}, encoding{[}, errors{]}{]}) -\textgreater{} str

Create a new string object from the given object. If encoding or
errors is specified, then the object must expose a data buffer
that will be decoded using the given encoding and error handler.
Otherwise, returns the result of object.\_\_str\_\_() (if defined)
or repr(object).
encoding defaults to sys.getdefaultencoding().
errors defaults to ‘strict’.

\end{fulllineitems}

\index{count\_mirnas\_by\_interaction\_type\_and\_data\_model\_and\_resource() (pypath.core.network.Network method)@\spxentry{count\_mirnas\_by\_interaction\_type\_and\_data\_model\_and\_resource()}\spxextra{pypath.core.network.Network method}}

\begin{fulllineitems}
\phantomsection\label{\detokenize{reference:pypath.core.network.Network.count_mirnas_by_interaction_type_and_data_model_and_resource}}\pysiglinewithargsret{\sphinxbfcode{\sphinxupquote{count\_mirnas\_by\_interaction\_type\_and\_data\_model\_and\_resource}}}{}{}
str(object=’‘) -\textgreater{} str
str(bytes\_or\_buffer{[}, encoding{[}, errors{]}{]}) -\textgreater{} str

Create a new string object from the given object. If encoding or
errors is specified, then the object must expose a data buffer
that will be decoded using the given encoding and error handler.
Otherwise, returns the result of object.\_\_str\_\_() (if defined)
or repr(object).
encoding defaults to sys.getdefaultencoding().
errors defaults to ‘strict’.

\end{fulllineitems}

\index{count\_mirnas\_by\_reference() (pypath.core.network.Network method)@\spxentry{count\_mirnas\_by\_reference()}\spxextra{pypath.core.network.Network method}}

\begin{fulllineitems}
\phantomsection\label{\detokenize{reference:pypath.core.network.Network.count_mirnas_by_reference}}\pysiglinewithargsret{\sphinxbfcode{\sphinxupquote{count\_mirnas\_by\_reference}}}{}{}
str(object=’‘) -\textgreater{} str
str(bytes\_or\_buffer{[}, encoding{[}, errors{]}{]}) -\textgreater{} str

Create a new string object from the given object. If encoding or
errors is specified, then the object must expose a data buffer
that will be decoded using the given encoding and error handler.
Otherwise, returns the result of object.\_\_str\_\_() (if defined)
or repr(object).
encoding defaults to sys.getdefaultencoding().
errors defaults to ‘strict’.

\end{fulllineitems}

\index{count\_mirnas\_by\_resource() (pypath.core.network.Network method)@\spxentry{count\_mirnas\_by\_resource()}\spxextra{pypath.core.network.Network method}}

\begin{fulllineitems}
\phantomsection\label{\detokenize{reference:pypath.core.network.Network.count_mirnas_by_resource}}\pysiglinewithargsret{\sphinxbfcode{\sphinxupquote{count\_mirnas\_by\_resource}}}{}{}
str(object=’‘) -\textgreater{} str
str(bytes\_or\_buffer{[}, encoding{[}, errors{]}{]}) -\textgreater{} str

Create a new string object from the given object. If encoding or
errors is specified, then the object must expose a data buffer
that will be decoded using the given encoding and error handler.
Otherwise, returns the result of object.\_\_str\_\_() (if defined)
or repr(object).
encoding defaults to sys.getdefaultencoding().
errors defaults to ‘strict’.

\end{fulllineitems}

\index{count\_partners() (pypath.core.network.Network method)@\spxentry{count\_partners()}\spxextra{pypath.core.network.Network method}}

\begin{fulllineitems}
\phantomsection\label{\detokenize{reference:pypath.core.network.Network.count_partners}}\pysiglinewithargsret{\sphinxbfcode{\sphinxupquote{count\_partners}}}{\emph{entity}, \emph{**kwargs}}{}
Returns the count of the interacting partners for one or more
entities according to the specified criteria.
Please refer to the docs of the \sphinxcode{\sphinxupquote{partners}} method.

\end{fulllineitems}

\index{count\_post\_transcriptionally\_activated\_by() (pypath.core.network.Network method)@\spxentry{count\_post\_transcriptionally\_activated\_by()}\spxextra{pypath.core.network.Network method}}

\begin{fulllineitems}
\phantomsection\label{\detokenize{reference:pypath.core.network.Network.count_post_transcriptionally_activated_by}}\pysiglinewithargsret{\sphinxbfcode{\sphinxupquote{count\_post\_transcriptionally\_activated\_by}}}{}{}
Returns the count of the interacting partners for one or more
entities according to the specified criteria.
Please refer to the docs of the \sphinxcode{\sphinxupquote{partners}} method.

\end{fulllineitems}

\index{count\_post\_transcriptionally\_activates() (pypath.core.network.Network method)@\spxentry{count\_post\_transcriptionally\_activates()}\spxextra{pypath.core.network.Network method}}

\begin{fulllineitems}
\phantomsection\label{\detokenize{reference:pypath.core.network.Network.count_post_transcriptionally_activates}}\pysiglinewithargsret{\sphinxbfcode{\sphinxupquote{count\_post\_transcriptionally\_activates}}}{}{}
Returns the count of the interacting partners for one or more
entities according to the specified criteria.
Please refer to the docs of the \sphinxcode{\sphinxupquote{partners}} method.

\end{fulllineitems}

\index{count\_post\_transcriptionally\_regulated\_by() (pypath.core.network.Network method)@\spxentry{count\_post\_transcriptionally\_regulated\_by()}\spxextra{pypath.core.network.Network method}}

\begin{fulllineitems}
\phantomsection\label{\detokenize{reference:pypath.core.network.Network.count_post_transcriptionally_regulated_by}}\pysiglinewithargsret{\sphinxbfcode{\sphinxupquote{count\_post\_transcriptionally\_regulated\_by}}}{}{}
Returns the count of the interacting partners for one or more
entities according to the specified criteria.
Please refer to the docs of the \sphinxcode{\sphinxupquote{partners}} method.

\end{fulllineitems}

\index{count\_post\_transcriptionally\_regulates() (pypath.core.network.Network method)@\spxentry{count\_post\_transcriptionally\_regulates()}\spxextra{pypath.core.network.Network method}}

\begin{fulllineitems}
\phantomsection\label{\detokenize{reference:pypath.core.network.Network.count_post_transcriptionally_regulates}}\pysiglinewithargsret{\sphinxbfcode{\sphinxupquote{count\_post\_transcriptionally\_regulates}}}{}{}
Returns the count of the interacting partners for one or more
entities according to the specified criteria.
Please refer to the docs of the \sphinxcode{\sphinxupquote{partners}} method.

\end{fulllineitems}

\index{count\_post\_transcriptionally\_suppressed\_by() (pypath.core.network.Network method)@\spxentry{count\_post\_transcriptionally\_suppressed\_by()}\spxextra{pypath.core.network.Network method}}

\begin{fulllineitems}
\phantomsection\label{\detokenize{reference:pypath.core.network.Network.count_post_transcriptionally_suppressed_by}}\pysiglinewithargsret{\sphinxbfcode{\sphinxupquote{count\_post\_transcriptionally\_suppressed\_by}}}{}{}
Returns the count of the interacting partners for one or more
entities according to the specified criteria.
Please refer to the docs of the \sphinxcode{\sphinxupquote{partners}} method.

\end{fulllineitems}

\index{count\_post\_transcriptionally\_suppresses() (pypath.core.network.Network method)@\spxentry{count\_post\_transcriptionally\_suppresses()}\spxextra{pypath.core.network.Network method}}

\begin{fulllineitems}
\phantomsection\label{\detokenize{reference:pypath.core.network.Network.count_post_transcriptionally_suppresses}}\pysiglinewithargsret{\sphinxbfcode{\sphinxupquote{count\_post\_transcriptionally\_suppresses}}}{}{}
Returns the count of the interacting partners for one or more
entities according to the specified criteria.
Please refer to the docs of the \sphinxcode{\sphinxupquote{partners}} method.

\end{fulllineitems}

\index{count\_post\_translationally\_activated\_by() (pypath.core.network.Network method)@\spxentry{count\_post\_translationally\_activated\_by()}\spxextra{pypath.core.network.Network method}}

\begin{fulllineitems}
\phantomsection\label{\detokenize{reference:pypath.core.network.Network.count_post_translationally_activated_by}}\pysiglinewithargsret{\sphinxbfcode{\sphinxupquote{count\_post\_translationally\_activated\_by}}}{}{}
Returns the count of the interacting partners for one or more
entities according to the specified criteria.
Please refer to the docs of the \sphinxcode{\sphinxupquote{partners}} method.

\end{fulllineitems}

\index{count\_post\_translationally\_activates() (pypath.core.network.Network method)@\spxentry{count\_post\_translationally\_activates()}\spxextra{pypath.core.network.Network method}}

\begin{fulllineitems}
\phantomsection\label{\detokenize{reference:pypath.core.network.Network.count_post_translationally_activates}}\pysiglinewithargsret{\sphinxbfcode{\sphinxupquote{count\_post\_translationally\_activates}}}{}{}
Returns the count of the interacting partners for one or more
entities according to the specified criteria.
Please refer to the docs of the \sphinxcode{\sphinxupquote{partners}} method.

\end{fulllineitems}

\index{count\_post\_translationally\_regulated\_by() (pypath.core.network.Network method)@\spxentry{count\_post\_translationally\_regulated\_by()}\spxextra{pypath.core.network.Network method}}

\begin{fulllineitems}
\phantomsection\label{\detokenize{reference:pypath.core.network.Network.count_post_translationally_regulated_by}}\pysiglinewithargsret{\sphinxbfcode{\sphinxupquote{count\_post\_translationally\_regulated\_by}}}{}{}
Returns the count of the interacting partners for one or more
entities according to the specified criteria.
Please refer to the docs of the \sphinxcode{\sphinxupquote{partners}} method.

\end{fulllineitems}

\index{count\_post\_translationally\_regulates() (pypath.core.network.Network method)@\spxentry{count\_post\_translationally\_regulates()}\spxextra{pypath.core.network.Network method}}

\begin{fulllineitems}
\phantomsection\label{\detokenize{reference:pypath.core.network.Network.count_post_translationally_regulates}}\pysiglinewithargsret{\sphinxbfcode{\sphinxupquote{count\_post\_translationally\_regulates}}}{}{}
Returns the count of the interacting partners for one or more
entities according to the specified criteria.
Please refer to the docs of the \sphinxcode{\sphinxupquote{partners}} method.

\end{fulllineitems}

\index{count\_post\_translationally\_suppressed\_by() (pypath.core.network.Network method)@\spxentry{count\_post\_translationally\_suppressed\_by()}\spxextra{pypath.core.network.Network method}}

\begin{fulllineitems}
\phantomsection\label{\detokenize{reference:pypath.core.network.Network.count_post_translationally_suppressed_by}}\pysiglinewithargsret{\sphinxbfcode{\sphinxupquote{count\_post\_translationally\_suppressed\_by}}}{}{}
Returns the count of the interacting partners for one or more
entities according to the specified criteria.
Please refer to the docs of the \sphinxcode{\sphinxupquote{partners}} method.

\end{fulllineitems}

\index{count\_post\_translationally\_suppresses() (pypath.core.network.Network method)@\spxentry{count\_post\_translationally\_suppresses()}\spxextra{pypath.core.network.Network method}}

\begin{fulllineitems}
\phantomsection\label{\detokenize{reference:pypath.core.network.Network.count_post_translationally_suppresses}}\pysiglinewithargsret{\sphinxbfcode{\sphinxupquote{count\_post\_translationally\_suppresses}}}{}{}
Returns the count of the interacting partners for one or more
entities according to the specified criteria.
Please refer to the docs of the \sphinxcode{\sphinxupquote{partners}} method.

\end{fulllineitems}

\index{count\_protein\_identifiers() (pypath.core.network.Network method)@\spxentry{count\_protein\_identifiers()}\spxextra{pypath.core.network.Network method}}

\begin{fulllineitems}
\phantomsection\label{\detokenize{reference:pypath.core.network.Network.count_protein_identifiers}}\pysiglinewithargsret{\sphinxbfcode{\sphinxupquote{count\_protein\_identifiers}}}{}{}
str(object=’‘) -\textgreater{} str
str(bytes\_or\_buffer{[}, encoding{[}, errors{]}{]}) -\textgreater{} str

Create a new string object from the given object. If encoding or
errors is specified, then the object must expose a data buffer
that will be decoded using the given encoding and error handler.
Otherwise, returns the result of object.\_\_str\_\_() (if defined)
or repr(object).
encoding defaults to sys.getdefaultencoding().
errors defaults to ‘strict’.

\end{fulllineitems}

\index{count\_protein\_identifiers\_by\_data\_model() (pypath.core.network.Network method)@\spxentry{count\_protein\_identifiers\_by\_data\_model()}\spxextra{pypath.core.network.Network method}}

\begin{fulllineitems}
\phantomsection\label{\detokenize{reference:pypath.core.network.Network.count_protein_identifiers_by_data_model}}\pysiglinewithargsret{\sphinxbfcode{\sphinxupquote{count\_protein\_identifiers\_by\_data\_model}}}{}{}
str(object=’‘) -\textgreater{} str
str(bytes\_or\_buffer{[}, encoding{[}, errors{]}{]}) -\textgreater{} str

Create a new string object from the given object. If encoding or
errors is specified, then the object must expose a data buffer
that will be decoded using the given encoding and error handler.
Otherwise, returns the result of object.\_\_str\_\_() (if defined)
or repr(object).
encoding defaults to sys.getdefaultencoding().
errors defaults to ‘strict’.

\end{fulllineitems}

\index{count\_protein\_identifiers\_by\_interaction\_type() (pypath.core.network.Network method)@\spxentry{count\_protein\_identifiers\_by\_interaction\_type()}\spxextra{pypath.core.network.Network method}}

\begin{fulllineitems}
\phantomsection\label{\detokenize{reference:pypath.core.network.Network.count_protein_identifiers_by_interaction_type}}\pysiglinewithargsret{\sphinxbfcode{\sphinxupquote{count\_protein\_identifiers\_by\_interaction\_type}}}{}{}
str(object=’‘) -\textgreater{} str
str(bytes\_or\_buffer{[}, encoding{[}, errors{]}{]}) -\textgreater{} str

Create a new string object from the given object. If encoding or
errors is specified, then the object must expose a data buffer
that will be decoded using the given encoding and error handler.
Otherwise, returns the result of object.\_\_str\_\_() (if defined)
or repr(object).
encoding defaults to sys.getdefaultencoding().
errors defaults to ‘strict’.

\end{fulllineitems}

\index{count\_protein\_identifiers\_by\_interaction\_type\_and\_data\_model() (pypath.core.network.Network method)@\spxentry{count\_protein\_identifiers\_by\_interaction\_type\_and\_data\_model()}\spxextra{pypath.core.network.Network method}}

\begin{fulllineitems}
\phantomsection\label{\detokenize{reference:pypath.core.network.Network.count_protein_identifiers_by_interaction_type_and_data_model}}\pysiglinewithargsret{\sphinxbfcode{\sphinxupquote{count\_protein\_identifiers\_by\_interaction\_type\_and\_data\_model}}}{}{}
str(object=’‘) -\textgreater{} str
str(bytes\_or\_buffer{[}, encoding{[}, errors{]}{]}) -\textgreater{} str

Create a new string object from the given object. If encoding or
errors is specified, then the object must expose a data buffer
that will be decoded using the given encoding and error handler.
Otherwise, returns the result of object.\_\_str\_\_() (if defined)
or repr(object).
encoding defaults to sys.getdefaultencoding().
errors defaults to ‘strict’.

\end{fulllineitems}

\index{count\_protein\_identifiers\_by\_interaction\_type\_and\_data\_model\_and\_resource() (pypath.core.network.Network method)@\spxentry{count\_protein\_identifiers\_by\_interaction\_type\_and\_data\_model\_and\_resource()}\spxextra{pypath.core.network.Network method}}

\begin{fulllineitems}
\phantomsection\label{\detokenize{reference:pypath.core.network.Network.count_protein_identifiers_by_interaction_type_and_data_model_and_resource}}\pysiglinewithargsret{\sphinxbfcode{\sphinxupquote{count\_protein\_identifiers\_by\_interaction\_type\_and\_data\_model\_and\_resource}}}{}{}
str(object=’‘) -\textgreater{} str
str(bytes\_or\_buffer{[}, encoding{[}, errors{]}{]}) -\textgreater{} str

Create a new string object from the given object. If encoding or
errors is specified, then the object must expose a data buffer
that will be decoded using the given encoding and error handler.
Otherwise, returns the result of object.\_\_str\_\_() (if defined)
or repr(object).
encoding defaults to sys.getdefaultencoding().
errors defaults to ‘strict’.

\end{fulllineitems}

\index{count\_protein\_identifiers\_by\_reference() (pypath.core.network.Network method)@\spxentry{count\_protein\_identifiers\_by\_reference()}\spxextra{pypath.core.network.Network method}}

\begin{fulllineitems}
\phantomsection\label{\detokenize{reference:pypath.core.network.Network.count_protein_identifiers_by_reference}}\pysiglinewithargsret{\sphinxbfcode{\sphinxupquote{count\_protein\_identifiers\_by\_reference}}}{}{}
str(object=’‘) -\textgreater{} str
str(bytes\_or\_buffer{[}, encoding{[}, errors{]}{]}) -\textgreater{} str

Create a new string object from the given object. If encoding or
errors is specified, then the object must expose a data buffer
that will be decoded using the given encoding and error handler.
Otherwise, returns the result of object.\_\_str\_\_() (if defined)
or repr(object).
encoding defaults to sys.getdefaultencoding().
errors defaults to ‘strict’.

\end{fulllineitems}

\index{count\_protein\_identifiers\_by\_resource() (pypath.core.network.Network method)@\spxentry{count\_protein\_identifiers\_by\_resource()}\spxextra{pypath.core.network.Network method}}

\begin{fulllineitems}
\phantomsection\label{\detokenize{reference:pypath.core.network.Network.count_protein_identifiers_by_resource}}\pysiglinewithargsret{\sphinxbfcode{\sphinxupquote{count\_protein\_identifiers\_by\_resource}}}{}{}
str(object=’‘) -\textgreater{} str
str(bytes\_or\_buffer{[}, encoding{[}, errors{]}{]}) -\textgreater{} str

Create a new string object from the given object. If encoding or
errors is specified, then the object must expose a data buffer
that will be decoded using the given encoding and error handler.
Otherwise, returns the result of object.\_\_str\_\_() (if defined)
or repr(object).
encoding defaults to sys.getdefaultencoding().
errors defaults to ‘strict’.

\end{fulllineitems}

\index{count\_protein\_labels() (pypath.core.network.Network method)@\spxentry{count\_protein\_labels()}\spxextra{pypath.core.network.Network method}}

\begin{fulllineitems}
\phantomsection\label{\detokenize{reference:pypath.core.network.Network.count_protein_labels}}\pysiglinewithargsret{\sphinxbfcode{\sphinxupquote{count\_protein\_labels}}}{}{}
str(object=’‘) -\textgreater{} str
str(bytes\_or\_buffer{[}, encoding{[}, errors{]}{]}) -\textgreater{} str

Create a new string object from the given object. If encoding or
errors is specified, then the object must expose a data buffer
that will be decoded using the given encoding and error handler.
Otherwise, returns the result of object.\_\_str\_\_() (if defined)
or repr(object).
encoding defaults to sys.getdefaultencoding().
errors defaults to ‘strict’.

\end{fulllineitems}

\index{count\_protein\_labels\_by\_data\_model() (pypath.core.network.Network method)@\spxentry{count\_protein\_labels\_by\_data\_model()}\spxextra{pypath.core.network.Network method}}

\begin{fulllineitems}
\phantomsection\label{\detokenize{reference:pypath.core.network.Network.count_protein_labels_by_data_model}}\pysiglinewithargsret{\sphinxbfcode{\sphinxupquote{count\_protein\_labels\_by\_data\_model}}}{}{}
str(object=’‘) -\textgreater{} str
str(bytes\_or\_buffer{[}, encoding{[}, errors{]}{]}) -\textgreater{} str

Create a new string object from the given object. If encoding or
errors is specified, then the object must expose a data buffer
that will be decoded using the given encoding and error handler.
Otherwise, returns the result of object.\_\_str\_\_() (if defined)
or repr(object).
encoding defaults to sys.getdefaultencoding().
errors defaults to ‘strict’.

\end{fulllineitems}

\index{count\_protein\_labels\_by\_interaction\_type() (pypath.core.network.Network method)@\spxentry{count\_protein\_labels\_by\_interaction\_type()}\spxextra{pypath.core.network.Network method}}

\begin{fulllineitems}
\phantomsection\label{\detokenize{reference:pypath.core.network.Network.count_protein_labels_by_interaction_type}}\pysiglinewithargsret{\sphinxbfcode{\sphinxupquote{count\_protein\_labels\_by\_interaction\_type}}}{}{}
str(object=’‘) -\textgreater{} str
str(bytes\_or\_buffer{[}, encoding{[}, errors{]}{]}) -\textgreater{} str

Create a new string object from the given object. If encoding or
errors is specified, then the object must expose a data buffer
that will be decoded using the given encoding and error handler.
Otherwise, returns the result of object.\_\_str\_\_() (if defined)
or repr(object).
encoding defaults to sys.getdefaultencoding().
errors defaults to ‘strict’.

\end{fulllineitems}

\index{count\_protein\_labels\_by\_interaction\_type\_and\_data\_model() (pypath.core.network.Network method)@\spxentry{count\_protein\_labels\_by\_interaction\_type\_and\_data\_model()}\spxextra{pypath.core.network.Network method}}

\begin{fulllineitems}
\phantomsection\label{\detokenize{reference:pypath.core.network.Network.count_protein_labels_by_interaction_type_and_data_model}}\pysiglinewithargsret{\sphinxbfcode{\sphinxupquote{count\_protein\_labels\_by\_interaction\_type\_and\_data\_model}}}{}{}
str(object=’‘) -\textgreater{} str
str(bytes\_or\_buffer{[}, encoding{[}, errors{]}{]}) -\textgreater{} str

Create a new string object from the given object. If encoding or
errors is specified, then the object must expose a data buffer
that will be decoded using the given encoding and error handler.
Otherwise, returns the result of object.\_\_str\_\_() (if defined)
or repr(object).
encoding defaults to sys.getdefaultencoding().
errors defaults to ‘strict’.

\end{fulllineitems}

\index{count\_protein\_labels\_by\_interaction\_type\_and\_data\_model\_and\_resource() (pypath.core.network.Network method)@\spxentry{count\_protein\_labels\_by\_interaction\_type\_and\_data\_model\_and\_resource()}\spxextra{pypath.core.network.Network method}}

\begin{fulllineitems}
\phantomsection\label{\detokenize{reference:pypath.core.network.Network.count_protein_labels_by_interaction_type_and_data_model_and_resource}}\pysiglinewithargsret{\sphinxbfcode{\sphinxupquote{count\_protein\_labels\_by\_interaction\_type\_and\_data\_model\_and\_resource}}}{}{}
str(object=’‘) -\textgreater{} str
str(bytes\_or\_buffer{[}, encoding{[}, errors{]}{]}) -\textgreater{} str

Create a new string object from the given object. If encoding or
errors is specified, then the object must expose a data buffer
that will be decoded using the given encoding and error handler.
Otherwise, returns the result of object.\_\_str\_\_() (if defined)
or repr(object).
encoding defaults to sys.getdefaultencoding().
errors defaults to ‘strict’.

\end{fulllineitems}

\index{count\_protein\_labels\_by\_reference() (pypath.core.network.Network method)@\spxentry{count\_protein\_labels\_by\_reference()}\spxextra{pypath.core.network.Network method}}

\begin{fulllineitems}
\phantomsection\label{\detokenize{reference:pypath.core.network.Network.count_protein_labels_by_reference}}\pysiglinewithargsret{\sphinxbfcode{\sphinxupquote{count\_protein\_labels\_by\_reference}}}{}{}
str(object=’‘) -\textgreater{} str
str(bytes\_or\_buffer{[}, encoding{[}, errors{]}{]}) -\textgreater{} str

Create a new string object from the given object. If encoding or
errors is specified, then the object must expose a data buffer
that will be decoded using the given encoding and error handler.
Otherwise, returns the result of object.\_\_str\_\_() (if defined)
or repr(object).
encoding defaults to sys.getdefaultencoding().
errors defaults to ‘strict’.

\end{fulllineitems}

\index{count\_protein\_labels\_by\_resource() (pypath.core.network.Network method)@\spxentry{count\_protein\_labels\_by\_resource()}\spxextra{pypath.core.network.Network method}}

\begin{fulllineitems}
\phantomsection\label{\detokenize{reference:pypath.core.network.Network.count_protein_labels_by_resource}}\pysiglinewithargsret{\sphinxbfcode{\sphinxupquote{count\_protein\_labels\_by\_resource}}}{}{}
str(object=’‘) -\textgreater{} str
str(bytes\_or\_buffer{[}, encoding{[}, errors{]}{]}) -\textgreater{} str

Create a new string object from the given object. If encoding or
errors is specified, then the object must expose a data buffer
that will be decoded using the given encoding and error handler.
Otherwise, returns the result of object.\_\_str\_\_() (if defined)
or repr(object).
encoding defaults to sys.getdefaultencoding().
errors defaults to ‘strict’.

\end{fulllineitems}

\index{count\_proteins() (pypath.core.network.Network method)@\spxentry{count\_proteins()}\spxextra{pypath.core.network.Network method}}

\begin{fulllineitems}
\phantomsection\label{\detokenize{reference:pypath.core.network.Network.count_proteins}}\pysiglinewithargsret{\sphinxbfcode{\sphinxupquote{count\_proteins}}}{}{}
str(object=’‘) -\textgreater{} str
str(bytes\_or\_buffer{[}, encoding{[}, errors{]}{]}) -\textgreater{} str

Create a new string object from the given object. If encoding or
errors is specified, then the object must expose a data buffer
that will be decoded using the given encoding and error handler.
Otherwise, returns the result of object.\_\_str\_\_() (if defined)
or repr(object).
encoding defaults to sys.getdefaultencoding().
errors defaults to ‘strict’.

\end{fulllineitems}

\index{count\_proteins\_by\_data\_model() (pypath.core.network.Network method)@\spxentry{count\_proteins\_by\_data\_model()}\spxextra{pypath.core.network.Network method}}

\begin{fulllineitems}
\phantomsection\label{\detokenize{reference:pypath.core.network.Network.count_proteins_by_data_model}}\pysiglinewithargsret{\sphinxbfcode{\sphinxupquote{count\_proteins\_by\_data\_model}}}{}{}
str(object=’‘) -\textgreater{} str
str(bytes\_or\_buffer{[}, encoding{[}, errors{]}{]}) -\textgreater{} str

Create a new string object from the given object. If encoding or
errors is specified, then the object must expose a data buffer
that will be decoded using the given encoding and error handler.
Otherwise, returns the result of object.\_\_str\_\_() (if defined)
or repr(object).
encoding defaults to sys.getdefaultencoding().
errors defaults to ‘strict’.

\end{fulllineitems}

\index{count\_proteins\_by\_interaction\_type() (pypath.core.network.Network method)@\spxentry{count\_proteins\_by\_interaction\_type()}\spxextra{pypath.core.network.Network method}}

\begin{fulllineitems}
\phantomsection\label{\detokenize{reference:pypath.core.network.Network.count_proteins_by_interaction_type}}\pysiglinewithargsret{\sphinxbfcode{\sphinxupquote{count\_proteins\_by\_interaction\_type}}}{}{}
str(object=’‘) -\textgreater{} str
str(bytes\_or\_buffer{[}, encoding{[}, errors{]}{]}) -\textgreater{} str

Create a new string object from the given object. If encoding or
errors is specified, then the object must expose a data buffer
that will be decoded using the given encoding and error handler.
Otherwise, returns the result of object.\_\_str\_\_() (if defined)
or repr(object).
encoding defaults to sys.getdefaultencoding().
errors defaults to ‘strict’.

\end{fulllineitems}

\index{count\_proteins\_by\_interaction\_type\_and\_data\_model() (pypath.core.network.Network method)@\spxentry{count\_proteins\_by\_interaction\_type\_and\_data\_model()}\spxextra{pypath.core.network.Network method}}

\begin{fulllineitems}
\phantomsection\label{\detokenize{reference:pypath.core.network.Network.count_proteins_by_interaction_type_and_data_model}}\pysiglinewithargsret{\sphinxbfcode{\sphinxupquote{count\_proteins\_by\_interaction\_type\_and\_data\_model}}}{}{}
str(object=’‘) -\textgreater{} str
str(bytes\_or\_buffer{[}, encoding{[}, errors{]}{]}) -\textgreater{} str

Create a new string object from the given object. If encoding or
errors is specified, then the object must expose a data buffer
that will be decoded using the given encoding and error handler.
Otherwise, returns the result of object.\_\_str\_\_() (if defined)
or repr(object).
encoding defaults to sys.getdefaultencoding().
errors defaults to ‘strict’.

\end{fulllineitems}

\index{count\_proteins\_by\_interaction\_type\_and\_data\_model\_and\_resource() (pypath.core.network.Network method)@\spxentry{count\_proteins\_by\_interaction\_type\_and\_data\_model\_and\_resource()}\spxextra{pypath.core.network.Network method}}

\begin{fulllineitems}
\phantomsection\label{\detokenize{reference:pypath.core.network.Network.count_proteins_by_interaction_type_and_data_model_and_resource}}\pysiglinewithargsret{\sphinxbfcode{\sphinxupquote{count\_proteins\_by\_interaction\_type\_and\_data\_model\_and\_resource}}}{}{}
str(object=’‘) -\textgreater{} str
str(bytes\_or\_buffer{[}, encoding{[}, errors{]}{]}) -\textgreater{} str

Create a new string object from the given object. If encoding or
errors is specified, then the object must expose a data buffer
that will be decoded using the given encoding and error handler.
Otherwise, returns the result of object.\_\_str\_\_() (if defined)
or repr(object).
encoding defaults to sys.getdefaultencoding().
errors defaults to ‘strict’.

\end{fulllineitems}

\index{count\_proteins\_by\_reference() (pypath.core.network.Network method)@\spxentry{count\_proteins\_by\_reference()}\spxextra{pypath.core.network.Network method}}

\begin{fulllineitems}
\phantomsection\label{\detokenize{reference:pypath.core.network.Network.count_proteins_by_reference}}\pysiglinewithargsret{\sphinxbfcode{\sphinxupquote{count\_proteins\_by\_reference}}}{}{}
str(object=’‘) -\textgreater{} str
str(bytes\_or\_buffer{[}, encoding{[}, errors{]}{]}) -\textgreater{} str

Create a new string object from the given object. If encoding or
errors is specified, then the object must expose a data buffer
that will be decoded using the given encoding and error handler.
Otherwise, returns the result of object.\_\_str\_\_() (if defined)
or repr(object).
encoding defaults to sys.getdefaultencoding().
errors defaults to ‘strict’.

\end{fulllineitems}

\index{count\_proteins\_by\_resource() (pypath.core.network.Network method)@\spxentry{count\_proteins\_by\_resource()}\spxextra{pypath.core.network.Network method}}

\begin{fulllineitems}
\phantomsection\label{\detokenize{reference:pypath.core.network.Network.count_proteins_by_resource}}\pysiglinewithargsret{\sphinxbfcode{\sphinxupquote{count\_proteins\_by\_resource}}}{}{}
str(object=’‘) -\textgreater{} str
str(bytes\_or\_buffer{[}, encoding{[}, errors{]}{]}) -\textgreater{} str

Create a new string object from the given object. If encoding or
errors is specified, then the object must expose a data buffer
that will be decoded using the given encoding and error handler.
Otherwise, returns the result of object.\_\_str\_\_() (if defined)
or repr(object).
encoding defaults to sys.getdefaultencoding().
errors defaults to ‘strict’.

\end{fulllineitems}

\index{count\_references() (pypath.core.network.Network method)@\spxentry{count\_references()}\spxextra{pypath.core.network.Network method}}

\begin{fulllineitems}
\phantomsection\label{\detokenize{reference:pypath.core.network.Network.count_references}}\pysiglinewithargsret{\sphinxbfcode{\sphinxupquote{count\_references}}}{}{}
str(object=’‘) -\textgreater{} str
str(bytes\_or\_buffer{[}, encoding{[}, errors{]}{]}) -\textgreater{} str

Create a new string object from the given object. If encoding or
errors is specified, then the object must expose a data buffer
that will be decoded using the given encoding and error handler.
Otherwise, returns the result of object.\_\_str\_\_() (if defined)
or repr(object).
encoding defaults to sys.getdefaultencoding().
errors defaults to ‘strict’.

\end{fulllineitems}

\index{count\_references\_by\_data\_model() (pypath.core.network.Network method)@\spxentry{count\_references\_by\_data\_model()}\spxextra{pypath.core.network.Network method}}

\begin{fulllineitems}
\phantomsection\label{\detokenize{reference:pypath.core.network.Network.count_references_by_data_model}}\pysiglinewithargsret{\sphinxbfcode{\sphinxupquote{count\_references\_by\_data\_model}}}{}{}
str(object=’‘) -\textgreater{} str
str(bytes\_or\_buffer{[}, encoding{[}, errors{]}{]}) -\textgreater{} str

Create a new string object from the given object. If encoding or
errors is specified, then the object must expose a data buffer
that will be decoded using the given encoding and error handler.
Otherwise, returns the result of object.\_\_str\_\_() (if defined)
or repr(object).
encoding defaults to sys.getdefaultencoding().
errors defaults to ‘strict’.

\end{fulllineitems}

\index{count\_references\_by\_interaction\_type() (pypath.core.network.Network method)@\spxentry{count\_references\_by\_interaction\_type()}\spxextra{pypath.core.network.Network method}}

\begin{fulllineitems}
\phantomsection\label{\detokenize{reference:pypath.core.network.Network.count_references_by_interaction_type}}\pysiglinewithargsret{\sphinxbfcode{\sphinxupquote{count\_references\_by\_interaction\_type}}}{}{}
str(object=’‘) -\textgreater{} str
str(bytes\_or\_buffer{[}, encoding{[}, errors{]}{]}) -\textgreater{} str

Create a new string object from the given object. If encoding or
errors is specified, then the object must expose a data buffer
that will be decoded using the given encoding and error handler.
Otherwise, returns the result of object.\_\_str\_\_() (if defined)
or repr(object).
encoding defaults to sys.getdefaultencoding().
errors defaults to ‘strict’.

\end{fulllineitems}

\index{count\_references\_by\_interaction\_type\_and\_data\_model() (pypath.core.network.Network method)@\spxentry{count\_references\_by\_interaction\_type\_and\_data\_model()}\spxextra{pypath.core.network.Network method}}

\begin{fulllineitems}
\phantomsection\label{\detokenize{reference:pypath.core.network.Network.count_references_by_interaction_type_and_data_model}}\pysiglinewithargsret{\sphinxbfcode{\sphinxupquote{count\_references\_by\_interaction\_type\_and\_data\_model}}}{}{}
str(object=’‘) -\textgreater{} str
str(bytes\_or\_buffer{[}, encoding{[}, errors{]}{]}) -\textgreater{} str

Create a new string object from the given object. If encoding or
errors is specified, then the object must expose a data buffer
that will be decoded using the given encoding and error handler.
Otherwise, returns the result of object.\_\_str\_\_() (if defined)
or repr(object).
encoding defaults to sys.getdefaultencoding().
errors defaults to ‘strict’.

\end{fulllineitems}

\index{count\_references\_by\_interaction\_type\_and\_data\_model\_and\_resource() (pypath.core.network.Network method)@\spxentry{count\_references\_by\_interaction\_type\_and\_data\_model\_and\_resource()}\spxextra{pypath.core.network.Network method}}

\begin{fulllineitems}
\phantomsection\label{\detokenize{reference:pypath.core.network.Network.count_references_by_interaction_type_and_data_model_and_resource}}\pysiglinewithargsret{\sphinxbfcode{\sphinxupquote{count\_references\_by\_interaction\_type\_and\_data\_model\_and\_resource}}}{}{}
str(object=’‘) -\textgreater{} str
str(bytes\_or\_buffer{[}, encoding{[}, errors{]}{]}) -\textgreater{} str

Create a new string object from the given object. If encoding or
errors is specified, then the object must expose a data buffer
that will be decoded using the given encoding and error handler.
Otherwise, returns the result of object.\_\_str\_\_() (if defined)
or repr(object).
encoding defaults to sys.getdefaultencoding().
errors defaults to ‘strict’.

\end{fulllineitems}

\index{count\_references\_by\_reference() (pypath.core.network.Network method)@\spxentry{count\_references\_by\_reference()}\spxextra{pypath.core.network.Network method}}

\begin{fulllineitems}
\phantomsection\label{\detokenize{reference:pypath.core.network.Network.count_references_by_reference}}\pysiglinewithargsret{\sphinxbfcode{\sphinxupquote{count\_references\_by\_reference}}}{}{}
str(object=’‘) -\textgreater{} str
str(bytes\_or\_buffer{[}, encoding{[}, errors{]}{]}) -\textgreater{} str

Create a new string object from the given object. If encoding or
errors is specified, then the object must expose a data buffer
that will be decoded using the given encoding and error handler.
Otherwise, returns the result of object.\_\_str\_\_() (if defined)
or repr(object).
encoding defaults to sys.getdefaultencoding().
errors defaults to ‘strict’.

\end{fulllineitems}

\index{count\_references\_by\_resource() (pypath.core.network.Network method)@\spxentry{count\_references\_by\_resource()}\spxextra{pypath.core.network.Network method}}

\begin{fulllineitems}
\phantomsection\label{\detokenize{reference:pypath.core.network.Network.count_references_by_resource}}\pysiglinewithargsret{\sphinxbfcode{\sphinxupquote{count\_references\_by\_resource}}}{}{}
str(object=’‘) -\textgreater{} str
str(bytes\_or\_buffer{[}, encoding{[}, errors{]}{]}) -\textgreater{} str

Create a new string object from the given object. If encoding or
errors is specified, then the object must expose a data buffer
that will be decoded using the given encoding and error handler.
Otherwise, returns the result of object.\_\_str\_\_() (if defined)
or repr(object).
encoding defaults to sys.getdefaultencoding().
errors defaults to ‘strict’.

\end{fulllineitems}

\index{count\_regulated\_by() (pypath.core.network.Network method)@\spxentry{count\_regulated\_by()}\spxextra{pypath.core.network.Network method}}

\begin{fulllineitems}
\phantomsection\label{\detokenize{reference:pypath.core.network.Network.count_regulated_by}}\pysiglinewithargsret{\sphinxbfcode{\sphinxupquote{count\_regulated\_by}}}{}{}
Returns the count of the interacting partners for one or more
entities according to the specified criteria.
Please refer to the docs of the \sphinxcode{\sphinxupquote{partners}} method.

\end{fulllineitems}

\index{count\_regulates() (pypath.core.network.Network method)@\spxentry{count\_regulates()}\spxextra{pypath.core.network.Network method}}

\begin{fulllineitems}
\phantomsection\label{\detokenize{reference:pypath.core.network.Network.count_regulates}}\pysiglinewithargsret{\sphinxbfcode{\sphinxupquote{count\_regulates}}}{}{}
Returns the count of the interacting partners for one or more
entities according to the specified criteria.
Please refer to the docs of the \sphinxcode{\sphinxupquote{partners}} method.

\end{fulllineitems}

\index{count\_resource\_names() (pypath.core.network.Network method)@\spxentry{count\_resource\_names()}\spxextra{pypath.core.network.Network method}}

\begin{fulllineitems}
\phantomsection\label{\detokenize{reference:pypath.core.network.Network.count_resource_names}}\pysiglinewithargsret{\sphinxbfcode{\sphinxupquote{count\_resource\_names}}}{}{}
str(object=’‘) -\textgreater{} str
str(bytes\_or\_buffer{[}, encoding{[}, errors{]}{]}) -\textgreater{} str

Create a new string object from the given object. If encoding or
errors is specified, then the object must expose a data buffer
that will be decoded using the given encoding and error handler.
Otherwise, returns the result of object.\_\_str\_\_() (if defined)
or repr(object).
encoding defaults to sys.getdefaultencoding().
errors defaults to ‘strict’.

\end{fulllineitems}

\index{count\_resource\_names\_by\_data\_model() (pypath.core.network.Network method)@\spxentry{count\_resource\_names\_by\_data\_model()}\spxextra{pypath.core.network.Network method}}

\begin{fulllineitems}
\phantomsection\label{\detokenize{reference:pypath.core.network.Network.count_resource_names_by_data_model}}\pysiglinewithargsret{\sphinxbfcode{\sphinxupquote{count\_resource\_names\_by\_data\_model}}}{}{}
str(object=’‘) -\textgreater{} str
str(bytes\_or\_buffer{[}, encoding{[}, errors{]}{]}) -\textgreater{} str

Create a new string object from the given object. If encoding or
errors is specified, then the object must expose a data buffer
that will be decoded using the given encoding and error handler.
Otherwise, returns the result of object.\_\_str\_\_() (if defined)
or repr(object).
encoding defaults to sys.getdefaultencoding().
errors defaults to ‘strict’.

\end{fulllineitems}

\index{count\_resource\_names\_by\_interaction\_type() (pypath.core.network.Network method)@\spxentry{count\_resource\_names\_by\_interaction\_type()}\spxextra{pypath.core.network.Network method}}

\begin{fulllineitems}
\phantomsection\label{\detokenize{reference:pypath.core.network.Network.count_resource_names_by_interaction_type}}\pysiglinewithargsret{\sphinxbfcode{\sphinxupquote{count\_resource\_names\_by\_interaction\_type}}}{}{}
str(object=’‘) -\textgreater{} str
str(bytes\_or\_buffer{[}, encoding{[}, errors{]}{]}) -\textgreater{} str

Create a new string object from the given object. If encoding or
errors is specified, then the object must expose a data buffer
that will be decoded using the given encoding and error handler.
Otherwise, returns the result of object.\_\_str\_\_() (if defined)
or repr(object).
encoding defaults to sys.getdefaultencoding().
errors defaults to ‘strict’.

\end{fulllineitems}

\index{count\_resource\_names\_by\_interaction\_type\_and\_data\_model() (pypath.core.network.Network method)@\spxentry{count\_resource\_names\_by\_interaction\_type\_and\_data\_model()}\spxextra{pypath.core.network.Network method}}

\begin{fulllineitems}
\phantomsection\label{\detokenize{reference:pypath.core.network.Network.count_resource_names_by_interaction_type_and_data_model}}\pysiglinewithargsret{\sphinxbfcode{\sphinxupquote{count\_resource\_names\_by\_interaction\_type\_and\_data\_model}}}{}{}
str(object=’‘) -\textgreater{} str
str(bytes\_or\_buffer{[}, encoding{[}, errors{]}{]}) -\textgreater{} str

Create a new string object from the given object. If encoding or
errors is specified, then the object must expose a data buffer
that will be decoded using the given encoding and error handler.
Otherwise, returns the result of object.\_\_str\_\_() (if defined)
or repr(object).
encoding defaults to sys.getdefaultencoding().
errors defaults to ‘strict’.

\end{fulllineitems}

\index{count\_resource\_names\_by\_interaction\_type\_and\_data\_model\_and\_resource() (pypath.core.network.Network method)@\spxentry{count\_resource\_names\_by\_interaction\_type\_and\_data\_model\_and\_resource()}\spxextra{pypath.core.network.Network method}}

\begin{fulllineitems}
\phantomsection\label{\detokenize{reference:pypath.core.network.Network.count_resource_names_by_interaction_type_and_data_model_and_resource}}\pysiglinewithargsret{\sphinxbfcode{\sphinxupquote{count\_resource\_names\_by\_interaction\_type\_and\_data\_model\_and\_resource}}}{}{}
str(object=’‘) -\textgreater{} str
str(bytes\_or\_buffer{[}, encoding{[}, errors{]}{]}) -\textgreater{} str

Create a new string object from the given object. If encoding or
errors is specified, then the object must expose a data buffer
that will be decoded using the given encoding and error handler.
Otherwise, returns the result of object.\_\_str\_\_() (if defined)
or repr(object).
encoding defaults to sys.getdefaultencoding().
errors defaults to ‘strict’.

\end{fulllineitems}

\index{count\_resource\_names\_by\_reference() (pypath.core.network.Network method)@\spxentry{count\_resource\_names\_by\_reference()}\spxextra{pypath.core.network.Network method}}

\begin{fulllineitems}
\phantomsection\label{\detokenize{reference:pypath.core.network.Network.count_resource_names_by_reference}}\pysiglinewithargsret{\sphinxbfcode{\sphinxupquote{count\_resource\_names\_by\_reference}}}{}{}
str(object=’‘) -\textgreater{} str
str(bytes\_or\_buffer{[}, encoding{[}, errors{]}{]}) -\textgreater{} str

Create a new string object from the given object. If encoding or
errors is specified, then the object must expose a data buffer
that will be decoded using the given encoding and error handler.
Otherwise, returns the result of object.\_\_str\_\_() (if defined)
or repr(object).
encoding defaults to sys.getdefaultencoding().
errors defaults to ‘strict’.

\end{fulllineitems}

\index{count\_resource\_names\_by\_resource() (pypath.core.network.Network method)@\spxentry{count\_resource\_names\_by\_resource()}\spxextra{pypath.core.network.Network method}}

\begin{fulllineitems}
\phantomsection\label{\detokenize{reference:pypath.core.network.Network.count_resource_names_by_resource}}\pysiglinewithargsret{\sphinxbfcode{\sphinxupquote{count\_resource\_names\_by\_resource}}}{}{}
str(object=’‘) -\textgreater{} str
str(bytes\_or\_buffer{[}, encoding{[}, errors{]}{]}) -\textgreater{} str

Create a new string object from the given object. If encoding or
errors is specified, then the object must expose a data buffer
that will be decoded using the given encoding and error handler.
Otherwise, returns the result of object.\_\_str\_\_() (if defined)
or repr(object).
encoding defaults to sys.getdefaultencoding().
errors defaults to ‘strict’.

\end{fulllineitems}

\index{count\_resource\_names\_via() (pypath.core.network.Network method)@\spxentry{count\_resource\_names\_via()}\spxextra{pypath.core.network.Network method}}

\begin{fulllineitems}
\phantomsection\label{\detokenize{reference:pypath.core.network.Network.count_resource_names_via}}\pysiglinewithargsret{\sphinxbfcode{\sphinxupquote{count\_resource\_names\_via}}}{}{}
str(object=’‘) -\textgreater{} str
str(bytes\_or\_buffer{[}, encoding{[}, errors{]}{]}) -\textgreater{} str

Create a new string object from the given object. If encoding or
errors is specified, then the object must expose a data buffer
that will be decoded using the given encoding and error handler.
Otherwise, returns the result of object.\_\_str\_\_() (if defined)
or repr(object).
encoding defaults to sys.getdefaultencoding().
errors defaults to ‘strict’.

\end{fulllineitems}

\index{count\_resource\_names\_via\_by\_data\_model() (pypath.core.network.Network method)@\spxentry{count\_resource\_names\_via\_by\_data\_model()}\spxextra{pypath.core.network.Network method}}

\begin{fulllineitems}
\phantomsection\label{\detokenize{reference:pypath.core.network.Network.count_resource_names_via_by_data_model}}\pysiglinewithargsret{\sphinxbfcode{\sphinxupquote{count\_resource\_names\_via\_by\_data\_model}}}{}{}
str(object=’‘) -\textgreater{} str
str(bytes\_or\_buffer{[}, encoding{[}, errors{]}{]}) -\textgreater{} str

Create a new string object from the given object. If encoding or
errors is specified, then the object must expose a data buffer
that will be decoded using the given encoding and error handler.
Otherwise, returns the result of object.\_\_str\_\_() (if defined)
or repr(object).
encoding defaults to sys.getdefaultencoding().
errors defaults to ‘strict’.

\end{fulllineitems}

\index{count\_resource\_names\_via\_by\_interaction\_type() (pypath.core.network.Network method)@\spxentry{count\_resource\_names\_via\_by\_interaction\_type()}\spxextra{pypath.core.network.Network method}}

\begin{fulllineitems}
\phantomsection\label{\detokenize{reference:pypath.core.network.Network.count_resource_names_via_by_interaction_type}}\pysiglinewithargsret{\sphinxbfcode{\sphinxupquote{count\_resource\_names\_via\_by\_interaction\_type}}}{}{}
str(object=’‘) -\textgreater{} str
str(bytes\_or\_buffer{[}, encoding{[}, errors{]}{]}) -\textgreater{} str

Create a new string object from the given object. If encoding or
errors is specified, then the object must expose a data buffer
that will be decoded using the given encoding and error handler.
Otherwise, returns the result of object.\_\_str\_\_() (if defined)
or repr(object).
encoding defaults to sys.getdefaultencoding().
errors defaults to ‘strict’.

\end{fulllineitems}

\index{count\_resource\_names\_via\_by\_interaction\_type\_and\_data\_model() (pypath.core.network.Network method)@\spxentry{count\_resource\_names\_via\_by\_interaction\_type\_and\_data\_model()}\spxextra{pypath.core.network.Network method}}

\begin{fulllineitems}
\phantomsection\label{\detokenize{reference:pypath.core.network.Network.count_resource_names_via_by_interaction_type_and_data_model}}\pysiglinewithargsret{\sphinxbfcode{\sphinxupquote{count\_resource\_names\_via\_by\_interaction\_type\_and\_data\_model}}}{}{}
str(object=’‘) -\textgreater{} str
str(bytes\_or\_buffer{[}, encoding{[}, errors{]}{]}) -\textgreater{} str

Create a new string object from the given object. If encoding or
errors is specified, then the object must expose a data buffer
that will be decoded using the given encoding and error handler.
Otherwise, returns the result of object.\_\_str\_\_() (if defined)
or repr(object).
encoding defaults to sys.getdefaultencoding().
errors defaults to ‘strict’.

\end{fulllineitems}

\index{count\_resource\_names\_via\_by\_interaction\_type\_and\_data\_model\_and\_resource() (pypath.core.network.Network method)@\spxentry{count\_resource\_names\_via\_by\_interaction\_type\_and\_data\_model\_and\_resource()}\spxextra{pypath.core.network.Network method}}

\begin{fulllineitems}
\phantomsection\label{\detokenize{reference:pypath.core.network.Network.count_resource_names_via_by_interaction_type_and_data_model_and_resource}}\pysiglinewithargsret{\sphinxbfcode{\sphinxupquote{count\_resource\_names\_via\_by\_interaction\_type\_and\_data\_model\_and\_resource}}}{}{}
str(object=’‘) -\textgreater{} str
str(bytes\_or\_buffer{[}, encoding{[}, errors{]}{]}) -\textgreater{} str

Create a new string object from the given object. If encoding or
errors is specified, then the object must expose a data buffer
that will be decoded using the given encoding and error handler.
Otherwise, returns the result of object.\_\_str\_\_() (if defined)
or repr(object).
encoding defaults to sys.getdefaultencoding().
errors defaults to ‘strict’.

\end{fulllineitems}

\index{count\_resource\_names\_via\_by\_reference() (pypath.core.network.Network method)@\spxentry{count\_resource\_names\_via\_by\_reference()}\spxextra{pypath.core.network.Network method}}

\begin{fulllineitems}
\phantomsection\label{\detokenize{reference:pypath.core.network.Network.count_resource_names_via_by_reference}}\pysiglinewithargsret{\sphinxbfcode{\sphinxupquote{count\_resource\_names\_via\_by\_reference}}}{}{}
str(object=’‘) -\textgreater{} str
str(bytes\_or\_buffer{[}, encoding{[}, errors{]}{]}) -\textgreater{} str

Create a new string object from the given object. If encoding or
errors is specified, then the object must expose a data buffer
that will be decoded using the given encoding and error handler.
Otherwise, returns the result of object.\_\_str\_\_() (if defined)
or repr(object).
encoding defaults to sys.getdefaultencoding().
errors defaults to ‘strict’.

\end{fulllineitems}

\index{count\_resource\_names\_via\_by\_resource() (pypath.core.network.Network method)@\spxentry{count\_resource\_names\_via\_by\_resource()}\spxextra{pypath.core.network.Network method}}

\begin{fulllineitems}
\phantomsection\label{\detokenize{reference:pypath.core.network.Network.count_resource_names_via_by_resource}}\pysiglinewithargsret{\sphinxbfcode{\sphinxupquote{count\_resource\_names\_via\_by\_resource}}}{}{}
str(object=’‘) -\textgreater{} str
str(bytes\_or\_buffer{[}, encoding{[}, errors{]}{]}) -\textgreater{} str

Create a new string object from the given object. If encoding or
errors is specified, then the object must expose a data buffer
that will be decoded using the given encoding and error handler.
Otherwise, returns the result of object.\_\_str\_\_() (if defined)
or repr(object).
encoding defaults to sys.getdefaultencoding().
errors defaults to ‘strict’.

\end{fulllineitems}

\index{count\_resources() (pypath.core.network.Network method)@\spxentry{count\_resources()}\spxextra{pypath.core.network.Network method}}

\begin{fulllineitems}
\phantomsection\label{\detokenize{reference:pypath.core.network.Network.count_resources}}\pysiglinewithargsret{\sphinxbfcode{\sphinxupquote{count\_resources}}}{}{}
str(object=’‘) -\textgreater{} str
str(bytes\_or\_buffer{[}, encoding{[}, errors{]}{]}) -\textgreater{} str

Create a new string object from the given object. If encoding or
errors is specified, then the object must expose a data buffer
that will be decoded using the given encoding and error handler.
Otherwise, returns the result of object.\_\_str\_\_() (if defined)
or repr(object).
encoding defaults to sys.getdefaultencoding().
errors defaults to ‘strict’.

\end{fulllineitems}

\index{count\_resources\_by\_data\_model() (pypath.core.network.Network method)@\spxentry{count\_resources\_by\_data\_model()}\spxextra{pypath.core.network.Network method}}

\begin{fulllineitems}
\phantomsection\label{\detokenize{reference:pypath.core.network.Network.count_resources_by_data_model}}\pysiglinewithargsret{\sphinxbfcode{\sphinxupquote{count\_resources\_by\_data\_model}}}{}{}
str(object=’‘) -\textgreater{} str
str(bytes\_or\_buffer{[}, encoding{[}, errors{]}{]}) -\textgreater{} str

Create a new string object from the given object. If encoding or
errors is specified, then the object must expose a data buffer
that will be decoded using the given encoding and error handler.
Otherwise, returns the result of object.\_\_str\_\_() (if defined)
or repr(object).
encoding defaults to sys.getdefaultencoding().
errors defaults to ‘strict’.

\end{fulllineitems}

\index{count\_resources\_by\_interaction\_type() (pypath.core.network.Network method)@\spxentry{count\_resources\_by\_interaction\_type()}\spxextra{pypath.core.network.Network method}}

\begin{fulllineitems}
\phantomsection\label{\detokenize{reference:pypath.core.network.Network.count_resources_by_interaction_type}}\pysiglinewithargsret{\sphinxbfcode{\sphinxupquote{count\_resources\_by\_interaction\_type}}}{}{}
str(object=’‘) -\textgreater{} str
str(bytes\_or\_buffer{[}, encoding{[}, errors{]}{]}) -\textgreater{} str

Create a new string object from the given object. If encoding or
errors is specified, then the object must expose a data buffer
that will be decoded using the given encoding and error handler.
Otherwise, returns the result of object.\_\_str\_\_() (if defined)
or repr(object).
encoding defaults to sys.getdefaultencoding().
errors defaults to ‘strict’.

\end{fulllineitems}

\index{count\_resources\_by\_interaction\_type\_and\_data\_model() (pypath.core.network.Network method)@\spxentry{count\_resources\_by\_interaction\_type\_and\_data\_model()}\spxextra{pypath.core.network.Network method}}

\begin{fulllineitems}
\phantomsection\label{\detokenize{reference:pypath.core.network.Network.count_resources_by_interaction_type_and_data_model}}\pysiglinewithargsret{\sphinxbfcode{\sphinxupquote{count\_resources\_by\_interaction\_type\_and\_data\_model}}}{}{}
str(object=’‘) -\textgreater{} str
str(bytes\_or\_buffer{[}, encoding{[}, errors{]}{]}) -\textgreater{} str

Create a new string object from the given object. If encoding or
errors is specified, then the object must expose a data buffer
that will be decoded using the given encoding and error handler.
Otherwise, returns the result of object.\_\_str\_\_() (if defined)
or repr(object).
encoding defaults to sys.getdefaultencoding().
errors defaults to ‘strict’.

\end{fulllineitems}

\index{count\_resources\_by\_interaction\_type\_and\_data\_model\_and\_resource() (pypath.core.network.Network method)@\spxentry{count\_resources\_by\_interaction\_type\_and\_data\_model\_and\_resource()}\spxextra{pypath.core.network.Network method}}

\begin{fulllineitems}
\phantomsection\label{\detokenize{reference:pypath.core.network.Network.count_resources_by_interaction_type_and_data_model_and_resource}}\pysiglinewithargsret{\sphinxbfcode{\sphinxupquote{count\_resources\_by\_interaction\_type\_and\_data\_model\_and\_resource}}}{}{}
str(object=’‘) -\textgreater{} str
str(bytes\_or\_buffer{[}, encoding{[}, errors{]}{]}) -\textgreater{} str

Create a new string object from the given object. If encoding or
errors is specified, then the object must expose a data buffer
that will be decoded using the given encoding and error handler.
Otherwise, returns the result of object.\_\_str\_\_() (if defined)
or repr(object).
encoding defaults to sys.getdefaultencoding().
errors defaults to ‘strict’.

\end{fulllineitems}

\index{count\_resources\_by\_reference() (pypath.core.network.Network method)@\spxentry{count\_resources\_by\_reference()}\spxextra{pypath.core.network.Network method}}

\begin{fulllineitems}
\phantomsection\label{\detokenize{reference:pypath.core.network.Network.count_resources_by_reference}}\pysiglinewithargsret{\sphinxbfcode{\sphinxupquote{count\_resources\_by\_reference}}}{}{}
str(object=’‘) -\textgreater{} str
str(bytes\_or\_buffer{[}, encoding{[}, errors{]}{]}) -\textgreater{} str

Create a new string object from the given object. If encoding or
errors is specified, then the object must expose a data buffer
that will be decoded using the given encoding and error handler.
Otherwise, returns the result of object.\_\_str\_\_() (if defined)
or repr(object).
encoding defaults to sys.getdefaultencoding().
errors defaults to ‘strict’.

\end{fulllineitems}

\index{count\_resources\_by\_resource() (pypath.core.network.Network method)@\spxentry{count\_resources\_by\_resource()}\spxextra{pypath.core.network.Network method}}

\begin{fulllineitems}
\phantomsection\label{\detokenize{reference:pypath.core.network.Network.count_resources_by_resource}}\pysiglinewithargsret{\sphinxbfcode{\sphinxupquote{count\_resources\_by\_resource}}}{}{}
str(object=’‘) -\textgreater{} str
str(bytes\_or\_buffer{[}, encoding{[}, errors{]}{]}) -\textgreater{} str

Create a new string object from the given object. If encoding or
errors is specified, then the object must expose a data buffer
that will be decoded using the given encoding and error handler.
Otherwise, returns the result of object.\_\_str\_\_() (if defined)
or repr(object).
encoding defaults to sys.getdefaultencoding().
errors defaults to ‘strict’.

\end{fulllineitems}

\index{count\_resources\_via() (pypath.core.network.Network method)@\spxentry{count\_resources\_via()}\spxextra{pypath.core.network.Network method}}

\begin{fulllineitems}
\phantomsection\label{\detokenize{reference:pypath.core.network.Network.count_resources_via}}\pysiglinewithargsret{\sphinxbfcode{\sphinxupquote{count\_resources\_via}}}{}{}
str(object=’‘) -\textgreater{} str
str(bytes\_or\_buffer{[}, encoding{[}, errors{]}{]}) -\textgreater{} str

Create a new string object from the given object. If encoding or
errors is specified, then the object must expose a data buffer
that will be decoded using the given encoding and error handler.
Otherwise, returns the result of object.\_\_str\_\_() (if defined)
or repr(object).
encoding defaults to sys.getdefaultencoding().
errors defaults to ‘strict’.

\end{fulllineitems}

\index{count\_resources\_via\_by\_data\_model() (pypath.core.network.Network method)@\spxentry{count\_resources\_via\_by\_data\_model()}\spxextra{pypath.core.network.Network method}}

\begin{fulllineitems}
\phantomsection\label{\detokenize{reference:pypath.core.network.Network.count_resources_via_by_data_model}}\pysiglinewithargsret{\sphinxbfcode{\sphinxupquote{count\_resources\_via\_by\_data\_model}}}{}{}
str(object=’‘) -\textgreater{} str
str(bytes\_or\_buffer{[}, encoding{[}, errors{]}{]}) -\textgreater{} str

Create a new string object from the given object. If encoding or
errors is specified, then the object must expose a data buffer
that will be decoded using the given encoding and error handler.
Otherwise, returns the result of object.\_\_str\_\_() (if defined)
or repr(object).
encoding defaults to sys.getdefaultencoding().
errors defaults to ‘strict’.

\end{fulllineitems}

\index{count\_resources\_via\_by\_interaction\_type() (pypath.core.network.Network method)@\spxentry{count\_resources\_via\_by\_interaction\_type()}\spxextra{pypath.core.network.Network method}}

\begin{fulllineitems}
\phantomsection\label{\detokenize{reference:pypath.core.network.Network.count_resources_via_by_interaction_type}}\pysiglinewithargsret{\sphinxbfcode{\sphinxupquote{count\_resources\_via\_by\_interaction\_type}}}{}{}
str(object=’‘) -\textgreater{} str
str(bytes\_or\_buffer{[}, encoding{[}, errors{]}{]}) -\textgreater{} str

Create a new string object from the given object. If encoding or
errors is specified, then the object must expose a data buffer
that will be decoded using the given encoding and error handler.
Otherwise, returns the result of object.\_\_str\_\_() (if defined)
or repr(object).
encoding defaults to sys.getdefaultencoding().
errors defaults to ‘strict’.

\end{fulllineitems}

\index{count\_resources\_via\_by\_interaction\_type\_and\_data\_model() (pypath.core.network.Network method)@\spxentry{count\_resources\_via\_by\_interaction\_type\_and\_data\_model()}\spxextra{pypath.core.network.Network method}}

\begin{fulllineitems}
\phantomsection\label{\detokenize{reference:pypath.core.network.Network.count_resources_via_by_interaction_type_and_data_model}}\pysiglinewithargsret{\sphinxbfcode{\sphinxupquote{count\_resources\_via\_by\_interaction\_type\_and\_data\_model}}}{}{}
str(object=’‘) -\textgreater{} str
str(bytes\_or\_buffer{[}, encoding{[}, errors{]}{]}) -\textgreater{} str

Create a new string object from the given object. If encoding or
errors is specified, then the object must expose a data buffer
that will be decoded using the given encoding and error handler.
Otherwise, returns the result of object.\_\_str\_\_() (if defined)
or repr(object).
encoding defaults to sys.getdefaultencoding().
errors defaults to ‘strict’.

\end{fulllineitems}

\index{count\_resources\_via\_by\_interaction\_type\_and\_data\_model\_and\_resource() (pypath.core.network.Network method)@\spxentry{count\_resources\_via\_by\_interaction\_type\_and\_data\_model\_and\_resource()}\spxextra{pypath.core.network.Network method}}

\begin{fulllineitems}
\phantomsection\label{\detokenize{reference:pypath.core.network.Network.count_resources_via_by_interaction_type_and_data_model_and_resource}}\pysiglinewithargsret{\sphinxbfcode{\sphinxupquote{count\_resources\_via\_by\_interaction\_type\_and\_data\_model\_and\_resource}}}{}{}
str(object=’‘) -\textgreater{} str
str(bytes\_or\_buffer{[}, encoding{[}, errors{]}{]}) -\textgreater{} str

Create a new string object from the given object. If encoding or
errors is specified, then the object must expose a data buffer
that will be decoded using the given encoding and error handler.
Otherwise, returns the result of object.\_\_str\_\_() (if defined)
or repr(object).
encoding defaults to sys.getdefaultencoding().
errors defaults to ‘strict’.

\end{fulllineitems}

\index{count\_resources\_via\_by\_reference() (pypath.core.network.Network method)@\spxentry{count\_resources\_via\_by\_reference()}\spxextra{pypath.core.network.Network method}}

\begin{fulllineitems}
\phantomsection\label{\detokenize{reference:pypath.core.network.Network.count_resources_via_by_reference}}\pysiglinewithargsret{\sphinxbfcode{\sphinxupquote{count\_resources\_via\_by\_reference}}}{}{}
str(object=’‘) -\textgreater{} str
str(bytes\_or\_buffer{[}, encoding{[}, errors{]}{]}) -\textgreater{} str

Create a new string object from the given object. If encoding or
errors is specified, then the object must expose a data buffer
that will be decoded using the given encoding and error handler.
Otherwise, returns the result of object.\_\_str\_\_() (if defined)
or repr(object).
encoding defaults to sys.getdefaultencoding().
errors defaults to ‘strict’.

\end{fulllineitems}

\index{count\_resources\_via\_by\_resource() (pypath.core.network.Network method)@\spxentry{count\_resources\_via\_by\_resource()}\spxextra{pypath.core.network.Network method}}

\begin{fulllineitems}
\phantomsection\label{\detokenize{reference:pypath.core.network.Network.count_resources_via_by_resource}}\pysiglinewithargsret{\sphinxbfcode{\sphinxupquote{count\_resources\_via\_by\_resource}}}{}{}
str(object=’‘) -\textgreater{} str
str(bytes\_or\_buffer{[}, encoding{[}, errors{]}{]}) -\textgreater{} str

Create a new string object from the given object. If encoding or
errors is specified, then the object must expose a data buffer
that will be decoded using the given encoding and error handler.
Otherwise, returns the result of object.\_\_str\_\_() (if defined)
or repr(object).
encoding defaults to sys.getdefaultencoding().
errors defaults to ‘strict’.

\end{fulllineitems}

\index{count\_small\_molecule\_identifiers() (pypath.core.network.Network method)@\spxentry{count\_small\_molecule\_identifiers()}\spxextra{pypath.core.network.Network method}}

\begin{fulllineitems}
\phantomsection\label{\detokenize{reference:pypath.core.network.Network.count_small_molecule_identifiers}}\pysiglinewithargsret{\sphinxbfcode{\sphinxupquote{count\_small\_molecule\_identifiers}}}{}{}
str(object=’‘) -\textgreater{} str
str(bytes\_or\_buffer{[}, encoding{[}, errors{]}{]}) -\textgreater{} str

Create a new string object from the given object. If encoding or
errors is specified, then the object must expose a data buffer
that will be decoded using the given encoding and error handler.
Otherwise, returns the result of object.\_\_str\_\_() (if defined)
or repr(object).
encoding defaults to sys.getdefaultencoding().
errors defaults to ‘strict’.

\end{fulllineitems}

\index{count\_small\_molecule\_identifiers\_by\_data\_model() (pypath.core.network.Network method)@\spxentry{count\_small\_molecule\_identifiers\_by\_data\_model()}\spxextra{pypath.core.network.Network method}}

\begin{fulllineitems}
\phantomsection\label{\detokenize{reference:pypath.core.network.Network.count_small_molecule_identifiers_by_data_model}}\pysiglinewithargsret{\sphinxbfcode{\sphinxupquote{count\_small\_molecule\_identifiers\_by\_data\_model}}}{}{}
str(object=’‘) -\textgreater{} str
str(bytes\_or\_buffer{[}, encoding{[}, errors{]}{]}) -\textgreater{} str

Create a new string object from the given object. If encoding or
errors is specified, then the object must expose a data buffer
that will be decoded using the given encoding and error handler.
Otherwise, returns the result of object.\_\_str\_\_() (if defined)
or repr(object).
encoding defaults to sys.getdefaultencoding().
errors defaults to ‘strict’.

\end{fulllineitems}

\index{count\_small\_molecule\_identifiers\_by\_interaction\_type() (pypath.core.network.Network method)@\spxentry{count\_small\_molecule\_identifiers\_by\_interaction\_type()}\spxextra{pypath.core.network.Network method}}

\begin{fulllineitems}
\phantomsection\label{\detokenize{reference:pypath.core.network.Network.count_small_molecule_identifiers_by_interaction_type}}\pysiglinewithargsret{\sphinxbfcode{\sphinxupquote{count\_small\_molecule\_identifiers\_by\_interaction\_type}}}{}{}
str(object=’‘) -\textgreater{} str
str(bytes\_or\_buffer{[}, encoding{[}, errors{]}{]}) -\textgreater{} str

Create a new string object from the given object. If encoding or
errors is specified, then the object must expose a data buffer
that will be decoded using the given encoding and error handler.
Otherwise, returns the result of object.\_\_str\_\_() (if defined)
or repr(object).
encoding defaults to sys.getdefaultencoding().
errors defaults to ‘strict’.

\end{fulllineitems}

\index{count\_small\_molecule\_identifiers\_by\_interaction\_type\_and\_data\_model() (pypath.core.network.Network method)@\spxentry{count\_small\_molecule\_identifiers\_by\_interaction\_type\_and\_data\_model()}\spxextra{pypath.core.network.Network method}}

\begin{fulllineitems}
\phantomsection\label{\detokenize{reference:pypath.core.network.Network.count_small_molecule_identifiers_by_interaction_type_and_data_model}}\pysiglinewithargsret{\sphinxbfcode{\sphinxupquote{count\_small\_molecule\_identifiers\_by\_interaction\_type\_and\_data\_model}}}{}{}
str(object=’‘) -\textgreater{} str
str(bytes\_or\_buffer{[}, encoding{[}, errors{]}{]}) -\textgreater{} str

Create a new string object from the given object. If encoding or
errors is specified, then the object must expose a data buffer
that will be decoded using the given encoding and error handler.
Otherwise, returns the result of object.\_\_str\_\_() (if defined)
or repr(object).
encoding defaults to sys.getdefaultencoding().
errors defaults to ‘strict’.

\end{fulllineitems}

\index{count\_small\_molecule\_identifiers\_by\_interaction\_type\_and\_data\_model\_and\_resource() (pypath.core.network.Network method)@\spxentry{count\_small\_molecule\_identifiers\_by\_interaction\_type\_and\_data\_model\_and\_resource()}\spxextra{pypath.core.network.Network method}}

\begin{fulllineitems}
\phantomsection\label{\detokenize{reference:pypath.core.network.Network.count_small_molecule_identifiers_by_interaction_type_and_data_model_and_resource}}\pysiglinewithargsret{\sphinxbfcode{\sphinxupquote{count\_small\_molecule\_identifiers\_by\_interaction\_type\_and\_data\_model\_and\_resource}}}{}{}
str(object=’‘) -\textgreater{} str
str(bytes\_or\_buffer{[}, encoding{[}, errors{]}{]}) -\textgreater{} str

Create a new string object from the given object. If encoding or
errors is specified, then the object must expose a data buffer
that will be decoded using the given encoding and error handler.
Otherwise, returns the result of object.\_\_str\_\_() (if defined)
or repr(object).
encoding defaults to sys.getdefaultencoding().
errors defaults to ‘strict’.

\end{fulllineitems}

\index{count\_small\_molecule\_identifiers\_by\_reference() (pypath.core.network.Network method)@\spxentry{count\_small\_molecule\_identifiers\_by\_reference()}\spxextra{pypath.core.network.Network method}}

\begin{fulllineitems}
\phantomsection\label{\detokenize{reference:pypath.core.network.Network.count_small_molecule_identifiers_by_reference}}\pysiglinewithargsret{\sphinxbfcode{\sphinxupquote{count\_small\_molecule\_identifiers\_by\_reference}}}{}{}
str(object=’‘) -\textgreater{} str
str(bytes\_or\_buffer{[}, encoding{[}, errors{]}{]}) -\textgreater{} str

Create a new string object from the given object. If encoding or
errors is specified, then the object must expose a data buffer
that will be decoded using the given encoding and error handler.
Otherwise, returns the result of object.\_\_str\_\_() (if defined)
or repr(object).
encoding defaults to sys.getdefaultencoding().
errors defaults to ‘strict’.

\end{fulllineitems}

\index{count\_small\_molecule\_identifiers\_by\_resource() (pypath.core.network.Network method)@\spxentry{count\_small\_molecule\_identifiers\_by\_resource()}\spxextra{pypath.core.network.Network method}}

\begin{fulllineitems}
\phantomsection\label{\detokenize{reference:pypath.core.network.Network.count_small_molecule_identifiers_by_resource}}\pysiglinewithargsret{\sphinxbfcode{\sphinxupquote{count\_small\_molecule\_identifiers\_by\_resource}}}{}{}
str(object=’‘) -\textgreater{} str
str(bytes\_or\_buffer{[}, encoding{[}, errors{]}{]}) -\textgreater{} str

Create a new string object from the given object. If encoding or
errors is specified, then the object must expose a data buffer
that will be decoded using the given encoding and error handler.
Otherwise, returns the result of object.\_\_str\_\_() (if defined)
or repr(object).
encoding defaults to sys.getdefaultencoding().
errors defaults to ‘strict’.

\end{fulllineitems}

\index{count\_small\_molecule\_labels() (pypath.core.network.Network method)@\spxentry{count\_small\_molecule\_labels()}\spxextra{pypath.core.network.Network method}}

\begin{fulllineitems}
\phantomsection\label{\detokenize{reference:pypath.core.network.Network.count_small_molecule_labels}}\pysiglinewithargsret{\sphinxbfcode{\sphinxupquote{count\_small\_molecule\_labels}}}{}{}
str(object=’‘) -\textgreater{} str
str(bytes\_or\_buffer{[}, encoding{[}, errors{]}{]}) -\textgreater{} str

Create a new string object from the given object. If encoding or
errors is specified, then the object must expose a data buffer
that will be decoded using the given encoding and error handler.
Otherwise, returns the result of object.\_\_str\_\_() (if defined)
or repr(object).
encoding defaults to sys.getdefaultencoding().
errors defaults to ‘strict’.

\end{fulllineitems}

\index{count\_small\_molecule\_labels\_by\_data\_model() (pypath.core.network.Network method)@\spxentry{count\_small\_molecule\_labels\_by\_data\_model()}\spxextra{pypath.core.network.Network method}}

\begin{fulllineitems}
\phantomsection\label{\detokenize{reference:pypath.core.network.Network.count_small_molecule_labels_by_data_model}}\pysiglinewithargsret{\sphinxbfcode{\sphinxupquote{count\_small\_molecule\_labels\_by\_data\_model}}}{}{}
str(object=’‘) -\textgreater{} str
str(bytes\_or\_buffer{[}, encoding{[}, errors{]}{]}) -\textgreater{} str

Create a new string object from the given object. If encoding or
errors is specified, then the object must expose a data buffer
that will be decoded using the given encoding and error handler.
Otherwise, returns the result of object.\_\_str\_\_() (if defined)
or repr(object).
encoding defaults to sys.getdefaultencoding().
errors defaults to ‘strict’.

\end{fulllineitems}

\index{count\_small\_molecule\_labels\_by\_interaction\_type() (pypath.core.network.Network method)@\spxentry{count\_small\_molecule\_labels\_by\_interaction\_type()}\spxextra{pypath.core.network.Network method}}

\begin{fulllineitems}
\phantomsection\label{\detokenize{reference:pypath.core.network.Network.count_small_molecule_labels_by_interaction_type}}\pysiglinewithargsret{\sphinxbfcode{\sphinxupquote{count\_small\_molecule\_labels\_by\_interaction\_type}}}{}{}
str(object=’‘) -\textgreater{} str
str(bytes\_or\_buffer{[}, encoding{[}, errors{]}{]}) -\textgreater{} str

Create a new string object from the given object. If encoding or
errors is specified, then the object must expose a data buffer
that will be decoded using the given encoding and error handler.
Otherwise, returns the result of object.\_\_str\_\_() (if defined)
or repr(object).
encoding defaults to sys.getdefaultencoding().
errors defaults to ‘strict’.

\end{fulllineitems}

\index{count\_small\_molecule\_labels\_by\_interaction\_type\_and\_data\_model() (pypath.core.network.Network method)@\spxentry{count\_small\_molecule\_labels\_by\_interaction\_type\_and\_data\_model()}\spxextra{pypath.core.network.Network method}}

\begin{fulllineitems}
\phantomsection\label{\detokenize{reference:pypath.core.network.Network.count_small_molecule_labels_by_interaction_type_and_data_model}}\pysiglinewithargsret{\sphinxbfcode{\sphinxupquote{count\_small\_molecule\_labels\_by\_interaction\_type\_and\_data\_model}}}{}{}
str(object=’‘) -\textgreater{} str
str(bytes\_or\_buffer{[}, encoding{[}, errors{]}{]}) -\textgreater{} str

Create a new string object from the given object. If encoding or
errors is specified, then the object must expose a data buffer
that will be decoded using the given encoding and error handler.
Otherwise, returns the result of object.\_\_str\_\_() (if defined)
or repr(object).
encoding defaults to sys.getdefaultencoding().
errors defaults to ‘strict’.

\end{fulllineitems}

\index{count\_small\_molecule\_labels\_by\_interaction\_type\_and\_data\_model\_and\_resource() (pypath.core.network.Network method)@\spxentry{count\_small\_molecule\_labels\_by\_interaction\_type\_and\_data\_model\_and\_resource()}\spxextra{pypath.core.network.Network method}}

\begin{fulllineitems}
\phantomsection\label{\detokenize{reference:pypath.core.network.Network.count_small_molecule_labels_by_interaction_type_and_data_model_and_resource}}\pysiglinewithargsret{\sphinxbfcode{\sphinxupquote{count\_small\_molecule\_labels\_by\_interaction\_type\_and\_data\_model\_and\_resource}}}{}{}
str(object=’‘) -\textgreater{} str
str(bytes\_or\_buffer{[}, encoding{[}, errors{]}{]}) -\textgreater{} str

Create a new string object from the given object. If encoding or
errors is specified, then the object must expose a data buffer
that will be decoded using the given encoding and error handler.
Otherwise, returns the result of object.\_\_str\_\_() (if defined)
or repr(object).
encoding defaults to sys.getdefaultencoding().
errors defaults to ‘strict’.

\end{fulllineitems}

\index{count\_small\_molecule\_labels\_by\_reference() (pypath.core.network.Network method)@\spxentry{count\_small\_molecule\_labels\_by\_reference()}\spxextra{pypath.core.network.Network method}}

\begin{fulllineitems}
\phantomsection\label{\detokenize{reference:pypath.core.network.Network.count_small_molecule_labels_by_reference}}\pysiglinewithargsret{\sphinxbfcode{\sphinxupquote{count\_small\_molecule\_labels\_by\_reference}}}{}{}
str(object=’‘) -\textgreater{} str
str(bytes\_or\_buffer{[}, encoding{[}, errors{]}{]}) -\textgreater{} str

Create a new string object from the given object. If encoding or
errors is specified, then the object must expose a data buffer
that will be decoded using the given encoding and error handler.
Otherwise, returns the result of object.\_\_str\_\_() (if defined)
or repr(object).
encoding defaults to sys.getdefaultencoding().
errors defaults to ‘strict’.

\end{fulllineitems}

\index{count\_small\_molecule\_labels\_by\_resource() (pypath.core.network.Network method)@\spxentry{count\_small\_molecule\_labels\_by\_resource()}\spxextra{pypath.core.network.Network method}}

\begin{fulllineitems}
\phantomsection\label{\detokenize{reference:pypath.core.network.Network.count_small_molecule_labels_by_resource}}\pysiglinewithargsret{\sphinxbfcode{\sphinxupquote{count\_small\_molecule\_labels\_by\_resource}}}{}{}
str(object=’‘) -\textgreater{} str
str(bytes\_or\_buffer{[}, encoding{[}, errors{]}{]}) -\textgreater{} str

Create a new string object from the given object. If encoding or
errors is specified, then the object must expose a data buffer
that will be decoded using the given encoding and error handler.
Otherwise, returns the result of object.\_\_str\_\_() (if defined)
or repr(object).
encoding defaults to sys.getdefaultencoding().
errors defaults to ‘strict’.

\end{fulllineitems}

\index{count\_small\_molecules() (pypath.core.network.Network method)@\spxentry{count\_small\_molecules()}\spxextra{pypath.core.network.Network method}}

\begin{fulllineitems}
\phantomsection\label{\detokenize{reference:pypath.core.network.Network.count_small_molecules}}\pysiglinewithargsret{\sphinxbfcode{\sphinxupquote{count\_small\_molecules}}}{}{}
str(object=’‘) -\textgreater{} str
str(bytes\_or\_buffer{[}, encoding{[}, errors{]}{]}) -\textgreater{} str

Create a new string object from the given object. If encoding or
errors is specified, then the object must expose a data buffer
that will be decoded using the given encoding and error handler.
Otherwise, returns the result of object.\_\_str\_\_() (if defined)
or repr(object).
encoding defaults to sys.getdefaultencoding().
errors defaults to ‘strict’.

\end{fulllineitems}

\index{count\_small\_molecules\_by\_data\_model() (pypath.core.network.Network method)@\spxentry{count\_small\_molecules\_by\_data\_model()}\spxextra{pypath.core.network.Network method}}

\begin{fulllineitems}
\phantomsection\label{\detokenize{reference:pypath.core.network.Network.count_small_molecules_by_data_model}}\pysiglinewithargsret{\sphinxbfcode{\sphinxupquote{count\_small\_molecules\_by\_data\_model}}}{}{}
str(object=’‘) -\textgreater{} str
str(bytes\_or\_buffer{[}, encoding{[}, errors{]}{]}) -\textgreater{} str

Create a new string object from the given object. If encoding or
errors is specified, then the object must expose a data buffer
that will be decoded using the given encoding and error handler.
Otherwise, returns the result of object.\_\_str\_\_() (if defined)
or repr(object).
encoding defaults to sys.getdefaultencoding().
errors defaults to ‘strict’.

\end{fulllineitems}

\index{count\_small\_molecules\_by\_interaction\_type() (pypath.core.network.Network method)@\spxentry{count\_small\_molecules\_by\_interaction\_type()}\spxextra{pypath.core.network.Network method}}

\begin{fulllineitems}
\phantomsection\label{\detokenize{reference:pypath.core.network.Network.count_small_molecules_by_interaction_type}}\pysiglinewithargsret{\sphinxbfcode{\sphinxupquote{count\_small\_molecules\_by\_interaction\_type}}}{}{}
str(object=’‘) -\textgreater{} str
str(bytes\_or\_buffer{[}, encoding{[}, errors{]}{]}) -\textgreater{} str

Create a new string object from the given object. If encoding or
errors is specified, then the object must expose a data buffer
that will be decoded using the given encoding and error handler.
Otherwise, returns the result of object.\_\_str\_\_() (if defined)
or repr(object).
encoding defaults to sys.getdefaultencoding().
errors defaults to ‘strict’.

\end{fulllineitems}

\index{count\_small\_molecules\_by\_interaction\_type\_and\_data\_model() (pypath.core.network.Network method)@\spxentry{count\_small\_molecules\_by\_interaction\_type\_and\_data\_model()}\spxextra{pypath.core.network.Network method}}

\begin{fulllineitems}
\phantomsection\label{\detokenize{reference:pypath.core.network.Network.count_small_molecules_by_interaction_type_and_data_model}}\pysiglinewithargsret{\sphinxbfcode{\sphinxupquote{count\_small\_molecules\_by\_interaction\_type\_and\_data\_model}}}{}{}
str(object=’‘) -\textgreater{} str
str(bytes\_or\_buffer{[}, encoding{[}, errors{]}{]}) -\textgreater{} str

Create a new string object from the given object. If encoding or
errors is specified, then the object must expose a data buffer
that will be decoded using the given encoding and error handler.
Otherwise, returns the result of object.\_\_str\_\_() (if defined)
or repr(object).
encoding defaults to sys.getdefaultencoding().
errors defaults to ‘strict’.

\end{fulllineitems}

\index{count\_small\_molecules\_by\_interaction\_type\_and\_data\_model\_and\_resource() (pypath.core.network.Network method)@\spxentry{count\_small\_molecules\_by\_interaction\_type\_and\_data\_model\_and\_resource()}\spxextra{pypath.core.network.Network method}}

\begin{fulllineitems}
\phantomsection\label{\detokenize{reference:pypath.core.network.Network.count_small_molecules_by_interaction_type_and_data_model_and_resource}}\pysiglinewithargsret{\sphinxbfcode{\sphinxupquote{count\_small\_molecules\_by\_interaction\_type\_and\_data\_model\_and\_resource}}}{}{}
str(object=’‘) -\textgreater{} str
str(bytes\_or\_buffer{[}, encoding{[}, errors{]}{]}) -\textgreater{} str

Create a new string object from the given object. If encoding or
errors is specified, then the object must expose a data buffer
that will be decoded using the given encoding and error handler.
Otherwise, returns the result of object.\_\_str\_\_() (if defined)
or repr(object).
encoding defaults to sys.getdefaultencoding().
errors defaults to ‘strict’.

\end{fulllineitems}

\index{count\_small\_molecules\_by\_reference() (pypath.core.network.Network method)@\spxentry{count\_small\_molecules\_by\_reference()}\spxextra{pypath.core.network.Network method}}

\begin{fulllineitems}
\phantomsection\label{\detokenize{reference:pypath.core.network.Network.count_small_molecules_by_reference}}\pysiglinewithargsret{\sphinxbfcode{\sphinxupquote{count\_small\_molecules\_by\_reference}}}{}{}
str(object=’‘) -\textgreater{} str
str(bytes\_or\_buffer{[}, encoding{[}, errors{]}{]}) -\textgreater{} str

Create a new string object from the given object. If encoding or
errors is specified, then the object must expose a data buffer
that will be decoded using the given encoding and error handler.
Otherwise, returns the result of object.\_\_str\_\_() (if defined)
or repr(object).
encoding defaults to sys.getdefaultencoding().
errors defaults to ‘strict’.

\end{fulllineitems}

\index{count\_small\_molecules\_by\_resource() (pypath.core.network.Network method)@\spxentry{count\_small\_molecules\_by\_resource()}\spxextra{pypath.core.network.Network method}}

\begin{fulllineitems}
\phantomsection\label{\detokenize{reference:pypath.core.network.Network.count_small_molecules_by_resource}}\pysiglinewithargsret{\sphinxbfcode{\sphinxupquote{count\_small\_molecules\_by\_resource}}}{}{}
str(object=’‘) -\textgreater{} str
str(bytes\_or\_buffer{[}, encoding{[}, errors{]}{]}) -\textgreater{} str

Create a new string object from the given object. If encoding or
errors is specified, then the object must expose a data buffer
that will be decoded using the given encoding and error handler.
Otherwise, returns the result of object.\_\_str\_\_() (if defined)
or repr(object).
encoding defaults to sys.getdefaultencoding().
errors defaults to ‘strict’.

\end{fulllineitems}

\index{count\_suppressed\_by() (pypath.core.network.Network method)@\spxentry{count\_suppressed\_by()}\spxextra{pypath.core.network.Network method}}

\begin{fulllineitems}
\phantomsection\label{\detokenize{reference:pypath.core.network.Network.count_suppressed_by}}\pysiglinewithargsret{\sphinxbfcode{\sphinxupquote{count\_suppressed\_by}}}{}{}
Returns the count of the interacting partners for one or more
entities according to the specified criteria.
Please refer to the docs of the \sphinxcode{\sphinxupquote{partners}} method.

\end{fulllineitems}

\index{count\_suppresses() (pypath.core.network.Network method)@\spxentry{count\_suppresses()}\spxextra{pypath.core.network.Network method}}

\begin{fulllineitems}
\phantomsection\label{\detokenize{reference:pypath.core.network.Network.count_suppresses}}\pysiglinewithargsret{\sphinxbfcode{\sphinxupquote{count\_suppresses}}}{}{}
Returns the count of the interacting partners for one or more
entities according to the specified criteria.
Please refer to the docs of the \sphinxcode{\sphinxupquote{partners}} method.

\end{fulllineitems}

\index{count\_transcriptionally\_activated\_by() (pypath.core.network.Network method)@\spxentry{count\_transcriptionally\_activated\_by()}\spxextra{pypath.core.network.Network method}}

\begin{fulllineitems}
\phantomsection\label{\detokenize{reference:pypath.core.network.Network.count_transcriptionally_activated_by}}\pysiglinewithargsret{\sphinxbfcode{\sphinxupquote{count\_transcriptionally\_activated\_by}}}{}{}
Returns the count of the interacting partners for one or more
entities according to the specified criteria.
Please refer to the docs of the \sphinxcode{\sphinxupquote{partners}} method.

\end{fulllineitems}

\index{count\_transcriptionally\_activates() (pypath.core.network.Network method)@\spxentry{count\_transcriptionally\_activates()}\spxextra{pypath.core.network.Network method}}

\begin{fulllineitems}
\phantomsection\label{\detokenize{reference:pypath.core.network.Network.count_transcriptionally_activates}}\pysiglinewithargsret{\sphinxbfcode{\sphinxupquote{count\_transcriptionally\_activates}}}{}{}
Returns the count of the interacting partners for one or more
entities according to the specified criteria.
Please refer to the docs of the \sphinxcode{\sphinxupquote{partners}} method.

\end{fulllineitems}

\index{count\_transcriptionally\_regulated\_by() (pypath.core.network.Network method)@\spxentry{count\_transcriptionally\_regulated\_by()}\spxextra{pypath.core.network.Network method}}

\begin{fulllineitems}
\phantomsection\label{\detokenize{reference:pypath.core.network.Network.count_transcriptionally_regulated_by}}\pysiglinewithargsret{\sphinxbfcode{\sphinxupquote{count\_transcriptionally\_regulated\_by}}}{}{}
Returns the count of the interacting partners for one or more
entities according to the specified criteria.
Please refer to the docs of the \sphinxcode{\sphinxupquote{partners}} method.

\end{fulllineitems}

\index{count\_transcriptionally\_regulates() (pypath.core.network.Network method)@\spxentry{count\_transcriptionally\_regulates()}\spxextra{pypath.core.network.Network method}}

\begin{fulllineitems}
\phantomsection\label{\detokenize{reference:pypath.core.network.Network.count_transcriptionally_regulates}}\pysiglinewithargsret{\sphinxbfcode{\sphinxupquote{count\_transcriptionally\_regulates}}}{}{}
Returns the count of the interacting partners for one or more
entities according to the specified criteria.
Please refer to the docs of the \sphinxcode{\sphinxupquote{partners}} method.

\end{fulllineitems}

\index{count\_transcriptionally\_suppressed\_by() (pypath.core.network.Network method)@\spxentry{count\_transcriptionally\_suppressed\_by()}\spxextra{pypath.core.network.Network method}}

\begin{fulllineitems}
\phantomsection\label{\detokenize{reference:pypath.core.network.Network.count_transcriptionally_suppressed_by}}\pysiglinewithargsret{\sphinxbfcode{\sphinxupquote{count\_transcriptionally\_suppressed\_by}}}{}{}
Returns the count of the interacting partners for one or more
entities according to the specified criteria.
Please refer to the docs of the \sphinxcode{\sphinxupquote{partners}} method.

\end{fulllineitems}

\index{count\_transcriptionally\_suppresses() (pypath.core.network.Network method)@\spxentry{count\_transcriptionally\_suppresses()}\spxextra{pypath.core.network.Network method}}

\begin{fulllineitems}
\phantomsection\label{\detokenize{reference:pypath.core.network.Network.count_transcriptionally_suppresses}}\pysiglinewithargsret{\sphinxbfcode{\sphinxupquote{count\_transcriptionally\_suppresses}}}{}{}
Returns the count of the interacting partners for one or more
entities according to the specified criteria.
Please refer to the docs of the \sphinxcode{\sphinxupquote{partners}} method.

\end{fulllineitems}

\index{curation\_effort\_by\_data\_model() (pypath.core.network.Network method)@\spxentry{curation\_effort\_by\_data\_model()}\spxextra{pypath.core.network.Network method}}

\begin{fulllineitems}
\phantomsection\label{\detokenize{reference:pypath.core.network.Network.curation_effort_by_data_model}}\pysiglinewithargsret{\sphinxbfcode{\sphinxupquote{curation\_effort\_by\_data\_model}}}{}{}
Built-in immutable sequence.

If no argument is given, the constructor returns an empty tuple.
If iterable is specified the tuple is initialized from iterable’s items.

If the argument is a tuple, the return value is the same object.

\end{fulllineitems}

\index{curation\_effort\_by\_interaction\_type() (pypath.core.network.Network method)@\spxentry{curation\_effort\_by\_interaction\_type()}\spxextra{pypath.core.network.Network method}}

\begin{fulllineitems}
\phantomsection\label{\detokenize{reference:pypath.core.network.Network.curation_effort_by_interaction_type}}\pysiglinewithargsret{\sphinxbfcode{\sphinxupquote{curation\_effort\_by\_interaction\_type}}}{}{}
Built-in immutable sequence.

If no argument is given, the constructor returns an empty tuple.
If iterable is specified the tuple is initialized from iterable’s items.

If the argument is a tuple, the return value is the same object.

\end{fulllineitems}

\index{curation\_effort\_by\_interaction\_type\_and\_data\_model() (pypath.core.network.Network method)@\spxentry{curation\_effort\_by\_interaction\_type\_and\_data\_model()}\spxextra{pypath.core.network.Network method}}

\begin{fulllineitems}
\phantomsection\label{\detokenize{reference:pypath.core.network.Network.curation_effort_by_interaction_type_and_data_model}}\pysiglinewithargsret{\sphinxbfcode{\sphinxupquote{curation\_effort\_by\_interaction\_type\_and\_data\_model}}}{}{}
Built-in immutable sequence.

If no argument is given, the constructor returns an empty tuple.
If iterable is specified the tuple is initialized from iterable’s items.

If the argument is a tuple, the return value is the same object.

\end{fulllineitems}

\index{curation\_effort\_by\_interaction\_type\_and\_data\_model\_and\_resource() (pypath.core.network.Network method)@\spxentry{curation\_effort\_by\_interaction\_type\_and\_data\_model\_and\_resource()}\spxextra{pypath.core.network.Network method}}

\begin{fulllineitems}
\phantomsection\label{\detokenize{reference:pypath.core.network.Network.curation_effort_by_interaction_type_and_data_model_and_resource}}\pysiglinewithargsret{\sphinxbfcode{\sphinxupquote{curation\_effort\_by\_interaction\_type\_and\_data\_model\_and\_resource}}}{}{}
Built-in immutable sequence.

If no argument is given, the constructor returns an empty tuple.
If iterable is specified the tuple is initialized from iterable’s items.

If the argument is a tuple, the return value is the same object.

\end{fulllineitems}

\index{curation\_effort\_by\_reference() (pypath.core.network.Network method)@\spxentry{curation\_effort\_by\_reference()}\spxextra{pypath.core.network.Network method}}

\begin{fulllineitems}
\phantomsection\label{\detokenize{reference:pypath.core.network.Network.curation_effort_by_reference}}\pysiglinewithargsret{\sphinxbfcode{\sphinxupquote{curation\_effort\_by\_reference}}}{}{}
Built-in immutable sequence.

If no argument is given, the constructor returns an empty tuple.
If iterable is specified the tuple is initialized from iterable’s items.

If the argument is a tuple, the return value is the same object.

\end{fulllineitems}

\index{curation\_effort\_by\_resource() (pypath.core.network.Network method)@\spxentry{curation\_effort\_by\_resource()}\spxextra{pypath.core.network.Network method}}

\begin{fulllineitems}
\phantomsection\label{\detokenize{reference:pypath.core.network.Network.curation_effort_by_resource}}\pysiglinewithargsret{\sphinxbfcode{\sphinxupquote{curation\_effort\_by\_resource}}}{}{}
Built-in immutable sequence.

If no argument is given, the constructor returns an empty tuple.
If iterable is specified the tuple is initialized from iterable’s items.

If the argument is a tuple, the return value is the same object.

\end{fulllineitems}

\index{data\_models\_by\_data\_model() (pypath.core.network.Network method)@\spxentry{data\_models\_by\_data\_model()}\spxextra{pypath.core.network.Network method}}

\begin{fulllineitems}
\phantomsection\label{\detokenize{reference:pypath.core.network.Network.data_models_by_data_model}}\pysiglinewithargsret{\sphinxbfcode{\sphinxupquote{data\_models\_by\_data\_model}}}{}{}
Built-in immutable sequence.

If no argument is given, the constructor returns an empty tuple.
If iterable is specified the tuple is initialized from iterable’s items.

If the argument is a tuple, the return value is the same object.

\end{fulllineitems}

\index{data\_models\_by\_interaction\_type() (pypath.core.network.Network method)@\spxentry{data\_models\_by\_interaction\_type()}\spxextra{pypath.core.network.Network method}}

\begin{fulllineitems}
\phantomsection\label{\detokenize{reference:pypath.core.network.Network.data_models_by_interaction_type}}\pysiglinewithargsret{\sphinxbfcode{\sphinxupquote{data\_models\_by\_interaction\_type}}}{}{}
Built-in immutable sequence.

If no argument is given, the constructor returns an empty tuple.
If iterable is specified the tuple is initialized from iterable’s items.

If the argument is a tuple, the return value is the same object.

\end{fulllineitems}

\index{data\_models\_by\_interaction\_type\_and\_data\_model() (pypath.core.network.Network method)@\spxentry{data\_models\_by\_interaction\_type\_and\_data\_model()}\spxextra{pypath.core.network.Network method}}

\begin{fulllineitems}
\phantomsection\label{\detokenize{reference:pypath.core.network.Network.data_models_by_interaction_type_and_data_model}}\pysiglinewithargsret{\sphinxbfcode{\sphinxupquote{data\_models\_by\_interaction\_type\_and\_data\_model}}}{}{}
Built-in immutable sequence.

If no argument is given, the constructor returns an empty tuple.
If iterable is specified the tuple is initialized from iterable’s items.

If the argument is a tuple, the return value is the same object.

\end{fulllineitems}

\index{data\_models\_by\_interaction\_type\_and\_data\_model\_and\_resource() (pypath.core.network.Network method)@\spxentry{data\_models\_by\_interaction\_type\_and\_data\_model\_and\_resource()}\spxextra{pypath.core.network.Network method}}

\begin{fulllineitems}
\phantomsection\label{\detokenize{reference:pypath.core.network.Network.data_models_by_interaction_type_and_data_model_and_resource}}\pysiglinewithargsret{\sphinxbfcode{\sphinxupquote{data\_models\_by\_interaction\_type\_and\_data\_model\_and\_resource}}}{}{}
Built-in immutable sequence.

If no argument is given, the constructor returns an empty tuple.
If iterable is specified the tuple is initialized from iterable’s items.

If the argument is a tuple, the return value is the same object.

\end{fulllineitems}

\index{data\_models\_by\_reference() (pypath.core.network.Network method)@\spxentry{data\_models\_by\_reference()}\spxextra{pypath.core.network.Network method}}

\begin{fulllineitems}
\phantomsection\label{\detokenize{reference:pypath.core.network.Network.data_models_by_reference}}\pysiglinewithargsret{\sphinxbfcode{\sphinxupquote{data\_models\_by\_reference}}}{}{}
Built-in immutable sequence.

If no argument is given, the constructor returns an empty tuple.
If iterable is specified the tuple is initialized from iterable’s items.

If the argument is a tuple, the return value is the same object.

\end{fulllineitems}

\index{data\_models\_by\_resource() (pypath.core.network.Network method)@\spxentry{data\_models\_by\_resource()}\spxextra{pypath.core.network.Network method}}

\begin{fulllineitems}
\phantomsection\label{\detokenize{reference:pypath.core.network.Network.data_models_by_resource}}\pysiglinewithargsret{\sphinxbfcode{\sphinxupquote{data\_models\_by\_resource}}}{}{}
Built-in immutable sequence.

If no argument is given, the constructor returns an empty tuple.
If iterable is specified the tuple is initialized from iterable’s items.

If the argument is a tuple, the return value is the same object.

\end{fulllineitems}

\index{degrees\_directed\_by\_data\_model() (pypath.core.network.Network method)@\spxentry{degrees\_directed\_by\_data\_model()}\spxextra{pypath.core.network.Network method}}

\begin{fulllineitems}
\phantomsection\label{\detokenize{reference:pypath.core.network.Network.degrees_directed_by_data_model}}\pysiglinewithargsret{\sphinxbfcode{\sphinxupquote{degrees\_directed\_by\_data\_model}}}{}{}
Built-in immutable sequence.

If no argument is given, the constructor returns an empty tuple.
If iterable is specified the tuple is initialized from iterable’s items.

If the argument is a tuple, the return value is the same object.

\end{fulllineitems}

\index{degrees\_directed\_by\_interaction\_type() (pypath.core.network.Network method)@\spxentry{degrees\_directed\_by\_interaction\_type()}\spxextra{pypath.core.network.Network method}}

\begin{fulllineitems}
\phantomsection\label{\detokenize{reference:pypath.core.network.Network.degrees_directed_by_interaction_type}}\pysiglinewithargsret{\sphinxbfcode{\sphinxupquote{degrees\_directed\_by\_interaction\_type}}}{}{}
Built-in immutable sequence.

If no argument is given, the constructor returns an empty tuple.
If iterable is specified the tuple is initialized from iterable’s items.

If the argument is a tuple, the return value is the same object.

\end{fulllineitems}

\index{degrees\_directed\_by\_interaction\_type\_and\_data\_model() (pypath.core.network.Network method)@\spxentry{degrees\_directed\_by\_interaction\_type\_and\_data\_model()}\spxextra{pypath.core.network.Network method}}

\begin{fulllineitems}
\phantomsection\label{\detokenize{reference:pypath.core.network.Network.degrees_directed_by_interaction_type_and_data_model}}\pysiglinewithargsret{\sphinxbfcode{\sphinxupquote{degrees\_directed\_by\_interaction\_type\_and\_data\_model}}}{}{}
Built-in immutable sequence.

If no argument is given, the constructor returns an empty tuple.
If iterable is specified the tuple is initialized from iterable’s items.

If the argument is a tuple, the return value is the same object.

\end{fulllineitems}

\index{degrees\_directed\_by\_interaction\_type\_and\_data\_model\_and\_resource() (pypath.core.network.Network method)@\spxentry{degrees\_directed\_by\_interaction\_type\_and\_data\_model\_and\_resource()}\spxextra{pypath.core.network.Network method}}

\begin{fulllineitems}
\phantomsection\label{\detokenize{reference:pypath.core.network.Network.degrees_directed_by_interaction_type_and_data_model_and_resource}}\pysiglinewithargsret{\sphinxbfcode{\sphinxupquote{degrees\_directed\_by\_interaction\_type\_and\_data\_model\_and\_resource}}}{}{}
Built-in immutable sequence.

If no argument is given, the constructor returns an empty tuple.
If iterable is specified the tuple is initialized from iterable’s items.

If the argument is a tuple, the return value is the same object.

\end{fulllineitems}

\index{degrees\_directed\_by\_reference() (pypath.core.network.Network method)@\spxentry{degrees\_directed\_by\_reference()}\spxextra{pypath.core.network.Network method}}

\begin{fulllineitems}
\phantomsection\label{\detokenize{reference:pypath.core.network.Network.degrees_directed_by_reference}}\pysiglinewithargsret{\sphinxbfcode{\sphinxupquote{degrees\_directed\_by\_reference}}}{}{}
Built-in immutable sequence.

If no argument is given, the constructor returns an empty tuple.
If iterable is specified the tuple is initialized from iterable’s items.

If the argument is a tuple, the return value is the same object.

\end{fulllineitems}

\index{degrees\_directed\_by\_resource() (pypath.core.network.Network method)@\spxentry{degrees\_directed\_by\_resource()}\spxextra{pypath.core.network.Network method}}

\begin{fulllineitems}
\phantomsection\label{\detokenize{reference:pypath.core.network.Network.degrees_directed_by_resource}}\pysiglinewithargsret{\sphinxbfcode{\sphinxupquote{degrees\_directed\_by\_resource}}}{}{}
Built-in immutable sequence.

If no argument is given, the constructor returns an empty tuple.
If iterable is specified the tuple is initialized from iterable’s items.

If the argument is a tuple, the return value is the same object.

\end{fulllineitems}

\index{degrees\_directed\_in\_by\_data\_model() (pypath.core.network.Network method)@\spxentry{degrees\_directed\_in\_by\_data\_model()}\spxextra{pypath.core.network.Network method}}

\begin{fulllineitems}
\phantomsection\label{\detokenize{reference:pypath.core.network.Network.degrees_directed_in_by_data_model}}\pysiglinewithargsret{\sphinxbfcode{\sphinxupquote{degrees\_directed\_in\_by\_data\_model}}}{}{}
Built-in immutable sequence.

If no argument is given, the constructor returns an empty tuple.
If iterable is specified the tuple is initialized from iterable’s items.

If the argument is a tuple, the return value is the same object.

\end{fulllineitems}

\index{degrees\_directed\_in\_by\_interaction\_type() (pypath.core.network.Network method)@\spxentry{degrees\_directed\_in\_by\_interaction\_type()}\spxextra{pypath.core.network.Network method}}

\begin{fulllineitems}
\phantomsection\label{\detokenize{reference:pypath.core.network.Network.degrees_directed_in_by_interaction_type}}\pysiglinewithargsret{\sphinxbfcode{\sphinxupquote{degrees\_directed\_in\_by\_interaction\_type}}}{}{}
Built-in immutable sequence.

If no argument is given, the constructor returns an empty tuple.
If iterable is specified the tuple is initialized from iterable’s items.

If the argument is a tuple, the return value is the same object.

\end{fulllineitems}

\index{degrees\_directed\_in\_by\_interaction\_type\_and\_data\_model() (pypath.core.network.Network method)@\spxentry{degrees\_directed\_in\_by\_interaction\_type\_and\_data\_model()}\spxextra{pypath.core.network.Network method}}

\begin{fulllineitems}
\phantomsection\label{\detokenize{reference:pypath.core.network.Network.degrees_directed_in_by_interaction_type_and_data_model}}\pysiglinewithargsret{\sphinxbfcode{\sphinxupquote{degrees\_directed\_in\_by\_interaction\_type\_and\_data\_model}}}{}{}
Built-in immutable sequence.

If no argument is given, the constructor returns an empty tuple.
If iterable is specified the tuple is initialized from iterable’s items.

If the argument is a tuple, the return value is the same object.

\end{fulllineitems}

\index{degrees\_directed\_in\_by\_interaction\_type\_and\_data\_model\_and\_resource() (pypath.core.network.Network method)@\spxentry{degrees\_directed\_in\_by\_interaction\_type\_and\_data\_model\_and\_resource()}\spxextra{pypath.core.network.Network method}}

\begin{fulllineitems}
\phantomsection\label{\detokenize{reference:pypath.core.network.Network.degrees_directed_in_by_interaction_type_and_data_model_and_resource}}\pysiglinewithargsret{\sphinxbfcode{\sphinxupquote{degrees\_directed\_in\_by\_interaction\_type\_and\_data\_model\_and\_resource}}}{}{}
Built-in immutable sequence.

If no argument is given, the constructor returns an empty tuple.
If iterable is specified the tuple is initialized from iterable’s items.

If the argument is a tuple, the return value is the same object.

\end{fulllineitems}

\index{degrees\_directed\_in\_by\_reference() (pypath.core.network.Network method)@\spxentry{degrees\_directed\_in\_by\_reference()}\spxextra{pypath.core.network.Network method}}

\begin{fulllineitems}
\phantomsection\label{\detokenize{reference:pypath.core.network.Network.degrees_directed_in_by_reference}}\pysiglinewithargsret{\sphinxbfcode{\sphinxupquote{degrees\_directed\_in\_by\_reference}}}{}{}
Built-in immutable sequence.

If no argument is given, the constructor returns an empty tuple.
If iterable is specified the tuple is initialized from iterable’s items.

If the argument is a tuple, the return value is the same object.

\end{fulllineitems}

\index{degrees\_directed\_in\_by\_resource() (pypath.core.network.Network method)@\spxentry{degrees\_directed\_in\_by\_resource()}\spxextra{pypath.core.network.Network method}}

\begin{fulllineitems}
\phantomsection\label{\detokenize{reference:pypath.core.network.Network.degrees_directed_in_by_resource}}\pysiglinewithargsret{\sphinxbfcode{\sphinxupquote{degrees\_directed\_in\_by\_resource}}}{}{}
Built-in immutable sequence.

If no argument is given, the constructor returns an empty tuple.
If iterable is specified the tuple is initialized from iterable’s items.

If the argument is a tuple, the return value is the same object.

\end{fulllineitems}

\index{degrees\_directed\_out\_by\_data\_model() (pypath.core.network.Network method)@\spxentry{degrees\_directed\_out\_by\_data\_model()}\spxextra{pypath.core.network.Network method}}

\begin{fulllineitems}
\phantomsection\label{\detokenize{reference:pypath.core.network.Network.degrees_directed_out_by_data_model}}\pysiglinewithargsret{\sphinxbfcode{\sphinxupquote{degrees\_directed\_out\_by\_data\_model}}}{}{}
Built-in immutable sequence.

If no argument is given, the constructor returns an empty tuple.
If iterable is specified the tuple is initialized from iterable’s items.

If the argument is a tuple, the return value is the same object.

\end{fulllineitems}

\index{degrees\_directed\_out\_by\_interaction\_type() (pypath.core.network.Network method)@\spxentry{degrees\_directed\_out\_by\_interaction\_type()}\spxextra{pypath.core.network.Network method}}

\begin{fulllineitems}
\phantomsection\label{\detokenize{reference:pypath.core.network.Network.degrees_directed_out_by_interaction_type}}\pysiglinewithargsret{\sphinxbfcode{\sphinxupquote{degrees\_directed\_out\_by\_interaction\_type}}}{}{}
Built-in immutable sequence.

If no argument is given, the constructor returns an empty tuple.
If iterable is specified the tuple is initialized from iterable’s items.

If the argument is a tuple, the return value is the same object.

\end{fulllineitems}

\index{degrees\_directed\_out\_by\_interaction\_type\_and\_data\_model() (pypath.core.network.Network method)@\spxentry{degrees\_directed\_out\_by\_interaction\_type\_and\_data\_model()}\spxextra{pypath.core.network.Network method}}

\begin{fulllineitems}
\phantomsection\label{\detokenize{reference:pypath.core.network.Network.degrees_directed_out_by_interaction_type_and_data_model}}\pysiglinewithargsret{\sphinxbfcode{\sphinxupquote{degrees\_directed\_out\_by\_interaction\_type\_and\_data\_model}}}{}{}
Built-in immutable sequence.

If no argument is given, the constructor returns an empty tuple.
If iterable is specified the tuple is initialized from iterable’s items.

If the argument is a tuple, the return value is the same object.

\end{fulllineitems}

\index{degrees\_directed\_out\_by\_interaction\_type\_and\_data\_model\_and\_resource() (pypath.core.network.Network method)@\spxentry{degrees\_directed\_out\_by\_interaction\_type\_and\_data\_model\_and\_resource()}\spxextra{pypath.core.network.Network method}}

\begin{fulllineitems}
\phantomsection\label{\detokenize{reference:pypath.core.network.Network.degrees_directed_out_by_interaction_type_and_data_model_and_resource}}\pysiglinewithargsret{\sphinxbfcode{\sphinxupquote{degrees\_directed\_out\_by\_interaction\_type\_and\_data\_model\_and\_resource}}}{}{}
Built-in immutable sequence.

If no argument is given, the constructor returns an empty tuple.
If iterable is specified the tuple is initialized from iterable’s items.

If the argument is a tuple, the return value is the same object.

\end{fulllineitems}

\index{degrees\_directed\_out\_by\_reference() (pypath.core.network.Network method)@\spxentry{degrees\_directed\_out\_by\_reference()}\spxextra{pypath.core.network.Network method}}

\begin{fulllineitems}
\phantomsection\label{\detokenize{reference:pypath.core.network.Network.degrees_directed_out_by_reference}}\pysiglinewithargsret{\sphinxbfcode{\sphinxupquote{degrees\_directed\_out\_by\_reference}}}{}{}
Built-in immutable sequence.

If no argument is given, the constructor returns an empty tuple.
If iterable is specified the tuple is initialized from iterable’s items.

If the argument is a tuple, the return value is the same object.

\end{fulllineitems}

\index{degrees\_directed\_out\_by\_resource() (pypath.core.network.Network method)@\spxentry{degrees\_directed\_out\_by\_resource()}\spxextra{pypath.core.network.Network method}}

\begin{fulllineitems}
\phantomsection\label{\detokenize{reference:pypath.core.network.Network.degrees_directed_out_by_resource}}\pysiglinewithargsret{\sphinxbfcode{\sphinxupquote{degrees\_directed\_out\_by\_resource}}}{}{}
Built-in immutable sequence.

If no argument is given, the constructor returns an empty tuple.
If iterable is specified the tuple is initialized from iterable’s items.

If the argument is a tuple, the return value is the same object.

\end{fulllineitems}

\index{degrees\_negative\_by\_data\_model() (pypath.core.network.Network method)@\spxentry{degrees\_negative\_by\_data\_model()}\spxextra{pypath.core.network.Network method}}

\begin{fulllineitems}
\phantomsection\label{\detokenize{reference:pypath.core.network.Network.degrees_negative_by_data_model}}\pysiglinewithargsret{\sphinxbfcode{\sphinxupquote{degrees\_negative\_by\_data\_model}}}{}{}
Built-in immutable sequence.

If no argument is given, the constructor returns an empty tuple.
If iterable is specified the tuple is initialized from iterable’s items.

If the argument is a tuple, the return value is the same object.

\end{fulllineitems}

\index{degrees\_negative\_by\_interaction\_type() (pypath.core.network.Network method)@\spxentry{degrees\_negative\_by\_interaction\_type()}\spxextra{pypath.core.network.Network method}}

\begin{fulllineitems}
\phantomsection\label{\detokenize{reference:pypath.core.network.Network.degrees_negative_by_interaction_type}}\pysiglinewithargsret{\sphinxbfcode{\sphinxupquote{degrees\_negative\_by\_interaction\_type}}}{}{}
Built-in immutable sequence.

If no argument is given, the constructor returns an empty tuple.
If iterable is specified the tuple is initialized from iterable’s items.

If the argument is a tuple, the return value is the same object.

\end{fulllineitems}

\index{degrees\_negative\_by\_interaction\_type\_and\_data\_model() (pypath.core.network.Network method)@\spxentry{degrees\_negative\_by\_interaction\_type\_and\_data\_model()}\spxextra{pypath.core.network.Network method}}

\begin{fulllineitems}
\phantomsection\label{\detokenize{reference:pypath.core.network.Network.degrees_negative_by_interaction_type_and_data_model}}\pysiglinewithargsret{\sphinxbfcode{\sphinxupquote{degrees\_negative\_by\_interaction\_type\_and\_data\_model}}}{}{}
Built-in immutable sequence.

If no argument is given, the constructor returns an empty tuple.
If iterable is specified the tuple is initialized from iterable’s items.

If the argument is a tuple, the return value is the same object.

\end{fulllineitems}

\index{degrees\_negative\_by\_interaction\_type\_and\_data\_model\_and\_resource() (pypath.core.network.Network method)@\spxentry{degrees\_negative\_by\_interaction\_type\_and\_data\_model\_and\_resource()}\spxextra{pypath.core.network.Network method}}

\begin{fulllineitems}
\phantomsection\label{\detokenize{reference:pypath.core.network.Network.degrees_negative_by_interaction_type_and_data_model_and_resource}}\pysiglinewithargsret{\sphinxbfcode{\sphinxupquote{degrees\_negative\_by\_interaction\_type\_and\_data\_model\_and\_resource}}}{}{}
Built-in immutable sequence.

If no argument is given, the constructor returns an empty tuple.
If iterable is specified the tuple is initialized from iterable’s items.

If the argument is a tuple, the return value is the same object.

\end{fulllineitems}

\index{degrees\_negative\_by\_reference() (pypath.core.network.Network method)@\spxentry{degrees\_negative\_by\_reference()}\spxextra{pypath.core.network.Network method}}

\begin{fulllineitems}
\phantomsection\label{\detokenize{reference:pypath.core.network.Network.degrees_negative_by_reference}}\pysiglinewithargsret{\sphinxbfcode{\sphinxupquote{degrees\_negative\_by\_reference}}}{}{}
Built-in immutable sequence.

If no argument is given, the constructor returns an empty tuple.
If iterable is specified the tuple is initialized from iterable’s items.

If the argument is a tuple, the return value is the same object.

\end{fulllineitems}

\index{degrees\_negative\_by\_resource() (pypath.core.network.Network method)@\spxentry{degrees\_negative\_by\_resource()}\spxextra{pypath.core.network.Network method}}

\begin{fulllineitems}
\phantomsection\label{\detokenize{reference:pypath.core.network.Network.degrees_negative_by_resource}}\pysiglinewithargsret{\sphinxbfcode{\sphinxupquote{degrees\_negative\_by\_resource}}}{}{}
Built-in immutable sequence.

If no argument is given, the constructor returns an empty tuple.
If iterable is specified the tuple is initialized from iterable’s items.

If the argument is a tuple, the return value is the same object.

\end{fulllineitems}

\index{degrees\_negative\_in\_by\_data\_model() (pypath.core.network.Network method)@\spxentry{degrees\_negative\_in\_by\_data\_model()}\spxextra{pypath.core.network.Network method}}

\begin{fulllineitems}
\phantomsection\label{\detokenize{reference:pypath.core.network.Network.degrees_negative_in_by_data_model}}\pysiglinewithargsret{\sphinxbfcode{\sphinxupquote{degrees\_negative\_in\_by\_data\_model}}}{}{}
Built-in immutable sequence.

If no argument is given, the constructor returns an empty tuple.
If iterable is specified the tuple is initialized from iterable’s items.

If the argument is a tuple, the return value is the same object.

\end{fulllineitems}

\index{degrees\_negative\_in\_by\_interaction\_type() (pypath.core.network.Network method)@\spxentry{degrees\_negative\_in\_by\_interaction\_type()}\spxextra{pypath.core.network.Network method}}

\begin{fulllineitems}
\phantomsection\label{\detokenize{reference:pypath.core.network.Network.degrees_negative_in_by_interaction_type}}\pysiglinewithargsret{\sphinxbfcode{\sphinxupquote{degrees\_negative\_in\_by\_interaction\_type}}}{}{}
Built-in immutable sequence.

If no argument is given, the constructor returns an empty tuple.
If iterable is specified the tuple is initialized from iterable’s items.

If the argument is a tuple, the return value is the same object.

\end{fulllineitems}

\index{degrees\_negative\_in\_by\_interaction\_type\_and\_data\_model() (pypath.core.network.Network method)@\spxentry{degrees\_negative\_in\_by\_interaction\_type\_and\_data\_model()}\spxextra{pypath.core.network.Network method}}

\begin{fulllineitems}
\phantomsection\label{\detokenize{reference:pypath.core.network.Network.degrees_negative_in_by_interaction_type_and_data_model}}\pysiglinewithargsret{\sphinxbfcode{\sphinxupquote{degrees\_negative\_in\_by\_interaction\_type\_and\_data\_model}}}{}{}
Built-in immutable sequence.

If no argument is given, the constructor returns an empty tuple.
If iterable is specified the tuple is initialized from iterable’s items.

If the argument is a tuple, the return value is the same object.

\end{fulllineitems}

\index{degrees\_negative\_in\_by\_interaction\_type\_and\_data\_model\_and\_resource() (pypath.core.network.Network method)@\spxentry{degrees\_negative\_in\_by\_interaction\_type\_and\_data\_model\_and\_resource()}\spxextra{pypath.core.network.Network method}}

\begin{fulllineitems}
\phantomsection\label{\detokenize{reference:pypath.core.network.Network.degrees_negative_in_by_interaction_type_and_data_model_and_resource}}\pysiglinewithargsret{\sphinxbfcode{\sphinxupquote{degrees\_negative\_in\_by\_interaction\_type\_and\_data\_model\_and\_resource}}}{}{}
Built-in immutable sequence.

If no argument is given, the constructor returns an empty tuple.
If iterable is specified the tuple is initialized from iterable’s items.

If the argument is a tuple, the return value is the same object.

\end{fulllineitems}

\index{degrees\_negative\_in\_by\_reference() (pypath.core.network.Network method)@\spxentry{degrees\_negative\_in\_by\_reference()}\spxextra{pypath.core.network.Network method}}

\begin{fulllineitems}
\phantomsection\label{\detokenize{reference:pypath.core.network.Network.degrees_negative_in_by_reference}}\pysiglinewithargsret{\sphinxbfcode{\sphinxupquote{degrees\_negative\_in\_by\_reference}}}{}{}
Built-in immutable sequence.

If no argument is given, the constructor returns an empty tuple.
If iterable is specified the tuple is initialized from iterable’s items.

If the argument is a tuple, the return value is the same object.

\end{fulllineitems}

\index{degrees\_negative\_in\_by\_resource() (pypath.core.network.Network method)@\spxentry{degrees\_negative\_in\_by\_resource()}\spxextra{pypath.core.network.Network method}}

\begin{fulllineitems}
\phantomsection\label{\detokenize{reference:pypath.core.network.Network.degrees_negative_in_by_resource}}\pysiglinewithargsret{\sphinxbfcode{\sphinxupquote{degrees\_negative\_in\_by\_resource}}}{}{}
Built-in immutable sequence.

If no argument is given, the constructor returns an empty tuple.
If iterable is specified the tuple is initialized from iterable’s items.

If the argument is a tuple, the return value is the same object.

\end{fulllineitems}

\index{degrees\_negative\_out\_by\_data\_model() (pypath.core.network.Network method)@\spxentry{degrees\_negative\_out\_by\_data\_model()}\spxextra{pypath.core.network.Network method}}

\begin{fulllineitems}
\phantomsection\label{\detokenize{reference:pypath.core.network.Network.degrees_negative_out_by_data_model}}\pysiglinewithargsret{\sphinxbfcode{\sphinxupquote{degrees\_negative\_out\_by\_data\_model}}}{}{}
Built-in immutable sequence.

If no argument is given, the constructor returns an empty tuple.
If iterable is specified the tuple is initialized from iterable’s items.

If the argument is a tuple, the return value is the same object.

\end{fulllineitems}

\index{degrees\_negative\_out\_by\_interaction\_type() (pypath.core.network.Network method)@\spxentry{degrees\_negative\_out\_by\_interaction\_type()}\spxextra{pypath.core.network.Network method}}

\begin{fulllineitems}
\phantomsection\label{\detokenize{reference:pypath.core.network.Network.degrees_negative_out_by_interaction_type}}\pysiglinewithargsret{\sphinxbfcode{\sphinxupquote{degrees\_negative\_out\_by\_interaction\_type}}}{}{}
Built-in immutable sequence.

If no argument is given, the constructor returns an empty tuple.
If iterable is specified the tuple is initialized from iterable’s items.

If the argument is a tuple, the return value is the same object.

\end{fulllineitems}

\index{degrees\_negative\_out\_by\_interaction\_type\_and\_data\_model() (pypath.core.network.Network method)@\spxentry{degrees\_negative\_out\_by\_interaction\_type\_and\_data\_model()}\spxextra{pypath.core.network.Network method}}

\begin{fulllineitems}
\phantomsection\label{\detokenize{reference:pypath.core.network.Network.degrees_negative_out_by_interaction_type_and_data_model}}\pysiglinewithargsret{\sphinxbfcode{\sphinxupquote{degrees\_negative\_out\_by\_interaction\_type\_and\_data\_model}}}{}{}
Built-in immutable sequence.

If no argument is given, the constructor returns an empty tuple.
If iterable is specified the tuple is initialized from iterable’s items.

If the argument is a tuple, the return value is the same object.

\end{fulllineitems}

\index{degrees\_negative\_out\_by\_interaction\_type\_and\_data\_model\_and\_resource() (pypath.core.network.Network method)@\spxentry{degrees\_negative\_out\_by\_interaction\_type\_and\_data\_model\_and\_resource()}\spxextra{pypath.core.network.Network method}}

\begin{fulllineitems}
\phantomsection\label{\detokenize{reference:pypath.core.network.Network.degrees_negative_out_by_interaction_type_and_data_model_and_resource}}\pysiglinewithargsret{\sphinxbfcode{\sphinxupquote{degrees\_negative\_out\_by\_interaction\_type\_and\_data\_model\_and\_resource}}}{}{}
Built-in immutable sequence.

If no argument is given, the constructor returns an empty tuple.
If iterable is specified the tuple is initialized from iterable’s items.

If the argument is a tuple, the return value is the same object.

\end{fulllineitems}

\index{degrees\_negative\_out\_by\_reference() (pypath.core.network.Network method)@\spxentry{degrees\_negative\_out\_by\_reference()}\spxextra{pypath.core.network.Network method}}

\begin{fulllineitems}
\phantomsection\label{\detokenize{reference:pypath.core.network.Network.degrees_negative_out_by_reference}}\pysiglinewithargsret{\sphinxbfcode{\sphinxupquote{degrees\_negative\_out\_by\_reference}}}{}{}
Built-in immutable sequence.

If no argument is given, the constructor returns an empty tuple.
If iterable is specified the tuple is initialized from iterable’s items.

If the argument is a tuple, the return value is the same object.

\end{fulllineitems}

\index{degrees\_negative\_out\_by\_resource() (pypath.core.network.Network method)@\spxentry{degrees\_negative\_out\_by\_resource()}\spxextra{pypath.core.network.Network method}}

\begin{fulllineitems}
\phantomsection\label{\detokenize{reference:pypath.core.network.Network.degrees_negative_out_by_resource}}\pysiglinewithargsret{\sphinxbfcode{\sphinxupquote{degrees\_negative\_out\_by\_resource}}}{}{}
Built-in immutable sequence.

If no argument is given, the constructor returns an empty tuple.
If iterable is specified the tuple is initialized from iterable’s items.

If the argument is a tuple, the return value is the same object.

\end{fulllineitems}

\index{degrees\_non\_directed\_by\_data\_model() (pypath.core.network.Network method)@\spxentry{degrees\_non\_directed\_by\_data\_model()}\spxextra{pypath.core.network.Network method}}

\begin{fulllineitems}
\phantomsection\label{\detokenize{reference:pypath.core.network.Network.degrees_non_directed_by_data_model}}\pysiglinewithargsret{\sphinxbfcode{\sphinxupquote{degrees\_non\_directed\_by\_data\_model}}}{}{}
Built-in immutable sequence.

If no argument is given, the constructor returns an empty tuple.
If iterable is specified the tuple is initialized from iterable’s items.

If the argument is a tuple, the return value is the same object.

\end{fulllineitems}

\index{degrees\_non\_directed\_by\_interaction\_type() (pypath.core.network.Network method)@\spxentry{degrees\_non\_directed\_by\_interaction\_type()}\spxextra{pypath.core.network.Network method}}

\begin{fulllineitems}
\phantomsection\label{\detokenize{reference:pypath.core.network.Network.degrees_non_directed_by_interaction_type}}\pysiglinewithargsret{\sphinxbfcode{\sphinxupquote{degrees\_non\_directed\_by\_interaction\_type}}}{}{}
Built-in immutable sequence.

If no argument is given, the constructor returns an empty tuple.
If iterable is specified the tuple is initialized from iterable’s items.

If the argument is a tuple, the return value is the same object.

\end{fulllineitems}

\index{degrees\_non\_directed\_by\_interaction\_type\_and\_data\_model() (pypath.core.network.Network method)@\spxentry{degrees\_non\_directed\_by\_interaction\_type\_and\_data\_model()}\spxextra{pypath.core.network.Network method}}

\begin{fulllineitems}
\phantomsection\label{\detokenize{reference:pypath.core.network.Network.degrees_non_directed_by_interaction_type_and_data_model}}\pysiglinewithargsret{\sphinxbfcode{\sphinxupquote{degrees\_non\_directed\_by\_interaction\_type\_and\_data\_model}}}{}{}
Built-in immutable sequence.

If no argument is given, the constructor returns an empty tuple.
If iterable is specified the tuple is initialized from iterable’s items.

If the argument is a tuple, the return value is the same object.

\end{fulllineitems}

\index{degrees\_non\_directed\_by\_interaction\_type\_and\_data\_model\_and\_resource() (pypath.core.network.Network method)@\spxentry{degrees\_non\_directed\_by\_interaction\_type\_and\_data\_model\_and\_resource()}\spxextra{pypath.core.network.Network method}}

\begin{fulllineitems}
\phantomsection\label{\detokenize{reference:pypath.core.network.Network.degrees_non_directed_by_interaction_type_and_data_model_and_resource}}\pysiglinewithargsret{\sphinxbfcode{\sphinxupquote{degrees\_non\_directed\_by\_interaction\_type\_and\_data\_model\_and\_resource}}}{}{}
Built-in immutable sequence.

If no argument is given, the constructor returns an empty tuple.
If iterable is specified the tuple is initialized from iterable’s items.

If the argument is a tuple, the return value is the same object.

\end{fulllineitems}

\index{degrees\_non\_directed\_by\_reference() (pypath.core.network.Network method)@\spxentry{degrees\_non\_directed\_by\_reference()}\spxextra{pypath.core.network.Network method}}

\begin{fulllineitems}
\phantomsection\label{\detokenize{reference:pypath.core.network.Network.degrees_non_directed_by_reference}}\pysiglinewithargsret{\sphinxbfcode{\sphinxupquote{degrees\_non\_directed\_by\_reference}}}{}{}
Built-in immutable sequence.

If no argument is given, the constructor returns an empty tuple.
If iterable is specified the tuple is initialized from iterable’s items.

If the argument is a tuple, the return value is the same object.

\end{fulllineitems}

\index{degrees\_non\_directed\_by\_resource() (pypath.core.network.Network method)@\spxentry{degrees\_non\_directed\_by\_resource()}\spxextra{pypath.core.network.Network method}}

\begin{fulllineitems}
\phantomsection\label{\detokenize{reference:pypath.core.network.Network.degrees_non_directed_by_resource}}\pysiglinewithargsret{\sphinxbfcode{\sphinxupquote{degrees\_non\_directed\_by\_resource}}}{}{}
Built-in immutable sequence.

If no argument is given, the constructor returns an empty tuple.
If iterable is specified the tuple is initialized from iterable’s items.

If the argument is a tuple, the return value is the same object.

\end{fulllineitems}

\index{degrees\_positive\_by\_data\_model() (pypath.core.network.Network method)@\spxentry{degrees\_positive\_by\_data\_model()}\spxextra{pypath.core.network.Network method}}

\begin{fulllineitems}
\phantomsection\label{\detokenize{reference:pypath.core.network.Network.degrees_positive_by_data_model}}\pysiglinewithargsret{\sphinxbfcode{\sphinxupquote{degrees\_positive\_by\_data\_model}}}{}{}
Built-in immutable sequence.

If no argument is given, the constructor returns an empty tuple.
If iterable is specified the tuple is initialized from iterable’s items.

If the argument is a tuple, the return value is the same object.

\end{fulllineitems}

\index{degrees\_positive\_by\_interaction\_type() (pypath.core.network.Network method)@\spxentry{degrees\_positive\_by\_interaction\_type()}\spxextra{pypath.core.network.Network method}}

\begin{fulllineitems}
\phantomsection\label{\detokenize{reference:pypath.core.network.Network.degrees_positive_by_interaction_type}}\pysiglinewithargsret{\sphinxbfcode{\sphinxupquote{degrees\_positive\_by\_interaction\_type}}}{}{}
Built-in immutable sequence.

If no argument is given, the constructor returns an empty tuple.
If iterable is specified the tuple is initialized from iterable’s items.

If the argument is a tuple, the return value is the same object.

\end{fulllineitems}

\index{degrees\_positive\_by\_interaction\_type\_and\_data\_model() (pypath.core.network.Network method)@\spxentry{degrees\_positive\_by\_interaction\_type\_and\_data\_model()}\spxextra{pypath.core.network.Network method}}

\begin{fulllineitems}
\phantomsection\label{\detokenize{reference:pypath.core.network.Network.degrees_positive_by_interaction_type_and_data_model}}\pysiglinewithargsret{\sphinxbfcode{\sphinxupquote{degrees\_positive\_by\_interaction\_type\_and\_data\_model}}}{}{}
Built-in immutable sequence.

If no argument is given, the constructor returns an empty tuple.
If iterable is specified the tuple is initialized from iterable’s items.

If the argument is a tuple, the return value is the same object.

\end{fulllineitems}

\index{degrees\_positive\_by\_interaction\_type\_and\_data\_model\_and\_resource() (pypath.core.network.Network method)@\spxentry{degrees\_positive\_by\_interaction\_type\_and\_data\_model\_and\_resource()}\spxextra{pypath.core.network.Network method}}

\begin{fulllineitems}
\phantomsection\label{\detokenize{reference:pypath.core.network.Network.degrees_positive_by_interaction_type_and_data_model_and_resource}}\pysiglinewithargsret{\sphinxbfcode{\sphinxupquote{degrees\_positive\_by\_interaction\_type\_and\_data\_model\_and\_resource}}}{}{}
Built-in immutable sequence.

If no argument is given, the constructor returns an empty tuple.
If iterable is specified the tuple is initialized from iterable’s items.

If the argument is a tuple, the return value is the same object.

\end{fulllineitems}

\index{degrees\_positive\_by\_reference() (pypath.core.network.Network method)@\spxentry{degrees\_positive\_by\_reference()}\spxextra{pypath.core.network.Network method}}

\begin{fulllineitems}
\phantomsection\label{\detokenize{reference:pypath.core.network.Network.degrees_positive_by_reference}}\pysiglinewithargsret{\sphinxbfcode{\sphinxupquote{degrees\_positive\_by\_reference}}}{}{}
Built-in immutable sequence.

If no argument is given, the constructor returns an empty tuple.
If iterable is specified the tuple is initialized from iterable’s items.

If the argument is a tuple, the return value is the same object.

\end{fulllineitems}

\index{degrees\_positive\_by\_resource() (pypath.core.network.Network method)@\spxentry{degrees\_positive\_by\_resource()}\spxextra{pypath.core.network.Network method}}

\begin{fulllineitems}
\phantomsection\label{\detokenize{reference:pypath.core.network.Network.degrees_positive_by_resource}}\pysiglinewithargsret{\sphinxbfcode{\sphinxupquote{degrees\_positive\_by\_resource}}}{}{}
Built-in immutable sequence.

If no argument is given, the constructor returns an empty tuple.
If iterable is specified the tuple is initialized from iterable’s items.

If the argument is a tuple, the return value is the same object.

\end{fulllineitems}

\index{degrees\_positive\_in\_by\_data\_model() (pypath.core.network.Network method)@\spxentry{degrees\_positive\_in\_by\_data\_model()}\spxextra{pypath.core.network.Network method}}

\begin{fulllineitems}
\phantomsection\label{\detokenize{reference:pypath.core.network.Network.degrees_positive_in_by_data_model}}\pysiglinewithargsret{\sphinxbfcode{\sphinxupquote{degrees\_positive\_in\_by\_data\_model}}}{}{}
Built-in immutable sequence.

If no argument is given, the constructor returns an empty tuple.
If iterable is specified the tuple is initialized from iterable’s items.

If the argument is a tuple, the return value is the same object.

\end{fulllineitems}

\index{degrees\_positive\_in\_by\_interaction\_type() (pypath.core.network.Network method)@\spxentry{degrees\_positive\_in\_by\_interaction\_type()}\spxextra{pypath.core.network.Network method}}

\begin{fulllineitems}
\phantomsection\label{\detokenize{reference:pypath.core.network.Network.degrees_positive_in_by_interaction_type}}\pysiglinewithargsret{\sphinxbfcode{\sphinxupquote{degrees\_positive\_in\_by\_interaction\_type}}}{}{}
Built-in immutable sequence.

If no argument is given, the constructor returns an empty tuple.
If iterable is specified the tuple is initialized from iterable’s items.

If the argument is a tuple, the return value is the same object.

\end{fulllineitems}

\index{degrees\_positive\_in\_by\_interaction\_type\_and\_data\_model() (pypath.core.network.Network method)@\spxentry{degrees\_positive\_in\_by\_interaction\_type\_and\_data\_model()}\spxextra{pypath.core.network.Network method}}

\begin{fulllineitems}
\phantomsection\label{\detokenize{reference:pypath.core.network.Network.degrees_positive_in_by_interaction_type_and_data_model}}\pysiglinewithargsret{\sphinxbfcode{\sphinxupquote{degrees\_positive\_in\_by\_interaction\_type\_and\_data\_model}}}{}{}
Built-in immutable sequence.

If no argument is given, the constructor returns an empty tuple.
If iterable is specified the tuple is initialized from iterable’s items.

If the argument is a tuple, the return value is the same object.

\end{fulllineitems}

\index{degrees\_positive\_in\_by\_interaction\_type\_and\_data\_model\_and\_resource() (pypath.core.network.Network method)@\spxentry{degrees\_positive\_in\_by\_interaction\_type\_and\_data\_model\_and\_resource()}\spxextra{pypath.core.network.Network method}}

\begin{fulllineitems}
\phantomsection\label{\detokenize{reference:pypath.core.network.Network.degrees_positive_in_by_interaction_type_and_data_model_and_resource}}\pysiglinewithargsret{\sphinxbfcode{\sphinxupquote{degrees\_positive\_in\_by\_interaction\_type\_and\_data\_model\_and\_resource}}}{}{}
Built-in immutable sequence.

If no argument is given, the constructor returns an empty tuple.
If iterable is specified the tuple is initialized from iterable’s items.

If the argument is a tuple, the return value is the same object.

\end{fulllineitems}

\index{degrees\_positive\_in\_by\_reference() (pypath.core.network.Network method)@\spxentry{degrees\_positive\_in\_by\_reference()}\spxextra{pypath.core.network.Network method}}

\begin{fulllineitems}
\phantomsection\label{\detokenize{reference:pypath.core.network.Network.degrees_positive_in_by_reference}}\pysiglinewithargsret{\sphinxbfcode{\sphinxupquote{degrees\_positive\_in\_by\_reference}}}{}{}
Built-in immutable sequence.

If no argument is given, the constructor returns an empty tuple.
If iterable is specified the tuple is initialized from iterable’s items.

If the argument is a tuple, the return value is the same object.

\end{fulllineitems}

\index{degrees\_positive\_in\_by\_resource() (pypath.core.network.Network method)@\spxentry{degrees\_positive\_in\_by\_resource()}\spxextra{pypath.core.network.Network method}}

\begin{fulllineitems}
\phantomsection\label{\detokenize{reference:pypath.core.network.Network.degrees_positive_in_by_resource}}\pysiglinewithargsret{\sphinxbfcode{\sphinxupquote{degrees\_positive\_in\_by\_resource}}}{}{}
Built-in immutable sequence.

If no argument is given, the constructor returns an empty tuple.
If iterable is specified the tuple is initialized from iterable’s items.

If the argument is a tuple, the return value is the same object.

\end{fulllineitems}

\index{degrees\_positive\_out\_by\_data\_model() (pypath.core.network.Network method)@\spxentry{degrees\_positive\_out\_by\_data\_model()}\spxextra{pypath.core.network.Network method}}

\begin{fulllineitems}
\phantomsection\label{\detokenize{reference:pypath.core.network.Network.degrees_positive_out_by_data_model}}\pysiglinewithargsret{\sphinxbfcode{\sphinxupquote{degrees\_positive\_out\_by\_data\_model}}}{}{}
Built-in immutable sequence.

If no argument is given, the constructor returns an empty tuple.
If iterable is specified the tuple is initialized from iterable’s items.

If the argument is a tuple, the return value is the same object.

\end{fulllineitems}

\index{degrees\_positive\_out\_by\_interaction\_type() (pypath.core.network.Network method)@\spxentry{degrees\_positive\_out\_by\_interaction\_type()}\spxextra{pypath.core.network.Network method}}

\begin{fulllineitems}
\phantomsection\label{\detokenize{reference:pypath.core.network.Network.degrees_positive_out_by_interaction_type}}\pysiglinewithargsret{\sphinxbfcode{\sphinxupquote{degrees\_positive\_out\_by\_interaction\_type}}}{}{}
Built-in immutable sequence.

If no argument is given, the constructor returns an empty tuple.
If iterable is specified the tuple is initialized from iterable’s items.

If the argument is a tuple, the return value is the same object.

\end{fulllineitems}

\index{degrees\_positive\_out\_by\_interaction\_type\_and\_data\_model() (pypath.core.network.Network method)@\spxentry{degrees\_positive\_out\_by\_interaction\_type\_and\_data\_model()}\spxextra{pypath.core.network.Network method}}

\begin{fulllineitems}
\phantomsection\label{\detokenize{reference:pypath.core.network.Network.degrees_positive_out_by_interaction_type_and_data_model}}\pysiglinewithargsret{\sphinxbfcode{\sphinxupquote{degrees\_positive\_out\_by\_interaction\_type\_and\_data\_model}}}{}{}
Built-in immutable sequence.

If no argument is given, the constructor returns an empty tuple.
If iterable is specified the tuple is initialized from iterable’s items.

If the argument is a tuple, the return value is the same object.

\end{fulllineitems}

\index{degrees\_positive\_out\_by\_interaction\_type\_and\_data\_model\_and\_resource() (pypath.core.network.Network method)@\spxentry{degrees\_positive\_out\_by\_interaction\_type\_and\_data\_model\_and\_resource()}\spxextra{pypath.core.network.Network method}}

\begin{fulllineitems}
\phantomsection\label{\detokenize{reference:pypath.core.network.Network.degrees_positive_out_by_interaction_type_and_data_model_and_resource}}\pysiglinewithargsret{\sphinxbfcode{\sphinxupquote{degrees\_positive\_out\_by\_interaction\_type\_and\_data\_model\_and\_resource}}}{}{}
Built-in immutable sequence.

If no argument is given, the constructor returns an empty tuple.
If iterable is specified the tuple is initialized from iterable’s items.

If the argument is a tuple, the return value is the same object.

\end{fulllineitems}

\index{degrees\_positive\_out\_by\_reference() (pypath.core.network.Network method)@\spxentry{degrees\_positive\_out\_by\_reference()}\spxextra{pypath.core.network.Network method}}

\begin{fulllineitems}
\phantomsection\label{\detokenize{reference:pypath.core.network.Network.degrees_positive_out_by_reference}}\pysiglinewithargsret{\sphinxbfcode{\sphinxupquote{degrees\_positive\_out\_by\_reference}}}{}{}
Built-in immutable sequence.

If no argument is given, the constructor returns an empty tuple.
If iterable is specified the tuple is initialized from iterable’s items.

If the argument is a tuple, the return value is the same object.

\end{fulllineitems}

\index{degrees\_positive\_out\_by\_resource() (pypath.core.network.Network method)@\spxentry{degrees\_positive\_out\_by\_resource()}\spxextra{pypath.core.network.Network method}}

\begin{fulllineitems}
\phantomsection\label{\detokenize{reference:pypath.core.network.Network.degrees_positive_out_by_resource}}\pysiglinewithargsret{\sphinxbfcode{\sphinxupquote{degrees\_positive\_out\_by\_resource}}}{}{}
Built-in immutable sequence.

If no argument is given, the constructor returns an empty tuple.
If iterable is specified the tuple is initialized from iterable’s items.

If the argument is a tuple, the return value is the same object.

\end{fulllineitems}

\index{degrees\_signed\_by\_data\_model() (pypath.core.network.Network method)@\spxentry{degrees\_signed\_by\_data\_model()}\spxextra{pypath.core.network.Network method}}

\begin{fulllineitems}
\phantomsection\label{\detokenize{reference:pypath.core.network.Network.degrees_signed_by_data_model}}\pysiglinewithargsret{\sphinxbfcode{\sphinxupquote{degrees\_signed\_by\_data\_model}}}{}{}
Built-in immutable sequence.

If no argument is given, the constructor returns an empty tuple.
If iterable is specified the tuple is initialized from iterable’s items.

If the argument is a tuple, the return value is the same object.

\end{fulllineitems}

\index{degrees\_signed\_by\_interaction\_type() (pypath.core.network.Network method)@\spxentry{degrees\_signed\_by\_interaction\_type()}\spxextra{pypath.core.network.Network method}}

\begin{fulllineitems}
\phantomsection\label{\detokenize{reference:pypath.core.network.Network.degrees_signed_by_interaction_type}}\pysiglinewithargsret{\sphinxbfcode{\sphinxupquote{degrees\_signed\_by\_interaction\_type}}}{}{}
Built-in immutable sequence.

If no argument is given, the constructor returns an empty tuple.
If iterable is specified the tuple is initialized from iterable’s items.

If the argument is a tuple, the return value is the same object.

\end{fulllineitems}

\index{degrees\_signed\_by\_interaction\_type\_and\_data\_model() (pypath.core.network.Network method)@\spxentry{degrees\_signed\_by\_interaction\_type\_and\_data\_model()}\spxextra{pypath.core.network.Network method}}

\begin{fulllineitems}
\phantomsection\label{\detokenize{reference:pypath.core.network.Network.degrees_signed_by_interaction_type_and_data_model}}\pysiglinewithargsret{\sphinxbfcode{\sphinxupquote{degrees\_signed\_by\_interaction\_type\_and\_data\_model}}}{}{}
Built-in immutable sequence.

If no argument is given, the constructor returns an empty tuple.
If iterable is specified the tuple is initialized from iterable’s items.

If the argument is a tuple, the return value is the same object.

\end{fulllineitems}

\index{degrees\_signed\_by\_interaction\_type\_and\_data\_model\_and\_resource() (pypath.core.network.Network method)@\spxentry{degrees\_signed\_by\_interaction\_type\_and\_data\_model\_and\_resource()}\spxextra{pypath.core.network.Network method}}

\begin{fulllineitems}
\phantomsection\label{\detokenize{reference:pypath.core.network.Network.degrees_signed_by_interaction_type_and_data_model_and_resource}}\pysiglinewithargsret{\sphinxbfcode{\sphinxupquote{degrees\_signed\_by\_interaction\_type\_and\_data\_model\_and\_resource}}}{}{}
Built-in immutable sequence.

If no argument is given, the constructor returns an empty tuple.
If iterable is specified the tuple is initialized from iterable’s items.

If the argument is a tuple, the return value is the same object.

\end{fulllineitems}

\index{degrees\_signed\_by\_reference() (pypath.core.network.Network method)@\spxentry{degrees\_signed\_by\_reference()}\spxextra{pypath.core.network.Network method}}

\begin{fulllineitems}
\phantomsection\label{\detokenize{reference:pypath.core.network.Network.degrees_signed_by_reference}}\pysiglinewithargsret{\sphinxbfcode{\sphinxupquote{degrees\_signed\_by\_reference}}}{}{}
Built-in immutable sequence.

If no argument is given, the constructor returns an empty tuple.
If iterable is specified the tuple is initialized from iterable’s items.

If the argument is a tuple, the return value is the same object.

\end{fulllineitems}

\index{degrees\_signed\_by\_resource() (pypath.core.network.Network method)@\spxentry{degrees\_signed\_by\_resource()}\spxextra{pypath.core.network.Network method}}

\begin{fulllineitems}
\phantomsection\label{\detokenize{reference:pypath.core.network.Network.degrees_signed_by_resource}}\pysiglinewithargsret{\sphinxbfcode{\sphinxupquote{degrees\_signed\_by\_resource}}}{}{}
Built-in immutable sequence.

If no argument is given, the constructor returns an empty tuple.
If iterable is specified the tuple is initialized from iterable’s items.

If the argument is a tuple, the return value is the same object.

\end{fulllineitems}

\index{degrees\_signed\_in\_by\_data\_model() (pypath.core.network.Network method)@\spxentry{degrees\_signed\_in\_by\_data\_model()}\spxextra{pypath.core.network.Network method}}

\begin{fulllineitems}
\phantomsection\label{\detokenize{reference:pypath.core.network.Network.degrees_signed_in_by_data_model}}\pysiglinewithargsret{\sphinxbfcode{\sphinxupquote{degrees\_signed\_in\_by\_data\_model}}}{}{}
Built-in immutable sequence.

If no argument is given, the constructor returns an empty tuple.
If iterable is specified the tuple is initialized from iterable’s items.

If the argument is a tuple, the return value is the same object.

\end{fulllineitems}

\index{degrees\_signed\_in\_by\_interaction\_type() (pypath.core.network.Network method)@\spxentry{degrees\_signed\_in\_by\_interaction\_type()}\spxextra{pypath.core.network.Network method}}

\begin{fulllineitems}
\phantomsection\label{\detokenize{reference:pypath.core.network.Network.degrees_signed_in_by_interaction_type}}\pysiglinewithargsret{\sphinxbfcode{\sphinxupquote{degrees\_signed\_in\_by\_interaction\_type}}}{}{}
Built-in immutable sequence.

If no argument is given, the constructor returns an empty tuple.
If iterable is specified the tuple is initialized from iterable’s items.

If the argument is a tuple, the return value is the same object.

\end{fulllineitems}

\index{degrees\_signed\_in\_by\_interaction\_type\_and\_data\_model() (pypath.core.network.Network method)@\spxentry{degrees\_signed\_in\_by\_interaction\_type\_and\_data\_model()}\spxextra{pypath.core.network.Network method}}

\begin{fulllineitems}
\phantomsection\label{\detokenize{reference:pypath.core.network.Network.degrees_signed_in_by_interaction_type_and_data_model}}\pysiglinewithargsret{\sphinxbfcode{\sphinxupquote{degrees\_signed\_in\_by\_interaction\_type\_and\_data\_model}}}{}{}
Built-in immutable sequence.

If no argument is given, the constructor returns an empty tuple.
If iterable is specified the tuple is initialized from iterable’s items.

If the argument is a tuple, the return value is the same object.

\end{fulllineitems}

\index{degrees\_signed\_in\_by\_interaction\_type\_and\_data\_model\_and\_resource() (pypath.core.network.Network method)@\spxentry{degrees\_signed\_in\_by\_interaction\_type\_and\_data\_model\_and\_resource()}\spxextra{pypath.core.network.Network method}}

\begin{fulllineitems}
\phantomsection\label{\detokenize{reference:pypath.core.network.Network.degrees_signed_in_by_interaction_type_and_data_model_and_resource}}\pysiglinewithargsret{\sphinxbfcode{\sphinxupquote{degrees\_signed\_in\_by\_interaction\_type\_and\_data\_model\_and\_resource}}}{}{}
Built-in immutable sequence.

If no argument is given, the constructor returns an empty tuple.
If iterable is specified the tuple is initialized from iterable’s items.

If the argument is a tuple, the return value is the same object.

\end{fulllineitems}

\index{degrees\_signed\_in\_by\_reference() (pypath.core.network.Network method)@\spxentry{degrees\_signed\_in\_by\_reference()}\spxextra{pypath.core.network.Network method}}

\begin{fulllineitems}
\phantomsection\label{\detokenize{reference:pypath.core.network.Network.degrees_signed_in_by_reference}}\pysiglinewithargsret{\sphinxbfcode{\sphinxupquote{degrees\_signed\_in\_by\_reference}}}{}{}
Built-in immutable sequence.

If no argument is given, the constructor returns an empty tuple.
If iterable is specified the tuple is initialized from iterable’s items.

If the argument is a tuple, the return value is the same object.

\end{fulllineitems}

\index{degrees\_signed\_in\_by\_resource() (pypath.core.network.Network method)@\spxentry{degrees\_signed\_in\_by\_resource()}\spxextra{pypath.core.network.Network method}}

\begin{fulllineitems}
\phantomsection\label{\detokenize{reference:pypath.core.network.Network.degrees_signed_in_by_resource}}\pysiglinewithargsret{\sphinxbfcode{\sphinxupquote{degrees\_signed\_in\_by\_resource}}}{}{}
Built-in immutable sequence.

If no argument is given, the constructor returns an empty tuple.
If iterable is specified the tuple is initialized from iterable’s items.

If the argument is a tuple, the return value is the same object.

\end{fulllineitems}

\index{degrees\_signed\_out\_by\_data\_model() (pypath.core.network.Network method)@\spxentry{degrees\_signed\_out\_by\_data\_model()}\spxextra{pypath.core.network.Network method}}

\begin{fulllineitems}
\phantomsection\label{\detokenize{reference:pypath.core.network.Network.degrees_signed_out_by_data_model}}\pysiglinewithargsret{\sphinxbfcode{\sphinxupquote{degrees\_signed\_out\_by\_data\_model}}}{}{}
Built-in immutable sequence.

If no argument is given, the constructor returns an empty tuple.
If iterable is specified the tuple is initialized from iterable’s items.

If the argument is a tuple, the return value is the same object.

\end{fulllineitems}

\index{degrees\_signed\_out\_by\_interaction\_type() (pypath.core.network.Network method)@\spxentry{degrees\_signed\_out\_by\_interaction\_type()}\spxextra{pypath.core.network.Network method}}

\begin{fulllineitems}
\phantomsection\label{\detokenize{reference:pypath.core.network.Network.degrees_signed_out_by_interaction_type}}\pysiglinewithargsret{\sphinxbfcode{\sphinxupquote{degrees\_signed\_out\_by\_interaction\_type}}}{}{}
Built-in immutable sequence.

If no argument is given, the constructor returns an empty tuple.
If iterable is specified the tuple is initialized from iterable’s items.

If the argument is a tuple, the return value is the same object.

\end{fulllineitems}

\index{degrees\_signed\_out\_by\_interaction\_type\_and\_data\_model() (pypath.core.network.Network method)@\spxentry{degrees\_signed\_out\_by\_interaction\_type\_and\_data\_model()}\spxextra{pypath.core.network.Network method}}

\begin{fulllineitems}
\phantomsection\label{\detokenize{reference:pypath.core.network.Network.degrees_signed_out_by_interaction_type_and_data_model}}\pysiglinewithargsret{\sphinxbfcode{\sphinxupquote{degrees\_signed\_out\_by\_interaction\_type\_and\_data\_model}}}{}{}
Built-in immutable sequence.

If no argument is given, the constructor returns an empty tuple.
If iterable is specified the tuple is initialized from iterable’s items.

If the argument is a tuple, the return value is the same object.

\end{fulllineitems}

\index{degrees\_signed\_out\_by\_interaction\_type\_and\_data\_model\_and\_resource() (pypath.core.network.Network method)@\spxentry{degrees\_signed\_out\_by\_interaction\_type\_and\_data\_model\_and\_resource()}\spxextra{pypath.core.network.Network method}}

\begin{fulllineitems}
\phantomsection\label{\detokenize{reference:pypath.core.network.Network.degrees_signed_out_by_interaction_type_and_data_model_and_resource}}\pysiglinewithargsret{\sphinxbfcode{\sphinxupquote{degrees\_signed\_out\_by\_interaction\_type\_and\_data\_model\_and\_resource}}}{}{}
Built-in immutable sequence.

If no argument is given, the constructor returns an empty tuple.
If iterable is specified the tuple is initialized from iterable’s items.

If the argument is a tuple, the return value is the same object.

\end{fulllineitems}

\index{degrees\_signed\_out\_by\_reference() (pypath.core.network.Network method)@\spxentry{degrees\_signed\_out\_by\_reference()}\spxextra{pypath.core.network.Network method}}

\begin{fulllineitems}
\phantomsection\label{\detokenize{reference:pypath.core.network.Network.degrees_signed_out_by_reference}}\pysiglinewithargsret{\sphinxbfcode{\sphinxupquote{degrees\_signed\_out\_by\_reference}}}{}{}
Built-in immutable sequence.

If no argument is given, the constructor returns an empty tuple.
If iterable is specified the tuple is initialized from iterable’s items.

If the argument is a tuple, the return value is the same object.

\end{fulllineitems}

\index{degrees\_signed\_out\_by\_resource() (pypath.core.network.Network method)@\spxentry{degrees\_signed\_out\_by\_resource()}\spxextra{pypath.core.network.Network method}}

\begin{fulllineitems}
\phantomsection\label{\detokenize{reference:pypath.core.network.Network.degrees_signed_out_by_resource}}\pysiglinewithargsret{\sphinxbfcode{\sphinxupquote{degrees\_signed\_out\_by\_resource}}}{}{}
Built-in immutable sequence.

If no argument is given, the constructor returns an empty tuple.
If iterable is specified the tuple is initialized from iterable’s items.

If the argument is a tuple, the return value is the same object.

\end{fulllineitems}

\index{degrees\_undirected\_by\_data\_model() (pypath.core.network.Network method)@\spxentry{degrees\_undirected\_by\_data\_model()}\spxextra{pypath.core.network.Network method}}

\begin{fulllineitems}
\phantomsection\label{\detokenize{reference:pypath.core.network.Network.degrees_undirected_by_data_model}}\pysiglinewithargsret{\sphinxbfcode{\sphinxupquote{degrees\_undirected\_by\_data\_model}}}{}{}
Built-in immutable sequence.

If no argument is given, the constructor returns an empty tuple.
If iterable is specified the tuple is initialized from iterable’s items.

If the argument is a tuple, the return value is the same object.

\end{fulllineitems}

\index{degrees\_undirected\_by\_interaction\_type() (pypath.core.network.Network method)@\spxentry{degrees\_undirected\_by\_interaction\_type()}\spxextra{pypath.core.network.Network method}}

\begin{fulllineitems}
\phantomsection\label{\detokenize{reference:pypath.core.network.Network.degrees_undirected_by_interaction_type}}\pysiglinewithargsret{\sphinxbfcode{\sphinxupquote{degrees\_undirected\_by\_interaction\_type}}}{}{}
Built-in immutable sequence.

If no argument is given, the constructor returns an empty tuple.
If iterable is specified the tuple is initialized from iterable’s items.

If the argument is a tuple, the return value is the same object.

\end{fulllineitems}

\index{degrees\_undirected\_by\_interaction\_type\_and\_data\_model() (pypath.core.network.Network method)@\spxentry{degrees\_undirected\_by\_interaction\_type\_and\_data\_model()}\spxextra{pypath.core.network.Network method}}

\begin{fulllineitems}
\phantomsection\label{\detokenize{reference:pypath.core.network.Network.degrees_undirected_by_interaction_type_and_data_model}}\pysiglinewithargsret{\sphinxbfcode{\sphinxupquote{degrees\_undirected\_by\_interaction\_type\_and\_data\_model}}}{}{}
Built-in immutable sequence.

If no argument is given, the constructor returns an empty tuple.
If iterable is specified the tuple is initialized from iterable’s items.

If the argument is a tuple, the return value is the same object.

\end{fulllineitems}

\index{degrees\_undirected\_by\_interaction\_type\_and\_data\_model\_and\_resource() (pypath.core.network.Network method)@\spxentry{degrees\_undirected\_by\_interaction\_type\_and\_data\_model\_and\_resource()}\spxextra{pypath.core.network.Network method}}

\begin{fulllineitems}
\phantomsection\label{\detokenize{reference:pypath.core.network.Network.degrees_undirected_by_interaction_type_and_data_model_and_resource}}\pysiglinewithargsret{\sphinxbfcode{\sphinxupquote{degrees\_undirected\_by\_interaction\_type\_and\_data\_model\_and\_resource}}}{}{}
Built-in immutable sequence.

If no argument is given, the constructor returns an empty tuple.
If iterable is specified the tuple is initialized from iterable’s items.

If the argument is a tuple, the return value is the same object.

\end{fulllineitems}

\index{degrees\_undirected\_by\_reference() (pypath.core.network.Network method)@\spxentry{degrees\_undirected\_by\_reference()}\spxextra{pypath.core.network.Network method}}

\begin{fulllineitems}
\phantomsection\label{\detokenize{reference:pypath.core.network.Network.degrees_undirected_by_reference}}\pysiglinewithargsret{\sphinxbfcode{\sphinxupquote{degrees\_undirected\_by\_reference}}}{}{}
Built-in immutable sequence.

If no argument is given, the constructor returns an empty tuple.
If iterable is specified the tuple is initialized from iterable’s items.

If the argument is a tuple, the return value is the same object.

\end{fulllineitems}

\index{degrees\_undirected\_by\_resource() (pypath.core.network.Network method)@\spxentry{degrees\_undirected\_by\_resource()}\spxextra{pypath.core.network.Network method}}

\begin{fulllineitems}
\phantomsection\label{\detokenize{reference:pypath.core.network.Network.degrees_undirected_by_resource}}\pysiglinewithargsret{\sphinxbfcode{\sphinxupquote{degrees\_undirected\_by\_resource}}}{}{}
Built-in immutable sequence.

If no argument is given, the constructor returns an empty tuple.
If iterable is specified the tuple is initialized from iterable’s items.

If the argument is a tuple, the return value is the same object.

\end{fulllineitems}

\index{dorothea() (pypath.core.network.Network class method)@\spxentry{dorothea()}\spxextra{pypath.core.network.Network class method}}

\begin{fulllineitems}
\phantomsection\label{\detokenize{reference:pypath.core.network.Network.dorothea}}\pysiglinewithargsret{\sphinxbfcode{\sphinxupquote{classmethod }}\sphinxbfcode{\sphinxupquote{dorothea}}}{\emph{levels=None}, \emph{ncbi\_tax\_id=9606}, \emph{**kwargs}}{}
Initializes a new \sphinxcode{\sphinxupquote{Network}} object with loading the transcriptional
regulation network from DoRothEA.
\begin{quote}\begin{description}
\item[{Parameters}] \leavevmode
\sphinxstyleliteralstrong{\sphinxupquote{levels}} (\sphinxstyleliteralemphasis{\sphinxupquote{NontType}}\sphinxstyleliteralemphasis{\sphinxupquote{,}}\sphinxstyleliteralemphasis{\sphinxupquote{set}}) \textendash{} The confidence levels to include.

\end{description}\end{quote}

\end{fulllineitems}

\index{entities\_by\_data\_model() (pypath.core.network.Network method)@\spxentry{entities\_by\_data\_model()}\spxextra{pypath.core.network.Network method}}

\begin{fulllineitems}
\phantomsection\label{\detokenize{reference:pypath.core.network.Network.entities_by_data_model}}\pysiglinewithargsret{\sphinxbfcode{\sphinxupquote{entities\_by\_data\_model}}}{}{}
Built-in immutable sequence.

If no argument is given, the constructor returns an empty tuple.
If iterable is specified the tuple is initialized from iterable’s items.

If the argument is a tuple, the return value is the same object.

\end{fulllineitems}

\index{entities\_by\_interaction\_type() (pypath.core.network.Network method)@\spxentry{entities\_by\_interaction\_type()}\spxextra{pypath.core.network.Network method}}

\begin{fulllineitems}
\phantomsection\label{\detokenize{reference:pypath.core.network.Network.entities_by_interaction_type}}\pysiglinewithargsret{\sphinxbfcode{\sphinxupquote{entities\_by\_interaction\_type}}}{}{}
Built-in immutable sequence.

If no argument is given, the constructor returns an empty tuple.
If iterable is specified the tuple is initialized from iterable’s items.

If the argument is a tuple, the return value is the same object.

\end{fulllineitems}

\index{entities\_by\_interaction\_type\_and\_data\_model() (pypath.core.network.Network method)@\spxentry{entities\_by\_interaction\_type\_and\_data\_model()}\spxextra{pypath.core.network.Network method}}

\begin{fulllineitems}
\phantomsection\label{\detokenize{reference:pypath.core.network.Network.entities_by_interaction_type_and_data_model}}\pysiglinewithargsret{\sphinxbfcode{\sphinxupquote{entities\_by\_interaction\_type\_and\_data\_model}}}{}{}
Built-in immutable sequence.

If no argument is given, the constructor returns an empty tuple.
If iterable is specified the tuple is initialized from iterable’s items.

If the argument is a tuple, the return value is the same object.

\end{fulllineitems}

\index{entities\_by\_interaction\_type\_and\_data\_model\_and\_resource() (pypath.core.network.Network method)@\spxentry{entities\_by\_interaction\_type\_and\_data\_model\_and\_resource()}\spxextra{pypath.core.network.Network method}}

\begin{fulllineitems}
\phantomsection\label{\detokenize{reference:pypath.core.network.Network.entities_by_interaction_type_and_data_model_and_resource}}\pysiglinewithargsret{\sphinxbfcode{\sphinxupquote{entities\_by\_interaction\_type\_and\_data\_model\_and\_resource}}}{}{}
Built-in immutable sequence.

If no argument is given, the constructor returns an empty tuple.
If iterable is specified the tuple is initialized from iterable’s items.

If the argument is a tuple, the return value is the same object.

\end{fulllineitems}

\index{entities\_by\_reference() (pypath.core.network.Network method)@\spxentry{entities\_by\_reference()}\spxextra{pypath.core.network.Network method}}

\begin{fulllineitems}
\phantomsection\label{\detokenize{reference:pypath.core.network.Network.entities_by_reference}}\pysiglinewithargsret{\sphinxbfcode{\sphinxupquote{entities\_by\_reference}}}{}{}
Built-in immutable sequence.

If no argument is given, the constructor returns an empty tuple.
If iterable is specified the tuple is initialized from iterable’s items.

If the argument is a tuple, the return value is the same object.

\end{fulllineitems}

\index{entities\_by\_resource() (pypath.core.network.Network method)@\spxentry{entities\_by\_resource()}\spxextra{pypath.core.network.Network method}}

\begin{fulllineitems}
\phantomsection\label{\detokenize{reference:pypath.core.network.Network.entities_by_resource}}\pysiglinewithargsret{\sphinxbfcode{\sphinxupquote{entities\_by\_resource}}}{}{}
Built-in immutable sequence.

If no argument is given, the constructor returns an empty tuple.
If iterable is specified the tuple is initialized from iterable’s items.

If the argument is a tuple, the return value is the same object.

\end{fulllineitems}

\index{entity\_by\_id() (pypath.core.network.Network method)@\spxentry{entity\_by\_id()}\spxextra{pypath.core.network.Network method}}

\begin{fulllineitems}
\phantomsection\label{\detokenize{reference:pypath.core.network.Network.entity_by_id}}\pysiglinewithargsret{\sphinxbfcode{\sphinxupquote{entity\_by\_id}}}{\emph{identifier}}{}
Returns a \sphinxcode{\sphinxupquote{pypath.entity.Entity}} object representing a molecular
entity by looking it up by its identifier. If the molecule does not
present in the current network \sphinxcode{\sphinxupquote{None}} will be returned.
\begin{quote}\begin{description}
\item[{Parameters}] \leavevmode
\sphinxstyleliteralstrong{\sphinxupquote{identifier}} (\sphinxstyleliteralemphasis{\sphinxupquote{str}}) \textendash{} The identifier of a molecular entity. Unless it’s been set
otherwise for genes/proteins it is the UniProt ID.
E.g. \sphinxcode{\sphinxupquote{'P00533'}}.

\end{description}\end{quote}

\end{fulllineitems}

\index{entity\_by\_label() (pypath.core.network.Network method)@\spxentry{entity\_by\_label()}\spxextra{pypath.core.network.Network method}}

\begin{fulllineitems}
\phantomsection\label{\detokenize{reference:pypath.core.network.Network.entity_by_label}}\pysiglinewithargsret{\sphinxbfcode{\sphinxupquote{entity\_by\_label}}}{\emph{label}}{}
Returns a \sphinxcode{\sphinxupquote{pypath.entity.Entity}} object representing a molecular
entity by looking it up by its label. If the molecule does not
present in the current network \sphinxcode{\sphinxupquote{None}} will be returned.
\begin{quote}\begin{description}
\item[{Parameters}] \leavevmode
\sphinxstyleliteralstrong{\sphinxupquote{label}} (\sphinxstyleliteralemphasis{\sphinxupquote{str}}) \textendash{} The label of a molecular entity. Unless it’s been set otherwise
for genes/proteins it is the Gene Symbol. E.g. \sphinxcode{\sphinxupquote{'EGFR'}}.

\end{description}\end{quote}

\end{fulllineitems}

\index{evidences\_by\_data\_model() (pypath.core.network.Network method)@\spxentry{evidences\_by\_data\_model()}\spxextra{pypath.core.network.Network method}}

\begin{fulllineitems}
\phantomsection\label{\detokenize{reference:pypath.core.network.Network.evidences_by_data_model}}\pysiglinewithargsret{\sphinxbfcode{\sphinxupquote{evidences\_by\_data\_model}}}{}{}
Built-in immutable sequence.

If no argument is given, the constructor returns an empty tuple.
If iterable is specified the tuple is initialized from iterable’s items.

If the argument is a tuple, the return value is the same object.

\end{fulllineitems}

\index{evidences\_by\_interaction\_type() (pypath.core.network.Network method)@\spxentry{evidences\_by\_interaction\_type()}\spxextra{pypath.core.network.Network method}}

\begin{fulllineitems}
\phantomsection\label{\detokenize{reference:pypath.core.network.Network.evidences_by_interaction_type}}\pysiglinewithargsret{\sphinxbfcode{\sphinxupquote{evidences\_by\_interaction\_type}}}{}{}
Built-in immutable sequence.

If no argument is given, the constructor returns an empty tuple.
If iterable is specified the tuple is initialized from iterable’s items.

If the argument is a tuple, the return value is the same object.

\end{fulllineitems}

\index{evidences\_by\_interaction\_type\_and\_data\_model() (pypath.core.network.Network method)@\spxentry{evidences\_by\_interaction\_type\_and\_data\_model()}\spxextra{pypath.core.network.Network method}}

\begin{fulllineitems}
\phantomsection\label{\detokenize{reference:pypath.core.network.Network.evidences_by_interaction_type_and_data_model}}\pysiglinewithargsret{\sphinxbfcode{\sphinxupquote{evidences\_by\_interaction\_type\_and\_data\_model}}}{}{}
Built-in immutable sequence.

If no argument is given, the constructor returns an empty tuple.
If iterable is specified the tuple is initialized from iterable’s items.

If the argument is a tuple, the return value is the same object.

\end{fulllineitems}

\index{evidences\_by\_interaction\_type\_and\_data\_model\_and\_resource() (pypath.core.network.Network method)@\spxentry{evidences\_by\_interaction\_type\_and\_data\_model\_and\_resource()}\spxextra{pypath.core.network.Network method}}

\begin{fulllineitems}
\phantomsection\label{\detokenize{reference:pypath.core.network.Network.evidences_by_interaction_type_and_data_model_and_resource}}\pysiglinewithargsret{\sphinxbfcode{\sphinxupquote{evidences\_by\_interaction\_type\_and\_data\_model\_and\_resource}}}{}{}
Built-in immutable sequence.

If no argument is given, the constructor returns an empty tuple.
If iterable is specified the tuple is initialized from iterable’s items.

If the argument is a tuple, the return value is the same object.

\end{fulllineitems}

\index{evidences\_by\_reference() (pypath.core.network.Network method)@\spxentry{evidences\_by\_reference()}\spxextra{pypath.core.network.Network method}}

\begin{fulllineitems}
\phantomsection\label{\detokenize{reference:pypath.core.network.Network.evidences_by_reference}}\pysiglinewithargsret{\sphinxbfcode{\sphinxupquote{evidences\_by\_reference}}}{}{}
Built-in immutable sequence.

If no argument is given, the constructor returns an empty tuple.
If iterable is specified the tuple is initialized from iterable’s items.

If the argument is a tuple, the return value is the same object.

\end{fulllineitems}

\index{evidences\_by\_resource() (pypath.core.network.Network method)@\spxentry{evidences\_by\_resource()}\spxextra{pypath.core.network.Network method}}

\begin{fulllineitems}
\phantomsection\label{\detokenize{reference:pypath.core.network.Network.evidences_by_resource}}\pysiglinewithargsret{\sphinxbfcode{\sphinxupquote{evidences\_by\_resource}}}{}{}
Built-in immutable sequence.

If no argument is given, the constructor returns an empty tuple.
If iterable is specified the tuple is initialized from iterable’s items.

If the argument is a tuple, the return value is the same object.

\end{fulllineitems}

\index{extra\_directions() (pypath.core.network.Network method)@\spxentry{extra\_directions()}\spxextra{pypath.core.network.Network method}}

\begin{fulllineitems}
\phantomsection\label{\detokenize{reference:pypath.core.network.Network.extra_directions}}\pysiglinewithargsret{\sphinxbfcode{\sphinxupquote{extra\_directions}}}{\emph{resources='extra\_directions'}, \emph{use\_laudanna=False}, \emph{use\_string=False}}{}
Adds additional direction \& effect information from resources having
no literature curated references, but giving sufficient evidence
about the directionality for interactions already supported by
literature evidences from other sources.

\end{fulllineitems}

\index{find\_paths() (pypath.core.network.Network method)@\spxentry{find\_paths()}\spxextra{pypath.core.network.Network method}}

\begin{fulllineitems}
\phantomsection\label{\detokenize{reference:pypath.core.network.Network.find_paths}}\pysiglinewithargsret{\sphinxbfcode{\sphinxupquote{find\_paths}}}{\emph{start}, \emph{end=None}, \emph{loops=False}, \emph{mode='OUT'}, \emph{maxlen=2}, \emph{minlen=1}, \emph{direction=None}, \emph{effect=None}, \emph{resources=None}, \emph{interaction\_type=None}, \emph{data\_model=None}, \emph{via=None}, \emph{references=None}, \emph{silent=False}}{}
Finds all paths up to length \sphinxcode{\sphinxupquote{maxlen}} between groups of nodes.
In addition is able to search for motifs or select the nodes of a
subnetwork around certain nodes.
\begin{quote}\begin{description}
\item[{Parameters}] \leavevmode\begin{itemize}
\item {} 
\sphinxstyleliteralstrong{\sphinxupquote{start}} (\sphinxstyleliteralemphasis{\sphinxupquote{str}}\sphinxstyleliteralemphasis{\sphinxupquote{,}}{\hyperref[\detokenize{reference:pypath.core.entity.Entity}]{\sphinxcrossref{\sphinxstyleliteralemphasis{\sphinxupquote{Entity}}}}}\sphinxstyleliteralemphasis{\sphinxupquote{,}}\sphinxstyleliteralemphasis{\sphinxupquote{list}}\sphinxstyleliteralemphasis{\sphinxupquote{,}}\sphinxstyleliteralemphasis{\sphinxupquote{tuple}}\sphinxstyleliteralemphasis{\sphinxupquote{,}}\sphinxstyleliteralemphasis{\sphinxupquote{set}}\sphinxstyleliteralemphasis{\sphinxupquote{,}}\sphinxstyleliteralemphasis{\sphinxupquote{EntityList}}) \textendash{} Starting node(s) of the paths.

\item {} 
\sphinxstyleliteralstrong{\sphinxupquote{end}} (\sphinxstyleliteralemphasis{\sphinxupquote{str}}\sphinxstyleliteralemphasis{\sphinxupquote{,}}{\hyperref[\detokenize{reference:pypath.core.entity.Entity}]{\sphinxcrossref{\sphinxstyleliteralemphasis{\sphinxupquote{Entity}}}}}\sphinxstyleliteralemphasis{\sphinxupquote{,}}\sphinxstyleliteralemphasis{\sphinxupquote{list}}\sphinxstyleliteralemphasis{\sphinxupquote{,}}\sphinxstyleliteralemphasis{\sphinxupquote{tuple}}\sphinxstyleliteralemphasis{\sphinxupquote{,}}\sphinxstyleliteralemphasis{\sphinxupquote{set}}\sphinxstyleliteralemphasis{\sphinxupquote{,}}\sphinxstyleliteralemphasis{\sphinxupquote{EntityList}}\sphinxstyleliteralemphasis{\sphinxupquote{,}}\sphinxstyleliteralemphasis{\sphinxupquote{NoneType}}) \textendash{} Target node(s) of the paths. If \sphinxcode{\sphinxupquote{None}} any target node will
be accepted and all paths from the starting nodes with length
\sphinxcode{\sphinxupquote{maxlen}} will be returned.

\item {} 
\sphinxstyleliteralstrong{\sphinxupquote{loops}} (\sphinxstyleliteralemphasis{\sphinxupquote{bool}}) \textendash{} Search for loops, i.e. the start and end nodes of each path
should be the same.

\item {} 
\sphinxstyleliteralstrong{\sphinxupquote{mode}} (\sphinxstyleliteralemphasis{\sphinxupquote{str}}) \textendash{} Direction of the paths. \sphinxcode{\sphinxupquote{'OUT'}} means from \sphinxcode{\sphinxupquote{start}} to \sphinxcode{\sphinxupquote{end}},
\sphinxcode{\sphinxupquote{'IN'}} the opposite direction while \sphinxcode{\sphinxupquote{'ALL'}} both directions.

\item {} 
\sphinxstyleliteralstrong{\sphinxupquote{maxlen}} (\sphinxstyleliteralemphasis{\sphinxupquote{int}}) \textendash{} Maximum length of paths in steps, i.e. if maxlen = 3, then
the longest path may consist of 3 edges and 4 nodes.

\item {} 
\sphinxstyleliteralstrong{\sphinxupquote{minlen}} (\sphinxstyleliteralemphasis{\sphinxupquote{int}}) \textendash{} Minimum length of the path.

\item {} 
\sphinxstyleliteralstrong{\sphinxupquote{silent}} (\sphinxstyleliteralemphasis{\sphinxupquote{bool}}) \textendash{} Indicate progress by showing a progress bar.

\end{itemize}

\item[{Details}] \leavevmode
\end{description}\end{quote}

The arguments: \sphinxcode{\sphinxupquote{direction}}, \sphinxcode{\sphinxupquote{effect}}, \sphinxcode{\sphinxupquote{resources}},
\sphinxcode{\sphinxupquote{interaction\_type}}, \sphinxcode{\sphinxupquote{data\_model}}, \sphinxcode{\sphinxupquote{via}} and \sphinxcode{\sphinxupquote{references}}
will be passed to the \sphinxcode{\sphinxupquote{partners}} method of this object and from
there to the relevant methods of the \sphinxcode{\sphinxupquote{Interaction}} and \sphinxcode{\sphinxupquote{Evidence}}
objects. By these arguments it is possible to filter the interactions
in the paths according to custom criteria. If any of these arguments
is a \sphinxcode{\sphinxupquote{tuple}} or \sphinxcode{\sphinxupquote{list}}, its first value will be used to match the
first interaction in the path, the second for the second one and so
on. If the list or tuple is shorter then \sphinxcode{\sphinxupquote{maxlen}}, its last
element will be used for all interactions. If it’s longer than
\sphinxcode{\sphinxupquote{maxlen}}, the remaining elements will be discarded. This way the
method is able to search for custom motives.
For example, let’s say you want to find the motives where the
estrogen receptor transcription factor \sphinxstyleemphasis{ESR1} transcriptionally
regulates a gene encoding a protein which then has some effect
post-translationally on \sphinxstyleemphasis{ESR1}:

\begin{sphinxVerbatim}[commandchars=\\\{\}]
\PYG{g+gp}{\PYGZgt{}\PYGZgt{}\PYGZgt{} }\PYG{n}{n}\PYG{o}{.}\PYG{n}{find\PYGZus{}paths}\PYG{p}{(}
\PYG{g+gp}{... }    \PYG{l+s+s1}{\PYGZsq{}}\PYG{l+s+s1}{ESR1}\PYG{l+s+s1}{\PYGZsq{}}\PYG{p}{,}
\PYG{g+gp}{... }    \PYG{n}{loops} \PYG{o}{=} \PYG{k+kc}{True}\PYG{p}{,}
\PYG{g+gp}{... }    \PYG{n}{minlen} \PYG{o}{=} \PYG{l+m+mi}{2}\PYG{p}{,}
\PYG{g+gp}{... }    \PYG{n}{interaction\PYGZus{}type} \PYG{o}{=} \PYG{p}{(}\PYG{l+s+s1}{\PYGZsq{}}\PYG{l+s+s1}{transcriptional}\PYG{l+s+s1}{\PYGZsq{}}\PYG{p}{,} \PYG{l+s+s1}{\PYGZsq{}}\PYG{l+s+s1}{post\PYGZus{}translational}\PYG{l+s+s1}{\PYGZsq{}}\PYG{p}{)}\PYG{p}{,}
\PYG{g+gp}{... }\PYG{p}{)}
\end{sphinxVerbatim}

Or if you are interested only in the -/+ feedback loops i.e.
\sphinxstyleemphasis{ESR1 \textendash{}(-)\textendash{}\textgreater{} X \textendash{}(+)\textendash{}\textgreater{} ESR1}:

\begin{sphinxVerbatim}[commandchars=\\\{\}]
\PYG{g+gp}{\PYGZgt{}\PYGZgt{}\PYGZgt{} }\PYG{n}{n}\PYG{o}{.}\PYG{n}{find\PYGZus{}paths}\PYG{p}{(}
\PYG{g+gp}{... }    \PYG{l+s+s1}{\PYGZsq{}}\PYG{l+s+s1}{ESR1}\PYG{l+s+s1}{\PYGZsq{}}\PYG{p}{,}
\PYG{g+gp}{... }    \PYG{n}{loops} \PYG{o}{=} \PYG{k+kc}{True}\PYG{p}{,}
\PYG{g+gp}{... }    \PYG{n}{minlen} \PYG{o}{=} \PYG{l+m+mi}{2}\PYG{p}{,}
\PYG{g+gp}{... }    \PYG{n}{interaction\PYGZus{}type} \PYG{o}{=} \PYG{p}{(}\PYG{l+s+s1}{\PYGZsq{}}\PYG{l+s+s1}{transcriptional}\PYG{l+s+s1}{\PYGZsq{}}\PYG{p}{,} \PYG{l+s+s1}{\PYGZsq{}}\PYG{l+s+s1}{post\PYGZus{}translational}\PYG{l+s+s1}{\PYGZsq{}}\PYG{p}{)}\PYG{p}{,}
\PYG{g+gp}{... }    \PYG{n}{effect} \PYG{o}{=} \PYG{p}{(}\PYG{l+s+s1}{\PYGZsq{}}\PYG{l+s+s1}{negative}\PYG{l+s+s1}{\PYGZsq{}}\PYG{p}{,} \PYG{l+s+s1}{\PYGZsq{}}\PYG{l+s+s1}{positive}\PYG{l+s+s1}{\PYGZsq{}}\PYG{p}{)}\PYG{p}{,}
\PYG{g+gp}{... }\PYG{p}{)}
\end{sphinxVerbatim}

\end{fulllineitems}

\index{from\_igraph() (pypath.core.network.Network class method)@\spxentry{from\_igraph()}\spxextra{pypath.core.network.Network class method}}

\begin{fulllineitems}
\phantomsection\label{\detokenize{reference:pypath.core.network.Network.from_igraph}}\pysiglinewithargsret{\sphinxbfcode{\sphinxupquote{classmethod }}\sphinxbfcode{\sphinxupquote{from\_igraph}}}{\emph{pa}, \emph{**kwargs}}{}
Creates an instance from an \sphinxcode{\sphinxupquote{igraph.Graph}} based
\sphinxcode{\sphinxupquote{pypath.main.PyPath}} object.
\begin{quote}\begin{description}
\item[{Parameters}] \leavevmode
\sphinxstyleliteralstrong{\sphinxupquote{pa}} (\sphinxstyleliteralemphasis{\sphinxupquote{pypath.main.PyPath}}) \textendash{} A \sphinxcode{\sphinxupquote{pypath.main.PyPath}} object with network data loaded.

\end{description}\end{quote}

\end{fulllineitems}

\index{from\_pickle() (pypath.core.network.Network class method)@\spxentry{from\_pickle()}\spxextra{pypath.core.network.Network class method}}

\begin{fulllineitems}
\phantomsection\label{\detokenize{reference:pypath.core.network.Network.from_pickle}}\pysiglinewithargsret{\sphinxbfcode{\sphinxupquote{classmethod }}\sphinxbfcode{\sphinxupquote{from\_pickle}}}{\emph{pickle\_file}, \emph{**kwargs}}{}
Initializes a new \sphinxcode{\sphinxupquote{Network}} object by loading it from a pickle
file. Returns a \sphinxcode{\sphinxupquote{Network}} object.
\begin{quote}\begin{description}
\item[{Parameters}] \leavevmode
\sphinxstyleliteralstrong{\sphinxupquote{pickle\_file}} (\sphinxstyleliteralemphasis{\sphinxupquote{str}}) \textendash{} Path to a pickle file.

\end{description}\end{quote}
\begin{description}
\item[{{\color{red}\bfseries{}**}kwargs:}] \leavevmode
Passed to \sphinxcode{\sphinxupquote{Network.\_\_init\_\_}}.

\end{description}

\end{fulllineitems}

\index{get\_complex\_identifiers() (pypath.core.network.Network method)@\spxentry{get\_complex\_identifiers()}\spxextra{pypath.core.network.Network method}}

\begin{fulllineitems}
\phantomsection\label{\detokenize{reference:pypath.core.network.Network.get_complex_identifiers}}\pysiglinewithargsret{\sphinxbfcode{\sphinxupquote{get\_complex\_identifiers}}}{}{}
Built-in immutable sequence.

If no argument is given, the constructor returns an empty tuple.
If iterable is specified the tuple is initialized from iterable’s items.

If the argument is a tuple, the return value is the same object.

\end{fulllineitems}

\index{get\_complex\_labels() (pypath.core.network.Network method)@\spxentry{get\_complex\_labels()}\spxextra{pypath.core.network.Network method}}

\begin{fulllineitems}
\phantomsection\label{\detokenize{reference:pypath.core.network.Network.get_complex_labels}}\pysiglinewithargsret{\sphinxbfcode{\sphinxupquote{get\_complex\_labels}}}{}{}
Built-in immutable sequence.

If no argument is given, the constructor returns an empty tuple.
If iterable is specified the tuple is initialized from iterable’s items.

If the argument is a tuple, the return value is the same object.

\end{fulllineitems}

\index{get\_complexes() (pypath.core.network.Network method)@\spxentry{get\_complexes()}\spxextra{pypath.core.network.Network method}}

\begin{fulllineitems}
\phantomsection\label{\detokenize{reference:pypath.core.network.Network.get_complexes}}\pysiglinewithargsret{\sphinxbfcode{\sphinxupquote{get\_complexes}}}{}{}
Built-in immutable sequence.

If no argument is given, the constructor returns an empty tuple.
If iterable is specified the tuple is initialized from iterable’s items.

If the argument is a tuple, the return value is the same object.

\end{fulllineitems}

\index{get\_curation\_effort() (pypath.core.network.Network method)@\spxentry{get\_curation\_effort()}\spxextra{pypath.core.network.Network method}}

\begin{fulllineitems}
\phantomsection\label{\detokenize{reference:pypath.core.network.Network.get_curation_effort}}\pysiglinewithargsret{\sphinxbfcode{\sphinxupquote{get\_curation\_effort}}}{}{}
Built-in immutable sequence.

If no argument is given, the constructor returns an empty tuple.
If iterable is specified the tuple is initialized from iterable’s items.

If the argument is a tuple, the return value is the same object.

\end{fulllineitems}

\index{get\_data\_models() (pypath.core.network.Network method)@\spxentry{get\_data\_models()}\spxextra{pypath.core.network.Network method}}

\begin{fulllineitems}
\phantomsection\label{\detokenize{reference:pypath.core.network.Network.get_data_models}}\pysiglinewithargsret{\sphinxbfcode{\sphinxupquote{get\_data\_models}}}{}{}
Built-in immutable sequence.

If no argument is given, the constructor returns an empty tuple.
If iterable is specified the tuple is initialized from iterable’s items.

If the argument is a tuple, the return value is the same object.

\end{fulllineitems}

\index{get\_degrees\_directed() (pypath.core.network.Network method)@\spxentry{get\_degrees\_directed()}\spxextra{pypath.core.network.Network method}}

\begin{fulllineitems}
\phantomsection\label{\detokenize{reference:pypath.core.network.Network.get_degrees_directed}}\pysiglinewithargsret{\sphinxbfcode{\sphinxupquote{get\_degrees\_directed}}}{}{}
Built-in immutable sequence.

If no argument is given, the constructor returns an empty tuple.
If iterable is specified the tuple is initialized from iterable’s items.

If the argument is a tuple, the return value is the same object.

\end{fulllineitems}

\index{get\_degrees\_directed\_in() (pypath.core.network.Network method)@\spxentry{get\_degrees\_directed\_in()}\spxextra{pypath.core.network.Network method}}

\begin{fulllineitems}
\phantomsection\label{\detokenize{reference:pypath.core.network.Network.get_degrees_directed_in}}\pysiglinewithargsret{\sphinxbfcode{\sphinxupquote{get\_degrees\_directed\_in}}}{}{}
Built-in immutable sequence.

If no argument is given, the constructor returns an empty tuple.
If iterable is specified the tuple is initialized from iterable’s items.

If the argument is a tuple, the return value is the same object.

\end{fulllineitems}

\index{get\_degrees\_directed\_out() (pypath.core.network.Network method)@\spxentry{get\_degrees\_directed\_out()}\spxextra{pypath.core.network.Network method}}

\begin{fulllineitems}
\phantomsection\label{\detokenize{reference:pypath.core.network.Network.get_degrees_directed_out}}\pysiglinewithargsret{\sphinxbfcode{\sphinxupquote{get\_degrees\_directed\_out}}}{}{}
Built-in immutable sequence.

If no argument is given, the constructor returns an empty tuple.
If iterable is specified the tuple is initialized from iterable’s items.

If the argument is a tuple, the return value is the same object.

\end{fulllineitems}

\index{get\_degrees\_negative() (pypath.core.network.Network method)@\spxentry{get\_degrees\_negative()}\spxextra{pypath.core.network.Network method}}

\begin{fulllineitems}
\phantomsection\label{\detokenize{reference:pypath.core.network.Network.get_degrees_negative}}\pysiglinewithargsret{\sphinxbfcode{\sphinxupquote{get\_degrees\_negative}}}{}{}
Built-in immutable sequence.

If no argument is given, the constructor returns an empty tuple.
If iterable is specified the tuple is initialized from iterable’s items.

If the argument is a tuple, the return value is the same object.

\end{fulllineitems}

\index{get\_degrees\_negative\_in() (pypath.core.network.Network method)@\spxentry{get\_degrees\_negative\_in()}\spxextra{pypath.core.network.Network method}}

\begin{fulllineitems}
\phantomsection\label{\detokenize{reference:pypath.core.network.Network.get_degrees_negative_in}}\pysiglinewithargsret{\sphinxbfcode{\sphinxupquote{get\_degrees\_negative\_in}}}{}{}
Built-in immutable sequence.

If no argument is given, the constructor returns an empty tuple.
If iterable is specified the tuple is initialized from iterable’s items.

If the argument is a tuple, the return value is the same object.

\end{fulllineitems}

\index{get\_degrees\_negative\_out() (pypath.core.network.Network method)@\spxentry{get\_degrees\_negative\_out()}\spxextra{pypath.core.network.Network method}}

\begin{fulllineitems}
\phantomsection\label{\detokenize{reference:pypath.core.network.Network.get_degrees_negative_out}}\pysiglinewithargsret{\sphinxbfcode{\sphinxupquote{get\_degrees\_negative\_out}}}{}{}
Built-in immutable sequence.

If no argument is given, the constructor returns an empty tuple.
If iterable is specified the tuple is initialized from iterable’s items.

If the argument is a tuple, the return value is the same object.

\end{fulllineitems}

\index{get\_degrees\_non\_directed() (pypath.core.network.Network method)@\spxentry{get\_degrees\_non\_directed()}\spxextra{pypath.core.network.Network method}}

\begin{fulllineitems}
\phantomsection\label{\detokenize{reference:pypath.core.network.Network.get_degrees_non_directed}}\pysiglinewithargsret{\sphinxbfcode{\sphinxupquote{get\_degrees\_non\_directed}}}{}{}
Built-in immutable sequence.

If no argument is given, the constructor returns an empty tuple.
If iterable is specified the tuple is initialized from iterable’s items.

If the argument is a tuple, the return value is the same object.

\end{fulllineitems}

\index{get\_degrees\_positive() (pypath.core.network.Network method)@\spxentry{get\_degrees\_positive()}\spxextra{pypath.core.network.Network method}}

\begin{fulllineitems}
\phantomsection\label{\detokenize{reference:pypath.core.network.Network.get_degrees_positive}}\pysiglinewithargsret{\sphinxbfcode{\sphinxupquote{get\_degrees\_positive}}}{}{}
Built-in immutable sequence.

If no argument is given, the constructor returns an empty tuple.
If iterable is specified the tuple is initialized from iterable’s items.

If the argument is a tuple, the return value is the same object.

\end{fulllineitems}

\index{get\_degrees\_positive\_in() (pypath.core.network.Network method)@\spxentry{get\_degrees\_positive\_in()}\spxextra{pypath.core.network.Network method}}

\begin{fulllineitems}
\phantomsection\label{\detokenize{reference:pypath.core.network.Network.get_degrees_positive_in}}\pysiglinewithargsret{\sphinxbfcode{\sphinxupquote{get\_degrees\_positive\_in}}}{}{}
Built-in immutable sequence.

If no argument is given, the constructor returns an empty tuple.
If iterable is specified the tuple is initialized from iterable’s items.

If the argument is a tuple, the return value is the same object.

\end{fulllineitems}

\index{get\_degrees\_positive\_out() (pypath.core.network.Network method)@\spxentry{get\_degrees\_positive\_out()}\spxextra{pypath.core.network.Network method}}

\begin{fulllineitems}
\phantomsection\label{\detokenize{reference:pypath.core.network.Network.get_degrees_positive_out}}\pysiglinewithargsret{\sphinxbfcode{\sphinxupquote{get\_degrees\_positive\_out}}}{}{}
Built-in immutable sequence.

If no argument is given, the constructor returns an empty tuple.
If iterable is specified the tuple is initialized from iterable’s items.

If the argument is a tuple, the return value is the same object.

\end{fulllineitems}

\index{get\_degrees\_signed() (pypath.core.network.Network method)@\spxentry{get\_degrees\_signed()}\spxextra{pypath.core.network.Network method}}

\begin{fulllineitems}
\phantomsection\label{\detokenize{reference:pypath.core.network.Network.get_degrees_signed}}\pysiglinewithargsret{\sphinxbfcode{\sphinxupquote{get\_degrees\_signed}}}{}{}
Built-in immutable sequence.

If no argument is given, the constructor returns an empty tuple.
If iterable is specified the tuple is initialized from iterable’s items.

If the argument is a tuple, the return value is the same object.

\end{fulllineitems}

\index{get\_degrees\_signed\_in() (pypath.core.network.Network method)@\spxentry{get\_degrees\_signed\_in()}\spxextra{pypath.core.network.Network method}}

\begin{fulllineitems}
\phantomsection\label{\detokenize{reference:pypath.core.network.Network.get_degrees_signed_in}}\pysiglinewithargsret{\sphinxbfcode{\sphinxupquote{get\_degrees\_signed\_in}}}{}{}
Built-in immutable sequence.

If no argument is given, the constructor returns an empty tuple.
If iterable is specified the tuple is initialized from iterable’s items.

If the argument is a tuple, the return value is the same object.

\end{fulllineitems}

\index{get\_degrees\_signed\_out() (pypath.core.network.Network method)@\spxentry{get\_degrees\_signed\_out()}\spxextra{pypath.core.network.Network method}}

\begin{fulllineitems}
\phantomsection\label{\detokenize{reference:pypath.core.network.Network.get_degrees_signed_out}}\pysiglinewithargsret{\sphinxbfcode{\sphinxupquote{get\_degrees\_signed\_out}}}{}{}
Built-in immutable sequence.

If no argument is given, the constructor returns an empty tuple.
If iterable is specified the tuple is initialized from iterable’s items.

If the argument is a tuple, the return value is the same object.

\end{fulllineitems}

\index{get\_degrees\_undirected() (pypath.core.network.Network method)@\spxentry{get\_degrees\_undirected()}\spxextra{pypath.core.network.Network method}}

\begin{fulllineitems}
\phantomsection\label{\detokenize{reference:pypath.core.network.Network.get_degrees_undirected}}\pysiglinewithargsret{\sphinxbfcode{\sphinxupquote{get\_degrees\_undirected}}}{}{}
Built-in immutable sequence.

If no argument is given, the constructor returns an empty tuple.
If iterable is specified the tuple is initialized from iterable’s items.

If the argument is a tuple, the return value is the same object.

\end{fulllineitems}

\index{get\_entities() (pypath.core.network.Network method)@\spxentry{get\_entities()}\spxextra{pypath.core.network.Network method}}

\begin{fulllineitems}
\phantomsection\label{\detokenize{reference:pypath.core.network.Network.get_entities}}\pysiglinewithargsret{\sphinxbfcode{\sphinxupquote{get\_entities}}}{}{}
Built-in immutable sequence.

If no argument is given, the constructor returns an empty tuple.
If iterable is specified the tuple is initialized from iterable’s items.

If the argument is a tuple, the return value is the same object.

\end{fulllineitems}

\index{get\_evidences() (pypath.core.network.Network method)@\spxentry{get\_evidences()}\spxextra{pypath.core.network.Network method}}

\begin{fulllineitems}
\phantomsection\label{\detokenize{reference:pypath.core.network.Network.get_evidences}}\pysiglinewithargsret{\sphinxbfcode{\sphinxupquote{get\_evidences}}}{}{}
Built-in immutable sequence.

If no argument is given, the constructor returns an empty tuple.
If iterable is specified the tuple is initialized from iterable’s items.

If the argument is a tuple, the return value is the same object.

\end{fulllineitems}

\index{get\_identifiers() (pypath.core.network.Network method)@\spxentry{get\_identifiers()}\spxextra{pypath.core.network.Network method}}

\begin{fulllineitems}
\phantomsection\label{\detokenize{reference:pypath.core.network.Network.get_identifiers}}\pysiglinewithargsret{\sphinxbfcode{\sphinxupquote{get\_identifiers}}}{}{}
Built-in immutable sequence.

If no argument is given, the constructor returns an empty tuple.
If iterable is specified the tuple is initialized from iterable’s items.

If the argument is a tuple, the return value is the same object.

\end{fulllineitems}

\index{get\_interaction\_types() (pypath.core.network.Network method)@\spxentry{get\_interaction\_types()}\spxextra{pypath.core.network.Network method}}

\begin{fulllineitems}
\phantomsection\label{\detokenize{reference:pypath.core.network.Network.get_interaction_types}}\pysiglinewithargsret{\sphinxbfcode{\sphinxupquote{get\_interaction\_types}}}{}{}
Built-in immutable sequence.

If no argument is given, the constructor returns an empty tuple.
If iterable is specified the tuple is initialized from iterable’s items.

If the argument is a tuple, the return value is the same object.

\end{fulllineitems}

\index{get\_interactions() (pypath.core.network.Network method)@\spxentry{get\_interactions()}\spxextra{pypath.core.network.Network method}}

\begin{fulllineitems}
\phantomsection\label{\detokenize{reference:pypath.core.network.Network.get_interactions}}\pysiglinewithargsret{\sphinxbfcode{\sphinxupquote{get\_interactions}}}{}{}
Built-in immutable sequence.

If no argument is given, the constructor returns an empty tuple.
If iterable is specified the tuple is initialized from iterable’s items.

If the argument is a tuple, the return value is the same object.

\end{fulllineitems}

\index{get\_interactions\_0() (pypath.core.network.Network method)@\spxentry{get\_interactions\_0()}\spxextra{pypath.core.network.Network method}}

\begin{fulllineitems}
\phantomsection\label{\detokenize{reference:pypath.core.network.Network.get_interactions_0}}\pysiglinewithargsret{\sphinxbfcode{\sphinxupquote{get\_interactions\_0}}}{}{}
Built-in immutable sequence.

If no argument is given, the constructor returns an empty tuple.
If iterable is specified the tuple is initialized from iterable’s items.

If the argument is a tuple, the return value is the same object.

\end{fulllineitems}

\index{get\_interactions\_directed() (pypath.core.network.Network method)@\spxentry{get\_interactions\_directed()}\spxextra{pypath.core.network.Network method}}

\begin{fulllineitems}
\phantomsection\label{\detokenize{reference:pypath.core.network.Network.get_interactions_directed}}\pysiglinewithargsret{\sphinxbfcode{\sphinxupquote{get\_interactions\_directed}}}{}{}
Built-in immutable sequence.

If no argument is given, the constructor returns an empty tuple.
If iterable is specified the tuple is initialized from iterable’s items.

If the argument is a tuple, the return value is the same object.

\end{fulllineitems}

\index{get\_interactions\_mutual() (pypath.core.network.Network method)@\spxentry{get\_interactions\_mutual()}\spxextra{pypath.core.network.Network method}}

\begin{fulllineitems}
\phantomsection\label{\detokenize{reference:pypath.core.network.Network.get_interactions_mutual}}\pysiglinewithargsret{\sphinxbfcode{\sphinxupquote{get\_interactions\_mutual}}}{}{}
Built-in immutable sequence.

If no argument is given, the constructor returns an empty tuple.
If iterable is specified the tuple is initialized from iterable’s items.

If the argument is a tuple, the return value is the same object.

\end{fulllineitems}

\index{get\_interactions\_negative() (pypath.core.network.Network method)@\spxentry{get\_interactions\_negative()}\spxextra{pypath.core.network.Network method}}

\begin{fulllineitems}
\phantomsection\label{\detokenize{reference:pypath.core.network.Network.get_interactions_negative}}\pysiglinewithargsret{\sphinxbfcode{\sphinxupquote{get\_interactions\_negative}}}{}{}
Built-in immutable sequence.

If no argument is given, the constructor returns an empty tuple.
If iterable is specified the tuple is initialized from iterable’s items.

If the argument is a tuple, the return value is the same object.

\end{fulllineitems}

\index{get\_interactions\_non\_directed() (pypath.core.network.Network method)@\spxentry{get\_interactions\_non\_directed()}\spxextra{pypath.core.network.Network method}}

\begin{fulllineitems}
\phantomsection\label{\detokenize{reference:pypath.core.network.Network.get_interactions_non_directed}}\pysiglinewithargsret{\sphinxbfcode{\sphinxupquote{get\_interactions\_non\_directed}}}{}{}
Built-in immutable sequence.

If no argument is given, the constructor returns an empty tuple.
If iterable is specified the tuple is initialized from iterable’s items.

If the argument is a tuple, the return value is the same object.

\end{fulllineitems}

\index{get\_interactions\_non\_directed\_0() (pypath.core.network.Network method)@\spxentry{get\_interactions\_non\_directed\_0()}\spxextra{pypath.core.network.Network method}}

\begin{fulllineitems}
\phantomsection\label{\detokenize{reference:pypath.core.network.Network.get_interactions_non_directed_0}}\pysiglinewithargsret{\sphinxbfcode{\sphinxupquote{get\_interactions\_non\_directed\_0}}}{}{}
Built-in immutable sequence.

If no argument is given, the constructor returns an empty tuple.
If iterable is specified the tuple is initialized from iterable’s items.

If the argument is a tuple, the return value is the same object.

\end{fulllineitems}

\index{get\_interactions\_positive() (pypath.core.network.Network method)@\spxentry{get\_interactions\_positive()}\spxextra{pypath.core.network.Network method}}

\begin{fulllineitems}
\phantomsection\label{\detokenize{reference:pypath.core.network.Network.get_interactions_positive}}\pysiglinewithargsret{\sphinxbfcode{\sphinxupquote{get\_interactions\_positive}}}{}{}
Built-in immutable sequence.

If no argument is given, the constructor returns an empty tuple.
If iterable is specified the tuple is initialized from iterable’s items.

If the argument is a tuple, the return value is the same object.

\end{fulllineitems}

\index{get\_interactions\_signed() (pypath.core.network.Network method)@\spxentry{get\_interactions\_signed()}\spxextra{pypath.core.network.Network method}}

\begin{fulllineitems}
\phantomsection\label{\detokenize{reference:pypath.core.network.Network.get_interactions_signed}}\pysiglinewithargsret{\sphinxbfcode{\sphinxupquote{get\_interactions\_signed}}}{}{}
Built-in immutable sequence.

If no argument is given, the constructor returns an empty tuple.
If iterable is specified the tuple is initialized from iterable’s items.

If the argument is a tuple, the return value is the same object.

\end{fulllineitems}

\index{get\_interactions\_undirected() (pypath.core.network.Network method)@\spxentry{get\_interactions\_undirected()}\spxextra{pypath.core.network.Network method}}

\begin{fulllineitems}
\phantomsection\label{\detokenize{reference:pypath.core.network.Network.get_interactions_undirected}}\pysiglinewithargsret{\sphinxbfcode{\sphinxupquote{get\_interactions\_undirected}}}{}{}
Built-in immutable sequence.

If no argument is given, the constructor returns an empty tuple.
If iterable is specified the tuple is initialized from iterable’s items.

If the argument is a tuple, the return value is the same object.

\end{fulllineitems}

\index{get\_interactions\_undirected\_0() (pypath.core.network.Network method)@\spxentry{get\_interactions\_undirected\_0()}\spxextra{pypath.core.network.Network method}}

\begin{fulllineitems}
\phantomsection\label{\detokenize{reference:pypath.core.network.Network.get_interactions_undirected_0}}\pysiglinewithargsret{\sphinxbfcode{\sphinxupquote{get\_interactions\_undirected\_0}}}{}{}
Built-in immutable sequence.

If no argument is given, the constructor returns an empty tuple.
If iterable is specified the tuple is initialized from iterable’s items.

If the argument is a tuple, the return value is the same object.

\end{fulllineitems}

\index{get\_labels() (pypath.core.network.Network method)@\spxentry{get\_labels()}\spxextra{pypath.core.network.Network method}}

\begin{fulllineitems}
\phantomsection\label{\detokenize{reference:pypath.core.network.Network.get_labels}}\pysiglinewithargsret{\sphinxbfcode{\sphinxupquote{get\_labels}}}{}{}
Built-in immutable sequence.

If no argument is given, the constructor returns an empty tuple.
If iterable is specified the tuple is initialized from iterable’s items.

If the argument is a tuple, the return value is the same object.

\end{fulllineitems}

\index{get\_lncrna\_identifiers() (pypath.core.network.Network method)@\spxentry{get\_lncrna\_identifiers()}\spxextra{pypath.core.network.Network method}}

\begin{fulllineitems}
\phantomsection\label{\detokenize{reference:pypath.core.network.Network.get_lncrna_identifiers}}\pysiglinewithargsret{\sphinxbfcode{\sphinxupquote{get\_lncrna\_identifiers}}}{}{}
Built-in immutable sequence.

If no argument is given, the constructor returns an empty tuple.
If iterable is specified the tuple is initialized from iterable’s items.

If the argument is a tuple, the return value is the same object.

\end{fulllineitems}

\index{get\_lncrna\_labels() (pypath.core.network.Network method)@\spxentry{get\_lncrna\_labels()}\spxextra{pypath.core.network.Network method}}

\begin{fulllineitems}
\phantomsection\label{\detokenize{reference:pypath.core.network.Network.get_lncrna_labels}}\pysiglinewithargsret{\sphinxbfcode{\sphinxupquote{get\_lncrna\_labels}}}{}{}
Built-in immutable sequence.

If no argument is given, the constructor returns an empty tuple.
If iterable is specified the tuple is initialized from iterable’s items.

If the argument is a tuple, the return value is the same object.

\end{fulllineitems}

\index{get\_lncrnas() (pypath.core.network.Network method)@\spxentry{get\_lncrnas()}\spxextra{pypath.core.network.Network method}}

\begin{fulllineitems}
\phantomsection\label{\detokenize{reference:pypath.core.network.Network.get_lncrnas}}\pysiglinewithargsret{\sphinxbfcode{\sphinxupquote{get\_lncrnas}}}{}{}
Built-in immutable sequence.

If no argument is given, the constructor returns an empty tuple.
If iterable is specified the tuple is initialized from iterable’s items.

If the argument is a tuple, the return value is the same object.

\end{fulllineitems}

\index{get\_mirna\_identifiers() (pypath.core.network.Network method)@\spxentry{get\_mirna\_identifiers()}\spxextra{pypath.core.network.Network method}}

\begin{fulllineitems}
\phantomsection\label{\detokenize{reference:pypath.core.network.Network.get_mirna_identifiers}}\pysiglinewithargsret{\sphinxbfcode{\sphinxupquote{get\_mirna\_identifiers}}}{}{}
Built-in immutable sequence.

If no argument is given, the constructor returns an empty tuple.
If iterable is specified the tuple is initialized from iterable’s items.

If the argument is a tuple, the return value is the same object.

\end{fulllineitems}

\index{get\_mirna\_labels() (pypath.core.network.Network method)@\spxentry{get\_mirna\_labels()}\spxextra{pypath.core.network.Network method}}

\begin{fulllineitems}
\phantomsection\label{\detokenize{reference:pypath.core.network.Network.get_mirna_labels}}\pysiglinewithargsret{\sphinxbfcode{\sphinxupquote{get\_mirna\_labels}}}{}{}
Built-in immutable sequence.

If no argument is given, the constructor returns an empty tuple.
If iterable is specified the tuple is initialized from iterable’s items.

If the argument is a tuple, the return value is the same object.

\end{fulllineitems}

\index{get\_mirnas() (pypath.core.network.Network method)@\spxentry{get\_mirnas()}\spxextra{pypath.core.network.Network method}}

\begin{fulllineitems}
\phantomsection\label{\detokenize{reference:pypath.core.network.Network.get_mirnas}}\pysiglinewithargsret{\sphinxbfcode{\sphinxupquote{get\_mirnas}}}{}{}
Built-in immutable sequence.

If no argument is given, the constructor returns an empty tuple.
If iterable is specified the tuple is initialized from iterable’s items.

If the argument is a tuple, the return value is the same object.

\end{fulllineitems}

\index{get\_organisms() (pypath.core.network.Network method)@\spxentry{get\_organisms()}\spxextra{pypath.core.network.Network method}}

\begin{fulllineitems}
\phantomsection\label{\detokenize{reference:pypath.core.network.Network.get_organisms}}\pysiglinewithargsret{\sphinxbfcode{\sphinxupquote{get\_organisms}}}{}{}
Returns the set of all NCBI Taxonomy IDs occurring in the network.

\end{fulllineitems}

\index{get\_protein\_identifiers() (pypath.core.network.Network method)@\spxentry{get\_protein\_identifiers()}\spxextra{pypath.core.network.Network method}}

\begin{fulllineitems}
\phantomsection\label{\detokenize{reference:pypath.core.network.Network.get_protein_identifiers}}\pysiglinewithargsret{\sphinxbfcode{\sphinxupquote{get\_protein\_identifiers}}}{}{}
Built-in immutable sequence.

If no argument is given, the constructor returns an empty tuple.
If iterable is specified the tuple is initialized from iterable’s items.

If the argument is a tuple, the return value is the same object.

\end{fulllineitems}

\index{get\_protein\_labels() (pypath.core.network.Network method)@\spxentry{get\_protein\_labels()}\spxextra{pypath.core.network.Network method}}

\begin{fulllineitems}
\phantomsection\label{\detokenize{reference:pypath.core.network.Network.get_protein_labels}}\pysiglinewithargsret{\sphinxbfcode{\sphinxupquote{get\_protein\_labels}}}{}{}
Built-in immutable sequence.

If no argument is given, the constructor returns an empty tuple.
If iterable is specified the tuple is initialized from iterable’s items.

If the argument is a tuple, the return value is the same object.

\end{fulllineitems}

\index{get\_proteins() (pypath.core.network.Network method)@\spxentry{get\_proteins()}\spxextra{pypath.core.network.Network method}}

\begin{fulllineitems}
\phantomsection\label{\detokenize{reference:pypath.core.network.Network.get_proteins}}\pysiglinewithargsret{\sphinxbfcode{\sphinxupquote{get\_proteins}}}{}{}
Built-in immutable sequence.

If no argument is given, the constructor returns an empty tuple.
If iterable is specified the tuple is initialized from iterable’s items.

If the argument is a tuple, the return value is the same object.

\end{fulllineitems}

\index{get\_references() (pypath.core.network.Network method)@\spxentry{get\_references()}\spxextra{pypath.core.network.Network method}}

\begin{fulllineitems}
\phantomsection\label{\detokenize{reference:pypath.core.network.Network.get_references}}\pysiglinewithargsret{\sphinxbfcode{\sphinxupquote{get\_references}}}{}{}
Built-in immutable sequence.

If no argument is given, the constructor returns an empty tuple.
If iterable is specified the tuple is initialized from iterable’s items.

If the argument is a tuple, the return value is the same object.

\end{fulllineitems}

\index{get\_resource\_names() (pypath.core.network.Network method)@\spxentry{get\_resource\_names()}\spxextra{pypath.core.network.Network method}}

\begin{fulllineitems}
\phantomsection\label{\detokenize{reference:pypath.core.network.Network.get_resource_names}}\pysiglinewithargsret{\sphinxbfcode{\sphinxupquote{get\_resource\_names}}}{}{}
Built-in immutable sequence.

If no argument is given, the constructor returns an empty tuple.
If iterable is specified the tuple is initialized from iterable’s items.

If the argument is a tuple, the return value is the same object.

\end{fulllineitems}

\index{get\_resource\_names\_via() (pypath.core.network.Network method)@\spxentry{get\_resource\_names\_via()}\spxextra{pypath.core.network.Network method}}

\begin{fulllineitems}
\phantomsection\label{\detokenize{reference:pypath.core.network.Network.get_resource_names_via}}\pysiglinewithargsret{\sphinxbfcode{\sphinxupquote{get\_resource\_names\_via}}}{}{}
Built-in immutable sequence.

If no argument is given, the constructor returns an empty tuple.
If iterable is specified the tuple is initialized from iterable’s items.

If the argument is a tuple, the return value is the same object.

\end{fulllineitems}

\index{get\_resources() (pypath.core.network.Network method)@\spxentry{get\_resources()}\spxextra{pypath.core.network.Network method}}

\begin{fulllineitems}
\phantomsection\label{\detokenize{reference:pypath.core.network.Network.get_resources}}\pysiglinewithargsret{\sphinxbfcode{\sphinxupquote{get\_resources}}}{}{}
Built-in immutable sequence.

If no argument is given, the constructor returns an empty tuple.
If iterable is specified the tuple is initialized from iterable’s items.

If the argument is a tuple, the return value is the same object.

\end{fulllineitems}

\index{get\_resources\_via() (pypath.core.network.Network method)@\spxentry{get\_resources\_via()}\spxextra{pypath.core.network.Network method}}

\begin{fulllineitems}
\phantomsection\label{\detokenize{reference:pypath.core.network.Network.get_resources_via}}\pysiglinewithargsret{\sphinxbfcode{\sphinxupquote{get\_resources\_via}}}{}{}
Built-in immutable sequence.

If no argument is given, the constructor returns an empty tuple.
If iterable is specified the tuple is initialized from iterable’s items.

If the argument is a tuple, the return value is the same object.

\end{fulllineitems}

\index{get\_small\_molecule\_identifiers() (pypath.core.network.Network method)@\spxentry{get\_small\_molecule\_identifiers()}\spxextra{pypath.core.network.Network method}}

\begin{fulllineitems}
\phantomsection\label{\detokenize{reference:pypath.core.network.Network.get_small_molecule_identifiers}}\pysiglinewithargsret{\sphinxbfcode{\sphinxupquote{get\_small\_molecule\_identifiers}}}{}{}
Built-in immutable sequence.

If no argument is given, the constructor returns an empty tuple.
If iterable is specified the tuple is initialized from iterable’s items.

If the argument is a tuple, the return value is the same object.

\end{fulllineitems}

\index{get\_small\_molecule\_labels() (pypath.core.network.Network method)@\spxentry{get\_small\_molecule\_labels()}\spxextra{pypath.core.network.Network method}}

\begin{fulllineitems}
\phantomsection\label{\detokenize{reference:pypath.core.network.Network.get_small_molecule_labels}}\pysiglinewithargsret{\sphinxbfcode{\sphinxupquote{get\_small\_molecule\_labels}}}{}{}
Built-in immutable sequence.

If no argument is given, the constructor returns an empty tuple.
If iterable is specified the tuple is initialized from iterable’s items.

If the argument is a tuple, the return value is the same object.

\end{fulllineitems}

\index{get\_small\_molecules() (pypath.core.network.Network method)@\spxentry{get\_small\_molecules()}\spxextra{pypath.core.network.Network method}}

\begin{fulllineitems}
\phantomsection\label{\detokenize{reference:pypath.core.network.Network.get_small_molecules}}\pysiglinewithargsret{\sphinxbfcode{\sphinxupquote{get\_small\_molecules}}}{}{}
Built-in immutable sequence.

If no argument is given, the constructor returns an empty tuple.
If iterable is specified the tuple is initialized from iterable’s items.

If the argument is a tuple, the return value is the same object.

\end{fulllineitems}

\index{identifiers\_by\_data\_model() (pypath.core.network.Network method)@\spxentry{identifiers\_by\_data\_model()}\spxextra{pypath.core.network.Network method}}

\begin{fulllineitems}
\phantomsection\label{\detokenize{reference:pypath.core.network.Network.identifiers_by_data_model}}\pysiglinewithargsret{\sphinxbfcode{\sphinxupquote{identifiers\_by\_data\_model}}}{}{}
Built-in immutable sequence.

If no argument is given, the constructor returns an empty tuple.
If iterable is specified the tuple is initialized from iterable’s items.

If the argument is a tuple, the return value is the same object.

\end{fulllineitems}

\index{identifiers\_by\_interaction\_type() (pypath.core.network.Network method)@\spxentry{identifiers\_by\_interaction\_type()}\spxextra{pypath.core.network.Network method}}

\begin{fulllineitems}
\phantomsection\label{\detokenize{reference:pypath.core.network.Network.identifiers_by_interaction_type}}\pysiglinewithargsret{\sphinxbfcode{\sphinxupquote{identifiers\_by\_interaction\_type}}}{}{}
Built-in immutable sequence.

If no argument is given, the constructor returns an empty tuple.
If iterable is specified the tuple is initialized from iterable’s items.

If the argument is a tuple, the return value is the same object.

\end{fulllineitems}

\index{identifiers\_by\_interaction\_type\_and\_data\_model() (pypath.core.network.Network method)@\spxentry{identifiers\_by\_interaction\_type\_and\_data\_model()}\spxextra{pypath.core.network.Network method}}

\begin{fulllineitems}
\phantomsection\label{\detokenize{reference:pypath.core.network.Network.identifiers_by_interaction_type_and_data_model}}\pysiglinewithargsret{\sphinxbfcode{\sphinxupquote{identifiers\_by\_interaction\_type\_and\_data\_model}}}{}{}
Built-in immutable sequence.

If no argument is given, the constructor returns an empty tuple.
If iterable is specified the tuple is initialized from iterable’s items.

If the argument is a tuple, the return value is the same object.

\end{fulllineitems}

\index{identifiers\_by\_interaction\_type\_and\_data\_model\_and\_resource() (pypath.core.network.Network method)@\spxentry{identifiers\_by\_interaction\_type\_and\_data\_model\_and\_resource()}\spxextra{pypath.core.network.Network method}}

\begin{fulllineitems}
\phantomsection\label{\detokenize{reference:pypath.core.network.Network.identifiers_by_interaction_type_and_data_model_and_resource}}\pysiglinewithargsret{\sphinxbfcode{\sphinxupquote{identifiers\_by\_interaction\_type\_and\_data\_model\_and\_resource}}}{}{}
Built-in immutable sequence.

If no argument is given, the constructor returns an empty tuple.
If iterable is specified the tuple is initialized from iterable’s items.

If the argument is a tuple, the return value is the same object.

\end{fulllineitems}

\index{identifiers\_by\_reference() (pypath.core.network.Network method)@\spxentry{identifiers\_by\_reference()}\spxextra{pypath.core.network.Network method}}

\begin{fulllineitems}
\phantomsection\label{\detokenize{reference:pypath.core.network.Network.identifiers_by_reference}}\pysiglinewithargsret{\sphinxbfcode{\sphinxupquote{identifiers\_by\_reference}}}{}{}
Built-in immutable sequence.

If no argument is given, the constructor returns an empty tuple.
If iterable is specified the tuple is initialized from iterable’s items.

If the argument is a tuple, the return value is the same object.

\end{fulllineitems}

\index{identifiers\_by\_resource() (pypath.core.network.Network method)@\spxentry{identifiers\_by\_resource()}\spxextra{pypath.core.network.Network method}}

\begin{fulllineitems}
\phantomsection\label{\detokenize{reference:pypath.core.network.Network.identifiers_by_resource}}\pysiglinewithargsret{\sphinxbfcode{\sphinxupquote{identifiers\_by\_resource}}}{}{}
Built-in immutable sequence.

If no argument is given, the constructor returns an empty tuple.
If iterable is specified the tuple is initialized from iterable’s items.

If the argument is a tuple, the return value is the same object.

\end{fulllineitems}

\index{init\_network() (pypath.core.network.Network method)@\spxentry{init\_network()}\spxextra{pypath.core.network.Network method}}

\begin{fulllineitems}
\phantomsection\label{\detokenize{reference:pypath.core.network.Network.init_network}}\pysiglinewithargsret{\sphinxbfcode{\sphinxupquote{init\_network}}}{\emph{resources=None}, \emph{make\_df=False}, \emph{exclude=None}, \emph{reread=False}, \emph{redownload=False}, \emph{keep\_raw=False}, \emph{top\_call=True}, \emph{cache\_files=None}, \emph{only\_directions=False}, \emph{pickle\_file=None}}{}
Loads data from a network resource or a collection of resources.
\begin{quote}\begin{description}
\item[{Parameters}] \leavevmode\begin{itemize}
\item {} 
\sphinxstyleliteralstrong{\sphinxupquote{resources}} (\sphinxstyleliteralemphasis{\sphinxupquote{str}}\sphinxstyleliteralemphasis{\sphinxupquote{,}}\sphinxstyleliteralemphasis{\sphinxupquote{dict}}\sphinxstyleliteralemphasis{\sphinxupquote{,}}\sphinxstyleliteralemphasis{\sphinxupquote{list}}\sphinxstyleliteralemphasis{\sphinxupquote{,}}\sphinxstyleliteralemphasis{\sphinxupquote{resource.NetworkResource}}) \textendash{} An object defining one or more network resources. If \sphinxstyleemphasis{str} it
will be looked up among the collections in the
\sphinxcode{\sphinxupquote{pypath.resources.network}} module (e.g. \sphinxcode{\sphinxupquote{'pathway'}} will load
all resources in the \sphinxtitleref{pathway} collection). If \sphinxstyleemphasis{dict} or \sphinxstyleemphasis{list}
it will be processed recursively i.e. the \sphinxcode{\sphinxupquote{load}} method will be
called for each element. If it is a
\sphinxcode{\sphinxupquote{pypath.resource.NetworkResource}} object it will be processed
and added to the network.

\item {} 
\sphinxstyleliteralstrong{\sphinxupquote{make\_df}} (\sphinxstyleliteralemphasis{\sphinxupquote{bool}}) \textendash{} Whether to create a \sphinxcode{\sphinxupquote{pandas.DataFrame}} after loading all
resources.

\item {} 
\sphinxstyleliteralstrong{\sphinxupquote{exclude}} (\sphinxstyleliteralemphasis{\sphinxupquote{NoneType}}\sphinxstyleliteralemphasis{\sphinxupquote{,}}\sphinxstyleliteralemphasis{\sphinxupquote{set}}) \textendash{} A \sphinxstyleemphasis{set} of resource names to be ignored. It is useful if you want
to load a collection with the exception of a few resources.

\end{itemize}

\end{description}\end{quote}

\end{fulllineitems}

\index{interaction() (pypath.core.network.Network method)@\spxentry{interaction()}\spxextra{pypath.core.network.Network method}}

\begin{fulllineitems}
\phantomsection\label{\detokenize{reference:pypath.core.network.Network.interaction}}\pysiglinewithargsret{\sphinxbfcode{\sphinxupquote{interaction}}}{\emph{a}, \emph{b}}{}
Retrieves the interaction \sphinxtitleref{a \textendash{}\textgreater{} b} if it exists in the network,
otherwise \sphinxtitleref{b \textendash{}\textgreater{} a}. If no interaction exist between \sphinxtitleref{a} and \sphinxtitleref{b}
returns \sphinxtitleref{None}.

\end{fulllineitems}

\index{interaction\_by\_id() (pypath.core.network.Network method)@\spxentry{interaction\_by\_id()}\spxextra{pypath.core.network.Network method}}

\begin{fulllineitems}
\phantomsection\label{\detokenize{reference:pypath.core.network.Network.interaction_by_id}}\pysiglinewithargsret{\sphinxbfcode{\sphinxupquote{interaction\_by\_id}}}{\emph{id\_a}, \emph{id\_b}}{}
Returns a \sphinxcode{\sphinxupquote{pypath.interaction.Interaction}} object by looking it up
based on a pair of identifiers. If the interaction does not exist
in the network \sphinxcode{\sphinxupquote{None}} will be returned.
\begin{quote}\begin{description}
\item[{Parameters}] \leavevmode\begin{itemize}
\item {} 
\sphinxstyleliteralstrong{\sphinxupquote{id\_a}} (\sphinxstyleliteralemphasis{\sphinxupquote{str}}) \textendash{} The identifier of one of the partners in the interaction. Unless
it’s been set otherwise for genes/proteins it is the UniProt ID.
E.g. \sphinxcode{\sphinxupquote{'P00533'}}.

\item {} 
\sphinxstyleliteralstrong{\sphinxupquote{id\_b}} (\sphinxstyleliteralemphasis{\sphinxupquote{str}}) \textendash{} The other partner, similarly to \sphinxcode{\sphinxupquote{id\_a}}. The order of the
partners does not matter here.

\end{itemize}

\end{description}\end{quote}

\end{fulllineitems}

\index{interaction\_by\_label() (pypath.core.network.Network method)@\spxentry{interaction\_by\_label()}\spxextra{pypath.core.network.Network method}}

\begin{fulllineitems}
\phantomsection\label{\detokenize{reference:pypath.core.network.Network.interaction_by_label}}\pysiglinewithargsret{\sphinxbfcode{\sphinxupquote{interaction\_by\_label}}}{\emph{label\_a}, \emph{label\_b}}{}
Returns a \sphinxcode{\sphinxupquote{pypath.interaction.Interaction}} object by looking it up
based on a pair of labels. If the interaction does not exist
in the network \sphinxcode{\sphinxupquote{None}} will be returned.
\begin{quote}\begin{description}
\item[{Parameters}] \leavevmode\begin{itemize}
\item {} 
\sphinxstyleliteralstrong{\sphinxupquote{label\_a}} (\sphinxstyleliteralemphasis{\sphinxupquote{str}}) \textendash{} The label of one of the partners in the interaction. Unless
it’s been set otherwise for genes/proteins it is the Gene Symbol.
E.g. \sphinxcode{\sphinxupquote{'EGFR'}}.

\item {} 
\sphinxstyleliteralstrong{\sphinxupquote{label\_b}} (\sphinxstyleliteralemphasis{\sphinxupquote{str}}) \textendash{} The other partner, similarly to \sphinxcode{\sphinxupquote{label\_a}}. The order of the
partners does not matter here.

\end{itemize}

\end{description}\end{quote}

\end{fulllineitems}

\index{interaction\_types\_by\_data\_model() (pypath.core.network.Network method)@\spxentry{interaction\_types\_by\_data\_model()}\spxextra{pypath.core.network.Network method}}

\begin{fulllineitems}
\phantomsection\label{\detokenize{reference:pypath.core.network.Network.interaction_types_by_data_model}}\pysiglinewithargsret{\sphinxbfcode{\sphinxupquote{interaction\_types\_by\_data\_model}}}{}{}
Built-in immutable sequence.

If no argument is given, the constructor returns an empty tuple.
If iterable is specified the tuple is initialized from iterable’s items.

If the argument is a tuple, the return value is the same object.

\end{fulllineitems}

\index{interaction\_types\_by\_interaction\_type() (pypath.core.network.Network method)@\spxentry{interaction\_types\_by\_interaction\_type()}\spxextra{pypath.core.network.Network method}}

\begin{fulllineitems}
\phantomsection\label{\detokenize{reference:pypath.core.network.Network.interaction_types_by_interaction_type}}\pysiglinewithargsret{\sphinxbfcode{\sphinxupquote{interaction\_types\_by\_interaction\_type}}}{}{}
Built-in immutable sequence.

If no argument is given, the constructor returns an empty tuple.
If iterable is specified the tuple is initialized from iterable’s items.

If the argument is a tuple, the return value is the same object.

\end{fulllineitems}

\index{interaction\_types\_by\_interaction\_type\_and\_data\_model() (pypath.core.network.Network method)@\spxentry{interaction\_types\_by\_interaction\_type\_and\_data\_model()}\spxextra{pypath.core.network.Network method}}

\begin{fulllineitems}
\phantomsection\label{\detokenize{reference:pypath.core.network.Network.interaction_types_by_interaction_type_and_data_model}}\pysiglinewithargsret{\sphinxbfcode{\sphinxupquote{interaction\_types\_by\_interaction\_type\_and\_data\_model}}}{}{}
Built-in immutable sequence.

If no argument is given, the constructor returns an empty tuple.
If iterable is specified the tuple is initialized from iterable’s items.

If the argument is a tuple, the return value is the same object.

\end{fulllineitems}

\index{interaction\_types\_by\_interaction\_type\_and\_data\_model\_and\_resource() (pypath.core.network.Network method)@\spxentry{interaction\_types\_by\_interaction\_type\_and\_data\_model\_and\_resource()}\spxextra{pypath.core.network.Network method}}

\begin{fulllineitems}
\phantomsection\label{\detokenize{reference:pypath.core.network.Network.interaction_types_by_interaction_type_and_data_model_and_resource}}\pysiglinewithargsret{\sphinxbfcode{\sphinxupquote{interaction\_types\_by\_interaction\_type\_and\_data\_model\_and\_resource}}}{}{}
Built-in immutable sequence.

If no argument is given, the constructor returns an empty tuple.
If iterable is specified the tuple is initialized from iterable’s items.

If the argument is a tuple, the return value is the same object.

\end{fulllineitems}

\index{interaction\_types\_by\_reference() (pypath.core.network.Network method)@\spxentry{interaction\_types\_by\_reference()}\spxextra{pypath.core.network.Network method}}

\begin{fulllineitems}
\phantomsection\label{\detokenize{reference:pypath.core.network.Network.interaction_types_by_reference}}\pysiglinewithargsret{\sphinxbfcode{\sphinxupquote{interaction\_types\_by\_reference}}}{}{}
Built-in immutable sequence.

If no argument is given, the constructor returns an empty tuple.
If iterable is specified the tuple is initialized from iterable’s items.

If the argument is a tuple, the return value is the same object.

\end{fulllineitems}

\index{interaction\_types\_by\_resource() (pypath.core.network.Network method)@\spxentry{interaction\_types\_by\_resource()}\spxextra{pypath.core.network.Network method}}

\begin{fulllineitems}
\phantomsection\label{\detokenize{reference:pypath.core.network.Network.interaction_types_by_resource}}\pysiglinewithargsret{\sphinxbfcode{\sphinxupquote{interaction\_types\_by\_resource}}}{}{}
Built-in immutable sequence.

If no argument is given, the constructor returns an empty tuple.
If iterable is specified the tuple is initialized from iterable’s items.

If the argument is a tuple, the return value is the same object.

\end{fulllineitems}

\index{interactions\_0\_by\_data\_model() (pypath.core.network.Network method)@\spxentry{interactions\_0\_by\_data\_model()}\spxextra{pypath.core.network.Network method}}

\begin{fulllineitems}
\phantomsection\label{\detokenize{reference:pypath.core.network.Network.interactions_0_by_data_model}}\pysiglinewithargsret{\sphinxbfcode{\sphinxupquote{interactions\_0\_by\_data\_model}}}{}{}
Built-in immutable sequence.

If no argument is given, the constructor returns an empty tuple.
If iterable is specified the tuple is initialized from iterable’s items.

If the argument is a tuple, the return value is the same object.

\end{fulllineitems}

\index{interactions\_0\_by\_interaction\_type() (pypath.core.network.Network method)@\spxentry{interactions\_0\_by\_interaction\_type()}\spxextra{pypath.core.network.Network method}}

\begin{fulllineitems}
\phantomsection\label{\detokenize{reference:pypath.core.network.Network.interactions_0_by_interaction_type}}\pysiglinewithargsret{\sphinxbfcode{\sphinxupquote{interactions\_0\_by\_interaction\_type}}}{}{}
Built-in immutable sequence.

If no argument is given, the constructor returns an empty tuple.
If iterable is specified the tuple is initialized from iterable’s items.

If the argument is a tuple, the return value is the same object.

\end{fulllineitems}

\index{interactions\_0\_by\_interaction\_type\_and\_data\_model() (pypath.core.network.Network method)@\spxentry{interactions\_0\_by\_interaction\_type\_and\_data\_model()}\spxextra{pypath.core.network.Network method}}

\begin{fulllineitems}
\phantomsection\label{\detokenize{reference:pypath.core.network.Network.interactions_0_by_interaction_type_and_data_model}}\pysiglinewithargsret{\sphinxbfcode{\sphinxupquote{interactions\_0\_by\_interaction\_type\_and\_data\_model}}}{}{}
Built-in immutable sequence.

If no argument is given, the constructor returns an empty tuple.
If iterable is specified the tuple is initialized from iterable’s items.

If the argument is a tuple, the return value is the same object.

\end{fulllineitems}

\index{interactions\_0\_by\_interaction\_type\_and\_data\_model\_and\_resource() (pypath.core.network.Network method)@\spxentry{interactions\_0\_by\_interaction\_type\_and\_data\_model\_and\_resource()}\spxextra{pypath.core.network.Network method}}

\begin{fulllineitems}
\phantomsection\label{\detokenize{reference:pypath.core.network.Network.interactions_0_by_interaction_type_and_data_model_and_resource}}\pysiglinewithargsret{\sphinxbfcode{\sphinxupquote{interactions\_0\_by\_interaction\_type\_and\_data\_model\_and\_resource}}}{}{}
Built-in immutable sequence.

If no argument is given, the constructor returns an empty tuple.
If iterable is specified the tuple is initialized from iterable’s items.

If the argument is a tuple, the return value is the same object.

\end{fulllineitems}

\index{interactions\_0\_by\_reference() (pypath.core.network.Network method)@\spxentry{interactions\_0\_by\_reference()}\spxextra{pypath.core.network.Network method}}

\begin{fulllineitems}
\phantomsection\label{\detokenize{reference:pypath.core.network.Network.interactions_0_by_reference}}\pysiglinewithargsret{\sphinxbfcode{\sphinxupquote{interactions\_0\_by\_reference}}}{}{}
Built-in immutable sequence.

If no argument is given, the constructor returns an empty tuple.
If iterable is specified the tuple is initialized from iterable’s items.

If the argument is a tuple, the return value is the same object.

\end{fulllineitems}

\index{interactions\_0\_by\_resource() (pypath.core.network.Network method)@\spxentry{interactions\_0\_by\_resource()}\spxextra{pypath.core.network.Network method}}

\begin{fulllineitems}
\phantomsection\label{\detokenize{reference:pypath.core.network.Network.interactions_0_by_resource}}\pysiglinewithargsret{\sphinxbfcode{\sphinxupquote{interactions\_0\_by\_resource}}}{}{}
Built-in immutable sequence.

If no argument is given, the constructor returns an empty tuple.
If iterable is specified the tuple is initialized from iterable’s items.

If the argument is a tuple, the return value is the same object.

\end{fulllineitems}

\index{interactions\_by\_data\_model() (pypath.core.network.Network method)@\spxentry{interactions\_by\_data\_model()}\spxextra{pypath.core.network.Network method}}

\begin{fulllineitems}
\phantomsection\label{\detokenize{reference:pypath.core.network.Network.interactions_by_data_model}}\pysiglinewithargsret{\sphinxbfcode{\sphinxupquote{interactions\_by\_data\_model}}}{}{}
Built-in immutable sequence.

If no argument is given, the constructor returns an empty tuple.
If iterable is specified the tuple is initialized from iterable’s items.

If the argument is a tuple, the return value is the same object.

\end{fulllineitems}

\index{interactions\_by\_interaction\_type() (pypath.core.network.Network method)@\spxentry{interactions\_by\_interaction\_type()}\spxextra{pypath.core.network.Network method}}

\begin{fulllineitems}
\phantomsection\label{\detokenize{reference:pypath.core.network.Network.interactions_by_interaction_type}}\pysiglinewithargsret{\sphinxbfcode{\sphinxupquote{interactions\_by\_interaction\_type}}}{}{}
Built-in immutable sequence.

If no argument is given, the constructor returns an empty tuple.
If iterable is specified the tuple is initialized from iterable’s items.

If the argument is a tuple, the return value is the same object.

\end{fulllineitems}

\index{interactions\_by\_interaction\_type\_and\_data\_model() (pypath.core.network.Network method)@\spxentry{interactions\_by\_interaction\_type\_and\_data\_model()}\spxextra{pypath.core.network.Network method}}

\begin{fulllineitems}
\phantomsection\label{\detokenize{reference:pypath.core.network.Network.interactions_by_interaction_type_and_data_model}}\pysiglinewithargsret{\sphinxbfcode{\sphinxupquote{interactions\_by\_interaction\_type\_and\_data\_model}}}{}{}
Built-in immutable sequence.

If no argument is given, the constructor returns an empty tuple.
If iterable is specified the tuple is initialized from iterable’s items.

If the argument is a tuple, the return value is the same object.

\end{fulllineitems}

\index{interactions\_by\_interaction\_type\_and\_data\_model\_and\_resource() (pypath.core.network.Network method)@\spxentry{interactions\_by\_interaction\_type\_and\_data\_model\_and\_resource()}\spxextra{pypath.core.network.Network method}}

\begin{fulllineitems}
\phantomsection\label{\detokenize{reference:pypath.core.network.Network.interactions_by_interaction_type_and_data_model_and_resource}}\pysiglinewithargsret{\sphinxbfcode{\sphinxupquote{interactions\_by\_interaction\_type\_and\_data\_model\_and\_resource}}}{}{}
Built-in immutable sequence.

If no argument is given, the constructor returns an empty tuple.
If iterable is specified the tuple is initialized from iterable’s items.

If the argument is a tuple, the return value is the same object.

\end{fulllineitems}

\index{interactions\_by\_reference() (pypath.core.network.Network method)@\spxentry{interactions\_by\_reference()}\spxextra{pypath.core.network.Network method}}

\begin{fulllineitems}
\phantomsection\label{\detokenize{reference:pypath.core.network.Network.interactions_by_reference}}\pysiglinewithargsret{\sphinxbfcode{\sphinxupquote{interactions\_by\_reference}}}{}{}
Built-in immutable sequence.

If no argument is given, the constructor returns an empty tuple.
If iterable is specified the tuple is initialized from iterable’s items.

If the argument is a tuple, the return value is the same object.

\end{fulllineitems}

\index{interactions\_by\_resource() (pypath.core.network.Network method)@\spxentry{interactions\_by\_resource()}\spxextra{pypath.core.network.Network method}}

\begin{fulllineitems}
\phantomsection\label{\detokenize{reference:pypath.core.network.Network.interactions_by_resource}}\pysiglinewithargsret{\sphinxbfcode{\sphinxupquote{interactions\_by\_resource}}}{}{}
Built-in immutable sequence.

If no argument is given, the constructor returns an empty tuple.
If iterable is specified the tuple is initialized from iterable’s items.

If the argument is a tuple, the return value is the same object.

\end{fulllineitems}

\index{interactions\_directed\_by\_data\_model() (pypath.core.network.Network method)@\spxentry{interactions\_directed\_by\_data\_model()}\spxextra{pypath.core.network.Network method}}

\begin{fulllineitems}
\phantomsection\label{\detokenize{reference:pypath.core.network.Network.interactions_directed_by_data_model}}\pysiglinewithargsret{\sphinxbfcode{\sphinxupquote{interactions\_directed\_by\_data\_model}}}{}{}
Built-in immutable sequence.

If no argument is given, the constructor returns an empty tuple.
If iterable is specified the tuple is initialized from iterable’s items.

If the argument is a tuple, the return value is the same object.

\end{fulllineitems}

\index{interactions\_directed\_by\_interaction\_type() (pypath.core.network.Network method)@\spxentry{interactions\_directed\_by\_interaction\_type()}\spxextra{pypath.core.network.Network method}}

\begin{fulllineitems}
\phantomsection\label{\detokenize{reference:pypath.core.network.Network.interactions_directed_by_interaction_type}}\pysiglinewithargsret{\sphinxbfcode{\sphinxupquote{interactions\_directed\_by\_interaction\_type}}}{}{}
Built-in immutable sequence.

If no argument is given, the constructor returns an empty tuple.
If iterable is specified the tuple is initialized from iterable’s items.

If the argument is a tuple, the return value is the same object.

\end{fulllineitems}

\index{interactions\_directed\_by\_interaction\_type\_and\_data\_model() (pypath.core.network.Network method)@\spxentry{interactions\_directed\_by\_interaction\_type\_and\_data\_model()}\spxextra{pypath.core.network.Network method}}

\begin{fulllineitems}
\phantomsection\label{\detokenize{reference:pypath.core.network.Network.interactions_directed_by_interaction_type_and_data_model}}\pysiglinewithargsret{\sphinxbfcode{\sphinxupquote{interactions\_directed\_by\_interaction\_type\_and\_data\_model}}}{}{}
Built-in immutable sequence.

If no argument is given, the constructor returns an empty tuple.
If iterable is specified the tuple is initialized from iterable’s items.

If the argument is a tuple, the return value is the same object.

\end{fulllineitems}

\index{interactions\_directed\_by\_interaction\_type\_and\_data\_model\_and\_resource() (pypath.core.network.Network method)@\spxentry{interactions\_directed\_by\_interaction\_type\_and\_data\_model\_and\_resource()}\spxextra{pypath.core.network.Network method}}

\begin{fulllineitems}
\phantomsection\label{\detokenize{reference:pypath.core.network.Network.interactions_directed_by_interaction_type_and_data_model_and_resource}}\pysiglinewithargsret{\sphinxbfcode{\sphinxupquote{interactions\_directed\_by\_interaction\_type\_and\_data\_model\_and\_resource}}}{}{}
Built-in immutable sequence.

If no argument is given, the constructor returns an empty tuple.
If iterable is specified the tuple is initialized from iterable’s items.

If the argument is a tuple, the return value is the same object.

\end{fulllineitems}

\index{interactions\_directed\_by\_reference() (pypath.core.network.Network method)@\spxentry{interactions\_directed\_by\_reference()}\spxextra{pypath.core.network.Network method}}

\begin{fulllineitems}
\phantomsection\label{\detokenize{reference:pypath.core.network.Network.interactions_directed_by_reference}}\pysiglinewithargsret{\sphinxbfcode{\sphinxupquote{interactions\_directed\_by\_reference}}}{}{}
Built-in immutable sequence.

If no argument is given, the constructor returns an empty tuple.
If iterable is specified the tuple is initialized from iterable’s items.

If the argument is a tuple, the return value is the same object.

\end{fulllineitems}

\index{interactions\_directed\_by\_resource() (pypath.core.network.Network method)@\spxentry{interactions\_directed\_by\_resource()}\spxextra{pypath.core.network.Network method}}

\begin{fulllineitems}
\phantomsection\label{\detokenize{reference:pypath.core.network.Network.interactions_directed_by_resource}}\pysiglinewithargsret{\sphinxbfcode{\sphinxupquote{interactions\_directed\_by\_resource}}}{}{}
Built-in immutable sequence.

If no argument is given, the constructor returns an empty tuple.
If iterable is specified the tuple is initialized from iterable’s items.

If the argument is a tuple, the return value is the same object.

\end{fulllineitems}

\index{interactions\_mutual\_by\_data\_model() (pypath.core.network.Network method)@\spxentry{interactions\_mutual\_by\_data\_model()}\spxextra{pypath.core.network.Network method}}

\begin{fulllineitems}
\phantomsection\label{\detokenize{reference:pypath.core.network.Network.interactions_mutual_by_data_model}}\pysiglinewithargsret{\sphinxbfcode{\sphinxupquote{interactions\_mutual\_by\_data\_model}}}{}{}
Built-in immutable sequence.

If no argument is given, the constructor returns an empty tuple.
If iterable is specified the tuple is initialized from iterable’s items.

If the argument is a tuple, the return value is the same object.

\end{fulllineitems}

\index{interactions\_mutual\_by\_interaction\_type() (pypath.core.network.Network method)@\spxentry{interactions\_mutual\_by\_interaction\_type()}\spxextra{pypath.core.network.Network method}}

\begin{fulllineitems}
\phantomsection\label{\detokenize{reference:pypath.core.network.Network.interactions_mutual_by_interaction_type}}\pysiglinewithargsret{\sphinxbfcode{\sphinxupquote{interactions\_mutual\_by\_interaction\_type}}}{}{}
Built-in immutable sequence.

If no argument is given, the constructor returns an empty tuple.
If iterable is specified the tuple is initialized from iterable’s items.

If the argument is a tuple, the return value is the same object.

\end{fulllineitems}

\index{interactions\_mutual\_by\_interaction\_type\_and\_data\_model() (pypath.core.network.Network method)@\spxentry{interactions\_mutual\_by\_interaction\_type\_and\_data\_model()}\spxextra{pypath.core.network.Network method}}

\begin{fulllineitems}
\phantomsection\label{\detokenize{reference:pypath.core.network.Network.interactions_mutual_by_interaction_type_and_data_model}}\pysiglinewithargsret{\sphinxbfcode{\sphinxupquote{interactions\_mutual\_by\_interaction\_type\_and\_data\_model}}}{}{}
Built-in immutable sequence.

If no argument is given, the constructor returns an empty tuple.
If iterable is specified the tuple is initialized from iterable’s items.

If the argument is a tuple, the return value is the same object.

\end{fulllineitems}

\index{interactions\_mutual\_by\_interaction\_type\_and\_data\_model\_and\_resource() (pypath.core.network.Network method)@\spxentry{interactions\_mutual\_by\_interaction\_type\_and\_data\_model\_and\_resource()}\spxextra{pypath.core.network.Network method}}

\begin{fulllineitems}
\phantomsection\label{\detokenize{reference:pypath.core.network.Network.interactions_mutual_by_interaction_type_and_data_model_and_resource}}\pysiglinewithargsret{\sphinxbfcode{\sphinxupquote{interactions\_mutual\_by\_interaction\_type\_and\_data\_model\_and\_resource}}}{}{}
Built-in immutable sequence.

If no argument is given, the constructor returns an empty tuple.
If iterable is specified the tuple is initialized from iterable’s items.

If the argument is a tuple, the return value is the same object.

\end{fulllineitems}

\index{interactions\_mutual\_by\_reference() (pypath.core.network.Network method)@\spxentry{interactions\_mutual\_by\_reference()}\spxextra{pypath.core.network.Network method}}

\begin{fulllineitems}
\phantomsection\label{\detokenize{reference:pypath.core.network.Network.interactions_mutual_by_reference}}\pysiglinewithargsret{\sphinxbfcode{\sphinxupquote{interactions\_mutual\_by\_reference}}}{}{}
Built-in immutable sequence.

If no argument is given, the constructor returns an empty tuple.
If iterable is specified the tuple is initialized from iterable’s items.

If the argument is a tuple, the return value is the same object.

\end{fulllineitems}

\index{interactions\_mutual\_by\_resource() (pypath.core.network.Network method)@\spxentry{interactions\_mutual\_by\_resource()}\spxextra{pypath.core.network.Network method}}

\begin{fulllineitems}
\phantomsection\label{\detokenize{reference:pypath.core.network.Network.interactions_mutual_by_resource}}\pysiglinewithargsret{\sphinxbfcode{\sphinxupquote{interactions\_mutual\_by\_resource}}}{}{}
Built-in immutable sequence.

If no argument is given, the constructor returns an empty tuple.
If iterable is specified the tuple is initialized from iterable’s items.

If the argument is a tuple, the return value is the same object.

\end{fulllineitems}

\index{interactions\_negative\_by\_data\_model() (pypath.core.network.Network method)@\spxentry{interactions\_negative\_by\_data\_model()}\spxextra{pypath.core.network.Network method}}

\begin{fulllineitems}
\phantomsection\label{\detokenize{reference:pypath.core.network.Network.interactions_negative_by_data_model}}\pysiglinewithargsret{\sphinxbfcode{\sphinxupquote{interactions\_negative\_by\_data\_model}}}{}{}
Built-in immutable sequence.

If no argument is given, the constructor returns an empty tuple.
If iterable is specified the tuple is initialized from iterable’s items.

If the argument is a tuple, the return value is the same object.

\end{fulllineitems}

\index{interactions\_negative\_by\_interaction\_type() (pypath.core.network.Network method)@\spxentry{interactions\_negative\_by\_interaction\_type()}\spxextra{pypath.core.network.Network method}}

\begin{fulllineitems}
\phantomsection\label{\detokenize{reference:pypath.core.network.Network.interactions_negative_by_interaction_type}}\pysiglinewithargsret{\sphinxbfcode{\sphinxupquote{interactions\_negative\_by\_interaction\_type}}}{}{}
Built-in immutable sequence.

If no argument is given, the constructor returns an empty tuple.
If iterable is specified the tuple is initialized from iterable’s items.

If the argument is a tuple, the return value is the same object.

\end{fulllineitems}

\index{interactions\_negative\_by\_interaction\_type\_and\_data\_model() (pypath.core.network.Network method)@\spxentry{interactions\_negative\_by\_interaction\_type\_and\_data\_model()}\spxextra{pypath.core.network.Network method}}

\begin{fulllineitems}
\phantomsection\label{\detokenize{reference:pypath.core.network.Network.interactions_negative_by_interaction_type_and_data_model}}\pysiglinewithargsret{\sphinxbfcode{\sphinxupquote{interactions\_negative\_by\_interaction\_type\_and\_data\_model}}}{}{}
Built-in immutable sequence.

If no argument is given, the constructor returns an empty tuple.
If iterable is specified the tuple is initialized from iterable’s items.

If the argument is a tuple, the return value is the same object.

\end{fulllineitems}

\index{interactions\_negative\_by\_interaction\_type\_and\_data\_model\_and\_resource() (pypath.core.network.Network method)@\spxentry{interactions\_negative\_by\_interaction\_type\_and\_data\_model\_and\_resource()}\spxextra{pypath.core.network.Network method}}

\begin{fulllineitems}
\phantomsection\label{\detokenize{reference:pypath.core.network.Network.interactions_negative_by_interaction_type_and_data_model_and_resource}}\pysiglinewithargsret{\sphinxbfcode{\sphinxupquote{interactions\_negative\_by\_interaction\_type\_and\_data\_model\_and\_resource}}}{}{}
Built-in immutable sequence.

If no argument is given, the constructor returns an empty tuple.
If iterable is specified the tuple is initialized from iterable’s items.

If the argument is a tuple, the return value is the same object.

\end{fulllineitems}

\index{interactions\_negative\_by\_reference() (pypath.core.network.Network method)@\spxentry{interactions\_negative\_by\_reference()}\spxextra{pypath.core.network.Network method}}

\begin{fulllineitems}
\phantomsection\label{\detokenize{reference:pypath.core.network.Network.interactions_negative_by_reference}}\pysiglinewithargsret{\sphinxbfcode{\sphinxupquote{interactions\_negative\_by\_reference}}}{}{}
Built-in immutable sequence.

If no argument is given, the constructor returns an empty tuple.
If iterable is specified the tuple is initialized from iterable’s items.

If the argument is a tuple, the return value is the same object.

\end{fulllineitems}

\index{interactions\_negative\_by\_resource() (pypath.core.network.Network method)@\spxentry{interactions\_negative\_by\_resource()}\spxextra{pypath.core.network.Network method}}

\begin{fulllineitems}
\phantomsection\label{\detokenize{reference:pypath.core.network.Network.interactions_negative_by_resource}}\pysiglinewithargsret{\sphinxbfcode{\sphinxupquote{interactions\_negative\_by\_resource}}}{}{}
Built-in immutable sequence.

If no argument is given, the constructor returns an empty tuple.
If iterable is specified the tuple is initialized from iterable’s items.

If the argument is a tuple, the return value is the same object.

\end{fulllineitems}

\index{interactions\_non\_directed\_0\_by\_data\_model() (pypath.core.network.Network method)@\spxentry{interactions\_non\_directed\_0\_by\_data\_model()}\spxextra{pypath.core.network.Network method}}

\begin{fulllineitems}
\phantomsection\label{\detokenize{reference:pypath.core.network.Network.interactions_non_directed_0_by_data_model}}\pysiglinewithargsret{\sphinxbfcode{\sphinxupquote{interactions\_non\_directed\_0\_by\_data\_model}}}{}{}
Built-in immutable sequence.

If no argument is given, the constructor returns an empty tuple.
If iterable is specified the tuple is initialized from iterable’s items.

If the argument is a tuple, the return value is the same object.

\end{fulllineitems}

\index{interactions\_non\_directed\_0\_by\_interaction\_type() (pypath.core.network.Network method)@\spxentry{interactions\_non\_directed\_0\_by\_interaction\_type()}\spxextra{pypath.core.network.Network method}}

\begin{fulllineitems}
\phantomsection\label{\detokenize{reference:pypath.core.network.Network.interactions_non_directed_0_by_interaction_type}}\pysiglinewithargsret{\sphinxbfcode{\sphinxupquote{interactions\_non\_directed\_0\_by\_interaction\_type}}}{}{}
Built-in immutable sequence.

If no argument is given, the constructor returns an empty tuple.
If iterable is specified the tuple is initialized from iterable’s items.

If the argument is a tuple, the return value is the same object.

\end{fulllineitems}

\index{interactions\_non\_directed\_0\_by\_interaction\_type\_and\_data\_model() (pypath.core.network.Network method)@\spxentry{interactions\_non\_directed\_0\_by\_interaction\_type\_and\_data\_model()}\spxextra{pypath.core.network.Network method}}

\begin{fulllineitems}
\phantomsection\label{\detokenize{reference:pypath.core.network.Network.interactions_non_directed_0_by_interaction_type_and_data_model}}\pysiglinewithargsret{\sphinxbfcode{\sphinxupquote{interactions\_non\_directed\_0\_by\_interaction\_type\_and\_data\_model}}}{}{}
Built-in immutable sequence.

If no argument is given, the constructor returns an empty tuple.
If iterable is specified the tuple is initialized from iterable’s items.

If the argument is a tuple, the return value is the same object.

\end{fulllineitems}

\index{interactions\_non\_directed\_0\_by\_interaction\_type\_and\_data\_model\_and\_resource() (pypath.core.network.Network method)@\spxentry{interactions\_non\_directed\_0\_by\_interaction\_type\_and\_data\_model\_and\_resource()}\spxextra{pypath.core.network.Network method}}

\begin{fulllineitems}
\phantomsection\label{\detokenize{reference:pypath.core.network.Network.interactions_non_directed_0_by_interaction_type_and_data_model_and_resource}}\pysiglinewithargsret{\sphinxbfcode{\sphinxupquote{interactions\_non\_directed\_0\_by\_interaction\_type\_and\_data\_model\_and\_resource}}}{}{}
Built-in immutable sequence.

If no argument is given, the constructor returns an empty tuple.
If iterable is specified the tuple is initialized from iterable’s items.

If the argument is a tuple, the return value is the same object.

\end{fulllineitems}

\index{interactions\_non\_directed\_0\_by\_reference() (pypath.core.network.Network method)@\spxentry{interactions\_non\_directed\_0\_by\_reference()}\spxextra{pypath.core.network.Network method}}

\begin{fulllineitems}
\phantomsection\label{\detokenize{reference:pypath.core.network.Network.interactions_non_directed_0_by_reference}}\pysiglinewithargsret{\sphinxbfcode{\sphinxupquote{interactions\_non\_directed\_0\_by\_reference}}}{}{}
Built-in immutable sequence.

If no argument is given, the constructor returns an empty tuple.
If iterable is specified the tuple is initialized from iterable’s items.

If the argument is a tuple, the return value is the same object.

\end{fulllineitems}

\index{interactions\_non\_directed\_0\_by\_resource() (pypath.core.network.Network method)@\spxentry{interactions\_non\_directed\_0\_by\_resource()}\spxextra{pypath.core.network.Network method}}

\begin{fulllineitems}
\phantomsection\label{\detokenize{reference:pypath.core.network.Network.interactions_non_directed_0_by_resource}}\pysiglinewithargsret{\sphinxbfcode{\sphinxupquote{interactions\_non\_directed\_0\_by\_resource}}}{}{}
Built-in immutable sequence.

If no argument is given, the constructor returns an empty tuple.
If iterable is specified the tuple is initialized from iterable’s items.

If the argument is a tuple, the return value is the same object.

\end{fulllineitems}

\index{interactions\_non\_directed\_by\_data\_model() (pypath.core.network.Network method)@\spxentry{interactions\_non\_directed\_by\_data\_model()}\spxextra{pypath.core.network.Network method}}

\begin{fulllineitems}
\phantomsection\label{\detokenize{reference:pypath.core.network.Network.interactions_non_directed_by_data_model}}\pysiglinewithargsret{\sphinxbfcode{\sphinxupquote{interactions\_non\_directed\_by\_data\_model}}}{}{}
Built-in immutable sequence.

If no argument is given, the constructor returns an empty tuple.
If iterable is specified the tuple is initialized from iterable’s items.

If the argument is a tuple, the return value is the same object.

\end{fulllineitems}

\index{interactions\_non\_directed\_by\_interaction\_type() (pypath.core.network.Network method)@\spxentry{interactions\_non\_directed\_by\_interaction\_type()}\spxextra{pypath.core.network.Network method}}

\begin{fulllineitems}
\phantomsection\label{\detokenize{reference:pypath.core.network.Network.interactions_non_directed_by_interaction_type}}\pysiglinewithargsret{\sphinxbfcode{\sphinxupquote{interactions\_non\_directed\_by\_interaction\_type}}}{}{}
Built-in immutable sequence.

If no argument is given, the constructor returns an empty tuple.
If iterable is specified the tuple is initialized from iterable’s items.

If the argument is a tuple, the return value is the same object.

\end{fulllineitems}

\index{interactions\_non\_directed\_by\_interaction\_type\_and\_data\_model() (pypath.core.network.Network method)@\spxentry{interactions\_non\_directed\_by\_interaction\_type\_and\_data\_model()}\spxextra{pypath.core.network.Network method}}

\begin{fulllineitems}
\phantomsection\label{\detokenize{reference:pypath.core.network.Network.interactions_non_directed_by_interaction_type_and_data_model}}\pysiglinewithargsret{\sphinxbfcode{\sphinxupquote{interactions\_non\_directed\_by\_interaction\_type\_and\_data\_model}}}{}{}
Built-in immutable sequence.

If no argument is given, the constructor returns an empty tuple.
If iterable is specified the tuple is initialized from iterable’s items.

If the argument is a tuple, the return value is the same object.

\end{fulllineitems}

\index{interactions\_non\_directed\_by\_interaction\_type\_and\_data\_model\_and\_resource() (pypath.core.network.Network method)@\spxentry{interactions\_non\_directed\_by\_interaction\_type\_and\_data\_model\_and\_resource()}\spxextra{pypath.core.network.Network method}}

\begin{fulllineitems}
\phantomsection\label{\detokenize{reference:pypath.core.network.Network.interactions_non_directed_by_interaction_type_and_data_model_and_resource}}\pysiglinewithargsret{\sphinxbfcode{\sphinxupquote{interactions\_non\_directed\_by\_interaction\_type\_and\_data\_model\_and\_resource}}}{}{}
Built-in immutable sequence.

If no argument is given, the constructor returns an empty tuple.
If iterable is specified the tuple is initialized from iterable’s items.

If the argument is a tuple, the return value is the same object.

\end{fulllineitems}

\index{interactions\_non\_directed\_by\_reference() (pypath.core.network.Network method)@\spxentry{interactions\_non\_directed\_by\_reference()}\spxextra{pypath.core.network.Network method}}

\begin{fulllineitems}
\phantomsection\label{\detokenize{reference:pypath.core.network.Network.interactions_non_directed_by_reference}}\pysiglinewithargsret{\sphinxbfcode{\sphinxupquote{interactions\_non\_directed\_by\_reference}}}{}{}
Built-in immutable sequence.

If no argument is given, the constructor returns an empty tuple.
If iterable is specified the tuple is initialized from iterable’s items.

If the argument is a tuple, the return value is the same object.

\end{fulllineitems}

\index{interactions\_non\_directed\_by\_resource() (pypath.core.network.Network method)@\spxentry{interactions\_non\_directed\_by\_resource()}\spxextra{pypath.core.network.Network method}}

\begin{fulllineitems}
\phantomsection\label{\detokenize{reference:pypath.core.network.Network.interactions_non_directed_by_resource}}\pysiglinewithargsret{\sphinxbfcode{\sphinxupquote{interactions\_non\_directed\_by\_resource}}}{}{}
Built-in immutable sequence.

If no argument is given, the constructor returns an empty tuple.
If iterable is specified the tuple is initialized from iterable’s items.

If the argument is a tuple, the return value is the same object.

\end{fulllineitems}

\index{interactions\_positive\_by\_data\_model() (pypath.core.network.Network method)@\spxentry{interactions\_positive\_by\_data\_model()}\spxextra{pypath.core.network.Network method}}

\begin{fulllineitems}
\phantomsection\label{\detokenize{reference:pypath.core.network.Network.interactions_positive_by_data_model}}\pysiglinewithargsret{\sphinxbfcode{\sphinxupquote{interactions\_positive\_by\_data\_model}}}{}{}
Built-in immutable sequence.

If no argument is given, the constructor returns an empty tuple.
If iterable is specified the tuple is initialized from iterable’s items.

If the argument is a tuple, the return value is the same object.

\end{fulllineitems}

\index{interactions\_positive\_by\_interaction\_type() (pypath.core.network.Network method)@\spxentry{interactions\_positive\_by\_interaction\_type()}\spxextra{pypath.core.network.Network method}}

\begin{fulllineitems}
\phantomsection\label{\detokenize{reference:pypath.core.network.Network.interactions_positive_by_interaction_type}}\pysiglinewithargsret{\sphinxbfcode{\sphinxupquote{interactions\_positive\_by\_interaction\_type}}}{}{}
Built-in immutable sequence.

If no argument is given, the constructor returns an empty tuple.
If iterable is specified the tuple is initialized from iterable’s items.

If the argument is a tuple, the return value is the same object.

\end{fulllineitems}

\index{interactions\_positive\_by\_interaction\_type\_and\_data\_model() (pypath.core.network.Network method)@\spxentry{interactions\_positive\_by\_interaction\_type\_and\_data\_model()}\spxextra{pypath.core.network.Network method}}

\begin{fulllineitems}
\phantomsection\label{\detokenize{reference:pypath.core.network.Network.interactions_positive_by_interaction_type_and_data_model}}\pysiglinewithargsret{\sphinxbfcode{\sphinxupquote{interactions\_positive\_by\_interaction\_type\_and\_data\_model}}}{}{}
Built-in immutable sequence.

If no argument is given, the constructor returns an empty tuple.
If iterable is specified the tuple is initialized from iterable’s items.

If the argument is a tuple, the return value is the same object.

\end{fulllineitems}

\index{interactions\_positive\_by\_interaction\_type\_and\_data\_model\_and\_resource() (pypath.core.network.Network method)@\spxentry{interactions\_positive\_by\_interaction\_type\_and\_data\_model\_and\_resource()}\spxextra{pypath.core.network.Network method}}

\begin{fulllineitems}
\phantomsection\label{\detokenize{reference:pypath.core.network.Network.interactions_positive_by_interaction_type_and_data_model_and_resource}}\pysiglinewithargsret{\sphinxbfcode{\sphinxupquote{interactions\_positive\_by\_interaction\_type\_and\_data\_model\_and\_resource}}}{}{}
Built-in immutable sequence.

If no argument is given, the constructor returns an empty tuple.
If iterable is specified the tuple is initialized from iterable’s items.

If the argument is a tuple, the return value is the same object.

\end{fulllineitems}

\index{interactions\_positive\_by\_reference() (pypath.core.network.Network method)@\spxentry{interactions\_positive\_by\_reference()}\spxextra{pypath.core.network.Network method}}

\begin{fulllineitems}
\phantomsection\label{\detokenize{reference:pypath.core.network.Network.interactions_positive_by_reference}}\pysiglinewithargsret{\sphinxbfcode{\sphinxupquote{interactions\_positive\_by\_reference}}}{}{}
Built-in immutable sequence.

If no argument is given, the constructor returns an empty tuple.
If iterable is specified the tuple is initialized from iterable’s items.

If the argument is a tuple, the return value is the same object.

\end{fulllineitems}

\index{interactions\_positive\_by\_resource() (pypath.core.network.Network method)@\spxentry{interactions\_positive\_by\_resource()}\spxextra{pypath.core.network.Network method}}

\begin{fulllineitems}
\phantomsection\label{\detokenize{reference:pypath.core.network.Network.interactions_positive_by_resource}}\pysiglinewithargsret{\sphinxbfcode{\sphinxupquote{interactions\_positive\_by\_resource}}}{}{}
Built-in immutable sequence.

If no argument is given, the constructor returns an empty tuple.
If iterable is specified the tuple is initialized from iterable’s items.

If the argument is a tuple, the return value is the same object.

\end{fulllineitems}

\index{interactions\_signed\_by\_data\_model() (pypath.core.network.Network method)@\spxentry{interactions\_signed\_by\_data\_model()}\spxextra{pypath.core.network.Network method}}

\begin{fulllineitems}
\phantomsection\label{\detokenize{reference:pypath.core.network.Network.interactions_signed_by_data_model}}\pysiglinewithargsret{\sphinxbfcode{\sphinxupquote{interactions\_signed\_by\_data\_model}}}{}{}
Built-in immutable sequence.

If no argument is given, the constructor returns an empty tuple.
If iterable is specified the tuple is initialized from iterable’s items.

If the argument is a tuple, the return value is the same object.

\end{fulllineitems}

\index{interactions\_signed\_by\_interaction\_type() (pypath.core.network.Network method)@\spxentry{interactions\_signed\_by\_interaction\_type()}\spxextra{pypath.core.network.Network method}}

\begin{fulllineitems}
\phantomsection\label{\detokenize{reference:pypath.core.network.Network.interactions_signed_by_interaction_type}}\pysiglinewithargsret{\sphinxbfcode{\sphinxupquote{interactions\_signed\_by\_interaction\_type}}}{}{}
Built-in immutable sequence.

If no argument is given, the constructor returns an empty tuple.
If iterable is specified the tuple is initialized from iterable’s items.

If the argument is a tuple, the return value is the same object.

\end{fulllineitems}

\index{interactions\_signed\_by\_interaction\_type\_and\_data\_model() (pypath.core.network.Network method)@\spxentry{interactions\_signed\_by\_interaction\_type\_and\_data\_model()}\spxextra{pypath.core.network.Network method}}

\begin{fulllineitems}
\phantomsection\label{\detokenize{reference:pypath.core.network.Network.interactions_signed_by_interaction_type_and_data_model}}\pysiglinewithargsret{\sphinxbfcode{\sphinxupquote{interactions\_signed\_by\_interaction\_type\_and\_data\_model}}}{}{}
Built-in immutable sequence.

If no argument is given, the constructor returns an empty tuple.
If iterable is specified the tuple is initialized from iterable’s items.

If the argument is a tuple, the return value is the same object.

\end{fulllineitems}

\index{interactions\_signed\_by\_interaction\_type\_and\_data\_model\_and\_resource() (pypath.core.network.Network method)@\spxentry{interactions\_signed\_by\_interaction\_type\_and\_data\_model\_and\_resource()}\spxextra{pypath.core.network.Network method}}

\begin{fulllineitems}
\phantomsection\label{\detokenize{reference:pypath.core.network.Network.interactions_signed_by_interaction_type_and_data_model_and_resource}}\pysiglinewithargsret{\sphinxbfcode{\sphinxupquote{interactions\_signed\_by\_interaction\_type\_and\_data\_model\_and\_resource}}}{}{}
Built-in immutable sequence.

If no argument is given, the constructor returns an empty tuple.
If iterable is specified the tuple is initialized from iterable’s items.

If the argument is a tuple, the return value is the same object.

\end{fulllineitems}

\index{interactions\_signed\_by\_reference() (pypath.core.network.Network method)@\spxentry{interactions\_signed\_by\_reference()}\spxextra{pypath.core.network.Network method}}

\begin{fulllineitems}
\phantomsection\label{\detokenize{reference:pypath.core.network.Network.interactions_signed_by_reference}}\pysiglinewithargsret{\sphinxbfcode{\sphinxupquote{interactions\_signed\_by\_reference}}}{}{}
Built-in immutable sequence.

If no argument is given, the constructor returns an empty tuple.
If iterable is specified the tuple is initialized from iterable’s items.

If the argument is a tuple, the return value is the same object.

\end{fulllineitems}

\index{interactions\_signed\_by\_resource() (pypath.core.network.Network method)@\spxentry{interactions\_signed\_by\_resource()}\spxextra{pypath.core.network.Network method}}

\begin{fulllineitems}
\phantomsection\label{\detokenize{reference:pypath.core.network.Network.interactions_signed_by_resource}}\pysiglinewithargsret{\sphinxbfcode{\sphinxupquote{interactions\_signed\_by\_resource}}}{}{}
Built-in immutable sequence.

If no argument is given, the constructor returns an empty tuple.
If iterable is specified the tuple is initialized from iterable’s items.

If the argument is a tuple, the return value is the same object.

\end{fulllineitems}

\index{interactions\_undirected\_0\_by\_data\_model() (pypath.core.network.Network method)@\spxentry{interactions\_undirected\_0\_by\_data\_model()}\spxextra{pypath.core.network.Network method}}

\begin{fulllineitems}
\phantomsection\label{\detokenize{reference:pypath.core.network.Network.interactions_undirected_0_by_data_model}}\pysiglinewithargsret{\sphinxbfcode{\sphinxupquote{interactions\_undirected\_0\_by\_data\_model}}}{}{}
Built-in immutable sequence.

If no argument is given, the constructor returns an empty tuple.
If iterable is specified the tuple is initialized from iterable’s items.

If the argument is a tuple, the return value is the same object.

\end{fulllineitems}

\index{interactions\_undirected\_0\_by\_interaction\_type() (pypath.core.network.Network method)@\spxentry{interactions\_undirected\_0\_by\_interaction\_type()}\spxextra{pypath.core.network.Network method}}

\begin{fulllineitems}
\phantomsection\label{\detokenize{reference:pypath.core.network.Network.interactions_undirected_0_by_interaction_type}}\pysiglinewithargsret{\sphinxbfcode{\sphinxupquote{interactions\_undirected\_0\_by\_interaction\_type}}}{}{}
Built-in immutable sequence.

If no argument is given, the constructor returns an empty tuple.
If iterable is specified the tuple is initialized from iterable’s items.

If the argument is a tuple, the return value is the same object.

\end{fulllineitems}

\index{interactions\_undirected\_0\_by\_interaction\_type\_and\_data\_model() (pypath.core.network.Network method)@\spxentry{interactions\_undirected\_0\_by\_interaction\_type\_and\_data\_model()}\spxextra{pypath.core.network.Network method}}

\begin{fulllineitems}
\phantomsection\label{\detokenize{reference:pypath.core.network.Network.interactions_undirected_0_by_interaction_type_and_data_model}}\pysiglinewithargsret{\sphinxbfcode{\sphinxupquote{interactions\_undirected\_0\_by\_interaction\_type\_and\_data\_model}}}{}{}
Built-in immutable sequence.

If no argument is given, the constructor returns an empty tuple.
If iterable is specified the tuple is initialized from iterable’s items.

If the argument is a tuple, the return value is the same object.

\end{fulllineitems}

\index{interactions\_undirected\_0\_by\_interaction\_type\_and\_data\_model\_and\_resource() (pypath.core.network.Network method)@\spxentry{interactions\_undirected\_0\_by\_interaction\_type\_and\_data\_model\_and\_resource()}\spxextra{pypath.core.network.Network method}}

\begin{fulllineitems}
\phantomsection\label{\detokenize{reference:pypath.core.network.Network.interactions_undirected_0_by_interaction_type_and_data_model_and_resource}}\pysiglinewithargsret{\sphinxbfcode{\sphinxupquote{interactions\_undirected\_0\_by\_interaction\_type\_and\_data\_model\_and\_resource}}}{}{}
Built-in immutable sequence.

If no argument is given, the constructor returns an empty tuple.
If iterable is specified the tuple is initialized from iterable’s items.

If the argument is a tuple, the return value is the same object.

\end{fulllineitems}

\index{interactions\_undirected\_0\_by\_reference() (pypath.core.network.Network method)@\spxentry{interactions\_undirected\_0\_by\_reference()}\spxextra{pypath.core.network.Network method}}

\begin{fulllineitems}
\phantomsection\label{\detokenize{reference:pypath.core.network.Network.interactions_undirected_0_by_reference}}\pysiglinewithargsret{\sphinxbfcode{\sphinxupquote{interactions\_undirected\_0\_by\_reference}}}{}{}
Built-in immutable sequence.

If no argument is given, the constructor returns an empty tuple.
If iterable is specified the tuple is initialized from iterable’s items.

If the argument is a tuple, the return value is the same object.

\end{fulllineitems}

\index{interactions\_undirected\_0\_by\_resource() (pypath.core.network.Network method)@\spxentry{interactions\_undirected\_0\_by\_resource()}\spxextra{pypath.core.network.Network method}}

\begin{fulllineitems}
\phantomsection\label{\detokenize{reference:pypath.core.network.Network.interactions_undirected_0_by_resource}}\pysiglinewithargsret{\sphinxbfcode{\sphinxupquote{interactions\_undirected\_0\_by\_resource}}}{}{}
Built-in immutable sequence.

If no argument is given, the constructor returns an empty tuple.
If iterable is specified the tuple is initialized from iterable’s items.

If the argument is a tuple, the return value is the same object.

\end{fulllineitems}

\index{interactions\_undirected\_by\_data\_model() (pypath.core.network.Network method)@\spxentry{interactions\_undirected\_by\_data\_model()}\spxextra{pypath.core.network.Network method}}

\begin{fulllineitems}
\phantomsection\label{\detokenize{reference:pypath.core.network.Network.interactions_undirected_by_data_model}}\pysiglinewithargsret{\sphinxbfcode{\sphinxupquote{interactions\_undirected\_by\_data\_model}}}{}{}
Built-in immutable sequence.

If no argument is given, the constructor returns an empty tuple.
If iterable is specified the tuple is initialized from iterable’s items.

If the argument is a tuple, the return value is the same object.

\end{fulllineitems}

\index{interactions\_undirected\_by\_interaction\_type() (pypath.core.network.Network method)@\spxentry{interactions\_undirected\_by\_interaction\_type()}\spxextra{pypath.core.network.Network method}}

\begin{fulllineitems}
\phantomsection\label{\detokenize{reference:pypath.core.network.Network.interactions_undirected_by_interaction_type}}\pysiglinewithargsret{\sphinxbfcode{\sphinxupquote{interactions\_undirected\_by\_interaction\_type}}}{}{}
Built-in immutable sequence.

If no argument is given, the constructor returns an empty tuple.
If iterable is specified the tuple is initialized from iterable’s items.

If the argument is a tuple, the return value is the same object.

\end{fulllineitems}

\index{interactions\_undirected\_by\_interaction\_type\_and\_data\_model() (pypath.core.network.Network method)@\spxentry{interactions\_undirected\_by\_interaction\_type\_and\_data\_model()}\spxextra{pypath.core.network.Network method}}

\begin{fulllineitems}
\phantomsection\label{\detokenize{reference:pypath.core.network.Network.interactions_undirected_by_interaction_type_and_data_model}}\pysiglinewithargsret{\sphinxbfcode{\sphinxupquote{interactions\_undirected\_by\_interaction\_type\_and\_data\_model}}}{}{}
Built-in immutable sequence.

If no argument is given, the constructor returns an empty tuple.
If iterable is specified the tuple is initialized from iterable’s items.

If the argument is a tuple, the return value is the same object.

\end{fulllineitems}

\index{interactions\_undirected\_by\_interaction\_type\_and\_data\_model\_and\_resource() (pypath.core.network.Network method)@\spxentry{interactions\_undirected\_by\_interaction\_type\_and\_data\_model\_and\_resource()}\spxextra{pypath.core.network.Network method}}

\begin{fulllineitems}
\phantomsection\label{\detokenize{reference:pypath.core.network.Network.interactions_undirected_by_interaction_type_and_data_model_and_resource}}\pysiglinewithargsret{\sphinxbfcode{\sphinxupquote{interactions\_undirected\_by\_interaction\_type\_and\_data\_model\_and\_resource}}}{}{}
Built-in immutable sequence.

If no argument is given, the constructor returns an empty tuple.
If iterable is specified the tuple is initialized from iterable’s items.

If the argument is a tuple, the return value is the same object.

\end{fulllineitems}

\index{interactions\_undirected\_by\_reference() (pypath.core.network.Network method)@\spxentry{interactions\_undirected\_by\_reference()}\spxextra{pypath.core.network.Network method}}

\begin{fulllineitems}
\phantomsection\label{\detokenize{reference:pypath.core.network.Network.interactions_undirected_by_reference}}\pysiglinewithargsret{\sphinxbfcode{\sphinxupquote{interactions\_undirected\_by\_reference}}}{}{}
Built-in immutable sequence.

If no argument is given, the constructor returns an empty tuple.
If iterable is specified the tuple is initialized from iterable’s items.

If the argument is a tuple, the return value is the same object.

\end{fulllineitems}

\index{interactions\_undirected\_by\_resource() (pypath.core.network.Network method)@\spxentry{interactions\_undirected\_by\_resource()}\spxextra{pypath.core.network.Network method}}

\begin{fulllineitems}
\phantomsection\label{\detokenize{reference:pypath.core.network.Network.interactions_undirected_by_resource}}\pysiglinewithargsret{\sphinxbfcode{\sphinxupquote{interactions\_undirected\_by\_resource}}}{}{}
Built-in immutable sequence.

If no argument is given, the constructor returns an empty tuple.
If iterable is specified the tuple is initialized from iterable’s items.

If the argument is a tuple, the return value is the same object.

\end{fulllineitems}

\index{labels\_by\_data\_model() (pypath.core.network.Network method)@\spxentry{labels\_by\_data\_model()}\spxextra{pypath.core.network.Network method}}

\begin{fulllineitems}
\phantomsection\label{\detokenize{reference:pypath.core.network.Network.labels_by_data_model}}\pysiglinewithargsret{\sphinxbfcode{\sphinxupquote{labels\_by\_data\_model}}}{}{}
Built-in immutable sequence.

If no argument is given, the constructor returns an empty tuple.
If iterable is specified the tuple is initialized from iterable’s items.

If the argument is a tuple, the return value is the same object.

\end{fulllineitems}

\index{labels\_by\_interaction\_type() (pypath.core.network.Network method)@\spxentry{labels\_by\_interaction\_type()}\spxextra{pypath.core.network.Network method}}

\begin{fulllineitems}
\phantomsection\label{\detokenize{reference:pypath.core.network.Network.labels_by_interaction_type}}\pysiglinewithargsret{\sphinxbfcode{\sphinxupquote{labels\_by\_interaction\_type}}}{}{}
Built-in immutable sequence.

If no argument is given, the constructor returns an empty tuple.
If iterable is specified the tuple is initialized from iterable’s items.

If the argument is a tuple, the return value is the same object.

\end{fulllineitems}

\index{labels\_by\_interaction\_type\_and\_data\_model() (pypath.core.network.Network method)@\spxentry{labels\_by\_interaction\_type\_and\_data\_model()}\spxextra{pypath.core.network.Network method}}

\begin{fulllineitems}
\phantomsection\label{\detokenize{reference:pypath.core.network.Network.labels_by_interaction_type_and_data_model}}\pysiglinewithargsret{\sphinxbfcode{\sphinxupquote{labels\_by\_interaction\_type\_and\_data\_model}}}{}{}
Built-in immutable sequence.

If no argument is given, the constructor returns an empty tuple.
If iterable is specified the tuple is initialized from iterable’s items.

If the argument is a tuple, the return value is the same object.

\end{fulllineitems}

\index{labels\_by\_interaction\_type\_and\_data\_model\_and\_resource() (pypath.core.network.Network method)@\spxentry{labels\_by\_interaction\_type\_and\_data\_model\_and\_resource()}\spxextra{pypath.core.network.Network method}}

\begin{fulllineitems}
\phantomsection\label{\detokenize{reference:pypath.core.network.Network.labels_by_interaction_type_and_data_model_and_resource}}\pysiglinewithargsret{\sphinxbfcode{\sphinxupquote{labels\_by\_interaction\_type\_and\_data\_model\_and\_resource}}}{}{}
Built-in immutable sequence.

If no argument is given, the constructor returns an empty tuple.
If iterable is specified the tuple is initialized from iterable’s items.

If the argument is a tuple, the return value is the same object.

\end{fulllineitems}

\index{labels\_by\_reference() (pypath.core.network.Network method)@\spxentry{labels\_by\_reference()}\spxextra{pypath.core.network.Network method}}

\begin{fulllineitems}
\phantomsection\label{\detokenize{reference:pypath.core.network.Network.labels_by_reference}}\pysiglinewithargsret{\sphinxbfcode{\sphinxupquote{labels\_by\_reference}}}{}{}
Built-in immutable sequence.

If no argument is given, the constructor returns an empty tuple.
If iterable is specified the tuple is initialized from iterable’s items.

If the argument is a tuple, the return value is the same object.

\end{fulllineitems}

\index{labels\_by\_resource() (pypath.core.network.Network method)@\spxentry{labels\_by\_resource()}\spxextra{pypath.core.network.Network method}}

\begin{fulllineitems}
\phantomsection\label{\detokenize{reference:pypath.core.network.Network.labels_by_resource}}\pysiglinewithargsret{\sphinxbfcode{\sphinxupquote{labels\_by\_resource}}}{}{}
Built-in immutable sequence.

If no argument is given, the constructor returns an empty tuple.
If iterable is specified the tuple is initialized from iterable’s items.

If the argument is a tuple, the return value is the same object.

\end{fulllineitems}

\index{lncrna\_identifiers\_by\_data\_model() (pypath.core.network.Network method)@\spxentry{lncrna\_identifiers\_by\_data\_model()}\spxextra{pypath.core.network.Network method}}

\begin{fulllineitems}
\phantomsection\label{\detokenize{reference:pypath.core.network.Network.lncrna_identifiers_by_data_model}}\pysiglinewithargsret{\sphinxbfcode{\sphinxupquote{lncrna\_identifiers\_by\_data\_model}}}{}{}
Built-in immutable sequence.

If no argument is given, the constructor returns an empty tuple.
If iterable is specified the tuple is initialized from iterable’s items.

If the argument is a tuple, the return value is the same object.

\end{fulllineitems}

\index{lncrna\_identifiers\_by\_interaction\_type() (pypath.core.network.Network method)@\spxentry{lncrna\_identifiers\_by\_interaction\_type()}\spxextra{pypath.core.network.Network method}}

\begin{fulllineitems}
\phantomsection\label{\detokenize{reference:pypath.core.network.Network.lncrna_identifiers_by_interaction_type}}\pysiglinewithargsret{\sphinxbfcode{\sphinxupquote{lncrna\_identifiers\_by\_interaction\_type}}}{}{}
Built-in immutable sequence.

If no argument is given, the constructor returns an empty tuple.
If iterable is specified the tuple is initialized from iterable’s items.

If the argument is a tuple, the return value is the same object.

\end{fulllineitems}

\index{lncrna\_identifiers\_by\_interaction\_type\_and\_data\_model() (pypath.core.network.Network method)@\spxentry{lncrna\_identifiers\_by\_interaction\_type\_and\_data\_model()}\spxextra{pypath.core.network.Network method}}

\begin{fulllineitems}
\phantomsection\label{\detokenize{reference:pypath.core.network.Network.lncrna_identifiers_by_interaction_type_and_data_model}}\pysiglinewithargsret{\sphinxbfcode{\sphinxupquote{lncrna\_identifiers\_by\_interaction\_type\_and\_data\_model}}}{}{}
Built-in immutable sequence.

If no argument is given, the constructor returns an empty tuple.
If iterable is specified the tuple is initialized from iterable’s items.

If the argument is a tuple, the return value is the same object.

\end{fulllineitems}

\index{lncrna\_identifiers\_by\_interaction\_type\_and\_data\_model\_and\_resource() (pypath.core.network.Network method)@\spxentry{lncrna\_identifiers\_by\_interaction\_type\_and\_data\_model\_and\_resource()}\spxextra{pypath.core.network.Network method}}

\begin{fulllineitems}
\phantomsection\label{\detokenize{reference:pypath.core.network.Network.lncrna_identifiers_by_interaction_type_and_data_model_and_resource}}\pysiglinewithargsret{\sphinxbfcode{\sphinxupquote{lncrna\_identifiers\_by\_interaction\_type\_and\_data\_model\_and\_resource}}}{}{}
Built-in immutable sequence.

If no argument is given, the constructor returns an empty tuple.
If iterable is specified the tuple is initialized from iterable’s items.

If the argument is a tuple, the return value is the same object.

\end{fulllineitems}

\index{lncrna\_identifiers\_by\_reference() (pypath.core.network.Network method)@\spxentry{lncrna\_identifiers\_by\_reference()}\spxextra{pypath.core.network.Network method}}

\begin{fulllineitems}
\phantomsection\label{\detokenize{reference:pypath.core.network.Network.lncrna_identifiers_by_reference}}\pysiglinewithargsret{\sphinxbfcode{\sphinxupquote{lncrna\_identifiers\_by\_reference}}}{}{}
Built-in immutable sequence.

If no argument is given, the constructor returns an empty tuple.
If iterable is specified the tuple is initialized from iterable’s items.

If the argument is a tuple, the return value is the same object.

\end{fulllineitems}

\index{lncrna\_identifiers\_by\_resource() (pypath.core.network.Network method)@\spxentry{lncrna\_identifiers\_by\_resource()}\spxextra{pypath.core.network.Network method}}

\begin{fulllineitems}
\phantomsection\label{\detokenize{reference:pypath.core.network.Network.lncrna_identifiers_by_resource}}\pysiglinewithargsret{\sphinxbfcode{\sphinxupquote{lncrna\_identifiers\_by\_resource}}}{}{}
Built-in immutable sequence.

If no argument is given, the constructor returns an empty tuple.
If iterable is specified the tuple is initialized from iterable’s items.

If the argument is a tuple, the return value is the same object.

\end{fulllineitems}

\index{lncrna\_labels\_by\_data\_model() (pypath.core.network.Network method)@\spxentry{lncrna\_labels\_by\_data\_model()}\spxextra{pypath.core.network.Network method}}

\begin{fulllineitems}
\phantomsection\label{\detokenize{reference:pypath.core.network.Network.lncrna_labels_by_data_model}}\pysiglinewithargsret{\sphinxbfcode{\sphinxupquote{lncrna\_labels\_by\_data\_model}}}{}{}
Built-in immutable sequence.

If no argument is given, the constructor returns an empty tuple.
If iterable is specified the tuple is initialized from iterable’s items.

If the argument is a tuple, the return value is the same object.

\end{fulllineitems}

\index{lncrna\_labels\_by\_interaction\_type() (pypath.core.network.Network method)@\spxentry{lncrna\_labels\_by\_interaction\_type()}\spxextra{pypath.core.network.Network method}}

\begin{fulllineitems}
\phantomsection\label{\detokenize{reference:pypath.core.network.Network.lncrna_labels_by_interaction_type}}\pysiglinewithargsret{\sphinxbfcode{\sphinxupquote{lncrna\_labels\_by\_interaction\_type}}}{}{}
Built-in immutable sequence.

If no argument is given, the constructor returns an empty tuple.
If iterable is specified the tuple is initialized from iterable’s items.

If the argument is a tuple, the return value is the same object.

\end{fulllineitems}

\index{lncrna\_labels\_by\_interaction\_type\_and\_data\_model() (pypath.core.network.Network method)@\spxentry{lncrna\_labels\_by\_interaction\_type\_and\_data\_model()}\spxextra{pypath.core.network.Network method}}

\begin{fulllineitems}
\phantomsection\label{\detokenize{reference:pypath.core.network.Network.lncrna_labels_by_interaction_type_and_data_model}}\pysiglinewithargsret{\sphinxbfcode{\sphinxupquote{lncrna\_labels\_by\_interaction\_type\_and\_data\_model}}}{}{}
Built-in immutable sequence.

If no argument is given, the constructor returns an empty tuple.
If iterable is specified the tuple is initialized from iterable’s items.

If the argument is a tuple, the return value is the same object.

\end{fulllineitems}

\index{lncrna\_labels\_by\_interaction\_type\_and\_data\_model\_and\_resource() (pypath.core.network.Network method)@\spxentry{lncrna\_labels\_by\_interaction\_type\_and\_data\_model\_and\_resource()}\spxextra{pypath.core.network.Network method}}

\begin{fulllineitems}
\phantomsection\label{\detokenize{reference:pypath.core.network.Network.lncrna_labels_by_interaction_type_and_data_model_and_resource}}\pysiglinewithargsret{\sphinxbfcode{\sphinxupquote{lncrna\_labels\_by\_interaction\_type\_and\_data\_model\_and\_resource}}}{}{}
Built-in immutable sequence.

If no argument is given, the constructor returns an empty tuple.
If iterable is specified the tuple is initialized from iterable’s items.

If the argument is a tuple, the return value is the same object.

\end{fulllineitems}

\index{lncrna\_labels\_by\_reference() (pypath.core.network.Network method)@\spxentry{lncrna\_labels\_by\_reference()}\spxextra{pypath.core.network.Network method}}

\begin{fulllineitems}
\phantomsection\label{\detokenize{reference:pypath.core.network.Network.lncrna_labels_by_reference}}\pysiglinewithargsret{\sphinxbfcode{\sphinxupquote{lncrna\_labels\_by\_reference}}}{}{}
Built-in immutable sequence.

If no argument is given, the constructor returns an empty tuple.
If iterable is specified the tuple is initialized from iterable’s items.

If the argument is a tuple, the return value is the same object.

\end{fulllineitems}

\index{lncrna\_labels\_by\_resource() (pypath.core.network.Network method)@\spxentry{lncrna\_labels\_by\_resource()}\spxextra{pypath.core.network.Network method}}

\begin{fulllineitems}
\phantomsection\label{\detokenize{reference:pypath.core.network.Network.lncrna_labels_by_resource}}\pysiglinewithargsret{\sphinxbfcode{\sphinxupquote{lncrna\_labels\_by\_resource}}}{}{}
Built-in immutable sequence.

If no argument is given, the constructor returns an empty tuple.
If iterable is specified the tuple is initialized from iterable’s items.

If the argument is a tuple, the return value is the same object.

\end{fulllineitems}

\index{lncrnas\_by\_data\_model() (pypath.core.network.Network method)@\spxentry{lncrnas\_by\_data\_model()}\spxextra{pypath.core.network.Network method}}

\begin{fulllineitems}
\phantomsection\label{\detokenize{reference:pypath.core.network.Network.lncrnas_by_data_model}}\pysiglinewithargsret{\sphinxbfcode{\sphinxupquote{lncrnas\_by\_data\_model}}}{}{}
Built-in immutable sequence.

If no argument is given, the constructor returns an empty tuple.
If iterable is specified the tuple is initialized from iterable’s items.

If the argument is a tuple, the return value is the same object.

\end{fulllineitems}

\index{lncrnas\_by\_interaction\_type() (pypath.core.network.Network method)@\spxentry{lncrnas\_by\_interaction\_type()}\spxextra{pypath.core.network.Network method}}

\begin{fulllineitems}
\phantomsection\label{\detokenize{reference:pypath.core.network.Network.lncrnas_by_interaction_type}}\pysiglinewithargsret{\sphinxbfcode{\sphinxupquote{lncrnas\_by\_interaction\_type}}}{}{}
Built-in immutable sequence.

If no argument is given, the constructor returns an empty tuple.
If iterable is specified the tuple is initialized from iterable’s items.

If the argument is a tuple, the return value is the same object.

\end{fulllineitems}

\index{lncrnas\_by\_interaction\_type\_and\_data\_model() (pypath.core.network.Network method)@\spxentry{lncrnas\_by\_interaction\_type\_and\_data\_model()}\spxextra{pypath.core.network.Network method}}

\begin{fulllineitems}
\phantomsection\label{\detokenize{reference:pypath.core.network.Network.lncrnas_by_interaction_type_and_data_model}}\pysiglinewithargsret{\sphinxbfcode{\sphinxupquote{lncrnas\_by\_interaction\_type\_and\_data\_model}}}{}{}
Built-in immutable sequence.

If no argument is given, the constructor returns an empty tuple.
If iterable is specified the tuple is initialized from iterable’s items.

If the argument is a tuple, the return value is the same object.

\end{fulllineitems}

\index{lncrnas\_by\_interaction\_type\_and\_data\_model\_and\_resource() (pypath.core.network.Network method)@\spxentry{lncrnas\_by\_interaction\_type\_and\_data\_model\_and\_resource()}\spxextra{pypath.core.network.Network method}}

\begin{fulllineitems}
\phantomsection\label{\detokenize{reference:pypath.core.network.Network.lncrnas_by_interaction_type_and_data_model_and_resource}}\pysiglinewithargsret{\sphinxbfcode{\sphinxupquote{lncrnas\_by\_interaction\_type\_and\_data\_model\_and\_resource}}}{}{}
Built-in immutable sequence.

If no argument is given, the constructor returns an empty tuple.
If iterable is specified the tuple is initialized from iterable’s items.

If the argument is a tuple, the return value is the same object.

\end{fulllineitems}

\index{lncrnas\_by\_reference() (pypath.core.network.Network method)@\spxentry{lncrnas\_by\_reference()}\spxextra{pypath.core.network.Network method}}

\begin{fulllineitems}
\phantomsection\label{\detokenize{reference:pypath.core.network.Network.lncrnas_by_reference}}\pysiglinewithargsret{\sphinxbfcode{\sphinxupquote{lncrnas\_by\_reference}}}{}{}
Built-in immutable sequence.

If no argument is given, the constructor returns an empty tuple.
If iterable is specified the tuple is initialized from iterable’s items.

If the argument is a tuple, the return value is the same object.

\end{fulllineitems}

\index{lncrnas\_by\_resource() (pypath.core.network.Network method)@\spxentry{lncrnas\_by\_resource()}\spxextra{pypath.core.network.Network method}}

\begin{fulllineitems}
\phantomsection\label{\detokenize{reference:pypath.core.network.Network.lncrnas_by_resource}}\pysiglinewithargsret{\sphinxbfcode{\sphinxupquote{lncrnas\_by\_resource}}}{}{}
Built-in immutable sequence.

If no argument is given, the constructor returns an empty tuple.
If iterable is specified the tuple is initialized from iterable’s items.

If the argument is a tuple, the return value is the same object.

\end{fulllineitems}

\index{load() (pypath.core.network.Network method)@\spxentry{load()}\spxextra{pypath.core.network.Network method}}

\begin{fulllineitems}
\phantomsection\label{\detokenize{reference:pypath.core.network.Network.load}}\pysiglinewithargsret{\sphinxbfcode{\sphinxupquote{load}}}{\emph{resources=None}, \emph{make\_df=False}, \emph{exclude=None}, \emph{reread=False}, \emph{redownload=False}, \emph{keep\_raw=False}, \emph{top\_call=True}, \emph{cache\_files=None}, \emph{only\_directions=False}, \emph{pickle\_file=None}}{}
Loads data from a network resource or a collection of resources.
\begin{quote}\begin{description}
\item[{Parameters}] \leavevmode\begin{itemize}
\item {} 
\sphinxstyleliteralstrong{\sphinxupquote{resources}} (\sphinxstyleliteralemphasis{\sphinxupquote{str}}\sphinxstyleliteralemphasis{\sphinxupquote{,}}\sphinxstyleliteralemphasis{\sphinxupquote{dict}}\sphinxstyleliteralemphasis{\sphinxupquote{,}}\sphinxstyleliteralemphasis{\sphinxupquote{list}}\sphinxstyleliteralemphasis{\sphinxupquote{,}}\sphinxstyleliteralemphasis{\sphinxupquote{resource.NetworkResource}}) \textendash{} An object defining one or more network resources. If \sphinxstyleemphasis{str} it
will be looked up among the collections in the
\sphinxcode{\sphinxupquote{pypath.resources.network}} module (e.g. \sphinxcode{\sphinxupquote{'pathway'}} will load
all resources in the \sphinxtitleref{pathway} collection). If \sphinxstyleemphasis{dict} or \sphinxstyleemphasis{list}
it will be processed recursively i.e. the \sphinxcode{\sphinxupquote{load}} method will be
called for each element. If it is a
\sphinxcode{\sphinxupquote{pypath.resource.NetworkResource}} object it will be processed
and added to the network.

\item {} 
\sphinxstyleliteralstrong{\sphinxupquote{make\_df}} (\sphinxstyleliteralemphasis{\sphinxupquote{bool}}) \textendash{} Whether to create a \sphinxcode{\sphinxupquote{pandas.DataFrame}} after loading all
resources.

\item {} 
\sphinxstyleliteralstrong{\sphinxupquote{exclude}} (\sphinxstyleliteralemphasis{\sphinxupquote{NoneType}}\sphinxstyleliteralemphasis{\sphinxupquote{,}}\sphinxstyleliteralemphasis{\sphinxupquote{set}}) \textendash{} A \sphinxstyleemphasis{set} of resource names to be ignored. It is useful if you want
to load a collection with the exception of a few resources.

\end{itemize}

\end{description}\end{quote}

\end{fulllineitems}

\index{load\_from\_pickle() (pypath.core.network.Network method)@\spxentry{load\_from\_pickle()}\spxextra{pypath.core.network.Network method}}

\begin{fulllineitems}
\phantomsection\label{\detokenize{reference:pypath.core.network.Network.load_from_pickle}}\pysiglinewithargsret{\sphinxbfcode{\sphinxupquote{load\_from\_pickle}}}{\emph{pickle\_file}}{}
Loads the network to a pickle file.
\begin{quote}\begin{description}
\item[{Parameters}] \leavevmode
\sphinxstyleliteralstrong{\sphinxupquote{pickle\_file}} (\sphinxstyleliteralemphasis{\sphinxupquote{str}}) \textendash{} Path to the pickle file.

\end{description}\end{quote}

\end{fulllineitems}

\index{load\_resource() (pypath.core.network.Network method)@\spxentry{load\_resource()}\spxextra{pypath.core.network.Network method}}

\begin{fulllineitems}
\phantomsection\label{\detokenize{reference:pypath.core.network.Network.load_resource}}\pysiglinewithargsret{\sphinxbfcode{\sphinxupquote{load\_resource}}}{\emph{resource}, \emph{clean=True}, \emph{reread=None}, \emph{redownload=None}, \emph{keep\_raw=False}, \emph{only\_directions=False}, \emph{**kwargs}}{}
Loads the data from a single resource and attaches it to the
network
\begin{quote}\begin{description}
\item[{Parameters}] \leavevmode\begin{itemize}
\item {} 
\sphinxstyleliteralstrong{\sphinxupquote{resource}} (\sphinxstyleliteralemphasis{\sphinxupquote{pypath.input\_formats.NetworkInput}}) \textendash{} \sphinxcode{\sphinxupquote{pypath.input\_formats.NetworkInput}} instance
containing the detailed definition of the input format to
the downloaded file.

\item {} 
\sphinxstyleliteralstrong{\sphinxupquote{clean}} (\sphinxstyleliteralemphasis{\sphinxupquote{bool}}) \textendash{} Legacy parameter, has no effect at the moment.
Optional, \sphinxcode{\sphinxupquote{True}} by default. Whether to clean the graph
after importing the data or not. See
\sphinxcode{\sphinxupquote{pypath.main.PyPath.clean\_graph()}} for more
information.

\item {} 
\sphinxstyleliteralstrong{\sphinxupquote{cache\_files}} (\sphinxstyleliteralemphasis{\sphinxupquote{dict}}) \textendash{} Legacy parameter, has no effect at the moment.
Optional, \sphinxcode{\sphinxupquote{\{\}}} by default. Contains the resource name(s)
{[}str{]} (keys) and the corresponding cached file name {[}str{]}.
If provided (and file exists) bypasses the download of the
data for that resource and uses the cache file instead.

\item {} 
\sphinxstyleliteralstrong{\sphinxupquote{reread}} (\sphinxstyleliteralemphasis{\sphinxupquote{bool}}) \textendash{} Optional, \sphinxcode{\sphinxupquote{False}} by default. Specifies whether to reread
the data files from the cache or omit them (similar to
\sphinxstyleemphasis{redownload}).

\item {} 
\sphinxstyleliteralstrong{\sphinxupquote{redownload}} (\sphinxstyleliteralemphasis{\sphinxupquote{bool}}) \textendash{} Optional, \sphinxcode{\sphinxupquote{False}} by default. Specifies whether to
re-download the data and ignore the cache.

\item {} 
\sphinxstyleliteralstrong{\sphinxupquote{only\_directions}} (\sphinxstyleliteralemphasis{\sphinxupquote{bool}}) \textendash{} If \sphinxcode{\sphinxupquote{True}}, no new interactions will be created but direction
and effect sign evidences will be added to existing interactions.

\end{itemize}

\end{description}\end{quote}

\end{fulllineitems}

\index{load\_resources() (pypath.core.network.Network method)@\spxentry{load\_resources()}\spxextra{pypath.core.network.Network method}}

\begin{fulllineitems}
\phantomsection\label{\detokenize{reference:pypath.core.network.Network.load_resources}}\pysiglinewithargsret{\sphinxbfcode{\sphinxupquote{load\_resources}}}{\emph{resources=None}, \emph{make\_df=False}, \emph{exclude=None}, \emph{reread=False}, \emph{redownload=False}, \emph{keep\_raw=False}, \emph{top\_call=True}, \emph{cache\_files=None}, \emph{only\_directions=False}, \emph{pickle\_file=None}}{}
Loads data from a network resource or a collection of resources.
\begin{quote}\begin{description}
\item[{Parameters}] \leavevmode\begin{itemize}
\item {} 
\sphinxstyleliteralstrong{\sphinxupquote{resources}} (\sphinxstyleliteralemphasis{\sphinxupquote{str}}\sphinxstyleliteralemphasis{\sphinxupquote{,}}\sphinxstyleliteralemphasis{\sphinxupquote{dict}}\sphinxstyleliteralemphasis{\sphinxupquote{,}}\sphinxstyleliteralemphasis{\sphinxupquote{list}}\sphinxstyleliteralemphasis{\sphinxupquote{,}}\sphinxstyleliteralemphasis{\sphinxupquote{resource.NetworkResource}}) \textendash{} An object defining one or more network resources. If \sphinxstyleemphasis{str} it
will be looked up among the collections in the
\sphinxcode{\sphinxupquote{pypath.resources.network}} module (e.g. \sphinxcode{\sphinxupquote{'pathway'}} will load
all resources in the \sphinxtitleref{pathway} collection). If \sphinxstyleemphasis{dict} or \sphinxstyleemphasis{list}
it will be processed recursively i.e. the \sphinxcode{\sphinxupquote{load}} method will be
called for each element. If it is a
\sphinxcode{\sphinxupquote{pypath.resource.NetworkResource}} object it will be processed
and added to the network.

\item {} 
\sphinxstyleliteralstrong{\sphinxupquote{make\_df}} (\sphinxstyleliteralemphasis{\sphinxupquote{bool}}) \textendash{} Whether to create a \sphinxcode{\sphinxupquote{pandas.DataFrame}} after loading all
resources.

\item {} 
\sphinxstyleliteralstrong{\sphinxupquote{exclude}} (\sphinxstyleliteralemphasis{\sphinxupquote{NoneType}}\sphinxstyleliteralemphasis{\sphinxupquote{,}}\sphinxstyleliteralemphasis{\sphinxupquote{set}}) \textendash{} A \sphinxstyleemphasis{set} of resource names to be ignored. It is useful if you want
to load a collection with the exception of a few resources.

\end{itemize}

\end{description}\end{quote}

\end{fulllineitems}

\index{make\_df() (pypath.core.network.Network method)@\spxentry{make\_df()}\spxextra{pypath.core.network.Network method}}

\begin{fulllineitems}
\phantomsection\label{\detokenize{reference:pypath.core.network.Network.make_df}}\pysiglinewithargsret{\sphinxbfcode{\sphinxupquote{make\_df}}}{\emph{records=None}, \emph{by\_source=None}, \emph{with\_references=None}, \emph{columns=None}, \emph{dtype=None}}{}
Creates a \sphinxcode{\sphinxupquote{pandas.DataFrame}} from the interactions.

\end{fulllineitems}

\index{mirna\_identifiers\_by\_data\_model() (pypath.core.network.Network method)@\spxentry{mirna\_identifiers\_by\_data\_model()}\spxextra{pypath.core.network.Network method}}

\begin{fulllineitems}
\phantomsection\label{\detokenize{reference:pypath.core.network.Network.mirna_identifiers_by_data_model}}\pysiglinewithargsret{\sphinxbfcode{\sphinxupquote{mirna\_identifiers\_by\_data\_model}}}{}{}
Built-in immutable sequence.

If no argument is given, the constructor returns an empty tuple.
If iterable is specified the tuple is initialized from iterable’s items.

If the argument is a tuple, the return value is the same object.

\end{fulllineitems}

\index{mirna\_identifiers\_by\_interaction\_type() (pypath.core.network.Network method)@\spxentry{mirna\_identifiers\_by\_interaction\_type()}\spxextra{pypath.core.network.Network method}}

\begin{fulllineitems}
\phantomsection\label{\detokenize{reference:pypath.core.network.Network.mirna_identifiers_by_interaction_type}}\pysiglinewithargsret{\sphinxbfcode{\sphinxupquote{mirna\_identifiers\_by\_interaction\_type}}}{}{}
Built-in immutable sequence.

If no argument is given, the constructor returns an empty tuple.
If iterable is specified the tuple is initialized from iterable’s items.

If the argument is a tuple, the return value is the same object.

\end{fulllineitems}

\index{mirna\_identifiers\_by\_interaction\_type\_and\_data\_model() (pypath.core.network.Network method)@\spxentry{mirna\_identifiers\_by\_interaction\_type\_and\_data\_model()}\spxextra{pypath.core.network.Network method}}

\begin{fulllineitems}
\phantomsection\label{\detokenize{reference:pypath.core.network.Network.mirna_identifiers_by_interaction_type_and_data_model}}\pysiglinewithargsret{\sphinxbfcode{\sphinxupquote{mirna\_identifiers\_by\_interaction\_type\_and\_data\_model}}}{}{}
Built-in immutable sequence.

If no argument is given, the constructor returns an empty tuple.
If iterable is specified the tuple is initialized from iterable’s items.

If the argument is a tuple, the return value is the same object.

\end{fulllineitems}

\index{mirna\_identifiers\_by\_interaction\_type\_and\_data\_model\_and\_resource() (pypath.core.network.Network method)@\spxentry{mirna\_identifiers\_by\_interaction\_type\_and\_data\_model\_and\_resource()}\spxextra{pypath.core.network.Network method}}

\begin{fulllineitems}
\phantomsection\label{\detokenize{reference:pypath.core.network.Network.mirna_identifiers_by_interaction_type_and_data_model_and_resource}}\pysiglinewithargsret{\sphinxbfcode{\sphinxupquote{mirna\_identifiers\_by\_interaction\_type\_and\_data\_model\_and\_resource}}}{}{}
Built-in immutable sequence.

If no argument is given, the constructor returns an empty tuple.
If iterable is specified the tuple is initialized from iterable’s items.

If the argument is a tuple, the return value is the same object.

\end{fulllineitems}

\index{mirna\_identifiers\_by\_reference() (pypath.core.network.Network method)@\spxentry{mirna\_identifiers\_by\_reference()}\spxextra{pypath.core.network.Network method}}

\begin{fulllineitems}
\phantomsection\label{\detokenize{reference:pypath.core.network.Network.mirna_identifiers_by_reference}}\pysiglinewithargsret{\sphinxbfcode{\sphinxupquote{mirna\_identifiers\_by\_reference}}}{}{}
Built-in immutable sequence.

If no argument is given, the constructor returns an empty tuple.
If iterable is specified the tuple is initialized from iterable’s items.

If the argument is a tuple, the return value is the same object.

\end{fulllineitems}

\index{mirna\_identifiers\_by\_resource() (pypath.core.network.Network method)@\spxentry{mirna\_identifiers\_by\_resource()}\spxextra{pypath.core.network.Network method}}

\begin{fulllineitems}
\phantomsection\label{\detokenize{reference:pypath.core.network.Network.mirna_identifiers_by_resource}}\pysiglinewithargsret{\sphinxbfcode{\sphinxupquote{mirna\_identifiers\_by\_resource}}}{}{}
Built-in immutable sequence.

If no argument is given, the constructor returns an empty tuple.
If iterable is specified the tuple is initialized from iterable’s items.

If the argument is a tuple, the return value is the same object.

\end{fulllineitems}

\index{mirna\_labels\_by\_data\_model() (pypath.core.network.Network method)@\spxentry{mirna\_labels\_by\_data\_model()}\spxextra{pypath.core.network.Network method}}

\begin{fulllineitems}
\phantomsection\label{\detokenize{reference:pypath.core.network.Network.mirna_labels_by_data_model}}\pysiglinewithargsret{\sphinxbfcode{\sphinxupquote{mirna\_labels\_by\_data\_model}}}{}{}
Built-in immutable sequence.

If no argument is given, the constructor returns an empty tuple.
If iterable is specified the tuple is initialized from iterable’s items.

If the argument is a tuple, the return value is the same object.

\end{fulllineitems}

\index{mirna\_labels\_by\_interaction\_type() (pypath.core.network.Network method)@\spxentry{mirna\_labels\_by\_interaction\_type()}\spxextra{pypath.core.network.Network method}}

\begin{fulllineitems}
\phantomsection\label{\detokenize{reference:pypath.core.network.Network.mirna_labels_by_interaction_type}}\pysiglinewithargsret{\sphinxbfcode{\sphinxupquote{mirna\_labels\_by\_interaction\_type}}}{}{}
Built-in immutable sequence.

If no argument is given, the constructor returns an empty tuple.
If iterable is specified the tuple is initialized from iterable’s items.

If the argument is a tuple, the return value is the same object.

\end{fulllineitems}

\index{mirna\_labels\_by\_interaction\_type\_and\_data\_model() (pypath.core.network.Network method)@\spxentry{mirna\_labels\_by\_interaction\_type\_and\_data\_model()}\spxextra{pypath.core.network.Network method}}

\begin{fulllineitems}
\phantomsection\label{\detokenize{reference:pypath.core.network.Network.mirna_labels_by_interaction_type_and_data_model}}\pysiglinewithargsret{\sphinxbfcode{\sphinxupquote{mirna\_labels\_by\_interaction\_type\_and\_data\_model}}}{}{}
Built-in immutable sequence.

If no argument is given, the constructor returns an empty tuple.
If iterable is specified the tuple is initialized from iterable’s items.

If the argument is a tuple, the return value is the same object.

\end{fulllineitems}

\index{mirna\_labels\_by\_interaction\_type\_and\_data\_model\_and\_resource() (pypath.core.network.Network method)@\spxentry{mirna\_labels\_by\_interaction\_type\_and\_data\_model\_and\_resource()}\spxextra{pypath.core.network.Network method}}

\begin{fulllineitems}
\phantomsection\label{\detokenize{reference:pypath.core.network.Network.mirna_labels_by_interaction_type_and_data_model_and_resource}}\pysiglinewithargsret{\sphinxbfcode{\sphinxupquote{mirna\_labels\_by\_interaction\_type\_and\_data\_model\_and\_resource}}}{}{}
Built-in immutable sequence.

If no argument is given, the constructor returns an empty tuple.
If iterable is specified the tuple is initialized from iterable’s items.

If the argument is a tuple, the return value is the same object.

\end{fulllineitems}

\index{mirna\_labels\_by\_reference() (pypath.core.network.Network method)@\spxentry{mirna\_labels\_by\_reference()}\spxextra{pypath.core.network.Network method}}

\begin{fulllineitems}
\phantomsection\label{\detokenize{reference:pypath.core.network.Network.mirna_labels_by_reference}}\pysiglinewithargsret{\sphinxbfcode{\sphinxupquote{mirna\_labels\_by\_reference}}}{}{}
Built-in immutable sequence.

If no argument is given, the constructor returns an empty tuple.
If iterable is specified the tuple is initialized from iterable’s items.

If the argument is a tuple, the return value is the same object.

\end{fulllineitems}

\index{mirna\_labels\_by\_resource() (pypath.core.network.Network method)@\spxentry{mirna\_labels\_by\_resource()}\spxextra{pypath.core.network.Network method}}

\begin{fulllineitems}
\phantomsection\label{\detokenize{reference:pypath.core.network.Network.mirna_labels_by_resource}}\pysiglinewithargsret{\sphinxbfcode{\sphinxupquote{mirna\_labels\_by\_resource}}}{}{}
Built-in immutable sequence.

If no argument is given, the constructor returns an empty tuple.
If iterable is specified the tuple is initialized from iterable’s items.

If the argument is a tuple, the return value is the same object.

\end{fulllineitems}

\index{mirna\_target() (pypath.core.network.Network class method)@\spxentry{mirna\_target()}\spxextra{pypath.core.network.Network class method}}

\begin{fulllineitems}
\phantomsection\label{\detokenize{reference:pypath.core.network.Network.mirna_target}}\pysiglinewithargsret{\sphinxbfcode{\sphinxupquote{classmethod }}\sphinxbfcode{\sphinxupquote{mirna\_target}}}{\emph{resources=None}, \emph{make\_df=None}, \emph{reread=False}, \emph{redownload=False}, \emph{exclude=None}, \emph{ncbi\_tax\_id=9606}, \emph{**kwargs}}{}
Initializes a new \sphinxcode{\sphinxupquote{Network}} object with loading a miRNA-mRNA
regulation network from all databases by default.

{\color{red}\bfseries{}**}kwargs: passed to \sphinxcode{\sphinxupquote{Network.\_\_init\_\_}}.

\end{fulllineitems}

\index{mirnas\_by\_data\_model() (pypath.core.network.Network method)@\spxentry{mirnas\_by\_data\_model()}\spxextra{pypath.core.network.Network method}}

\begin{fulllineitems}
\phantomsection\label{\detokenize{reference:pypath.core.network.Network.mirnas_by_data_model}}\pysiglinewithargsret{\sphinxbfcode{\sphinxupquote{mirnas\_by\_data\_model}}}{}{}
Built-in immutable sequence.

If no argument is given, the constructor returns an empty tuple.
If iterable is specified the tuple is initialized from iterable’s items.

If the argument is a tuple, the return value is the same object.

\end{fulllineitems}

\index{mirnas\_by\_interaction\_type() (pypath.core.network.Network method)@\spxentry{mirnas\_by\_interaction\_type()}\spxextra{pypath.core.network.Network method}}

\begin{fulllineitems}
\phantomsection\label{\detokenize{reference:pypath.core.network.Network.mirnas_by_interaction_type}}\pysiglinewithargsret{\sphinxbfcode{\sphinxupquote{mirnas\_by\_interaction\_type}}}{}{}
Built-in immutable sequence.

If no argument is given, the constructor returns an empty tuple.
If iterable is specified the tuple is initialized from iterable’s items.

If the argument is a tuple, the return value is the same object.

\end{fulllineitems}

\index{mirnas\_by\_interaction\_type\_and\_data\_model() (pypath.core.network.Network method)@\spxentry{mirnas\_by\_interaction\_type\_and\_data\_model()}\spxextra{pypath.core.network.Network method}}

\begin{fulllineitems}
\phantomsection\label{\detokenize{reference:pypath.core.network.Network.mirnas_by_interaction_type_and_data_model}}\pysiglinewithargsret{\sphinxbfcode{\sphinxupquote{mirnas\_by\_interaction\_type\_and\_data\_model}}}{}{}
Built-in immutable sequence.

If no argument is given, the constructor returns an empty tuple.
If iterable is specified the tuple is initialized from iterable’s items.

If the argument is a tuple, the return value is the same object.

\end{fulllineitems}

\index{mirnas\_by\_interaction\_type\_and\_data\_model\_and\_resource() (pypath.core.network.Network method)@\spxentry{mirnas\_by\_interaction\_type\_and\_data\_model\_and\_resource()}\spxextra{pypath.core.network.Network method}}

\begin{fulllineitems}
\phantomsection\label{\detokenize{reference:pypath.core.network.Network.mirnas_by_interaction_type_and_data_model_and_resource}}\pysiglinewithargsret{\sphinxbfcode{\sphinxupquote{mirnas\_by\_interaction\_type\_and\_data\_model\_and\_resource}}}{}{}
Built-in immutable sequence.

If no argument is given, the constructor returns an empty tuple.
If iterable is specified the tuple is initialized from iterable’s items.

If the argument is a tuple, the return value is the same object.

\end{fulllineitems}

\index{mirnas\_by\_reference() (pypath.core.network.Network method)@\spxentry{mirnas\_by\_reference()}\spxextra{pypath.core.network.Network method}}

\begin{fulllineitems}
\phantomsection\label{\detokenize{reference:pypath.core.network.Network.mirnas_by_reference}}\pysiglinewithargsret{\sphinxbfcode{\sphinxupquote{mirnas\_by\_reference}}}{}{}
Built-in immutable sequence.

If no argument is given, the constructor returns an empty tuple.
If iterable is specified the tuple is initialized from iterable’s items.

If the argument is a tuple, the return value is the same object.

\end{fulllineitems}

\index{mirnas\_by\_resource() (pypath.core.network.Network method)@\spxentry{mirnas\_by\_resource()}\spxextra{pypath.core.network.Network method}}

\begin{fulllineitems}
\phantomsection\label{\detokenize{reference:pypath.core.network.Network.mirnas_by_resource}}\pysiglinewithargsret{\sphinxbfcode{\sphinxupquote{mirnas\_by\_resource}}}{}{}
Built-in immutable sequence.

If no argument is given, the constructor returns an empty tuple.
If iterable is specified the tuple is initialized from iterable’s items.

If the argument is a tuple, the return value is the same object.

\end{fulllineitems}

\index{numof\_interactions\_per\_reference() (pypath.core.network.Network method)@\spxentry{numof\_interactions\_per\_reference()}\spxextra{pypath.core.network.Network method}}

\begin{fulllineitems}
\phantomsection\label{\detokenize{reference:pypath.core.network.Network.numof_interactions_per_reference}}\pysiglinewithargsret{\sphinxbfcode{\sphinxupquote{numof\_interactions\_per\_reference}}}{}{}
Counts the number of interactions for each literature reference.
Returns a \sphinxcode{\sphinxupquote{collections.Counter}} object (similar to \sphinxcode{\sphinxupquote{dict}}).

\end{fulllineitems}

\index{organisms\_check() (pypath.core.network.Network method)@\spxentry{organisms\_check()}\spxextra{pypath.core.network.Network method}}

\begin{fulllineitems}
\phantomsection\label{\detokenize{reference:pypath.core.network.Network.organisms_check}}\pysiglinewithargsret{\sphinxbfcode{\sphinxupquote{organisms\_check}}}{\emph{organisms=None}, \emph{remove\_mismatches=True}, \emph{remove\_nonspecific=False}}{}
Scans the network for one or more organisms and removes the nodes
and interactions which belong to any other organism.
\begin{quote}\begin{description}
\item[{Parameters}] \leavevmode\begin{itemize}
\item {} 
\sphinxstyleliteralstrong{\sphinxupquote{organisms}} (\sphinxstyleliteralemphasis{\sphinxupquote{int}}\sphinxstyleliteralemphasis{\sphinxupquote{,}}\sphinxstyleliteralemphasis{\sphinxupquote{set}}\sphinxstyleliteralemphasis{\sphinxupquote{,}}\sphinxstyleliteralemphasis{\sphinxupquote{NoneType}}) \textendash{} One or more NCBI Taxonomy IDs. If \sphinxcode{\sphinxupquote{None}} the value in
\sphinxcode{\sphinxupquote{ncbi\_tax\_id}} will be used. If that’s too is \sphinxcode{\sphinxupquote{None}}
then only the entities with discrepancy between their stated
organism and their identifier.

\item {} 
\sphinxstyleliteralstrong{\sphinxupquote{remove\_mismatches}} (\sphinxstyleliteralemphasis{\sphinxupquote{bool}}) \textendash{} Remove the entities where their \sphinxcode{\sphinxupquote{identifier}} can not be found
in the reference list from the database for their \sphinxcode{\sphinxupquote{taxon}}.

\item {} 
\sphinxstyleliteralstrong{\sphinxupquote{remove\_nonspecific}} (\sphinxstyleliteralemphasis{\sphinxupquote{bool}}) \textendash{} Remove the entities with taxonomy ID zero, which is used to
represent the non taxon specific entities such as metabolites
or drug compounds.

\end{itemize}

\end{description}\end{quote}

\end{fulllineitems}

\index{partners() (pypath.core.network.Network method)@\spxentry{partners()}\spxextra{pypath.core.network.Network method}}

\begin{fulllineitems}
\phantomsection\label{\detokenize{reference:pypath.core.network.Network.partners}}\pysiglinewithargsret{\sphinxbfcode{\sphinxupquote{partners}}}{\emph{entity}, \emph{mode='ALL'}, \emph{direction=None}, \emph{effect=None}, \emph{resources=None}, \emph{interaction\_type=None}, \emph{data\_model=None}, \emph{via=None}, \emph{references=None}, \emph{return\_interactions=False}}{}~\begin{quote}\begin{description}
\item[{Parameters}] \leavevmode\begin{itemize}
\item {} 
\sphinxstyleliteralstrong{\sphinxupquote{entity}} (\sphinxstyleliteralemphasis{\sphinxupquote{str}}\sphinxstyleliteralemphasis{\sphinxupquote{,}}{\hyperref[\detokenize{reference:pypath.core.entity.Entity}]{\sphinxcrossref{\sphinxstyleliteralemphasis{\sphinxupquote{Entity}}}}}\sphinxstyleliteralemphasis{\sphinxupquote{,}}\sphinxstyleliteralemphasis{\sphinxupquote{list}}\sphinxstyleliteralemphasis{\sphinxupquote{,}}\sphinxstyleliteralemphasis{\sphinxupquote{set}}\sphinxstyleliteralemphasis{\sphinxupquote{,}}\sphinxstyleliteralemphasis{\sphinxupquote{tuple}}\sphinxstyleliteralemphasis{\sphinxupquote{,}}\sphinxstyleliteralemphasis{\sphinxupquote{EntityList}}) \textendash{} An identifier or label of a molecular entity or an
\sphinxcode{\sphinxupquote{Entity}} object. Alternatively an iterator with the
elements of any of the types valid for a single entity argument,
e.g. a list of gene symbols.

\item {} 
\sphinxstyleliteralstrong{\sphinxupquote{mode}} (\sphinxstyleliteralemphasis{\sphinxupquote{str}}) \textendash{} Mode of counting the interactions: \sphinxtitleref{IN}, \sphinxtitleref{OUT} or \sphinxtitleref{ALL} , whether
to consider incoming, outgoing or all edges, respectively,
respective to the \sphinxtitleref{node defined in {}`entity{}`}.

\end{itemize}

\item[{Returns}] \leavevmode
\sphinxcode{\sphinxupquote{EntityList}} object containing the partners having
interactions to the queried node(s) matching all the criteria.
If \sphinxcode{\sphinxupquote{entity}} doesn’t present in the network the returned
\sphinxcode{\sphinxupquote{EntityList}} will be empty just like if no interaction matches
the criteria.

\end{description}\end{quote}

\end{fulllineitems}

\index{post\_transcriptionally\_activated\_by() (pypath.core.network.Network method)@\spxentry{post\_transcriptionally\_activated\_by()}\spxextra{pypath.core.network.Network method}}

\begin{fulllineitems}
\phantomsection\label{\detokenize{reference:pypath.core.network.Network.post_transcriptionally_activated_by}}\pysiglinewithargsret{\sphinxbfcode{\sphinxupquote{post\_transcriptionally\_activated\_by}}}{}{}~\begin{quote}\begin{description}
\item[{Parameters}] \leavevmode\begin{itemize}
\item {} 
\sphinxstyleliteralstrong{\sphinxupquote{entity}} (\sphinxstyleliteralemphasis{\sphinxupquote{str}}\sphinxstyleliteralemphasis{\sphinxupquote{,}}{\hyperref[\detokenize{reference:pypath.core.entity.Entity}]{\sphinxcrossref{\sphinxstyleliteralemphasis{\sphinxupquote{Entity}}}}}\sphinxstyleliteralemphasis{\sphinxupquote{,}}\sphinxstyleliteralemphasis{\sphinxupquote{list}}\sphinxstyleliteralemphasis{\sphinxupquote{,}}\sphinxstyleliteralemphasis{\sphinxupquote{set}}\sphinxstyleliteralemphasis{\sphinxupquote{,}}\sphinxstyleliteralemphasis{\sphinxupquote{tuple}}\sphinxstyleliteralemphasis{\sphinxupquote{,}}\sphinxstyleliteralemphasis{\sphinxupquote{EntityList}}) \textendash{} An identifier or label of a molecular entity or an
\sphinxcode{\sphinxupquote{Entity}} object. Alternatively an iterator with the
elements of any of the types valid for a single entity argument,
e.g. a list of gene symbols.

\item {} 
\sphinxstyleliteralstrong{\sphinxupquote{mode}} (\sphinxstyleliteralemphasis{\sphinxupquote{str}}) \textendash{} Mode of counting the interactions: \sphinxtitleref{IN}, \sphinxtitleref{OUT} or \sphinxtitleref{ALL} , whether
to consider incoming, outgoing or all edges, respectively,
respective to the \sphinxtitleref{node defined in {}`entity{}`}.

\end{itemize}

\item[{Returns}] \leavevmode
\sphinxcode{\sphinxupquote{EntityList}} object containing the partners having
interactions to the queried node(s) matching all the criteria.
If \sphinxcode{\sphinxupquote{entity}} doesn’t present in the network the returned
\sphinxcode{\sphinxupquote{EntityList}} will be empty just like if no interaction matches
the criteria.

\end{description}\end{quote}

\end{fulllineitems}

\index{post\_transcriptionally\_activates() (pypath.core.network.Network method)@\spxentry{post\_transcriptionally\_activates()}\spxextra{pypath.core.network.Network method}}

\begin{fulllineitems}
\phantomsection\label{\detokenize{reference:pypath.core.network.Network.post_transcriptionally_activates}}\pysiglinewithargsret{\sphinxbfcode{\sphinxupquote{post\_transcriptionally\_activates}}}{}{}~\begin{quote}\begin{description}
\item[{Parameters}] \leavevmode\begin{itemize}
\item {} 
\sphinxstyleliteralstrong{\sphinxupquote{entity}} (\sphinxstyleliteralemphasis{\sphinxupquote{str}}\sphinxstyleliteralemphasis{\sphinxupquote{,}}{\hyperref[\detokenize{reference:pypath.core.entity.Entity}]{\sphinxcrossref{\sphinxstyleliteralemphasis{\sphinxupquote{Entity}}}}}\sphinxstyleliteralemphasis{\sphinxupquote{,}}\sphinxstyleliteralemphasis{\sphinxupquote{list}}\sphinxstyleliteralemphasis{\sphinxupquote{,}}\sphinxstyleliteralemphasis{\sphinxupquote{set}}\sphinxstyleliteralemphasis{\sphinxupquote{,}}\sphinxstyleliteralemphasis{\sphinxupquote{tuple}}\sphinxstyleliteralemphasis{\sphinxupquote{,}}\sphinxstyleliteralemphasis{\sphinxupquote{EntityList}}) \textendash{} An identifier or label of a molecular entity or an
\sphinxcode{\sphinxupquote{Entity}} object. Alternatively an iterator with the
elements of any of the types valid for a single entity argument,
e.g. a list of gene symbols.

\item {} 
\sphinxstyleliteralstrong{\sphinxupquote{mode}} (\sphinxstyleliteralemphasis{\sphinxupquote{str}}) \textendash{} Mode of counting the interactions: \sphinxtitleref{IN}, \sphinxtitleref{OUT} or \sphinxtitleref{ALL} , whether
to consider incoming, outgoing or all edges, respectively,
respective to the \sphinxtitleref{node defined in {}`entity{}`}.

\end{itemize}

\item[{Returns}] \leavevmode
\sphinxcode{\sphinxupquote{EntityList}} object containing the partners having
interactions to the queried node(s) matching all the criteria.
If \sphinxcode{\sphinxupquote{entity}} doesn’t present in the network the returned
\sphinxcode{\sphinxupquote{EntityList}} will be empty just like if no interaction matches
the criteria.

\end{description}\end{quote}

\end{fulllineitems}

\index{post\_transcriptionally\_regulated\_by() (pypath.core.network.Network method)@\spxentry{post\_transcriptionally\_regulated\_by()}\spxextra{pypath.core.network.Network method}}

\begin{fulllineitems}
\phantomsection\label{\detokenize{reference:pypath.core.network.Network.post_transcriptionally_regulated_by}}\pysiglinewithargsret{\sphinxbfcode{\sphinxupquote{post\_transcriptionally\_regulated\_by}}}{}{}~\begin{quote}\begin{description}
\item[{Parameters}] \leavevmode\begin{itemize}
\item {} 
\sphinxstyleliteralstrong{\sphinxupquote{entity}} (\sphinxstyleliteralemphasis{\sphinxupquote{str}}\sphinxstyleliteralemphasis{\sphinxupquote{,}}{\hyperref[\detokenize{reference:pypath.core.entity.Entity}]{\sphinxcrossref{\sphinxstyleliteralemphasis{\sphinxupquote{Entity}}}}}\sphinxstyleliteralemphasis{\sphinxupquote{,}}\sphinxstyleliteralemphasis{\sphinxupquote{list}}\sphinxstyleliteralemphasis{\sphinxupquote{,}}\sphinxstyleliteralemphasis{\sphinxupquote{set}}\sphinxstyleliteralemphasis{\sphinxupquote{,}}\sphinxstyleliteralemphasis{\sphinxupquote{tuple}}\sphinxstyleliteralemphasis{\sphinxupquote{,}}\sphinxstyleliteralemphasis{\sphinxupquote{EntityList}}) \textendash{} An identifier or label of a molecular entity or an
\sphinxcode{\sphinxupquote{Entity}} object. Alternatively an iterator with the
elements of any of the types valid for a single entity argument,
e.g. a list of gene symbols.

\item {} 
\sphinxstyleliteralstrong{\sphinxupquote{mode}} (\sphinxstyleliteralemphasis{\sphinxupquote{str}}) \textendash{} Mode of counting the interactions: \sphinxtitleref{IN}, \sphinxtitleref{OUT} or \sphinxtitleref{ALL} , whether
to consider incoming, outgoing or all edges, respectively,
respective to the \sphinxtitleref{node defined in {}`entity{}`}.

\end{itemize}

\item[{Returns}] \leavevmode
\sphinxcode{\sphinxupquote{EntityList}} object containing the partners having
interactions to the queried node(s) matching all the criteria.
If \sphinxcode{\sphinxupquote{entity}} doesn’t present in the network the returned
\sphinxcode{\sphinxupquote{EntityList}} will be empty just like if no interaction matches
the criteria.

\end{description}\end{quote}

\end{fulllineitems}

\index{post\_transcriptionally\_regulates() (pypath.core.network.Network method)@\spxentry{post\_transcriptionally\_regulates()}\spxextra{pypath.core.network.Network method}}

\begin{fulllineitems}
\phantomsection\label{\detokenize{reference:pypath.core.network.Network.post_transcriptionally_regulates}}\pysiglinewithargsret{\sphinxbfcode{\sphinxupquote{post\_transcriptionally\_regulates}}}{}{}~\begin{quote}\begin{description}
\item[{Parameters}] \leavevmode\begin{itemize}
\item {} 
\sphinxstyleliteralstrong{\sphinxupquote{entity}} (\sphinxstyleliteralemphasis{\sphinxupquote{str}}\sphinxstyleliteralemphasis{\sphinxupquote{,}}{\hyperref[\detokenize{reference:pypath.core.entity.Entity}]{\sphinxcrossref{\sphinxstyleliteralemphasis{\sphinxupquote{Entity}}}}}\sphinxstyleliteralemphasis{\sphinxupquote{,}}\sphinxstyleliteralemphasis{\sphinxupquote{list}}\sphinxstyleliteralemphasis{\sphinxupquote{,}}\sphinxstyleliteralemphasis{\sphinxupquote{set}}\sphinxstyleliteralemphasis{\sphinxupquote{,}}\sphinxstyleliteralemphasis{\sphinxupquote{tuple}}\sphinxstyleliteralemphasis{\sphinxupquote{,}}\sphinxstyleliteralemphasis{\sphinxupquote{EntityList}}) \textendash{} An identifier or label of a molecular entity or an
\sphinxcode{\sphinxupquote{Entity}} object. Alternatively an iterator with the
elements of any of the types valid for a single entity argument,
e.g. a list of gene symbols.

\item {} 
\sphinxstyleliteralstrong{\sphinxupquote{mode}} (\sphinxstyleliteralemphasis{\sphinxupquote{str}}) \textendash{} Mode of counting the interactions: \sphinxtitleref{IN}, \sphinxtitleref{OUT} or \sphinxtitleref{ALL} , whether
to consider incoming, outgoing or all edges, respectively,
respective to the \sphinxtitleref{node defined in {}`entity{}`}.

\end{itemize}

\item[{Returns}] \leavevmode
\sphinxcode{\sphinxupquote{EntityList}} object containing the partners having
interactions to the queried node(s) matching all the criteria.
If \sphinxcode{\sphinxupquote{entity}} doesn’t present in the network the returned
\sphinxcode{\sphinxupquote{EntityList}} will be empty just like if no interaction matches
the criteria.

\end{description}\end{quote}

\end{fulllineitems}

\index{post\_transcriptionally\_suppressed\_by() (pypath.core.network.Network method)@\spxentry{post\_transcriptionally\_suppressed\_by()}\spxextra{pypath.core.network.Network method}}

\begin{fulllineitems}
\phantomsection\label{\detokenize{reference:pypath.core.network.Network.post_transcriptionally_suppressed_by}}\pysiglinewithargsret{\sphinxbfcode{\sphinxupquote{post\_transcriptionally\_suppressed\_by}}}{}{}~\begin{quote}\begin{description}
\item[{Parameters}] \leavevmode\begin{itemize}
\item {} 
\sphinxstyleliteralstrong{\sphinxupquote{entity}} (\sphinxstyleliteralemphasis{\sphinxupquote{str}}\sphinxstyleliteralemphasis{\sphinxupquote{,}}{\hyperref[\detokenize{reference:pypath.core.entity.Entity}]{\sphinxcrossref{\sphinxstyleliteralemphasis{\sphinxupquote{Entity}}}}}\sphinxstyleliteralemphasis{\sphinxupquote{,}}\sphinxstyleliteralemphasis{\sphinxupquote{list}}\sphinxstyleliteralemphasis{\sphinxupquote{,}}\sphinxstyleliteralemphasis{\sphinxupquote{set}}\sphinxstyleliteralemphasis{\sphinxupquote{,}}\sphinxstyleliteralemphasis{\sphinxupquote{tuple}}\sphinxstyleliteralemphasis{\sphinxupquote{,}}\sphinxstyleliteralemphasis{\sphinxupquote{EntityList}}) \textendash{} An identifier or label of a molecular entity or an
\sphinxcode{\sphinxupquote{Entity}} object. Alternatively an iterator with the
elements of any of the types valid for a single entity argument,
e.g. a list of gene symbols.

\item {} 
\sphinxstyleliteralstrong{\sphinxupquote{mode}} (\sphinxstyleliteralemphasis{\sphinxupquote{str}}) \textendash{} Mode of counting the interactions: \sphinxtitleref{IN}, \sphinxtitleref{OUT} or \sphinxtitleref{ALL} , whether
to consider incoming, outgoing or all edges, respectively,
respective to the \sphinxtitleref{node defined in {}`entity{}`}.

\end{itemize}

\item[{Returns}] \leavevmode
\sphinxcode{\sphinxupquote{EntityList}} object containing the partners having
interactions to the queried node(s) matching all the criteria.
If \sphinxcode{\sphinxupquote{entity}} doesn’t present in the network the returned
\sphinxcode{\sphinxupquote{EntityList}} will be empty just like if no interaction matches
the criteria.

\end{description}\end{quote}

\end{fulllineitems}

\index{post\_transcriptionally\_suppresses() (pypath.core.network.Network method)@\spxentry{post\_transcriptionally\_suppresses()}\spxextra{pypath.core.network.Network method}}

\begin{fulllineitems}
\phantomsection\label{\detokenize{reference:pypath.core.network.Network.post_transcriptionally_suppresses}}\pysiglinewithargsret{\sphinxbfcode{\sphinxupquote{post\_transcriptionally\_suppresses}}}{}{}~\begin{quote}\begin{description}
\item[{Parameters}] \leavevmode\begin{itemize}
\item {} 
\sphinxstyleliteralstrong{\sphinxupquote{entity}} (\sphinxstyleliteralemphasis{\sphinxupquote{str}}\sphinxstyleliteralemphasis{\sphinxupquote{,}}{\hyperref[\detokenize{reference:pypath.core.entity.Entity}]{\sphinxcrossref{\sphinxstyleliteralemphasis{\sphinxupquote{Entity}}}}}\sphinxstyleliteralemphasis{\sphinxupquote{,}}\sphinxstyleliteralemphasis{\sphinxupquote{list}}\sphinxstyleliteralemphasis{\sphinxupquote{,}}\sphinxstyleliteralemphasis{\sphinxupquote{set}}\sphinxstyleliteralemphasis{\sphinxupquote{,}}\sphinxstyleliteralemphasis{\sphinxupquote{tuple}}\sphinxstyleliteralemphasis{\sphinxupquote{,}}\sphinxstyleliteralemphasis{\sphinxupquote{EntityList}}) \textendash{} An identifier or label of a molecular entity or an
\sphinxcode{\sphinxupquote{Entity}} object. Alternatively an iterator with the
elements of any of the types valid for a single entity argument,
e.g. a list of gene symbols.

\item {} 
\sphinxstyleliteralstrong{\sphinxupquote{mode}} (\sphinxstyleliteralemphasis{\sphinxupquote{str}}) \textendash{} Mode of counting the interactions: \sphinxtitleref{IN}, \sphinxtitleref{OUT} or \sphinxtitleref{ALL} , whether
to consider incoming, outgoing or all edges, respectively,
respective to the \sphinxtitleref{node defined in {}`entity{}`}.

\end{itemize}

\item[{Returns}] \leavevmode
\sphinxcode{\sphinxupquote{EntityList}} object containing the partners having
interactions to the queried node(s) matching all the criteria.
If \sphinxcode{\sphinxupquote{entity}} doesn’t present in the network the returned
\sphinxcode{\sphinxupquote{EntityList}} will be empty just like if no interaction matches
the criteria.

\end{description}\end{quote}

\end{fulllineitems}

\index{post\_translationally\_activated\_by() (pypath.core.network.Network method)@\spxentry{post\_translationally\_activated\_by()}\spxextra{pypath.core.network.Network method}}

\begin{fulllineitems}
\phantomsection\label{\detokenize{reference:pypath.core.network.Network.post_translationally_activated_by}}\pysiglinewithargsret{\sphinxbfcode{\sphinxupquote{post\_translationally\_activated\_by}}}{}{}~\begin{quote}\begin{description}
\item[{Parameters}] \leavevmode\begin{itemize}
\item {} 
\sphinxstyleliteralstrong{\sphinxupquote{entity}} (\sphinxstyleliteralemphasis{\sphinxupquote{str}}\sphinxstyleliteralemphasis{\sphinxupquote{,}}{\hyperref[\detokenize{reference:pypath.core.entity.Entity}]{\sphinxcrossref{\sphinxstyleliteralemphasis{\sphinxupquote{Entity}}}}}\sphinxstyleliteralemphasis{\sphinxupquote{,}}\sphinxstyleliteralemphasis{\sphinxupquote{list}}\sphinxstyleliteralemphasis{\sphinxupquote{,}}\sphinxstyleliteralemphasis{\sphinxupquote{set}}\sphinxstyleliteralemphasis{\sphinxupquote{,}}\sphinxstyleliteralemphasis{\sphinxupquote{tuple}}\sphinxstyleliteralemphasis{\sphinxupquote{,}}\sphinxstyleliteralemphasis{\sphinxupquote{EntityList}}) \textendash{} An identifier or label of a molecular entity or an
\sphinxcode{\sphinxupquote{Entity}} object. Alternatively an iterator with the
elements of any of the types valid for a single entity argument,
e.g. a list of gene symbols.

\item {} 
\sphinxstyleliteralstrong{\sphinxupquote{mode}} (\sphinxstyleliteralemphasis{\sphinxupquote{str}}) \textendash{} Mode of counting the interactions: \sphinxtitleref{IN}, \sphinxtitleref{OUT} or \sphinxtitleref{ALL} , whether
to consider incoming, outgoing or all edges, respectively,
respective to the \sphinxtitleref{node defined in {}`entity{}`}.

\end{itemize}

\item[{Returns}] \leavevmode
\sphinxcode{\sphinxupquote{EntityList}} object containing the partners having
interactions to the queried node(s) matching all the criteria.
If \sphinxcode{\sphinxupquote{entity}} doesn’t present in the network the returned
\sphinxcode{\sphinxupquote{EntityList}} will be empty just like if no interaction matches
the criteria.

\end{description}\end{quote}

\end{fulllineitems}

\index{post\_translationally\_activates() (pypath.core.network.Network method)@\spxentry{post\_translationally\_activates()}\spxextra{pypath.core.network.Network method}}

\begin{fulllineitems}
\phantomsection\label{\detokenize{reference:pypath.core.network.Network.post_translationally_activates}}\pysiglinewithargsret{\sphinxbfcode{\sphinxupquote{post\_translationally\_activates}}}{}{}~\begin{quote}\begin{description}
\item[{Parameters}] \leavevmode\begin{itemize}
\item {} 
\sphinxstyleliteralstrong{\sphinxupquote{entity}} (\sphinxstyleliteralemphasis{\sphinxupquote{str}}\sphinxstyleliteralemphasis{\sphinxupquote{,}}{\hyperref[\detokenize{reference:pypath.core.entity.Entity}]{\sphinxcrossref{\sphinxstyleliteralemphasis{\sphinxupquote{Entity}}}}}\sphinxstyleliteralemphasis{\sphinxupquote{,}}\sphinxstyleliteralemphasis{\sphinxupquote{list}}\sphinxstyleliteralemphasis{\sphinxupquote{,}}\sphinxstyleliteralemphasis{\sphinxupquote{set}}\sphinxstyleliteralemphasis{\sphinxupquote{,}}\sphinxstyleliteralemphasis{\sphinxupquote{tuple}}\sphinxstyleliteralemphasis{\sphinxupquote{,}}\sphinxstyleliteralemphasis{\sphinxupquote{EntityList}}) \textendash{} An identifier or label of a molecular entity or an
\sphinxcode{\sphinxupquote{Entity}} object. Alternatively an iterator with the
elements of any of the types valid for a single entity argument,
e.g. a list of gene symbols.

\item {} 
\sphinxstyleliteralstrong{\sphinxupquote{mode}} (\sphinxstyleliteralemphasis{\sphinxupquote{str}}) \textendash{} Mode of counting the interactions: \sphinxtitleref{IN}, \sphinxtitleref{OUT} or \sphinxtitleref{ALL} , whether
to consider incoming, outgoing or all edges, respectively,
respective to the \sphinxtitleref{node defined in {}`entity{}`}.

\end{itemize}

\item[{Returns}] \leavevmode
\sphinxcode{\sphinxupquote{EntityList}} object containing the partners having
interactions to the queried node(s) matching all the criteria.
If \sphinxcode{\sphinxupquote{entity}} doesn’t present in the network the returned
\sphinxcode{\sphinxupquote{EntityList}} will be empty just like if no interaction matches
the criteria.

\end{description}\end{quote}

\end{fulllineitems}

\index{post\_translationally\_regulated\_by() (pypath.core.network.Network method)@\spxentry{post\_translationally\_regulated\_by()}\spxextra{pypath.core.network.Network method}}

\begin{fulllineitems}
\phantomsection\label{\detokenize{reference:pypath.core.network.Network.post_translationally_regulated_by}}\pysiglinewithargsret{\sphinxbfcode{\sphinxupquote{post\_translationally\_regulated\_by}}}{}{}~\begin{quote}\begin{description}
\item[{Parameters}] \leavevmode\begin{itemize}
\item {} 
\sphinxstyleliteralstrong{\sphinxupquote{entity}} (\sphinxstyleliteralemphasis{\sphinxupquote{str}}\sphinxstyleliteralemphasis{\sphinxupquote{,}}{\hyperref[\detokenize{reference:pypath.core.entity.Entity}]{\sphinxcrossref{\sphinxstyleliteralemphasis{\sphinxupquote{Entity}}}}}\sphinxstyleliteralemphasis{\sphinxupquote{,}}\sphinxstyleliteralemphasis{\sphinxupquote{list}}\sphinxstyleliteralemphasis{\sphinxupquote{,}}\sphinxstyleliteralemphasis{\sphinxupquote{set}}\sphinxstyleliteralemphasis{\sphinxupquote{,}}\sphinxstyleliteralemphasis{\sphinxupquote{tuple}}\sphinxstyleliteralemphasis{\sphinxupquote{,}}\sphinxstyleliteralemphasis{\sphinxupquote{EntityList}}) \textendash{} An identifier or label of a molecular entity or an
\sphinxcode{\sphinxupquote{Entity}} object. Alternatively an iterator with the
elements of any of the types valid for a single entity argument,
e.g. a list of gene symbols.

\item {} 
\sphinxstyleliteralstrong{\sphinxupquote{mode}} (\sphinxstyleliteralemphasis{\sphinxupquote{str}}) \textendash{} Mode of counting the interactions: \sphinxtitleref{IN}, \sphinxtitleref{OUT} or \sphinxtitleref{ALL} , whether
to consider incoming, outgoing or all edges, respectively,
respective to the \sphinxtitleref{node defined in {}`entity{}`}.

\end{itemize}

\item[{Returns}] \leavevmode
\sphinxcode{\sphinxupquote{EntityList}} object containing the partners having
interactions to the queried node(s) matching all the criteria.
If \sphinxcode{\sphinxupquote{entity}} doesn’t present in the network the returned
\sphinxcode{\sphinxupquote{EntityList}} will be empty just like if no interaction matches
the criteria.

\end{description}\end{quote}

\end{fulllineitems}

\index{post\_translationally\_regulates() (pypath.core.network.Network method)@\spxentry{post\_translationally\_regulates()}\spxextra{pypath.core.network.Network method}}

\begin{fulllineitems}
\phantomsection\label{\detokenize{reference:pypath.core.network.Network.post_translationally_regulates}}\pysiglinewithargsret{\sphinxbfcode{\sphinxupquote{post\_translationally\_regulates}}}{}{}~\begin{quote}\begin{description}
\item[{Parameters}] \leavevmode\begin{itemize}
\item {} 
\sphinxstyleliteralstrong{\sphinxupquote{entity}} (\sphinxstyleliteralemphasis{\sphinxupquote{str}}\sphinxstyleliteralemphasis{\sphinxupquote{,}}{\hyperref[\detokenize{reference:pypath.core.entity.Entity}]{\sphinxcrossref{\sphinxstyleliteralemphasis{\sphinxupquote{Entity}}}}}\sphinxstyleliteralemphasis{\sphinxupquote{,}}\sphinxstyleliteralemphasis{\sphinxupquote{list}}\sphinxstyleliteralemphasis{\sphinxupquote{,}}\sphinxstyleliteralemphasis{\sphinxupquote{set}}\sphinxstyleliteralemphasis{\sphinxupquote{,}}\sphinxstyleliteralemphasis{\sphinxupquote{tuple}}\sphinxstyleliteralemphasis{\sphinxupquote{,}}\sphinxstyleliteralemphasis{\sphinxupquote{EntityList}}) \textendash{} An identifier or label of a molecular entity or an
\sphinxcode{\sphinxupquote{Entity}} object. Alternatively an iterator with the
elements of any of the types valid for a single entity argument,
e.g. a list of gene symbols.

\item {} 
\sphinxstyleliteralstrong{\sphinxupquote{mode}} (\sphinxstyleliteralemphasis{\sphinxupquote{str}}) \textendash{} Mode of counting the interactions: \sphinxtitleref{IN}, \sphinxtitleref{OUT} or \sphinxtitleref{ALL} , whether
to consider incoming, outgoing or all edges, respectively,
respective to the \sphinxtitleref{node defined in {}`entity{}`}.

\end{itemize}

\item[{Returns}] \leavevmode
\sphinxcode{\sphinxupquote{EntityList}} object containing the partners having
interactions to the queried node(s) matching all the criteria.
If \sphinxcode{\sphinxupquote{entity}} doesn’t present in the network the returned
\sphinxcode{\sphinxupquote{EntityList}} will be empty just like if no interaction matches
the criteria.

\end{description}\end{quote}

\end{fulllineitems}

\index{post\_translationally\_suppressed\_by() (pypath.core.network.Network method)@\spxentry{post\_translationally\_suppressed\_by()}\spxextra{pypath.core.network.Network method}}

\begin{fulllineitems}
\phantomsection\label{\detokenize{reference:pypath.core.network.Network.post_translationally_suppressed_by}}\pysiglinewithargsret{\sphinxbfcode{\sphinxupquote{post\_translationally\_suppressed\_by}}}{}{}~\begin{quote}\begin{description}
\item[{Parameters}] \leavevmode\begin{itemize}
\item {} 
\sphinxstyleliteralstrong{\sphinxupquote{entity}} (\sphinxstyleliteralemphasis{\sphinxupquote{str}}\sphinxstyleliteralemphasis{\sphinxupquote{,}}{\hyperref[\detokenize{reference:pypath.core.entity.Entity}]{\sphinxcrossref{\sphinxstyleliteralemphasis{\sphinxupquote{Entity}}}}}\sphinxstyleliteralemphasis{\sphinxupquote{,}}\sphinxstyleliteralemphasis{\sphinxupquote{list}}\sphinxstyleliteralemphasis{\sphinxupquote{,}}\sphinxstyleliteralemphasis{\sphinxupquote{set}}\sphinxstyleliteralemphasis{\sphinxupquote{,}}\sphinxstyleliteralemphasis{\sphinxupquote{tuple}}\sphinxstyleliteralemphasis{\sphinxupquote{,}}\sphinxstyleliteralemphasis{\sphinxupquote{EntityList}}) \textendash{} An identifier or label of a molecular entity or an
\sphinxcode{\sphinxupquote{Entity}} object. Alternatively an iterator with the
elements of any of the types valid for a single entity argument,
e.g. a list of gene symbols.

\item {} 
\sphinxstyleliteralstrong{\sphinxupquote{mode}} (\sphinxstyleliteralemphasis{\sphinxupquote{str}}) \textendash{} Mode of counting the interactions: \sphinxtitleref{IN}, \sphinxtitleref{OUT} or \sphinxtitleref{ALL} , whether
to consider incoming, outgoing or all edges, respectively,
respective to the \sphinxtitleref{node defined in {}`entity{}`}.

\end{itemize}

\item[{Returns}] \leavevmode
\sphinxcode{\sphinxupquote{EntityList}} object containing the partners having
interactions to the queried node(s) matching all the criteria.
If \sphinxcode{\sphinxupquote{entity}} doesn’t present in the network the returned
\sphinxcode{\sphinxupquote{EntityList}} will be empty just like if no interaction matches
the criteria.

\end{description}\end{quote}

\end{fulllineitems}

\index{post\_translationally\_suppresses() (pypath.core.network.Network method)@\spxentry{post\_translationally\_suppresses()}\spxextra{pypath.core.network.Network method}}

\begin{fulllineitems}
\phantomsection\label{\detokenize{reference:pypath.core.network.Network.post_translationally_suppresses}}\pysiglinewithargsret{\sphinxbfcode{\sphinxupquote{post\_translationally\_suppresses}}}{}{}~\begin{quote}\begin{description}
\item[{Parameters}] \leavevmode\begin{itemize}
\item {} 
\sphinxstyleliteralstrong{\sphinxupquote{entity}} (\sphinxstyleliteralemphasis{\sphinxupquote{str}}\sphinxstyleliteralemphasis{\sphinxupquote{,}}{\hyperref[\detokenize{reference:pypath.core.entity.Entity}]{\sphinxcrossref{\sphinxstyleliteralemphasis{\sphinxupquote{Entity}}}}}\sphinxstyleliteralemphasis{\sphinxupquote{,}}\sphinxstyleliteralemphasis{\sphinxupquote{list}}\sphinxstyleliteralemphasis{\sphinxupquote{,}}\sphinxstyleliteralemphasis{\sphinxupquote{set}}\sphinxstyleliteralemphasis{\sphinxupquote{,}}\sphinxstyleliteralemphasis{\sphinxupquote{tuple}}\sphinxstyleliteralemphasis{\sphinxupquote{,}}\sphinxstyleliteralemphasis{\sphinxupquote{EntityList}}) \textendash{} An identifier or label of a molecular entity or an
\sphinxcode{\sphinxupquote{Entity}} object. Alternatively an iterator with the
elements of any of the types valid for a single entity argument,
e.g. a list of gene symbols.

\item {} 
\sphinxstyleliteralstrong{\sphinxupquote{mode}} (\sphinxstyleliteralemphasis{\sphinxupquote{str}}) \textendash{} Mode of counting the interactions: \sphinxtitleref{IN}, \sphinxtitleref{OUT} or \sphinxtitleref{ALL} , whether
to consider incoming, outgoing or all edges, respectively,
respective to the \sphinxtitleref{node defined in {}`entity{}`}.

\end{itemize}

\item[{Returns}] \leavevmode
\sphinxcode{\sphinxupquote{EntityList}} object containing the partners having
interactions to the queried node(s) matching all the criteria.
If \sphinxcode{\sphinxupquote{entity}} doesn’t present in the network the returned
\sphinxcode{\sphinxupquote{EntityList}} will be empty just like if no interaction matches
the criteria.

\end{description}\end{quote}

\end{fulllineitems}

\index{protein\_identifiers\_by\_data\_model() (pypath.core.network.Network method)@\spxentry{protein\_identifiers\_by\_data\_model()}\spxextra{pypath.core.network.Network method}}

\begin{fulllineitems}
\phantomsection\label{\detokenize{reference:pypath.core.network.Network.protein_identifiers_by_data_model}}\pysiglinewithargsret{\sphinxbfcode{\sphinxupquote{protein\_identifiers\_by\_data\_model}}}{}{}
Built-in immutable sequence.

If no argument is given, the constructor returns an empty tuple.
If iterable is specified the tuple is initialized from iterable’s items.

If the argument is a tuple, the return value is the same object.

\end{fulllineitems}

\index{protein\_identifiers\_by\_interaction\_type() (pypath.core.network.Network method)@\spxentry{protein\_identifiers\_by\_interaction\_type()}\spxextra{pypath.core.network.Network method}}

\begin{fulllineitems}
\phantomsection\label{\detokenize{reference:pypath.core.network.Network.protein_identifiers_by_interaction_type}}\pysiglinewithargsret{\sphinxbfcode{\sphinxupquote{protein\_identifiers\_by\_interaction\_type}}}{}{}
Built-in immutable sequence.

If no argument is given, the constructor returns an empty tuple.
If iterable is specified the tuple is initialized from iterable’s items.

If the argument is a tuple, the return value is the same object.

\end{fulllineitems}

\index{protein\_identifiers\_by\_interaction\_type\_and\_data\_model() (pypath.core.network.Network method)@\spxentry{protein\_identifiers\_by\_interaction\_type\_and\_data\_model()}\spxextra{pypath.core.network.Network method}}

\begin{fulllineitems}
\phantomsection\label{\detokenize{reference:pypath.core.network.Network.protein_identifiers_by_interaction_type_and_data_model}}\pysiglinewithargsret{\sphinxbfcode{\sphinxupquote{protein\_identifiers\_by\_interaction\_type\_and\_data\_model}}}{}{}
Built-in immutable sequence.

If no argument is given, the constructor returns an empty tuple.
If iterable is specified the tuple is initialized from iterable’s items.

If the argument is a tuple, the return value is the same object.

\end{fulllineitems}

\index{protein\_identifiers\_by\_interaction\_type\_and\_data\_model\_and\_resource() (pypath.core.network.Network method)@\spxentry{protein\_identifiers\_by\_interaction\_type\_and\_data\_model\_and\_resource()}\spxextra{pypath.core.network.Network method}}

\begin{fulllineitems}
\phantomsection\label{\detokenize{reference:pypath.core.network.Network.protein_identifiers_by_interaction_type_and_data_model_and_resource}}\pysiglinewithargsret{\sphinxbfcode{\sphinxupquote{protein\_identifiers\_by\_interaction\_type\_and\_data\_model\_and\_resource}}}{}{}
Built-in immutable sequence.

If no argument is given, the constructor returns an empty tuple.
If iterable is specified the tuple is initialized from iterable’s items.

If the argument is a tuple, the return value is the same object.

\end{fulllineitems}

\index{protein\_identifiers\_by\_reference() (pypath.core.network.Network method)@\spxentry{protein\_identifiers\_by\_reference()}\spxextra{pypath.core.network.Network method}}

\begin{fulllineitems}
\phantomsection\label{\detokenize{reference:pypath.core.network.Network.protein_identifiers_by_reference}}\pysiglinewithargsret{\sphinxbfcode{\sphinxupquote{protein\_identifiers\_by\_reference}}}{}{}
Built-in immutable sequence.

If no argument is given, the constructor returns an empty tuple.
If iterable is specified the tuple is initialized from iterable’s items.

If the argument is a tuple, the return value is the same object.

\end{fulllineitems}

\index{protein\_identifiers\_by\_resource() (pypath.core.network.Network method)@\spxentry{protein\_identifiers\_by\_resource()}\spxextra{pypath.core.network.Network method}}

\begin{fulllineitems}
\phantomsection\label{\detokenize{reference:pypath.core.network.Network.protein_identifiers_by_resource}}\pysiglinewithargsret{\sphinxbfcode{\sphinxupquote{protein\_identifiers\_by\_resource}}}{}{}
Built-in immutable sequence.

If no argument is given, the constructor returns an empty tuple.
If iterable is specified the tuple is initialized from iterable’s items.

If the argument is a tuple, the return value is the same object.

\end{fulllineitems}

\index{protein\_labels\_by\_data\_model() (pypath.core.network.Network method)@\spxentry{protein\_labels\_by\_data\_model()}\spxextra{pypath.core.network.Network method}}

\begin{fulllineitems}
\phantomsection\label{\detokenize{reference:pypath.core.network.Network.protein_labels_by_data_model}}\pysiglinewithargsret{\sphinxbfcode{\sphinxupquote{protein\_labels\_by\_data\_model}}}{}{}
Built-in immutable sequence.

If no argument is given, the constructor returns an empty tuple.
If iterable is specified the tuple is initialized from iterable’s items.

If the argument is a tuple, the return value is the same object.

\end{fulllineitems}

\index{protein\_labels\_by\_interaction\_type() (pypath.core.network.Network method)@\spxentry{protein\_labels\_by\_interaction\_type()}\spxextra{pypath.core.network.Network method}}

\begin{fulllineitems}
\phantomsection\label{\detokenize{reference:pypath.core.network.Network.protein_labels_by_interaction_type}}\pysiglinewithargsret{\sphinxbfcode{\sphinxupquote{protein\_labels\_by\_interaction\_type}}}{}{}
Built-in immutable sequence.

If no argument is given, the constructor returns an empty tuple.
If iterable is specified the tuple is initialized from iterable’s items.

If the argument is a tuple, the return value is the same object.

\end{fulllineitems}

\index{protein\_labels\_by\_interaction\_type\_and\_data\_model() (pypath.core.network.Network method)@\spxentry{protein\_labels\_by\_interaction\_type\_and\_data\_model()}\spxextra{pypath.core.network.Network method}}

\begin{fulllineitems}
\phantomsection\label{\detokenize{reference:pypath.core.network.Network.protein_labels_by_interaction_type_and_data_model}}\pysiglinewithargsret{\sphinxbfcode{\sphinxupquote{protein\_labels\_by\_interaction\_type\_and\_data\_model}}}{}{}
Built-in immutable sequence.

If no argument is given, the constructor returns an empty tuple.
If iterable is specified the tuple is initialized from iterable’s items.

If the argument is a tuple, the return value is the same object.

\end{fulllineitems}

\index{protein\_labels\_by\_interaction\_type\_and\_data\_model\_and\_resource() (pypath.core.network.Network method)@\spxentry{protein\_labels\_by\_interaction\_type\_and\_data\_model\_and\_resource()}\spxextra{pypath.core.network.Network method}}

\begin{fulllineitems}
\phantomsection\label{\detokenize{reference:pypath.core.network.Network.protein_labels_by_interaction_type_and_data_model_and_resource}}\pysiglinewithargsret{\sphinxbfcode{\sphinxupquote{protein\_labels\_by\_interaction\_type\_and\_data\_model\_and\_resource}}}{}{}
Built-in immutable sequence.

If no argument is given, the constructor returns an empty tuple.
If iterable is specified the tuple is initialized from iterable’s items.

If the argument is a tuple, the return value is the same object.

\end{fulllineitems}

\index{protein\_labels\_by\_reference() (pypath.core.network.Network method)@\spxentry{protein\_labels\_by\_reference()}\spxextra{pypath.core.network.Network method}}

\begin{fulllineitems}
\phantomsection\label{\detokenize{reference:pypath.core.network.Network.protein_labels_by_reference}}\pysiglinewithargsret{\sphinxbfcode{\sphinxupquote{protein\_labels\_by\_reference}}}{}{}
Built-in immutable sequence.

If no argument is given, the constructor returns an empty tuple.
If iterable is specified the tuple is initialized from iterable’s items.

If the argument is a tuple, the return value is the same object.

\end{fulllineitems}

\index{protein\_labels\_by\_resource() (pypath.core.network.Network method)@\spxentry{protein\_labels\_by\_resource()}\spxextra{pypath.core.network.Network method}}

\begin{fulllineitems}
\phantomsection\label{\detokenize{reference:pypath.core.network.Network.protein_labels_by_resource}}\pysiglinewithargsret{\sphinxbfcode{\sphinxupquote{protein\_labels\_by\_resource}}}{}{}
Built-in immutable sequence.

If no argument is given, the constructor returns an empty tuple.
If iterable is specified the tuple is initialized from iterable’s items.

If the argument is a tuple, the return value is the same object.

\end{fulllineitems}

\index{proteins\_by\_data\_model() (pypath.core.network.Network method)@\spxentry{proteins\_by\_data\_model()}\spxextra{pypath.core.network.Network method}}

\begin{fulllineitems}
\phantomsection\label{\detokenize{reference:pypath.core.network.Network.proteins_by_data_model}}\pysiglinewithargsret{\sphinxbfcode{\sphinxupquote{proteins\_by\_data\_model}}}{}{}
Built-in immutable sequence.

If no argument is given, the constructor returns an empty tuple.
If iterable is specified the tuple is initialized from iterable’s items.

If the argument is a tuple, the return value is the same object.

\end{fulllineitems}

\index{proteins\_by\_interaction\_type() (pypath.core.network.Network method)@\spxentry{proteins\_by\_interaction\_type()}\spxextra{pypath.core.network.Network method}}

\begin{fulllineitems}
\phantomsection\label{\detokenize{reference:pypath.core.network.Network.proteins_by_interaction_type}}\pysiglinewithargsret{\sphinxbfcode{\sphinxupquote{proteins\_by\_interaction\_type}}}{}{}
Built-in immutable sequence.

If no argument is given, the constructor returns an empty tuple.
If iterable is specified the tuple is initialized from iterable’s items.

If the argument is a tuple, the return value is the same object.

\end{fulllineitems}

\index{proteins\_by\_interaction\_type\_and\_data\_model() (pypath.core.network.Network method)@\spxentry{proteins\_by\_interaction\_type\_and\_data\_model()}\spxextra{pypath.core.network.Network method}}

\begin{fulllineitems}
\phantomsection\label{\detokenize{reference:pypath.core.network.Network.proteins_by_interaction_type_and_data_model}}\pysiglinewithargsret{\sphinxbfcode{\sphinxupquote{proteins\_by\_interaction\_type\_and\_data\_model}}}{}{}
Built-in immutable sequence.

If no argument is given, the constructor returns an empty tuple.
If iterable is specified the tuple is initialized from iterable’s items.

If the argument is a tuple, the return value is the same object.

\end{fulllineitems}

\index{proteins\_by\_interaction\_type\_and\_data\_model\_and\_resource() (pypath.core.network.Network method)@\spxentry{proteins\_by\_interaction\_type\_and\_data\_model\_and\_resource()}\spxextra{pypath.core.network.Network method}}

\begin{fulllineitems}
\phantomsection\label{\detokenize{reference:pypath.core.network.Network.proteins_by_interaction_type_and_data_model_and_resource}}\pysiglinewithargsret{\sphinxbfcode{\sphinxupquote{proteins\_by\_interaction\_type\_and\_data\_model\_and\_resource}}}{}{}
Built-in immutable sequence.

If no argument is given, the constructor returns an empty tuple.
If iterable is specified the tuple is initialized from iterable’s items.

If the argument is a tuple, the return value is the same object.

\end{fulllineitems}

\index{proteins\_by\_reference() (pypath.core.network.Network method)@\spxentry{proteins\_by\_reference()}\spxextra{pypath.core.network.Network method}}

\begin{fulllineitems}
\phantomsection\label{\detokenize{reference:pypath.core.network.Network.proteins_by_reference}}\pysiglinewithargsret{\sphinxbfcode{\sphinxupquote{proteins\_by\_reference}}}{}{}
Built-in immutable sequence.

If no argument is given, the constructor returns an empty tuple.
If iterable is specified the tuple is initialized from iterable’s items.

If the argument is a tuple, the return value is the same object.

\end{fulllineitems}

\index{proteins\_by\_resource() (pypath.core.network.Network method)@\spxentry{proteins\_by\_resource()}\spxextra{pypath.core.network.Network method}}

\begin{fulllineitems}
\phantomsection\label{\detokenize{reference:pypath.core.network.Network.proteins_by_resource}}\pysiglinewithargsret{\sphinxbfcode{\sphinxupquote{proteins\_by\_resource}}}{}{}
Built-in immutable sequence.

If no argument is given, the constructor returns an empty tuple.
If iterable is specified the tuple is initialized from iterable’s items.

If the argument is a tuple, the return value is the same object.

\end{fulllineitems}

\index{references\_by\_data\_model() (pypath.core.network.Network method)@\spxentry{references\_by\_data\_model()}\spxextra{pypath.core.network.Network method}}

\begin{fulllineitems}
\phantomsection\label{\detokenize{reference:pypath.core.network.Network.references_by_data_model}}\pysiglinewithargsret{\sphinxbfcode{\sphinxupquote{references\_by\_data\_model}}}{}{}
Built-in immutable sequence.

If no argument is given, the constructor returns an empty tuple.
If iterable is specified the tuple is initialized from iterable’s items.

If the argument is a tuple, the return value is the same object.

\end{fulllineitems}

\index{references\_by\_interaction\_type() (pypath.core.network.Network method)@\spxentry{references\_by\_interaction\_type()}\spxextra{pypath.core.network.Network method}}

\begin{fulllineitems}
\phantomsection\label{\detokenize{reference:pypath.core.network.Network.references_by_interaction_type}}\pysiglinewithargsret{\sphinxbfcode{\sphinxupquote{references\_by\_interaction\_type}}}{}{}
Built-in immutable sequence.

If no argument is given, the constructor returns an empty tuple.
If iterable is specified the tuple is initialized from iterable’s items.

If the argument is a tuple, the return value is the same object.

\end{fulllineitems}

\index{references\_by\_interaction\_type\_and\_data\_model() (pypath.core.network.Network method)@\spxentry{references\_by\_interaction\_type\_and\_data\_model()}\spxextra{pypath.core.network.Network method}}

\begin{fulllineitems}
\phantomsection\label{\detokenize{reference:pypath.core.network.Network.references_by_interaction_type_and_data_model}}\pysiglinewithargsret{\sphinxbfcode{\sphinxupquote{references\_by\_interaction\_type\_and\_data\_model}}}{}{}
Built-in immutable sequence.

If no argument is given, the constructor returns an empty tuple.
If iterable is specified the tuple is initialized from iterable’s items.

If the argument is a tuple, the return value is the same object.

\end{fulllineitems}

\index{references\_by\_interaction\_type\_and\_data\_model\_and\_resource() (pypath.core.network.Network method)@\spxentry{references\_by\_interaction\_type\_and\_data\_model\_and\_resource()}\spxextra{pypath.core.network.Network method}}

\begin{fulllineitems}
\phantomsection\label{\detokenize{reference:pypath.core.network.Network.references_by_interaction_type_and_data_model_and_resource}}\pysiglinewithargsret{\sphinxbfcode{\sphinxupquote{references\_by\_interaction\_type\_and\_data\_model\_and\_resource}}}{}{}
Built-in immutable sequence.

If no argument is given, the constructor returns an empty tuple.
If iterable is specified the tuple is initialized from iterable’s items.

If the argument is a tuple, the return value is the same object.

\end{fulllineitems}

\index{references\_by\_reference() (pypath.core.network.Network method)@\spxentry{references\_by\_reference()}\spxextra{pypath.core.network.Network method}}

\begin{fulllineitems}
\phantomsection\label{\detokenize{reference:pypath.core.network.Network.references_by_reference}}\pysiglinewithargsret{\sphinxbfcode{\sphinxupquote{references\_by\_reference}}}{}{}
Built-in immutable sequence.

If no argument is given, the constructor returns an empty tuple.
If iterable is specified the tuple is initialized from iterable’s items.

If the argument is a tuple, the return value is the same object.

\end{fulllineitems}

\index{references\_by\_resource() (pypath.core.network.Network method)@\spxentry{references\_by\_resource()}\spxextra{pypath.core.network.Network method}}

\begin{fulllineitems}
\phantomsection\label{\detokenize{reference:pypath.core.network.Network.references_by_resource}}\pysiglinewithargsret{\sphinxbfcode{\sphinxupquote{references\_by\_resource}}}{}{}
Built-in immutable sequence.

If no argument is given, the constructor returns an empty tuple.
If iterable is specified the tuple is initialized from iterable’s items.

If the argument is a tuple, the return value is the same object.

\end{fulllineitems}

\index{regulated\_by() (pypath.core.network.Network method)@\spxentry{regulated\_by()}\spxextra{pypath.core.network.Network method}}

\begin{fulllineitems}
\phantomsection\label{\detokenize{reference:pypath.core.network.Network.regulated_by}}\pysiglinewithargsret{\sphinxbfcode{\sphinxupquote{regulated\_by}}}{}{}~\begin{quote}\begin{description}
\item[{Parameters}] \leavevmode\begin{itemize}
\item {} 
\sphinxstyleliteralstrong{\sphinxupquote{entity}} (\sphinxstyleliteralemphasis{\sphinxupquote{str}}\sphinxstyleliteralemphasis{\sphinxupquote{,}}{\hyperref[\detokenize{reference:pypath.core.entity.Entity}]{\sphinxcrossref{\sphinxstyleliteralemphasis{\sphinxupquote{Entity}}}}}\sphinxstyleliteralemphasis{\sphinxupquote{,}}\sphinxstyleliteralemphasis{\sphinxupquote{list}}\sphinxstyleliteralemphasis{\sphinxupquote{,}}\sphinxstyleliteralemphasis{\sphinxupquote{set}}\sphinxstyleliteralemphasis{\sphinxupquote{,}}\sphinxstyleliteralemphasis{\sphinxupquote{tuple}}\sphinxstyleliteralemphasis{\sphinxupquote{,}}\sphinxstyleliteralemphasis{\sphinxupquote{EntityList}}) \textendash{} An identifier or label of a molecular entity or an
\sphinxcode{\sphinxupquote{Entity}} object. Alternatively an iterator with the
elements of any of the types valid for a single entity argument,
e.g. a list of gene symbols.

\item {} 
\sphinxstyleliteralstrong{\sphinxupquote{mode}} (\sphinxstyleliteralemphasis{\sphinxupquote{str}}) \textendash{} Mode of counting the interactions: \sphinxtitleref{IN}, \sphinxtitleref{OUT} or \sphinxtitleref{ALL} , whether
to consider incoming, outgoing or all edges, respectively,
respective to the \sphinxtitleref{node defined in {}`entity{}`}.

\end{itemize}

\item[{Returns}] \leavevmode
\sphinxcode{\sphinxupquote{EntityList}} object containing the partners having
interactions to the queried node(s) matching all the criteria.
If \sphinxcode{\sphinxupquote{entity}} doesn’t present in the network the returned
\sphinxcode{\sphinxupquote{EntityList}} will be empty just like if no interaction matches
the criteria.

\end{description}\end{quote}

\end{fulllineitems}

\index{regulates() (pypath.core.network.Network method)@\spxentry{regulates()}\spxextra{pypath.core.network.Network method}}

\begin{fulllineitems}
\phantomsection\label{\detokenize{reference:pypath.core.network.Network.regulates}}\pysiglinewithargsret{\sphinxbfcode{\sphinxupquote{regulates}}}{}{}~\begin{quote}\begin{description}
\item[{Parameters}] \leavevmode\begin{itemize}
\item {} 
\sphinxstyleliteralstrong{\sphinxupquote{entity}} (\sphinxstyleliteralemphasis{\sphinxupquote{str}}\sphinxstyleliteralemphasis{\sphinxupquote{,}}{\hyperref[\detokenize{reference:pypath.core.entity.Entity}]{\sphinxcrossref{\sphinxstyleliteralemphasis{\sphinxupquote{Entity}}}}}\sphinxstyleliteralemphasis{\sphinxupquote{,}}\sphinxstyleliteralemphasis{\sphinxupquote{list}}\sphinxstyleliteralemphasis{\sphinxupquote{,}}\sphinxstyleliteralemphasis{\sphinxupquote{set}}\sphinxstyleliteralemphasis{\sphinxupquote{,}}\sphinxstyleliteralemphasis{\sphinxupquote{tuple}}\sphinxstyleliteralemphasis{\sphinxupquote{,}}\sphinxstyleliteralemphasis{\sphinxupquote{EntityList}}) \textendash{} An identifier or label of a molecular entity or an
\sphinxcode{\sphinxupquote{Entity}} object. Alternatively an iterator with the
elements of any of the types valid for a single entity argument,
e.g. a list of gene symbols.

\item {} 
\sphinxstyleliteralstrong{\sphinxupquote{mode}} (\sphinxstyleliteralemphasis{\sphinxupquote{str}}) \textendash{} Mode of counting the interactions: \sphinxtitleref{IN}, \sphinxtitleref{OUT} or \sphinxtitleref{ALL} , whether
to consider incoming, outgoing or all edges, respectively,
respective to the \sphinxtitleref{node defined in {}`entity{}`}.

\end{itemize}

\item[{Returns}] \leavevmode
\sphinxcode{\sphinxupquote{EntityList}} object containing the partners having
interactions to the queried node(s) matching all the criteria.
If \sphinxcode{\sphinxupquote{entity}} doesn’t present in the network the returned
\sphinxcode{\sphinxupquote{EntityList}} will be empty just like if no interaction matches
the criteria.

\end{description}\end{quote}

\end{fulllineitems}

\index{reload() (pypath.core.network.Network method)@\spxentry{reload()}\spxextra{pypath.core.network.Network method}}

\begin{fulllineitems}
\phantomsection\label{\detokenize{reference:pypath.core.network.Network.reload}}\pysiglinewithargsret{\sphinxbfcode{\sphinxupquote{reload}}}{}{}
Reloads the object from the module level.

\end{fulllineitems}

\index{remove\_interaction() (pypath.core.network.Network method)@\spxentry{remove\_interaction()}\spxextra{pypath.core.network.Network method}}

\begin{fulllineitems}
\phantomsection\label{\detokenize{reference:pypath.core.network.Network.remove_interaction}}\pysiglinewithargsret{\sphinxbfcode{\sphinxupquote{remove\_interaction}}}{\emph{entity\_a}, \emph{entity\_b}}{}
Removes the interaction between two nodes if exists.
\begin{quote}\begin{description}
\item[{Parameters}] \leavevmode
\sphinxstyleliteralstrong{\sphinxupquote{entity\_a}}\sphinxstyleliteralstrong{\sphinxupquote{,}}\sphinxstyleliteralstrong{\sphinxupquote{entity\_b}} (\sphinxstyleliteralemphasis{\sphinxupquote{str}}\sphinxstyleliteralemphasis{\sphinxupquote{,}}{\hyperref[\detokenize{reference:pypath.core.entity.Entity}]{\sphinxcrossref{\sphinxstyleliteralemphasis{\sphinxupquote{Entity}}}}}) \textendash{} A pair of molecular entity identifiers, labels or \sphinxcode{\sphinxupquote{Entity}}
objects.

\end{description}\end{quote}

\end{fulllineitems}

\index{remove\_loops() (pypath.core.network.Network method)@\spxentry{remove\_loops()}\spxextra{pypath.core.network.Network method}}

\begin{fulllineitems}
\phantomsection\label{\detokenize{reference:pypath.core.network.Network.remove_loops}}\pysiglinewithargsret{\sphinxbfcode{\sphinxupquote{remove\_loops}}}{}{}
Removes the loop interactions from the network i.e. the ones with
their two endpoints being the same entity.

\end{fulllineitems}

\index{remove\_node() (pypath.core.network.Network method)@\spxentry{remove\_node()}\spxextra{pypath.core.network.Network method}}

\begin{fulllineitems}
\phantomsection\label{\detokenize{reference:pypath.core.network.Network.remove_node}}\pysiglinewithargsret{\sphinxbfcode{\sphinxupquote{remove\_node}}}{\emph{entity}}{}
Removes a node with all its interactions.
If the removal of the interactions leaves any of the partner nodes
without interactions it will be removed too.
\begin{quote}\begin{description}
\item[{Parameters}] \leavevmode
\sphinxstyleliteralstrong{\sphinxupquote{entity}} (\sphinxstyleliteralemphasis{\sphinxupquote{str}}\sphinxstyleliteralemphasis{\sphinxupquote{,}}{\hyperref[\detokenize{reference:pypath.core.entity.Entity}]{\sphinxcrossref{\sphinxstyleliteralemphasis{\sphinxupquote{Entity}}}}}) \textendash{} A molecular entity identifier, label or \sphinxcode{\sphinxupquote{Entity}} object.

\end{description}\end{quote}

\end{fulllineitems}

\index{remove\_zero\_degree() (pypath.core.network.Network method)@\spxentry{remove\_zero\_degree()}\spxextra{pypath.core.network.Network method}}

\begin{fulllineitems}
\phantomsection\label{\detokenize{reference:pypath.core.network.Network.remove_zero_degree}}\pysiglinewithargsret{\sphinxbfcode{\sphinxupquote{remove\_zero\_degree}}}{}{}
Removes all nodes with no interaction.

\end{fulllineitems}

\index{reset() (pypath.core.network.Network method)@\spxentry{reset()}\spxextra{pypath.core.network.Network method}}

\begin{fulllineitems}
\phantomsection\label{\detokenize{reference:pypath.core.network.Network.reset}}\pysiglinewithargsret{\sphinxbfcode{\sphinxupquote{reset}}}{}{}
Removes network data i.e. creates empty interaction and node
dictionaries.

\end{fulllineitems}

\index{resource\_names (pypath.core.network.Network attribute)@\spxentry{resource\_names}\spxextra{pypath.core.network.Network attribute}}

\begin{fulllineitems}
\phantomsection\label{\detokenize{reference:pypath.core.network.Network.resource_names}}\pysigline{\sphinxbfcode{\sphinxupquote{resource\_names}}}
Returns a set of all resource names.

\end{fulllineitems}

\index{resource\_names\_by\_data\_model() (pypath.core.network.Network method)@\spxentry{resource\_names\_by\_data\_model()}\spxextra{pypath.core.network.Network method}}

\begin{fulllineitems}
\phantomsection\label{\detokenize{reference:pypath.core.network.Network.resource_names_by_data_model}}\pysiglinewithargsret{\sphinxbfcode{\sphinxupquote{resource\_names\_by\_data\_model}}}{}{}
Built-in immutable sequence.

If no argument is given, the constructor returns an empty tuple.
If iterable is specified the tuple is initialized from iterable’s items.

If the argument is a tuple, the return value is the same object.

\end{fulllineitems}

\index{resource\_names\_by\_interaction\_type() (pypath.core.network.Network method)@\spxentry{resource\_names\_by\_interaction\_type()}\spxextra{pypath.core.network.Network method}}

\begin{fulllineitems}
\phantomsection\label{\detokenize{reference:pypath.core.network.Network.resource_names_by_interaction_type}}\pysiglinewithargsret{\sphinxbfcode{\sphinxupquote{resource\_names\_by\_interaction\_type}}}{}{}
Built-in immutable sequence.

If no argument is given, the constructor returns an empty tuple.
If iterable is specified the tuple is initialized from iterable’s items.

If the argument is a tuple, the return value is the same object.

\end{fulllineitems}

\index{resource\_names\_by\_interaction\_type\_and\_data\_model() (pypath.core.network.Network method)@\spxentry{resource\_names\_by\_interaction\_type\_and\_data\_model()}\spxextra{pypath.core.network.Network method}}

\begin{fulllineitems}
\phantomsection\label{\detokenize{reference:pypath.core.network.Network.resource_names_by_interaction_type_and_data_model}}\pysiglinewithargsret{\sphinxbfcode{\sphinxupquote{resource\_names\_by\_interaction\_type\_and\_data\_model}}}{}{}
Built-in immutable sequence.

If no argument is given, the constructor returns an empty tuple.
If iterable is specified the tuple is initialized from iterable’s items.

If the argument is a tuple, the return value is the same object.

\end{fulllineitems}

\index{resource\_names\_by\_interaction\_type\_and\_data\_model\_and\_resource() (pypath.core.network.Network method)@\spxentry{resource\_names\_by\_interaction\_type\_and\_data\_model\_and\_resource()}\spxextra{pypath.core.network.Network method}}

\begin{fulllineitems}
\phantomsection\label{\detokenize{reference:pypath.core.network.Network.resource_names_by_interaction_type_and_data_model_and_resource}}\pysiglinewithargsret{\sphinxbfcode{\sphinxupquote{resource\_names\_by\_interaction\_type\_and\_data\_model\_and\_resource}}}{}{}
Built-in immutable sequence.

If no argument is given, the constructor returns an empty tuple.
If iterable is specified the tuple is initialized from iterable’s items.

If the argument is a tuple, the return value is the same object.

\end{fulllineitems}

\index{resource\_names\_by\_reference() (pypath.core.network.Network method)@\spxentry{resource\_names\_by\_reference()}\spxextra{pypath.core.network.Network method}}

\begin{fulllineitems}
\phantomsection\label{\detokenize{reference:pypath.core.network.Network.resource_names_by_reference}}\pysiglinewithargsret{\sphinxbfcode{\sphinxupquote{resource\_names\_by\_reference}}}{}{}
Built-in immutable sequence.

If no argument is given, the constructor returns an empty tuple.
If iterable is specified the tuple is initialized from iterable’s items.

If the argument is a tuple, the return value is the same object.

\end{fulllineitems}

\index{resource\_names\_by\_resource() (pypath.core.network.Network method)@\spxentry{resource\_names\_by\_resource()}\spxextra{pypath.core.network.Network method}}

\begin{fulllineitems}
\phantomsection\label{\detokenize{reference:pypath.core.network.Network.resource_names_by_resource}}\pysiglinewithargsret{\sphinxbfcode{\sphinxupquote{resource\_names\_by\_resource}}}{}{}
Built-in immutable sequence.

If no argument is given, the constructor returns an empty tuple.
If iterable is specified the tuple is initialized from iterable’s items.

If the argument is a tuple, the return value is the same object.

\end{fulllineitems}

\index{resource\_names\_via\_by\_data\_model() (pypath.core.network.Network method)@\spxentry{resource\_names\_via\_by\_data\_model()}\spxextra{pypath.core.network.Network method}}

\begin{fulllineitems}
\phantomsection\label{\detokenize{reference:pypath.core.network.Network.resource_names_via_by_data_model}}\pysiglinewithargsret{\sphinxbfcode{\sphinxupquote{resource\_names\_via\_by\_data\_model}}}{}{}
Built-in immutable sequence.

If no argument is given, the constructor returns an empty tuple.
If iterable is specified the tuple is initialized from iterable’s items.

If the argument is a tuple, the return value is the same object.

\end{fulllineitems}

\index{resource\_names\_via\_by\_interaction\_type() (pypath.core.network.Network method)@\spxentry{resource\_names\_via\_by\_interaction\_type()}\spxextra{pypath.core.network.Network method}}

\begin{fulllineitems}
\phantomsection\label{\detokenize{reference:pypath.core.network.Network.resource_names_via_by_interaction_type}}\pysiglinewithargsret{\sphinxbfcode{\sphinxupquote{resource\_names\_via\_by\_interaction\_type}}}{}{}
Built-in immutable sequence.

If no argument is given, the constructor returns an empty tuple.
If iterable is specified the tuple is initialized from iterable’s items.

If the argument is a tuple, the return value is the same object.

\end{fulllineitems}

\index{resource\_names\_via\_by\_interaction\_type\_and\_data\_model() (pypath.core.network.Network method)@\spxentry{resource\_names\_via\_by\_interaction\_type\_and\_data\_model()}\spxextra{pypath.core.network.Network method}}

\begin{fulllineitems}
\phantomsection\label{\detokenize{reference:pypath.core.network.Network.resource_names_via_by_interaction_type_and_data_model}}\pysiglinewithargsret{\sphinxbfcode{\sphinxupquote{resource\_names\_via\_by\_interaction\_type\_and\_data\_model}}}{}{}
Built-in immutable sequence.

If no argument is given, the constructor returns an empty tuple.
If iterable is specified the tuple is initialized from iterable’s items.

If the argument is a tuple, the return value is the same object.

\end{fulllineitems}

\index{resource\_names\_via\_by\_interaction\_type\_and\_data\_model\_and\_resource() (pypath.core.network.Network method)@\spxentry{resource\_names\_via\_by\_interaction\_type\_and\_data\_model\_and\_resource()}\spxextra{pypath.core.network.Network method}}

\begin{fulllineitems}
\phantomsection\label{\detokenize{reference:pypath.core.network.Network.resource_names_via_by_interaction_type_and_data_model_and_resource}}\pysiglinewithargsret{\sphinxbfcode{\sphinxupquote{resource\_names\_via\_by\_interaction\_type\_and\_data\_model\_and\_resource}}}{}{}
Built-in immutable sequence.

If no argument is given, the constructor returns an empty tuple.
If iterable is specified the tuple is initialized from iterable’s items.

If the argument is a tuple, the return value is the same object.

\end{fulllineitems}

\index{resource\_names\_via\_by\_reference() (pypath.core.network.Network method)@\spxentry{resource\_names\_via\_by\_reference()}\spxextra{pypath.core.network.Network method}}

\begin{fulllineitems}
\phantomsection\label{\detokenize{reference:pypath.core.network.Network.resource_names_via_by_reference}}\pysiglinewithargsret{\sphinxbfcode{\sphinxupquote{resource\_names\_via\_by\_reference}}}{}{}
Built-in immutable sequence.

If no argument is given, the constructor returns an empty tuple.
If iterable is specified the tuple is initialized from iterable’s items.

If the argument is a tuple, the return value is the same object.

\end{fulllineitems}

\index{resource\_names\_via\_by\_resource() (pypath.core.network.Network method)@\spxentry{resource\_names\_via\_by\_resource()}\spxextra{pypath.core.network.Network method}}

\begin{fulllineitems}
\phantomsection\label{\detokenize{reference:pypath.core.network.Network.resource_names_via_by_resource}}\pysiglinewithargsret{\sphinxbfcode{\sphinxupquote{resource\_names\_via\_by\_resource}}}{}{}
Built-in immutable sequence.

If no argument is given, the constructor returns an empty tuple.
If iterable is specified the tuple is initialized from iterable’s items.

If the argument is a tuple, the return value is the same object.

\end{fulllineitems}

\index{resources (pypath.core.network.Network attribute)@\spxentry{resources}\spxextra{pypath.core.network.Network attribute}}

\begin{fulllineitems}
\phantomsection\label{\detokenize{reference:pypath.core.network.Network.resources}}\pysigline{\sphinxbfcode{\sphinxupquote{resources}}}
Returns a set of all resources.

\end{fulllineitems}

\index{resources\_by\_data\_model() (pypath.core.network.Network method)@\spxentry{resources\_by\_data\_model()}\spxextra{pypath.core.network.Network method}}

\begin{fulllineitems}
\phantomsection\label{\detokenize{reference:pypath.core.network.Network.resources_by_data_model}}\pysiglinewithargsret{\sphinxbfcode{\sphinxupquote{resources\_by\_data\_model}}}{}{}
Built-in immutable sequence.

If no argument is given, the constructor returns an empty tuple.
If iterable is specified the tuple is initialized from iterable’s items.

If the argument is a tuple, the return value is the same object.

\end{fulllineitems}

\index{resources\_by\_interaction\_type() (pypath.core.network.Network method)@\spxentry{resources\_by\_interaction\_type()}\spxextra{pypath.core.network.Network method}}

\begin{fulllineitems}
\phantomsection\label{\detokenize{reference:pypath.core.network.Network.resources_by_interaction_type}}\pysiglinewithargsret{\sphinxbfcode{\sphinxupquote{resources\_by\_interaction\_type}}}{}{}
Built-in immutable sequence.

If no argument is given, the constructor returns an empty tuple.
If iterable is specified the tuple is initialized from iterable’s items.

If the argument is a tuple, the return value is the same object.

\end{fulllineitems}

\index{resources\_by\_interaction\_type\_and\_data\_model() (pypath.core.network.Network method)@\spxentry{resources\_by\_interaction\_type\_and\_data\_model()}\spxextra{pypath.core.network.Network method}}

\begin{fulllineitems}
\phantomsection\label{\detokenize{reference:pypath.core.network.Network.resources_by_interaction_type_and_data_model}}\pysiglinewithargsret{\sphinxbfcode{\sphinxupquote{resources\_by\_interaction\_type\_and\_data\_model}}}{}{}
Built-in immutable sequence.

If no argument is given, the constructor returns an empty tuple.
If iterable is specified the tuple is initialized from iterable’s items.

If the argument is a tuple, the return value is the same object.

\end{fulllineitems}

\index{resources\_by\_interaction\_type\_and\_data\_model\_and\_resource() (pypath.core.network.Network method)@\spxentry{resources\_by\_interaction\_type\_and\_data\_model\_and\_resource()}\spxextra{pypath.core.network.Network method}}

\begin{fulllineitems}
\phantomsection\label{\detokenize{reference:pypath.core.network.Network.resources_by_interaction_type_and_data_model_and_resource}}\pysiglinewithargsret{\sphinxbfcode{\sphinxupquote{resources\_by\_interaction\_type\_and\_data\_model\_and\_resource}}}{}{}
Built-in immutable sequence.

If no argument is given, the constructor returns an empty tuple.
If iterable is specified the tuple is initialized from iterable’s items.

If the argument is a tuple, the return value is the same object.

\end{fulllineitems}

\index{resources\_by\_reference() (pypath.core.network.Network method)@\spxentry{resources\_by\_reference()}\spxextra{pypath.core.network.Network method}}

\begin{fulllineitems}
\phantomsection\label{\detokenize{reference:pypath.core.network.Network.resources_by_reference}}\pysiglinewithargsret{\sphinxbfcode{\sphinxupquote{resources\_by\_reference}}}{}{}
Built-in immutable sequence.

If no argument is given, the constructor returns an empty tuple.
If iterable is specified the tuple is initialized from iterable’s items.

If the argument is a tuple, the return value is the same object.

\end{fulllineitems}

\index{resources\_by\_resource() (pypath.core.network.Network method)@\spxentry{resources\_by\_resource()}\spxextra{pypath.core.network.Network method}}

\begin{fulllineitems}
\phantomsection\label{\detokenize{reference:pypath.core.network.Network.resources_by_resource}}\pysiglinewithargsret{\sphinxbfcode{\sphinxupquote{resources\_by\_resource}}}{}{}
Built-in immutable sequence.

If no argument is given, the constructor returns an empty tuple.
If iterable is specified the tuple is initialized from iterable’s items.

If the argument is a tuple, the return value is the same object.

\end{fulllineitems}

\index{resources\_via\_by\_data\_model() (pypath.core.network.Network method)@\spxentry{resources\_via\_by\_data\_model()}\spxextra{pypath.core.network.Network method}}

\begin{fulllineitems}
\phantomsection\label{\detokenize{reference:pypath.core.network.Network.resources_via_by_data_model}}\pysiglinewithargsret{\sphinxbfcode{\sphinxupquote{resources\_via\_by\_data\_model}}}{}{}
Built-in immutable sequence.

If no argument is given, the constructor returns an empty tuple.
If iterable is specified the tuple is initialized from iterable’s items.

If the argument is a tuple, the return value is the same object.

\end{fulllineitems}

\index{resources\_via\_by\_interaction\_type() (pypath.core.network.Network method)@\spxentry{resources\_via\_by\_interaction\_type()}\spxextra{pypath.core.network.Network method}}

\begin{fulllineitems}
\phantomsection\label{\detokenize{reference:pypath.core.network.Network.resources_via_by_interaction_type}}\pysiglinewithargsret{\sphinxbfcode{\sphinxupquote{resources\_via\_by\_interaction\_type}}}{}{}
Built-in immutable sequence.

If no argument is given, the constructor returns an empty tuple.
If iterable is specified the tuple is initialized from iterable’s items.

If the argument is a tuple, the return value is the same object.

\end{fulllineitems}

\index{resources\_via\_by\_interaction\_type\_and\_data\_model() (pypath.core.network.Network method)@\spxentry{resources\_via\_by\_interaction\_type\_and\_data\_model()}\spxextra{pypath.core.network.Network method}}

\begin{fulllineitems}
\phantomsection\label{\detokenize{reference:pypath.core.network.Network.resources_via_by_interaction_type_and_data_model}}\pysiglinewithargsret{\sphinxbfcode{\sphinxupquote{resources\_via\_by\_interaction\_type\_and\_data\_model}}}{}{}
Built-in immutable sequence.

If no argument is given, the constructor returns an empty tuple.
If iterable is specified the tuple is initialized from iterable’s items.

If the argument is a tuple, the return value is the same object.

\end{fulllineitems}

\index{resources\_via\_by\_interaction\_type\_and\_data\_model\_and\_resource() (pypath.core.network.Network method)@\spxentry{resources\_via\_by\_interaction\_type\_and\_data\_model\_and\_resource()}\spxextra{pypath.core.network.Network method}}

\begin{fulllineitems}
\phantomsection\label{\detokenize{reference:pypath.core.network.Network.resources_via_by_interaction_type_and_data_model_and_resource}}\pysiglinewithargsret{\sphinxbfcode{\sphinxupquote{resources\_via\_by\_interaction\_type\_and\_data\_model\_and\_resource}}}{}{}
Built-in immutable sequence.

If no argument is given, the constructor returns an empty tuple.
If iterable is specified the tuple is initialized from iterable’s items.

If the argument is a tuple, the return value is the same object.

\end{fulllineitems}

\index{resources\_via\_by\_reference() (pypath.core.network.Network method)@\spxentry{resources\_via\_by\_reference()}\spxextra{pypath.core.network.Network method}}

\begin{fulllineitems}
\phantomsection\label{\detokenize{reference:pypath.core.network.Network.resources_via_by_reference}}\pysiglinewithargsret{\sphinxbfcode{\sphinxupquote{resources\_via\_by\_reference}}}{}{}
Built-in immutable sequence.

If no argument is given, the constructor returns an empty tuple.
If iterable is specified the tuple is initialized from iterable’s items.

If the argument is a tuple, the return value is the same object.

\end{fulllineitems}

\index{resources\_via\_by\_resource() (pypath.core.network.Network method)@\spxentry{resources\_via\_by\_resource()}\spxextra{pypath.core.network.Network method}}

\begin{fulllineitems}
\phantomsection\label{\detokenize{reference:pypath.core.network.Network.resources_via_by_resource}}\pysiglinewithargsret{\sphinxbfcode{\sphinxupquote{resources\_via\_by\_resource}}}{}{}
Built-in immutable sequence.

If no argument is given, the constructor returns an empty tuple.
If iterable is specified the tuple is initialized from iterable’s items.

If the argument is a tuple, the return value is the same object.

\end{fulllineitems}

\index{save\_to\_pickle() (pypath.core.network.Network method)@\spxentry{save\_to\_pickle()}\spxextra{pypath.core.network.Network method}}

\begin{fulllineitems}
\phantomsection\label{\detokenize{reference:pypath.core.network.Network.save_to_pickle}}\pysiglinewithargsret{\sphinxbfcode{\sphinxupquote{save\_to\_pickle}}}{\emph{pickle\_file}}{}
Saves the network to a pickle file.
\begin{quote}\begin{description}
\item[{Parameters}] \leavevmode
\sphinxstyleliteralstrong{\sphinxupquote{pickle\_file}} (\sphinxstyleliteralemphasis{\sphinxupquote{str}}) \textendash{} Path to the pickle file.

\end{description}\end{quote}

\end{fulllineitems}

\index{small\_molecule\_identifiers\_by\_data\_model() (pypath.core.network.Network method)@\spxentry{small\_molecule\_identifiers\_by\_data\_model()}\spxextra{pypath.core.network.Network method}}

\begin{fulllineitems}
\phantomsection\label{\detokenize{reference:pypath.core.network.Network.small_molecule_identifiers_by_data_model}}\pysiglinewithargsret{\sphinxbfcode{\sphinxupquote{small\_molecule\_identifiers\_by\_data\_model}}}{}{}
Built-in immutable sequence.

If no argument is given, the constructor returns an empty tuple.
If iterable is specified the tuple is initialized from iterable’s items.

If the argument is a tuple, the return value is the same object.

\end{fulllineitems}

\index{small\_molecule\_identifiers\_by\_interaction\_type() (pypath.core.network.Network method)@\spxentry{small\_molecule\_identifiers\_by\_interaction\_type()}\spxextra{pypath.core.network.Network method}}

\begin{fulllineitems}
\phantomsection\label{\detokenize{reference:pypath.core.network.Network.small_molecule_identifiers_by_interaction_type}}\pysiglinewithargsret{\sphinxbfcode{\sphinxupquote{small\_molecule\_identifiers\_by\_interaction\_type}}}{}{}
Built-in immutable sequence.

If no argument is given, the constructor returns an empty tuple.
If iterable is specified the tuple is initialized from iterable’s items.

If the argument is a tuple, the return value is the same object.

\end{fulllineitems}

\index{small\_molecule\_identifiers\_by\_interaction\_type\_and\_data\_model() (pypath.core.network.Network method)@\spxentry{small\_molecule\_identifiers\_by\_interaction\_type\_and\_data\_model()}\spxextra{pypath.core.network.Network method}}

\begin{fulllineitems}
\phantomsection\label{\detokenize{reference:pypath.core.network.Network.small_molecule_identifiers_by_interaction_type_and_data_model}}\pysiglinewithargsret{\sphinxbfcode{\sphinxupquote{small\_molecule\_identifiers\_by\_interaction\_type\_and\_data\_model}}}{}{}
Built-in immutable sequence.

If no argument is given, the constructor returns an empty tuple.
If iterable is specified the tuple is initialized from iterable’s items.

If the argument is a tuple, the return value is the same object.

\end{fulllineitems}

\index{small\_molecule\_identifiers\_by\_interaction\_type\_and\_data\_model\_and\_resource() (pypath.core.network.Network method)@\spxentry{small\_molecule\_identifiers\_by\_interaction\_type\_and\_data\_model\_and\_resource()}\spxextra{pypath.core.network.Network method}}

\begin{fulllineitems}
\phantomsection\label{\detokenize{reference:pypath.core.network.Network.small_molecule_identifiers_by_interaction_type_and_data_model_and_resource}}\pysiglinewithargsret{\sphinxbfcode{\sphinxupquote{small\_molecule\_identifiers\_by\_interaction\_type\_and\_data\_model\_and\_resource}}}{}{}
Built-in immutable sequence.

If no argument is given, the constructor returns an empty tuple.
If iterable is specified the tuple is initialized from iterable’s items.

If the argument is a tuple, the return value is the same object.

\end{fulllineitems}

\index{small\_molecule\_identifiers\_by\_reference() (pypath.core.network.Network method)@\spxentry{small\_molecule\_identifiers\_by\_reference()}\spxextra{pypath.core.network.Network method}}

\begin{fulllineitems}
\phantomsection\label{\detokenize{reference:pypath.core.network.Network.small_molecule_identifiers_by_reference}}\pysiglinewithargsret{\sphinxbfcode{\sphinxupquote{small\_molecule\_identifiers\_by\_reference}}}{}{}
Built-in immutable sequence.

If no argument is given, the constructor returns an empty tuple.
If iterable is specified the tuple is initialized from iterable’s items.

If the argument is a tuple, the return value is the same object.

\end{fulllineitems}

\index{small\_molecule\_identifiers\_by\_resource() (pypath.core.network.Network method)@\spxentry{small\_molecule\_identifiers\_by\_resource()}\spxextra{pypath.core.network.Network method}}

\begin{fulllineitems}
\phantomsection\label{\detokenize{reference:pypath.core.network.Network.small_molecule_identifiers_by_resource}}\pysiglinewithargsret{\sphinxbfcode{\sphinxupquote{small\_molecule\_identifiers\_by\_resource}}}{}{}
Built-in immutable sequence.

If no argument is given, the constructor returns an empty tuple.
If iterable is specified the tuple is initialized from iterable’s items.

If the argument is a tuple, the return value is the same object.

\end{fulllineitems}

\index{small\_molecule\_labels\_by\_data\_model() (pypath.core.network.Network method)@\spxentry{small\_molecule\_labels\_by\_data\_model()}\spxextra{pypath.core.network.Network method}}

\begin{fulllineitems}
\phantomsection\label{\detokenize{reference:pypath.core.network.Network.small_molecule_labels_by_data_model}}\pysiglinewithargsret{\sphinxbfcode{\sphinxupquote{small\_molecule\_labels\_by\_data\_model}}}{}{}
Built-in immutable sequence.

If no argument is given, the constructor returns an empty tuple.
If iterable is specified the tuple is initialized from iterable’s items.

If the argument is a tuple, the return value is the same object.

\end{fulllineitems}

\index{small\_molecule\_labels\_by\_interaction\_type() (pypath.core.network.Network method)@\spxentry{small\_molecule\_labels\_by\_interaction\_type()}\spxextra{pypath.core.network.Network method}}

\begin{fulllineitems}
\phantomsection\label{\detokenize{reference:pypath.core.network.Network.small_molecule_labels_by_interaction_type}}\pysiglinewithargsret{\sphinxbfcode{\sphinxupquote{small\_molecule\_labels\_by\_interaction\_type}}}{}{}
Built-in immutable sequence.

If no argument is given, the constructor returns an empty tuple.
If iterable is specified the tuple is initialized from iterable’s items.

If the argument is a tuple, the return value is the same object.

\end{fulllineitems}

\index{small\_molecule\_labels\_by\_interaction\_type\_and\_data\_model() (pypath.core.network.Network method)@\spxentry{small\_molecule\_labels\_by\_interaction\_type\_and\_data\_model()}\spxextra{pypath.core.network.Network method}}

\begin{fulllineitems}
\phantomsection\label{\detokenize{reference:pypath.core.network.Network.small_molecule_labels_by_interaction_type_and_data_model}}\pysiglinewithargsret{\sphinxbfcode{\sphinxupquote{small\_molecule\_labels\_by\_interaction\_type\_and\_data\_model}}}{}{}
Built-in immutable sequence.

If no argument is given, the constructor returns an empty tuple.
If iterable is specified the tuple is initialized from iterable’s items.

If the argument is a tuple, the return value is the same object.

\end{fulllineitems}

\index{small\_molecule\_labels\_by\_interaction\_type\_and\_data\_model\_and\_resource() (pypath.core.network.Network method)@\spxentry{small\_molecule\_labels\_by\_interaction\_type\_and\_data\_model\_and\_resource()}\spxextra{pypath.core.network.Network method}}

\begin{fulllineitems}
\phantomsection\label{\detokenize{reference:pypath.core.network.Network.small_molecule_labels_by_interaction_type_and_data_model_and_resource}}\pysiglinewithargsret{\sphinxbfcode{\sphinxupquote{small\_molecule\_labels\_by\_interaction\_type\_and\_data\_model\_and\_resource}}}{}{}
Built-in immutable sequence.

If no argument is given, the constructor returns an empty tuple.
If iterable is specified the tuple is initialized from iterable’s items.

If the argument is a tuple, the return value is the same object.

\end{fulllineitems}

\index{small\_molecule\_labels\_by\_reference() (pypath.core.network.Network method)@\spxentry{small\_molecule\_labels\_by\_reference()}\spxextra{pypath.core.network.Network method}}

\begin{fulllineitems}
\phantomsection\label{\detokenize{reference:pypath.core.network.Network.small_molecule_labels_by_reference}}\pysiglinewithargsret{\sphinxbfcode{\sphinxupquote{small\_molecule\_labels\_by\_reference}}}{}{}
Built-in immutable sequence.

If no argument is given, the constructor returns an empty tuple.
If iterable is specified the tuple is initialized from iterable’s items.

If the argument is a tuple, the return value is the same object.

\end{fulllineitems}

\index{small\_molecule\_labels\_by\_resource() (pypath.core.network.Network method)@\spxentry{small\_molecule\_labels\_by\_resource()}\spxextra{pypath.core.network.Network method}}

\begin{fulllineitems}
\phantomsection\label{\detokenize{reference:pypath.core.network.Network.small_molecule_labels_by_resource}}\pysiglinewithargsret{\sphinxbfcode{\sphinxupquote{small\_molecule\_labels\_by\_resource}}}{}{}
Built-in immutable sequence.

If no argument is given, the constructor returns an empty tuple.
If iterable is specified the tuple is initialized from iterable’s items.

If the argument is a tuple, the return value is the same object.

\end{fulllineitems}

\index{small\_molecules\_by\_data\_model() (pypath.core.network.Network method)@\spxentry{small\_molecules\_by\_data\_model()}\spxextra{pypath.core.network.Network method}}

\begin{fulllineitems}
\phantomsection\label{\detokenize{reference:pypath.core.network.Network.small_molecules_by_data_model}}\pysiglinewithargsret{\sphinxbfcode{\sphinxupquote{small\_molecules\_by\_data\_model}}}{}{}
Built-in immutable sequence.

If no argument is given, the constructor returns an empty tuple.
If iterable is specified the tuple is initialized from iterable’s items.

If the argument is a tuple, the return value is the same object.

\end{fulllineitems}

\index{small\_molecules\_by\_interaction\_type() (pypath.core.network.Network method)@\spxentry{small\_molecules\_by\_interaction\_type()}\spxextra{pypath.core.network.Network method}}

\begin{fulllineitems}
\phantomsection\label{\detokenize{reference:pypath.core.network.Network.small_molecules_by_interaction_type}}\pysiglinewithargsret{\sphinxbfcode{\sphinxupquote{small\_molecules\_by\_interaction\_type}}}{}{}
Built-in immutable sequence.

If no argument is given, the constructor returns an empty tuple.
If iterable is specified the tuple is initialized from iterable’s items.

If the argument is a tuple, the return value is the same object.

\end{fulllineitems}

\index{small\_molecules\_by\_interaction\_type\_and\_data\_model() (pypath.core.network.Network method)@\spxentry{small\_molecules\_by\_interaction\_type\_and\_data\_model()}\spxextra{pypath.core.network.Network method}}

\begin{fulllineitems}
\phantomsection\label{\detokenize{reference:pypath.core.network.Network.small_molecules_by_interaction_type_and_data_model}}\pysiglinewithargsret{\sphinxbfcode{\sphinxupquote{small\_molecules\_by\_interaction\_type\_and\_data\_model}}}{}{}
Built-in immutable sequence.

If no argument is given, the constructor returns an empty tuple.
If iterable is specified the tuple is initialized from iterable’s items.

If the argument is a tuple, the return value is the same object.

\end{fulllineitems}

\index{small\_molecules\_by\_interaction\_type\_and\_data\_model\_and\_resource() (pypath.core.network.Network method)@\spxentry{small\_molecules\_by\_interaction\_type\_and\_data\_model\_and\_resource()}\spxextra{pypath.core.network.Network method}}

\begin{fulllineitems}
\phantomsection\label{\detokenize{reference:pypath.core.network.Network.small_molecules_by_interaction_type_and_data_model_and_resource}}\pysiglinewithargsret{\sphinxbfcode{\sphinxupquote{small\_molecules\_by\_interaction\_type\_and\_data\_model\_and\_resource}}}{}{}
Built-in immutable sequence.

If no argument is given, the constructor returns an empty tuple.
If iterable is specified the tuple is initialized from iterable’s items.

If the argument is a tuple, the return value is the same object.

\end{fulllineitems}

\index{small\_molecules\_by\_reference() (pypath.core.network.Network method)@\spxentry{small\_molecules\_by\_reference()}\spxextra{pypath.core.network.Network method}}

\begin{fulllineitems}
\phantomsection\label{\detokenize{reference:pypath.core.network.Network.small_molecules_by_reference}}\pysiglinewithargsret{\sphinxbfcode{\sphinxupquote{small\_molecules\_by\_reference}}}{}{}
Built-in immutable sequence.

If no argument is given, the constructor returns an empty tuple.
If iterable is specified the tuple is initialized from iterable’s items.

If the argument is a tuple, the return value is the same object.

\end{fulllineitems}

\index{small\_molecules\_by\_resource() (pypath.core.network.Network method)@\spxentry{small\_molecules\_by\_resource()}\spxextra{pypath.core.network.Network method}}

\begin{fulllineitems}
\phantomsection\label{\detokenize{reference:pypath.core.network.Network.small_molecules_by_resource}}\pysiglinewithargsret{\sphinxbfcode{\sphinxupquote{small\_molecules\_by\_resource}}}{}{}
Built-in immutable sequence.

If no argument is given, the constructor returns an empty tuple.
If iterable is specified the tuple is initialized from iterable’s items.

If the argument is a tuple, the return value is the same object.

\end{fulllineitems}

\index{summaries\_tab() (pypath.core.network.Network method)@\spxentry{summaries\_tab()}\spxextra{pypath.core.network.Network method}}

\begin{fulllineitems}
\phantomsection\label{\detokenize{reference:pypath.core.network.Network.summaries_tab}}\pysiglinewithargsret{\sphinxbfcode{\sphinxupquote{summaries\_tab}}}{\emph{outfile=None}, \emph{return\_table=False}, \emph{label\_type=1}}{}
Creates a table from resource vs. entity counts and optionally
writes it to \sphinxcode{\sphinxupquote{outfile}} and returns it.

\end{fulllineitems}

\index{suppressed\_by() (pypath.core.network.Network method)@\spxentry{suppressed\_by()}\spxextra{pypath.core.network.Network method}}

\begin{fulllineitems}
\phantomsection\label{\detokenize{reference:pypath.core.network.Network.suppressed_by}}\pysiglinewithargsret{\sphinxbfcode{\sphinxupquote{suppressed\_by}}}{}{}~\begin{quote}\begin{description}
\item[{Parameters}] \leavevmode\begin{itemize}
\item {} 
\sphinxstyleliteralstrong{\sphinxupquote{entity}} (\sphinxstyleliteralemphasis{\sphinxupquote{str}}\sphinxstyleliteralemphasis{\sphinxupquote{,}}{\hyperref[\detokenize{reference:pypath.core.entity.Entity}]{\sphinxcrossref{\sphinxstyleliteralemphasis{\sphinxupquote{Entity}}}}}\sphinxstyleliteralemphasis{\sphinxupquote{,}}\sphinxstyleliteralemphasis{\sphinxupquote{list}}\sphinxstyleliteralemphasis{\sphinxupquote{,}}\sphinxstyleliteralemphasis{\sphinxupquote{set}}\sphinxstyleliteralemphasis{\sphinxupquote{,}}\sphinxstyleliteralemphasis{\sphinxupquote{tuple}}\sphinxstyleliteralemphasis{\sphinxupquote{,}}\sphinxstyleliteralemphasis{\sphinxupquote{EntityList}}) \textendash{} An identifier or label of a molecular entity or an
\sphinxcode{\sphinxupquote{Entity}} object. Alternatively an iterator with the
elements of any of the types valid for a single entity argument,
e.g. a list of gene symbols.

\item {} 
\sphinxstyleliteralstrong{\sphinxupquote{mode}} (\sphinxstyleliteralemphasis{\sphinxupquote{str}}) \textendash{} Mode of counting the interactions: \sphinxtitleref{IN}, \sphinxtitleref{OUT} or \sphinxtitleref{ALL} , whether
to consider incoming, outgoing or all edges, respectively,
respective to the \sphinxtitleref{node defined in {}`entity{}`}.

\end{itemize}

\item[{Returns}] \leavevmode
\sphinxcode{\sphinxupquote{EntityList}} object containing the partners having
interactions to the queried node(s) matching all the criteria.
If \sphinxcode{\sphinxupquote{entity}} doesn’t present in the network the returned
\sphinxcode{\sphinxupquote{EntityList}} will be empty just like if no interaction matches
the criteria.

\end{description}\end{quote}

\end{fulllineitems}

\index{suppresses() (pypath.core.network.Network method)@\spxentry{suppresses()}\spxextra{pypath.core.network.Network method}}

\begin{fulllineitems}
\phantomsection\label{\detokenize{reference:pypath.core.network.Network.suppresses}}\pysiglinewithargsret{\sphinxbfcode{\sphinxupquote{suppresses}}}{}{}~\begin{quote}\begin{description}
\item[{Parameters}] \leavevmode\begin{itemize}
\item {} 
\sphinxstyleliteralstrong{\sphinxupquote{entity}} (\sphinxstyleliteralemphasis{\sphinxupquote{str}}\sphinxstyleliteralemphasis{\sphinxupquote{,}}{\hyperref[\detokenize{reference:pypath.core.entity.Entity}]{\sphinxcrossref{\sphinxstyleliteralemphasis{\sphinxupquote{Entity}}}}}\sphinxstyleliteralemphasis{\sphinxupquote{,}}\sphinxstyleliteralemphasis{\sphinxupquote{list}}\sphinxstyleliteralemphasis{\sphinxupquote{,}}\sphinxstyleliteralemphasis{\sphinxupquote{set}}\sphinxstyleliteralemphasis{\sphinxupquote{,}}\sphinxstyleliteralemphasis{\sphinxupquote{tuple}}\sphinxstyleliteralemphasis{\sphinxupquote{,}}\sphinxstyleliteralemphasis{\sphinxupquote{EntityList}}) \textendash{} An identifier or label of a molecular entity or an
\sphinxcode{\sphinxupquote{Entity}} object. Alternatively an iterator with the
elements of any of the types valid for a single entity argument,
e.g. a list of gene symbols.

\item {} 
\sphinxstyleliteralstrong{\sphinxupquote{mode}} (\sphinxstyleliteralemphasis{\sphinxupquote{str}}) \textendash{} Mode of counting the interactions: \sphinxtitleref{IN}, \sphinxtitleref{OUT} or \sphinxtitleref{ALL} , whether
to consider incoming, outgoing or all edges, respectively,
respective to the \sphinxtitleref{node defined in {}`entity{}`}.

\end{itemize}

\item[{Returns}] \leavevmode
\sphinxcode{\sphinxupquote{EntityList}} object containing the partners having
interactions to the queried node(s) matching all the criteria.
If \sphinxcode{\sphinxupquote{entity}} doesn’t present in the network the returned
\sphinxcode{\sphinxupquote{EntityList}} will be empty just like if no interaction matches
the criteria.

\end{description}\end{quote}

\end{fulllineitems}

\index{to\_igraph() (pypath.core.network.Network method)@\spxentry{to\_igraph()}\spxextra{pypath.core.network.Network method}}

\begin{fulllineitems}
\phantomsection\label{\detokenize{reference:pypath.core.network.Network.to_igraph}}\pysiglinewithargsret{\sphinxbfcode{\sphinxupquote{to\_igraph}}}{}{}
Converts the network to the legacy \sphinxcode{\sphinxupquote{igraph.Graph}} based \sphinxcode{\sphinxupquote{PyPath}}
object.

\end{fulllineitems}

\index{transcription() (pypath.core.network.Network class method)@\spxentry{transcription()}\spxextra{pypath.core.network.Network class method}}

\begin{fulllineitems}
\phantomsection\label{\detokenize{reference:pypath.core.network.Network.transcription}}\pysiglinewithargsret{\sphinxbfcode{\sphinxupquote{classmethod }}\sphinxbfcode{\sphinxupquote{transcription}}}{\emph{dorothea=True}, \emph{original\_resources=True}, \emph{dorothea\_levels=None}, \emph{exclude=None}, \emph{reread=False}, \emph{redownload=False}, \emph{make\_df=False}, \emph{ncbi\_tax\_id=9606}, \emph{**kwargs}}{}
Initializes a new \sphinxcode{\sphinxupquote{Network}} object with loading a transcriptional
regulation network from all databases by default.

{\color{red}\bfseries{}**}kwargs: passed to \sphinxcode{\sphinxupquote{Network.\_\_init\_\_}}.

\end{fulllineitems}

\index{transcriptionally\_activated\_by() (pypath.core.network.Network method)@\spxentry{transcriptionally\_activated\_by()}\spxextra{pypath.core.network.Network method}}

\begin{fulllineitems}
\phantomsection\label{\detokenize{reference:pypath.core.network.Network.transcriptionally_activated_by}}\pysiglinewithargsret{\sphinxbfcode{\sphinxupquote{transcriptionally\_activated\_by}}}{}{}~\begin{quote}\begin{description}
\item[{Parameters}] \leavevmode\begin{itemize}
\item {} 
\sphinxstyleliteralstrong{\sphinxupquote{entity}} (\sphinxstyleliteralemphasis{\sphinxupquote{str}}\sphinxstyleliteralemphasis{\sphinxupquote{,}}{\hyperref[\detokenize{reference:pypath.core.entity.Entity}]{\sphinxcrossref{\sphinxstyleliteralemphasis{\sphinxupquote{Entity}}}}}\sphinxstyleliteralemphasis{\sphinxupquote{,}}\sphinxstyleliteralemphasis{\sphinxupquote{list}}\sphinxstyleliteralemphasis{\sphinxupquote{,}}\sphinxstyleliteralemphasis{\sphinxupquote{set}}\sphinxstyleliteralemphasis{\sphinxupquote{,}}\sphinxstyleliteralemphasis{\sphinxupquote{tuple}}\sphinxstyleliteralemphasis{\sphinxupquote{,}}\sphinxstyleliteralemphasis{\sphinxupquote{EntityList}}) \textendash{} An identifier or label of a molecular entity or an
\sphinxcode{\sphinxupquote{Entity}} object. Alternatively an iterator with the
elements of any of the types valid for a single entity argument,
e.g. a list of gene symbols.

\item {} 
\sphinxstyleliteralstrong{\sphinxupquote{mode}} (\sphinxstyleliteralemphasis{\sphinxupquote{str}}) \textendash{} Mode of counting the interactions: \sphinxtitleref{IN}, \sphinxtitleref{OUT} or \sphinxtitleref{ALL} , whether
to consider incoming, outgoing or all edges, respectively,
respective to the \sphinxtitleref{node defined in {}`entity{}`}.

\end{itemize}

\item[{Returns}] \leavevmode
\sphinxcode{\sphinxupquote{EntityList}} object containing the partners having
interactions to the queried node(s) matching all the criteria.
If \sphinxcode{\sphinxupquote{entity}} doesn’t present in the network the returned
\sphinxcode{\sphinxupquote{EntityList}} will be empty just like if no interaction matches
the criteria.

\end{description}\end{quote}

\end{fulllineitems}

\index{transcriptionally\_activates() (pypath.core.network.Network method)@\spxentry{transcriptionally\_activates()}\spxextra{pypath.core.network.Network method}}

\begin{fulllineitems}
\phantomsection\label{\detokenize{reference:pypath.core.network.Network.transcriptionally_activates}}\pysiglinewithargsret{\sphinxbfcode{\sphinxupquote{transcriptionally\_activates}}}{}{}~\begin{quote}\begin{description}
\item[{Parameters}] \leavevmode\begin{itemize}
\item {} 
\sphinxstyleliteralstrong{\sphinxupquote{entity}} (\sphinxstyleliteralemphasis{\sphinxupquote{str}}\sphinxstyleliteralemphasis{\sphinxupquote{,}}{\hyperref[\detokenize{reference:pypath.core.entity.Entity}]{\sphinxcrossref{\sphinxstyleliteralemphasis{\sphinxupquote{Entity}}}}}\sphinxstyleliteralemphasis{\sphinxupquote{,}}\sphinxstyleliteralemphasis{\sphinxupquote{list}}\sphinxstyleliteralemphasis{\sphinxupquote{,}}\sphinxstyleliteralemphasis{\sphinxupquote{set}}\sphinxstyleliteralemphasis{\sphinxupquote{,}}\sphinxstyleliteralemphasis{\sphinxupquote{tuple}}\sphinxstyleliteralemphasis{\sphinxupquote{,}}\sphinxstyleliteralemphasis{\sphinxupquote{EntityList}}) \textendash{} An identifier or label of a molecular entity or an
\sphinxcode{\sphinxupquote{Entity}} object. Alternatively an iterator with the
elements of any of the types valid for a single entity argument,
e.g. a list of gene symbols.

\item {} 
\sphinxstyleliteralstrong{\sphinxupquote{mode}} (\sphinxstyleliteralemphasis{\sphinxupquote{str}}) \textendash{} Mode of counting the interactions: \sphinxtitleref{IN}, \sphinxtitleref{OUT} or \sphinxtitleref{ALL} , whether
to consider incoming, outgoing or all edges, respectively,
respective to the \sphinxtitleref{node defined in {}`entity{}`}.

\end{itemize}

\item[{Returns}] \leavevmode
\sphinxcode{\sphinxupquote{EntityList}} object containing the partners having
interactions to the queried node(s) matching all the criteria.
If \sphinxcode{\sphinxupquote{entity}} doesn’t present in the network the returned
\sphinxcode{\sphinxupquote{EntityList}} will be empty just like if no interaction matches
the criteria.

\end{description}\end{quote}

\end{fulllineitems}

\index{transcriptionally\_regulated\_by() (pypath.core.network.Network method)@\spxentry{transcriptionally\_regulated\_by()}\spxextra{pypath.core.network.Network method}}

\begin{fulllineitems}
\phantomsection\label{\detokenize{reference:pypath.core.network.Network.transcriptionally_regulated_by}}\pysiglinewithargsret{\sphinxbfcode{\sphinxupquote{transcriptionally\_regulated\_by}}}{}{}~\begin{quote}\begin{description}
\item[{Parameters}] \leavevmode\begin{itemize}
\item {} 
\sphinxstyleliteralstrong{\sphinxupquote{entity}} (\sphinxstyleliteralemphasis{\sphinxupquote{str}}\sphinxstyleliteralemphasis{\sphinxupquote{,}}{\hyperref[\detokenize{reference:pypath.core.entity.Entity}]{\sphinxcrossref{\sphinxstyleliteralemphasis{\sphinxupquote{Entity}}}}}\sphinxstyleliteralemphasis{\sphinxupquote{,}}\sphinxstyleliteralemphasis{\sphinxupquote{list}}\sphinxstyleliteralemphasis{\sphinxupquote{,}}\sphinxstyleliteralemphasis{\sphinxupquote{set}}\sphinxstyleliteralemphasis{\sphinxupquote{,}}\sphinxstyleliteralemphasis{\sphinxupquote{tuple}}\sphinxstyleliteralemphasis{\sphinxupquote{,}}\sphinxstyleliteralemphasis{\sphinxupquote{EntityList}}) \textendash{} An identifier or label of a molecular entity or an
\sphinxcode{\sphinxupquote{Entity}} object. Alternatively an iterator with the
elements of any of the types valid for a single entity argument,
e.g. a list of gene symbols.

\item {} 
\sphinxstyleliteralstrong{\sphinxupquote{mode}} (\sphinxstyleliteralemphasis{\sphinxupquote{str}}) \textendash{} Mode of counting the interactions: \sphinxtitleref{IN}, \sphinxtitleref{OUT} or \sphinxtitleref{ALL} , whether
to consider incoming, outgoing or all edges, respectively,
respective to the \sphinxtitleref{node defined in {}`entity{}`}.

\end{itemize}

\item[{Returns}] \leavevmode
\sphinxcode{\sphinxupquote{EntityList}} object containing the partners having
interactions to the queried node(s) matching all the criteria.
If \sphinxcode{\sphinxupquote{entity}} doesn’t present in the network the returned
\sphinxcode{\sphinxupquote{EntityList}} will be empty just like if no interaction matches
the criteria.

\end{description}\end{quote}

\end{fulllineitems}

\index{transcriptionally\_regulates() (pypath.core.network.Network method)@\spxentry{transcriptionally\_regulates()}\spxextra{pypath.core.network.Network method}}

\begin{fulllineitems}
\phantomsection\label{\detokenize{reference:pypath.core.network.Network.transcriptionally_regulates}}\pysiglinewithargsret{\sphinxbfcode{\sphinxupquote{transcriptionally\_regulates}}}{}{}~\begin{quote}\begin{description}
\item[{Parameters}] \leavevmode\begin{itemize}
\item {} 
\sphinxstyleliteralstrong{\sphinxupquote{entity}} (\sphinxstyleliteralemphasis{\sphinxupquote{str}}\sphinxstyleliteralemphasis{\sphinxupquote{,}}{\hyperref[\detokenize{reference:pypath.core.entity.Entity}]{\sphinxcrossref{\sphinxstyleliteralemphasis{\sphinxupquote{Entity}}}}}\sphinxstyleliteralemphasis{\sphinxupquote{,}}\sphinxstyleliteralemphasis{\sphinxupquote{list}}\sphinxstyleliteralemphasis{\sphinxupquote{,}}\sphinxstyleliteralemphasis{\sphinxupquote{set}}\sphinxstyleliteralemphasis{\sphinxupquote{,}}\sphinxstyleliteralemphasis{\sphinxupquote{tuple}}\sphinxstyleliteralemphasis{\sphinxupquote{,}}\sphinxstyleliteralemphasis{\sphinxupquote{EntityList}}) \textendash{} An identifier or label of a molecular entity or an
\sphinxcode{\sphinxupquote{Entity}} object. Alternatively an iterator with the
elements of any of the types valid for a single entity argument,
e.g. a list of gene symbols.

\item {} 
\sphinxstyleliteralstrong{\sphinxupquote{mode}} (\sphinxstyleliteralemphasis{\sphinxupquote{str}}) \textendash{} Mode of counting the interactions: \sphinxtitleref{IN}, \sphinxtitleref{OUT} or \sphinxtitleref{ALL} , whether
to consider incoming, outgoing or all edges, respectively,
respective to the \sphinxtitleref{node defined in {}`entity{}`}.

\end{itemize}

\item[{Returns}] \leavevmode
\sphinxcode{\sphinxupquote{EntityList}} object containing the partners having
interactions to the queried node(s) matching all the criteria.
If \sphinxcode{\sphinxupquote{entity}} doesn’t present in the network the returned
\sphinxcode{\sphinxupquote{EntityList}} will be empty just like if no interaction matches
the criteria.

\end{description}\end{quote}

\end{fulllineitems}

\index{transcriptionally\_suppressed\_by() (pypath.core.network.Network method)@\spxentry{transcriptionally\_suppressed\_by()}\spxextra{pypath.core.network.Network method}}

\begin{fulllineitems}
\phantomsection\label{\detokenize{reference:pypath.core.network.Network.transcriptionally_suppressed_by}}\pysiglinewithargsret{\sphinxbfcode{\sphinxupquote{transcriptionally\_suppressed\_by}}}{}{}~\begin{quote}\begin{description}
\item[{Parameters}] \leavevmode\begin{itemize}
\item {} 
\sphinxstyleliteralstrong{\sphinxupquote{entity}} (\sphinxstyleliteralemphasis{\sphinxupquote{str}}\sphinxstyleliteralemphasis{\sphinxupquote{,}}{\hyperref[\detokenize{reference:pypath.core.entity.Entity}]{\sphinxcrossref{\sphinxstyleliteralemphasis{\sphinxupquote{Entity}}}}}\sphinxstyleliteralemphasis{\sphinxupquote{,}}\sphinxstyleliteralemphasis{\sphinxupquote{list}}\sphinxstyleliteralemphasis{\sphinxupquote{,}}\sphinxstyleliteralemphasis{\sphinxupquote{set}}\sphinxstyleliteralemphasis{\sphinxupquote{,}}\sphinxstyleliteralemphasis{\sphinxupquote{tuple}}\sphinxstyleliteralemphasis{\sphinxupquote{,}}\sphinxstyleliteralemphasis{\sphinxupquote{EntityList}}) \textendash{} An identifier or label of a molecular entity or an
\sphinxcode{\sphinxupquote{Entity}} object. Alternatively an iterator with the
elements of any of the types valid for a single entity argument,
e.g. a list of gene symbols.

\item {} 
\sphinxstyleliteralstrong{\sphinxupquote{mode}} (\sphinxstyleliteralemphasis{\sphinxupquote{str}}) \textendash{} Mode of counting the interactions: \sphinxtitleref{IN}, \sphinxtitleref{OUT} or \sphinxtitleref{ALL} , whether
to consider incoming, outgoing or all edges, respectively,
respective to the \sphinxtitleref{node defined in {}`entity{}`}.

\end{itemize}

\item[{Returns}] \leavevmode
\sphinxcode{\sphinxupquote{EntityList}} object containing the partners having
interactions to the queried node(s) matching all the criteria.
If \sphinxcode{\sphinxupquote{entity}} doesn’t present in the network the returned
\sphinxcode{\sphinxupquote{EntityList}} will be empty just like if no interaction matches
the criteria.

\end{description}\end{quote}

\end{fulllineitems}

\index{transcriptionally\_suppresses() (pypath.core.network.Network method)@\spxentry{transcriptionally\_suppresses()}\spxextra{pypath.core.network.Network method}}

\begin{fulllineitems}
\phantomsection\label{\detokenize{reference:pypath.core.network.Network.transcriptionally_suppresses}}\pysiglinewithargsret{\sphinxbfcode{\sphinxupquote{transcriptionally\_suppresses}}}{}{}~\begin{quote}\begin{description}
\item[{Parameters}] \leavevmode\begin{itemize}
\item {} 
\sphinxstyleliteralstrong{\sphinxupquote{entity}} (\sphinxstyleliteralemphasis{\sphinxupquote{str}}\sphinxstyleliteralemphasis{\sphinxupquote{,}}{\hyperref[\detokenize{reference:pypath.core.entity.Entity}]{\sphinxcrossref{\sphinxstyleliteralemphasis{\sphinxupquote{Entity}}}}}\sphinxstyleliteralemphasis{\sphinxupquote{,}}\sphinxstyleliteralemphasis{\sphinxupquote{list}}\sphinxstyleliteralemphasis{\sphinxupquote{,}}\sphinxstyleliteralemphasis{\sphinxupquote{set}}\sphinxstyleliteralemphasis{\sphinxupquote{,}}\sphinxstyleliteralemphasis{\sphinxupquote{tuple}}\sphinxstyleliteralemphasis{\sphinxupquote{,}}\sphinxstyleliteralemphasis{\sphinxupquote{EntityList}}) \textendash{} An identifier or label of a molecular entity or an
\sphinxcode{\sphinxupquote{Entity}} object. Alternatively an iterator with the
elements of any of the types valid for a single entity argument,
e.g. a list of gene symbols.

\item {} 
\sphinxstyleliteralstrong{\sphinxupquote{mode}} (\sphinxstyleliteralemphasis{\sphinxupquote{str}}) \textendash{} Mode of counting the interactions: \sphinxtitleref{IN}, \sphinxtitleref{OUT} or \sphinxtitleref{ALL} , whether
to consider incoming, outgoing or all edges, respectively,
respective to the \sphinxtitleref{node defined in {}`entity{}`}.

\end{itemize}

\item[{Returns}] \leavevmode
\sphinxcode{\sphinxupquote{EntityList}} object containing the partners having
interactions to the queried node(s) matching all the criteria.
If \sphinxcode{\sphinxupquote{entity}} doesn’t present in the network the returned
\sphinxcode{\sphinxupquote{EntityList}} will be empty just like if no interaction matches
the criteria.

\end{description}\end{quote}

\end{fulllineitems}


\end{fulllineitems}

\index{NetworkStatsRecord (class in pypath.core.network)@\spxentry{NetworkStatsRecord}\spxextra{class in pypath.core.network}}

\begin{fulllineitems}
\phantomsection\label{\detokenize{reference:pypath.core.network.NetworkStatsRecord}}\pysiglinewithargsret{\sphinxbfcode{\sphinxupquote{class }}\sphinxcode{\sphinxupquote{pypath.core.network.}}\sphinxbfcode{\sphinxupquote{NetworkStatsRecord}}}{\emph{total}, \emph{by\_resource}, \emph{by\_category}, \emph{shared}, \emph{unique}, \emph{percent}, \emph{shared\_res\_cat}, \emph{unique\_res\_cat}, \emph{percent\_res\_cat}, \emph{shared\_cat}, \emph{unique\_cat}, \emph{percent\_cat}, \emph{resource\_cat}, \emph{cat\_resource}, \emph{method}, \emph{label}}{}~\index{by\_category (pypath.core.network.NetworkStatsRecord attribute)@\spxentry{by\_category}\spxextra{pypath.core.network.NetworkStatsRecord attribute}}

\begin{fulllineitems}
\phantomsection\label{\detokenize{reference:pypath.core.network.NetworkStatsRecord.by_category}}\pysigline{\sphinxbfcode{\sphinxupquote{by\_category}}}
Alias for field number 2

\end{fulllineitems}

\index{by\_resource (pypath.core.network.NetworkStatsRecord attribute)@\spxentry{by\_resource}\spxextra{pypath.core.network.NetworkStatsRecord attribute}}

\begin{fulllineitems}
\phantomsection\label{\detokenize{reference:pypath.core.network.NetworkStatsRecord.by_resource}}\pysigline{\sphinxbfcode{\sphinxupquote{by\_resource}}}
Alias for field number 1

\end{fulllineitems}

\index{cat\_resource (pypath.core.network.NetworkStatsRecord attribute)@\spxentry{cat\_resource}\spxextra{pypath.core.network.NetworkStatsRecord attribute}}

\begin{fulllineitems}
\phantomsection\label{\detokenize{reference:pypath.core.network.NetworkStatsRecord.cat_resource}}\pysigline{\sphinxbfcode{\sphinxupquote{cat\_resource}}}
Alias for field number 13

\end{fulllineitems}

\index{label (pypath.core.network.NetworkStatsRecord attribute)@\spxentry{label}\spxextra{pypath.core.network.NetworkStatsRecord attribute}}

\begin{fulllineitems}
\phantomsection\label{\detokenize{reference:pypath.core.network.NetworkStatsRecord.label}}\pysigline{\sphinxbfcode{\sphinxupquote{label}}}
Alias for field number 15

\end{fulllineitems}

\index{method (pypath.core.network.NetworkStatsRecord attribute)@\spxentry{method}\spxextra{pypath.core.network.NetworkStatsRecord attribute}}

\begin{fulllineitems}
\phantomsection\label{\detokenize{reference:pypath.core.network.NetworkStatsRecord.method}}\pysigline{\sphinxbfcode{\sphinxupquote{method}}}
Alias for field number 14

\end{fulllineitems}

\index{percent (pypath.core.network.NetworkStatsRecord attribute)@\spxentry{percent}\spxextra{pypath.core.network.NetworkStatsRecord attribute}}

\begin{fulllineitems}
\phantomsection\label{\detokenize{reference:pypath.core.network.NetworkStatsRecord.percent}}\pysigline{\sphinxbfcode{\sphinxupquote{percent}}}
Alias for field number 5

\end{fulllineitems}

\index{percent\_cat (pypath.core.network.NetworkStatsRecord attribute)@\spxentry{percent\_cat}\spxextra{pypath.core.network.NetworkStatsRecord attribute}}

\begin{fulllineitems}
\phantomsection\label{\detokenize{reference:pypath.core.network.NetworkStatsRecord.percent_cat}}\pysigline{\sphinxbfcode{\sphinxupquote{percent\_cat}}}
Alias for field number 11

\end{fulllineitems}

\index{percent\_res\_cat (pypath.core.network.NetworkStatsRecord attribute)@\spxentry{percent\_res\_cat}\spxextra{pypath.core.network.NetworkStatsRecord attribute}}

\begin{fulllineitems}
\phantomsection\label{\detokenize{reference:pypath.core.network.NetworkStatsRecord.percent_res_cat}}\pysigline{\sphinxbfcode{\sphinxupquote{percent\_res\_cat}}}
Alias for field number 8

\end{fulllineitems}

\index{resource\_cat (pypath.core.network.NetworkStatsRecord attribute)@\spxentry{resource\_cat}\spxextra{pypath.core.network.NetworkStatsRecord attribute}}

\begin{fulllineitems}
\phantomsection\label{\detokenize{reference:pypath.core.network.NetworkStatsRecord.resource_cat}}\pysigline{\sphinxbfcode{\sphinxupquote{resource\_cat}}}
Alias for field number 12

\end{fulllineitems}

\index{shared (pypath.core.network.NetworkStatsRecord attribute)@\spxentry{shared}\spxextra{pypath.core.network.NetworkStatsRecord attribute}}

\begin{fulllineitems}
\phantomsection\label{\detokenize{reference:pypath.core.network.NetworkStatsRecord.shared}}\pysigline{\sphinxbfcode{\sphinxupquote{shared}}}
Alias for field number 3

\end{fulllineitems}

\index{shared\_cat (pypath.core.network.NetworkStatsRecord attribute)@\spxentry{shared\_cat}\spxextra{pypath.core.network.NetworkStatsRecord attribute}}

\begin{fulllineitems}
\phantomsection\label{\detokenize{reference:pypath.core.network.NetworkStatsRecord.shared_cat}}\pysigline{\sphinxbfcode{\sphinxupquote{shared\_cat}}}
Alias for field number 9

\end{fulllineitems}

\index{shared\_res\_cat (pypath.core.network.NetworkStatsRecord attribute)@\spxentry{shared\_res\_cat}\spxextra{pypath.core.network.NetworkStatsRecord attribute}}

\begin{fulllineitems}
\phantomsection\label{\detokenize{reference:pypath.core.network.NetworkStatsRecord.shared_res_cat}}\pysigline{\sphinxbfcode{\sphinxupquote{shared\_res\_cat}}}
Alias for field number 6

\end{fulllineitems}

\index{total (pypath.core.network.NetworkStatsRecord attribute)@\spxentry{total}\spxextra{pypath.core.network.NetworkStatsRecord attribute}}

\begin{fulllineitems}
\phantomsection\label{\detokenize{reference:pypath.core.network.NetworkStatsRecord.total}}\pysigline{\sphinxbfcode{\sphinxupquote{total}}}
Alias for field number 0

\end{fulllineitems}

\index{unique (pypath.core.network.NetworkStatsRecord attribute)@\spxentry{unique}\spxextra{pypath.core.network.NetworkStatsRecord attribute}}

\begin{fulllineitems}
\phantomsection\label{\detokenize{reference:pypath.core.network.NetworkStatsRecord.unique}}\pysigline{\sphinxbfcode{\sphinxupquote{unique}}}
Alias for field number 4

\end{fulllineitems}

\index{unique\_cat (pypath.core.network.NetworkStatsRecord attribute)@\spxentry{unique\_cat}\spxextra{pypath.core.network.NetworkStatsRecord attribute}}

\begin{fulllineitems}
\phantomsection\label{\detokenize{reference:pypath.core.network.NetworkStatsRecord.unique_cat}}\pysigline{\sphinxbfcode{\sphinxupquote{unique\_cat}}}
Alias for field number 10

\end{fulllineitems}

\index{unique\_res\_cat (pypath.core.network.NetworkStatsRecord attribute)@\spxentry{unique\_res\_cat}\spxextra{pypath.core.network.NetworkStatsRecord attribute}}

\begin{fulllineitems}
\phantomsection\label{\detokenize{reference:pypath.core.network.NetworkStatsRecord.unique_res_cat}}\pysigline{\sphinxbfcode{\sphinxupquote{unique\_res\_cat}}}
Alias for field number 7

\end{fulllineitems}


\end{fulllineitems}



\section{omnipath}
\label{\detokenize{reference:omnipath}}

\section{pdb}
\label{\detokenize{reference:module-pypath.utils.pdb}}\label{\detokenize{reference:pdb}}\index{pypath.utils.pdb (module)@\spxentry{pypath.utils.pdb}\spxextra{module}}\index{ResidueMapper (class in pypath.utils.pdb)@\spxentry{ResidueMapper}\spxextra{class in pypath.utils.pdb}}

\begin{fulllineitems}
\phantomsection\label{\detokenize{reference:pypath.utils.pdb.ResidueMapper}}\pysigline{\sphinxbfcode{\sphinxupquote{class }}\sphinxcode{\sphinxupquote{pypath.utils.pdb.}}\sphinxbfcode{\sphinxupquote{ResidueMapper}}}
This class stores and serves the PDB \textendash{}\textgreater{} UniProt 
residue level mapping. Attempts to download the 
mapping, and stores it for further use. Converts 
PDB residue numbers to the corresponding UniProt ones.
\index{clean() (pypath.utils.pdb.ResidueMapper method)@\spxentry{clean()}\spxextra{pypath.utils.pdb.ResidueMapper method}}

\begin{fulllineitems}
\phantomsection\label{\detokenize{reference:pypath.utils.pdb.ResidueMapper.clean}}\pysiglinewithargsret{\sphinxbfcode{\sphinxupquote{clean}}}{}{}
Removes cached mappings, freeing up memory.

\end{fulllineitems}


\end{fulllineitems}



\section{plot}
\label{\detokenize{reference:module-pypath.visual.plot}}\label{\detokenize{reference:plot}}\index{pypath.visual.plot (module)@\spxentry{pypath.visual.plot}\spxextra{module}}\index{is\_opentype\_cff\_font() (in module pypath.visual.plot)@\spxentry{is\_opentype\_cff\_font()}\spxextra{in module pypath.visual.plot}}

\begin{fulllineitems}
\phantomsection\label{\detokenize{reference:pypath.visual.plot.is_opentype_cff_font}}\pysiglinewithargsret{\sphinxcode{\sphinxupquote{pypath.visual.plot.}}\sphinxbfcode{\sphinxupquote{is\_opentype\_cff\_font}}}{\emph{filename}}{}
This is necessary to fix a bug in matplotlib:
\sphinxurl{https://github.com/matplotlib/matplotlib/pull/6714}
Returns True if the given font is a Postscript Compact Font Format
Font embedded in an OpenType wrapper.  Used by the PostScript and
PDF backends that can not subset these fonts.

\end{fulllineitems}

\index{randn() (in module pypath.visual.plot)@\spxentry{randn()}\spxextra{in module pypath.visual.plot}}

\begin{fulllineitems}
\phantomsection\label{\detokenize{reference:pypath.visual.plot.randn}}\pysiglinewithargsret{\sphinxcode{\sphinxupquote{pypath.visual.plot.}}\sphinxbfcode{\sphinxupquote{randn}}}{\emph{d0}, \emph{d1}, \emph{...}, \emph{dn}}{}
Return a sample (or samples) from the “standard normal” distribution.

If positive, int\_like or int-convertible arguments are provided,
\sphinxtitleref{randn} generates an array of shape \sphinxcode{\sphinxupquote{(d0, d1, ..., dn)}}, filled
with random floats sampled from a univariate “normal” (Gaussian)
distribution of mean 0 and variance 1 (if any of the \(d_i\) are
floats, they are first converted to integers by truncation). A single
float randomly sampled from the distribution is returned if no
argument is provided.

This is a convenience function.  If you want an interface that takes a
tuple as the first argument, use \sphinxtitleref{numpy.random.standard\_normal} instead.
\begin{description}
\item[{d0, d1, …, dn}] \leavevmode{[}int, optional{]}
The dimensions of the returned array, should be all positive.
If no argument is given a single Python float is returned.

\end{description}
\begin{description}
\item[{Z}] \leavevmode{[}ndarray or float{]}
A \sphinxcode{\sphinxupquote{(d0, d1, ..., dn)}}-shaped array of floating-point samples from
the standard normal distribution, or a single such float if
no parameters were supplied.

\end{description}

standard\_normal : Similar, but takes a tuple as its argument.

For random samples from \(N(\mu, \sigma^2)\), use:

\sphinxcode{\sphinxupquote{sigma * np.random.randn(...) + mu}}

\begin{sphinxVerbatim}[commandchars=\\\{\}]
\PYG{g+gp}{\PYGZgt{}\PYGZgt{}\PYGZgt{} }\PYG{n}{np}\PYG{o}{.}\PYG{n}{random}\PYG{o}{.}\PYG{n}{randn}\PYG{p}{(}\PYG{p}{)}
\PYG{g+go}{2.1923875335537315 \PYGZsh{}random}
\end{sphinxVerbatim}

Two-by-four array of samples from N(3, 6.25):

\begin{sphinxVerbatim}[commandchars=\\\{\}]
\PYG{g+gp}{\PYGZgt{}\PYGZgt{}\PYGZgt{} }\PYG{l+m+mf}{2.5} \PYG{o}{*} \PYG{n}{np}\PYG{o}{.}\PYG{n}{random}\PYG{o}{.}\PYG{n}{randn}\PYG{p}{(}\PYG{l+m+mi}{2}\PYG{p}{,} \PYG{l+m+mi}{4}\PYG{p}{)} \PYG{o}{+} \PYG{l+m+mi}{3}
\PYG{g+go}{array([[\PYGZhy{}4.49401501,  4.00950034, \PYGZhy{}1.81814867,  7.29718677],  \PYGZsh{}random}
\PYG{g+go}{       [ 0.39924804,  4.68456316,  4.99394529,  4.84057254]]) \PYGZsh{}random}
\end{sphinxVerbatim}

\end{fulllineitems}



\section{progress}
\label{\detokenize{reference:module-pypath.share.progress}}\label{\detokenize{reference:progress}}\index{pypath.share.progress (module)@\spxentry{pypath.share.progress}\spxextra{module}}\index{Progress (class in pypath.share.progress)@\spxentry{Progress}\spxextra{class in pypath.share.progress}}

\begin{fulllineitems}
\phantomsection\label{\detokenize{reference:pypath.share.progress.Progress}}\pysiglinewithargsret{\sphinxbfcode{\sphinxupquote{class }}\sphinxcode{\sphinxupquote{pypath.share.progress.}}\sphinxbfcode{\sphinxupquote{Progress}}}{\emph{total=None}, \emph{name='Progress'}, \emph{interval=None}, \emph{percent=True}, \emph{status='initializing'}, \emph{done=0}, \emph{init=True}, \emph{unit='it'}, \emph{off=None}}{}
Before I had my custom progressbar here.
Now it is a wrapper around the great progressbar \sphinxtitleref{tqdm}.
Old implementation moved to \sphinxtitleref{OldProgress} class.
\index{get\_desc() (pypath.share.progress.Progress method)@\spxentry{get\_desc()}\spxextra{pypath.share.progress.Progress method}}

\begin{fulllineitems}
\phantomsection\label{\detokenize{reference:pypath.share.progress.Progress.get_desc}}\pysiglinewithargsret{\sphinxbfcode{\sphinxupquote{get\_desc}}}{}{}
Returns a formatted string of the description, consisted of
the name and the status. The name supposed something constant
within the life of the progressbar, while the status is there
to give information about the current stage of the task.

\end{fulllineitems}

\index{init\_tqdm() (pypath.share.progress.Progress method)@\spxentry{init\_tqdm()}\spxextra{pypath.share.progress.Progress method}}

\begin{fulllineitems}
\phantomsection\label{\detokenize{reference:pypath.share.progress.Progress.init_tqdm}}\pysiglinewithargsret{\sphinxbfcode{\sphinxupquote{init\_tqdm}}}{}{}
Creates a tqdm instance.

\end{fulllineitems}

\index{set\_done() (pypath.share.progress.Progress method)@\spxentry{set\_done()}\spxextra{pypath.share.progress.Progress method}}

\begin{fulllineitems}
\phantomsection\label{\detokenize{reference:pypath.share.progress.Progress.set_done}}\pysiglinewithargsret{\sphinxbfcode{\sphinxupquote{set\_done}}}{\emph{done}}{}
Sets the position of the progress bar.

\end{fulllineitems}

\index{set\_status() (pypath.share.progress.Progress method)@\spxentry{set\_status()}\spxextra{pypath.share.progress.Progress method}}

\begin{fulllineitems}
\phantomsection\label{\detokenize{reference:pypath.share.progress.Progress.set_status}}\pysiglinewithargsret{\sphinxbfcode{\sphinxupquote{set\_status}}}{\emph{status}}{}
Changes the prefix of the progressbar.

\end{fulllineitems}

\index{set\_total() (pypath.share.progress.Progress method)@\spxentry{set\_total()}\spxextra{pypath.share.progress.Progress method}}

\begin{fulllineitems}
\phantomsection\label{\detokenize{reference:pypath.share.progress.Progress.set_total}}\pysiglinewithargsret{\sphinxbfcode{\sphinxupquote{set\_total}}}{\emph{total}}{}
Changes the total value of the progress bar.

\end{fulllineitems}

\index{step() (pypath.share.progress.Progress method)@\spxentry{step()}\spxextra{pypath.share.progress.Progress method}}

\begin{fulllineitems}
\phantomsection\label{\detokenize{reference:pypath.share.progress.Progress.step}}\pysiglinewithargsret{\sphinxbfcode{\sphinxupquote{step}}}{\emph{step=1}, \emph{msg=None}, \emph{status='busy'}, \emph{force=False}}{}
Updates the progressbar by the desired number of steps.
\begin{quote}\begin{description}
\item[{Parameters}] \leavevmode
\sphinxstyleliteralstrong{\sphinxupquote{step}} (\sphinxstyleliteralemphasis{\sphinxupquote{int}}) \textendash{} Number of steps or items.

\end{description}\end{quote}

\end{fulllineitems}

\index{terminate() (pypath.share.progress.Progress method)@\spxentry{terminate()}\spxextra{pypath.share.progress.Progress method}}

\begin{fulllineitems}
\phantomsection\label{\detokenize{reference:pypath.share.progress.Progress.terminate}}\pysiglinewithargsret{\sphinxbfcode{\sphinxupquote{terminate}}}{\emph{status='finished'}}{}
Terminates the progressbar and destroys the tqdm object.

\end{fulllineitems}


\end{fulllineitems}



\section{enz\_sub}
\label{\detokenize{reference:module-pypath.core.enz_sub}}\label{\detokenize{reference:enz-sub}}\index{pypath.core.enz\_sub (module)@\spxentry{pypath.core.enz\_sub}\spxextra{module}}

\section{pyreact}
\label{\detokenize{reference:module-pypath.utils.pyreact}}\label{\detokenize{reference:pyreact}}\index{pypath.utils.pyreact (module)@\spxentry{pypath.utils.pyreact}\spxextra{module}}\index{BioPaxReader (class in pypath.utils.pyreact)@\spxentry{BioPaxReader}\spxextra{class in pypath.utils.pyreact}}

\begin{fulllineitems}
\phantomsection\label{\detokenize{reference:pypath.utils.pyreact.BioPaxReader}}\pysiglinewithargsret{\sphinxbfcode{\sphinxupquote{class }}\sphinxcode{\sphinxupquote{pypath.utils.pyreact.}}\sphinxbfcode{\sphinxupquote{BioPaxReader}}}{\emph{biopax}, \emph{source}, \emph{cleanup\_period=800}, \emph{file\_from\_archive=None}, \emph{silent=False}}{}
This class parses a BioPAX file and exposes its content easily accessible
for further processing. First it opens the file, if necessary it extracts
from the archive. Then an \sphinxtitleref{lxml.etree.iterparse} object is created, so the
iteration is efficient and memory requirements are minimal. The iterparse
object is iterated then, and for each tag included in the
\sphinxtitleref{BioPaxReader.methods} dict, the appropriate method is called. These me-
thods extract information from the BioPAX entity, and store it in arbit-
rary data structures: strings, lists or dicts. These are stored in dicts
where keys are the original IDs of the tags, prefixed with the unique ID
of the parser object. This is necessary to give a way to merge later the
result of parsing more BioPAX files. For example, \sphinxtitleref{id42} may identify
EGFR in one file, but AKT1 in the other. Then, the parser of the first
file has a unique ID of a 5 letter random string, the second parser a
different one, and the molecules with the same ID can be distinguished
at merging, e.g. EGFR will be \sphinxtitleref{ffjh2@id42} and AKT1 will be \sphinxtitleref{tr9gy@id42}.
The methods and the resulted dicts are named after the BioPAX elements,
sometimes abbreviated. For example, \sphinxtitleref{BioPaxReader.protein()} processes
the \sphinxtitleref{\textless{}bp:Protein\textgreater{}} elements, and stores the results in
\sphinxtitleref{BioPaxReader.proteins}.

In its current state, this class does not parse every information and
all BioPax entities. For example, nucleic acid related entities and
interactions are omitted. But these easily can be added with minor mo-
difications.
\index{biopax\_size() (pypath.utils.pyreact.BioPaxReader method)@\spxentry{biopax\_size()}\spxextra{pypath.utils.pyreact.BioPaxReader method}}

\begin{fulllineitems}
\phantomsection\label{\detokenize{reference:pypath.utils.pyreact.BioPaxReader.biopax_size}}\pysiglinewithargsret{\sphinxbfcode{\sphinxupquote{biopax\_size}}}{}{}
Gets the uncompressed size of the BioPax XML. This is needed in
order to have a progress bar. This method should not be called
directly, \sphinxcode{\sphinxupquote{BioPaxReader.process()}} calls it.

\end{fulllineitems}

\index{cleanup\_hook() (pypath.utils.pyreact.BioPaxReader method)@\spxentry{cleanup\_hook()}\spxextra{pypath.utils.pyreact.BioPaxReader method}}

\begin{fulllineitems}
\phantomsection\label{\detokenize{reference:pypath.utils.pyreact.BioPaxReader.cleanup_hook}}\pysiglinewithargsret{\sphinxbfcode{\sphinxupquote{cleanup\_hook}}}{}{}
Removes the used elements to free up memory.
This method should not be called directly,
\sphinxcode{\sphinxupquote{BioPaxReader.iterate()}} calls it.

\end{fulllineitems}

\index{close\_biopax() (pypath.utils.pyreact.BioPaxReader method)@\spxentry{close\_biopax()}\spxextra{pypath.utils.pyreact.BioPaxReader method}}

\begin{fulllineitems}
\phantomsection\label{\detokenize{reference:pypath.utils.pyreact.BioPaxReader.close_biopax}}\pysiglinewithargsret{\sphinxbfcode{\sphinxupquote{close\_biopax}}}{}{}
Deletes the iterator and closes the file object.
This method should not be called directly,
\sphinxcode{\sphinxupquote{BioPaxReader.process()}} calls it.

\end{fulllineitems}

\index{extract() (pypath.utils.pyreact.BioPaxReader method)@\spxentry{extract()}\spxextra{pypath.utils.pyreact.BioPaxReader method}}

\begin{fulllineitems}
\phantomsection\label{\detokenize{reference:pypath.utils.pyreact.BioPaxReader.extract}}\pysiglinewithargsret{\sphinxbfcode{\sphinxupquote{extract}}}{}{}
Extracts the BioPax file from compressed archive. Creates a
temporary file. This is needed to trace the progress of
processing, which is useful in case of large files.
This method should not be called directly,
\sphinxcode{\sphinxupquote{BioPaxReader.process()}} calls it.

\end{fulllineitems}

\index{init\_etree() (pypath.utils.pyreact.BioPaxReader method)@\spxentry{init\_etree()}\spxextra{pypath.utils.pyreact.BioPaxReader method}}

\begin{fulllineitems}
\phantomsection\label{\detokenize{reference:pypath.utils.pyreact.BioPaxReader.init_etree}}\pysiglinewithargsret{\sphinxbfcode{\sphinxupquote{init\_etree}}}{}{}
Creates the \sphinxcode{\sphinxupquote{lxml.etree.iterparse}} object.
This method should not be called directly,
\sphinxcode{\sphinxupquote{BioPaxReader.process()}} calls it.

\end{fulllineitems}

\index{iterate() (pypath.utils.pyreact.BioPaxReader method)@\spxentry{iterate()}\spxextra{pypath.utils.pyreact.BioPaxReader method}}

\begin{fulllineitems}
\phantomsection\label{\detokenize{reference:pypath.utils.pyreact.BioPaxReader.iterate}}\pysiglinewithargsret{\sphinxbfcode{\sphinxupquote{iterate}}}{}{}
Iterates the BioPax XML and calls the appropriate methods
for each element.
This method should not be called directly,
\sphinxcode{\sphinxupquote{BioPaxReader.process()}} calls it.

\end{fulllineitems}

\index{open\_biopax() (pypath.utils.pyreact.BioPaxReader method)@\spxentry{open\_biopax()}\spxextra{pypath.utils.pyreact.BioPaxReader method}}

\begin{fulllineitems}
\phantomsection\label{\detokenize{reference:pypath.utils.pyreact.BioPaxReader.open_biopax}}\pysiglinewithargsret{\sphinxbfcode{\sphinxupquote{open\_biopax}}}{}{}
Opens the BioPax file. This method should not be called directly,
\sphinxcode{\sphinxupquote{BioPaxReader.process()}} calls it.

\end{fulllineitems}

\index{process() (pypath.utils.pyreact.BioPaxReader method)@\spxentry{process()}\spxextra{pypath.utils.pyreact.BioPaxReader method}}

\begin{fulllineitems}
\phantomsection\label{\detokenize{reference:pypath.utils.pyreact.BioPaxReader.process}}\pysiglinewithargsret{\sphinxbfcode{\sphinxupquote{process}}}{\emph{silent=False}}{}
This method executes the total workflow of BioPax processing.
\begin{quote}\begin{description}
\item[{Parameters}] \leavevmode
\sphinxstyleliteralstrong{\sphinxupquote{silent}} (\sphinxstyleliteralemphasis{\sphinxupquote{bool}}) \textendash{} whether to print status messages and progress bars.

\end{description}\end{quote}

\end{fulllineitems}

\index{set\_progress() (pypath.utils.pyreact.BioPaxReader method)@\spxentry{set\_progress()}\spxextra{pypath.utils.pyreact.BioPaxReader method}}

\begin{fulllineitems}
\phantomsection\label{\detokenize{reference:pypath.utils.pyreact.BioPaxReader.set_progress}}\pysiglinewithargsret{\sphinxbfcode{\sphinxupquote{set\_progress}}}{}{}
Initializes a progress bar.
This method should not be called directly,
\sphinxcode{\sphinxupquote{BioPaxReader.process()}} calls it.

\end{fulllineitems}


\end{fulllineitems}



\section{reflists}
\label{\detokenize{reference:module-pypath.utils.reflists}}\label{\detokenize{reference:reflists}}\index{pypath.utils.reflists (module)@\spxentry{pypath.utils.reflists}\spxextra{module}}\index{check() (in module pypath.utils.reflists)@\spxentry{check()}\spxextra{in module pypath.utils.reflists}}

\begin{fulllineitems}
\phantomsection\label{\detokenize{reference:pypath.utils.reflists.check}}\pysiglinewithargsret{\sphinxcode{\sphinxupquote{pypath.utils.reflists.}}\sphinxbfcode{\sphinxupquote{check}}}{\emph{name}, \emph{id\_type}, \emph{ncbi\_tax\_id=None}}{}
Checks if the identifier \sphinxcode{\sphinxupquote{name}} is in the reference list with
the provided \sphinxcode{\sphinxupquote{id\_type}} and organism.

\end{fulllineitems}

\index{is\_not() (in module pypath.utils.reflists)@\spxentry{is\_not()}\spxextra{in module pypath.utils.reflists}}

\begin{fulllineitems}
\phantomsection\label{\detokenize{reference:pypath.utils.reflists.is_not}}\pysiglinewithargsret{\sphinxcode{\sphinxupquote{pypath.utils.reflists.}}\sphinxbfcode{\sphinxupquote{is\_not}}}{\emph{names}, \emph{id\_type}, \emph{ncbi\_tax\_id=None}}{}
Returns the identifiers from \sphinxcode{\sphinxupquote{names}} which are not instances of
the provided \sphinxcode{\sphinxupquote{id\_type}} and from the given organism.

\end{fulllineitems}

\index{select() (in module pypath.utils.reflists)@\spxentry{select()}\spxextra{in module pypath.utils.reflists}}

\begin{fulllineitems}
\phantomsection\label{\detokenize{reference:pypath.utils.reflists.select}}\pysiglinewithargsret{\sphinxcode{\sphinxupquote{pypath.utils.reflists.}}\sphinxbfcode{\sphinxupquote{select}}}{\emph{names}, \emph{id\_type}, \emph{ncbi\_tax\_id=None}}{}
Selects the identifiers in \sphinxcode{\sphinxupquote{names}} which are in the reference list
with the provided \sphinxcode{\sphinxupquote{id\_type}} and organism.

\end{fulllineitems}



\section{refs}
\label{\detokenize{reference:module-pypath.internals.refs}}\label{\detokenize{reference:refs}}\index{pypath.internals.refs (module)@\spxentry{pypath.internals.refs}\spxextra{module}}\index{get\_pmid() (in module pypath.internals.refs)@\spxentry{get\_pmid()}\spxextra{in module pypath.internals.refs}}

\begin{fulllineitems}
\phantomsection\label{\detokenize{reference:pypath.internals.refs.get_pmid}}\pysiglinewithargsret{\sphinxcode{\sphinxupquote{pypath.internals.refs.}}\sphinxbfcode{\sphinxupquote{get\_pmid}}}{\emph{idList}}{}
For a list of doi or PMC IDs 
fetches the corresponding PMIDs.

\end{fulllineitems}

\index{get\_pubmed\_data() (in module pypath.internals.refs)@\spxentry{get\_pubmed\_data()}\spxextra{in module pypath.internals.refs}}

\begin{fulllineitems}
\phantomsection\label{\detokenize{reference:pypath.internals.refs.get_pubmed_data}}\pysiglinewithargsret{\sphinxcode{\sphinxupquote{pypath.internals.refs.}}\sphinxbfcode{\sphinxupquote{get\_pubmed\_data}}}{\emph{pp}, \emph{cachefile=None}, \emph{htp\_threshold=20}}{}
For one PyPath object, obtains metadata for all PubMed IDs
through NCBI E-utils.
\begin{quote}\begin{description}
\item[{Parameters}] \leavevmode\begin{itemize}
\item {} 
\sphinxstyleliteralstrong{\sphinxupquote{pp}} \textendash{} \sphinxcode{\sphinxupquote{pypath.PyPath}} object

\item {} 
\sphinxstyleliteralstrong{\sphinxupquote{htp\_threshold}} \textendash{} The number of interactions for one reference
above the study considered to be high-throughput.

\end{itemize}

\end{description}\end{quote}

\end{fulllineitems}

\index{only\_pmids() (in module pypath.internals.refs)@\spxentry{only\_pmids()}\spxextra{in module pypath.internals.refs}}

\begin{fulllineitems}
\phantomsection\label{\detokenize{reference:pypath.internals.refs.only_pmids}}\pysiglinewithargsret{\sphinxcode{\sphinxupquote{pypath.internals.refs.}}\sphinxbfcode{\sphinxupquote{only\_pmids}}}{\emph{idList}, \emph{strict=True}}{}
Return elements unchanged which comply with PubMed ID format,
and attempts to translate the DOIs and PMC IDs using NCBI
E-utils.
Returns list containing only PMIDs.
\begin{description}
\item[{@idList}] \leavevmode{[}list, str{]}
List of IDs or one single ID.

\item[{@strict}] \leavevmode{[}bool{]}
Whether keep in the list those IDs which are not PMIDs,
neither DOIs or PMC IDs or NIH manuscript IDs.

\end{description}

\end{fulllineitems}

\index{open\_pubmed() (in module pypath.internals.refs)@\spxentry{open\_pubmed()}\spxextra{in module pypath.internals.refs}}

\begin{fulllineitems}
\phantomsection\label{\detokenize{reference:pypath.internals.refs.open_pubmed}}\pysiglinewithargsret{\sphinxcode{\sphinxupquote{pypath.internals.refs.}}\sphinxbfcode{\sphinxupquote{open\_pubmed}}}{\emph{pmid}}{}
Opens PubMed record in web browser.
\begin{description}
\item[{@pmid}] \leavevmode{[}str or int{]}
PubMed ID

\end{description}

\end{fulllineitems}



\section{residues}
\label{\detokenize{reference:module-pypath.utils.residues}}\label{\detokenize{reference:residues}}\index{pypath.utils.residues (module)@\spxentry{pypath.utils.residues}\spxextra{module}}\index{ResidueMapper (class in pypath.utils.residues)@\spxentry{ResidueMapper}\spxextra{class in pypath.utils.residues}}

\begin{fulllineitems}
\phantomsection\label{\detokenize{reference:pypath.utils.residues.ResidueMapper}}\pysigline{\sphinxbfcode{\sphinxupquote{class }}\sphinxcode{\sphinxupquote{pypath.utils.residues.}}\sphinxbfcode{\sphinxupquote{ResidueMapper}}}
This class stores and serves the PDB \textendash{}\textgreater{} UniProt 
residue level mapping. Attempts to download the 
mapping, and stores it for further use. Converts 
PDB residue numbers to the corresponding UniProt ones.
\index{clean() (pypath.utils.residues.ResidueMapper method)@\spxentry{clean()}\spxextra{pypath.utils.residues.ResidueMapper method}}

\begin{fulllineitems}
\phantomsection\label{\detokenize{reference:pypath.utils.residues.ResidueMapper.clean}}\pysiglinewithargsret{\sphinxbfcode{\sphinxupquote{clean}}}{}{}
Removes cached mappings, freeing up memory.

\end{fulllineitems}


\end{fulllineitems}



\section{resource}
\label{\detokenize{reference:module-pypath.internals.resource}}\label{\detokenize{reference:resource}}\index{pypath.internals.resource (module)@\spxentry{pypath.internals.resource}\spxextra{module}}\index{AbstractResource (class in pypath.internals.resource)@\spxentry{AbstractResource}\spxextra{class in pypath.internals.resource}}

\begin{fulllineitems}
\phantomsection\label{\detokenize{reference:pypath.internals.resource.AbstractResource}}\pysiglinewithargsret{\sphinxbfcode{\sphinxupquote{class }}\sphinxcode{\sphinxupquote{pypath.internals.resource.}}\sphinxbfcode{\sphinxupquote{AbstractResource}}}{\emph{name}, \emph{ncbi\_tax\_id=9606}, \emph{input\_method=None}, \emph{input\_args=None}, \emph{dump=None}, \emph{data\_attr\_name=None}, \emph{**kwargs}}{}
Generic class for downloading, processing and serving
data from a resource.
\index{load\_data() (pypath.internals.resource.AbstractResource method)@\spxentry{load\_data()}\spxextra{pypath.internals.resource.AbstractResource method}}

\begin{fulllineitems}
\phantomsection\label{\detokenize{reference:pypath.internals.resource.AbstractResource.load_data}}\pysiglinewithargsret{\sphinxbfcode{\sphinxupquote{load\_data}}}{}{}
Loads the data by calling \sphinxcode{\sphinxupquote{input\_method}}.

\end{fulllineitems}

\index{process() (pypath.internals.resource.AbstractResource method)@\spxentry{process()}\spxextra{pypath.internals.resource.AbstractResource method}}

\begin{fulllineitems}
\phantomsection\label{\detokenize{reference:pypath.internals.resource.AbstractResource.process}}\pysiglinewithargsret{\sphinxbfcode{\sphinxupquote{process}}}{}{}
Calls the \sphinxcode{\sphinxupquote{\_process\_method}}.

\end{fulllineitems}

\index{set\_method() (pypath.internals.resource.AbstractResource method)@\spxentry{set\_method()}\spxextra{pypath.internals.resource.AbstractResource method}}

\begin{fulllineitems}
\phantomsection\label{\detokenize{reference:pypath.internals.resource.AbstractResource.set_method}}\pysiglinewithargsret{\sphinxbfcode{\sphinxupquote{set\_method}}}{}{}
Sets the data input method by looking up in \sphinxcode{\sphinxupquote{inputs}} module if
necessary.

\end{fulllineitems}


\end{fulllineitems}

\index{EnzymeSubstrateResourceKey (class in pypath.internals.resource)@\spxentry{EnzymeSubstrateResourceKey}\spxextra{class in pypath.internals.resource}}

\begin{fulllineitems}
\phantomsection\label{\detokenize{reference:pypath.internals.resource.EnzymeSubstrateResourceKey}}\pysiglinewithargsret{\sphinxbfcode{\sphinxupquote{class }}\sphinxcode{\sphinxupquote{pypath.internals.resource.}}\sphinxbfcode{\sphinxupquote{EnzymeSubstrateResourceKey}}}{\emph{name}, \emph{data\_type}, \emph{via}}{}~\index{data\_type (pypath.internals.resource.EnzymeSubstrateResourceKey attribute)@\spxentry{data\_type}\spxextra{pypath.internals.resource.EnzymeSubstrateResourceKey attribute}}

\begin{fulllineitems}
\phantomsection\label{\detokenize{reference:pypath.internals.resource.EnzymeSubstrateResourceKey.data_type}}\pysigline{\sphinxbfcode{\sphinxupquote{data\_type}}}
Alias for field number 1

\end{fulllineitems}

\index{name (pypath.internals.resource.EnzymeSubstrateResourceKey attribute)@\spxentry{name}\spxextra{pypath.internals.resource.EnzymeSubstrateResourceKey attribute}}

\begin{fulllineitems}
\phantomsection\label{\detokenize{reference:pypath.internals.resource.EnzymeSubstrateResourceKey.name}}\pysigline{\sphinxbfcode{\sphinxupquote{name}}}
Alias for field number 0

\end{fulllineitems}

\index{via (pypath.internals.resource.EnzymeSubstrateResourceKey attribute)@\spxentry{via}\spxextra{pypath.internals.resource.EnzymeSubstrateResourceKey attribute}}

\begin{fulllineitems}
\phantomsection\label{\detokenize{reference:pypath.internals.resource.EnzymeSubstrateResourceKey.via}}\pysigline{\sphinxbfcode{\sphinxupquote{via}}}
Alias for field number 2

\end{fulllineitems}


\end{fulllineitems}

\index{NetworkResourceKey (class in pypath.internals.resource)@\spxentry{NetworkResourceKey}\spxextra{class in pypath.internals.resource}}

\begin{fulllineitems}
\phantomsection\label{\detokenize{reference:pypath.internals.resource.NetworkResourceKey}}\pysiglinewithargsret{\sphinxbfcode{\sphinxupquote{class }}\sphinxcode{\sphinxupquote{pypath.internals.resource.}}\sphinxbfcode{\sphinxupquote{NetworkResourceKey}}}{\emph{name}, \emph{data\_type}, \emph{interaction\_type}, \emph{data\_model}, \emph{via}}{}~\index{data\_model (pypath.internals.resource.NetworkResourceKey attribute)@\spxentry{data\_model}\spxextra{pypath.internals.resource.NetworkResourceKey attribute}}

\begin{fulllineitems}
\phantomsection\label{\detokenize{reference:pypath.internals.resource.NetworkResourceKey.data_model}}\pysigline{\sphinxbfcode{\sphinxupquote{data\_model}}}
Alias for field number 3

\end{fulllineitems}

\index{data\_type (pypath.internals.resource.NetworkResourceKey attribute)@\spxentry{data\_type}\spxextra{pypath.internals.resource.NetworkResourceKey attribute}}

\begin{fulllineitems}
\phantomsection\label{\detokenize{reference:pypath.internals.resource.NetworkResourceKey.data_type}}\pysigline{\sphinxbfcode{\sphinxupquote{data\_type}}}
Alias for field number 1

\end{fulllineitems}

\index{interaction\_type (pypath.internals.resource.NetworkResourceKey attribute)@\spxentry{interaction\_type}\spxextra{pypath.internals.resource.NetworkResourceKey attribute}}

\begin{fulllineitems}
\phantomsection\label{\detokenize{reference:pypath.internals.resource.NetworkResourceKey.interaction_type}}\pysigline{\sphinxbfcode{\sphinxupquote{interaction\_type}}}
Alias for field number 2

\end{fulllineitems}

\index{name (pypath.internals.resource.NetworkResourceKey attribute)@\spxentry{name}\spxextra{pypath.internals.resource.NetworkResourceKey attribute}}

\begin{fulllineitems}
\phantomsection\label{\detokenize{reference:pypath.internals.resource.NetworkResourceKey.name}}\pysigline{\sphinxbfcode{\sphinxupquote{name}}}
Alias for field number 0

\end{fulllineitems}

\index{via (pypath.internals.resource.NetworkResourceKey attribute)@\spxentry{via}\spxextra{pypath.internals.resource.NetworkResourceKey attribute}}

\begin{fulllineitems}
\phantomsection\label{\detokenize{reference:pypath.internals.resource.NetworkResourceKey.via}}\pysigline{\sphinxbfcode{\sphinxupquote{via}}}
Alias for field number 4

\end{fulllineitems}


\end{fulllineitems}



\section{seq}
\label{\detokenize{reference:module-pypath.utils.seq}}\label{\detokenize{reference:seq}}\index{pypath.utils.seq (module)@\spxentry{pypath.utils.seq}\spxextra{module}}\index{get\_isoforms() (in module pypath.utils.seq)@\spxentry{get\_isoforms()}\spxextra{in module pypath.utils.seq}}

\begin{fulllineitems}
\phantomsection\label{\detokenize{reference:pypath.utils.seq.get_isoforms}}\pysiglinewithargsret{\sphinxcode{\sphinxupquote{pypath.utils.seq.}}\sphinxbfcode{\sphinxupquote{get\_isoforms}}}{\emph{organism=9606}}{}
Loads UniProt sequences for all isoforms.

\end{fulllineitems}

\index{read\_fasta() (in module pypath.utils.seq)@\spxentry{read\_fasta()}\spxextra{in module pypath.utils.seq}}

\begin{fulllineitems}
\phantomsection\label{\detokenize{reference:pypath.utils.seq.read_fasta}}\pysiglinewithargsret{\sphinxcode{\sphinxupquote{pypath.utils.seq.}}\sphinxbfcode{\sphinxupquote{read\_fasta}}}{\emph{fasta}}{}
Parses a fasta file.
Returns dict with headers as keys and sequences as values.

\end{fulllineitems}

\index{swissprot\_seq() (in module pypath.utils.seq)@\spxentry{swissprot\_seq()}\spxextra{in module pypath.utils.seq}}

\begin{fulllineitems}
\phantomsection\label{\detokenize{reference:pypath.utils.seq.swissprot_seq}}\pysiglinewithargsret{\sphinxcode{\sphinxupquote{pypath.utils.seq.}}\sphinxbfcode{\sphinxupquote{swissprot\_seq}}}{\emph{organism=9606}, \emph{isoforms=False}}{}
Loads all sequences for an organism, optionally
for all isoforms, by default only first isoform.

\end{fulllineitems}



\section{server}
\label{\detokenize{reference:server}}

\section{session}
\label{\detokenize{reference:module-pypath.share.session}}\label{\detokenize{reference:session}}\index{pypath.share.session (module)@\spxentry{pypath.share.session}\spxextra{module}}\index{get\_log() (in module pypath.share.session)@\spxentry{get\_log()}\spxextra{in module pypath.share.session}}

\begin{fulllineitems}
\phantomsection\label{\detokenize{reference:pypath.share.session.get_log}}\pysiglinewithargsret{\sphinxcode{\sphinxupquote{pypath.share.session.}}\sphinxbfcode{\sphinxupquote{get\_log}}}{}{}
Returns the \sphinxcode{\sphinxupquote{log.Logger}} instance belonging to the session.

\end{fulllineitems}

\index{get\_session() (in module pypath.share.session)@\spxentry{get\_session()}\spxextra{in module pypath.share.session}}

\begin{fulllineitems}
\phantomsection\label{\detokenize{reference:pypath.share.session.get_session}}\pysiglinewithargsret{\sphinxcode{\sphinxupquote{pypath.share.session.}}\sphinxbfcode{\sphinxupquote{get\_session}}}{}{}
Creates new session or returns the one already created.

\end{fulllineitems}

\index{new\_session() (in module pypath.share.session)@\spxentry{new\_session()}\spxextra{in module pypath.share.session}}

\begin{fulllineitems}
\phantomsection\label{\detokenize{reference:pypath.share.session.new_session}}\pysiglinewithargsret{\sphinxcode{\sphinxupquote{pypath.share.session.}}\sphinxbfcode{\sphinxupquote{new\_session}}}{\emph{label=None}, \emph{log\_verbosity=0}}{}
Creates a new session. In case one already exists it will be deleted.
\begin{description}
\item[{label}] \leavevmode{[}str{]}
A custom name for the session.

\item[{log\_verbosity}] \leavevmode{[}int{]}
Verbositiy level passed to the logger.

\end{description}

\end{fulllineitems}



\section{settings}
\label{\detokenize{reference:module-pypath.share.settings}}\label{\detokenize{reference:settings}}\index{pypath.share.settings (module)@\spxentry{pypath.share.settings}\spxextra{module}}\index{Defaults (class in pypath.share.settings)@\spxentry{Defaults}\spxextra{class in pypath.share.settings}}

\begin{fulllineitems}
\phantomsection\label{\detokenize{reference:pypath.share.settings.Defaults}}\pysiglinewithargsret{\sphinxbfcode{\sphinxupquote{class }}\sphinxcode{\sphinxupquote{pypath.share.settings.}}\sphinxbfcode{\sphinxupquote{Defaults}}}{\emph{acsn}, \emph{acsn\_names}, \emph{alzpw\_ppi}, \emph{annotations\_mod}, \emph{annotations\_pickle}, \emph{arn}, \emph{basedir}, \emph{cachedir}, \emph{complex\_mod}, \emph{complex\_pickle}, \emph{console\_verbosity}, \emph{curated\_mod}, \emph{curated\_pickle}, \emph{data\_basedir}, \emph{datasets}, \emph{deathdomain}, \emph{default\_name\_types}, \emph{default\_organism}, \emph{dependencies}, \emph{dorothea\_expand\_levels}, \emph{enz\_sub\_mod}, \emph{enz\_sub\_pickle}, \emph{figures\_dir}, \emph{go\_pickle\_cache}, \emph{go\_pickle\_cache\_fname}, \emph{goose\_ancest\_sql}, \emph{goose\_annot\_sql}, \emph{goose\_terms\_sql}, \emph{hpmr\_preprocessed}, \emph{intercell\_mod}, \emph{intercell\_pickle}, \emph{keep\_noref}, \emph{latex\_dir}, \emph{lmpid}, \emph{lncrna\_mrna\_mod}, \emph{lncrna\_mrna\_pickle}, \emph{log\_flush\_interval}, \emph{log\_verbosity}, \emph{mapper\_cleanup\_interval}, \emph{mapping\_use\_cache}, \emph{mirna\_mrna\_mod}, \emph{mirna\_mrna\_pickle}, \emph{module\_name}, \emph{msigdb\_email}, \emph{nci\_pid}, \emph{network\_expand\_complexes}, \emph{network\_extra\_directions}, \emph{network\_keep\_original\_names}, \emph{network\_pickle\_cache}, \emph{nrf2ome}, \emph{old\_dbptm}, \emph{omnipath\_args}, \emph{omnipath\_mod}, \emph{omnipath\_pickle}, \emph{path\_root}, \emph{pickle\_dir}, \emph{ppoint}, \emph{progressbars}, \emph{pubmed\_cache}, \emph{slk01human}, \emph{slk3\_edges}, \emph{slk3\_nodes}, \emph{tables\_dir}, \emph{tf\_mirna\_mod}, \emph{tf\_mirna\_pickle}, \emph{tf\_target\_mod}, \emph{tf\_target\_pickle}, \emph{tfregulons\_levels}, \emph{timestamp\_dirs}, \emph{timestamp\_format}, \emph{trip\_preprocessed}, \emph{uniprot\_uploadlists\_chunk\_size}, \emph{use\_intermediate\_cache}, \emph{webpage\_main}}{}~\index{acsn (pypath.share.settings.Defaults attribute)@\spxentry{acsn}\spxextra{pypath.share.settings.Defaults attribute}}

\begin{fulllineitems}
\phantomsection\label{\detokenize{reference:pypath.share.settings.Defaults.acsn}}\pysigline{\sphinxbfcode{\sphinxupquote{acsn}}}
Alias for field number 0

\end{fulllineitems}

\index{acsn\_names (pypath.share.settings.Defaults attribute)@\spxentry{acsn\_names}\spxextra{pypath.share.settings.Defaults attribute}}

\begin{fulllineitems}
\phantomsection\label{\detokenize{reference:pypath.share.settings.Defaults.acsn_names}}\pysigline{\sphinxbfcode{\sphinxupquote{acsn\_names}}}
Alias for field number 1

\end{fulllineitems}

\index{alzpw\_ppi (pypath.share.settings.Defaults attribute)@\spxentry{alzpw\_ppi}\spxextra{pypath.share.settings.Defaults attribute}}

\begin{fulllineitems}
\phantomsection\label{\detokenize{reference:pypath.share.settings.Defaults.alzpw_ppi}}\pysigline{\sphinxbfcode{\sphinxupquote{alzpw\_ppi}}}
Alias for field number 2

\end{fulllineitems}

\index{annotations\_mod (pypath.share.settings.Defaults attribute)@\spxentry{annotations\_mod}\spxextra{pypath.share.settings.Defaults attribute}}

\begin{fulllineitems}
\phantomsection\label{\detokenize{reference:pypath.share.settings.Defaults.annotations_mod}}\pysigline{\sphinxbfcode{\sphinxupquote{annotations\_mod}}}
Alias for field number 3

\end{fulllineitems}

\index{annotations\_pickle (pypath.share.settings.Defaults attribute)@\spxentry{annotations\_pickle}\spxextra{pypath.share.settings.Defaults attribute}}

\begin{fulllineitems}
\phantomsection\label{\detokenize{reference:pypath.share.settings.Defaults.annotations_pickle}}\pysigline{\sphinxbfcode{\sphinxupquote{annotations\_pickle}}}
Alias for field number 4

\end{fulllineitems}

\index{arn (pypath.share.settings.Defaults attribute)@\spxentry{arn}\spxextra{pypath.share.settings.Defaults attribute}}

\begin{fulllineitems}
\phantomsection\label{\detokenize{reference:pypath.share.settings.Defaults.arn}}\pysigline{\sphinxbfcode{\sphinxupquote{arn}}}
Alias for field number 5

\end{fulllineitems}

\index{basedir (pypath.share.settings.Defaults attribute)@\spxentry{basedir}\spxextra{pypath.share.settings.Defaults attribute}}

\begin{fulllineitems}
\phantomsection\label{\detokenize{reference:pypath.share.settings.Defaults.basedir}}\pysigline{\sphinxbfcode{\sphinxupquote{basedir}}}
Alias for field number 6

\end{fulllineitems}

\index{cachedir (pypath.share.settings.Defaults attribute)@\spxentry{cachedir}\spxextra{pypath.share.settings.Defaults attribute}}

\begin{fulllineitems}
\phantomsection\label{\detokenize{reference:pypath.share.settings.Defaults.cachedir}}\pysigline{\sphinxbfcode{\sphinxupquote{cachedir}}}
Alias for field number 7

\end{fulllineitems}

\index{complex\_mod (pypath.share.settings.Defaults attribute)@\spxentry{complex\_mod}\spxextra{pypath.share.settings.Defaults attribute}}

\begin{fulllineitems}
\phantomsection\label{\detokenize{reference:pypath.share.settings.Defaults.complex_mod}}\pysigline{\sphinxbfcode{\sphinxupquote{complex\_mod}}}
Alias for field number 8

\end{fulllineitems}

\index{complex\_pickle (pypath.share.settings.Defaults attribute)@\spxentry{complex\_pickle}\spxextra{pypath.share.settings.Defaults attribute}}

\begin{fulllineitems}
\phantomsection\label{\detokenize{reference:pypath.share.settings.Defaults.complex_pickle}}\pysigline{\sphinxbfcode{\sphinxupquote{complex\_pickle}}}
Alias for field number 9

\end{fulllineitems}

\index{console\_verbosity (pypath.share.settings.Defaults attribute)@\spxentry{console\_verbosity}\spxextra{pypath.share.settings.Defaults attribute}}

\begin{fulllineitems}
\phantomsection\label{\detokenize{reference:pypath.share.settings.Defaults.console_verbosity}}\pysigline{\sphinxbfcode{\sphinxupquote{console\_verbosity}}}
Alias for field number 10

\end{fulllineitems}

\index{curated\_mod (pypath.share.settings.Defaults attribute)@\spxentry{curated\_mod}\spxextra{pypath.share.settings.Defaults attribute}}

\begin{fulllineitems}
\phantomsection\label{\detokenize{reference:pypath.share.settings.Defaults.curated_mod}}\pysigline{\sphinxbfcode{\sphinxupquote{curated\_mod}}}
Alias for field number 11

\end{fulllineitems}

\index{curated\_pickle (pypath.share.settings.Defaults attribute)@\spxentry{curated\_pickle}\spxextra{pypath.share.settings.Defaults attribute}}

\begin{fulllineitems}
\phantomsection\label{\detokenize{reference:pypath.share.settings.Defaults.curated_pickle}}\pysigline{\sphinxbfcode{\sphinxupquote{curated\_pickle}}}
Alias for field number 12

\end{fulllineitems}

\index{data\_basedir (pypath.share.settings.Defaults attribute)@\spxentry{data\_basedir}\spxextra{pypath.share.settings.Defaults attribute}}

\begin{fulllineitems}
\phantomsection\label{\detokenize{reference:pypath.share.settings.Defaults.data_basedir}}\pysigline{\sphinxbfcode{\sphinxupquote{data\_basedir}}}
Alias for field number 13

\end{fulllineitems}

\index{datasets (pypath.share.settings.Defaults attribute)@\spxentry{datasets}\spxextra{pypath.share.settings.Defaults attribute}}

\begin{fulllineitems}
\phantomsection\label{\detokenize{reference:pypath.share.settings.Defaults.datasets}}\pysigline{\sphinxbfcode{\sphinxupquote{datasets}}}
Alias for field number 14

\end{fulllineitems}

\index{deathdomain (pypath.share.settings.Defaults attribute)@\spxentry{deathdomain}\spxextra{pypath.share.settings.Defaults attribute}}

\begin{fulllineitems}
\phantomsection\label{\detokenize{reference:pypath.share.settings.Defaults.deathdomain}}\pysigline{\sphinxbfcode{\sphinxupquote{deathdomain}}}
Alias for field number 15

\end{fulllineitems}

\index{default\_name\_types (pypath.share.settings.Defaults attribute)@\spxentry{default\_name\_types}\spxextra{pypath.share.settings.Defaults attribute}}

\begin{fulllineitems}
\phantomsection\label{\detokenize{reference:pypath.share.settings.Defaults.default_name_types}}\pysigline{\sphinxbfcode{\sphinxupquote{default\_name\_types}}}
Alias for field number 16

\end{fulllineitems}

\index{default\_organism (pypath.share.settings.Defaults attribute)@\spxentry{default\_organism}\spxextra{pypath.share.settings.Defaults attribute}}

\begin{fulllineitems}
\phantomsection\label{\detokenize{reference:pypath.share.settings.Defaults.default_organism}}\pysigline{\sphinxbfcode{\sphinxupquote{default\_organism}}}
Alias for field number 17

\end{fulllineitems}

\index{dependencies (pypath.share.settings.Defaults attribute)@\spxentry{dependencies}\spxextra{pypath.share.settings.Defaults attribute}}

\begin{fulllineitems}
\phantomsection\label{\detokenize{reference:pypath.share.settings.Defaults.dependencies}}\pysigline{\sphinxbfcode{\sphinxupquote{dependencies}}}
Alias for field number 18

\end{fulllineitems}

\index{dorothea\_expand\_levels (pypath.share.settings.Defaults attribute)@\spxentry{dorothea\_expand\_levels}\spxextra{pypath.share.settings.Defaults attribute}}

\begin{fulllineitems}
\phantomsection\label{\detokenize{reference:pypath.share.settings.Defaults.dorothea_expand_levels}}\pysigline{\sphinxbfcode{\sphinxupquote{dorothea\_expand\_levels}}}
Alias for field number 19

\end{fulllineitems}

\index{enz\_sub\_mod (pypath.share.settings.Defaults attribute)@\spxentry{enz\_sub\_mod}\spxextra{pypath.share.settings.Defaults attribute}}

\begin{fulllineitems}
\phantomsection\label{\detokenize{reference:pypath.share.settings.Defaults.enz_sub_mod}}\pysigline{\sphinxbfcode{\sphinxupquote{enz\_sub\_mod}}}
Alias for field number 20

\end{fulllineitems}

\index{enz\_sub\_pickle (pypath.share.settings.Defaults attribute)@\spxentry{enz\_sub\_pickle}\spxextra{pypath.share.settings.Defaults attribute}}

\begin{fulllineitems}
\phantomsection\label{\detokenize{reference:pypath.share.settings.Defaults.enz_sub_pickle}}\pysigline{\sphinxbfcode{\sphinxupquote{enz\_sub\_pickle}}}
Alias for field number 21

\end{fulllineitems}

\index{figures\_dir (pypath.share.settings.Defaults attribute)@\spxentry{figures\_dir}\spxextra{pypath.share.settings.Defaults attribute}}

\begin{fulllineitems}
\phantomsection\label{\detokenize{reference:pypath.share.settings.Defaults.figures_dir}}\pysigline{\sphinxbfcode{\sphinxupquote{figures\_dir}}}
Alias for field number 22

\end{fulllineitems}

\index{go\_pickle\_cache (pypath.share.settings.Defaults attribute)@\spxentry{go\_pickle\_cache}\spxextra{pypath.share.settings.Defaults attribute}}

\begin{fulllineitems}
\phantomsection\label{\detokenize{reference:pypath.share.settings.Defaults.go_pickle_cache}}\pysigline{\sphinxbfcode{\sphinxupquote{go\_pickle\_cache}}}
Alias for field number 23

\end{fulllineitems}

\index{go\_pickle\_cache\_fname (pypath.share.settings.Defaults attribute)@\spxentry{go\_pickle\_cache\_fname}\spxextra{pypath.share.settings.Defaults attribute}}

\begin{fulllineitems}
\phantomsection\label{\detokenize{reference:pypath.share.settings.Defaults.go_pickle_cache_fname}}\pysigline{\sphinxbfcode{\sphinxupquote{go\_pickle\_cache\_fname}}}
Alias for field number 24

\end{fulllineitems}

\index{goose\_ancest\_sql (pypath.share.settings.Defaults attribute)@\spxentry{goose\_ancest\_sql}\spxextra{pypath.share.settings.Defaults attribute}}

\begin{fulllineitems}
\phantomsection\label{\detokenize{reference:pypath.share.settings.Defaults.goose_ancest_sql}}\pysigline{\sphinxbfcode{\sphinxupquote{goose\_ancest\_sql}}}
Alias for field number 25

\end{fulllineitems}

\index{goose\_annot\_sql (pypath.share.settings.Defaults attribute)@\spxentry{goose\_annot\_sql}\spxextra{pypath.share.settings.Defaults attribute}}

\begin{fulllineitems}
\phantomsection\label{\detokenize{reference:pypath.share.settings.Defaults.goose_annot_sql}}\pysigline{\sphinxbfcode{\sphinxupquote{goose\_annot\_sql}}}
Alias for field number 26

\end{fulllineitems}

\index{goose\_terms\_sql (pypath.share.settings.Defaults attribute)@\spxentry{goose\_terms\_sql}\spxextra{pypath.share.settings.Defaults attribute}}

\begin{fulllineitems}
\phantomsection\label{\detokenize{reference:pypath.share.settings.Defaults.goose_terms_sql}}\pysigline{\sphinxbfcode{\sphinxupquote{goose\_terms\_sql}}}
Alias for field number 27

\end{fulllineitems}

\index{hpmr\_preprocessed (pypath.share.settings.Defaults attribute)@\spxentry{hpmr\_preprocessed}\spxextra{pypath.share.settings.Defaults attribute}}

\begin{fulllineitems}
\phantomsection\label{\detokenize{reference:pypath.share.settings.Defaults.hpmr_preprocessed}}\pysigline{\sphinxbfcode{\sphinxupquote{hpmr\_preprocessed}}}
Alias for field number 28

\end{fulllineitems}

\index{intercell\_mod (pypath.share.settings.Defaults attribute)@\spxentry{intercell\_mod}\spxextra{pypath.share.settings.Defaults attribute}}

\begin{fulllineitems}
\phantomsection\label{\detokenize{reference:pypath.share.settings.Defaults.intercell_mod}}\pysigline{\sphinxbfcode{\sphinxupquote{intercell\_mod}}}
Alias for field number 29

\end{fulllineitems}

\index{intercell\_pickle (pypath.share.settings.Defaults attribute)@\spxentry{intercell\_pickle}\spxextra{pypath.share.settings.Defaults attribute}}

\begin{fulllineitems}
\phantomsection\label{\detokenize{reference:pypath.share.settings.Defaults.intercell_pickle}}\pysigline{\sphinxbfcode{\sphinxupquote{intercell\_pickle}}}
Alias for field number 30

\end{fulllineitems}

\index{keep\_noref (pypath.share.settings.Defaults attribute)@\spxentry{keep\_noref}\spxextra{pypath.share.settings.Defaults attribute}}

\begin{fulllineitems}
\phantomsection\label{\detokenize{reference:pypath.share.settings.Defaults.keep_noref}}\pysigline{\sphinxbfcode{\sphinxupquote{keep\_noref}}}
Alias for field number 31

\end{fulllineitems}

\index{latex\_dir (pypath.share.settings.Defaults attribute)@\spxentry{latex\_dir}\spxextra{pypath.share.settings.Defaults attribute}}

\begin{fulllineitems}
\phantomsection\label{\detokenize{reference:pypath.share.settings.Defaults.latex_dir}}\pysigline{\sphinxbfcode{\sphinxupquote{latex\_dir}}}
Alias for field number 32

\end{fulllineitems}

\index{lmpid (pypath.share.settings.Defaults attribute)@\spxentry{lmpid}\spxextra{pypath.share.settings.Defaults attribute}}

\begin{fulllineitems}
\phantomsection\label{\detokenize{reference:pypath.share.settings.Defaults.lmpid}}\pysigline{\sphinxbfcode{\sphinxupquote{lmpid}}}
Alias for field number 33

\end{fulllineitems}

\index{lncrna\_mrna\_mod (pypath.share.settings.Defaults attribute)@\spxentry{lncrna\_mrna\_mod}\spxextra{pypath.share.settings.Defaults attribute}}

\begin{fulllineitems}
\phantomsection\label{\detokenize{reference:pypath.share.settings.Defaults.lncrna_mrna_mod}}\pysigline{\sphinxbfcode{\sphinxupquote{lncrna\_mrna\_mod}}}
Alias for field number 34

\end{fulllineitems}

\index{lncrna\_mrna\_pickle (pypath.share.settings.Defaults attribute)@\spxentry{lncrna\_mrna\_pickle}\spxextra{pypath.share.settings.Defaults attribute}}

\begin{fulllineitems}
\phantomsection\label{\detokenize{reference:pypath.share.settings.Defaults.lncrna_mrna_pickle}}\pysigline{\sphinxbfcode{\sphinxupquote{lncrna\_mrna\_pickle}}}
Alias for field number 35

\end{fulllineitems}

\index{log\_flush\_interval (pypath.share.settings.Defaults attribute)@\spxentry{log\_flush\_interval}\spxextra{pypath.share.settings.Defaults attribute}}

\begin{fulllineitems}
\phantomsection\label{\detokenize{reference:pypath.share.settings.Defaults.log_flush_interval}}\pysigline{\sphinxbfcode{\sphinxupquote{log\_flush\_interval}}}
Alias for field number 36

\end{fulllineitems}

\index{log\_verbosity (pypath.share.settings.Defaults attribute)@\spxentry{log\_verbosity}\spxextra{pypath.share.settings.Defaults attribute}}

\begin{fulllineitems}
\phantomsection\label{\detokenize{reference:pypath.share.settings.Defaults.log_verbosity}}\pysigline{\sphinxbfcode{\sphinxupquote{log\_verbosity}}}
Alias for field number 37

\end{fulllineitems}

\index{mapper\_cleanup\_interval (pypath.share.settings.Defaults attribute)@\spxentry{mapper\_cleanup\_interval}\spxextra{pypath.share.settings.Defaults attribute}}

\begin{fulllineitems}
\phantomsection\label{\detokenize{reference:pypath.share.settings.Defaults.mapper_cleanup_interval}}\pysigline{\sphinxbfcode{\sphinxupquote{mapper\_cleanup\_interval}}}
Alias for field number 38

\end{fulllineitems}

\index{mapping\_use\_cache (pypath.share.settings.Defaults attribute)@\spxentry{mapping\_use\_cache}\spxextra{pypath.share.settings.Defaults attribute}}

\begin{fulllineitems}
\phantomsection\label{\detokenize{reference:pypath.share.settings.Defaults.mapping_use_cache}}\pysigline{\sphinxbfcode{\sphinxupquote{mapping\_use\_cache}}}
Alias for field number 39

\end{fulllineitems}

\index{mirna\_mrna\_mod (pypath.share.settings.Defaults attribute)@\spxentry{mirna\_mrna\_mod}\spxextra{pypath.share.settings.Defaults attribute}}

\begin{fulllineitems}
\phantomsection\label{\detokenize{reference:pypath.share.settings.Defaults.mirna_mrna_mod}}\pysigline{\sphinxbfcode{\sphinxupquote{mirna\_mrna\_mod}}}
Alias for field number 40

\end{fulllineitems}

\index{mirna\_mrna\_pickle (pypath.share.settings.Defaults attribute)@\spxentry{mirna\_mrna\_pickle}\spxextra{pypath.share.settings.Defaults attribute}}

\begin{fulllineitems}
\phantomsection\label{\detokenize{reference:pypath.share.settings.Defaults.mirna_mrna_pickle}}\pysigline{\sphinxbfcode{\sphinxupquote{mirna\_mrna\_pickle}}}
Alias for field number 41

\end{fulllineitems}

\index{module\_name (pypath.share.settings.Defaults attribute)@\spxentry{module\_name}\spxextra{pypath.share.settings.Defaults attribute}}

\begin{fulllineitems}
\phantomsection\label{\detokenize{reference:pypath.share.settings.Defaults.module_name}}\pysigline{\sphinxbfcode{\sphinxupquote{module\_name}}}
Alias for field number 42

\end{fulllineitems}

\index{msigdb\_email (pypath.share.settings.Defaults attribute)@\spxentry{msigdb\_email}\spxextra{pypath.share.settings.Defaults attribute}}

\begin{fulllineitems}
\phantomsection\label{\detokenize{reference:pypath.share.settings.Defaults.msigdb_email}}\pysigline{\sphinxbfcode{\sphinxupquote{msigdb\_email}}}
Alias for field number 43

\end{fulllineitems}

\index{nci\_pid (pypath.share.settings.Defaults attribute)@\spxentry{nci\_pid}\spxextra{pypath.share.settings.Defaults attribute}}

\begin{fulllineitems}
\phantomsection\label{\detokenize{reference:pypath.share.settings.Defaults.nci_pid}}\pysigline{\sphinxbfcode{\sphinxupquote{nci\_pid}}}
Alias for field number 44

\end{fulllineitems}

\index{network\_expand\_complexes (pypath.share.settings.Defaults attribute)@\spxentry{network\_expand\_complexes}\spxextra{pypath.share.settings.Defaults attribute}}

\begin{fulllineitems}
\phantomsection\label{\detokenize{reference:pypath.share.settings.Defaults.network_expand_complexes}}\pysigline{\sphinxbfcode{\sphinxupquote{network\_expand\_complexes}}}
Alias for field number 45

\end{fulllineitems}

\index{network\_extra\_directions (pypath.share.settings.Defaults attribute)@\spxentry{network\_extra\_directions}\spxextra{pypath.share.settings.Defaults attribute}}

\begin{fulllineitems}
\phantomsection\label{\detokenize{reference:pypath.share.settings.Defaults.network_extra_directions}}\pysigline{\sphinxbfcode{\sphinxupquote{network\_extra\_directions}}}
Alias for field number 46

\end{fulllineitems}

\index{network\_keep\_original\_names (pypath.share.settings.Defaults attribute)@\spxentry{network\_keep\_original\_names}\spxextra{pypath.share.settings.Defaults attribute}}

\begin{fulllineitems}
\phantomsection\label{\detokenize{reference:pypath.share.settings.Defaults.network_keep_original_names}}\pysigline{\sphinxbfcode{\sphinxupquote{network\_keep\_original\_names}}}
Alias for field number 47

\end{fulllineitems}

\index{network\_pickle\_cache (pypath.share.settings.Defaults attribute)@\spxentry{network\_pickle\_cache}\spxextra{pypath.share.settings.Defaults attribute}}

\begin{fulllineitems}
\phantomsection\label{\detokenize{reference:pypath.share.settings.Defaults.network_pickle_cache}}\pysigline{\sphinxbfcode{\sphinxupquote{network\_pickle\_cache}}}
Alias for field number 48

\end{fulllineitems}

\index{nrf2ome (pypath.share.settings.Defaults attribute)@\spxentry{nrf2ome}\spxextra{pypath.share.settings.Defaults attribute}}

\begin{fulllineitems}
\phantomsection\label{\detokenize{reference:pypath.share.settings.Defaults.nrf2ome}}\pysigline{\sphinxbfcode{\sphinxupquote{nrf2ome}}}
Alias for field number 49

\end{fulllineitems}

\index{old\_dbptm (pypath.share.settings.Defaults attribute)@\spxentry{old\_dbptm}\spxextra{pypath.share.settings.Defaults attribute}}

\begin{fulllineitems}
\phantomsection\label{\detokenize{reference:pypath.share.settings.Defaults.old_dbptm}}\pysigline{\sphinxbfcode{\sphinxupquote{old\_dbptm}}}
Alias for field number 50

\end{fulllineitems}

\index{omnipath\_args (pypath.share.settings.Defaults attribute)@\spxentry{omnipath\_args}\spxextra{pypath.share.settings.Defaults attribute}}

\begin{fulllineitems}
\phantomsection\label{\detokenize{reference:pypath.share.settings.Defaults.omnipath_args}}\pysigline{\sphinxbfcode{\sphinxupquote{omnipath\_args}}}
Alias for field number 51

\end{fulllineitems}

\index{omnipath\_mod (pypath.share.settings.Defaults attribute)@\spxentry{omnipath\_mod}\spxextra{pypath.share.settings.Defaults attribute}}

\begin{fulllineitems}
\phantomsection\label{\detokenize{reference:pypath.share.settings.Defaults.omnipath_mod}}\pysigline{\sphinxbfcode{\sphinxupquote{omnipath\_mod}}}
Alias for field number 52

\end{fulllineitems}

\index{omnipath\_pickle (pypath.share.settings.Defaults attribute)@\spxentry{omnipath\_pickle}\spxextra{pypath.share.settings.Defaults attribute}}

\begin{fulllineitems}
\phantomsection\label{\detokenize{reference:pypath.share.settings.Defaults.omnipath_pickle}}\pysigline{\sphinxbfcode{\sphinxupquote{omnipath\_pickle}}}
Alias for field number 53

\end{fulllineitems}

\index{path\_root (pypath.share.settings.Defaults attribute)@\spxentry{path\_root}\spxextra{pypath.share.settings.Defaults attribute}}

\begin{fulllineitems}
\phantomsection\label{\detokenize{reference:pypath.share.settings.Defaults.path_root}}\pysigline{\sphinxbfcode{\sphinxupquote{path\_root}}}
Alias for field number 54

\end{fulllineitems}

\index{pickle\_dir (pypath.share.settings.Defaults attribute)@\spxentry{pickle\_dir}\spxextra{pypath.share.settings.Defaults attribute}}

\begin{fulllineitems}
\phantomsection\label{\detokenize{reference:pypath.share.settings.Defaults.pickle_dir}}\pysigline{\sphinxbfcode{\sphinxupquote{pickle\_dir}}}
Alias for field number 55

\end{fulllineitems}

\index{ppoint (pypath.share.settings.Defaults attribute)@\spxentry{ppoint}\spxextra{pypath.share.settings.Defaults attribute}}

\begin{fulllineitems}
\phantomsection\label{\detokenize{reference:pypath.share.settings.Defaults.ppoint}}\pysigline{\sphinxbfcode{\sphinxupquote{ppoint}}}
Alias for field number 56

\end{fulllineitems}

\index{progressbars (pypath.share.settings.Defaults attribute)@\spxentry{progressbars}\spxextra{pypath.share.settings.Defaults attribute}}

\begin{fulllineitems}
\phantomsection\label{\detokenize{reference:pypath.share.settings.Defaults.progressbars}}\pysigline{\sphinxbfcode{\sphinxupquote{progressbars}}}
Alias for field number 57

\end{fulllineitems}

\index{pubmed\_cache (pypath.share.settings.Defaults attribute)@\spxentry{pubmed\_cache}\spxextra{pypath.share.settings.Defaults attribute}}

\begin{fulllineitems}
\phantomsection\label{\detokenize{reference:pypath.share.settings.Defaults.pubmed_cache}}\pysigline{\sphinxbfcode{\sphinxupquote{pubmed\_cache}}}
Alias for field number 58

\end{fulllineitems}

\index{slk01human (pypath.share.settings.Defaults attribute)@\spxentry{slk01human}\spxextra{pypath.share.settings.Defaults attribute}}

\begin{fulllineitems}
\phantomsection\label{\detokenize{reference:pypath.share.settings.Defaults.slk01human}}\pysigline{\sphinxbfcode{\sphinxupquote{slk01human}}}
Alias for field number 59

\end{fulllineitems}

\index{slk3\_edges (pypath.share.settings.Defaults attribute)@\spxentry{slk3\_edges}\spxextra{pypath.share.settings.Defaults attribute}}

\begin{fulllineitems}
\phantomsection\label{\detokenize{reference:pypath.share.settings.Defaults.slk3_edges}}\pysigline{\sphinxbfcode{\sphinxupquote{slk3\_edges}}}
Alias for field number 60

\end{fulllineitems}

\index{slk3\_nodes (pypath.share.settings.Defaults attribute)@\spxentry{slk3\_nodes}\spxextra{pypath.share.settings.Defaults attribute}}

\begin{fulllineitems}
\phantomsection\label{\detokenize{reference:pypath.share.settings.Defaults.slk3_nodes}}\pysigline{\sphinxbfcode{\sphinxupquote{slk3\_nodes}}}
Alias for field number 61

\end{fulllineitems}

\index{tables\_dir (pypath.share.settings.Defaults attribute)@\spxentry{tables\_dir}\spxextra{pypath.share.settings.Defaults attribute}}

\begin{fulllineitems}
\phantomsection\label{\detokenize{reference:pypath.share.settings.Defaults.tables_dir}}\pysigline{\sphinxbfcode{\sphinxupquote{tables\_dir}}}
Alias for field number 62

\end{fulllineitems}

\index{tf\_mirna\_mod (pypath.share.settings.Defaults attribute)@\spxentry{tf\_mirna\_mod}\spxextra{pypath.share.settings.Defaults attribute}}

\begin{fulllineitems}
\phantomsection\label{\detokenize{reference:pypath.share.settings.Defaults.tf_mirna_mod}}\pysigline{\sphinxbfcode{\sphinxupquote{tf\_mirna\_mod}}}
Alias for field number 63

\end{fulllineitems}

\index{tf\_mirna\_pickle (pypath.share.settings.Defaults attribute)@\spxentry{tf\_mirna\_pickle}\spxextra{pypath.share.settings.Defaults attribute}}

\begin{fulllineitems}
\phantomsection\label{\detokenize{reference:pypath.share.settings.Defaults.tf_mirna_pickle}}\pysigline{\sphinxbfcode{\sphinxupquote{tf\_mirna\_pickle}}}
Alias for field number 64

\end{fulllineitems}

\index{tf\_target\_mod (pypath.share.settings.Defaults attribute)@\spxentry{tf\_target\_mod}\spxextra{pypath.share.settings.Defaults attribute}}

\begin{fulllineitems}
\phantomsection\label{\detokenize{reference:pypath.share.settings.Defaults.tf_target_mod}}\pysigline{\sphinxbfcode{\sphinxupquote{tf\_target\_mod}}}
Alias for field number 65

\end{fulllineitems}

\index{tf\_target\_pickle (pypath.share.settings.Defaults attribute)@\spxentry{tf\_target\_pickle}\spxextra{pypath.share.settings.Defaults attribute}}

\begin{fulllineitems}
\phantomsection\label{\detokenize{reference:pypath.share.settings.Defaults.tf_target_pickle}}\pysigline{\sphinxbfcode{\sphinxupquote{tf\_target\_pickle}}}
Alias for field number 66

\end{fulllineitems}

\index{tfregulons\_levels (pypath.share.settings.Defaults attribute)@\spxentry{tfregulons\_levels}\spxextra{pypath.share.settings.Defaults attribute}}

\begin{fulllineitems}
\phantomsection\label{\detokenize{reference:pypath.share.settings.Defaults.tfregulons_levels}}\pysigline{\sphinxbfcode{\sphinxupquote{tfregulons\_levels}}}
Alias for field number 67

\end{fulllineitems}

\index{timestamp\_dirs (pypath.share.settings.Defaults attribute)@\spxentry{timestamp\_dirs}\spxextra{pypath.share.settings.Defaults attribute}}

\begin{fulllineitems}
\phantomsection\label{\detokenize{reference:pypath.share.settings.Defaults.timestamp_dirs}}\pysigline{\sphinxbfcode{\sphinxupquote{timestamp\_dirs}}}
Alias for field number 68

\end{fulllineitems}

\index{timestamp\_format (pypath.share.settings.Defaults attribute)@\spxentry{timestamp\_format}\spxextra{pypath.share.settings.Defaults attribute}}

\begin{fulllineitems}
\phantomsection\label{\detokenize{reference:pypath.share.settings.Defaults.timestamp_format}}\pysigline{\sphinxbfcode{\sphinxupquote{timestamp\_format}}}
Alias for field number 69

\end{fulllineitems}

\index{trip\_preprocessed (pypath.share.settings.Defaults attribute)@\spxentry{trip\_preprocessed}\spxextra{pypath.share.settings.Defaults attribute}}

\begin{fulllineitems}
\phantomsection\label{\detokenize{reference:pypath.share.settings.Defaults.trip_preprocessed}}\pysigline{\sphinxbfcode{\sphinxupquote{trip\_preprocessed}}}
Alias for field number 70

\end{fulllineitems}

\index{uniprot\_uploadlists\_chunk\_size (pypath.share.settings.Defaults attribute)@\spxentry{uniprot\_uploadlists\_chunk\_size}\spxextra{pypath.share.settings.Defaults attribute}}

\begin{fulllineitems}
\phantomsection\label{\detokenize{reference:pypath.share.settings.Defaults.uniprot_uploadlists_chunk_size}}\pysigline{\sphinxbfcode{\sphinxupquote{uniprot\_uploadlists\_chunk\_size}}}
Alias for field number 71

\end{fulllineitems}

\index{use\_intermediate\_cache (pypath.share.settings.Defaults attribute)@\spxentry{use\_intermediate\_cache}\spxextra{pypath.share.settings.Defaults attribute}}

\begin{fulllineitems}
\phantomsection\label{\detokenize{reference:pypath.share.settings.Defaults.use_intermediate_cache}}\pysigline{\sphinxbfcode{\sphinxupquote{use\_intermediate\_cache}}}
Alias for field number 72

\end{fulllineitems}

\index{webpage\_main (pypath.share.settings.Defaults attribute)@\spxentry{webpage\_main}\spxextra{pypath.share.settings.Defaults attribute}}

\begin{fulllineitems}
\phantomsection\label{\detokenize{reference:pypath.share.settings.Defaults.webpage_main}}\pysigline{\sphinxbfcode{\sphinxupquote{webpage\_main}}}
Alias for field number 73

\end{fulllineitems}


\end{fulllineitems}



\section{taxonomy}
\label{\detokenize{reference:module-pypath.utils.taxonomy}}\label{\detokenize{reference:taxonomy}}\index{pypath.utils.taxonomy (module)@\spxentry{pypath.utils.taxonomy}\spxextra{module}}

\section{unichem}
\label{\detokenize{reference:module-pypath.utils.unichem}}\label{\detokenize{reference:unichem}}\index{pypath.utils.unichem (module)@\spxentry{pypath.utils.unichem}\spxextra{module}}

\section{uniprot}
\label{\detokenize{reference:uniprot}}

\section{urls}
\label{\detokenize{reference:module-pypath.resources.urls}}\label{\detokenize{reference:urls}}\index{pypath.resources.urls (module)@\spxentry{pypath.resources.urls}\spxextra{module}}

\section{build}
\label{\detokenize{reference:module-pypath.omnipath.server.build}}\label{\detokenize{reference:build}}\index{pypath.omnipath.server.build (module)@\spxentry{pypath.omnipath.server.build}\spxextra{module}}
This is a standalone module with the only purpose of
building the tables for the webservice.
\index{WebserviceTables (class in pypath.omnipath.server.build)@\spxentry{WebserviceTables}\spxextra{class in pypath.omnipath.server.build}}

\begin{fulllineitems}
\phantomsection\label{\detokenize{reference:pypath.omnipath.server.build.WebserviceTables}}\pysiglinewithargsret{\sphinxbfcode{\sphinxupquote{class }}\sphinxcode{\sphinxupquote{pypath.omnipath.server.build.}}\sphinxbfcode{\sphinxupquote{WebserviceTables}}}{\emph{only\_human=False}, \emph{outfile\_interactions='omnipath\_webservice\_interactions.tsv'}, \emph{outfile\_ptms='omnipath\_webservice\_enz\_sub.tsv'}, \emph{outfile\_complexes='omnipath\_webservice\_complexes.tsv'}, \emph{outfile\_annotations='omnipath\_webservice\_annotations.tsv'}, \emph{outfile\_intercell='omnipath\_webservice\_intercell.tsv'}, \emph{network\_datasets=None}}{}
Creates the data frames which the web service uses to serve the data from.

\end{fulllineitems}



\section{resources}
\label{\detokenize{reference:resources}}

\subsection{network}
\label{\detokenize{reference:id173}}\phantomsection\label{\detokenize{reference:module-pypath.resources.network}}\index{pypath.resources.network (module)@\spxentry{pypath.resources.network}\spxextra{module}}\index{dorothea\_expand\_levels() (in module pypath.resources.network)@\spxentry{dorothea\_expand\_levels()}\spxextra{in module pypath.resources.network}}

\begin{fulllineitems}
\phantomsection\label{\detokenize{reference:pypath.resources.network.dorothea_expand_levels}}\pysiglinewithargsret{\sphinxcode{\sphinxupquote{pypath.resources.network.}}\sphinxbfcode{\sphinxupquote{dorothea\_expand\_levels}}}{\emph{resources=None}, \emph{levels=None}}{}
In a dictionary of resource definitions creates a separate
\sphinxcode{\sphinxupquote{NetworkResource}} object for each confidence levels of DoRothEA
just like each level was a different resource.

No matter \sphinxcode{\sphinxupquote{resources}} is a \sphinxcode{\sphinxupquote{NetworkResource}} or a dict of network
resources, returns always a dict of network resources.

\end{fulllineitems}



\chapter{Webservice}
\label{\detokenize{webservice:webservice}}\label{\detokenize{webservice::doc}}
\sphinxstylestrong{New webservice} from 14 June 2018: the queries slightly changed, have been
largely extended. See the examples below.

One instance of the pypath webservice runs at the domain
\sphinxurl{http://omnipathdb.org/}, serving not only the OmniPath data but other datasets:
TF-target interactions from TF Regulons, a large collection additional
enzyme-substrate interactions, and literature curated miRNA-mRNA interacions
combined from 4 databases. The webservice implements a very simple REST style
API, you can make requests by HTTP protocol (browser, wget, curl or whatever).

The webservice currently recognizes 3 types of queries: \sphinxcode{\sphinxupquote{interactions}},
\sphinxcode{\sphinxupquote{ptms}} and \sphinxcode{\sphinxupquote{info}}. The query types \sphinxcode{\sphinxupquote{resources}}, \sphinxcode{\sphinxupquote{network}} and
\sphinxcode{\sphinxupquote{about}} have not been implemented yet in the new webservice.


\section{Mouse and rat}
\label{\detokenize{webservice:mouse-and-rat}}
Except the miRNA interactions all interactions are available for human, mouse
and rat. The rodent data has been translated from human using the NCBI
Homologene database. Many human proteins have no known homolog in rodents
hence rodent datasets are smaller than their human counterparts. Note, if you
work with mouse omics data you might do better to translate your dataset to
human (for example using the \sphinxcode{\sphinxupquote{pypath.homology}} module) and use human
interaction data.


\section{Examples}
\label{\detokenize{webservice:examples}}
A request without any parameter, gives some basic numbers about the actual
loaded dataset:
\begin{quote}

\sphinxurl{http://omnipathdb.org}
\end{quote}

The \sphinxcode{\sphinxupquote{info}} returns a HTML page with comprehensive information about the
resources:
\begin{quote}

\sphinxurl{http://omnipathdb.org/info}
\end{quote}

The \sphinxcode{\sphinxupquote{interactions}} query accepts some parameters and returns interactions in
tabular format. This example returns all interactions of EGFR (P00533), with
sources and references listed.
\begin{quote}

\sphinxurl{http://omnipathdb.org/interactions/?partners=P00533\&fields=sources,references}
\end{quote}

By default only the OmniPath dataset used, to query the TF Regulons or add the
extra enzyme-substrate interactions you need to set additional parameters. For
example to query the transcriptional regulators of EGFR:
\begin{quote}

\sphinxurl{http://omnipathdb.org/interactions/?targets=EGFR\&types=TF}
\end{quote}

The TF Regulons database assigns confidence levels to the interactions. You
might want to select only the highest confidence, \sphinxstyleemphasis{A} category:
\begin{quote}

\sphinxurl{http://omnipathdb.org/interactions/?targets=EGFR\&types=TF\&tfregulons\_levels=A}
\end{quote}

Show the transcriptional targets of Smad2 homology translated to rat including
the confidence levels from TF Regulons:
\begin{quote}

\sphinxurl{http://omnipathdb.org/interactions/?genesymbols=1\&fields=type,ncbi\_tax\_id,tfregulons\_level\&organisms=10116\&sources=Smad2\&types=TF}
\end{quote}

Query interactions from PhosphoNetworks which is part of the \sphinxstyleemphasis{kinaseextra}
dataset:
\begin{quote}

\sphinxurl{http://omnipathdb.org/interactions/?genesymbols=1\&fields=sources\&databases=PhosphoNetworks\&datasets=kinaseextra}
\end{quote}

Get the interactions from Signor, SPIKE and SignaLink3:
\begin{quote}

\sphinxurl{http://omnipathdb.org/interactions/?genesymbols=1\&fields=sources,references\&databases=Signor,SPIKE,SignaLink3}
\end{quote}

All interactions of MAP1LC3B:
\begin{quote}

\sphinxurl{http://omnipathdb.org/interactions/?genesymbols=1\&partners=MAP1LC3B}
\end{quote}

By default \sphinxcode{\sphinxupquote{partners}} queries the interaction where either the source or the
arget is among the partners. If you set the \sphinxcode{\sphinxupquote{source\_target}} parameter to
\sphinxcode{\sphinxupquote{AND}} both the source and the target must be in the queried set:
\begin{quote}

\sphinxurl{http://omnipathdb.org/interactions/?genesymbols=1\&fields=sources,references\&sources=ATG3,ATG7,ATG4B,SQSTM1\&targets=MAP1LC3B,MAP1LC3A,MAP1LC3C,Q9H0R8,GABARAP,GABARAPL2\&source\_target=AND}
\end{quote}

As you see above you can use UniProt IDs and Gene Symbols in the queries and
also mix them. Get the miRNA regulating NOTCH1:
\begin{quote}

\sphinxurl{http://omnipathdb.org/interactions/?genesymbols=1\&fields=sources,references\&datasets=mirnatarget\&targets=NOTCH1}
\end{quote}

Note: with the exception of mandatory fields and genesymbols, the columns
appear exactly in the order you provided in your query.

Another query type available is \sphinxcode{\sphinxupquote{ptms}} which provides enzyme-substrate
interactions. It is very similar to the \sphinxcode{\sphinxupquote{interactions}}:
\begin{quote}

\sphinxurl{http://omnipathdb.org/ptms?genesymbols=1\&fields=sources,references,isoforms\&enzymes=FYN}
\end{quote}

Is there any ubiquitination reaction?
\begin{quote}

\sphinxurl{http://omnipathdb.org/ptms?genesymbols=1\&fields=sources,references\&types=ubiquitination}
\end{quote}

And acetylation in mouse?
\begin{quote}

\sphinxurl{http://omnipathdb.org/ptms?genesymbols=1\&fields=sources,references\&types=acetylation\&organisms=10090}
\end{quote}

Rat interactions, both directly from rat and homology translated from human,
from the PhosphoSite database:
\begin{quote}

\sphinxurl{http://omnipathdb.org/ptms?genesymbols=1\&fields=sources,references\&organisms=10116\&databases=PhosphoSite,PhosphoSite\_noref}
\end{quote}


\chapter{Release history}
\label{\detokenize{changelog:release-history}}\label{\detokenize{changelog::doc}}
Main improvements in the past releases:


\section{0.1.0}
\label{\detokenize{changelog:id1}}\begin{itemize}
\item {} 
First release of pypath, for initial testing.

\end{itemize}


\section{0.2.0}
\label{\detokenize{changelog:id2}}\begin{itemize}
\item {} 
Lots of small improvements in almost every module

\item {} 
Networks can be read from local files, remote files, lists or provided by
any function

\item {} 
Almost all redistributed data have been removed, every source downloaded
from the original provider.

\end{itemize}


\section{0.3.0}
\label{\detokenize{changelog:id3}}\begin{itemize}
\item {} 
First version with partial Python 3 support.

\end{itemize}


\section{0.4.0}
\label{\detokenize{changelog:id4}}\begin{itemize}
\item {} 
\sphinxstylestrong{pyreact} module with \sphinxstylestrong{BioPaxReader} and \sphinxstylestrong{PyReact} classes added

\item {} 
Process description databases, BioPax and PathwayCommons SIF conversion
rules are supported

\item {} 
Format definitions for 6 process description databases included.

\end{itemize}


\section{0.5.0}
\label{\detokenize{changelog:id5}}\begin{itemize}
\item {} 
Many classes have been added to the \sphinxstylestrong{plot} module

\item {} 
All figures and tables in the manuscript can be generated automatically

\item {} 
This is supported by a new module, \sphinxstylestrong{analysis}, which implements a generic
workflow in its \sphinxstylestrong{Workflow} class.

\end{itemize}


\section{0.7.74}
\label{\detokenize{changelog:id6}}\begin{itemize}
\item {} 
\sphinxstylestrong{homology} module: finds the homologs of proteins using the NCBI
Homologene database and the homologs of PTM sites using UniProt sequences
and PhosphoSitePlus homology table

\item {} 
\sphinxstylestrong{ptm} module: fully integrated way of processing enzyme-substrate
interactions from many databases and their translation by homology to other
species

\item {} 
\sphinxstylestrong{export} module: creates \sphinxcode{\sphinxupquote{pandas.DataFrame}} or exports the network into
tabular file

\item {} 
New webservice

\item {} 
TF Regulons database included and provides much more comprehensive
transcriptional regulation resources, including literature curated, in silico
predicted, ChIP-Seq and expression pattern based approaches

\item {} 
Many network resources added, including miRNA-mRNA and TF-miRNA interactions

\end{itemize}


\section{Upcoming}
\label{\detokenize{changelog:upcoming}}\begin{itemize}
\item {} 
New, more flexible network reader class

\item {} 
Full support for multi-species molecular interaction networks
(e.g. pathogene-host)

\item {} 
Better support for not protein only molecular interaction networks
(metabolites, drug compounds, RNA)

\item {} 
ChEMBL webservice interface, interface for PubChem and eventually
forDrugBank

\item {} 
Silent mode: a way to suppress messages and progress bars

\end{itemize}

\sphinxstylestrong{pypath} consists of a number of submodules to build various databases.
Most of these are provided as \sphinxstylestrong{pandas} data frames. The network database
is built around igraph to work with molecular network representations e.g.
protein, miRNA and drug compound interaction networks.


\chapter{Webservice}
\label{\detokenize{index:webservice}}
\sphinxstylestrong{New webservice} from 14 June 2018: the queries slightly changed, have been
largely extended. See the examples below.

The webservice implements a very simple REST style API, you can make requests
by the HTTP protocol (browser, wget, curl or whatever). After defining the
query type and optionally a set of molecular entities (proteins) you can
add further GET parameters encoded in the URL.


\section{Query types}
\label{\detokenize{index:query-types}}
The webservice currently recognizes 7 types of queries: \sphinxcode{\sphinxupquote{interactions}},
\sphinxcode{\sphinxupquote{ptms}}, \sphinxcode{\sphinxupquote{annotations}}, \sphinxcode{\sphinxupquote{complexes}}, \sphinxcode{\sphinxupquote{intercell}}, \sphinxcode{\sphinxupquote{queries}} and
\sphinxcode{\sphinxupquote{info}}.
The query types \sphinxcode{\sphinxupquote{resources}}, \sphinxcode{\sphinxupquote{network}} and \sphinxcode{\sphinxupquote{about}} have not been
implemented yet in the new webservice.


\section{Interaction datasets}
\label{\detokenize{index:interaction-datasets}}
The instance of the \sphinxcode{\sphinxupquote{pypath}} webserver running at the domain
\sphinxurl{http://omnipathdb.org/}, serves not only the OmniPath data but also other
datasets. Each of them has a short name what you can use in the queries
(e.g. \sphinxcode{\sphinxupquote{\&datasets=omnipath,pathwayextra}}).
\begin{itemize}
\item {} 
\sphinxcode{\sphinxupquote{omnipath}}: the OmniPath data as defined in the paper, an arbitrary
optimum between coverage and quality

\item {} 
\sphinxcode{\sphinxupquote{pathwayextra}}: activity flow interactions without literature reference

\item {} 
\sphinxcode{\sphinxupquote{kinaseextra}}: enzyme-substrate interactions without literature reference

\item {} 
\sphinxcode{\sphinxupquote{ligrecextra}}: ligand-receptor interactions without literature reference

\item {} 
\sphinxcode{\sphinxupquote{tfregulons}}: transcription factor (TF)-target interactions from DoRothEA

\item {} 
\sphinxcode{\sphinxupquote{mirnatarget}}: miRNA-mRNA and TF-miRNA interactions

\end{itemize}

TF-target interactions from TF Regulons, a large collection additional
enzyme-substrate interactions, and literature curated miRNA-mRNA interacions
combined from 4 databases.


\section{Mouse and rat}
\label{\detokenize{index:mouse-and-rat}}
Except the miRNA interactions all interactions are available for human, mouse
and rat. The rodent data has been translated from human using the NCBI
Homologene database. Many human proteins do not have known homolog in rodents
hence rodent datasets are smaller than their human counterparts. Note, if you
work with mouse omics data you might do better to translate your dataset to
human (for example using the \sphinxcode{\sphinxupquote{pypath.homology}} module) and use human
interaction data.


\section{Examples}
\label{\detokenize{index:examples}}
A request without any parameter provides the main webpage:
\begin{quote}

\sphinxurl{http://omnipathdb.org}
\end{quote}

The \sphinxcode{\sphinxupquote{info}} returns a HTML page with comprehensive information about the
resources. The list here should be and will be updated as currently OmniPath
includes much more databases:
\begin{quote}

\sphinxurl{http://omnipathdb.org/info}
\end{quote}


\subsection{Molecular interaction network}
\label{\detokenize{index:molecular-interaction-network}}
The \sphinxcode{\sphinxupquote{interactions}} query accepts some parameters and returns interactions in
tabular format. This example returns all interactions of EGFR (P00533), with
sources and references listed.
\begin{quote}

\sphinxurl{http://omnipathdb.org/interactions/?partners=P00533\&fields=sources,references}
\end{quote}

By default only the OmniPath dataset used, to include any other dataset you
have to set additional parameters. For example to query the transcriptional regulators of EGFR:
\begin{quote}

\sphinxurl{http://omnipathdb.org/interactions/?targets=EGFR\&types=TF}
\end{quote}

The TF Regulons database assigns confidence levels to the interactions. You
might want to select only the highest confidence, \sphinxstyleemphasis{A} category:
\begin{quote}

\sphinxurl{http://omnipathdb.org/interactions/?targets=EGFR\&types=TF\&tfregulons\_levels=A}
\end{quote}

Show the transcriptional targets of Smad2 homology translated to rat including
the confidence levels from TF Regulons:
\begin{quote}

\sphinxurl{http://omnipathdb.org/interactions/?genesymbols=1\&fields=type,ncbi\_tax\_id,tfregulons\_level\&organisms=10116\&sources=Smad2\&types=TF}
\end{quote}

Query interactions from PhosphoNetworks which is part of the \sphinxstyleemphasis{kinaseextra}
dataset:
\begin{quote}

\sphinxurl{http://omnipathdb.org/interactions/?genesymbols=1\&fields=sources\&databases=PhosphoNetworks\&datasets=kinaseextra}
\end{quote}

Get the interactions from Signor, SPIKE and SignaLink3:
\begin{quote}

\sphinxurl{http://omnipathdb.org/interactions/?genesymbols=1\&fields=sources,references\&databases=Signor,SPIKE,SignaLink3}
\end{quote}

All interactions of MAP1LC3B:
\begin{quote}

\sphinxurl{http://omnipathdb.org/interactions/?genesymbols=1\&partners=MAP1LC3B}
\end{quote}

By default \sphinxcode{\sphinxupquote{partners}} queries the interaction where either the source or the
arget is among the partners. If you set the \sphinxcode{\sphinxupquote{source\_target}} parameter to
\sphinxcode{\sphinxupquote{AND}} both the source and the target must be in the queried set:
\begin{quote}

\sphinxurl{http://omnipathdb.org/interactions/?genesymbols=1\&fields=sources,references\&sources=ATG3,ATG7,ATG4B,SQSTM1\&targets=MAP1LC3B,MAP1LC3A,MAP1LC3C,Q9H0R8,GABARAP,GABARAPL2\&source\_target=AND}
\end{quote}

As you see above you can use UniProt IDs and Gene Symbols in the queries and
also mix them. Get the miRNA regulating NOTCH1:
\begin{quote}

\sphinxurl{http://omnipathdb.org/interactions/?genesymbols=1\&fields=sources,references\&datasets=mirnatarget\&targets=NOTCH1}
\end{quote}

Note: with the exception of mandatory fields and genesymbols, the columns
appear exactly in the order you provided in your query.


\subsection{Enzyme-substrate interactions}
\label{\detokenize{index:enzyme-substrate-interactions}}
Another query type available is \sphinxcode{\sphinxupquote{ptms}} which provides enzyme-substrate
interactions. It is very similar to the \sphinxcode{\sphinxupquote{interactions}}:
\begin{quote}

\sphinxurl{http://omnipathdb.org/ptms?genesymbols=1\&fields=sources,references,isoforms\&enzymes=FYN}
\end{quote}

Is there any ubiquitination reaction?
\begin{quote}

\sphinxurl{http://omnipathdb.org/ptms?genesymbols=1\&fields=sources,references\&types=ubiquitination}
\end{quote}

And acetylation in mouse?
\begin{quote}

\sphinxurl{http://omnipathdb.org/ptms?genesymbols=1\&fields=sources,references\&types=acetylation\&organisms=10090}
\end{quote}

Rat interactions, both directly from rat and homology translated from human,
from the PhosphoSite database:
\begin{quote}

\sphinxurl{http://omnipathdb.org/ptms?genesymbols=1\&fields=sources,references\&organisms=10116\&databases=PhosphoSite,PhosphoSite\_noref}
\end{quote}


\subsection{Molecular complexes}
\label{\detokenize{index:molecular-complexes}}
The \sphinxcode{\sphinxupquote{complexes}} query provides a comprehensive database of more than 22,000
protein complexes. For example, to query all complexes from CORUM and PDB
containing MTOR (P42345):
\begin{quote}

\sphinxurl{http://omnipathdb.org/complexes?proteins=P42345\&databases=CORUM,PDB}
\end{quote}


\subsection{Annotations}
\label{\detokenize{index:annotations}}
The \sphinxcode{\sphinxupquote{annotations}} query provides a large variety of data about proteins,
complexes and in the future other kinds of molecules. For example an
annotation can tell if a protein is a kinase, or if it is expressed in the
hearth muscle. These data come from dozens of databases and each kind of
annotation record contains different fields. Because of this here we have
a \sphinxcode{\sphinxupquote{record\_id}} field which is unique within the records of each database.
Each row contains one key value pair and you need to use the \sphinxcode{\sphinxupquote{record\_id}}
to connect the related key-value pairs. You can easily do this with \sphinxcode{\sphinxupquote{tidyr}}
and \sphinxcode{\sphinxupquote{dplyr}} in R or \sphinxcode{\sphinxupquote{pandas}} in Python. An example to query the pathway
annotations from SignaLink:
\begin{quote}

\sphinxurl{http://omnipathdb.org/annotations?databases=SignaLink3}
\end{quote}

Or the tissue expression of BMP7 from Human Protein Atlas:
\begin{quote}

\sphinxurl{http://omnipathdb.org/annotations?databases=HPA\&proteins=BMP7}
\end{quote}


\subsection{Roles in inter-cellular communication}
\label{\detokenize{index:roles-in-inter-cellular-communication}}
Another query type is \sphinxcode{\sphinxupquote{intercell}} providing information on the roles in
inter-cellular signaling. E.g. if a protein is a ligand, a receptor, an
extracellular matrix (ECM) component, etc. This query type is very similar
to \sphinxcode{\sphinxupquote{annotations}} but here the data does not come from original sources but
combined from several databases by us. However we refer also to the original
databases whenever the \sphinxcode{\sphinxupquote{class\_type}} is \sphinxcode{\sphinxupquote{sub}} (subclass). E.g. the main
class \sphinxcode{\sphinxupquote{ligand}} is a combination of \sphinxcode{\sphinxupquote{Ramilowski 2015}}, \sphinxcode{\sphinxupquote{CellPhoneDB}},
\sphinxcode{\sphinxupquote{HPMR}} and many other databases, hence besides the \sphinxcode{\sphinxupquote{ligand}} category you
will find sub-categories like \sphinxcode{\sphinxupquote{ligand\_ramilowski}}, \sphinxcode{\sphinxupquote{ligand\_cellphonedb}}
and so on. An example how to get all intercell annotations for 4 selected
proteins:
\begin{quote}

\sphinxurl{http://omnipathdb.org/intercell?proteins=EGFR,ULK1,ATG4A,BMP8B}
\end{quote}

Or all the main classes for one protein:
\begin{quote}

\sphinxurl{http://omnipathdb.org/intercell?levels=main\&proteins=P00533}
\end{quote}

Or a list of all ECM proteins:
\begin{quote}

\sphinxurl{http://omnipathdb.org/intercell?categories=ecm}
\end{quote}


\subsection{Exploring possible parameters}
\label{\detokenize{index:exploring-possible-parameters}}
Sometimes the names and values of the query parameters are not intuitive,
even though in many cases the server accepts multiple alternatives. To see
the possible parameters with all possible values you can use the \sphinxcode{\sphinxupquote{queries}}
query type. The server checks the paremeter names and values exactly against
these rules and if any of them don’t match you will get an error message
instead of reply. To see the parameters for the \sphinxcode{\sphinxupquote{interactions}} query:
\begin{quote}

\sphinxurl{http://omnipathdb.org/queries/interactions}
\end{quote}


\chapter{Can I use OmniPath in R?}
\label{\detokenize{index:can-i-use-omnipath-in-r}}
You can download the data from the webservice and load into R. Thanks to
our colleague Attila Gabor we have a dedicated package for this:
\begin{quote}

\sphinxurl{https://github.com/saezlab/OmnipathR}
\end{quote}

Alternatively here is a very simple example:
\begin{quote}

\sphinxurl{https://github.com/saezlab/pypath/tree/master/r\_import}
\end{quote}


\chapter{Installation}
\label{\detokenize{index:installation}}

\section{Linux}
\label{\detokenize{index:linux}}
In almost any up-to-date Linux distribution the dependencies of \sphinxstylestrong{pypath} are
built-in, or provided by the distributors. You only need to install a couple
of things in your package manager (cairo, py(2)cairo, igraph,
python(2)-igraph, graphviz, pygraphviz), and after install \sphinxstylestrong{pypath} by \sphinxstyleemphasis{pip}
(see below). If any module still missing, you can install them the usual way
by \sphinxstyleemphasis{pip} or your package manager.


\section{igraph C library, cairo and pycairo}
\label{\detokenize{index:igraph-c-library-cairo-and-pycairo}}
\sphinxstyleemphasis{python(2)-igraph} is a Python interface to use the igraph C library. The
C library must be installed. The same goes for \sphinxstyleemphasis{cairo}, \sphinxstyleemphasis{py(2)cairo} and
\sphinxstyleemphasis{graphviz}.


\section{Directly from git}
\label{\detokenize{index:directly-from-git}}
\begin{sphinxVerbatim}[commandchars=\\\{\}]
pip install git+https://github.com/saezlab/pypath.git
\end{sphinxVerbatim}


\section{With pip}
\label{\detokenize{index:with-pip}}
Download the package from /dist, and install with pip:

\begin{sphinxVerbatim}[commandchars=\\\{\}]
pip install pypath\PYGZhy{}x.y.z.tar.gz
\end{sphinxVerbatim}


\section{Build source distribution}
\label{\detokenize{index:build-source-distribution}}
Clone the git repo, and run setup.py:

\begin{sphinxVerbatim}[commandchars=\\\{\}]
python setup.py sdist
\end{sphinxVerbatim}


\section{Mac OS X}
\label{\detokenize{index:mac-os-x}}
On OS X installation is not straightforward primarily because cairo needs to
be compiled from source. We provide 2 scripts here: the
\sphinxstylestrong{mac-install-brew.sh} installs everything with HomeBrew, and
\sphinxstylestrong{mac-install-conda.sh} installs from Anaconda distribution. With these
scripts installation of igraph, cairo and graphviz goes smoothly most of the
time, and options are available for omitting the 2 latter. To know more see
the description in the script header. There is a third script
\sphinxstylestrong{mac-install-source.sh} which compiles everything from source and presumes
only Python 2.7 and Xcode installed. We do not recommend this as it is time
consuming and troubleshooting requires expertise.


\subsection{Troubleshooting}
\label{\detokenize{index:troubleshooting}}\begin{itemize}
\item {} 
\sphinxcode{\sphinxupquote{no module named ...}} when you try to load a module in Python. Did
theinstallation of the module run without error? Try to run again the specific
part from the mac install shell script to see if any error comes up. Is the
path where the module has been installed in your \sphinxcode{\sphinxupquote{\$PYTHONPATH}}? Try \sphinxcode{\sphinxupquote{echo
\$PYTHONPATH}} to see the current paths. Add your local install directories if
those are not there, e.g.
\sphinxcode{\sphinxupquote{export PYTHONPATH="/Users/me/local/python2.7/site-packages:\$PYTHONPATH"}}.
If it works afterwards, don’t forget to append these export path statements to
your \sphinxcode{\sphinxupquote{\textasciitilde{}/.bash\_profile}}, so these will be set every time you launch a new
shell.

\item {} 
\sphinxcode{\sphinxupquote{pkgconfig}} not found. Check if the \sphinxcode{\sphinxupquote{\$PKG\_CONFIG\_PATH}} variable is
set correctly, and pointing on a directory where pkgconfig really can be
found.

\item {} 
Error while trying to install py(2)cairo by pip. py(2)cairo could not be
installed by pip, but only by waf. Please set the \sphinxcode{\sphinxupquote{\$PKG\_CONFIG\_PATH}} before.
See \sphinxstylestrong{mac-install-source.sh} on how to install with waf.

\item {} 
Error at pygraphviz build: \sphinxcode{\sphinxupquote{graphviz/cgraph.h file not found}}. This is
because the directory of graphviz detected wrong by pkgconfig. See
\sphinxstylestrong{mac-install-source.sh} how to set include dirs and library dirs by
\sphinxcode{\sphinxupquote{-{-}global-option}} parameters.

\item {} 
Can not install bioservices, because installation of jurko-suds fails. Ok,
this fails because pip is not able to install the recent version of
setuptools, because a very old version present in the system path. The
development version of jurko-suds does not require setuptools, so you can
install it directly from git as it is done in \sphinxstylestrong{mac-install-source.sh}.

\item {} 
In \sphinxstylestrong{Anaconda}, \sphinxstyleemphasis{pypath} can be imported, but the modules and classes are
missing. Apparently Anaconda has some built-in stuff called \sphinxstyleemphasis{pypath}. This
has nothing to do with this module. Please be aware that Anaconda installs a
completely separated Python distribution, and does not detect modules in the
main Python installation. You need to install all modules within Anaconda’s
directory. \sphinxstylestrong{mac-install-conda.sh} does exactly this. If you still
experience issues, please contact us.

\end{itemize}


\section{Microsoft Windows}
\label{\detokenize{index:microsoft-windows}}
Not many people have used \sphinxstyleemphasis{pypath} on Microsoft computers so far. Please share
your experiences and contact us if you encounter any issue. We appreciate
your feedback, and it would be nice to have better support for other computer
systems.


\subsection{With Anaconda}
\label{\detokenize{index:with-anaconda}}
The same workflow like you see in \sphinxcode{\sphinxupquote{mac-install-conda.sh}} should work for
Anaconda on Windows. The only problem you certainly will encounter is that not
all the channels have packages for all platforms. If certain channel provides
no package for Windows, or for your Python version, you just need to find an
other one. For this, do a search:

\begin{sphinxVerbatim}[commandchars=\\\{\}]
anaconda search \PYGZhy{}t conda \PYGZlt{}package name\PYGZgt{}
\end{sphinxVerbatim}

For example, if you search for \sphinxstyleemphasis{pycairo}, you will find out that \sphinxstyleemphasis{vgauther}
provides it for osx-64, but only for Python 3.4, while \sphinxstyleemphasis{richlewis} provides
also for Python 3.5. And for win-64 platform, there is the channel of
\sphinxstyleemphasis{KristanAmstrong}. Go along all the commands in \sphinxcode{\sphinxupquote{mac-install-conda.sh}}, and
modify the channel if necessary, until all packages install successfully.


\subsection{With other Python distributions}
\label{\detokenize{index:with-other-python-distributions}}
Here the basic principles are the same as everywhere: first try to install all
external dependencies, after \sphinxstyleemphasis{pip} install should work. On Windows certain
packages can not be installed by compiled from source by \sphinxstyleemphasis{pip}, instead the
easiest to install them precompiled. These are in our case \sphinxstyleemphasis{fisher, lxml,
numpy (mkl version), pycairo, igraph, pygraphviz, scipy and statsmodels}. The
precompiled packages are available here:
\sphinxurl{http://www.lfd.uci.edu/~gohlke/pythonlibs/}. We tested the setup with Python
3.4.3 and Python 2.7.11. The former should just work fine, while with the
latter we have issues to be resolved.


\subsection{Known issues}
\label{\detokenize{index:known-issues}}\begin{itemize}
\item {} 
\sphinxstyleemphasis{“No module fabric available.”} \textendash{} or \sphinxstyleemphasis{pysftp} missing: this is not
important, only certain data download methods rely on these modules, but
likely you won’t call those at all.

\item {} 
Progress indicator floods terminal: sorry about that, will be fixed soon.

\item {} 
Encoding related exceptions in Python2: these might occur at some points in
the module, please send the traceback if you encounter one, and we will fix
as soon as possible.

\item {} 
For Mac OS X (v \textgreater{}= 10.11 El Capitan) import of pypath fails with error: “libcurl link-time ssl backend (openssl) is different from compile-time ssl backend (none/other)”. To fix it, you may need to reinstall pycurl library using special flags. More information and steps can be found e.g. {[}here{]}(\sphinxurl{https://cscheng.info/2018/01/26/installing-pycurl-on-macos-high-sierra.html})

\end{itemize}

\sphinxstyleemphasis{Special thanks to Jorge Ferreira for testing pypath on Windows!}


\chapter{Release History}
\label{\detokenize{index:release-history}}
Main improvements in the past releases:


\section{0.1.0}
\label{\detokenize{index:id1}}\begin{itemize}
\item {} 
First release of PyPath, for initial testing.

\end{itemize}


\section{0.2.0}
\label{\detokenize{index:id2}}\begin{itemize}
\item {} 
Lots of small improvements in almost every module

\item {} 
Networks can be read from local files, remote files, lists or provided by any function

\item {} 
Almost all redistributed data have been removed, every source downloaded from the original provider.

\end{itemize}


\section{0.3.0}
\label{\detokenize{index:id3}}\begin{itemize}
\item {} 
First version with partial Python 3 support.

\end{itemize}


\section{0.4.0}
\label{\detokenize{index:id4}}\begin{itemize}
\item {} 
\sphinxstylestrong{pyreact} module with \sphinxstylestrong{BioPaxReader} and \sphinxstylestrong{PyReact} classes added

\item {} 
Process description databases, BioPax and PathwayCommons SIF conversion rules are supported

\item {} 
Format definitions for 6 process description databases included.

\end{itemize}


\section{0.5.0}
\label{\detokenize{index:id5}}\begin{itemize}
\item {} 
Many classes have been added to the \sphinxstylestrong{plot} module

\item {} 
All figures and tables in the manuscript can be generated automatically

\item {} 
This is supported by a new module, \sphinxstylestrong{analysis}, which implements a generic workflow in its \sphinxstylestrong{Workflow} class.

\end{itemize}


\section{0.5.32}
\label{\detokenize{index:id6}}\begin{itemize}
\item {} 
\sphinxtitleref{chembl}, \sphinxtitleref{unichem}, \sphinxtitleref{mysql} and \sphinxtitleref{mysql\_connect} modules made Python3 compatible

\end{itemize}


\section{0.6.31}
\label{\detokenize{index:id7}}\begin{itemize}
\item {} 
Orthology translation of network

\item {} 
Homologene UniProt dict to translate between different organisms UniProt-to-UniProt

\item {} 
Orthology translation of PTMs

\item {} 
Better processing of PhosphoSite regulatory sites

\end{itemize}


\section{0.7.0}
\label{\detokenize{index:id8}}\begin{itemize}
\item {} 
TF-target, miRNA-mRNA and TF-miRNA interactions from many databases

\end{itemize}


\section{0.7.74}
\label{\detokenize{index:id9}}\begin{itemize}
\item {} 
New web server based on \sphinxtitleref{pandas} data frames

\item {} 
New module \sphinxtitleref{export} for generating data frames of interactions or enzyme-substrate interactions

\item {} 
New module \sphinxtitleref{websrvtab} for exporting data frames for the web server

\item {} 
TF-target interactions from DoRothEA

\end{itemize}


\section{0.7.93}
\label{\detokenize{index:id10}}\begin{itemize}
\item {} 
New \sphinxtitleref{dataio} methods for Gene Ontology

\end{itemize}


\section{0.7.110}
\label{\detokenize{index:id11}}\begin{itemize}
\item {} 
Many new docstrings

\end{itemize}


\section{0.8}
\label{\detokenize{index:id12}}\begin{itemize}
\item {} 
New module \sphinxtitleref{complex}: a comprehensive database of complexes

\item {} 
New module \sphinxtitleref{annot}: database of protein annotations (function, location)

\item {} 
New module \sphinxtitleref{intercell}: special methods for data integration focusing on intercellular communication

\item {} 
New module \sphinxtitleref{bel}: BEL integration

\item {} 
Module \sphinxtitleref{go} and all the connected \sphinxtitleref{dataio} methods have been rewritten offering a workaround for
data access despite GO’s terrible web services and providing much more versatile query methods

\item {} 
Removed MySQL support (e.g. loading mapping tables from MySQL)

\item {} 
Modules \sphinxtitleref{mapping}, \sphinxtitleref{reflists}, \sphinxtitleref{complex}, \sphinxtitleref{ptm}, \sphinxtitleref{annot}, \sphinxtitleref{go} became services:
these modules build databases and provide query methods, sometimes they even automatically
delete data to free memory

\item {} 
New interaction category in \sphinxtitleref{data\_formats}: \sphinxtitleref{ligand\_receptor}

\item {} 
Improved logging and control over verbosity

\item {} 
Better control over paremeters by the \sphinxtitleref{settings} module

\item {} 
Many methods in \sphinxtitleref{dataio} have been improved or fixed, docs and code style largely improved

\item {} 
Started to add tests especially for methods in \sphinxtitleref{dataio}

\end{itemize}


\section{0.9}
\label{\detokenize{index:id13}}\begin{itemize}
\item {} 
The network database is not dependent any more on \sphinxtitleref{python-igraph} hence it
has been removed from the mandatory dependencies

\item {} 
New API for the network, interactions, evidences, molecular entities

\end{itemize}


\section{Upcoming}
\label{\detokenize{index:upcoming}}\begin{itemize}
\item {} 
New, more flexible network reader class

\item {} 
Full support for multi-species molecular interaction networks
(e.g. pathogene-host)

\item {} 
Better support for not protein only molecular interaction networks
(metabolites, drug compounds, RNA)

\end{itemize}


\chapter{Features}
\label{\detokenize{index:features}}
In the beginning the primary aim of \sphinxstylestrong{pypath} was to build networks from
multiple sources using an igraph object as the fundament of the integrated
data structure. From version 0.7 and 0.8 this design principle started to
change. Today \sphinxstylestrong{pypath} builds a number of different databases each having
\sphinxstylestrong{pandas.DataFrame} as a final format. Each of these integrates a specific
kind of data from various databases (e.g. protein complexes, interactions,
enzyme-PTM relationships, etc). \sphinxstylestrong{pypath} has many submodules with standalone
functionality which can be used in other modules and scripts. For example
the ID conversion module \sphinxstylestrong{pypath.mapping}.

Submodules perform various features, e.g. graph visualization, working with
rug compound data, searching drug targets and compounds in \sphinxstylestrong{ChEMBL}.


\section{ID conversion}
\label{\detokenize{index:id-conversion}}
The ID conversion module \sphinxcode{\sphinxupquote{mapping}} can be used independently. It has the
feature to translate secondary UniProt IDs to primaries, and Trembl IDs to
SwissProt, using primary Gene Symbols to find the connections. This module
automatically loads and stores the necessary conversion tables. Many tables
are predefined, such as all the IDs in \sphinxstylestrong{UniProt mapping service,} while
users are able to load any table from \sphinxstylestrong{file} or \sphinxstylestrong{MySQL,} using the classes
provided in the module \sphinxcode{\sphinxupquote{input\_formats}}.


\section{Pathways}
\label{\detokenize{index:pathways}}
\sphinxstylestrong{pypath} includes data and predefined format descriptions for more than 25
high quality, literature curated databases. The inut formats are defined in
the \sphinxcode{\sphinxupquote{data\_formats}} module. For some resources data downloaded on the fly,
where it is not possible, data is redistributed with the module. Descriptions
and comprehensive information about the resources is available in the
\sphinxcode{\sphinxupquote{descriptions}} module.


\section{Structural features}
\label{\detokenize{index:structural-features}}
One of the modules called \sphinxcode{\sphinxupquote{intera}} provides many classes for representing
structures and mechanisms behind protein interactions. These are \sphinxcode{\sphinxupquote{Residue}}
(optionally mutated), \sphinxcode{\sphinxupquote{Motif}}, \sphinxcode{\sphinxupquote{Ptm}}, \sphinxcode{\sphinxupquote{Domain}}, \sphinxcode{\sphinxupquote{DomainMotif}},
\sphinxcode{\sphinxupquote{DomainDomain}} and \sphinxcode{\sphinxupquote{Interface}}. All these classes have \sphinxcode{\sphinxupquote{\_\_eq\_\_()}}
methods to test equality between instances, and also \sphinxcode{\sphinxupquote{\_\_contains\_\_()}}
methods to look up easily if a residue is within a short motif or protein
domain, or is the target residue of a PTM.


\section{Sequences}
\label{\detokenize{index:sequences}}
The module \sphinxcode{\sphinxupquote{seq}} contains a simple class for quick lookup any residue or
segment in \sphinxstylestrong{UniProt} protein sequences while being aware of isoforms.


\section{Tissue expression}
\label{\detokenize{index:tissue-expression}}
For 3 protein expression databases there are functions and modules for
downloading and combining the expression data with the network. These are the
Human Protein Atlas, the ProteomicsDB and GIANT. The \sphinxcode{\sphinxupquote{giant}} and
\sphinxcode{\sphinxupquote{proteomicsdb}} modules can be used also as stand alone Python clients for
these resources.


\section{Functional annotations}
\label{\detokenize{index:functional-annotations}}
\sphinxstylestrong{GSEA} and \sphinxstylestrong{Gene Ontology} are two approaches for annotating genes and
gene products, and enrichment analysis technics aims to use these annotations
to highlight the biological functions a given set of genes is related to. Here
the \sphinxcode{\sphinxupquote{enrich}} module gives abstract classes to calculate enrichment
statistics, while the \sphinxcode{\sphinxupquote{go}} and the \sphinxcode{\sphinxupquote{gsea}} modules give access to GO and
GSEA data, and make it easy to count enrichment statistics for sets of genes.


\section{Drug compounds}
\label{\detokenize{index:drug-compounds}}
\sphinxstylestrong{UniChem} submodule provides an interface to effectively query the UniChem
service, use connectivity search with custom settings, and translate SMILEs to
ChEMBL IDs with ChEMBL web service.

\sphinxstylestrong{ChEMBL} submodule queries directly your own ChEMBL MySQL instance, has the
features to search targets and compounds from custom assay types and
relationship types, to get activity values, binding domains, and action types.
You need to download the ChEMBL MySQL dump, and load into your own server.


\section{Technical}
\label{\detokenize{index:technical}}
The module \sphinxcode{\sphinxupquote{pypath.curl}} provides a very flexible \sphinxstylestrong{download manager}
built on top of \sphinxcode{\sphinxupquote{pycurl}}. The classes \sphinxcode{\sphinxupquote{pypath.curl.Curl()}} and
\sphinxcode{\sphinxupquote{pypath.curl.FileOpener}} accept numerous arguments, try to deal in a smart
way with local \sphinxstylestrong{cache,} authentication, redirects, uncompression, character
encodings, FTP and HTTP transactions, and many other stuff. Cache can grow to
several GBs, and takes place in \sphinxcode{\sphinxupquote{\textasciitilde{}/.pypath/cache}} by default. If you
experience issues using \sphinxcode{\sphinxupquote{pypath}} these are most often related to failed
downloads which often result nonsense cache contents. To debug such issues
you can see the cache file names and cache usage in the log, and you can use
the context managers in \sphinxcode{\sphinxupquote{pypath.curl}} to show, delete or bypass the cache
for some particular method calls (\sphinxcode{\sphinxupquote{pypath.curl.cache\_print\_on()}},
\sphinxcode{\sphinxupquote{pypath.curl.cache\_delete\_on()}} and \sphinxcode{\sphinxupquote{pypath.curl.cache\_off()}}.
You can always set up an alternative cache directory for the entire session
using the \sphinxcode{\sphinxupquote{pypath.settings}} module.

The \sphinxcode{\sphinxupquote{pypath.session}} and \sphinxcode{\sphinxupquote{pypath.log}} modules take care of setting up
session level parameters and logging. Each session has a random 5 character
identifier e.g. \sphinxcode{\sphinxupquote{y5jzx}}. The default log file in this case is
\sphinxcode{\sphinxupquote{pypath\_log/pypath-y5jzx.log}}. The log messages flushed in every 2 seconds
by default. You can always change these things by the \sphinxcode{\sphinxupquote{settings}} module.
In this module you can get and set the values of various parameters using
the \sphinxcode{\sphinxupquote{pypath.settings.setup()}} and the \sphinxcode{\sphinxupquote{pypath.settings.get()}} methods.

A simple \sphinxstylestrong{webservice} comes with this module: the \sphinxcode{\sphinxupquote{server}} module based on
\sphinxcode{\sphinxupquote{twisted.web.server}} opens a custom port and serves plain text tables over
HTTP with REST style querying.


\renewcommand{\indexname}{Python Module Index}
\begin{sphinxtheindex}
\let\bigletter\sphinxstyleindexlettergroup
\bigletter{p}
\item\relax\sphinxstyleindexentry{pypath.core.annot}\sphinxstyleindexpageref{reference:\detokenize{module-pypath.core.annot}}
\item\relax\sphinxstyleindexentry{pypath.core.complex}\sphinxstyleindexpageref{reference:\detokenize{module-pypath.core.complex}}
\item\relax\sphinxstyleindexentry{pypath.core.entity}\sphinxstyleindexpageref{reference:\detokenize{module-pypath.core.entity}}
\item\relax\sphinxstyleindexentry{pypath.core.enz\_sub}\sphinxstyleindexpageref{reference:\detokenize{module-pypath.core.enz_sub}}
\item\relax\sphinxstyleindexentry{pypath.core.evidence}\sphinxstyleindexpageref{reference:\detokenize{module-pypath.core.evidence}}
\item\relax\sphinxstyleindexentry{pypath.core.interaction}\sphinxstyleindexpageref{reference:\detokenize{module-pypath.core.interaction}}
\item\relax\sphinxstyleindexentry{pypath.core.intercell}\sphinxstyleindexpageref{reference:\detokenize{module-pypath.core.intercell}}
\item\relax\sphinxstyleindexentry{pypath.core.intercell\_annot}\sphinxstyleindexpageref{reference:\detokenize{module-pypath.core.intercell_annot}}
\item\relax\sphinxstyleindexentry{pypath.core.network}\sphinxstyleindexpageref{reference:\detokenize{module-pypath.core.network}}
\item\relax\sphinxstyleindexentry{pypath.inputs.main}\sphinxstyleindexpageref{reference:\detokenize{module-pypath.inputs.main}}
\item\relax\sphinxstyleindexentry{pypath.inputs.mirbase}\sphinxstyleindexpageref{reference:\detokenize{module-pypath.inputs.mirbase}}
\item\relax\sphinxstyleindexentry{pypath.internals.input\_formats}\sphinxstyleindexpageref{reference:\detokenize{module-pypath.internals.input_formats}}
\item\relax\sphinxstyleindexentry{pypath.internals.intera}\sphinxstyleindexpageref{reference:\detokenize{module-pypath.internals.intera}}
\item\relax\sphinxstyleindexentry{pypath.internals.maps}\sphinxstyleindexpageref{reference:\detokenize{module-pypath.internals.maps}}
\item\relax\sphinxstyleindexentry{pypath.internals.refs}\sphinxstyleindexpageref{reference:\detokenize{module-pypath.internals.refs}}
\item\relax\sphinxstyleindexentry{pypath.internals.resource}\sphinxstyleindexpageref{reference:\detokenize{module-pypath.internals.resource}}
\item\relax\sphinxstyleindexentry{pypath.legacy.main}\sphinxstyleindexpageref{reference:\detokenize{module-pypath.legacy.main}}
\item\relax\sphinxstyleindexentry{pypath.omnipath.export}\sphinxstyleindexpageref{reference:\detokenize{module-pypath.omnipath.export}}
\item\relax\sphinxstyleindexentry{pypath.omnipath.server.build}\sphinxstyleindexpageref{reference:\detokenize{module-pypath.omnipath.server.build}}
\item\relax\sphinxstyleindexentry{pypath.resources.data\_formats}\sphinxstyleindexpageref{reference:\detokenize{module-pypath.resources.data_formats}}
\item\relax\sphinxstyleindexentry{pypath.resources.descriptions}\sphinxstyleindexpageref{reference:\detokenize{module-pypath.resources.descriptions}}
\item\relax\sphinxstyleindexentry{pypath.resources.network}\sphinxstyleindexpageref{reference:\detokenize{module-pypath.resources.network}}
\item\relax\sphinxstyleindexentry{pypath.resources.urls}\sphinxstyleindexpageref{reference:\detokenize{module-pypath.resources.urls}}
\item\relax\sphinxstyleindexentry{pypath.share.cache}\sphinxstyleindexpageref{reference:\detokenize{module-pypath.share.cache}}
\item\relax\sphinxstyleindexentry{pypath.share.common}\sphinxstyleindexpageref{reference:\detokenize{module-pypath.share.common}}
\item\relax\sphinxstyleindexentry{pypath.share.curl}\sphinxstyleindexpageref{reference:\detokenize{module-pypath.share.curl}}
\item\relax\sphinxstyleindexentry{pypath.share.log}\sphinxstyleindexpageref{reference:\detokenize{module-pypath.share.log}}
\item\relax\sphinxstyleindexentry{pypath.share.progress}\sphinxstyleindexpageref{reference:\detokenize{module-pypath.share.progress}}
\item\relax\sphinxstyleindexentry{pypath.share.session}\sphinxstyleindexpageref{reference:\detokenize{module-pypath.share.session}}
\item\relax\sphinxstyleindexentry{pypath.share.settings}\sphinxstyleindexpageref{reference:\detokenize{module-pypath.share.settings}}
\item\relax\sphinxstyleindexentry{pypath.utils.go}\sphinxstyleindexpageref{reference:\detokenize{module-pypath.utils.go}}
\item\relax\sphinxstyleindexentry{pypath.utils.homology}\sphinxstyleindexpageref{reference:\detokenize{module-pypath.utils.homology}}
\item\relax\sphinxstyleindexentry{pypath.utils.mapping}\sphinxstyleindexpageref{reference:\detokenize{module-pypath.utils.mapping}}
\item\relax\sphinxstyleindexentry{pypath.utils.pdb}\sphinxstyleindexpageref{reference:\detokenize{module-pypath.utils.pdb}}
\item\relax\sphinxstyleindexentry{pypath.utils.pyreact}\sphinxstyleindexpageref{reference:\detokenize{module-pypath.utils.pyreact}}
\item\relax\sphinxstyleindexentry{pypath.utils.reflists}\sphinxstyleindexpageref{reference:\detokenize{module-pypath.utils.reflists}}
\item\relax\sphinxstyleindexentry{pypath.utils.residues}\sphinxstyleindexpageref{reference:\detokenize{module-pypath.utils.residues}}
\item\relax\sphinxstyleindexentry{pypath.utils.seq}\sphinxstyleindexpageref{reference:\detokenize{module-pypath.utils.seq}}
\item\relax\sphinxstyleindexentry{pypath.utils.taxonomy}\sphinxstyleindexpageref{reference:\detokenize{module-pypath.utils.taxonomy}}
\item\relax\sphinxstyleindexentry{pypath.utils.unichem}\sphinxstyleindexpageref{reference:\detokenize{module-pypath.utils.unichem}}
\item\relax\sphinxstyleindexentry{pypath.visual.plot}\sphinxstyleindexpageref{reference:\detokenize{module-pypath.visual.plot}}
\end{sphinxtheindex}

\renewcommand{\indexname}{Index}
\printindex
\end{document}